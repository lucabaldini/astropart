\chapter{Cross section and mean free path}\label{chap:xsec}


In scattering theory, the differential cross section measures the probability for
a given particle to be deflected in a given direction by a scattering
center\sidenote{In real life things can be more complicated, and a cross section
  can be \emph{differential} in an arbitrary number of variable, not limited
  to directions in space (e.g., the momentum transferred in the collisiom), but
  we shall stick to the simplest case.}.
It has the physical dimensions of an area, and, at least in particle physics, is
customarily measured in \emph{barns}, where
\begin{align*}
  1~b = 10^{-24}~\text{cm}^2.
\end{align*}

From a physical standpoint, given a uniform beam of particles impinging onto a
scattering center, the differential cross section is the ratio $\nicefrac{d\sigma}{d\Omega}$,
where $d\sigma$ represents the (infinitesimal) area, perpendicular to the beam
axis, across which particles are deflected in the direction $d\Omega$.

Integrating over all the directions yields the so called \emph{total} cross section
\begin{align}
  \sigma = \int_\Omega \dv{\sigma}{\Omega} \diff\Omega
\end{align}
which measures the total surface orthogonal to the beam direction over which
particles are scattered in \emph{some} direction\sidenote{It follows from the definition
that the total cross section is infinite for long-range forces, like the Coulomb
scattering off a naked nucleus: in this case the incoming particle is deflected
by some angle, no matter how large the impact parameter is.}.


\section{Diffusion from a hard sphere}

\begin{marginfigure}
  %% Creator: Matplotlib, PGF backend
%%
%% To include the figure in your LaTeX document, write
%%   \input{<filename>.pgf}
%%
%% Make sure the required packages are loaded in your preamble
%%   \usepackage{pgf}
%%
%% Also ensure that all the required font packages are loaded; for instance,
%% the lmodern package is sometimes necessary when using math font.
%%   \usepackage{lmodern}
%%
%% Figures using additional raster images can only be included by \input if
%% they are in the same directory as the main LaTeX file. For loading figures
%% from other directories you can use the `import` package
%%   \usepackage{import}
%%
%% and then include the figures with
%%   \import{<path to file>}{<filename>.pgf}
%%
%% Matplotlib used the following preamble
%%   \usepackage{fontspec}
%%   \setmainfont{DejaVuSerif.ttf}[Path=\detokenize{/usr/share/matplotlib/mpl-data/fonts/ttf/}]
%%   \setsansfont{DejaVuSans.ttf}[Path=\detokenize{/usr/share/matplotlib/mpl-data/fonts/ttf/}]
%%   \setmonofont{DejaVuSansMono.ttf}[Path=\detokenize{/usr/share/matplotlib/mpl-data/fonts/ttf/}]
%%
\begingroup%
\makeatletter%
\begin{pgfpicture}%
\pgfpathrectangle{\pgfpointorigin}{\pgfqpoint{1.950000in}{1.300000in}}%
\pgfusepath{use as bounding box, clip}%
\begin{pgfscope}%
\pgfsetbuttcap%
\pgfsetmiterjoin%
\definecolor{currentfill}{rgb}{1.000000,1.000000,1.000000}%
\pgfsetfillcolor{currentfill}%
\pgfsetlinewidth{0.000000pt}%
\definecolor{currentstroke}{rgb}{1.000000,1.000000,1.000000}%
\pgfsetstrokecolor{currentstroke}%
\pgfsetdash{}{0pt}%
\pgfpathmoveto{\pgfqpoint{0.000000in}{0.000000in}}%
\pgfpathlineto{\pgfqpoint{1.950000in}{0.000000in}}%
\pgfpathlineto{\pgfqpoint{1.950000in}{1.300000in}}%
\pgfpathlineto{\pgfqpoint{0.000000in}{1.300000in}}%
\pgfpathlineto{\pgfqpoint{0.000000in}{0.000000in}}%
\pgfpathclose%
\pgfusepath{fill}%
\end{pgfscope}%
\begin{pgfscope}%
\pgfpathrectangle{\pgfqpoint{0.108333in}{0.000000in}}{\pgfqpoint{1.733333in}{1.300000in}}%
\pgfusepath{clip}%
\pgfsetbuttcap%
\pgfsetmiterjoin%
\definecolor{currentfill}{rgb}{1.000000,1.000000,1.000000}%
\pgfsetfillcolor{currentfill}%
\pgfsetlinewidth{1.003750pt}%
\definecolor{currentstroke}{rgb}{0.000000,0.000000,0.000000}%
\pgfsetstrokecolor{currentstroke}%
\pgfsetdash{}{0pt}%
\pgfpathmoveto{\pgfqpoint{1.321667in}{0.130000in}}%
\pgfpathcurveto{\pgfqpoint{1.413604in}{0.130000in}}{\pgfqpoint{1.501788in}{0.166527in}}{\pgfqpoint{1.566797in}{0.231536in}}%
\pgfpathcurveto{\pgfqpoint{1.631806in}{0.296546in}}{\pgfqpoint{1.668333in}{0.384730in}}{\pgfqpoint{1.668333in}{0.476667in}}%
\pgfpathcurveto{\pgfqpoint{1.668333in}{0.568604in}}{\pgfqpoint{1.631806in}{0.656788in}}{\pgfqpoint{1.566797in}{0.721797in}}%
\pgfpathcurveto{\pgfqpoint{1.501788in}{0.786806in}}{\pgfqpoint{1.413604in}{0.823333in}}{\pgfqpoint{1.321667in}{0.823333in}}%
\pgfpathcurveto{\pgfqpoint{1.229730in}{0.823333in}}{\pgfqpoint{1.141546in}{0.786806in}}{\pgfqpoint{1.076536in}{0.721797in}}%
\pgfpathcurveto{\pgfqpoint{1.011527in}{0.656788in}}{\pgfqpoint{0.975000in}{0.568604in}}{\pgfqpoint{0.975000in}{0.476667in}}%
\pgfpathcurveto{\pgfqpoint{0.975000in}{0.384730in}}{\pgfqpoint{1.011527in}{0.296546in}}{\pgfqpoint{1.076536in}{0.231536in}}%
\pgfpathcurveto{\pgfqpoint{1.141546in}{0.166527in}}{\pgfqpoint{1.229730in}{0.130000in}}{\pgfqpoint{1.321667in}{0.130000in}}%
\pgfpathlineto{\pgfqpoint{1.321667in}{0.130000in}}%
\pgfpathclose%
\pgfusepath{stroke,fill}%
\end{pgfscope}%
\begin{pgfscope}%
\pgfpathrectangle{\pgfqpoint{0.108333in}{0.000000in}}{\pgfqpoint{1.733333in}{1.300000in}}%
\pgfusepath{clip}%
\pgfsetbuttcap%
\pgfsetmiterjoin%
\pgfsetlinewidth{1.003750pt}%
\definecolor{currentstroke}{rgb}{0.000000,0.000000,0.000000}%
\pgfsetstrokecolor{currentstroke}%
\pgfsetdash{}{0pt}%
\pgfpathmoveto{\pgfqpoint{1.186358in}{0.585000in}}%
\pgfpathcurveto{\pgfqpoint{1.174140in}{0.569739in}}{\pgfqpoint{1.164590in}{0.552520in}}{\pgfqpoint{1.158116in}{0.534073in}}%
\pgfpathcurveto{\pgfqpoint{1.151641in}{0.515626in}}{\pgfqpoint{1.148333in}{0.496217in}}{\pgfqpoint{1.148333in}{0.476667in}}%
\pgfusepath{stroke}%
\end{pgfscope}%
\begin{pgfscope}%
\pgfpathrectangle{\pgfqpoint{0.108333in}{0.000000in}}{\pgfqpoint{1.733333in}{1.300000in}}%
\pgfusepath{clip}%
\pgfsetbuttcap%
\pgfsetmiterjoin%
\pgfsetlinewidth{1.003750pt}%
\definecolor{currentstroke}{rgb}{0.000000,0.000000,0.000000}%
\pgfsetstrokecolor{currentstroke}%
\pgfsetdash{}{0pt}%
\pgfpathmoveto{\pgfqpoint{1.152531in}{0.612083in}}%
\pgfpathcurveto{\pgfqpoint{1.137258in}{0.593007in}}{\pgfqpoint{1.125321in}{0.571483in}}{\pgfqpoint{1.117228in}{0.548425in}}%
\pgfpathcurveto{\pgfqpoint{1.109134in}{0.525366in}}{\pgfqpoint{1.105000in}{0.501104in}}{\pgfqpoint{1.105000in}{0.476667in}}%
\pgfusepath{stroke}%
\end{pgfscope}%
\begin{pgfscope}%
\pgfpathrectangle{\pgfqpoint{0.108333in}{0.000000in}}{\pgfqpoint{1.733333in}{1.300000in}}%
\pgfusepath{clip}%
\pgfsetbuttcap%
\pgfsetmiterjoin%
\pgfsetlinewidth{1.003750pt}%
\definecolor{currentstroke}{rgb}{0.000000,0.000000,0.000000}%
\pgfsetstrokecolor{currentstroke}%
\pgfsetdash{}{0pt}%
\pgfpathmoveto{\pgfqpoint{1.013133in}{0.862469in}}%
\pgfpathcurveto{\pgfqpoint{0.974713in}{0.853856in}}{\pgfqpoint{0.940351in}{0.832403in}}{\pgfqpoint{0.915742in}{0.801667in}}%
\pgfpathcurveto{\pgfqpoint{0.891133in}{0.770930in}}{\pgfqpoint{0.877717in}{0.732707in}}{\pgfqpoint{0.877717in}{0.693333in}}%
\pgfusepath{stroke}%
\end{pgfscope}%
\begin{pgfscope}%
\pgfpathrectangle{\pgfqpoint{0.108333in}{0.000000in}}{\pgfqpoint{1.733333in}{1.300000in}}%
\pgfusepath{clip}%
\pgfsetbuttcap%
\pgfsetroundjoin%
\pgfsetlinewidth{1.003750pt}%
\definecolor{currentstroke}{rgb}{0.000000,0.000000,0.000000}%
\pgfsetstrokecolor{currentstroke}%
\pgfsetdash{{3.700000pt}{1.600000pt}}{0.000000pt}%
\pgfpathmoveto{\pgfqpoint{0.108333in}{0.476667in}}%
\pgfpathlineto{\pgfqpoint{1.841667in}{0.476667in}}%
\pgfusepath{stroke}%
\end{pgfscope}%
\begin{pgfscope}%
\pgfpathrectangle{\pgfqpoint{0.108333in}{0.000000in}}{\pgfqpoint{1.733333in}{1.300000in}}%
\pgfusepath{clip}%
\pgfsetbuttcap%
\pgfsetroundjoin%
\pgfsetlinewidth{1.003750pt}%
\definecolor{currentstroke}{rgb}{0.000000,0.000000,0.000000}%
\pgfsetstrokecolor{currentstroke}%
\pgfsetdash{{3.700000pt}{1.600000pt}}{0.000000pt}%
\pgfpathmoveto{\pgfqpoint{1.321667in}{0.476667in}}%
\pgfpathlineto{\pgfqpoint{0.645125in}{1.018333in}}%
\pgfusepath{stroke}%
\end{pgfscope}%
\begin{pgfscope}%
\pgfpathrectangle{\pgfqpoint{0.108333in}{0.000000in}}{\pgfqpoint{1.733333in}{1.300000in}}%
\pgfusepath{clip}%
\pgfsetrectcap%
\pgfsetroundjoin%
\pgfsetlinewidth{1.003750pt}%
\definecolor{currentstroke}{rgb}{0.000000,0.000000,0.000000}%
\pgfsetstrokecolor{currentstroke}%
\pgfsetdash{}{0pt}%
\pgfpathmoveto{\pgfqpoint{0.108333in}{0.693333in}}%
\pgfpathlineto{\pgfqpoint{1.051050in}{0.693333in}}%
\pgfusepath{stroke}%
\end{pgfscope}%
\begin{pgfscope}%
\pgfpathrectangle{\pgfqpoint{0.108333in}{0.000000in}}{\pgfqpoint{1.733333in}{1.300000in}}%
\pgfusepath{clip}%
\pgfsetrectcap%
\pgfsetroundjoin%
\pgfsetlinewidth{1.003750pt}%
\definecolor{currentstroke}{rgb}{0.000000,0.000000,0.000000}%
\pgfsetstrokecolor{currentstroke}%
\pgfsetdash{}{0pt}%
\pgfpathmoveto{\pgfqpoint{1.051050in}{0.693333in}}%
\pgfpathlineto{\pgfqpoint{0.912806in}{1.310000in}}%
\pgfusepath{stroke}%
\end{pgfscope}%
\begin{pgfscope}%
\pgfpathrectangle{\pgfqpoint{0.108333in}{0.000000in}}{\pgfqpoint{1.733333in}{1.300000in}}%
\pgfusepath{clip}%
\pgfsetrectcap%
\pgfsetroundjoin%
\pgfsetlinewidth{0.752812pt}%
\definecolor{currentstroke}{rgb}{0.000000,0.000000,0.000000}%
\pgfsetstrokecolor{currentstroke}%
\pgfsetdash{}{0pt}%
\pgfpathmoveto{\pgfqpoint{0.455000in}{0.476667in}}%
\pgfpathlineto{\pgfqpoint{0.455000in}{0.693333in}}%
\pgfusepath{stroke}%
\end{pgfscope}%
\begin{pgfscope}%
\definecolor{textcolor}{rgb}{0.000000,0.000000,0.000000}%
\pgfsetstrokecolor{textcolor}%
\pgfsetfillcolor{textcolor}%
\pgftext[x=0.541667in,y=0.585000in,,]{\color{textcolor}\rmfamily\fontsize{9.000000}{10.800000}\selectfont \(\displaystyle b\)}%
\end{pgfscope}%
\begin{pgfscope}%
\definecolor{textcolor}{rgb}{0.000000,0.000000,0.000000}%
\pgfsetstrokecolor{textcolor}%
\pgfsetfillcolor{textcolor}%
\pgftext[x=1.269667in,y=0.320667in,,]{\color{textcolor}\rmfamily\fontsize{7.497000}{8.996400}\selectfont \(\displaystyle \frac{\theta}{2}\)}%
\end{pgfscope}%
\begin{pgfscope}%
\definecolor{textcolor}{rgb}{0.000000,0.000000,0.000000}%
\pgfsetstrokecolor{textcolor}%
\pgfsetfillcolor{textcolor}%
\pgftext[x=1.137717in,y=0.866667in,,]{\color{textcolor}\rmfamily\fontsize{7.497000}{8.996400}\selectfont \(\displaystyle \theta\)}%
\end{pgfscope}%
\end{pgfpicture}%
\makeatother%
\endgroup%

  \caption{Schematic representation of the scattering of a point-like particle
  off a hard shpere.}
  \label{fig:hard_sphere_xsec}
\end{marginfigure}

Many of the features of the cross section can be effectively illustrated with the
simple example of the diffusion of a point-like particle by a hard sphere with
infinite mass. The scattering angle as a function of the impact parameter $b$
is given by
\begin{align*}
  \sin\nicefrac{\theta}{2} = \frac{b}{r}
  \quad\text{or}\quad
  b = r \sin\nicefrac{\theta}{2}.
\end{align*}
Since the problem has azimuthal simmetry, the infinitesimal area orthogonal to
the particle flux that is scattered at an angle $\theta$ is given by
\begin{align*}
  d\sigma = 2\pi b \diff b = 2\pi b \abs{\dv{b}{\theta}} \diff\theta =
  2\pi \frac{r^2}{2} \sin\nicefrac{\theta}{2}\cos\nicefrac{\theta}{2}\diff\theta =
  2\pi \frac{r^2}{4} \sin\theta \diff\theta
\end{align*}
and the differential cross section
\begin{align}
  \dv{\sigma}{\Omega} = \frac{d\sigma}{2\pi \sin\theta \diff\theta} = \frac{r^2}{4}
\end{align}
does not explicitely depend on $\theta$.

The total cross section can be easily calculated integrating over the entire
solid angle and, not surprisingly, reads
\begin{align}
  \sigma = \int_\Omega \dv{\sigma}{\Omega} \diff\Omega =
  \int_0^\pi \int_0^{2\pi} \frac{r^2}{4} \sin\theta \diff\theta \diff\phi =
  4 \pi \frac{r^2}{4} = \pi r^2,
\end{align}
that is, is the physical area of the hard sphere, projected on the plane orthohonal
to the incident flux.


\section{The mean free path}

The average length that a particle traverses before interacting depends on both
the number density (that is, the number per unit volume) of the scattering centers
and the cross section, for a given process, of the single scattering center
\begin{align}
  \lambda = \frac{1}{n \sigma}.
\end{align}
This is correct from a dimensional standpoint (the number density has the dimension
of the inverse of a volume and $\sigma$ is an area), and meets the intuition that
the mean free path is smaller when the density of targets and/or the cross section
are larger.

Since the mean free path scales naturally with the (mass) density of the medium,
it is customary to give the same name \emph{mean free path} to the quantity
$\lambda \density$, measured in g~cm$^{-2}$, with the understanding that one has
to divide the numerical value of the latter by the density of the medium to recover
the actual length in the proper physical units (e.g., cm). This is especially useful
for gases, as one can express the mean free path for a given process as an
intrinsic property of a given mixture, irrespectively of the pressure.


\todo{Add Fermi golder rule?}
