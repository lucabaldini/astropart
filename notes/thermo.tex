\chapter{Thermodynamics}\label{chap:thermodynamics}


We review some basic facts of thermodynamics that will be useful in specific
sections of these notes.

From a thermodynamical standpoint, a homogeneous fluid is essentially described
by its pressure $\pressure$, its volume $V$ and its temperature $T$. These
three quantities are not independent, and related to each other by the
\emph{equation of state} of the sistem which, in the most general form, can
be written as
\begin{align}\label{eq:equation_of_state}
  f(\pressure, V, T) = 0.
\end{align}


\section{The first law}

The first law of thermodynamics encapsulates the conservation of energy for an
infinitesimal transformation in the form
\begin{align}\label{eq:thermo_first_law}
  dU + W = \delta Q
  \quad\text{or, equivalently,}\quad
  dU + \pressure dV = \delta Q.
\end{align}
For a given equation of state~\eqref{eq:equation_of_state} we can express the internal
energy (and the first law) in terms of any of the three possible sets of independent
variables:
\begin{align*}
  \parderr{U}{T}{V} dT + \left[ \parderr{U}{V}{T} + \pressure\right] dV = \delta Q
  \quad \{T, V\}\\
  \left[ \parderr{U}{T}{\pressure} + \pressure\parderr{V}{T}{\pressure} \;\; \right] dT +
  \left[ \parderr{U}{\pressure}{T} + \parderr{V}{\pressure}{T} \;\; \right] d\pressure = \delta Q
  \quad \{T, \pressure\}\\
  \parderr{U}{V}{\pressure} + \left[ \parderr{U}{\pressure}{V} + p \right] dV = \delta Q
  \quad \{V, \pressure\}
\end{align*}

The heat capacity is defined as the amount of heat to be supplied to an object to
produce a unit change in its temperature, and it is different depending
on whether the transformation takes place at constant volume or pressure. In
the first case
\begin{align}
  C_V = \left. \frac{\delta Q}{dT} \right|_V = \parderr{U}{T}{V}
\end{align}
while in the second
\begin{align}
  C_\pressure = \left. \frac{\delta Q}{dT} \right|_\pressure =
    \parderr{U}{T}{\pressure} + \pressure\parderr{V}{T}{\pressure}
\end{align}

We note that, while the heat capacity is an extensive quantity, it is customary
to use $C_V$ and $C_\pressure$ as the molar heat capacities, or the heat capacities
for  1~mol of substance.



\section{Ideal gases}

You might remember from high school the equation of state for a perfect gas, sometimes
referred to as the ideal gas law\sidenote{Note calligraphic $\mathcal{n}$ used to
indicate the number of moles, which should not be confused with $n$, indicating
the number density throughout these lecture notes.}
\begin{align*}
  \pressure V = \mathcal{n} RT = \mathcal{n} N_A \kT,
\end{align*}
where $R = N_A \kB$ is the ideal gas constant, that is, the product of the Boltzmann
constant and the Avogadro number.

Generally speaking, we don't like talking moles unless it is strictly necessary
to do so, and we prefer writing the ideal gas law in terms of the total number of
particles $N$
\begin{align*}
  \pressure V = N \kT,
\end{align*}
or, even better, in terms of the number density, since the latter is an intensive
quantity:
\begin{align}
  \pressure = \frac{N}{V} \kT = n\kT.
\end{align}
Now, according to the kinetic theory of gas, each degree of freedom provides
a contribution $\nicefrac{1}{2}kT$ to the internal energy, and therefore the latter
can be written as
\begin{align}\label{eq:ideal_gas_internal_energy}
  U = \frac{f}{2} N \kT \quad\text{and}\quad
  u = \frac{U}{V} = \frac{f}{2} \frac{N}{V} \kT = \frac{f}{2} n \kT,
\end{align}
where $f$ is the number of degrees of freedom for the single particle. For a monoatomic
gas ()$f = 3$) we recover the well know non-relativistic relation between the pressure
and the energy density
\begin{align}
  u = \frac{3}{2} n \kT = \frac{3}{2} \pressure.
\end{align}
We note that the internal energy per unit mass (as opposed to unit volume) $\mathcal{u}$,
which is also useful at times, reads
\begin{align}
  \mathcal{u} = \frac{U}{m} = \frac{U}{\density V} =
  \frac{3}{2} \frac{n \kT}{\density} = \frac{3}{2} \frac{\pressure}{\density}.
\end{align}

Since for an ideal gas the internal energy~\label{eq:ideal_gas_internal_energy} is
proportional to the temperature, the molar heat capacities are temperature
independent\sidenote{Note that the equation of state and the internal energy for
$1$~mole of perfect gas read
\begin{align*}
  \pressure V = RT \quad\text{and}\quad U = \frac{f}{2}RT.
\end{align*}}
\begin{align*}
  \begin{cases}
    C_V = \dv{U}{T} = \frac{f}{2}R\\[10pt]
    C_p = \dv{U}{T} + \pressure \parderr{V}{T}{\pressure} = C_V + R.
  \end{cases}
\end{align*}
The ratio between the specific heat at constant volume and pressure, generally referred
to as the \emph{adiabatic index} often comes up in calculations, and can
be explicitely written down as\sidenote{For a monoatomic gas, with only the three
translational degrees of freedom, $f = 3$ and therefore
\begin{align*}
  C_V = \frac{3}{2}R, \quad C_\pressure = \frac{5}{2}R
  \quad\text{and}\quad
  \adiabaticindex = \frac{5}{3}.
\end{align*}
}
\begin{align}
  \adiabaticindex = \frac{C_\pressure}{C_V} = \frac{C_V + R}{C_V} = \frac{f + 2}{f}.
\end{align}



\subsection{Adiabatic transformations}

Adiabatic transformations are reversible tranformations that happen with no
heat exchange with the external world. In this case the first law reads
\begin{align*}
  dU + \pressure dV = C_V dT + \pressure dV = 0,
\end{align*}
and, using the equation of state for a mole of ideal gas $\pressure V = RT$ to
eliminate the pressure, we have
\begin{align*}
  C_V dT + \frac{RT}{V} dV = 0
  \quad\text{or}\quad
  \frac{dT}{T} + \frac{R}{C_V} \frac{dV}{V} = 0.
\end{align*}
This can be easily integrated, and noting that $\nicefrac{R}{C_V} = \gamma - 1$,
yields the basic equation for adiabatic tranformations
\begin{align*}
  TV^{\gamma - 1} = \text{const},
\end{align*}
which, using again the equation of state, can be recasted in twe two equivalent
forms
\begin{align}
  \pressure V^\gamma = \text{const}
  \quad\text{or}\quad
  T\pressure^{-\frac{\gamma - 1}{\gamma}} = \text{const}.
\end{align}


\section{The second law}

The entropy is defined for an infinitesimal reversible transformation by
\begin{align}
  dS = \frac{\delta Q_\text{rev}}{T},
\end{align}
which can be integrated along any reversible transformation to calculate the
entropy difference between two arbitrary states $A$ and $B$
\begin{align*}
  S(B) - S(A) = \int_A^B \frac{\delta Q_\text{rev}}{T}.
\end{align*}
(The integral is independent from the particular tranformation as long as the latter
is reversible, and therefore the entropy is a function of state.)

In the general case (i.e., including irreversible tranformation) we have
\begin{align}
  S(B) - S(A) \geq \int_A^B \frac{\delta Q}{T},
\end{align}
from which it follows that for any tranformation of isolated system ($\delta Q = 0$),
the entropy cannot decrease---and it remains constant for a reversible
transformation.
