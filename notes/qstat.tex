\chapter{Quantum Statistics}
\label{chap:qstat}



\section{All the wonders of the $\zeta$ function}
\label{sec:riemann_zeta}

Well, some of them. The Riemann zeta function can be defined on the real line with
$x > 1$ as
\begin{align}
  \zeta(x) = \frac{1}{\Gamma(x)} \int_0^\infty \frac{z^{x-1}}{e^z - 1} dz,
\end{align}
where $\Gamma(x)$ is the gamma function. When the argument is an integer $n > 1$,
the definition can be rewritten as
\begin{align*}
  \zeta(n) = \frac{1}{(n - 1)!} \int_0^\infty \frac{z^{n-1}}{e^z - 1} dz,
\end{align*}
which can in turn be inverted to provide a closed expression for the entire class
of integrals
\begin{align}
  \int_0^\infty \frac{z^{n-1} dz}{e^z - 1} = (n - 1)! \zeta(n).
\end{align}
In addition, a related result can be demonstrated for the similar situation where
the sign at the denominator is swapped
\begin{align}
  \int_0^\infty \frac{z^{n-1} dz}{e^z + 1} = (1 - 2^{1 - n}) (n - 1)! \zeta(n).
\end{align}
(If you are wondering what this has to do with the title of this section, bear with
me one more second, as this kind of integrals will pop up very shortly when
dealing with the Bose-Einstein, Fermi-Dirac, and Planck distributions.)

As it turns out, the other important parametrization of the Riemann zeta function
is in the form of the infinite sum
\begin{align*}
  \zeta(n) = \sum_{k=1}^\infty \frac{1}{n^k}
\end{align*}
and, amusingly enough, this can be actually summed in closed form for all even
integers, but not for for odd numbers, in which case one has to resort to
numerical estimates, e.g.
\begin{align*}
  \zeta(2) = \frac{\pi^2}{6}, \quad \zeta(3) \approx 1.2020569032
  \quad \text{and} \quad \zeta(4) = \frac{\pi^4}{90}.
\end{align*}

We can piece things together to calculate a couple of notable definite integrals
we shall make extensive use of in the following:
\begin{align}\label{eq:qstats_notable_integrals}
  \int_0^\infty \frac{z^2}{e^z \mp 1} = 2\zeta(3)
  \begin{cases}
    \times 1\\
    \times \frac{3}{4}
  \end{cases}
  \quad \text{and} \quad
  \int_0^\infty \frac{z^3}{e^z \mp 1} = \frac{\pi^4}{15}
  \begin{cases}
    \times 1\\
    \times \frac{7}{8}
  \end{cases}.
\end{align}
(The case with the non trivial multiplicative factor is for a $+$ sign at the
denominator.)



\section{Quantum distribution functions}
\label{sec:quantum_statistics}

\begin{marginfigure}
  %% Creator: Matplotlib, PGF backend
%%
%% To include the figure in your LaTeX document, write
%%   \input{<filename>.pgf}
%%
%% Make sure the required packages are loaded in your preamble
%%   \usepackage{pgf}
%%
%% Also ensure that all the required font packages are loaded; for instance,
%% the lmodern package is sometimes necessary when using math font.
%%   \usepackage{lmodern}
%%
%% Figures using additional raster images can only be included by \input if
%% they are in the same directory as the main LaTeX file. For loading figures
%% from other directories you can use the `import` package
%%   \usepackage{import}
%%
%% and then include the figures with
%%   \import{<path to file>}{<filename>.pgf}
%%
%% Matplotlib used the following preamble
%%   \def\mathdefault#1{#1}
%%   \everymath=\expandafter{\the\everymath\displaystyle}
%%   \IfFileExists{scrextend.sty}{
%%     \usepackage[fontsize=9.000000pt]{scrextend}
%%   }{
%%     \renewcommand{\normalsize}{\fontsize{9.000000}{10.800000}\selectfont}
%%     \normalsize
%%   }
%%   
%%   \ifdefined\pdftexversion\else  % non-pdftex case.
%%     \usepackage{fontspec}
%%     \setmainfont{DejaVuSerif.ttf}[Path=\detokenize{/home/users/lbaldini/.pyenv/versions/3.13.1/lib/python3.13/site-packages/matplotlib/mpl-data/fonts/ttf/}]
%%     \setsansfont{DejaVuSans.ttf}[Path=\detokenize{/home/users/lbaldini/.pyenv/versions/3.13.1/lib/python3.13/site-packages/matplotlib/mpl-data/fonts/ttf/}]
%%     \setmonofont{DejaVuSansMono.ttf}[Path=\detokenize{/home/users/lbaldini/.pyenv/versions/3.13.1/lib/python3.13/site-packages/matplotlib/mpl-data/fonts/ttf/}]
%%   \fi
%%   \makeatletter\@ifpackageloaded{underscore}{}{\usepackage[strings]{underscore}}\makeatother
%%
\begingroup%
\makeatletter%
\begin{pgfpicture}%
\pgfpathrectangle{\pgfpointorigin}{\pgfqpoint{1.950000in}{2.500000in}}%
\pgfusepath{use as bounding box, clip}%
\begin{pgfscope}%
\pgfsetbuttcap%
\pgfsetmiterjoin%
\definecolor{currentfill}{rgb}{1.000000,1.000000,1.000000}%
\pgfsetfillcolor{currentfill}%
\pgfsetlinewidth{0.000000pt}%
\definecolor{currentstroke}{rgb}{1.000000,1.000000,1.000000}%
\pgfsetstrokecolor{currentstroke}%
\pgfsetdash{}{0pt}%
\pgfpathmoveto{\pgfqpoint{0.000000in}{0.000000in}}%
\pgfpathlineto{\pgfqpoint{1.950000in}{0.000000in}}%
\pgfpathlineto{\pgfqpoint{1.950000in}{2.500000in}}%
\pgfpathlineto{\pgfqpoint{0.000000in}{2.500000in}}%
\pgfpathlineto{\pgfqpoint{0.000000in}{0.000000in}}%
\pgfpathclose%
\pgfusepath{fill}%
\end{pgfscope}%
\begin{pgfscope}%
\pgfsetbuttcap%
\pgfsetmiterjoin%
\definecolor{currentfill}{rgb}{1.000000,1.000000,1.000000}%
\pgfsetfillcolor{currentfill}%
\pgfsetlinewidth{0.000000pt}%
\definecolor{currentstroke}{rgb}{0.000000,0.000000,0.000000}%
\pgfsetstrokecolor{currentstroke}%
\pgfsetstrokeopacity{0.000000}%
\pgfsetdash{}{0pt}%
\pgfpathmoveto{\pgfqpoint{0.243750in}{0.525000in}}%
\pgfpathlineto{\pgfqpoint{1.846250in}{0.525000in}}%
\pgfpathlineto{\pgfqpoint{1.846250in}{2.412500in}}%
\pgfpathlineto{\pgfqpoint{0.243750in}{2.412500in}}%
\pgfpathlineto{\pgfqpoint{0.243750in}{0.525000in}}%
\pgfpathclose%
\pgfusepath{fill}%
\end{pgfscope}%
\begin{pgfscope}%
\pgfpathrectangle{\pgfqpoint{0.243750in}{0.525000in}}{\pgfqpoint{1.602500in}{1.887500in}}%
\pgfusepath{clip}%
\pgfsetbuttcap%
\pgfsetroundjoin%
\pgfsetlinewidth{0.803000pt}%
\definecolor{currentstroke}{rgb}{0.752941,0.752941,0.752941}%
\pgfsetstrokecolor{currentstroke}%
\pgfsetdash{{2.960000pt}{1.280000pt}}{0.000000pt}%
\pgfpathmoveto{\pgfqpoint{1.045000in}{0.525000in}}%
\pgfpathlineto{\pgfqpoint{1.045000in}{2.412500in}}%
\pgfusepath{stroke}%
\end{pgfscope}%
\begin{pgfscope}%
\pgfsetbuttcap%
\pgfsetroundjoin%
\definecolor{currentfill}{rgb}{0.000000,0.000000,0.000000}%
\pgfsetfillcolor{currentfill}%
\pgfsetlinewidth{0.803000pt}%
\definecolor{currentstroke}{rgb}{0.000000,0.000000,0.000000}%
\pgfsetstrokecolor{currentstroke}%
\pgfsetdash{}{0pt}%
\pgfsys@defobject{currentmarker}{\pgfqpoint{0.000000in}{-0.048611in}}{\pgfqpoint{0.000000in}{0.000000in}}{%
\pgfpathmoveto{\pgfqpoint{0.000000in}{0.000000in}}%
\pgfpathlineto{\pgfqpoint{0.000000in}{-0.048611in}}%
\pgfusepath{stroke,fill}%
}%
\begin{pgfscope}%
\pgfsys@transformshift{1.045000in}{0.525000in}%
\pgfsys@useobject{currentmarker}{}%
\end{pgfscope}%
\end{pgfscope}%
\begin{pgfscope}%
\definecolor{textcolor}{rgb}{0.000000,0.000000,0.000000}%
\pgfsetstrokecolor{textcolor}%
\pgfsetfillcolor{textcolor}%
\pgftext[x=1.045000in,y=0.427778in,,top]{\color{textcolor}{\rmfamily\fontsize{9.000000}{10.800000}\selectfont\catcode`\^=\active\def^{\ifmmode\sp\else\^{}\fi}\catcode`\%=\active\def%{\%}$\mu$}}%
\end{pgfscope}%
\begin{pgfscope}%
\definecolor{textcolor}{rgb}{0.000000,0.000000,0.000000}%
\pgfsetstrokecolor{textcolor}%
\pgfsetfillcolor{textcolor}%
\pgftext[x=1.045000in,y=0.251251in,,top]{\color{textcolor}{\rmfamily\fontsize{9.000000}{10.800000}\selectfont\catcode`\^=\active\def^{\ifmmode\sp\else\^{}\fi}\catcode`\%=\active\def%{\%}Energy}}%
\end{pgfscope}%
\begin{pgfscope}%
\definecolor{textcolor}{rgb}{0.000000,0.000000,0.000000}%
\pgfsetstrokecolor{textcolor}%
\pgfsetfillcolor{textcolor}%
\pgftext[x=0.188194in,y=1.468750in,,bottom,rotate=90.000000]{\color{textcolor}{\rmfamily\fontsize{9.000000}{10.800000}\selectfont\catcode`\^=\active\def^{\ifmmode\sp\else\^{}\fi}\catcode`\%=\active\def%{\%}$h^3 dn/d^3p$}}%
\end{pgfscope}%
\begin{pgfscope}%
\pgfpathrectangle{\pgfqpoint{0.243750in}{0.525000in}}{\pgfqpoint{1.602500in}{1.887500in}}%
\pgfusepath{clip}%
\pgfsetrectcap%
\pgfsetroundjoin%
\pgfsetlinewidth{1.003750pt}%
\definecolor{currentstroke}{rgb}{0.000000,0.000000,0.000000}%
\pgfsetstrokecolor{currentstroke}%
\pgfsetdash{}{0pt}%
\pgfpathmoveto{\pgfqpoint{1.085879in}{2.422500in}}%
\pgfpathlineto{\pgfqpoint{1.101654in}{1.749010in}}%
\pgfpathlineto{\pgfqpoint{1.117841in}{1.374041in}}%
\pgfpathlineto{\pgfqpoint{1.134028in}{1.142517in}}%
\pgfpathlineto{\pgfqpoint{1.150215in}{0.987998in}}%
\pgfpathlineto{\pgfqpoint{1.166402in}{0.879449in}}%
\pgfpathlineto{\pgfqpoint{1.182588in}{0.800421in}}%
\pgfpathlineto{\pgfqpoint{1.198775in}{0.741378in}}%
\pgfpathlineto{\pgfqpoint{1.214962in}{0.696406in}}%
\pgfpathlineto{\pgfqpoint{1.231149in}{0.661647in}}%
\pgfpathlineto{\pgfqpoint{1.247336in}{0.634477in}}%
\pgfpathlineto{\pgfqpoint{1.263523in}{0.613050in}}%
\pgfpathlineto{\pgfqpoint{1.279710in}{0.596035in}}%
\pgfpathlineto{\pgfqpoint{1.295896in}{0.582448in}}%
\pgfpathlineto{\pgfqpoint{1.312083in}{0.571550in}}%
\pgfpathlineto{\pgfqpoint{1.328270in}{0.562780in}}%
\pgfpathlineto{\pgfqpoint{1.344457in}{0.555700in}}%
\pgfpathlineto{\pgfqpoint{1.360644in}{0.549973in}}%
\pgfpathlineto{\pgfqpoint{1.376831in}{0.545331in}}%
\pgfpathlineto{\pgfqpoint{1.393018in}{0.541563in}}%
\pgfpathlineto{\pgfqpoint{1.409205in}{0.538501in}}%
\pgfpathlineto{\pgfqpoint{1.425391in}{0.536010in}}%
\pgfpathlineto{\pgfqpoint{1.441578in}{0.533981in}}%
\pgfpathlineto{\pgfqpoint{1.457765in}{0.532329in}}%
\pgfpathlineto{\pgfqpoint{1.473952in}{0.530982in}}%
\pgfpathlineto{\pgfqpoint{1.490139in}{0.529883in}}%
\pgfpathlineto{\pgfqpoint{1.506326in}{0.528987in}}%
\pgfpathlineto{\pgfqpoint{1.522513in}{0.528256in}}%
\pgfpathlineto{\pgfqpoint{1.538699in}{0.527659in}}%
\pgfpathlineto{\pgfqpoint{1.554886in}{0.527172in}}%
\pgfpathlineto{\pgfqpoint{1.571073in}{0.526774in}}%
\pgfpathlineto{\pgfqpoint{1.587260in}{0.526449in}}%
\pgfpathlineto{\pgfqpoint{1.603447in}{0.526184in}}%
\pgfpathlineto{\pgfqpoint{1.619634in}{0.525967in}}%
\pgfpathlineto{\pgfqpoint{1.635821in}{0.525790in}}%
\pgfpathlineto{\pgfqpoint{1.652008in}{0.525646in}}%
\pgfpathlineto{\pgfqpoint{1.668194in}{0.525527in}}%
\pgfpathlineto{\pgfqpoint{1.684381in}{0.525431in}}%
\pgfpathlineto{\pgfqpoint{1.700568in}{0.525352in}}%
\pgfpathlineto{\pgfqpoint{1.716755in}{0.525288in}}%
\pgfpathlineto{\pgfqpoint{1.732942in}{0.525235in}}%
\pgfpathlineto{\pgfqpoint{1.749129in}{0.525192in}}%
\pgfpathlineto{\pgfqpoint{1.765316in}{0.525157in}}%
\pgfpathlineto{\pgfqpoint{1.781503in}{0.525128in}}%
\pgfpathlineto{\pgfqpoint{1.797689in}{0.525105in}}%
\pgfpathlineto{\pgfqpoint{1.813876in}{0.525086in}}%
\pgfpathlineto{\pgfqpoint{1.830063in}{0.525070in}}%
\pgfpathlineto{\pgfqpoint{1.846250in}{0.525057in}}%
\pgfusepath{stroke}%
\end{pgfscope}%
\begin{pgfscope}%
\pgfpathrectangle{\pgfqpoint{0.243750in}{0.525000in}}{\pgfqpoint{1.602500in}{1.887500in}}%
\pgfusepath{clip}%
\pgfsetrectcap%
\pgfsetroundjoin%
\pgfsetlinewidth{1.003750pt}%
\definecolor{currentstroke}{rgb}{0.000000,0.000000,0.000000}%
\pgfsetstrokecolor{currentstroke}%
\pgfsetdash{}{0pt}%
\pgfpathmoveto{\pgfqpoint{0.243750in}{1.783276in}}%
\pgfpathlineto{\pgfqpoint{0.259937in}{1.783263in}}%
\pgfpathlineto{\pgfqpoint{0.276124in}{1.783248in}}%
\pgfpathlineto{\pgfqpoint{0.292311in}{1.783229in}}%
\pgfpathlineto{\pgfqpoint{0.308497in}{1.783205in}}%
\pgfpathlineto{\pgfqpoint{0.324684in}{1.783176in}}%
\pgfpathlineto{\pgfqpoint{0.340871in}{1.783141in}}%
\pgfpathlineto{\pgfqpoint{0.357058in}{1.783098in}}%
\pgfpathlineto{\pgfqpoint{0.373245in}{1.783046in}}%
\pgfpathlineto{\pgfqpoint{0.389432in}{1.782981in}}%
\pgfpathlineto{\pgfqpoint{0.405619in}{1.782903in}}%
\pgfpathlineto{\pgfqpoint{0.421806in}{1.782806in}}%
\pgfpathlineto{\pgfqpoint{0.437992in}{1.782688in}}%
\pgfpathlineto{\pgfqpoint{0.454179in}{1.782544in}}%
\pgfpathlineto{\pgfqpoint{0.470366in}{1.782368in}}%
\pgfpathlineto{\pgfqpoint{0.486553in}{1.782152in}}%
\pgfpathlineto{\pgfqpoint{0.502740in}{1.781887in}}%
\pgfpathlineto{\pgfqpoint{0.518927in}{1.781564in}}%
\pgfpathlineto{\pgfqpoint{0.535114in}{1.781169in}}%
\pgfpathlineto{\pgfqpoint{0.551301in}{1.780685in}}%
\pgfpathlineto{\pgfqpoint{0.567487in}{1.780094in}}%
\pgfpathlineto{\pgfqpoint{0.583674in}{1.779371in}}%
\pgfpathlineto{\pgfqpoint{0.599861in}{1.778487in}}%
\pgfpathlineto{\pgfqpoint{0.616048in}{1.777408in}}%
\pgfpathlineto{\pgfqpoint{0.632235in}{1.776089in}}%
\pgfpathlineto{\pgfqpoint{0.648422in}{1.774478in}}%
\pgfpathlineto{\pgfqpoint{0.664609in}{1.772513in}}%
\pgfpathlineto{\pgfqpoint{0.680795in}{1.770116in}}%
\pgfpathlineto{\pgfqpoint{0.696982in}{1.767195in}}%
\pgfpathlineto{\pgfqpoint{0.713169in}{1.763639in}}%
\pgfpathlineto{\pgfqpoint{0.729356in}{1.759314in}}%
\pgfpathlineto{\pgfqpoint{0.745543in}{1.754061in}}%
\pgfpathlineto{\pgfqpoint{0.761730in}{1.747694in}}%
\pgfpathlineto{\pgfqpoint{0.777917in}{1.739990in}}%
\pgfpathlineto{\pgfqpoint{0.794104in}{1.730692in}}%
\pgfpathlineto{\pgfqpoint{0.810290in}{1.719505in}}%
\pgfpathlineto{\pgfqpoint{0.826477in}{1.706093in}}%
\pgfpathlineto{\pgfqpoint{0.842664in}{1.690082in}}%
\pgfpathlineto{\pgfqpoint{0.858851in}{1.671069in}}%
\pgfpathlineto{\pgfqpoint{0.875038in}{1.648626in}}%
\pgfpathlineto{\pgfqpoint{0.891225in}{1.622327in}}%
\pgfpathlineto{\pgfqpoint{0.907412in}{1.591770in}}%
\pgfpathlineto{\pgfqpoint{0.923598in}{1.556611in}}%
\pgfpathlineto{\pgfqpoint{0.939785in}{1.516613in}}%
\pgfpathlineto{\pgfqpoint{0.955972in}{1.471690in}}%
\pgfpathlineto{\pgfqpoint{0.972159in}{1.421958in}}%
\pgfpathlineto{\pgfqpoint{0.988346in}{1.367773in}}%
\pgfpathlineto{\pgfqpoint{1.004533in}{1.309754in}}%
\pgfpathlineto{\pgfqpoint{1.020720in}{1.248772in}}%
\pgfpathlineto{\pgfqpoint{1.036907in}{1.185916in}}%
\pgfpathlineto{\pgfqpoint{1.053093in}{1.122418in}}%
\pgfpathlineto{\pgfqpoint{1.069280in}{1.059561in}}%
\pgfpathlineto{\pgfqpoint{1.085467in}{0.998579in}}%
\pgfpathlineto{\pgfqpoint{1.101654in}{0.940560in}}%
\pgfpathlineto{\pgfqpoint{1.117841in}{0.886376in}}%
\pgfpathlineto{\pgfqpoint{1.134028in}{0.836644in}}%
\pgfpathlineto{\pgfqpoint{1.150215in}{0.791721in}}%
\pgfpathlineto{\pgfqpoint{1.166402in}{0.751722in}}%
\pgfpathlineto{\pgfqpoint{1.182588in}{0.716563in}}%
\pgfpathlineto{\pgfqpoint{1.198775in}{0.686006in}}%
\pgfpathlineto{\pgfqpoint{1.214962in}{0.659707in}}%
\pgfpathlineto{\pgfqpoint{1.231149in}{0.637265in}}%
\pgfpathlineto{\pgfqpoint{1.247336in}{0.618251in}}%
\pgfpathlineto{\pgfqpoint{1.263523in}{0.602240in}}%
\pgfpathlineto{\pgfqpoint{1.279710in}{0.588828in}}%
\pgfpathlineto{\pgfqpoint{1.295896in}{0.577641in}}%
\pgfpathlineto{\pgfqpoint{1.312083in}{0.568344in}}%
\pgfpathlineto{\pgfqpoint{1.328270in}{0.560640in}}%
\pgfpathlineto{\pgfqpoint{1.344457in}{0.554272in}}%
\pgfpathlineto{\pgfqpoint{1.360644in}{0.549020in}}%
\pgfpathlineto{\pgfqpoint{1.376831in}{0.544695in}}%
\pgfpathlineto{\pgfqpoint{1.393018in}{0.541138in}}%
\pgfpathlineto{\pgfqpoint{1.409205in}{0.538217in}}%
\pgfpathlineto{\pgfqpoint{1.425391in}{0.535820in}}%
\pgfpathlineto{\pgfqpoint{1.441578in}{0.533855in}}%
\pgfpathlineto{\pgfqpoint{1.457765in}{0.532245in}}%
\pgfpathlineto{\pgfqpoint{1.473952in}{0.530926in}}%
\pgfpathlineto{\pgfqpoint{1.490139in}{0.529846in}}%
\pgfpathlineto{\pgfqpoint{1.506326in}{0.528962in}}%
\pgfpathlineto{\pgfqpoint{1.522513in}{0.528239in}}%
\pgfpathlineto{\pgfqpoint{1.538699in}{0.527648in}}%
\pgfpathlineto{\pgfqpoint{1.554886in}{0.527164in}}%
\pgfpathlineto{\pgfqpoint{1.571073in}{0.526769in}}%
\pgfpathlineto{\pgfqpoint{1.587260in}{0.526446in}}%
\pgfpathlineto{\pgfqpoint{1.603447in}{0.526182in}}%
\pgfpathlineto{\pgfqpoint{1.619634in}{0.525966in}}%
\pgfpathlineto{\pgfqpoint{1.635821in}{0.525789in}}%
\pgfpathlineto{\pgfqpoint{1.652008in}{0.525645in}}%
\pgfpathlineto{\pgfqpoint{1.668194in}{0.525527in}}%
\pgfpathlineto{\pgfqpoint{1.684381in}{0.525431in}}%
\pgfpathlineto{\pgfqpoint{1.700568in}{0.525352in}}%
\pgfpathlineto{\pgfqpoint{1.716755in}{0.525288in}}%
\pgfpathlineto{\pgfqpoint{1.732942in}{0.525235in}}%
\pgfpathlineto{\pgfqpoint{1.749129in}{0.525192in}}%
\pgfpathlineto{\pgfqpoint{1.765316in}{0.525157in}}%
\pgfpathlineto{\pgfqpoint{1.781503in}{0.525128in}}%
\pgfpathlineto{\pgfqpoint{1.797689in}{0.525105in}}%
\pgfpathlineto{\pgfqpoint{1.813876in}{0.525086in}}%
\pgfpathlineto{\pgfqpoint{1.830063in}{0.525070in}}%
\pgfpathlineto{\pgfqpoint{1.846250in}{0.525057in}}%
\pgfusepath{stroke}%
\end{pgfscope}%
\begin{pgfscope}%
\pgfpathrectangle{\pgfqpoint{0.243750in}{0.525000in}}{\pgfqpoint{1.602500in}{1.887500in}}%
\pgfusepath{clip}%
\pgfsetbuttcap%
\pgfsetroundjoin%
\pgfsetlinewidth{1.003750pt}%
\definecolor{currentstroke}{rgb}{0.000000,0.000000,0.000000}%
\pgfsetstrokecolor{currentstroke}%
\pgfsetdash{{3.700000pt}{1.600000pt}}{0.000000pt}%
\pgfpathmoveto{\pgfqpoint{1.045000in}{0.525000in}}%
\pgfpathlineto{\pgfqpoint{1.045000in}{2.412500in}}%
\pgfusepath{stroke}%
\end{pgfscope}%
\begin{pgfscope}%
\pgfpathrectangle{\pgfqpoint{0.243750in}{0.525000in}}{\pgfqpoint{1.602500in}{1.887500in}}%
\pgfusepath{clip}%
\pgfsetbuttcap%
\pgfsetroundjoin%
\pgfsetlinewidth{0.501875pt}%
\definecolor{currentstroke}{rgb}{0.827451,0.827451,0.827451}%
\pgfsetstrokecolor{currentstroke}%
\pgfsetdash{{1.850000pt}{0.800000pt}}{0.000000pt}%
\pgfpathmoveto{\pgfqpoint{0.243750in}{1.783333in}}%
\pgfpathlineto{\pgfqpoint{1.846250in}{1.783333in}}%
\pgfusepath{stroke}%
\end{pgfscope}%
\begin{pgfscope}%
\pgfpathrectangle{\pgfqpoint{0.243750in}{0.525000in}}{\pgfqpoint{1.602500in}{1.887500in}}%
\pgfusepath{clip}%
\pgfsetbuttcap%
\pgfsetroundjoin%
\pgfsetlinewidth{0.501875pt}%
\definecolor{currentstroke}{rgb}{0.827451,0.827451,0.827451}%
\pgfsetstrokecolor{currentstroke}%
\pgfsetdash{{1.850000pt}{0.800000pt}}{0.000000pt}%
\pgfpathmoveto{\pgfqpoint{0.243750in}{1.154167in}}%
\pgfpathlineto{\pgfqpoint{1.846250in}{1.154167in}}%
\pgfusepath{stroke}%
\end{pgfscope}%
\begin{pgfscope}%
\pgfsetrectcap%
\pgfsetmiterjoin%
\pgfsetlinewidth{1.003750pt}%
\definecolor{currentstroke}{rgb}{0.000000,0.000000,0.000000}%
\pgfsetstrokecolor{currentstroke}%
\pgfsetdash{}{0pt}%
\pgfpathmoveto{\pgfqpoint{0.243750in}{0.525000in}}%
\pgfpathlineto{\pgfqpoint{0.243750in}{2.412500in}}%
\pgfusepath{stroke}%
\end{pgfscope}%
\begin{pgfscope}%
\pgfsetrectcap%
\pgfsetmiterjoin%
\pgfsetlinewidth{1.003750pt}%
\definecolor{currentstroke}{rgb}{0.000000,0.000000,0.000000}%
\pgfsetstrokecolor{currentstroke}%
\pgfsetdash{}{0pt}%
\pgfpathmoveto{\pgfqpoint{1.846250in}{0.525000in}}%
\pgfpathlineto{\pgfqpoint{1.846250in}{2.412500in}}%
\pgfusepath{stroke}%
\end{pgfscope}%
\begin{pgfscope}%
\pgfsetrectcap%
\pgfsetmiterjoin%
\pgfsetlinewidth{1.003750pt}%
\definecolor{currentstroke}{rgb}{0.000000,0.000000,0.000000}%
\pgfsetstrokecolor{currentstroke}%
\pgfsetdash{}{0pt}%
\pgfpathmoveto{\pgfqpoint{0.243750in}{0.525000in}}%
\pgfpathlineto{\pgfqpoint{1.846250in}{0.525000in}}%
\pgfusepath{stroke}%
\end{pgfscope}%
\begin{pgfscope}%
\pgfsetrectcap%
\pgfsetmiterjoin%
\pgfsetlinewidth{1.003750pt}%
\definecolor{currentstroke}{rgb}{0.000000,0.000000,0.000000}%
\pgfsetstrokecolor{currentstroke}%
\pgfsetdash{}{0pt}%
\pgfpathmoveto{\pgfqpoint{0.243750in}{2.412500in}}%
\pgfpathlineto{\pgfqpoint{1.846250in}{2.412500in}}%
\pgfusepath{stroke}%
\end{pgfscope}%
\begin{pgfscope}%
\definecolor{textcolor}{rgb}{0.000000,0.000000,0.000000}%
\pgfsetstrokecolor{textcolor}%
\pgfsetfillcolor{textcolor}%
\pgftext[x=0.283813in,y=1.846250in,left,base]{\color{textcolor}{\rmfamily\fontsize{7.497000}{8.996400}\selectfont\catcode`\^=\active\def^{\ifmmode\sp\else\^{}\fi}\catcode`\%=\active\def%{\%}Fermi-Dirac}}%
\end{pgfscope}%
\begin{pgfscope}%
\definecolor{textcolor}{rgb}{0.000000,0.000000,0.000000}%
\pgfsetstrokecolor{textcolor}%
\pgfsetfillcolor{textcolor}%
\pgftext[x=1.165188in,y=2.097917in,left,base]{\color{textcolor}{\rmfamily\fontsize{7.497000}{8.996400}\selectfont\catcode`\^=\active\def^{\ifmmode\sp\else\^{}\fi}\catcode`\%=\active\def%{\%}Bose-Einstein}}%
\end{pgfscope}%
\end{pgfpicture}%
\makeatother%
\endgroup%

  \caption{Illustration of the Bose-Einstein and Fermi-Dirac distributions, for
    $\mu = 10 \kT$. Note that for the Fermi-Dirac distribution the chemical
    potential corresponds to the energy where the differential number density is
    50\% of the maximum value. In the Bose-Einstein case, instead $E \geq \mu$ for
    the distribution to be positive-defined.}
  \label{fig:quantum_statistics}
\end{marginfigure}

A system of $N$ identical, non interacting bosons/fermions in thermodynamic equilibrium
follows the Bose-Einstein/Fermi-Dirac distribution, respectively, according to which
the density of particles per unit volume in the phase space is
\begin{align*}
  \frac{dN}{\nicefrac{d^3p\,dV}{h^3}} = \frac{g}{e^\frac{E - \mu}{\kT} \mp 1} =
  \begin{cases}
    \frac{\displaystyle g}{\displaystyle e^\frac{E - \mu}{\kT} - 1} &
    \quad\text{for bosons (Bose-Einstein)}\\[12pt]
    \frac{\displaystyle g}{\displaystyle e^\frac{E - \mu}{\kT} + 1} &
    \quad\text{for fermions (Fermi Dirac)},
  \end{cases}
\end{align*}
where $T$ is the temperature, $\mu$ is the chemical potential\sidenote{The chemical
potential is fixed by the condition that the total number of particle in the system
is given by integrating the density function over the entire phase space.}, and $g$
is the number of internal degrees of freedom of the single particle\sidenote{This
is, in general, $2s + 1$ for a spin-$s$ particle. For a spin $\nicefrac{1}{2}$ fermion
such an electron $g = 2$, which encapsulates the usual Pauli exclusion principle,
stating that no two such particles can occupy the same quantum state.}. Since it
is generally more convenient to work with intensive quantities, we shall recast
the expression in terms of the number density $n = \nicefrac{N}{V}$. In addition,
in an isotropic setting, the differential momentum is more conveniently written in
polar coordinates, where the angular part is trivial, i.e., $d^3p = 4\pi p^2 dp$.
With that, the probability density function for the number density in momentum space,
at a fixed temperature and chemical potential is given by
\begin{align}\label{eq:qstats}
  \frac{dn}{dp}(p; T, \mu) =
  \frac{4\pi g}{h^3} \frac{p^2}{e^\frac{E - \mu}{\kT} \mp 1},
\end{align}
with the usual convention that we take the $-$~sign for bosons and the $+$~sign
for fermions. (Make sure you understand the implications of the $p^2$ at the
numerator, here, because the distributions in $\abs{p}$ space does not look at all
like the plots in figure~\ref{fig:quantum_statistics}.)

Equation~\eqref{eq:qstats} is the starting point for many of the topics we shall
touch upon, and we shall use it extensively. The overall number density and internal
energy density of an arbitrary gas of particles are readily calculated
\begin{align*}
  n = \int_0^\infty \frac{dn}{dp}(p) dp \quad \text{and} \quad
  u = \frac{U}{V} = \int_0^\infty E_k(p) \frac{dn}{dp}(p) dp
\end{align*}
once the dispersion relation, relating the kinetic energy to the corresponding
momentum, is known
\begin{align*}
  E_k = mc^2 \left[ \sqrt{1 + \left(\frac{p}{mc} \right)^2} - 1\right] \approx
  \begin{cases}
    \frac{p^2}{2m} & \quad \text{(non-relativistic)} \\
    pc & \quad \text{(ultra-relativistic)}.
  \end{cases}
\end{align*}

The pressure is slightly more complicated, but essentially it can be defined as
the flux of momentum per unit surface, integrated over all particles\sidenote{By
unit surface we mean a surface of unit area oriented in an arbitrary direction,
e.g., any of the axes in a given orthogonal reference frame. Since the velocity
momentum distributions are isotropic we have
\begin{align*}
  \ave{v_x p_x} = \ave{v_y p_y} = \ave{v_z p_z} = \frac{1}{3} vp.
\end{align*}
(And if you are unsure about the $\nicefrac{1}{3}$, remember that both $v$ and $p$
get their own factor $\nicefrac{1}{\sqrt{3}}$.)
}:
\begin{align*}
  \pressure = \frac{1}{3} \int_0^\infty v(p) p \frac{dn}{dp}(p) dp.
\end{align*}
The only additional piece of information that we need is the relation between the
momentum, and the velocity, which is readily calculated
\begin{align*}
  v = \frac{pc^2}{E} =
  \frac{p}{m} \left[1 + \left(\frac{p}{mc} \right)^2 \right]^{-\frac{1}{2}} \approx
  \begin{cases}
    \frac{p}{m} & \quad \text{(non-relativistic)} \\
    c & \quad \text{(ultra-relativistic)}.
  \end{cases}
\end{align*}

All of this is trivial in principle, quite convoluted in practice. In the non-relativistic
limit, where $E_k \approx \nicefrac{p^2}{2m}$ and $v \approx \nicefrac{p}{m}$
the expression for the energy density and the pressure read
\begin{align*}
  u = \int_0^\infty \frac{p^2}{2m} \frac{dn}{dp}(p) dp \quad \text{and} \quad
  \pressure = \frac{1}{3} \int_0^\infty \frac{p^2}{m} \frac{dn}{dp}(p) dp,
\end{align*}
while in the ultra-relativistic limit, where $E_k \approx E \approx pc$ and $v \approx c$
\begin{align*}
  u = \int_0^\infty pc \frac{dn}{dp}(p) dp \quad \text{and} \quad
  \pressure = \frac{1}{3} \int_0^\infty pc \frac{dn}{dp}(p) dp.
\end{align*}
Take note of the asymptotic relations between the pressure and the energy
density\sidenote{Note that, although they typically expressed in different units,
pressure (force per unit surface) and energy density (energy per unit volume) are
homogeneous quantities.} in
the two cases
\begin{align}
  P =
  \begin{cases}
    \frac{2}{3} u \quad \text{(non-relativistic)}\\
    \frac{1}{3} u \quad \text{(ultra-relativistic)}
  \end{cases}
\end{align}
because we shall encounter them again and again. The most familiar example of this
relation between pressure and energy density might very well be that of an ideal
gas, where
\begin{align*}
  P = n\kB T \quad\text{and}\quad u = \frac{3}{2} \kB T.
\end{align*}



\subsection{The degenerate Fermi gas}
\label{sec:fermi_gas}

This is the right time to spend a moment on a topic that will turn out to be handy
when discussing degenerate stars---that is, the degenerate Fermi gas.
Let us imagine a system of $N$ identical fermions (e.g., electrons) at the absolute
zero. Since the Pauli exclusion principle dictates that no more than 2 such electrons
can occupy the same energy level, the system possess a finite kinetic energy (and,
therefore, pressure) even for $T = 0$. The resulting pressure, which has no classical
counterpart, is customarily referred to as the \emph{degeneracy pressure}.

\begin{marginfigure}
  %% Creator: Matplotlib, PGF backend
%%
%% To include the figure in your LaTeX document, write
%%   \input{<filename>.pgf}
%%
%% Make sure the required packages are loaded in your preamble
%%   \usepackage{pgf}
%%
%% Also ensure that all the required font packages are loaded; for instance,
%% the lmodern package is sometimes necessary when using math font.
%%   \usepackage{lmodern}
%%
%% Figures using additional raster images can only be included by \input if
%% they are in the same directory as the main LaTeX file. For loading figures
%% from other directories you can use the `import` package
%%   \usepackage{import}
%%
%% and then include the figures with
%%   \import{<path to file>}{<filename>.pgf}
%%
%% Matplotlib used the following preamble
%%   \def\mathdefault#1{#1}
%%   \everymath=\expandafter{\the\everymath\displaystyle}
%%   \IfFileExists{scrextend.sty}{
%%     \usepackage[fontsize=9.000000pt]{scrextend}
%%   }{
%%     \renewcommand{\normalsize}{\fontsize{9.000000}{10.800000}\selectfont}
%%     \normalsize
%%   }
%%   
%%   \ifdefined\pdftexversion\else  % non-pdftex case.
%%     \usepackage{fontspec}
%%     \setmainfont{DejaVuSerif.ttf}[Path=\detokenize{/home/users/lbaldini/.pyenv/versions/3.13.1/lib/python3.13/site-packages/matplotlib/mpl-data/fonts/ttf/}]
%%     \setsansfont{DejaVuSans.ttf}[Path=\detokenize{/home/users/lbaldini/.pyenv/versions/3.13.1/lib/python3.13/site-packages/matplotlib/mpl-data/fonts/ttf/}]
%%     \setmonofont{DejaVuSansMono.ttf}[Path=\detokenize{/home/users/lbaldini/.pyenv/versions/3.13.1/lib/python3.13/site-packages/matplotlib/mpl-data/fonts/ttf/}]
%%   \fi
%%   \makeatletter\@ifpackageloaded{underscore}{}{\usepackage[strings]{underscore}}\makeatother
%%
\begingroup%
\makeatletter%
\begin{pgfpicture}%
\pgfpathrectangle{\pgfpointorigin}{\pgfqpoint{1.950000in}{2.500000in}}%
\pgfusepath{use as bounding box, clip}%
\begin{pgfscope}%
\pgfsetbuttcap%
\pgfsetmiterjoin%
\definecolor{currentfill}{rgb}{1.000000,1.000000,1.000000}%
\pgfsetfillcolor{currentfill}%
\pgfsetlinewidth{0.000000pt}%
\definecolor{currentstroke}{rgb}{1.000000,1.000000,1.000000}%
\pgfsetstrokecolor{currentstroke}%
\pgfsetdash{}{0pt}%
\pgfpathmoveto{\pgfqpoint{0.000000in}{0.000000in}}%
\pgfpathlineto{\pgfqpoint{1.950000in}{0.000000in}}%
\pgfpathlineto{\pgfqpoint{1.950000in}{2.500000in}}%
\pgfpathlineto{\pgfqpoint{0.000000in}{2.500000in}}%
\pgfpathlineto{\pgfqpoint{0.000000in}{0.000000in}}%
\pgfpathclose%
\pgfusepath{fill}%
\end{pgfscope}%
\begin{pgfscope}%
\pgfsetbuttcap%
\pgfsetmiterjoin%
\definecolor{currentfill}{rgb}{1.000000,1.000000,1.000000}%
\pgfsetfillcolor{currentfill}%
\pgfsetlinewidth{0.000000pt}%
\definecolor{currentstroke}{rgb}{0.000000,0.000000,0.000000}%
\pgfsetstrokecolor{currentstroke}%
\pgfsetstrokeopacity{0.000000}%
\pgfsetdash{}{0pt}%
\pgfpathmoveto{\pgfqpoint{0.243750in}{0.525000in}}%
\pgfpathlineto{\pgfqpoint{1.846250in}{0.525000in}}%
\pgfpathlineto{\pgfqpoint{1.846250in}{2.412500in}}%
\pgfpathlineto{\pgfqpoint{0.243750in}{2.412500in}}%
\pgfpathlineto{\pgfqpoint{0.243750in}{0.525000in}}%
\pgfpathclose%
\pgfusepath{fill}%
\end{pgfscope}%
\begin{pgfscope}%
\pgfpathrectangle{\pgfqpoint{0.243750in}{0.525000in}}{\pgfqpoint{1.602500in}{1.887500in}}%
\pgfusepath{clip}%
\pgfsetbuttcap%
\pgfsetroundjoin%
\pgfsetlinewidth{0.803000pt}%
\definecolor{currentstroke}{rgb}{0.752941,0.752941,0.752941}%
\pgfsetstrokecolor{currentstroke}%
\pgfsetdash{{2.960000pt}{1.280000pt}}{0.000000pt}%
\pgfpathmoveto{\pgfqpoint{1.045000in}{0.525000in}}%
\pgfpathlineto{\pgfqpoint{1.045000in}{2.412500in}}%
\pgfusepath{stroke}%
\end{pgfscope}%
\begin{pgfscope}%
\pgfsetbuttcap%
\pgfsetroundjoin%
\definecolor{currentfill}{rgb}{0.000000,0.000000,0.000000}%
\pgfsetfillcolor{currentfill}%
\pgfsetlinewidth{0.803000pt}%
\definecolor{currentstroke}{rgb}{0.000000,0.000000,0.000000}%
\pgfsetstrokecolor{currentstroke}%
\pgfsetdash{}{0pt}%
\pgfsys@defobject{currentmarker}{\pgfqpoint{0.000000in}{-0.048611in}}{\pgfqpoint{0.000000in}{0.000000in}}{%
\pgfpathmoveto{\pgfqpoint{0.000000in}{0.000000in}}%
\pgfpathlineto{\pgfqpoint{0.000000in}{-0.048611in}}%
\pgfusepath{stroke,fill}%
}%
\begin{pgfscope}%
\pgfsys@transformshift{1.045000in}{0.525000in}%
\pgfsys@useobject{currentmarker}{}%
\end{pgfscope}%
\end{pgfscope}%
\begin{pgfscope}%
\definecolor{textcolor}{rgb}{0.000000,0.000000,0.000000}%
\pgfsetstrokecolor{textcolor}%
\pgfsetfillcolor{textcolor}%
\pgftext[x=1.045000in,y=0.427778in,,top]{\color{textcolor}{\rmfamily\fontsize{9.000000}{10.800000}\selectfont\catcode`\^=\active\def^{\ifmmode\sp\else\^{}\fi}\catcode`\%=\active\def%{\%}$\mu$}}%
\end{pgfscope}%
\begin{pgfscope}%
\definecolor{textcolor}{rgb}{0.000000,0.000000,0.000000}%
\pgfsetstrokecolor{textcolor}%
\pgfsetfillcolor{textcolor}%
\pgftext[x=1.045000in,y=0.251251in,,top]{\color{textcolor}{\rmfamily\fontsize{9.000000}{10.800000}\selectfont\catcode`\^=\active\def^{\ifmmode\sp\else\^{}\fi}\catcode`\%=\active\def%{\%}Energy}}%
\end{pgfscope}%
\begin{pgfscope}%
\definecolor{textcolor}{rgb}{0.000000,0.000000,0.000000}%
\pgfsetstrokecolor{textcolor}%
\pgfsetfillcolor{textcolor}%
\pgftext[x=0.188194in,y=1.468750in,,bottom,rotate=90.000000]{\color{textcolor}{\rmfamily\fontsize{9.000000}{10.800000}\selectfont\catcode`\^=\active\def^{\ifmmode\sp\else\^{}\fi}\catcode`\%=\active\def%{\%}$h^3 dn/d^3p$}}%
\end{pgfscope}%
\begin{pgfscope}%
\pgfpathrectangle{\pgfqpoint{0.243750in}{0.525000in}}{\pgfqpoint{1.602500in}{1.887500in}}%
\pgfusepath{clip}%
\pgfsetrectcap%
\pgfsetroundjoin%
\pgfsetlinewidth{1.003750pt}%
\definecolor{currentstroke}{rgb}{0.000000,0.000000,0.000000}%
\pgfsetstrokecolor{currentstroke}%
\pgfsetdash{}{0pt}%
\pgfpathmoveto{\pgfqpoint{0.243750in}{1.783276in}}%
\pgfpathlineto{\pgfqpoint{0.494744in}{1.782025in}}%
\pgfpathlineto{\pgfqpoint{0.578409in}{1.779622in}}%
\pgfpathlineto{\pgfqpoint{0.629895in}{1.776296in}}%
\pgfpathlineto{\pgfqpoint{0.668509in}{1.771978in}}%
\pgfpathlineto{\pgfqpoint{0.700688in}{1.766441in}}%
\pgfpathlineto{\pgfqpoint{0.726431in}{1.760159in}}%
\pgfpathlineto{\pgfqpoint{0.752174in}{1.751600in}}%
\pgfpathlineto{\pgfqpoint{0.771481in}{1.743228in}}%
\pgfpathlineto{\pgfqpoint{0.790788in}{1.732740in}}%
\pgfpathlineto{\pgfqpoint{0.810095in}{1.719652in}}%
\pgfpathlineto{\pgfqpoint{0.829403in}{1.703403in}}%
\pgfpathlineto{\pgfqpoint{0.848710in}{1.683355in}}%
\pgfpathlineto{\pgfqpoint{0.868017in}{1.658809in}}%
\pgfpathlineto{\pgfqpoint{0.887324in}{1.629039in}}%
\pgfpathlineto{\pgfqpoint{0.906632in}{1.593346in}}%
\pgfpathlineto{\pgfqpoint{0.925939in}{1.551131in}}%
\pgfpathlineto{\pgfqpoint{0.945246in}{1.502007in}}%
\pgfpathlineto{\pgfqpoint{0.964553in}{1.445908in}}%
\pgfpathlineto{\pgfqpoint{0.983860in}{1.383204in}}%
\pgfpathlineto{\pgfqpoint{1.009603in}{1.290922in}}%
\pgfpathlineto{\pgfqpoint{1.048218in}{1.141534in}}%
\pgfpathlineto{\pgfqpoint{1.086832in}{0.993558in}}%
\pgfpathlineto{\pgfqpoint{1.112575in}{0.903538in}}%
\pgfpathlineto{\pgfqpoint{1.131883in}{0.842963in}}%
\pgfpathlineto{\pgfqpoint{1.151190in}{0.789172in}}%
\pgfpathlineto{\pgfqpoint{1.170497in}{0.742376in}}%
\pgfpathlineto{\pgfqpoint{1.189804in}{0.702391in}}%
\pgfpathlineto{\pgfqpoint{1.209111in}{0.668746in}}%
\pgfpathlineto{\pgfqpoint{1.228419in}{0.640798in}}%
\pgfpathlineto{\pgfqpoint{1.247726in}{0.617832in}}%
\pgfpathlineto{\pgfqpoint{1.267033in}{0.599124in}}%
\pgfpathlineto{\pgfqpoint{1.286340in}{0.583996in}}%
\pgfpathlineto{\pgfqpoint{1.305648in}{0.571834in}}%
\pgfpathlineto{\pgfqpoint{1.324955in}{0.562101in}}%
\pgfpathlineto{\pgfqpoint{1.350698in}{0.552126in}}%
\pgfpathlineto{\pgfqpoint{1.376441in}{0.544789in}}%
\pgfpathlineto{\pgfqpoint{1.408619in}{0.538313in}}%
\pgfpathlineto{\pgfqpoint{1.447234in}{0.533256in}}%
\pgfpathlineto{\pgfqpoint{1.492284in}{0.529718in}}%
\pgfpathlineto{\pgfqpoint{1.556642in}{0.527118in}}%
\pgfpathlineto{\pgfqpoint{1.659613in}{0.525586in}}%
\pgfpathlineto{\pgfqpoint{1.846250in}{0.525057in}}%
\pgfpathlineto{\pgfqpoint{1.846250in}{0.525057in}}%
\pgfusepath{stroke}%
\end{pgfscope}%
\begin{pgfscope}%
\pgfpathrectangle{\pgfqpoint{0.243750in}{0.525000in}}{\pgfqpoint{1.602500in}{1.887500in}}%
\pgfusepath{clip}%
\pgfsetrectcap%
\pgfsetroundjoin%
\pgfsetlinewidth{1.003750pt}%
\definecolor{currentstroke}{rgb}{0.000000,0.000000,0.000000}%
\pgfsetstrokecolor{currentstroke}%
\pgfsetdash{}{0pt}%
\pgfpathmoveto{\pgfqpoint{0.243750in}{1.783333in}}%
\pgfpathlineto{\pgfqpoint{0.983860in}{1.782723in}}%
\pgfpathlineto{\pgfqpoint{0.996732in}{1.780296in}}%
\pgfpathlineto{\pgfqpoint{1.003168in}{1.776572in}}%
\pgfpathlineto{\pgfqpoint{1.009603in}{1.768336in}}%
\pgfpathlineto{\pgfqpoint{1.016039in}{1.750333in}}%
\pgfpathlineto{\pgfqpoint{1.022475in}{1.711961in}}%
\pgfpathlineto{\pgfqpoint{1.028911in}{1.634394in}}%
\pgfpathlineto{\pgfqpoint{1.035346in}{1.493138in}}%
\pgfpathlineto{\pgfqpoint{1.054654in}{0.815196in}}%
\pgfpathlineto{\pgfqpoint{1.061089in}{0.673939in}}%
\pgfpathlineto{\pgfqpoint{1.067525in}{0.596372in}}%
\pgfpathlineto{\pgfqpoint{1.073961in}{0.558000in}}%
\pgfpathlineto{\pgfqpoint{1.080397in}{0.539997in}}%
\pgfpathlineto{\pgfqpoint{1.086832in}{0.531762in}}%
\pgfpathlineto{\pgfqpoint{1.093268in}{0.528037in}}%
\pgfpathlineto{\pgfqpoint{1.106140in}{0.525611in}}%
\pgfpathlineto{\pgfqpoint{1.138318in}{0.525011in}}%
\pgfpathlineto{\pgfqpoint{1.846250in}{0.525000in}}%
\pgfpathlineto{\pgfqpoint{1.846250in}{0.525000in}}%
\pgfusepath{stroke}%
\end{pgfscope}%
\begin{pgfscope}%
\pgfpathrectangle{\pgfqpoint{0.243750in}{0.525000in}}{\pgfqpoint{1.602500in}{1.887500in}}%
\pgfusepath{clip}%
\pgfsetbuttcap%
\pgfsetroundjoin%
\pgfsetlinewidth{1.003750pt}%
\definecolor{currentstroke}{rgb}{0.000000,0.000000,0.000000}%
\pgfsetstrokecolor{currentstroke}%
\pgfsetdash{{3.700000pt}{1.600000pt}}{0.000000pt}%
\pgfpathmoveto{\pgfqpoint{1.045000in}{0.525000in}}%
\pgfpathlineto{\pgfqpoint{1.045000in}{2.412500in}}%
\pgfusepath{stroke}%
\end{pgfscope}%
\begin{pgfscope}%
\pgfpathrectangle{\pgfqpoint{0.243750in}{0.525000in}}{\pgfqpoint{1.602500in}{1.887500in}}%
\pgfusepath{clip}%
\pgfsetbuttcap%
\pgfsetroundjoin%
\pgfsetlinewidth{0.501875pt}%
\definecolor{currentstroke}{rgb}{0.827451,0.827451,0.827451}%
\pgfsetstrokecolor{currentstroke}%
\pgfsetdash{{1.850000pt}{0.800000pt}}{0.000000pt}%
\pgfpathmoveto{\pgfqpoint{0.243750in}{1.783333in}}%
\pgfpathlineto{\pgfqpoint{1.846250in}{1.783333in}}%
\pgfusepath{stroke}%
\end{pgfscope}%
\begin{pgfscope}%
\pgfpathrectangle{\pgfqpoint{0.243750in}{0.525000in}}{\pgfqpoint{1.602500in}{1.887500in}}%
\pgfusepath{clip}%
\pgfsetbuttcap%
\pgfsetroundjoin%
\pgfsetlinewidth{0.501875pt}%
\definecolor{currentstroke}{rgb}{0.827451,0.827451,0.827451}%
\pgfsetstrokecolor{currentstroke}%
\pgfsetdash{{1.850000pt}{0.800000pt}}{0.000000pt}%
\pgfpathmoveto{\pgfqpoint{0.243750in}{1.154167in}}%
\pgfpathlineto{\pgfqpoint{1.846250in}{1.154167in}}%
\pgfusepath{stroke}%
\end{pgfscope}%
\begin{pgfscope}%
\pgfsetrectcap%
\pgfsetmiterjoin%
\pgfsetlinewidth{1.003750pt}%
\definecolor{currentstroke}{rgb}{0.000000,0.000000,0.000000}%
\pgfsetstrokecolor{currentstroke}%
\pgfsetdash{}{0pt}%
\pgfpathmoveto{\pgfqpoint{0.243750in}{0.525000in}}%
\pgfpathlineto{\pgfqpoint{0.243750in}{2.412500in}}%
\pgfusepath{stroke}%
\end{pgfscope}%
\begin{pgfscope}%
\pgfsetrectcap%
\pgfsetmiterjoin%
\pgfsetlinewidth{1.003750pt}%
\definecolor{currentstroke}{rgb}{0.000000,0.000000,0.000000}%
\pgfsetstrokecolor{currentstroke}%
\pgfsetdash{}{0pt}%
\pgfpathmoveto{\pgfqpoint{1.846250in}{0.525000in}}%
\pgfpathlineto{\pgfqpoint{1.846250in}{2.412500in}}%
\pgfusepath{stroke}%
\end{pgfscope}%
\begin{pgfscope}%
\pgfsetrectcap%
\pgfsetmiterjoin%
\pgfsetlinewidth{1.003750pt}%
\definecolor{currentstroke}{rgb}{0.000000,0.000000,0.000000}%
\pgfsetstrokecolor{currentstroke}%
\pgfsetdash{}{0pt}%
\pgfpathmoveto{\pgfqpoint{0.243750in}{0.525000in}}%
\pgfpathlineto{\pgfqpoint{1.846250in}{0.525000in}}%
\pgfusepath{stroke}%
\end{pgfscope}%
\begin{pgfscope}%
\pgfsetrectcap%
\pgfsetmiterjoin%
\pgfsetlinewidth{1.003750pt}%
\definecolor{currentstroke}{rgb}{0.000000,0.000000,0.000000}%
\pgfsetstrokecolor{currentstroke}%
\pgfsetdash{}{0pt}%
\pgfpathmoveto{\pgfqpoint{0.243750in}{2.412500in}}%
\pgfpathlineto{\pgfqpoint{1.846250in}{2.412500in}}%
\pgfusepath{stroke}%
\end{pgfscope}%
\end{pgfpicture}%
\makeatother%
\endgroup%

  \caption{Fermi-Dirac distribution for $\kT = \nicefrac{\mu}{10}$ and
    $\kT = \nicefrac{\mu}{100}$. As the temperature approaches absolute zero,
    the distribution approaches a step function.}
  \label{fig:fermi_dirac_vs_temp}
\end{marginfigure}

In our language, as the temperature goes to zero, the exponential at the denominator
of the Fermi-Dirac distribution tends to become very large when the argument is
positive ($E > \mu$) and very small when the argument is negative ($E < \mu$).
Conversely, the number density in the phase space tends to a step function, as
illustrated in figure~\ref{fig:fermi_dirac_vs_temp}, i.e.,
\begin{align*}
  \lim_{T \rightarrow 0} \dv{n}{p}{(p; T)} =
  \begin{cases}
    \frac{4\pi g}{h^3} p^2 \quad & (E \leq \mu)\\
    0 \quad & (E > \mu)
  \end{cases}
\end{align*}
From a physical standpoint, what happens is that the electrons start filling all
the available energy levels starting from the lowest one (with two electrons per
level, one spin-up and one spin-down) until all of them have a seat. The maximum
electron momentum, which is the equivalent in momentum space of the chemical
potential, is called \emph{Fermi momentum} and can be readily calculated by simply
integrating~\eqref{eq:qstats} and imposing that the result amounts to the total
number density (which is known once $N$ and $V$ are set)
\begin{align}\label{eq:fermi_momentum}
  n = \int_0^{\pF} \dv{n}{p}{(p)}dp = \frac{4\pi g \pF^3}{3h^3}
  \quad \text{or} \quad
  \pF = h \left( \frac{3n}{4\pi g} \right)^\frac{1}{3}.
\end{align}

All we have said before still holds and our degenerate Fermi gas posses a non-zero
energy density and, as a consequence, a non-zero pressure\sidenote{This is at odds to what
happens in a classical setting (or, for what matters, with a system of non-interacting
bosons), where at absolute zero all the particles sit happily in the lowest energy
level and the pressure vanishes.}.
The energy density reads
\begin{align*}
  u = \int_0^{p_F} E_k \dv{n}{p} dp = \frac{4\pi g}{h^3} \times
  \begin{cases}
    \displaystyle\int_0^{p_F} \frac{p^4}{2m} dp = \frac{p_F^5}{10 m}
    & \quad\text{(non-relativistic)}\\[8pt]
    \displaystyle\int_0^{p_F} p^3 c dp = \frac{P_F^4 c}{4}
    & \quad\text{(ultra-relativistic)}.
  \end{cases}
\end{align*}
It is interesting to close the loop and express things back in terms of the number
density, which is the physical parameter we have direct access to
\begin{align}
  u =
  \begin{cases}
    \displaystyle %\frac{2\pi}{5} \frac{g p_F^5}{h^3 m} =
    \frac{3}{10} \left(\frac{3}{4\pi g}\right)^\frac{2}{3} \frac{h^2}{m} n^\frac{5}{3}
    = \frac{3}{2} P
    & \quad\text{(non-relativistic)}\\[8pt]
    \displaystyle %\pi \frac{g p_F^4 c}{h^3} =
    \frac{3}{4} \left(\frac{3}{4\pi g}\right)^\frac{1}{3} hc\; n^\frac{4}{3}
    = 3 P
    & \quad\text{(ultra-relativistic)}.
  \end{cases}
\end{align}
Note the different exponent of the dependence on the number density $n$ in the
two cases ($\nicefrac{5}{3}$ vs. $\nicefrac{4}{3}$), because it will turn out to
be very important for the equilibrium of degenerate stars. Also note that, in the
non-relativistic case, the degeneracy pressure is inversely proportional to the
mass of the fermion, i.e., it will be roughly 2,000 times smaller for protons or
neutrons with respect to electrons.



\section{The Maxwell-Boltzmann distribution}
\label{sec:maxwel_boltzmann_dist}

In the non-relativistic limit, if we assume that $E - \mu \ll \kT$\sidenote{Note this
is a subtle assumption, as in this context $E$ is a random variable, whose average
value is of the order of $\kT$. What we are asking is that the fluctuations of
$E$ around the chemical potential are much smaller than the temperature.}, both the
Bose-Einstein and the Fermi-Dirac distributions reduce to
\begin{align*}
  \dv{n}{p}{(p; T, \mu)} = \frac{4\pi g}{h^3} p^2 e^{-\frac{E - \mu}{\kT}} =
  \frac{4\pi g}{h^3} e^{\frac{\mu}{\kT}} p^2 e^{-\frac{p^2}{2m \kT}}.
\end{align*}
The normalization of this probability density function is readily calculated
\begin{align*}
  \int_0^\infty \dv{n}{p}{(p)} dp = &
  \frac{4\pi g}{h^3} e^{\frac{\mu}{\kT}} \int_0^\infty p^2 e^{-\frac{p^2}{2m \kT}} dp =\\
   = & \frac{4\pi g}{h^3} e^{\frac{\mu}{\kT}} (2m \kT)^\frac{3}{2}
   \underbrace{\int_0^\infty z^2 e^{-z^2} dz}_{\nicefrac{\sqrt{\pi}}{4}} =
   \frac{g}{h^3} e^{\frac{\mu}{\kT}} (2 \pi m \kT)^\frac{3}{2},
\end{align*}
and we can rewrite the previous expression as a proper probability density function
normalized to unity, in momentum space, as
\begin{align*}
  p_p(p; T) = \frac{4\pi}{(2 \pi m \kT)^\frac{3}{2}} p^2 e^{-\frac{p^2}{2m \kT}}.
\end{align*}
A simple change of variables\sidenote{This is the first of many change of variables
we shall do in this section, and you should always pay attention not to forget the
Jacobian of the transformation, or you will get the wrong numerical value. In this
case you need to multiply by $m^3$ in order to get the correct mass dependence, and
a factor $m^2$ comes from the $p^2$, while the extra $m$ comes from the Jacobian
$$
  \abs{\dv{p}{v}} = m.
$$} allows to recast this in velocity space
\begin{align}
  p_v(v; T) = \qty(\frac{m}{2\pi \kT})^\frac{3}{2} 4\pi v^2 e^{-\frac{mv^2}{2\kT}}
\end{align}
which is the usual form of the Maxwell-Boltzmann distribution you find in every
textbook.

The very same distribution can be expressed in (kinetic) energy space
\begin{align}
  p_E(E_k; T) =
  \frac{2}{\sqrt{\pi}} \qty(\frac{1}{\kT})^\frac{3}{2} \sqrt{E_k} e^{-\frac{E_k}{\kT}},
\end{align}
and we shall occasionally use it in this last form. We note that the average kinetic
energy reads
\begin{align}
  \ave{E_k} = \int_0^\infty E_k p_E(E_k) dE =
  \frac{2}{\sqrt{\pi}} \kT \underbrace{\int_0^\infty z^\frac{3}{2} e^{-z} dz}_{\frac{3\sqrt{\pi}}{4}} =
  \frac{3}{2} \kT.
\end{align}



\section{The Planck distribution}
\label{sec:planck_distribution}

A photon gas is described by a Bose-Einstein distribution~\eqref{eq:qstats} with the
understanding that \emph{photons have 2 degrees of freedom\sidenote{The photon has
spin~$s = 1$, so formally comes with $2s + 1 = 3$ internal degrees of freedom, but
the fact that it is massless reduces the number by one unity. Classically, this
is related to the fact that electromagnetic radiation only comes with two independent
states of polarization, since the electric and magnetic fields are bound to oscillate
in the plane transverse to the direction of propagation.} and zero chemical potential}.

Let us spend a second clearing this last bit of information out, for the sake of
completeness. Formally, the chemical potential is the partial derivative of the
free energy of the system with respect to the number of elements of a particular
species, and from a physical standpoint it represents the energy that is absorbed
or released by a system when the number of particles of that particular species
is changed. In an ordinary gas the total number $N$ of particles is fixed, and that,
as we have seen, provides the condition determining the chemical potential $\mu$.
Photons are different in that they are extremely cheap to create and
destroy\sidenote{If you think about for a second, photons do not really interact
a lot with each other, and the typical mechanism by which they reach equilibrium
is absorption and emission by matter, which is essentially free.}, and so in typical
conditions they number is not fixed, but is free to vary, with no real cost in terms
of energy. How does it vary? Well, in such a way that the free energy $F$ of the
system is minimum, or
\begin{align*}
  \left(\pdv{F}{N}\right)_{T, V} \!\!\!\!\!\! = 0 = \mu.
\end{align*}
(Ok, we agree this is hand-waving. If you are not satisfied, you are welcome to
dig into the issue more, or accept the fact that Plank derived the
distribution~\eqref{eq:planck_spec_density_nu} bearing his name from first principles.)

With that, since the dispersion relation for photons is $E = h\nu = pc$
we can rewrite the Bose-Einstein distribution, with $g = 2$ and $\mu = 0$, as
\begin{align*}
  \frac{dn}{dp}(p; T) =
  \frac{8\pi}{h^3} \frac{p^2}{e^\frac{pc}{\kT} - 1} \quad\text{or}\quad
  \frac{dn}{d\nu}(\nu; T) =
  \frac{8\pi}{c^3} \frac{\nu^2}{e^\frac{h\nu}{\kT} - 1},
\end{align*}
and if we multiply by $E = h\nu$ we get the spectral energy density\sidenote{To be
precise this is the energy density per unit frequency interval and unit volume,
and is measured in erg~Hz$^{-1}$~cm$^{-3}$ (or erg~s~cm$^{-3}$) in the cgs system.},
which is the standard form in which the Plank distribution is found on textbooks
(and illustrated in figure~\ref{fig:planck_distribution})
\begin{align}\label{eq:planck_spec_density_nu}
  \dv{u}{\nu}{(\nu; T)} = u_\nu(\nu; T) =
  \frac{8\pi h}{c^3} \frac{\nu^3}{e^\frac{h\nu}{\kT} - 1}.
\end{align}
(And of course Plank derived this expression in 1900, while it was not until
25~years later that Bose begin developing his quantum statistics, but in the grand
scheme of things it is more conceptually inexpensive to let things flow this way.)

By simple changes of variables, the distribution can be rewritten in a number of
ways, that are all useful in different contexts. The Planck distribution in wavelength
space, e.g., reads
\begin{align}\label{eq:planck_spec_density_lambda}
  \dv{u}{\lambda}{(\lambda; T)} = u_\lambda(\lambda; T) =
  8\pi h c \frac{\lambda^{-5}}{e^\frac{hc}{\lambda\kT} - 1}
\end{align}
which, due to the Jacobian of the transformation $\nu = \nicefrac{c}{\lambda}$
has a different functional form than its counterpart in frequency space.
Similarly, if we take $E = h\nu$ as our dynamical variable, the spectral
energy density can be rewritten as
\begin{align}\label{eq:planck_spec_density_E}
  \dv{u}{E}{(E; T)} = u_E(E; T) =
  \frac{8\pi}{(hc)^3} \frac{E^3}{e^\frac{E}{\kT} - 1},
\end{align}
and is we now we take a step back toward where we started from, we can write the
number density in energy space just dividing by $E$
\begin{align}
  \dv{n}{E}{(E; T)} = n_E(E; T) =
  \frac{8\pi}{(hc)^3} \frac{E^2}{e^\frac{E}{\kT} - 1}.
\end{align}

\begin{marginfigure}
  %% Creator: Matplotlib, PGF backend
%%
%% To include the figure in your LaTeX document, write
%%   \input{<filename>.pgf}
%%
%% Make sure the required packages are loaded in your preamble
%%   \usepackage{pgf}
%%
%% Also ensure that all the required font packages are loaded; for instance,
%% the lmodern package is sometimes necessary when using math font.
%%   \usepackage{lmodern}
%%
%% Figures using additional raster images can only be included by \input if
%% they are in the same directory as the main LaTeX file. For loading figures
%% from other directories you can use the `import` package
%%   \usepackage{import}
%%
%% and then include the figures with
%%   \import{<path to file>}{<filename>.pgf}
%%
%% Matplotlib used the following preamble
%%   \def\mathdefault#1{#1}
%%   \everymath=\expandafter{\the\everymath\displaystyle}
%%   \IfFileExists{scrextend.sty}{
%%     \usepackage[fontsize=9.000000pt]{scrextend}
%%   }{
%%     \renewcommand{\normalsize}{\fontsize{9.000000}{10.800000}\selectfont}
%%     \normalsize
%%   }
%%   
%%   \ifdefined\pdftexversion\else  % non-pdftex case.
%%     \usepackage{fontspec}
%%     \setmainfont{DejaVuSerif.ttf}[Path=\detokenize{/home/users/lbaldini/.pyenv/versions/3.13.1/lib/python3.13/site-packages/matplotlib/mpl-data/fonts/ttf/}]
%%     \setsansfont{DejaVuSans.ttf}[Path=\detokenize{/home/users/lbaldini/.pyenv/versions/3.13.1/lib/python3.13/site-packages/matplotlib/mpl-data/fonts/ttf/}]
%%     \setmonofont{DejaVuSansMono.ttf}[Path=\detokenize{/home/users/lbaldini/.pyenv/versions/3.13.1/lib/python3.13/site-packages/matplotlib/mpl-data/fonts/ttf/}]
%%   \fi
%%   \makeatletter\@ifpackageloaded{underscore}{}{\usepackage[strings]{underscore}}\makeatother
%%
\begingroup%
\makeatletter%
\begin{pgfpicture}%
\pgfpathrectangle{\pgfpointorigin}{\pgfqpoint{1.950000in}{2.000000in}}%
\pgfusepath{use as bounding box, clip}%
\begin{pgfscope}%
\pgfsetbuttcap%
\pgfsetmiterjoin%
\definecolor{currentfill}{rgb}{1.000000,1.000000,1.000000}%
\pgfsetfillcolor{currentfill}%
\pgfsetlinewidth{0.000000pt}%
\definecolor{currentstroke}{rgb}{1.000000,1.000000,1.000000}%
\pgfsetstrokecolor{currentstroke}%
\pgfsetdash{}{0pt}%
\pgfpathmoveto{\pgfqpoint{0.000000in}{0.000000in}}%
\pgfpathlineto{\pgfqpoint{1.950000in}{0.000000in}}%
\pgfpathlineto{\pgfqpoint{1.950000in}{2.000000in}}%
\pgfpathlineto{\pgfqpoint{0.000000in}{2.000000in}}%
\pgfpathlineto{\pgfqpoint{0.000000in}{0.000000in}}%
\pgfpathclose%
\pgfusepath{fill}%
\end{pgfscope}%
\begin{pgfscope}%
\pgfsetbuttcap%
\pgfsetmiterjoin%
\definecolor{currentfill}{rgb}{1.000000,1.000000,1.000000}%
\pgfsetfillcolor{currentfill}%
\pgfsetlinewidth{0.000000pt}%
\definecolor{currentstroke}{rgb}{0.000000,0.000000,0.000000}%
\pgfsetstrokecolor{currentstroke}%
\pgfsetstrokeopacity{0.000000}%
\pgfsetdash{}{0pt}%
\pgfpathmoveto{\pgfqpoint{0.243750in}{0.525000in}}%
\pgfpathlineto{\pgfqpoint{1.846250in}{0.525000in}}%
\pgfpathlineto{\pgfqpoint{1.846250in}{1.912500in}}%
\pgfpathlineto{\pgfqpoint{0.243750in}{1.912500in}}%
\pgfpathlineto{\pgfqpoint{0.243750in}{0.525000in}}%
\pgfpathclose%
\pgfusepath{fill}%
\end{pgfscope}%
\begin{pgfscope}%
\pgfpathrectangle{\pgfqpoint{0.243750in}{0.525000in}}{\pgfqpoint{1.602500in}{1.387500in}}%
\pgfusepath{clip}%
\pgfsetbuttcap%
\pgfsetroundjoin%
\pgfsetlinewidth{0.803000pt}%
\definecolor{currentstroke}{rgb}{0.752941,0.752941,0.752941}%
\pgfsetstrokecolor{currentstroke}%
\pgfsetdash{{2.960000pt}{1.280000pt}}{0.000000pt}%
\pgfpathmoveto{\pgfqpoint{0.243750in}{0.525000in}}%
\pgfpathlineto{\pgfqpoint{0.243750in}{1.912500in}}%
\pgfusepath{stroke}%
\end{pgfscope}%
\begin{pgfscope}%
\pgfsetbuttcap%
\pgfsetroundjoin%
\definecolor{currentfill}{rgb}{0.000000,0.000000,0.000000}%
\pgfsetfillcolor{currentfill}%
\pgfsetlinewidth{0.803000pt}%
\definecolor{currentstroke}{rgb}{0.000000,0.000000,0.000000}%
\pgfsetstrokecolor{currentstroke}%
\pgfsetdash{}{0pt}%
\pgfsys@defobject{currentmarker}{\pgfqpoint{0.000000in}{-0.048611in}}{\pgfqpoint{0.000000in}{0.000000in}}{%
\pgfpathmoveto{\pgfqpoint{0.000000in}{0.000000in}}%
\pgfpathlineto{\pgfqpoint{0.000000in}{-0.048611in}}%
\pgfusepath{stroke,fill}%
}%
\begin{pgfscope}%
\pgfsys@transformshift{0.243750in}{0.525000in}%
\pgfsys@useobject{currentmarker}{}%
\end{pgfscope}%
\end{pgfscope}%
\begin{pgfscope}%
\definecolor{textcolor}{rgb}{0.000000,0.000000,0.000000}%
\pgfsetstrokecolor{textcolor}%
\pgfsetfillcolor{textcolor}%
\pgftext[x=0.243750in,y=0.427778in,,top]{\color{textcolor}{\rmfamily\fontsize{9.000000}{10.800000}\selectfont\catcode`\^=\active\def^{\ifmmode\sp\else\^{}\fi}\catcode`\%=\active\def%{\%}0}}%
\end{pgfscope}%
\begin{pgfscope}%
\pgfpathrectangle{\pgfqpoint{0.243750in}{0.525000in}}{\pgfqpoint{1.602500in}{1.387500in}}%
\pgfusepath{clip}%
\pgfsetbuttcap%
\pgfsetroundjoin%
\pgfsetlinewidth{0.803000pt}%
\definecolor{currentstroke}{rgb}{0.752941,0.752941,0.752941}%
\pgfsetstrokecolor{currentstroke}%
\pgfsetdash{{2.960000pt}{1.280000pt}}{0.000000pt}%
\pgfpathmoveto{\pgfqpoint{0.457417in}{0.525000in}}%
\pgfpathlineto{\pgfqpoint{0.457417in}{1.912500in}}%
\pgfusepath{stroke}%
\end{pgfscope}%
\begin{pgfscope}%
\pgfsetbuttcap%
\pgfsetroundjoin%
\definecolor{currentfill}{rgb}{0.000000,0.000000,0.000000}%
\pgfsetfillcolor{currentfill}%
\pgfsetlinewidth{0.803000pt}%
\definecolor{currentstroke}{rgb}{0.000000,0.000000,0.000000}%
\pgfsetstrokecolor{currentstroke}%
\pgfsetdash{}{0pt}%
\pgfsys@defobject{currentmarker}{\pgfqpoint{0.000000in}{-0.048611in}}{\pgfqpoint{0.000000in}{0.000000in}}{%
\pgfpathmoveto{\pgfqpoint{0.000000in}{0.000000in}}%
\pgfpathlineto{\pgfqpoint{0.000000in}{-0.048611in}}%
\pgfusepath{stroke,fill}%
}%
\begin{pgfscope}%
\pgfsys@transformshift{0.457417in}{0.525000in}%
\pgfsys@useobject{currentmarker}{}%
\end{pgfscope}%
\end{pgfscope}%
\begin{pgfscope}%
\definecolor{textcolor}{rgb}{0.000000,0.000000,0.000000}%
\pgfsetstrokecolor{textcolor}%
\pgfsetfillcolor{textcolor}%
\pgftext[x=0.457417in,y=0.427778in,,top]{\color{textcolor}{\rmfamily\fontsize{9.000000}{10.800000}\selectfont\catcode`\^=\active\def^{\ifmmode\sp\else\^{}\fi}\catcode`\%=\active\def%{\%}2}}%
\end{pgfscope}%
\begin{pgfscope}%
\pgfpathrectangle{\pgfqpoint{0.243750in}{0.525000in}}{\pgfqpoint{1.602500in}{1.387500in}}%
\pgfusepath{clip}%
\pgfsetbuttcap%
\pgfsetroundjoin%
\pgfsetlinewidth{0.803000pt}%
\definecolor{currentstroke}{rgb}{0.752941,0.752941,0.752941}%
\pgfsetstrokecolor{currentstroke}%
\pgfsetdash{{2.960000pt}{1.280000pt}}{0.000000pt}%
\pgfpathmoveto{\pgfqpoint{0.671083in}{0.525000in}}%
\pgfpathlineto{\pgfqpoint{0.671083in}{1.912500in}}%
\pgfusepath{stroke}%
\end{pgfscope}%
\begin{pgfscope}%
\pgfsetbuttcap%
\pgfsetroundjoin%
\definecolor{currentfill}{rgb}{0.000000,0.000000,0.000000}%
\pgfsetfillcolor{currentfill}%
\pgfsetlinewidth{0.803000pt}%
\definecolor{currentstroke}{rgb}{0.000000,0.000000,0.000000}%
\pgfsetstrokecolor{currentstroke}%
\pgfsetdash{}{0pt}%
\pgfsys@defobject{currentmarker}{\pgfqpoint{0.000000in}{-0.048611in}}{\pgfqpoint{0.000000in}{0.000000in}}{%
\pgfpathmoveto{\pgfqpoint{0.000000in}{0.000000in}}%
\pgfpathlineto{\pgfqpoint{0.000000in}{-0.048611in}}%
\pgfusepath{stroke,fill}%
}%
\begin{pgfscope}%
\pgfsys@transformshift{0.671083in}{0.525000in}%
\pgfsys@useobject{currentmarker}{}%
\end{pgfscope}%
\end{pgfscope}%
\begin{pgfscope}%
\definecolor{textcolor}{rgb}{0.000000,0.000000,0.000000}%
\pgfsetstrokecolor{textcolor}%
\pgfsetfillcolor{textcolor}%
\pgftext[x=0.671083in,y=0.427778in,,top]{\color{textcolor}{\rmfamily\fontsize{9.000000}{10.800000}\selectfont\catcode`\^=\active\def^{\ifmmode\sp\else\^{}\fi}\catcode`\%=\active\def%{\%}4}}%
\end{pgfscope}%
\begin{pgfscope}%
\pgfpathrectangle{\pgfqpoint{0.243750in}{0.525000in}}{\pgfqpoint{1.602500in}{1.387500in}}%
\pgfusepath{clip}%
\pgfsetbuttcap%
\pgfsetroundjoin%
\pgfsetlinewidth{0.803000pt}%
\definecolor{currentstroke}{rgb}{0.752941,0.752941,0.752941}%
\pgfsetstrokecolor{currentstroke}%
\pgfsetdash{{2.960000pt}{1.280000pt}}{0.000000pt}%
\pgfpathmoveto{\pgfqpoint{0.884750in}{0.525000in}}%
\pgfpathlineto{\pgfqpoint{0.884750in}{1.912500in}}%
\pgfusepath{stroke}%
\end{pgfscope}%
\begin{pgfscope}%
\pgfsetbuttcap%
\pgfsetroundjoin%
\definecolor{currentfill}{rgb}{0.000000,0.000000,0.000000}%
\pgfsetfillcolor{currentfill}%
\pgfsetlinewidth{0.803000pt}%
\definecolor{currentstroke}{rgb}{0.000000,0.000000,0.000000}%
\pgfsetstrokecolor{currentstroke}%
\pgfsetdash{}{0pt}%
\pgfsys@defobject{currentmarker}{\pgfqpoint{0.000000in}{-0.048611in}}{\pgfqpoint{0.000000in}{0.000000in}}{%
\pgfpathmoveto{\pgfqpoint{0.000000in}{0.000000in}}%
\pgfpathlineto{\pgfqpoint{0.000000in}{-0.048611in}}%
\pgfusepath{stroke,fill}%
}%
\begin{pgfscope}%
\pgfsys@transformshift{0.884750in}{0.525000in}%
\pgfsys@useobject{currentmarker}{}%
\end{pgfscope}%
\end{pgfscope}%
\begin{pgfscope}%
\definecolor{textcolor}{rgb}{0.000000,0.000000,0.000000}%
\pgfsetstrokecolor{textcolor}%
\pgfsetfillcolor{textcolor}%
\pgftext[x=0.884750in,y=0.427778in,,top]{\color{textcolor}{\rmfamily\fontsize{9.000000}{10.800000}\selectfont\catcode`\^=\active\def^{\ifmmode\sp\else\^{}\fi}\catcode`\%=\active\def%{\%}6}}%
\end{pgfscope}%
\begin{pgfscope}%
\pgfpathrectangle{\pgfqpoint{0.243750in}{0.525000in}}{\pgfqpoint{1.602500in}{1.387500in}}%
\pgfusepath{clip}%
\pgfsetbuttcap%
\pgfsetroundjoin%
\pgfsetlinewidth{0.803000pt}%
\definecolor{currentstroke}{rgb}{0.752941,0.752941,0.752941}%
\pgfsetstrokecolor{currentstroke}%
\pgfsetdash{{2.960000pt}{1.280000pt}}{0.000000pt}%
\pgfpathmoveto{\pgfqpoint{1.098417in}{0.525000in}}%
\pgfpathlineto{\pgfqpoint{1.098417in}{1.912500in}}%
\pgfusepath{stroke}%
\end{pgfscope}%
\begin{pgfscope}%
\pgfsetbuttcap%
\pgfsetroundjoin%
\definecolor{currentfill}{rgb}{0.000000,0.000000,0.000000}%
\pgfsetfillcolor{currentfill}%
\pgfsetlinewidth{0.803000pt}%
\definecolor{currentstroke}{rgb}{0.000000,0.000000,0.000000}%
\pgfsetstrokecolor{currentstroke}%
\pgfsetdash{}{0pt}%
\pgfsys@defobject{currentmarker}{\pgfqpoint{0.000000in}{-0.048611in}}{\pgfqpoint{0.000000in}{0.000000in}}{%
\pgfpathmoveto{\pgfqpoint{0.000000in}{0.000000in}}%
\pgfpathlineto{\pgfqpoint{0.000000in}{-0.048611in}}%
\pgfusepath{stroke,fill}%
}%
\begin{pgfscope}%
\pgfsys@transformshift{1.098417in}{0.525000in}%
\pgfsys@useobject{currentmarker}{}%
\end{pgfscope}%
\end{pgfscope}%
\begin{pgfscope}%
\definecolor{textcolor}{rgb}{0.000000,0.000000,0.000000}%
\pgfsetstrokecolor{textcolor}%
\pgfsetfillcolor{textcolor}%
\pgftext[x=1.098417in,y=0.427778in,,top]{\color{textcolor}{\rmfamily\fontsize{9.000000}{10.800000}\selectfont\catcode`\^=\active\def^{\ifmmode\sp\else\^{}\fi}\catcode`\%=\active\def%{\%}8}}%
\end{pgfscope}%
\begin{pgfscope}%
\pgfpathrectangle{\pgfqpoint{0.243750in}{0.525000in}}{\pgfqpoint{1.602500in}{1.387500in}}%
\pgfusepath{clip}%
\pgfsetbuttcap%
\pgfsetroundjoin%
\pgfsetlinewidth{0.803000pt}%
\definecolor{currentstroke}{rgb}{0.752941,0.752941,0.752941}%
\pgfsetstrokecolor{currentstroke}%
\pgfsetdash{{2.960000pt}{1.280000pt}}{0.000000pt}%
\pgfpathmoveto{\pgfqpoint{1.312083in}{0.525000in}}%
\pgfpathlineto{\pgfqpoint{1.312083in}{1.912500in}}%
\pgfusepath{stroke}%
\end{pgfscope}%
\begin{pgfscope}%
\pgfsetbuttcap%
\pgfsetroundjoin%
\definecolor{currentfill}{rgb}{0.000000,0.000000,0.000000}%
\pgfsetfillcolor{currentfill}%
\pgfsetlinewidth{0.803000pt}%
\definecolor{currentstroke}{rgb}{0.000000,0.000000,0.000000}%
\pgfsetstrokecolor{currentstroke}%
\pgfsetdash{}{0pt}%
\pgfsys@defobject{currentmarker}{\pgfqpoint{0.000000in}{-0.048611in}}{\pgfqpoint{0.000000in}{0.000000in}}{%
\pgfpathmoveto{\pgfqpoint{0.000000in}{0.000000in}}%
\pgfpathlineto{\pgfqpoint{0.000000in}{-0.048611in}}%
\pgfusepath{stroke,fill}%
}%
\begin{pgfscope}%
\pgfsys@transformshift{1.312083in}{0.525000in}%
\pgfsys@useobject{currentmarker}{}%
\end{pgfscope}%
\end{pgfscope}%
\begin{pgfscope}%
\definecolor{textcolor}{rgb}{0.000000,0.000000,0.000000}%
\pgfsetstrokecolor{textcolor}%
\pgfsetfillcolor{textcolor}%
\pgftext[x=1.312083in,y=0.427778in,,top]{\color{textcolor}{\rmfamily\fontsize{9.000000}{10.800000}\selectfont\catcode`\^=\active\def^{\ifmmode\sp\else\^{}\fi}\catcode`\%=\active\def%{\%}10}}%
\end{pgfscope}%
\begin{pgfscope}%
\pgfpathrectangle{\pgfqpoint{0.243750in}{0.525000in}}{\pgfqpoint{1.602500in}{1.387500in}}%
\pgfusepath{clip}%
\pgfsetbuttcap%
\pgfsetroundjoin%
\pgfsetlinewidth{0.803000pt}%
\definecolor{currentstroke}{rgb}{0.752941,0.752941,0.752941}%
\pgfsetstrokecolor{currentstroke}%
\pgfsetdash{{2.960000pt}{1.280000pt}}{0.000000pt}%
\pgfpathmoveto{\pgfqpoint{1.525750in}{0.525000in}}%
\pgfpathlineto{\pgfqpoint{1.525750in}{1.912500in}}%
\pgfusepath{stroke}%
\end{pgfscope}%
\begin{pgfscope}%
\pgfsetbuttcap%
\pgfsetroundjoin%
\definecolor{currentfill}{rgb}{0.000000,0.000000,0.000000}%
\pgfsetfillcolor{currentfill}%
\pgfsetlinewidth{0.803000pt}%
\definecolor{currentstroke}{rgb}{0.000000,0.000000,0.000000}%
\pgfsetstrokecolor{currentstroke}%
\pgfsetdash{}{0pt}%
\pgfsys@defobject{currentmarker}{\pgfqpoint{0.000000in}{-0.048611in}}{\pgfqpoint{0.000000in}{0.000000in}}{%
\pgfpathmoveto{\pgfqpoint{0.000000in}{0.000000in}}%
\pgfpathlineto{\pgfqpoint{0.000000in}{-0.048611in}}%
\pgfusepath{stroke,fill}%
}%
\begin{pgfscope}%
\pgfsys@transformshift{1.525750in}{0.525000in}%
\pgfsys@useobject{currentmarker}{}%
\end{pgfscope}%
\end{pgfscope}%
\begin{pgfscope}%
\definecolor{textcolor}{rgb}{0.000000,0.000000,0.000000}%
\pgfsetstrokecolor{textcolor}%
\pgfsetfillcolor{textcolor}%
\pgftext[x=1.525750in,y=0.427778in,,top]{\color{textcolor}{\rmfamily\fontsize{9.000000}{10.800000}\selectfont\catcode`\^=\active\def^{\ifmmode\sp\else\^{}\fi}\catcode`\%=\active\def%{\%}12}}%
\end{pgfscope}%
\begin{pgfscope}%
\pgfpathrectangle{\pgfqpoint{0.243750in}{0.525000in}}{\pgfqpoint{1.602500in}{1.387500in}}%
\pgfusepath{clip}%
\pgfsetbuttcap%
\pgfsetroundjoin%
\pgfsetlinewidth{0.803000pt}%
\definecolor{currentstroke}{rgb}{0.752941,0.752941,0.752941}%
\pgfsetstrokecolor{currentstroke}%
\pgfsetdash{{2.960000pt}{1.280000pt}}{0.000000pt}%
\pgfpathmoveto{\pgfqpoint{1.739417in}{0.525000in}}%
\pgfpathlineto{\pgfqpoint{1.739417in}{1.912500in}}%
\pgfusepath{stroke}%
\end{pgfscope}%
\begin{pgfscope}%
\pgfsetbuttcap%
\pgfsetroundjoin%
\definecolor{currentfill}{rgb}{0.000000,0.000000,0.000000}%
\pgfsetfillcolor{currentfill}%
\pgfsetlinewidth{0.803000pt}%
\definecolor{currentstroke}{rgb}{0.000000,0.000000,0.000000}%
\pgfsetstrokecolor{currentstroke}%
\pgfsetdash{}{0pt}%
\pgfsys@defobject{currentmarker}{\pgfqpoint{0.000000in}{-0.048611in}}{\pgfqpoint{0.000000in}{0.000000in}}{%
\pgfpathmoveto{\pgfqpoint{0.000000in}{0.000000in}}%
\pgfpathlineto{\pgfqpoint{0.000000in}{-0.048611in}}%
\pgfusepath{stroke,fill}%
}%
\begin{pgfscope}%
\pgfsys@transformshift{1.739417in}{0.525000in}%
\pgfsys@useobject{currentmarker}{}%
\end{pgfscope}%
\end{pgfscope}%
\begin{pgfscope}%
\definecolor{textcolor}{rgb}{0.000000,0.000000,0.000000}%
\pgfsetstrokecolor{textcolor}%
\pgfsetfillcolor{textcolor}%
\pgftext[x=1.739417in,y=0.427778in,,top]{\color{textcolor}{\rmfamily\fontsize{9.000000}{10.800000}\selectfont\catcode`\^=\active\def^{\ifmmode\sp\else\^{}\fi}\catcode`\%=\active\def%{\%}14}}%
\end{pgfscope}%
\begin{pgfscope}%
\definecolor{textcolor}{rgb}{0.000000,0.000000,0.000000}%
\pgfsetstrokecolor{textcolor}%
\pgfsetfillcolor{textcolor}%
\pgftext[x=1.045000in,y=0.251251in,,top]{\color{textcolor}{\rmfamily\fontsize{9.000000}{10.800000}\selectfont\catcode`\^=\active\def^{\ifmmode\sp\else\^{}\fi}\catcode`\%=\active\def%{\%}$h\nu / kT$}}%
\end{pgfscope}%
\begin{pgfscope}%
\definecolor{textcolor}{rgb}{0.000000,0.000000,0.000000}%
\pgfsetstrokecolor{textcolor}%
\pgfsetfillcolor{textcolor}%
\pgftext[x=0.188194in,y=1.218750in,,bottom,rotate=90.000000]{\color{textcolor}{\rmfamily\fontsize{9.000000}{10.800000}\selectfont\catcode`\^=\active\def^{\ifmmode\sp\else\^{}\fi}\catcode`\%=\active\def%{\%}$u_\nu(\nu)$}}%
\end{pgfscope}%
\begin{pgfscope}%
\pgfpathrectangle{\pgfqpoint{0.243750in}{0.525000in}}{\pgfqpoint{1.602500in}{1.387500in}}%
\pgfusepath{clip}%
\pgfsetrectcap%
\pgfsetroundjoin%
\pgfsetlinewidth{1.003750pt}%
\definecolor{currentstroke}{rgb}{0.000000,0.000000,0.000000}%
\pgfsetstrokecolor{currentstroke}%
\pgfsetdash{}{0pt}%
\pgfpathmoveto{\pgfqpoint{0.259937in}{0.544776in}}%
\pgfpathlineto{\pgfqpoint{0.276124in}{0.598124in}}%
\pgfpathlineto{\pgfqpoint{0.292311in}{0.676799in}}%
\pgfpathlineto{\pgfqpoint{0.308497in}{0.773514in}}%
\pgfpathlineto{\pgfqpoint{0.324684in}{0.881911in}}%
\pgfpathlineto{\pgfqpoint{0.340871in}{0.996525in}}%
\pgfpathlineto{\pgfqpoint{0.357058in}{1.112736in}}%
\pgfpathlineto{\pgfqpoint{0.373245in}{1.226722in}}%
\pgfpathlineto{\pgfqpoint{0.389432in}{1.335392in}}%
\pgfpathlineto{\pgfqpoint{0.405619in}{1.436332in}}%
\pgfpathlineto{\pgfqpoint{0.421806in}{1.527732in}}%
\pgfpathlineto{\pgfqpoint{0.437992in}{1.608323in}}%
\pgfpathlineto{\pgfqpoint{0.454179in}{1.677308in}}%
\pgfpathlineto{\pgfqpoint{0.470366in}{1.734297in}}%
\pgfpathlineto{\pgfqpoint{0.486553in}{1.779243in}}%
\pgfpathlineto{\pgfqpoint{0.502740in}{1.812384in}}%
\pgfpathlineto{\pgfqpoint{0.518927in}{1.834185in}}%
\pgfpathlineto{\pgfqpoint{0.535114in}{1.845290in}}%
\pgfpathlineto{\pgfqpoint{0.551301in}{1.846474in}}%
\pgfpathlineto{\pgfqpoint{0.567487in}{1.838605in}}%
\pgfpathlineto{\pgfqpoint{0.583674in}{1.822605in}}%
\pgfpathlineto{\pgfqpoint{0.599861in}{1.799425in}}%
\pgfpathlineto{\pgfqpoint{0.616048in}{1.770015in}}%
\pgfpathlineto{\pgfqpoint{0.632235in}{1.735307in}}%
\pgfpathlineto{\pgfqpoint{0.648422in}{1.696198in}}%
\pgfpathlineto{\pgfqpoint{0.664609in}{1.653537in}}%
\pgfpathlineto{\pgfqpoint{0.680795in}{1.608116in}}%
\pgfpathlineto{\pgfqpoint{0.696982in}{1.560662in}}%
\pgfpathlineto{\pgfqpoint{0.713169in}{1.511839in}}%
\pgfpathlineto{\pgfqpoint{0.729356in}{1.462240in}}%
\pgfpathlineto{\pgfqpoint{0.745543in}{1.412392in}}%
\pgfpathlineto{\pgfqpoint{0.761730in}{1.362754in}}%
\pgfpathlineto{\pgfqpoint{0.777917in}{1.313722in}}%
\pgfpathlineto{\pgfqpoint{0.794104in}{1.265632in}}%
\pgfpathlineto{\pgfqpoint{0.810290in}{1.218765in}}%
\pgfpathlineto{\pgfqpoint{0.826477in}{1.173348in}}%
\pgfpathlineto{\pgfqpoint{0.842664in}{1.129563in}}%
\pgfpathlineto{\pgfqpoint{0.858851in}{1.087548in}}%
\pgfpathlineto{\pgfqpoint{0.875038in}{1.047405in}}%
\pgfpathlineto{\pgfqpoint{0.891225in}{1.009200in}}%
\pgfpathlineto{\pgfqpoint{0.907412in}{0.972973in}}%
\pgfpathlineto{\pgfqpoint{0.923598in}{0.938736in}}%
\pgfpathlineto{\pgfqpoint{0.939785in}{0.906481in}}%
\pgfpathlineto{\pgfqpoint{0.955972in}{0.876182in}}%
\pgfpathlineto{\pgfqpoint{0.972159in}{0.847799in}}%
\pgfpathlineto{\pgfqpoint{0.988346in}{0.821278in}}%
\pgfpathlineto{\pgfqpoint{1.004533in}{0.796556in}}%
\pgfpathlineto{\pgfqpoint{1.020720in}{0.773564in}}%
\pgfpathlineto{\pgfqpoint{1.036907in}{0.752226in}}%
\pgfpathlineto{\pgfqpoint{1.053093in}{0.732463in}}%
\pgfpathlineto{\pgfqpoint{1.069280in}{0.714194in}}%
\pgfpathlineto{\pgfqpoint{1.085467in}{0.697337in}}%
\pgfpathlineto{\pgfqpoint{1.101654in}{0.681809in}}%
\pgfpathlineto{\pgfqpoint{1.117841in}{0.667529in}}%
\pgfpathlineto{\pgfqpoint{1.134028in}{0.654417in}}%
\pgfpathlineto{\pgfqpoint{1.150215in}{0.642395in}}%
\pgfpathlineto{\pgfqpoint{1.166402in}{0.631389in}}%
\pgfpathlineto{\pgfqpoint{1.182588in}{0.621325in}}%
\pgfpathlineto{\pgfqpoint{1.198775in}{0.612137in}}%
\pgfpathlineto{\pgfqpoint{1.214962in}{0.603757in}}%
\pgfpathlineto{\pgfqpoint{1.231149in}{0.596124in}}%
\pgfpathlineto{\pgfqpoint{1.247336in}{0.589179in}}%
\pgfpathlineto{\pgfqpoint{1.263523in}{0.582867in}}%
\pgfpathlineto{\pgfqpoint{1.279710in}{0.577137in}}%
\pgfpathlineto{\pgfqpoint{1.295896in}{0.571940in}}%
\pgfpathlineto{\pgfqpoint{1.312083in}{0.567230in}}%
\pgfpathlineto{\pgfqpoint{1.328270in}{0.562968in}}%
\pgfpathlineto{\pgfqpoint{1.344457in}{0.559112in}}%
\pgfpathlineto{\pgfqpoint{1.360644in}{0.555629in}}%
\pgfpathlineto{\pgfqpoint{1.376831in}{0.552483in}}%
\pgfpathlineto{\pgfqpoint{1.393018in}{0.549646in}}%
\pgfpathlineto{\pgfqpoint{1.409205in}{0.547088in}}%
\pgfpathlineto{\pgfqpoint{1.425391in}{0.544785in}}%
\pgfpathlineto{\pgfqpoint{1.441578in}{0.542712in}}%
\pgfpathlineto{\pgfqpoint{1.457765in}{0.540847in}}%
\pgfpathlineto{\pgfqpoint{1.473952in}{0.539171in}}%
\pgfpathlineto{\pgfqpoint{1.490139in}{0.537666in}}%
\pgfpathlineto{\pgfqpoint{1.506326in}{0.536314in}}%
\pgfpathlineto{\pgfqpoint{1.522513in}{0.535103in}}%
\pgfpathlineto{\pgfqpoint{1.538699in}{0.534016in}}%
\pgfpathlineto{\pgfqpoint{1.554886in}{0.533043in}}%
\pgfpathlineto{\pgfqpoint{1.571073in}{0.532171in}}%
\pgfpathlineto{\pgfqpoint{1.587260in}{0.531391in}}%
\pgfpathlineto{\pgfqpoint{1.603447in}{0.530693in}}%
\pgfpathlineto{\pgfqpoint{1.619634in}{0.530070in}}%
\pgfpathlineto{\pgfqpoint{1.635821in}{0.529513in}}%
\pgfpathlineto{\pgfqpoint{1.652008in}{0.529015in}}%
\pgfpathlineto{\pgfqpoint{1.668194in}{0.528571in}}%
\pgfpathlineto{\pgfqpoint{1.684381in}{0.528175in}}%
\pgfpathlineto{\pgfqpoint{1.700568in}{0.527821in}}%
\pgfpathlineto{\pgfqpoint{1.716755in}{0.527506in}}%
\pgfpathlineto{\pgfqpoint{1.732942in}{0.527226in}}%
\pgfpathlineto{\pgfqpoint{1.749129in}{0.526976in}}%
\pgfpathlineto{\pgfqpoint{1.765316in}{0.526753in}}%
\pgfpathlineto{\pgfqpoint{1.781503in}{0.526556in}}%
\pgfpathlineto{\pgfqpoint{1.797689in}{0.526379in}}%
\pgfpathlineto{\pgfqpoint{1.813876in}{0.526223in}}%
\pgfpathlineto{\pgfqpoint{1.830063in}{0.526084in}}%
\pgfpathlineto{\pgfqpoint{1.846250in}{0.525960in}}%
\pgfusepath{stroke}%
\end{pgfscope}%
\begin{pgfscope}%
\pgfsetrectcap%
\pgfsetmiterjoin%
\pgfsetlinewidth{1.003750pt}%
\definecolor{currentstroke}{rgb}{0.000000,0.000000,0.000000}%
\pgfsetstrokecolor{currentstroke}%
\pgfsetdash{}{0pt}%
\pgfpathmoveto{\pgfqpoint{0.243750in}{0.525000in}}%
\pgfpathlineto{\pgfqpoint{0.243750in}{1.912500in}}%
\pgfusepath{stroke}%
\end{pgfscope}%
\begin{pgfscope}%
\pgfsetrectcap%
\pgfsetmiterjoin%
\pgfsetlinewidth{1.003750pt}%
\definecolor{currentstroke}{rgb}{0.000000,0.000000,0.000000}%
\pgfsetstrokecolor{currentstroke}%
\pgfsetdash{}{0pt}%
\pgfpathmoveto{\pgfqpoint{1.846250in}{0.525000in}}%
\pgfpathlineto{\pgfqpoint{1.846250in}{1.912500in}}%
\pgfusepath{stroke}%
\end{pgfscope}%
\begin{pgfscope}%
\pgfsetrectcap%
\pgfsetmiterjoin%
\pgfsetlinewidth{1.003750pt}%
\definecolor{currentstroke}{rgb}{0.000000,0.000000,0.000000}%
\pgfsetstrokecolor{currentstroke}%
\pgfsetdash{}{0pt}%
\pgfpathmoveto{\pgfqpoint{0.243750in}{0.525000in}}%
\pgfpathlineto{\pgfqpoint{1.846250in}{0.525000in}}%
\pgfusepath{stroke}%
\end{pgfscope}%
\begin{pgfscope}%
\pgfsetrectcap%
\pgfsetmiterjoin%
\pgfsetlinewidth{1.003750pt}%
\definecolor{currentstroke}{rgb}{0.000000,0.000000,0.000000}%
\pgfsetstrokecolor{currentstroke}%
\pgfsetdash{}{0pt}%
\pgfpathmoveto{\pgfqpoint{0.243750in}{1.912500in}}%
\pgfpathlineto{\pgfqpoint{1.846250in}{1.912500in}}%
\pgfusepath{stroke}%
\end{pgfscope}%
\end{pgfpicture}%
\makeatother%
\endgroup%

  %% Creator: Matplotlib, PGF backend
%%
%% To include the figure in your LaTeX document, write
%%   \input{<filename>.pgf}
%%
%% Make sure the required packages are loaded in your preamble
%%   \usepackage{pgf}
%%
%% Also ensure that all the required font packages are loaded; for instance,
%% the lmodern package is sometimes necessary when using math font.
%%   \usepackage{lmodern}
%%
%% Figures using additional raster images can only be included by \input if
%% they are in the same directory as the main LaTeX file. For loading figures
%% from other directories you can use the `import` package
%%   \usepackage{import}
%%
%% and then include the figures with
%%   \import{<path to file>}{<filename>.pgf}
%%
%% Matplotlib used the following preamble
%%   \def\mathdefault#1{#1}
%%   \everymath=\expandafter{\the\everymath\displaystyle}
%%   \IfFileExists{scrextend.sty}{
%%     \usepackage[fontsize=9.000000pt]{scrextend}
%%   }{
%%     \renewcommand{\normalsize}{\fontsize{9.000000}{10.800000}\selectfont}
%%     \normalsize
%%   }
%%   
%%   \ifdefined\pdftexversion\else  % non-pdftex case.
%%     \usepackage{fontspec}
%%     \setmainfont{DejaVuSerif.ttf}[Path=\detokenize{/home/users/lbaldini/.pyenv/versions/3.13.1/lib/python3.13/site-packages/matplotlib/mpl-data/fonts/ttf/}]
%%     \setsansfont{DejaVuSans.ttf}[Path=\detokenize{/home/users/lbaldini/.pyenv/versions/3.13.1/lib/python3.13/site-packages/matplotlib/mpl-data/fonts/ttf/}]
%%     \setmonofont{DejaVuSansMono.ttf}[Path=\detokenize{/home/users/lbaldini/.pyenv/versions/3.13.1/lib/python3.13/site-packages/matplotlib/mpl-data/fonts/ttf/}]
%%   \fi
%%   \makeatletter\@ifpackageloaded{underscore}{}{\usepackage[strings]{underscore}}\makeatother
%%
\begingroup%
\makeatletter%
\begin{pgfpicture}%
\pgfpathrectangle{\pgfpointorigin}{\pgfqpoint{1.950000in}{2.000000in}}%
\pgfusepath{use as bounding box, clip}%
\begin{pgfscope}%
\pgfsetbuttcap%
\pgfsetmiterjoin%
\definecolor{currentfill}{rgb}{1.000000,1.000000,1.000000}%
\pgfsetfillcolor{currentfill}%
\pgfsetlinewidth{0.000000pt}%
\definecolor{currentstroke}{rgb}{1.000000,1.000000,1.000000}%
\pgfsetstrokecolor{currentstroke}%
\pgfsetdash{}{0pt}%
\pgfpathmoveto{\pgfqpoint{0.000000in}{0.000000in}}%
\pgfpathlineto{\pgfqpoint{1.950000in}{0.000000in}}%
\pgfpathlineto{\pgfqpoint{1.950000in}{2.000000in}}%
\pgfpathlineto{\pgfqpoint{0.000000in}{2.000000in}}%
\pgfpathlineto{\pgfqpoint{0.000000in}{0.000000in}}%
\pgfpathclose%
\pgfusepath{fill}%
\end{pgfscope}%
\begin{pgfscope}%
\pgfsetbuttcap%
\pgfsetmiterjoin%
\definecolor{currentfill}{rgb}{1.000000,1.000000,1.000000}%
\pgfsetfillcolor{currentfill}%
\pgfsetlinewidth{0.000000pt}%
\definecolor{currentstroke}{rgb}{0.000000,0.000000,0.000000}%
\pgfsetstrokecolor{currentstroke}%
\pgfsetstrokeopacity{0.000000}%
\pgfsetdash{}{0pt}%
\pgfpathmoveto{\pgfqpoint{0.243750in}{0.525000in}}%
\pgfpathlineto{\pgfqpoint{1.846250in}{0.525000in}}%
\pgfpathlineto{\pgfqpoint{1.846250in}{1.912500in}}%
\pgfpathlineto{\pgfqpoint{0.243750in}{1.912500in}}%
\pgfpathlineto{\pgfqpoint{0.243750in}{0.525000in}}%
\pgfpathclose%
\pgfusepath{fill}%
\end{pgfscope}%
\begin{pgfscope}%
\pgfpathrectangle{\pgfqpoint{0.243750in}{0.525000in}}{\pgfqpoint{1.602500in}{1.387500in}}%
\pgfusepath{clip}%
\pgfsetbuttcap%
\pgfsetroundjoin%
\pgfsetlinewidth{0.803000pt}%
\definecolor{currentstroke}{rgb}{0.752941,0.752941,0.752941}%
\pgfsetstrokecolor{currentstroke}%
\pgfsetdash{{2.960000pt}{1.280000pt}}{0.000000pt}%
\pgfpathmoveto{\pgfqpoint{0.243750in}{0.525000in}}%
\pgfpathlineto{\pgfqpoint{0.243750in}{1.912500in}}%
\pgfusepath{stroke}%
\end{pgfscope}%
\begin{pgfscope}%
\pgfsetbuttcap%
\pgfsetroundjoin%
\definecolor{currentfill}{rgb}{0.000000,0.000000,0.000000}%
\pgfsetfillcolor{currentfill}%
\pgfsetlinewidth{0.803000pt}%
\definecolor{currentstroke}{rgb}{0.000000,0.000000,0.000000}%
\pgfsetstrokecolor{currentstroke}%
\pgfsetdash{}{0pt}%
\pgfsys@defobject{currentmarker}{\pgfqpoint{0.000000in}{-0.048611in}}{\pgfqpoint{0.000000in}{0.000000in}}{%
\pgfpathmoveto{\pgfqpoint{0.000000in}{0.000000in}}%
\pgfpathlineto{\pgfqpoint{0.000000in}{-0.048611in}}%
\pgfusepath{stroke,fill}%
}%
\begin{pgfscope}%
\pgfsys@transformshift{0.243750in}{0.525000in}%
\pgfsys@useobject{currentmarker}{}%
\end{pgfscope}%
\end{pgfscope}%
\begin{pgfscope}%
\definecolor{textcolor}{rgb}{0.000000,0.000000,0.000000}%
\pgfsetstrokecolor{textcolor}%
\pgfsetfillcolor{textcolor}%
\pgftext[x=0.243750in,y=0.427778in,,top]{\color{textcolor}{\rmfamily\fontsize{9.000000}{10.800000}\selectfont\catcode`\^=\active\def^{\ifmmode\sp\else\^{}\fi}\catcode`\%=\active\def%{\%}0}}%
\end{pgfscope}%
\begin{pgfscope}%
\pgfpathrectangle{\pgfqpoint{0.243750in}{0.525000in}}{\pgfqpoint{1.602500in}{1.387500in}}%
\pgfusepath{clip}%
\pgfsetbuttcap%
\pgfsetroundjoin%
\pgfsetlinewidth{0.803000pt}%
\definecolor{currentstroke}{rgb}{0.752941,0.752941,0.752941}%
\pgfsetstrokecolor{currentstroke}%
\pgfsetdash{{2.960000pt}{1.280000pt}}{0.000000pt}%
\pgfpathmoveto{\pgfqpoint{0.457417in}{0.525000in}}%
\pgfpathlineto{\pgfqpoint{0.457417in}{1.912500in}}%
\pgfusepath{stroke}%
\end{pgfscope}%
\begin{pgfscope}%
\pgfsetbuttcap%
\pgfsetroundjoin%
\definecolor{currentfill}{rgb}{0.000000,0.000000,0.000000}%
\pgfsetfillcolor{currentfill}%
\pgfsetlinewidth{0.803000pt}%
\definecolor{currentstroke}{rgb}{0.000000,0.000000,0.000000}%
\pgfsetstrokecolor{currentstroke}%
\pgfsetdash{}{0pt}%
\pgfsys@defobject{currentmarker}{\pgfqpoint{0.000000in}{-0.048611in}}{\pgfqpoint{0.000000in}{0.000000in}}{%
\pgfpathmoveto{\pgfqpoint{0.000000in}{0.000000in}}%
\pgfpathlineto{\pgfqpoint{0.000000in}{-0.048611in}}%
\pgfusepath{stroke,fill}%
}%
\begin{pgfscope}%
\pgfsys@transformshift{0.457417in}{0.525000in}%
\pgfsys@useobject{currentmarker}{}%
\end{pgfscope}%
\end{pgfscope}%
\begin{pgfscope}%
\definecolor{textcolor}{rgb}{0.000000,0.000000,0.000000}%
\pgfsetstrokecolor{textcolor}%
\pgfsetfillcolor{textcolor}%
\pgftext[x=0.457417in,y=0.427778in,,top]{\color{textcolor}{\rmfamily\fontsize{9.000000}{10.800000}\selectfont\catcode`\^=\active\def^{\ifmmode\sp\else\^{}\fi}\catcode`\%=\active\def%{\%}2}}%
\end{pgfscope}%
\begin{pgfscope}%
\pgfpathrectangle{\pgfqpoint{0.243750in}{0.525000in}}{\pgfqpoint{1.602500in}{1.387500in}}%
\pgfusepath{clip}%
\pgfsetbuttcap%
\pgfsetroundjoin%
\pgfsetlinewidth{0.803000pt}%
\definecolor{currentstroke}{rgb}{0.752941,0.752941,0.752941}%
\pgfsetstrokecolor{currentstroke}%
\pgfsetdash{{2.960000pt}{1.280000pt}}{0.000000pt}%
\pgfpathmoveto{\pgfqpoint{0.671083in}{0.525000in}}%
\pgfpathlineto{\pgfqpoint{0.671083in}{1.912500in}}%
\pgfusepath{stroke}%
\end{pgfscope}%
\begin{pgfscope}%
\pgfsetbuttcap%
\pgfsetroundjoin%
\definecolor{currentfill}{rgb}{0.000000,0.000000,0.000000}%
\pgfsetfillcolor{currentfill}%
\pgfsetlinewidth{0.803000pt}%
\definecolor{currentstroke}{rgb}{0.000000,0.000000,0.000000}%
\pgfsetstrokecolor{currentstroke}%
\pgfsetdash{}{0pt}%
\pgfsys@defobject{currentmarker}{\pgfqpoint{0.000000in}{-0.048611in}}{\pgfqpoint{0.000000in}{0.000000in}}{%
\pgfpathmoveto{\pgfqpoint{0.000000in}{0.000000in}}%
\pgfpathlineto{\pgfqpoint{0.000000in}{-0.048611in}}%
\pgfusepath{stroke,fill}%
}%
\begin{pgfscope}%
\pgfsys@transformshift{0.671083in}{0.525000in}%
\pgfsys@useobject{currentmarker}{}%
\end{pgfscope}%
\end{pgfscope}%
\begin{pgfscope}%
\definecolor{textcolor}{rgb}{0.000000,0.000000,0.000000}%
\pgfsetstrokecolor{textcolor}%
\pgfsetfillcolor{textcolor}%
\pgftext[x=0.671083in,y=0.427778in,,top]{\color{textcolor}{\rmfamily\fontsize{9.000000}{10.800000}\selectfont\catcode`\^=\active\def^{\ifmmode\sp\else\^{}\fi}\catcode`\%=\active\def%{\%}4}}%
\end{pgfscope}%
\begin{pgfscope}%
\pgfpathrectangle{\pgfqpoint{0.243750in}{0.525000in}}{\pgfqpoint{1.602500in}{1.387500in}}%
\pgfusepath{clip}%
\pgfsetbuttcap%
\pgfsetroundjoin%
\pgfsetlinewidth{0.803000pt}%
\definecolor{currentstroke}{rgb}{0.752941,0.752941,0.752941}%
\pgfsetstrokecolor{currentstroke}%
\pgfsetdash{{2.960000pt}{1.280000pt}}{0.000000pt}%
\pgfpathmoveto{\pgfqpoint{0.884750in}{0.525000in}}%
\pgfpathlineto{\pgfqpoint{0.884750in}{1.912500in}}%
\pgfusepath{stroke}%
\end{pgfscope}%
\begin{pgfscope}%
\pgfsetbuttcap%
\pgfsetroundjoin%
\definecolor{currentfill}{rgb}{0.000000,0.000000,0.000000}%
\pgfsetfillcolor{currentfill}%
\pgfsetlinewidth{0.803000pt}%
\definecolor{currentstroke}{rgb}{0.000000,0.000000,0.000000}%
\pgfsetstrokecolor{currentstroke}%
\pgfsetdash{}{0pt}%
\pgfsys@defobject{currentmarker}{\pgfqpoint{0.000000in}{-0.048611in}}{\pgfqpoint{0.000000in}{0.000000in}}{%
\pgfpathmoveto{\pgfqpoint{0.000000in}{0.000000in}}%
\pgfpathlineto{\pgfqpoint{0.000000in}{-0.048611in}}%
\pgfusepath{stroke,fill}%
}%
\begin{pgfscope}%
\pgfsys@transformshift{0.884750in}{0.525000in}%
\pgfsys@useobject{currentmarker}{}%
\end{pgfscope}%
\end{pgfscope}%
\begin{pgfscope}%
\definecolor{textcolor}{rgb}{0.000000,0.000000,0.000000}%
\pgfsetstrokecolor{textcolor}%
\pgfsetfillcolor{textcolor}%
\pgftext[x=0.884750in,y=0.427778in,,top]{\color{textcolor}{\rmfamily\fontsize{9.000000}{10.800000}\selectfont\catcode`\^=\active\def^{\ifmmode\sp\else\^{}\fi}\catcode`\%=\active\def%{\%}6}}%
\end{pgfscope}%
\begin{pgfscope}%
\pgfpathrectangle{\pgfqpoint{0.243750in}{0.525000in}}{\pgfqpoint{1.602500in}{1.387500in}}%
\pgfusepath{clip}%
\pgfsetbuttcap%
\pgfsetroundjoin%
\pgfsetlinewidth{0.803000pt}%
\definecolor{currentstroke}{rgb}{0.752941,0.752941,0.752941}%
\pgfsetstrokecolor{currentstroke}%
\pgfsetdash{{2.960000pt}{1.280000pt}}{0.000000pt}%
\pgfpathmoveto{\pgfqpoint{1.098417in}{0.525000in}}%
\pgfpathlineto{\pgfqpoint{1.098417in}{1.912500in}}%
\pgfusepath{stroke}%
\end{pgfscope}%
\begin{pgfscope}%
\pgfsetbuttcap%
\pgfsetroundjoin%
\definecolor{currentfill}{rgb}{0.000000,0.000000,0.000000}%
\pgfsetfillcolor{currentfill}%
\pgfsetlinewidth{0.803000pt}%
\definecolor{currentstroke}{rgb}{0.000000,0.000000,0.000000}%
\pgfsetstrokecolor{currentstroke}%
\pgfsetdash{}{0pt}%
\pgfsys@defobject{currentmarker}{\pgfqpoint{0.000000in}{-0.048611in}}{\pgfqpoint{0.000000in}{0.000000in}}{%
\pgfpathmoveto{\pgfqpoint{0.000000in}{0.000000in}}%
\pgfpathlineto{\pgfqpoint{0.000000in}{-0.048611in}}%
\pgfusepath{stroke,fill}%
}%
\begin{pgfscope}%
\pgfsys@transformshift{1.098417in}{0.525000in}%
\pgfsys@useobject{currentmarker}{}%
\end{pgfscope}%
\end{pgfscope}%
\begin{pgfscope}%
\definecolor{textcolor}{rgb}{0.000000,0.000000,0.000000}%
\pgfsetstrokecolor{textcolor}%
\pgfsetfillcolor{textcolor}%
\pgftext[x=1.098417in,y=0.427778in,,top]{\color{textcolor}{\rmfamily\fontsize{9.000000}{10.800000}\selectfont\catcode`\^=\active\def^{\ifmmode\sp\else\^{}\fi}\catcode`\%=\active\def%{\%}8}}%
\end{pgfscope}%
\begin{pgfscope}%
\pgfpathrectangle{\pgfqpoint{0.243750in}{0.525000in}}{\pgfqpoint{1.602500in}{1.387500in}}%
\pgfusepath{clip}%
\pgfsetbuttcap%
\pgfsetroundjoin%
\pgfsetlinewidth{0.803000pt}%
\definecolor{currentstroke}{rgb}{0.752941,0.752941,0.752941}%
\pgfsetstrokecolor{currentstroke}%
\pgfsetdash{{2.960000pt}{1.280000pt}}{0.000000pt}%
\pgfpathmoveto{\pgfqpoint{1.312083in}{0.525000in}}%
\pgfpathlineto{\pgfqpoint{1.312083in}{1.912500in}}%
\pgfusepath{stroke}%
\end{pgfscope}%
\begin{pgfscope}%
\pgfsetbuttcap%
\pgfsetroundjoin%
\definecolor{currentfill}{rgb}{0.000000,0.000000,0.000000}%
\pgfsetfillcolor{currentfill}%
\pgfsetlinewidth{0.803000pt}%
\definecolor{currentstroke}{rgb}{0.000000,0.000000,0.000000}%
\pgfsetstrokecolor{currentstroke}%
\pgfsetdash{}{0pt}%
\pgfsys@defobject{currentmarker}{\pgfqpoint{0.000000in}{-0.048611in}}{\pgfqpoint{0.000000in}{0.000000in}}{%
\pgfpathmoveto{\pgfqpoint{0.000000in}{0.000000in}}%
\pgfpathlineto{\pgfqpoint{0.000000in}{-0.048611in}}%
\pgfusepath{stroke,fill}%
}%
\begin{pgfscope}%
\pgfsys@transformshift{1.312083in}{0.525000in}%
\pgfsys@useobject{currentmarker}{}%
\end{pgfscope}%
\end{pgfscope}%
\begin{pgfscope}%
\definecolor{textcolor}{rgb}{0.000000,0.000000,0.000000}%
\pgfsetstrokecolor{textcolor}%
\pgfsetfillcolor{textcolor}%
\pgftext[x=1.312083in,y=0.427778in,,top]{\color{textcolor}{\rmfamily\fontsize{9.000000}{10.800000}\selectfont\catcode`\^=\active\def^{\ifmmode\sp\else\^{}\fi}\catcode`\%=\active\def%{\%}10}}%
\end{pgfscope}%
\begin{pgfscope}%
\pgfpathrectangle{\pgfqpoint{0.243750in}{0.525000in}}{\pgfqpoint{1.602500in}{1.387500in}}%
\pgfusepath{clip}%
\pgfsetbuttcap%
\pgfsetroundjoin%
\pgfsetlinewidth{0.803000pt}%
\definecolor{currentstroke}{rgb}{0.752941,0.752941,0.752941}%
\pgfsetstrokecolor{currentstroke}%
\pgfsetdash{{2.960000pt}{1.280000pt}}{0.000000pt}%
\pgfpathmoveto{\pgfqpoint{1.525750in}{0.525000in}}%
\pgfpathlineto{\pgfqpoint{1.525750in}{1.912500in}}%
\pgfusepath{stroke}%
\end{pgfscope}%
\begin{pgfscope}%
\pgfsetbuttcap%
\pgfsetroundjoin%
\definecolor{currentfill}{rgb}{0.000000,0.000000,0.000000}%
\pgfsetfillcolor{currentfill}%
\pgfsetlinewidth{0.803000pt}%
\definecolor{currentstroke}{rgb}{0.000000,0.000000,0.000000}%
\pgfsetstrokecolor{currentstroke}%
\pgfsetdash{}{0pt}%
\pgfsys@defobject{currentmarker}{\pgfqpoint{0.000000in}{-0.048611in}}{\pgfqpoint{0.000000in}{0.000000in}}{%
\pgfpathmoveto{\pgfqpoint{0.000000in}{0.000000in}}%
\pgfpathlineto{\pgfqpoint{0.000000in}{-0.048611in}}%
\pgfusepath{stroke,fill}%
}%
\begin{pgfscope}%
\pgfsys@transformshift{1.525750in}{0.525000in}%
\pgfsys@useobject{currentmarker}{}%
\end{pgfscope}%
\end{pgfscope}%
\begin{pgfscope}%
\definecolor{textcolor}{rgb}{0.000000,0.000000,0.000000}%
\pgfsetstrokecolor{textcolor}%
\pgfsetfillcolor{textcolor}%
\pgftext[x=1.525750in,y=0.427778in,,top]{\color{textcolor}{\rmfamily\fontsize{9.000000}{10.800000}\selectfont\catcode`\^=\active\def^{\ifmmode\sp\else\^{}\fi}\catcode`\%=\active\def%{\%}12}}%
\end{pgfscope}%
\begin{pgfscope}%
\pgfpathrectangle{\pgfqpoint{0.243750in}{0.525000in}}{\pgfqpoint{1.602500in}{1.387500in}}%
\pgfusepath{clip}%
\pgfsetbuttcap%
\pgfsetroundjoin%
\pgfsetlinewidth{0.803000pt}%
\definecolor{currentstroke}{rgb}{0.752941,0.752941,0.752941}%
\pgfsetstrokecolor{currentstroke}%
\pgfsetdash{{2.960000pt}{1.280000pt}}{0.000000pt}%
\pgfpathmoveto{\pgfqpoint{1.739417in}{0.525000in}}%
\pgfpathlineto{\pgfqpoint{1.739417in}{1.912500in}}%
\pgfusepath{stroke}%
\end{pgfscope}%
\begin{pgfscope}%
\pgfsetbuttcap%
\pgfsetroundjoin%
\definecolor{currentfill}{rgb}{0.000000,0.000000,0.000000}%
\pgfsetfillcolor{currentfill}%
\pgfsetlinewidth{0.803000pt}%
\definecolor{currentstroke}{rgb}{0.000000,0.000000,0.000000}%
\pgfsetstrokecolor{currentstroke}%
\pgfsetdash{}{0pt}%
\pgfsys@defobject{currentmarker}{\pgfqpoint{0.000000in}{-0.048611in}}{\pgfqpoint{0.000000in}{0.000000in}}{%
\pgfpathmoveto{\pgfqpoint{0.000000in}{0.000000in}}%
\pgfpathlineto{\pgfqpoint{0.000000in}{-0.048611in}}%
\pgfusepath{stroke,fill}%
}%
\begin{pgfscope}%
\pgfsys@transformshift{1.739417in}{0.525000in}%
\pgfsys@useobject{currentmarker}{}%
\end{pgfscope}%
\end{pgfscope}%
\begin{pgfscope}%
\definecolor{textcolor}{rgb}{0.000000,0.000000,0.000000}%
\pgfsetstrokecolor{textcolor}%
\pgfsetfillcolor{textcolor}%
\pgftext[x=1.739417in,y=0.427778in,,top]{\color{textcolor}{\rmfamily\fontsize{9.000000}{10.800000}\selectfont\catcode`\^=\active\def^{\ifmmode\sp\else\^{}\fi}\catcode`\%=\active\def%{\%}14}}%
\end{pgfscope}%
\begin{pgfscope}%
\definecolor{textcolor}{rgb}{0.000000,0.000000,0.000000}%
\pgfsetstrokecolor{textcolor}%
\pgfsetfillcolor{textcolor}%
\pgftext[x=1.045000in,y=0.251251in,,top]{\color{textcolor}{\rmfamily\fontsize{9.000000}{10.800000}\selectfont\catcode`\^=\active\def^{\ifmmode\sp\else\^{}\fi}\catcode`\%=\active\def%{\%}$hc / \lambda kT$}}%
\end{pgfscope}%
\begin{pgfscope}%
\definecolor{textcolor}{rgb}{0.000000,0.000000,0.000000}%
\pgfsetstrokecolor{textcolor}%
\pgfsetfillcolor{textcolor}%
\pgftext[x=0.188194in,y=1.218750in,,bottom,rotate=90.000000]{\color{textcolor}{\rmfamily\fontsize{9.000000}{10.800000}\selectfont\catcode`\^=\active\def^{\ifmmode\sp\else\^{}\fi}\catcode`\%=\active\def%{\%}$u_\lambda(\lambda)$}}%
\end{pgfscope}%
\begin{pgfscope}%
\pgfpathrectangle{\pgfqpoint{0.243750in}{0.525000in}}{\pgfqpoint{1.602500in}{1.387500in}}%
\pgfusepath{clip}%
\pgfsetrectcap%
\pgfsetroundjoin%
\pgfsetlinewidth{1.003750pt}%
\definecolor{currentstroke}{rgb}{0.000000,0.000000,0.000000}%
\pgfsetstrokecolor{currentstroke}%
\pgfsetdash{}{0pt}%
\pgfpathmoveto{\pgfqpoint{0.259937in}{0.525030in}}%
\pgfpathlineto{\pgfqpoint{0.276124in}{0.525450in}}%
\pgfpathlineto{\pgfqpoint{0.292311in}{0.527102in}}%
\pgfpathlineto{\pgfqpoint{0.308497in}{0.531117in}}%
\pgfpathlineto{\pgfqpoint{0.324684in}{0.538728in}}%
\pgfpathlineto{\pgfqpoint{0.340871in}{0.551116in}}%
\pgfpathlineto{\pgfqpoint{0.357058in}{0.569307in}}%
\pgfpathlineto{\pgfqpoint{0.373245in}{0.594094in}}%
\pgfpathlineto{\pgfqpoint{0.389432in}{0.625989in}}%
\pgfpathlineto{\pgfqpoint{0.405619in}{0.665207in}}%
\pgfpathlineto{\pgfqpoint{0.421806in}{0.711665in}}%
\pgfpathlineto{\pgfqpoint{0.437992in}{0.765001in}}%
\pgfpathlineto{\pgfqpoint{0.454179in}{0.824604in}}%
\pgfpathlineto{\pgfqpoint{0.470366in}{0.889654in}}%
\pgfpathlineto{\pgfqpoint{0.486553in}{0.959167in}}%
\pgfpathlineto{\pgfqpoint{0.502740in}{1.032038in}}%
\pgfpathlineto{\pgfqpoint{0.518927in}{1.107091in}}%
\pgfpathlineto{\pgfqpoint{0.535114in}{1.183123in}}%
\pgfpathlineto{\pgfqpoint{0.551301in}{1.258936in}}%
\pgfpathlineto{\pgfqpoint{0.567487in}{1.333383in}}%
\pgfpathlineto{\pgfqpoint{0.583674in}{1.405387in}}%
\pgfpathlineto{\pgfqpoint{0.599861in}{1.473969in}}%
\pgfpathlineto{\pgfqpoint{0.616048in}{1.538264in}}%
\pgfpathlineto{\pgfqpoint{0.632235in}{1.597533in}}%
\pgfpathlineto{\pgfqpoint{0.648422in}{1.651167in}}%
\pgfpathlineto{\pgfqpoint{0.664609in}{1.698695in}}%
\pgfpathlineto{\pgfqpoint{0.680795in}{1.739772in}}%
\pgfpathlineto{\pgfqpoint{0.696982in}{1.774185in}}%
\pgfpathlineto{\pgfqpoint{0.713169in}{1.801835in}}%
\pgfpathlineto{\pgfqpoint{0.729356in}{1.822735in}}%
\pgfpathlineto{\pgfqpoint{0.745543in}{1.836993in}}%
\pgfpathlineto{\pgfqpoint{0.761730in}{1.844802in}}%
\pgfpathlineto{\pgfqpoint{0.777917in}{1.846430in}}%
\pgfpathlineto{\pgfqpoint{0.794104in}{1.842203in}}%
\pgfpathlineto{\pgfqpoint{0.810290in}{1.832497in}}%
\pgfpathlineto{\pgfqpoint{0.826477in}{1.817723in}}%
\pgfpathlineto{\pgfqpoint{0.842664in}{1.798319in}}%
\pgfpathlineto{\pgfqpoint{0.858851in}{1.774739in}}%
\pgfpathlineto{\pgfqpoint{0.875038in}{1.747443in}}%
\pgfpathlineto{\pgfqpoint{0.891225in}{1.716893in}}%
\pgfpathlineto{\pgfqpoint{0.907412in}{1.683542in}}%
\pgfpathlineto{\pgfqpoint{0.923598in}{1.647830in}}%
\pgfpathlineto{\pgfqpoint{0.939785in}{1.610181in}}%
\pgfpathlineto{\pgfqpoint{0.955972in}{1.570997in}}%
\pgfpathlineto{\pgfqpoint{0.972159in}{1.530657in}}%
\pgfpathlineto{\pgfqpoint{0.988346in}{1.489511in}}%
\pgfpathlineto{\pgfqpoint{1.004533in}{1.447886in}}%
\pgfpathlineto{\pgfqpoint{1.020720in}{1.406076in}}%
\pgfpathlineto{\pgfqpoint{1.036907in}{1.364349in}}%
\pgfpathlineto{\pgfqpoint{1.053093in}{1.322946in}}%
\pgfpathlineto{\pgfqpoint{1.069280in}{1.282079in}}%
\pgfpathlineto{\pgfqpoint{1.085467in}{1.241932in}}%
\pgfpathlineto{\pgfqpoint{1.101654in}{1.202665in}}%
\pgfpathlineto{\pgfqpoint{1.117841in}{1.164415in}}%
\pgfpathlineto{\pgfqpoint{1.134028in}{1.127293in}}%
\pgfpathlineto{\pgfqpoint{1.150215in}{1.091393in}}%
\pgfpathlineto{\pgfqpoint{1.166402in}{1.056786in}}%
\pgfpathlineto{\pgfqpoint{1.182588in}{1.023528in}}%
\pgfpathlineto{\pgfqpoint{1.198775in}{0.991658in}}%
\pgfpathlineto{\pgfqpoint{1.214962in}{0.961199in}}%
\pgfpathlineto{\pgfqpoint{1.231149in}{0.932163in}}%
\pgfpathlineto{\pgfqpoint{1.247336in}{0.904550in}}%
\pgfpathlineto{\pgfqpoint{1.263523in}{0.878351in}}%
\pgfpathlineto{\pgfqpoint{1.279710in}{0.853547in}}%
\pgfpathlineto{\pgfqpoint{1.295896in}{0.830111in}}%
\pgfpathlineto{\pgfqpoint{1.312083in}{0.808012in}}%
\pgfpathlineto{\pgfqpoint{1.328270in}{0.787213in}}%
\pgfpathlineto{\pgfqpoint{1.344457in}{0.767673in}}%
\pgfpathlineto{\pgfqpoint{1.360644in}{0.749345in}}%
\pgfpathlineto{\pgfqpoint{1.376831in}{0.732185in}}%
\pgfpathlineto{\pgfqpoint{1.393018in}{0.716142in}}%
\pgfpathlineto{\pgfqpoint{1.409205in}{0.701166in}}%
\pgfpathlineto{\pgfqpoint{1.425391in}{0.687208in}}%
\pgfpathlineto{\pgfqpoint{1.441578in}{0.674215in}}%
\pgfpathlineto{\pgfqpoint{1.457765in}{0.662138in}}%
\pgfpathlineto{\pgfqpoint{1.473952in}{0.650926in}}%
\pgfpathlineto{\pgfqpoint{1.490139in}{0.640531in}}%
\pgfpathlineto{\pgfqpoint{1.506326in}{0.630905in}}%
\pgfpathlineto{\pgfqpoint{1.522513in}{0.622001in}}%
\pgfpathlineto{\pgfqpoint{1.538699in}{0.613774in}}%
\pgfpathlineto{\pgfqpoint{1.554886in}{0.606182in}}%
\pgfpathlineto{\pgfqpoint{1.571073in}{0.599183in}}%
\pgfpathlineto{\pgfqpoint{1.587260in}{0.592736in}}%
\pgfpathlineto{\pgfqpoint{1.603447in}{0.586805in}}%
\pgfpathlineto{\pgfqpoint{1.619634in}{0.581353in}}%
\pgfpathlineto{\pgfqpoint{1.635821in}{0.576347in}}%
\pgfpathlineto{\pgfqpoint{1.652008in}{0.571754in}}%
\pgfpathlineto{\pgfqpoint{1.668194in}{0.567543in}}%
\pgfpathlineto{\pgfqpoint{1.684381in}{0.563687in}}%
\pgfpathlineto{\pgfqpoint{1.700568in}{0.560158in}}%
\pgfpathlineto{\pgfqpoint{1.716755in}{0.556931in}}%
\pgfpathlineto{\pgfqpoint{1.732942in}{0.553983in}}%
\pgfpathlineto{\pgfqpoint{1.749129in}{0.551292in}}%
\pgfpathlineto{\pgfqpoint{1.765316in}{0.548837in}}%
\pgfpathlineto{\pgfqpoint{1.781503in}{0.546598in}}%
\pgfpathlineto{\pgfqpoint{1.797689in}{0.544559in}}%
\pgfpathlineto{\pgfqpoint{1.813876in}{0.542703in}}%
\pgfpathlineto{\pgfqpoint{1.830063in}{0.541015in}}%
\pgfpathlineto{\pgfqpoint{1.846250in}{0.539480in}}%
\pgfusepath{stroke}%
\end{pgfscope}%
\begin{pgfscope}%
\pgfsetrectcap%
\pgfsetmiterjoin%
\pgfsetlinewidth{1.003750pt}%
\definecolor{currentstroke}{rgb}{0.000000,0.000000,0.000000}%
\pgfsetstrokecolor{currentstroke}%
\pgfsetdash{}{0pt}%
\pgfpathmoveto{\pgfqpoint{0.243750in}{0.525000in}}%
\pgfpathlineto{\pgfqpoint{0.243750in}{1.912500in}}%
\pgfusepath{stroke}%
\end{pgfscope}%
\begin{pgfscope}%
\pgfsetrectcap%
\pgfsetmiterjoin%
\pgfsetlinewidth{1.003750pt}%
\definecolor{currentstroke}{rgb}{0.000000,0.000000,0.000000}%
\pgfsetstrokecolor{currentstroke}%
\pgfsetdash{}{0pt}%
\pgfpathmoveto{\pgfqpoint{1.846250in}{0.525000in}}%
\pgfpathlineto{\pgfqpoint{1.846250in}{1.912500in}}%
\pgfusepath{stroke}%
\end{pgfscope}%
\begin{pgfscope}%
\pgfsetrectcap%
\pgfsetmiterjoin%
\pgfsetlinewidth{1.003750pt}%
\definecolor{currentstroke}{rgb}{0.000000,0.000000,0.000000}%
\pgfsetstrokecolor{currentstroke}%
\pgfsetdash{}{0pt}%
\pgfpathmoveto{\pgfqpoint{0.243750in}{0.525000in}}%
\pgfpathlineto{\pgfqpoint{1.846250in}{0.525000in}}%
\pgfusepath{stroke}%
\end{pgfscope}%
\begin{pgfscope}%
\pgfsetrectcap%
\pgfsetmiterjoin%
\pgfsetlinewidth{1.003750pt}%
\definecolor{currentstroke}{rgb}{0.000000,0.000000,0.000000}%
\pgfsetstrokecolor{currentstroke}%
\pgfsetdash{}{0pt}%
\pgfpathmoveto{\pgfqpoint{0.243750in}{1.912500in}}%
\pgfpathlineto{\pgfqpoint{1.846250in}{1.912500in}}%
\pgfusepath{stroke}%
\end{pgfscope}%
\end{pgfpicture}%
\makeatother%
\endgroup%

  \caption{Universal shape of the Planck spectral density in
  frequency~\eqref{eq:planck_spec_density_nu} and wavelength~\eqref{eq:planck_spec_density_lambda}
  space.}
  \label{fig:planck_distribution}
\end{marginfigure}



\subsection{Wien's displacement law}
\label{sec:wien_law}

Planck's formulation of the theory of blackbody radiation bears a number of important
consequences. The spectral energy density in frequency space can be written, e.g., as
\begin{align*}
  u_\nu(\nu; T) \propto \frac{x^3}{e^x - 1}
  \quad\text{with}\quad
  x = \frac{h\nu}{\kT},
\end{align*}
and the peak of the distribution satisfies the condition
\begin{align*}
  \dv{x} (\frac{x^3}{e^x - 1}) = 0
  \quad\text{or}\quad
  e^{-x} + \frac{1}{3}x - 1 = 0.
\end{align*}
This is a transcendental equation that can only be solved numerically, yielding
$x_\text{peak} \approx 2.82144$, or
\begin{align}
  \nu_\text{peak} = b_\nu T
  \quad\text{with}\quad
  b_\nu \approx \frac{2.82144~\kB}{h} \approx
  5.879 \times 10^{10}~\text{s}^{-1}~^\circ \text{K}^{-1},
\end{align}
which is historically knows as Wien's displacement law: the frequency $\nu_\text{peak}$
at which the spectral energy distribution peaks is proportional to the temperature
$T$ of the radiating body.

The same exact reasoning in wavelength space
\begin{align*}
  u_\lambda(\lambda; T) \propto \frac{x^5}{e^x - 1}
  \quad\text{with}\quad
  x = \frac{hc}{\lambda\kT},
\end{align*}
provides a similar trascendental equation, yielding $x_\text{peak} \approx 4.96511$,
which allows to recast Wien's displacement law as
\begin{align}
  \lambda_\text{peak} = \frac{b_\lambda}{T}
  \quad\text{with}\quad
  b_\nu \approx \frac{hc}{4.96511~\kB} \approx
  0.2898~\text{cm}~^\circ \text{K}.
\end{align}



\subsection{The Stefan-Boltzmann law}
\label{sec:stefan_boltzmann_law}

If we integrate the spectral energy distribution, in any of the equivalent representations,
we get the total energy per unit volume
\begin{align}
  u(T) & = \int_0^\infty u_E(E; T)\diff{E} =
  \frac{8\pi}{(hc)^3} \int_0^\infty \frac{E^3}{e^\frac{E}{\kT} - 1} \diff{E} =\\\nonumber
  & = \frac{8\pi \kB^4 T^4}{h^3 c^3}
  \underbrace{\int_0^\infty \frac{z^3}{e^z - 1} \diff{z}}_{\nicefrac{\pi^4}{15}} =
  \frac{8\pi^5 \kB^4}{15 h^3 c^3} T^4,
\end{align}
which, notably, is proportional to the fourth power of the temperature. Since, in
the same vein, the total number density reads
\begin{align}
  n(T) & = \int_0^\infty \dv{n}{E}{(E; T)}\diff{E} =
  \frac{8\pi}{(hc)^3} \int_0^\infty \frac{E^2}{e^\frac{E}{\kT} - 1} \diff{E} =\\\nonumber
  & = \frac{8\pi \kB^3 T^3}{h^3 c^3}
  \underbrace{\int_0^\infty \frac{z^2}{e^z - 1} \diff{z}}_{2 \zeta(3)} =
  \frac{16\pi \zeta(3) \kB^3}{h^3 c^3} T^3,
\end{align}
the average photon energy at a given temperature $T$ is readily calculated by
dividing the total energy by the total number of photons
\begin{align}
  \ave{E} = \frac{u(E)}{n(E)} = \frac{\pi^4}{30 \zeta(3)} \kT = 2.7012 \; \kT.
\end{align}
This is close (but not quite identical) to the peak of the spectral energy distribution
in energy space.

It is worth noting that, in this context, there are at least three tightly related
concepts we should distinguish:
\begin{itemize}
  \item the overall energy density $u(T)$ we have just calculated;
  \item the \emph{radiance} $I(T)$, which represents the energy flux per unit area,
    solid angle and time;
  \item the \emph{radiant emittance} $M(T)$, which represents the energy flux crossing
    a given unit area from all possible directions per unit time\sidenote{This is
    a fundamental concept for us, because whenever we have a blackbody (e.g., a star)
    at a given temperature, we get the overall radiated power (energy output per
    unit time) by just multiplying the radiant emittance by the emitting surface.}.
\end{itemize}

Since everything, in our discussion, is isotropic, the radiance is related to the
energy density by the simple relation
\begin{align}
  I(T) = \frac{c}{4\pi} u(T).
\end{align}
(You need a velocity to make the units right, and that has to be the speed of light,
at which the radiation travels, while $4\pi$ is the total solid angle.)

The radiant emittance, on the other hand, is readily calculated by integrating over
the solid angle the component of the radiance orthogonal to a given unit surface
\begin{align*}
  M(T) = \int_\Omega I(T) \cos\theta d\Omega =
  I(T) \int_0^{2\pi} d\phi \int_0^{\frac{\pi}{2}} \cos\theta \sin\theta d\theta = \pi I (T)
\end{align*}
We have therefore the fundamental relation between the energy density and the
radiant emittance
\begin{align}
  M(T) = \frac{c}{4} u(T) =
  \frac{2\pi^5 \kB^4}{15 h^3 c^2} T^4 = \sigma T^4,
\end{align}
which represents the well-known Stefan-Boltzmann law. The constant $\sigma$ is called the
Stefan-Boltzmann constant, and in the cgs system is
\begin{align*}
  \sigma = \frac{2\pi^5 \kB^4}{15 h^3 c^2} =
    5.67 \times 10^{-5}~\text{erg}~\text{cm}^{-2}~\text{s}^{-1}~^\circ~\text{K}^{-4}.
\end{align*}

For completeness, we can put everything together and derive compact and useful
expression for the energy density and pressure of a photon gas at the thermal
equilibrium
\begin{align}
  u(T) = \frac{4\sigma}{c} T^4 \quad\text{and}\quad \pressure(T) = \frac{4\sigma}{3c} T^4.
\end{align}
