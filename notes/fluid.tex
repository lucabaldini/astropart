\chapter{Fluidodynamics}\label{chap:fluidodynamics}

In this chapter we shall quickly review some basic fact from non-relativistic
fluidodynamic. We shall assume that
\begin{align*}
  \vbvelocity(x, y, z, t)
\end{align*}
is the velocity field of the fluid and all the properties of the latter can be
described as (scalar, vector, or tensor) functions of time and position.
(The discussion is largely tailored to cosmic-ray acceleration in shocks, that we
treat in section~\ref{sec:shock_acceleration}.)



\section{Vector calculus}

Since we shall use it extensively in the following, we recall here that for any
scalar function $f(x, y, z, t)$ the total time derivative reads
\begin{align*}
  \dv{f}{t} &= \pdv{f}{t} + \pdv{f}{x} \dv{x}{t} +
  \pdv{f}{y} \dv{y}{t} + \pdv{f}{z} \dv{z}{t} =
  \pdv{f}{t} + \velocity_x \pdv{f}{x} + \velocity_y \pdv{f}{y} + \velocity_z \pdv{f}{z},
\end{align*}
which can be written more succinctly\sidenote{It is obvious from the expansion of
the expression that
\begin{align*}
  (\vbvelocity \cdot \grad) f = \vbvelocity \cdot \grad{f} =
  \grad{f} \cdot \vbvelocity
\end{align*}
but the form with the parenthesis showcases the underlying physical meaning
of the total derivative as an operator, and can be applied to vectors with no
ambiguities.} as
\begin{align}
  \dv{f}{t} = \pdv{f}{t} + \overbrace{(\vbvelocity \cdot \grad)}^{\text{advection}} f
\end{align}
where the part containing the spatial derivatives is generally referred to as the
\emph{advective} term.

We shall also make use of the well-know divergence theorem, that we state here
for a generic vectorial field $\vb{f}$
\begin{align}\label{eq:divergence_theorem}
  \int_S \vb{f} \cdot d\vb{S} = \int_V \vnabla \cdot \vb{f} \dd{V}.
\end{align}



\section{Conservation laws}

Let us consider a non-relativistic fluid described by its mass density
$\density(x, y, z, t)$, function of position and time. We shall express in differential
form the conservation laws for mass, momentum and energy.



\subsection{Mass conservation}\label{sec:nrf_mass_conservation}

The conservation of mass can be expressed by affirming that, in absence of any
source (or sink) term, the change of mass within any given volume $V$ in the fluid
must be compensated by the flux of matter crossing the corresponding closed surface $S$
\begin{align*}
  \int_V \pdv{\density}{t} \dd{V} + \int_S \density \vbvelocity \cdot d\vb{S} = 0.
\end{align*}
By virtue of the divergence theorem this can be rewritten as
\begin{align*}
  \int_V \pdv{\density}{t} \dd{V} + \int_V \vnabla \cdot (\density \vbvelocity) \dd{V} = 0
\end{align*}
and, considering the arbitrariness of the integration volume, it yields the continuity
equation, in differential form, for the conservation of mass
\begin{align}\label{eq:continuity_equation}
  \pdv{\density}{t} + \vnabla \cdot (\density \vbvelocity) = 0.
\end{align}

We emphasize that this very same general structure, where the (partial) time derivative
of a quantity is equal and opposite to the divergence of a vectorial field, is
characteristic of conservation laws for scalar quantities. The conservation of the
electric charge\sidenote{The latter is generally expressed as
\begin{align*}
  \pdv{\density}{t} + \vnabla \cdot \vb{j} = 0,
\end{align*}
and the vectorial field, here, is represented by the electrical current $\vb{j}$.
Note we are overloading for a second the meaning of $\density$ to indicate the
charge density---this will be the only instance throughout these notes.}
is a prime example, and we shall see another one (the conservation of energy) in section~\ref{sec:nrf_energy_conservation}.



\subsection{Momentum conservation}\label{sec:nrf_momentum_conservation}

Since the momentum is a vectorial quantity, the associated conservation law does
not take the form of a continuity equation.
The force excerted on our volume of fluid from the rest of the fluid corresponds
(with the opposite sign) to the integral of the pressure over the surface \sidenote{We
have used the divergence theorem, as you can check for yourself on any cartesian
component, e.g.,
\begin{align*}
  \mathcal{F}_x &= \int_S (\pressure \vb*{i}) \cdot d\vb{S} =
  \int_V \nabla \cdot (\pressure \vb*{i}) \dd{V} =\\
  &= \int_V \left( \vb*{i} \cdot \vnabla \pressure +
  \pressure \cancel{\vnabla \cdot \vb*{i}} \right) \dd{V} =\\
  &= \int_V \pdv{\pressure}{x} \dd V.
\end{align*}
(The term $\vnabla \cdot \vb*{i}$ cancels because the versor $\vb*{i}$ is constant.)}
\begin{align*}
  \mathbfcal{F} = -\int_S \pressure \dd{\vb{S}} = -\int_V \vnabla \pressure \dd{V}.
\end{align*}

Now, in absence of other external forces, and neglecting any viscous effect
the equation of motion reads
\begin{align*}
  \mathbfcal{F} = %\int_V \dv{\vbvelocity}{t} \dd{m} =
  \int_V \density  \dv{\vbvelocity}{t} \dd{V} =
  \int_V \density \left[\pdv{\vbvelocity}{t} +
  (\vbvelocity \cdot \vnabla) \vbvelocity
  \right] \dd{V}.
\end{align*}

Given that the integration volume is, again, arbitrary, this yields what is generally
referred to as the Euler equation for the conservation of momentum in differential
form\sidenote{Note this is an instance where the $\vbvelocity \cdot \grad$ operator
is applied to a vector and needs to be written in parenthesis.}
\begin{align}\label{eq:euler_equation}
  \pdv{\vbvelocity}{t} + (\vbvelocity \cdot \grad) \vbvelocity =
  -\frac{\grad{\pressure}}{\density}.
\end{align}



\subsection{Energy conservation}\label{sec:nrf_energy_conservation}

If we neglect viscosity and thermal conductivity, we can write the total energy per
unit volume in any given volume as the sum of a kinetic and an internal term
\begin{align}
  \varepsilon = \frac{E}{V} = \frac{1}{2}\density \velocity^2 + \frac{U}{V} =
  \frac{1}{2}\density \velocity^2 + \frac{U}{m} \frac{m}{V} =
  \frac{1}{2}\density \velocity^2 + \mathcal{u} \density,
\end{align}
where we have implicitly defined the energy per unit mass $\mathcal{u}$ as
\begin{align*}
  \mathcal{u} = \frac{U}{m}.
\end{align*}
The energy being a scalar, we would like to express the associated conservation
law in a form similar to~\eqref{eq:continuity_equation}
\begin{align*}
  \pdv{\varepsilon}{t} + \vnabla \cdot \mathbfcal{J} = 0,
\end{align*}
where the vector $\mathbfcal{J}$ should be written in terms of the thermodynamical
quantities of the system (e.g., $\pressure$ and $\density$). We might event be tempted
to make a bold guess
\begin{align*}
  \mathbfcal{J} = \varepsilon \vbvelocity =
  \density \vbvelocity \left( \frac{1}{2}\velocity^2 + \mathcal{u} \right)
  \quad \text{(wrong!)}
\end{align*}
but in fact we would not be quite right if we did that\sidenote{This is actually
not too far from the right answer, and if you are not concerned with the gory details
of the calculation, you can move forward at the end of this section: you will see
that all you have to do to make this right is to change the internal energy per
unit mass into the enthalpy per unit mass.}, so let us take a breath and proceed
in a orderly manner---if you are in a rush, you can jump straight to the correct
answer~\eqref{eq:nrf_energy_conservation}.

Before we start, along the same lines of what we did for the internal energy, let
us define the entropy and enthalpy per unit mass:
\begin{align*}
  \mathcal{s} = \frac{S}{m}
  \quad\text{and}\quad
  \mathcal{h} = \frac{H}{m} = \frac{U + \pressure V}{m} =
  \mathcal{u} + \frac{\pressure}{\density}.
\end{align*}
We are now ready to start piecing things together. The (partial) time derivative
of the energy per unit volume reads
\begin{align*}
  \pdv{\varepsilon}{t} =
  \pdv{}{t}\left( \frac{1}{2} \density \velocity^2 + \mathcal{u} \density \right) =
  \frac{1}{2} \velocity^2 \pdv{\density}{t} + \density \vbvelocity \cdot \pdv{\vbvelocity}{t} +
  \density \pdv{\mathcal{u}}{t} + \mathcal{u} \pdv{\density}{t}.
\end{align*}
and all we have to do is to recast the right-end side of the expression as a divergence.

The task is actually less daunting than it might seem at a first glance, as the
conservation laws for mass and momentum do allow to write most of the time derivatives
as spatial gradients. The one bit where it is not completely obvious how to proceed
is the one involving the time derivative of the internal energy per unit mass, so we
shall begin from that. In our setting, where we are operating in terms of intensive
quantities\sidenote{One should note that the differential of the volume in the
$\pressure dV$ term becomes
\begin{align*}
  \pressure dV \rightarrow \pressure d\left(\frac{V}{m}\right) =
  \pressure d\left(\frac{1}{\density}\right) = -\frac{\pressure}{\density^2}d\density.
\end{align*}},
the first law of thermodynamics can be recasted as
\begin{align*}
  d\mathcal{u} - \frac{\pressure}{\density^2}d\density = T d\mathcal{s}
  \quad\text{or}\quad
  \pdv{\mathcal{u}}{t} =
  T \pdv{\mathcal{s}}{t} + \frac{\pressure}{\density^2}\pdv{\density}{t},
\end{align*}
which in turns allows to rewrite the part of our energy balance involving the
internal energy as
\begin{align*}
  \density \pdv{\mathcal{u}}{t} + \mathcal{u} \pdv{\density}{t} &=
  \density T \pdv{\mathcal{s}}{t} + \frac{\pressure}{\density}\pdv{\density}{t} +
  \mathcal{u} \pdv{\density}{t} =
  \density T \pdv{\mathcal{s}}{t} +
  \overbrace{\left( \mathcal{u} + \frac{\pressure}{\density} \right)}^{\text{enthalpy}}
  \pdv{\density}{t} =\\
  & = \density T \pdv{\mathcal{s}}{t} + \mathcal{h} \pdv{\density}{t}.
\end{align*}
(Oh, look what we have here: the enthalpy has appeared, as we did anticipate in
the sidenote a few lines above.)

If we assume adiabaticity, the total time derivative of the entropy vanishes, which
we can use at our advantage in order to write the partial time derivative as a spatial
derivative
\begin{align*}
  \dv{\mathcal{s}}{t} = \pdv{\mathcal{s}}{t} + (\vbvelocity \cdot \grad) \mathcal{s} = 0
  \quad\text{and}\quad
  \density T \pdv{\mathcal{s}}{t} = -\density T (\vbvelocity \cdot \grad) \mathcal{s}.
\end{align*}
At this point we are basically done, except for the fact that we have two non-independent
quantities in the mix (the entropy and the enthalpy), and we would rather express
everything in terms of one or the other. We shall pick the enthalpy, and use again
the first law of thermodynamics
\begin{align*}
  d\mathcal{h} = d\mathcal{u} + \frac{1}{\density}d\pressure -
  \frac{\pressure}{\density^2}d\density =
  Td\mathcal{s} + \frac{1}{\density}d\pressure
  \quad\text{or}\quad
  Td\mathcal{s} = d\mathcal{h} - \frac{1}{\density}d\pressure,
\end{align*}
which, finally, yields\sidenote{If you are not quite sure why this follows from the
previous relation, keep in mind that by $d\mathcal{s}$ we mean \emph{any} differential,
including our convective derivative. In other words:
\begin{align*}
  T(\vbvelocity \cdot \vnabla) \mathcal{s} =
  (\vbvelocity \cdot \vnabla) \mathcal{h} -
  \frac{1}{\density} (\vbvelocity \cdot \vnabla) \pressure
  % Don't ask me why this extra \mathcal is needed here to prevent the enthalpy
  % to appear in bold...
  \mathcal{},
\end{align*}
and since both $\mathcal{s}$ and $\pressure$ are scalars, we can get rid of the
parentheses with no ambiguities. Et voila.
}
\begin{align*}
  \density T \pdv{s}{t} =
  -\density \vbvelocity \cdot \grad{\mathcal{h}} +
  \vbvelocity \cdot \grad{\pressure}
  % Don't ask me why this extra \mathcal is needed here to prevent the enthalpy
  % to appear in bold...
  \mathcal{}.
\end{align*}

We are now ready to put everything together,
\begin{align*}
  \pdv{\varepsilon}{t} =
  %\pdv{t}(\frac{1}{2} \density \velocity^2 + \mathcal{u} \density) =
  \frac{1}{2} \velocity^2 \pdv{\density}{t} + \density \vbvelocity \cdot \pdv{\vbvelocity}{t}
  -\density\vbvelocity \cdot \grad{\mathcal{h}} + \vbvelocity \cdot \grad{\pressure} +
  \mathcal{h} \pdv{\density}{t}.
\end{align*}
Using the the continuity equation and the Euler equation we can rewrite the right-hand
side of the expression as
\begin{align*}
  \pdv{\varepsilon}{t} = &
  -\frac{1}{2}\velocity^2 \vnabla \cdot (\density \vbvelocity)
  - \density \vbvelocity \cdot (\vbvelocity \cdot \grad) \vbvelocity
  - \cancel{\vbvelocity \cdot \grad{\pressure}} + \mathcal{}\\
  & -\density(\vbvelocity \cdot \grad) \mathcal{h}
  + \cancel{\vbvelocity \cdot \grad{\pressure}} +
  - \mathcal{h} \vnabla \cdot (\density \vbvelocity) = \mathcal{}\\
  = & - \left( \frac{1}{2} \velocity^2 + \mathcal{h} \right) \vnabla \cdot (\density \vbvelocity)
  - \density \vbvelocity \cdot \grad(\frac{1}{2} \velocity^2 + \mathcal{h}) = \mathcal{}\\
  = & - \vnabla \cdot \left[
    \density \vbvelocity \left( \frac{1}{2} \velocity^2 + \mathcal{h} \right)
    \right]\mathcal{}.
\end{align*}
And we are done---this is the expression for $\mathbfcal{J}$ we were looking for,
and the full, glorious formula for the energy conservation reads.
\begin{align}\label{eq:nrf_energy_conservation}
  \pdv{t} \left( \frac{1}{2}\density \velocity^2 + \mathcal{u} \density \right) +
  \vnabla \cdot \left[ \density \vbvelocity
  \left( \frac{1}{2} \velocity^2 + \mathcal{u} + \frac{\pressure}{\density} \right)\right] = 0.
\end{align}



\subsection{Conservation laws in one dimension}

Before we move on, it is instructive to write down explicitely the three conservation
laws in the one-dimensional case\sidenote{If you are wondering what is the use for
that, be reassured this is relevant for the discussion of a parallel shock that
we shall do in section~\ref{sec:shock_acceleration}}, i.e, when the velocity field
and all the dynamical variables only depend on a single spatial coordinate (e.g., $x$),
and, possibly, on the time
\begin{align}
  \begin{cases}
  \displaystyle\pdv{\density}{t} + \pdv{x}(\density \velocity) = 0
  & \quad\text{(mass)}\\[5pt]
  \displaystyle\pdv{\velocity}{t} + \velocity \pdv{\velocity}{x} = -\frac{1}{\density} \pdv{\pressure}{x}
  & \quad\text{(momentum)}\\[5pt]
  \displaystyle\pdv{t}(\frac{1}{2} \density \velocity^2 + \mathcal{u} \density) +
  \pdv{x} \left[\density \velocity \left( \frac{1}{2} \velocity^2 + \mathcal{u} +
  \frac{\pressure}{\density} \right) \right] = 0
  & \quad\text{(energy)}
\end{cases}
\end{align}

In the stationary case, when all the time derivatives vanish, these reduce to
the simple expressions\sidenote{If you are unsure about why the momentum conservation
reads this way, note that
\begin{align*}
  \dv{x}(\density \velocity^2) =
  \density \velocity \dv{\velocity}{x} + \cancel{\velocity \dv{x}(\density \velocity)} =
  \density \velocity \dv{\velocity}{x}
\end{align*}
where the striked-through term vanishes in virtue of the continuity equation in
the stationary case.}
\begin{align}\label{eq:conservation_1d_stationary}
  \begin{cases}
  \displaystyle\dv{x}(\density \velocity) = 0
  & \quad\text{(mass)}\\[5pt]
  \displaystyle\dv{x}(\density \velocity^2 + \pressure) = 0
  & \quad\text{(momentum)}\\[5pt]
  \displaystyle\dv{x}
    \left[
    \density \velocity \left( \frac{1}{2} \velocity^2 + \mathcal{u} +
    \frac{\pressure}{\density} \right)
    \right] = 0
  & \quad\text{(energy)}
  \end{cases}
\end{align}
which are readily integrated yielding
\begin{align}
  \begin{cases}
    \density \velocity = \text{const}
    & \quad\text{(mass)}\\
    \density \velocity^2 + \pressure = \text{const}
    & \quad\text{(momentum)}\\
    \density \velocity \left( \frac{1}{2} \velocity^2 +
    \mathcal{u} + \frac{\pressure}{\density} \right) = \text{const}
    & \quad\text{(energy)}.
  \end{cases}
\end{align}


% \section{An application: sound waves in a fluid}\label{sec:fluid_waves}

% Let us consider the implications of the conservation laws for small perturbations
% of a non-relativistic fluid with zero velocity, i.e., let us assume
% that the density and the pressure can be written as
% \begin{align*}
%   \pressure(x, y, z, t) = \pressure_0 + \pressure'(x, y, z, t),
%   \quad\text{and}\quad
%   \density(x, y, z, t) = \density_0 + \density'(x, y, z, t)
% \end{align*}
% where
% \begin{align*}
%   \pressure'(x, y, z, t) \ll \pressure_0.
%   \quad\text{and}\quad
%   \density'(x, y, z, t) \ll \density_0
% \end{align*}

% The continuity equation reads
% \begin{align*}
%   \pdv{\density}{t} + \div(\density \vb{u}) \approx
%   \pdv{\density'}{t} + \div(\density_0 \vb{u}) =
%   \pdv{\density'}{t} + \density_0 \div{\vb{u}} = 0,
% \end{align*}
% while the conservation of momentum\sidenote{The fact that the second term vanishes
% is connected with the zero-velocity hypothesis, but we should be more explicit
% about this.} is
% \begin{align*}
%   \pdv{\vb{u}}{t} + \cancel{(\vb{u} \dotproduct \grad )\vb{u}} =
%   -\frac{1}{\density} \grad{\pressure} \approx
%   -\frac{1}{\density_0} \grad{\pressure'}.
% \end{align*}

% If we assume a generic equation of state $\pressure = \pressure(\density)$, small
% perturbations in the pressure are proportional to small perturbations in the
% density through
% \begin{align*}
%   \pressure'(x, y, z, t) = c_s^2 \density'(x, y, z, t)
%   \quad\text{where}\quad
%   c_s^2 = \eval{\pdv{\pressure}{\density}}_{\density = \density_0}
% \end{align*}
% (we shall see in a second why we have named $c_s^2$ the constant of proportionality).

% We are left with the system of equations
% \begin{equation*}
%   \begin{cases}
%     \displaystyle \frac{1}{c_s^2} \pdv{\pressure'}{t} + \density_0 \div{\vb{u}} = 0\\[8pt]
%     \displaystyle \pdv{\vb{u}}{t} = - \frac{1}{\density_0} \grad{\pressure'},
%   \end{cases}
% \end{equation*}
% to which we can search a solution in the form $\vb{u} = \grad{\varphi}$. We can
% make the substitution in the secon equation, yielding
% \begin{align*}
%   \pdv{\grad{\varphi}}{t} = \grad{\pdv{\varphi}{t}} = -\frac{1}{\density_0} \grad{\pressure'}
%   \quad\text{or}\quad
%   p' = -\density_0 \pdv{\varphi}{t}
% \end{align*}
% Finally, plugging this into the first equation yields
% \begin{align}
%   \frac{1}{c_s^2}\pdv[2]{\varphi}{t} - \laplacian{\varphi} = 0
% \end{align}
% which is a wave equation for sound waves propgating at speed~$c_s$ (which is why
% we called it that way in the first place).

% In case of an adiabatic perturbation of a perfect gas we have $\pressure V^\gamma$
% (where $\gamma$ is the adiabatic index defined in~appendix~\ref{chap:thermodynamics}).
% Since $V \density$ amounts to the total mass of gas (and is therefore rigorously
% constant), and we can write the equation of state as
% \begin{align*}
%   \pressure = k \density^\gamma
%   \quad\text{or}\quad
%   \pdv{\pressure}{\density} = \gamma k \density^{\gamma - 1} = \frac{\gamma\pressure}{\density}
% \end{align*}
% from which it follows that
% \begin{align}
%   c_s = \sqrt{\frac{\gamma\pressure}{\density}} = \sqrt{\frac{\gamma RT}{M}},
% \end{align}
% where $M$ is the molar gas for the gas. For dry air, assuming $\gamma = \nicefrac{7}{5}$
% and $M = 0.02897$~km~m$^-1$, the formula yields $c_s = 343$~m~s$^-1$ at $20^\circ$~C.
