\chapter{Motion in a magnetic field}\label{chap:bmotion}

In this chapter we briefly revise some basic facts about the motion of a charged
particle in a magntic field. This is relevant for several different subjects
covered in this book, such as the rigidity measurement in a magnetic spectrometer,
the geomagnetically trapped radiation and the acceleration of cosmic rays.

The equation of motion in a (purely) magnetic field is of a particle with charge
$q$ is dictated by the expression of the Lorentz force\sidenote{Since the
Lorentz force is orthogonal to the velocity, it classically does not do work, that,
is, it does not change the magnitude of $\vb{v}$, nor the relativistic $\gamma$
factor
\begin{align*}
  \dv{\vb{p}}{t} = \cancel{m \dv{\gamma}{t}\vb{v}} + \gamma m \dv{\vb{v}}{t} =
  \gamma m \dv{\vb{v}}{t}
\end{align*}}, which in the cgs system reads
\begin{align}
  \dv{\vb{p}}{t} = \gamma m \dv{\vb{v}}{t} = \frac{q}{c} (\vb{v} \cross \vb{B}).
\end{align}


\section{Motion in a uniform magnetic field}

Let us consider a uniform and stationary magnetic field $\vb{B}$. In this case the
motion can be decomposed into a part parallel to the magnetic field, whose
solution is trivial
\begin{align*}
  \dv{\vb{p}_\parallel}{t} = 0
  \quad\text{or}\quad
  \vb{p}_\parallel = \text{const},
\end{align*}
and a part orthogonal to the magnetic field
\begin{align*}
  \dv{\vb{p}_\perp}{t} = \frac{q}{c} (\vb{v}_\perp \cross \vb{B}).
\end{align*}
This is also easily integrated noting that the momentum change is always
perpendicular to the velocity, and the latter does only change in direction, not
in magnitude. The solution, when projected onto a plane perpendicular to the
magnetic field, is a circular motion whose radius is dictated by the condition
that the centrifugal force balances the Lorentz force
\begin{align}\label{eq:gyrorad}
  m \gamma \frac{v_\perp^2}{\gyrorad} = \frac{qv_\perp B}{c}
  \quad\text{or}\quad
  \gyrorad = \frac{m\gamma v_\perp c}{qB} = \frac{p_\perp c}{qB} =
  \frac{pc}{q} \frac{\sin\theta}{B}.
\end{align}
This is customarily called the cyclotron radius or  the \emph{gyroradius}, and it
is an important parameter characterizing the motion of a particle into a magnetic
field. The angle $\theta$ between the velocity and the magnetic field is usually
referred to as the \emph{pitch angle}. The quantity
\begin{align}
  \rigidity = \frac{pc}{q}
\end{align}
is called the \emph{rigidity}\sidenote{The rigidity is an important concept because
two particles with the same rigidity behave the same way in a magnetic field, no
matter what the specific values of the momentum and charge are. A magnetic spectrometer,
as we shall see, measures the rigidity.} of the particle, and allows to rewrite the
expression for the gyroradius as
\begin{align*}
  \gyrorad = \rigidity \frac{\sin\theta}{B}.
\end{align*}

As we shall see in a second, the concept of gyroradius is in fact important not
only for the case of a uniform magnetic field, since even in inhomogeneous fields
particles tend to exhibit a characteristic gyromotion, coupled to a drift along the
field lines. The period of this motion is given by
\begin{align}
  T_\gyrorad = \frac{2\pi \gyrorad}{v_\perp} = \frac{2\pi m \gamma c}{qB}, \quad
  \nu_\gyrorad = \frac{qB}{2\pi m \gamma c} \quad\text{and}\quad
  \omega_\gyrorad = \frac{qB}{m \gamma c}.
\end{align}

The motion of a particle in a inhomogeneous magnetic field gets typically complicated
very quickly, and in most cases the only complete solution of the problem is via
a numerical ray-tracing of the particle. Adiabatic invariants, however, that we shall
briefly review in the next section, provide a useful way of understanding the basic
properties of the motion in a simple fashion.


\section{Digression: adiabatic invariants}

It is knwow from classical mechanics that when the Hamiltonian of a one-dimensional,
periodic system depends on one or more parameters that change
\emph{slowly}\sidenote{By slow, here we mean that the fractional change of the
energy of the system in one period is much smaller than unity.} in time, the line
integral
\begin{align}\label{eq:adiabaitic_invariant_1d}
  J = \oint p \diff{q}
\end{align}
is approximately constant. Here $p$ and $q$ are a pair of canonically coniugate
variables, the integral is performed over a closed orbit of the system, and
the action $J$ is called an \emph{adiabatic invariant}. From a geometrical standpoint,
$J$ represents the area of the phase space enclosed in a closed orbit, and it is
easy to see that the line intergral~\eqref{eq:adiabaitic_invariant_1d} can be
expressed in the equivalent form
\begin{align*}
  J = \int\int \diff{p} \diff{q}.
\end{align*}

\begin{marginfigure}
  %% Creator: Matplotlib, PGF backend
%%
%% To include the figure in your LaTeX document, write
%%   \input{<filename>.pgf}
%%
%% Make sure the required packages are loaded in your preamble
%%   \usepackage{pgf}
%%
%% Also ensure that all the required font packages are loaded; for instance,
%% the lmodern package is sometimes necessary when using math font.
%%   \usepackage{lmodern}
%%
%% Figures using additional raster images can only be included by \input if
%% they are in the same directory as the main LaTeX file. For loading figures
%% from other directories you can use the `import` package
%%   \usepackage{import}
%%
%% and then include the figures with
%%   \import{<path to file>}{<filename>.pgf}
%%
%% Matplotlib used the following preamble
%%   \usepackage{fontspec}
%%   \setmainfont{DejaVuSerif.ttf}[Path=\detokenize{/usr/share/matplotlib/mpl-data/fonts/ttf/}]
%%   \setsansfont{DejaVuSans.ttf}[Path=\detokenize{/usr/share/matplotlib/mpl-data/fonts/ttf/}]
%%   \setmonofont{DejaVuSansMono.ttf}[Path=\detokenize{/usr/share/matplotlib/mpl-data/fonts/ttf/}]
%%
\begingroup%
\makeatletter%
\begin{pgfpicture}%
\pgfpathrectangle{\pgfpointorigin}{\pgfqpoint{1.950000in}{1.300000in}}%
\pgfusepath{use as bounding box, clip}%
\begin{pgfscope}%
\pgfsetbuttcap%
\pgfsetmiterjoin%
\definecolor{currentfill}{rgb}{1.000000,1.000000,1.000000}%
\pgfsetfillcolor{currentfill}%
\pgfsetlinewidth{0.000000pt}%
\definecolor{currentstroke}{rgb}{1.000000,1.000000,1.000000}%
\pgfsetstrokecolor{currentstroke}%
\pgfsetdash{}{0pt}%
\pgfpathmoveto{\pgfqpoint{0.000000in}{0.000000in}}%
\pgfpathlineto{\pgfqpoint{1.950000in}{0.000000in}}%
\pgfpathlineto{\pgfqpoint{1.950000in}{1.300000in}}%
\pgfpathlineto{\pgfqpoint{0.000000in}{1.300000in}}%
\pgfpathlineto{\pgfqpoint{0.000000in}{0.000000in}}%
\pgfpathclose%
\pgfusepath{fill}%
\end{pgfscope}%
\begin{pgfscope}%
\pgfpathrectangle{\pgfqpoint{0.000000in}{0.000000in}}{\pgfqpoint{1.950000in}{1.300000in}}%
\pgfusepath{clip}%
\pgfsetbuttcap%
\pgfsetmiterjoin%
\definecolor{currentfill}{rgb}{1.000000,1.000000,1.000000}%
\pgfsetfillcolor{currentfill}%
\pgfsetlinewidth{1.003750pt}%
\definecolor{currentstroke}{rgb}{0.000000,0.000000,0.000000}%
\pgfsetstrokecolor{currentstroke}%
\pgfsetdash{}{0pt}%
\pgfpathmoveto{\pgfqpoint{0.975000in}{0.216667in}}%
\pgfpathcurveto{\pgfqpoint{1.181858in}{0.216667in}}{\pgfqpoint{1.380272in}{0.262325in}}{\pgfqpoint{1.526543in}{0.343587in}}%
\pgfpathcurveto{\pgfqpoint{1.672814in}{0.424849in}}{\pgfqpoint{1.755000in}{0.535079in}}{\pgfqpoint{1.755000in}{0.650000in}}%
\pgfpathcurveto{\pgfqpoint{1.755000in}{0.764921in}}{\pgfqpoint{1.672814in}{0.875151in}}{\pgfqpoint{1.526543in}{0.956413in}}%
\pgfpathcurveto{\pgfqpoint{1.380272in}{1.037675in}}{\pgfqpoint{1.181858in}{1.083333in}}{\pgfqpoint{0.975000in}{1.083333in}}%
\pgfpathcurveto{\pgfqpoint{0.768142in}{1.083333in}}{\pgfqpoint{0.569728in}{1.037675in}}{\pgfqpoint{0.423457in}{0.956413in}}%
\pgfpathcurveto{\pgfqpoint{0.277186in}{0.875151in}}{\pgfqpoint{0.195000in}{0.764921in}}{\pgfqpoint{0.195000in}{0.650000in}}%
\pgfpathcurveto{\pgfqpoint{0.195000in}{0.535079in}}{\pgfqpoint{0.277186in}{0.424849in}}{\pgfqpoint{0.423457in}{0.343587in}}%
\pgfpathcurveto{\pgfqpoint{0.569728in}{0.262325in}}{\pgfqpoint{0.768142in}{0.216667in}}{\pgfqpoint{0.975000in}{0.216667in}}%
\pgfpathlineto{\pgfqpoint{0.975000in}{0.216667in}}%
\pgfpathclose%
\pgfusepath{stroke,fill}%
\end{pgfscope}%
\begin{pgfscope}%
\pgfsetroundcap%
\pgfsetroundjoin%
\pgfsetlinewidth{1.003750pt}%
\definecolor{currentstroke}{rgb}{0.000000,0.000000,0.000000}%
\pgfsetstrokecolor{currentstroke}%
\pgfsetdash{}{0pt}%
\pgfpathmoveto{\pgfqpoint{0.027761in}{0.650000in}}%
\pgfpathquadraticcurveto{\pgfqpoint{0.974992in}{0.650000in}}{\pgfqpoint{1.906695in}{0.650000in}}%
\pgfusepath{stroke}%
\end{pgfscope}%
\begin{pgfscope}%
\pgfsetroundcap%
\pgfsetroundjoin%
\pgfsetlinewidth{1.003750pt}%
\definecolor{currentstroke}{rgb}{0.000000,0.000000,0.000000}%
\pgfsetstrokecolor{currentstroke}%
\pgfsetdash{}{0pt}%
\pgfpathmoveto{\pgfqpoint{1.856695in}{0.675000in}}%
\pgfpathlineto{\pgfqpoint{1.906695in}{0.650000in}}%
\pgfpathlineto{\pgfqpoint{1.856695in}{0.625000in}}%
\pgfusepath{stroke}%
\end{pgfscope}%
\begin{pgfscope}%
\definecolor{textcolor}{rgb}{0.000000,0.000000,0.000000}%
\pgfsetstrokecolor{textcolor}%
\pgfsetfillcolor{textcolor}%
\pgftext[x=1.852500in,y=0.585000in,,top]{\color{textcolor}\rmfamily\fontsize{9.000000}{10.800000}\selectfont q}%
\end{pgfscope}%
\begin{pgfscope}%
\pgfsetroundcap%
\pgfsetroundjoin%
\pgfsetlinewidth{1.003750pt}%
\definecolor{currentstroke}{rgb}{0.000000,0.000000,0.000000}%
\pgfsetstrokecolor{currentstroke}%
\pgfsetdash{}{0pt}%
\pgfpathmoveto{\pgfqpoint{0.975000in}{0.027811in}}%
\pgfpathquadraticcurveto{\pgfqpoint{0.975000in}{0.650026in}}{\pgfqpoint{0.975000in}{1.256712in}}%
\pgfusepath{stroke}%
\end{pgfscope}%
\begin{pgfscope}%
\pgfsetroundcap%
\pgfsetroundjoin%
\pgfsetlinewidth{1.003750pt}%
\definecolor{currentstroke}{rgb}{0.000000,0.000000,0.000000}%
\pgfsetstrokecolor{currentstroke}%
\pgfsetdash{}{0pt}%
\pgfpathmoveto{\pgfqpoint{0.950000in}{1.206712in}}%
\pgfpathlineto{\pgfqpoint{0.975000in}{1.256712in}}%
\pgfpathlineto{\pgfqpoint{1.000000in}{1.206712in}}%
\pgfusepath{stroke}%
\end{pgfscope}%
\begin{pgfscope}%
\definecolor{textcolor}{rgb}{0.000000,0.000000,0.000000}%
\pgfsetstrokecolor{textcolor}%
\pgfsetfillcolor{textcolor}%
\pgftext[x=0.916500in,y=1.191667in,right,]{\color{textcolor}\rmfamily\fontsize{9.000000}{10.800000}\selectfont p}%
\end{pgfscope}%
\begin{pgfscope}%
\definecolor{textcolor}{rgb}{0.000000,0.000000,0.000000}%
\pgfsetstrokecolor{textcolor}%
\pgfsetfillcolor{textcolor}%
\pgftext[x=0.936000in,y=0.866667in,right,]{\color{textcolor}\rmfamily\fontsize{9.000000}{10.800000}\selectfont \(\displaystyle \sqrt{2mE}\)}%
\end{pgfscope}%
\begin{pgfscope}%
\definecolor{textcolor}{rgb}{0.000000,0.000000,0.000000}%
\pgfsetstrokecolor{textcolor}%
\pgfsetfillcolor{textcolor}%
\pgftext[x=1.306500in,y=0.606667in,,top]{\color{textcolor}\rmfamily\fontsize{9.000000}{10.800000}\selectfont \(\displaystyle \sqrt{2E/m\omega^2}\)}%
\end{pgfscope}%
\end{pgfpicture}%
\makeatother%
\endgroup%

  \caption{Trajectory of the periodic motion of a harmonic oscillator in the phse
  space.}
  \label{fig:adiabatic_invariant}
\end{marginfigure}

This is best illustrated by a harmonic oscillator whose angular frequency $\omega$
changes slowly with time. The Hamiltonian reads
\begin{align*}
  H = \frac{p^2}{2m} + \frac{m\omega^2 q^2}{2}
\end{align*}
and, for any given value of the energy $E$ of the system, the trajectory in the
phase space is an ellipse with semi-axes $\sqrt{\nicefrac{2E}{m\omega^2}}$ and
$\sqrt{2mE}$, and the area of the ellipse is
\begin{align*}
  J = 2\pi \frac{E}{\omega},
\end{align*}
that is, when $\omega$ changes adiabatically, the energy $E$ changes of the same
fractional amount.

We emphasize that it is sometimes the case that a system has more than one periodicity,
and in that case at each periodicity corresponds a distinct adiabatic invariant.


\section{Motion in inhomogeneous magnetic fields}

In plasma physics it is customary to define two invariants for the motion of a
charged particle in a magnetic field: the first is related to the the gyrantion
motion around the field lines, and the second is related to the periodic bouncing
between magnetic mirroring points, when they exist. This also applies to geomagnetically
trapped radiation, where a third periodic motion (a slow east-west drift about the Earth),
and therefore a third invariant can be defined. Since the trajectory of a particle
in a inhomogeneous magnetic field can be arbitrarily complicated, adiabatic invariants
are extremely useful to characterize in simple terms the properties of a system.

For illustrative purpose we shall calculate explicitly the first plasma adiabatic
invariant $J_1$ deriving from the gyration motion around the field lines, which
will be approximately constant for a magnetic field whose fractional variations
in time and space are small within a single period of the motion and across a gyroradius.
Since the canonical momentum of a charged particle moving in a magnetic field with
vector potential $\vb{A}$ is $\vb{p} + \nicefrac{q}{c} \vb{A}$, we can write the
adiabatic invariant as
\begin{align}
  J_1 = \oint \qty[\vb{p} + \frac{q}{c} \vb{A}] \dotproduct d\vb{l} =
  2\pi\gyrorad p_\perp + \frac{q}{c} \oint \vb{A} \dotproduct d\vb{l}.
\end{align}
The last line integral can be transformed into a flux across the surface $S$ enclosed
by the orbit by virtue of the Stokes' theorem
\begin{align*}
  \oint \vb{A} \dotproduct d\vb{l} = \int_S \curl{\vb{A}} \dotproduct d\vb{S} =
  \int_S \vb{B} \dotproduct d\vb{S} = -\pi \gyrorad^2 B,
\end{align*}
where $B$ is the absolute value of the magnetic field, assumed to be constant within
a single orbit\sidenote{The negative sign is due to the fact that the vector $d\vb{S}$
defined by the orbit of the particle is antiparallel to $B$. This is a general
principle related to Lenz's law.}.

Now, the two terms in our invariant contains both $p_\perp$ and $\gyrorad$, which
are not independent, as either one can be written in terms of the other using
equation~\eqref{eq:gyrorad}---by doing which we realize that the second is in fact
exactly a half of the first (with the opposite sign), and the sum reads
\begin{align}
  J_1 = \frac{\pi p_\perp^2 c}{qB} = \frac{\pi \gyrorad^2 qB}{c}.
\end{align}
We note, in passing, that the expression for $J_1$ as a function of the transverse
momentum makes it immediately obvious that the first adiabatic invariant is
proportional to the quantity
\begin{align*}
  J_1 \propto \mu = \frac{p_\perp^2}{2mB}
\end{align*}
which is customarily called the \emph{magnetic moment} of the motion, since it is
easy to show that it reduces to the actual magnetic moment of the current loop
induced by the particle gyromotion in the non-relativistic limit\sidenote{Here
$i = \nicefrac{q}{T\gyrorad}$ denotes the current associated to the moving change,
and $S = \pi \gyrorad^2$ is the surface of the loop. Note that, in the expression
for the gyroradius we have set $\gamma = 1$ since we are in the non-relativistics case.}
\begin{align*}
  \mu_\text{NR} = \frac{iS}{c} = \frac{q \pi \gyrorad^2}{T_\gyrorad c} =
  \frac{m v_\perp^2 }{2 B} = \frac{\kine_\perp}{B}
\end{align*}

It is also worth mentioning that the conservation of the magnetic moment in the
non relativistic case can be demonstrated directly for a uniform magnetic field
slowly varying in time. The time derivative of $\mu_\text{NR}$ reads
\begin{align}\label{eq:nr_mu_tder}
  \dv{\mu_\text{NR}}{t} =
  \frac{1}{B}\pdv{\kine_\perp}{t} - \frac{\kine_\perp}{B^2}\pdv{B}{t},
\end{align}
which includes two separate terms---one due to the (slow) change in kinetic energy
(and hence transverse velocity, and hence gyroradius) and the other due to the (slow)
change of magnetic field. As it turns out, the two are not independent, as the variation
of kinetic energy is in turn due to the electromotive force
\begin{align*}
  \curl\vb{E} = -\frac{1}{c}\pdv{\vb{B}}{t}.
\end{align*}
induced by the time variation of the magnetic field. In order to show that the
magnetic moment is constant in time, we can therefore calculate the change in
transverse kinetic energy in a full revolution as
\begin{align*}
  \Delta \kine = \oint q\vb{E} \dotproduct d\vb{l} =
  q \int_S \curl \vb{E} \dotproduct d\vb{S} =
  - \frac{q}{c} \int_S\pdv{\vb{B}}{t} \dotproduct d\vb{S} =
  \frac{q\pi \gyrorad^2}{c} \pdv{B}{t}
\end{align*}
(note that, as usual, the magnetic field and the surface element vector are antiparallel,
hence the additional minus sign in the last passage) and since the rate of variation
is constant through a period, the time derivative of the transverse kinetic energy as
\begin{align*}
  \pdv{\kine_\perp}{t} = \frac{\Delta \kine}{T_\gyrorad} =
  \frac{v_\perp}{2\pi\gyrorad} \frac{q\pi \gyrorad^2}{c} \pdv{B}{t} =
  \frac{v_\perp q \gyrorad}{2c}\pdv{B}{t} = \frac{\kine_\perp}{B}\pdv{B}{t}.
\end{align*}
By plugging this equation into~\eqref{eq:nr_mu_tder} we see that the time derivative
of $\mu_\text{NR}$ vanishes, as expected.

Either way, we have the fundamental result that the gyromotion in a slowly varying
magnetic field proceed with an approximately conserved quantities that can
be written as
\begin{align}
  \frac{p_\perp^2}{B} = \text{const}
  \quad\text{or}\quad
  \gyrorad^2 B = \text{const}.
\end{align}

This is a fundamental results as it immediately implies that, whenever a particle
is moving from a region with a weaker magnetic field toward a region with a
higher magnetic field (that is, $B$ increases), the transverse momentum is bound
to increase as well ($\propto \sqrt{B}$, as a matter of fact) so that our first
adiabatic invariant is preserved. Since the magnetic field is not doing work and
the kinetic energy is conserved, we have that
\begin{align*}
  p_\perp^2 +  p_\parallel^2 = \text{const}
\end{align*}
and the component of the momentum parallel to the magnetic field is bound to
decrease---\emph{charged particles feel a net repulsive force when moving in
the direction of a magnetic field gradient} at the point that, under suitable
conditions, the parallel velocity can get to zero and change sign, and a magnetic
field gradient can act as a magnetic mirror. This is the very principle at the basis
of plasma confinement, it is one of the fundamental mechenism at the base of the
geomagnetically trapped radiation, and is one of the ingredients of the Fermi
first order acceleration mechanism.

The parallel with the magnetic moment is actually more profound that we have
sketched here, as a particle undergoing gyration motion acts like a small loop
of current, and the component of the foce parallel to the magnetic field is
essentially that of a inhomogeneous magnetic field acting on a magnetic moment
\begin{align}
  F_\parallel = -\frac{\mu}{\gamma c}\pdv{B}{s}
\end{align}
Since the motion of the particle (and hence the current) is induced by the field
itself, the force is always repulsive.
