\chapter{Software and data}
\label{chap:software}


Since this writeup is intended for people to do useful stuff with its content,
we provide here a (biased and non comprehensive) list of open-source projects and
data sources that are related to the content of the document. The vast majority
of the figures in the writeup have been crafted using one or more of the packages
and sources described in this appendix.


\section{Python packages}

Since Python is by far the most popular scripting language for scientific applications,
we provide a list of useful Python packages intendend as easy-access resources
to the relevant data.


\subsection{\texttt{crdb} (\href{https://github.com/crdb-project/crdb}{code},
  \href{https://crdb.readthedocs.io/en/latest/}{docs})}

Python interface to the cosmic-ray database, compiling cosmic-ray data and meta-data
from $10^6$ to $10^{21}$~eV, for protons, nuclei and leptons.


\subsection{\texttt{pdg} (\href{https://github.com/particledatagroup/api}{code},
  \href{https://pdgapi.lbl.gov/doc/pythonapi.html}{docs})}

The PDG Python API provides a high-level tool for programmatically accessing PDG
data. This mainly includes particle data, such as particle properties and branching
fractions.

Atomic and nuclear properties of materials, such as stopping power for muons
between 1~MeV and 1~PeV, radiation and interaction lengths, critical energies, and
much more are available in machine-readable form from
\url{https://pdg.lbl.gov/2023/AtomicNuclearProperties/index.html}.


\subsection{\texttt{xcom} and \texttt{star} (\href{https://github.com/Zelenyy/nist-calculators}{code})}

Python interfaces to the NIST XCOM and STAR databases. The former provides
photon cross section data in elements and compounds from 1~keV to 100~GeV, while
the latter includes stopping-power and range tables for electrons, protons and He ions
between 1~keV and 10~GeV.


\subsection{\texttt{xraydb} (\href{https://github.com/xraypy/XrayDB}{code},
  \href{https://xraypy.github.io/XrayDB/}{docs})}

XrayDB provides atomic data, characteristic X-ray energies, and X-ray cross sections
(taylored for energues between 250~eV and 250~keV) for the elements.



%%% Candidates for addition

% astropy.cosmo
% pint https://pint.readthedocs.io/en/stable/ (units)
% uncertainties 
