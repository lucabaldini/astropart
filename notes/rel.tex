\chapter{Special relativity}\label{chap:relativity}


\section{Four-momentum}

Relativistic kinematics is naturally formulated in terms of the
four-momentum\sidenote{Since we shall note use the four-vector notation a lot,
and in the spirit of avoiding confusion, we shall use the funny-looking
calligraphic~$\mathcal{p}$ to indicate it, leaving the old, plain $p$ for the
norm of the three-dimensional momentum
\begin{align*}
  p = \abs{\vb{p}} = \gamma \beta mc.
\end{align*}}
\begin{align}
  \mathcal{p} = \qty(\frac{E}{c}, \vb{p}),
\end{align}
where
\begin{align*}
  E = \gamma m c^2, \quad
  \vb{p} = \gamma m \vb{v} \quad\text{and}\quad
  p = \gamma\beta m c.
\end{align*}
Here and in the following, for completeness, the relativistic $\beta$ and $\gamma$
are defined as\sidenote{The definition for $\gamma$ can be inverted, yielding
\begin{align*}
  \beta = \sqrt{1 - \frac{1}{\gamma^2}} = \sqrt{\frac{\gamma^2 - 1}{\gamma^2}},
\end{align*}
which is sometimes useful.}
\begin{align*}
  \beta = \frac{v}{c}
  \quad\text{and}\quad
  \gamma = \frac{1}{\sqrt{1 - \beta^2}}.
\end{align*}

It is trivial to show that norm squared of the four momentum is a Lorentz-invariant
quantity---more specifically, the proper mass of the particle squared. If we choose the
$(1, -1, -1, -1)$ metric signature, e.g., we have
\begin{align*}
  \mathcal{p}^2 = \frac{E^2}{c^2} - \abs{\vb{p}}^2 =
  m^2 \gamma^2 c^2 - m^2 \gamma^2 \beta^2 c^2 =
  m^2 \gamma^2 c^2 (1 - \beta^2) = m^2c^2,
\end{align*}
which represents the usual dispersion relation for a generic particle of mass $m$
\begin{align}
  m^2 c^4 = E^2 - p^2c^2.
\end{align}
The kinetic energy reads
\begin{align}
  E_k = E - mc^2.
\end{align}

With some trivial algebra one can derive some useful relations for $\gamma$ and
$\beta$, once $E$ and $p$ are known:
\begin{align*}
  \gamma = \frac{E}{mc^2}, \quad
  \beta = \frac{pc}{E} \quad\text{and}\quad
  \gamma\beta = \frac{p}{mc}.
\end{align*}


\section{Lorentz tranformations}

The Lorentz transformations for a generic contravariant four-vector can
be conveniently written in matrix form as:
\begin{equation}
{a'}^\mu = \Lambda^\mu_\nu a^\nu
\end{equation}
where the explicit form of $\Lambda^\mu_\nu$ is, in the particular case of
a boost in the direction of the $x$-axis with velocity $\beta c$, is:
\begin{equation}
  \Lambda^\mu_\nu =
  \begin{pmatrix}
    \gamma & -\beta\gamma & 0 & 0\\
    -\beta\gamma & \gamma & 0 & 0\\
    0 & 0 & 1 & 0\\
    0 & 0 & 0 & 1\\
  \end{pmatrix}
\end{equation}

The relativistic formula for the addition of velocities in the simple case of
collinear motion reads
\begin{equation}
u' = \frac{u + v}{1 + uv/c^2}.
\end{equation}


\section{Four momentum: lab and CM frames}

As a useful application of the Lorentz transformation in special relativity, we
shall calculate the energy available in the center of mass for a collision
between a particle of mass $m$ and energy $E$ approaching an identical particle
at rest (in a given reference system).

\begin{marginfigure}
  \begin{tikzpicture}
    \node at (0, 0) {$\left(\frac{E}{c},~\vb{p}\right)$};
    \node at (0, -0.5) [circle,fill,inner sep=1.5pt]{};
    \node at (-1, -0.5) {Lab};
    \draw[arrows = {-Computer Modern Rightarrow[line cap=butt]}] (0, -0.5)--(1, -0.5);
    \node at (2, 0) {$\left(mc,~\vb{0}\right)$};
    \node at (2, -0.5) [circle,fill,inner sep=1.5pt]{};
    \node at (0, -2.25) {$\left(\frac{E_\text{CM}}{2c},~\vb{p}_\text{CM}\right)$};
    \node at (0, -1.5) [circle,fill,inner sep=1.5pt]{};
    \draw[arrows = {-Computer Modern Rightarrow[line cap=butt]}] (0, -1.5)--(0.75, -1.5);
    \node at (2, -2.25) {$\left(\frac{E_\text{CM}}{2c},~-\vb{p}_\text{CM}\right)$};
    \node at (2, -1.5) [circle,fill,inner sep=1.5pt]{};
    \node at (-1, -1.5) {CM};
    \draw[arrows = {-Computer Modern Rightarrow[line cap=butt]}] (2, -1.5)--(1.25, -1.5);
  \end{tikzpicture}
  \caption{Cinematic of the collision between two identical particles of mass $m$
    in the reference frames of the laboratory and of the center of mass. (Note
    $E_\text{CM}$ is the \emph{total} energy in the center of mass system.)}
  \label{fig:relativistic_collision}
\end{marginfigure}

The four momentum conservation in the lab and the center-of-mass reference
systems, as shown in figure~\ref{fig:relativistic_collision}, reads
\begin{align*}
  \overbrace{\left(\frac{E}{c} + mc, \vb{p}\right)^2}^\text{Lab} =
  \overbrace{\left(\frac{E_\text{CM}}{c}, \vb{0}\right)^2}^\text{CM},
\end{align*}
(here $E_\text{CM}$ indicates the \emph{total} energy in the center of mass).
Taking the modulo squared of both sides yields
\begin{align*}
  E^2 + m^2c^4 + 2Emc^2 - p^2c^2 = E_\text{CM}^2,
\end{align*}
and, noting that $E^2 - p^2c^2 = m^2$, we have
\begin{align*}
  2m^2c^4 + 2Emc^2 = E_\text{CM}^2.
\end{align*}
In the ultra-relativistic limit, where $E \gg mc^2$, this reads
\begin{align}
  E = \frac{E_\text{CM}^2}{2mc^2}
  \quad\text{or}\quad
  E_\text{CM} = \sqrt{2Emc^2}.
\end{align}

A proton-proton collision at LHC, with a total energy $E_\text{CM} = 13.6$~TeV in the
center-of-mass reference frame, is kinematically equivalent to a collision
of a $9.3 \times 10^{16}$~eV proton (right above the knee in the cosmic-ray spectrum)
with another proton at rest.
