\chapter{Passage of particles through matter}

This is a fairly succinct recap of the basic facts concering the passage of
particles through matter. The reader will find more in depth information
in~\cite{PDG} and reference therein. We emphasize that the importance of the
material covered in this section is pervasive within this writeup, as it is
equally important in context as different as the acceleration of cosmic rays,
their propagation and their detection.


\section{Charged particles: energy losses}

Energy losses of charged particles are customarily discussed separately for
heavy\sidenote{In this context by \emph{heavy} we really mean anything much
heavier than an electron.} (e.g., protons, alpha particles and nuclei) and light
(e.g. electrons) particles---the reason behind that being that radiation losses
are, generally speaking, negligible for the the former. Given that, in principle, all
particles---be they heavy or light---suffer both ionization and radiation
losses, here we take a slightly different approach, focusing our discussion,
at the top level, on the distinction between these two types of losses.


\subsection{Ionization losses}

Relativistic charged heavy particles passing through matter loose energy by
ionization at an average rate that is reasonably well described by the Bethe
equation over most of energies we are interested. Numerical factor aside, this
reads
\begin{align}\label{eq:stopping_power}
  -\ave{\frac{dE}{dx}}_\text{ion} \!\!\!\!\!\!
  \propto z^2 \frac{Z}{A}\frac{1}{\beta^2}
  \left[ \frac{1}{2} \ln(C_0 \beta^2\gamma^2) - \beta^2 +
    \text{corrections} \right],
\end{align}
where $Z$ and $A$ refer to the target material and $z$, $\beta$ and $\gamma$
refer to the projectile. (The readed can refer to~\cite{PDG} for a detailed
description of all the terms in the equation).

\begin{marginfigure}
  %% Creator: Matplotlib, PGF backend
%%
%% To include the figure in your LaTeX document, write
%%   \input{<filename>.pgf}
%%
%% Make sure the required packages are loaded in your preamble
%%   \usepackage{pgf}
%%
%% Also ensure that all the required font packages are loaded; for instance,
%% the lmodern package is sometimes necessary when using math font.
%%   \usepackage{lmodern}
%%
%% Figures using additional raster images can only be included by \input if
%% they are in the same directory as the main LaTeX file. For loading figures
%% from other directories you can use the `import` package
%%   \usepackage{import}
%%
%% and then include the figures with
%%   \import{<path to file>}{<filename>.pgf}
%%
%% Matplotlib used the following preamble
%%   \usepackage{fontspec}
%%   \setmainfont{DejaVuSerif.ttf}[Path=\detokenize{/usr/share/matplotlib/mpl-data/fonts/ttf/}]
%%   \setsansfont{DejaVuSans.ttf}[Path=\detokenize{/usr/share/matplotlib/mpl-data/fonts/ttf/}]
%%   \setmonofont{DejaVuSansMono.ttf}[Path=\detokenize{/usr/share/matplotlib/mpl-data/fonts/ttf/}]
%%
\begingroup%
\makeatletter%
\begin{pgfpicture}%
\pgfpathrectangle{\pgfpointorigin}{\pgfqpoint{1.950000in}{2.750000in}}%
\pgfusepath{use as bounding box, clip}%
\begin{pgfscope}%
\pgfsetbuttcap%
\pgfsetmiterjoin%
\definecolor{currentfill}{rgb}{1.000000,1.000000,1.000000}%
\pgfsetfillcolor{currentfill}%
\pgfsetlinewidth{0.000000pt}%
\definecolor{currentstroke}{rgb}{1.000000,1.000000,1.000000}%
\pgfsetstrokecolor{currentstroke}%
\pgfsetdash{}{0pt}%
\pgfpathmoveto{\pgfqpoint{0.000000in}{0.000000in}}%
\pgfpathlineto{\pgfqpoint{1.950000in}{0.000000in}}%
\pgfpathlineto{\pgfqpoint{1.950000in}{2.750000in}}%
\pgfpathlineto{\pgfqpoint{0.000000in}{2.750000in}}%
\pgfpathlineto{\pgfqpoint{0.000000in}{0.000000in}}%
\pgfpathclose%
\pgfusepath{fill}%
\end{pgfscope}%
\begin{pgfscope}%
\pgfsetbuttcap%
\pgfsetmiterjoin%
\definecolor{currentfill}{rgb}{1.000000,1.000000,1.000000}%
\pgfsetfillcolor{currentfill}%
\pgfsetlinewidth{0.000000pt}%
\definecolor{currentstroke}{rgb}{0.000000,0.000000,0.000000}%
\pgfsetstrokecolor{currentstroke}%
\pgfsetstrokeopacity{0.000000}%
\pgfsetdash{}{0pt}%
\pgfpathmoveto{\pgfqpoint{0.536250in}{0.525000in}}%
\pgfpathlineto{\pgfqpoint{1.846250in}{0.525000in}}%
\pgfpathlineto{\pgfqpoint{1.846250in}{2.268750in}}%
\pgfpathlineto{\pgfqpoint{0.536250in}{2.268750in}}%
\pgfpathlineto{\pgfqpoint{0.536250in}{0.525000in}}%
\pgfpathclose%
\pgfusepath{fill}%
\end{pgfscope}%
\begin{pgfscope}%
\pgfpathrectangle{\pgfqpoint{0.536250in}{0.525000in}}{\pgfqpoint{1.310000in}{1.743750in}}%
\pgfusepath{clip}%
\pgfsetbuttcap%
\pgfsetroundjoin%
\pgfsetlinewidth{0.803000pt}%
\definecolor{currentstroke}{rgb}{0.752941,0.752941,0.752941}%
\pgfsetstrokecolor{currentstroke}%
\pgfsetdash{{2.960000pt}{1.280000pt}}{0.000000pt}%
\pgfpathmoveto{\pgfqpoint{0.700000in}{0.525000in}}%
\pgfpathlineto{\pgfqpoint{0.700000in}{2.268750in}}%
\pgfusepath{stroke}%
\end{pgfscope}%
\begin{pgfscope}%
\pgfsetbuttcap%
\pgfsetroundjoin%
\definecolor{currentfill}{rgb}{0.000000,0.000000,0.000000}%
\pgfsetfillcolor{currentfill}%
\pgfsetlinewidth{0.803000pt}%
\definecolor{currentstroke}{rgb}{0.000000,0.000000,0.000000}%
\pgfsetstrokecolor{currentstroke}%
\pgfsetdash{}{0pt}%
\pgfsys@defobject{currentmarker}{\pgfqpoint{0.000000in}{-0.048611in}}{\pgfqpoint{0.000000in}{0.000000in}}{%
\pgfpathmoveto{\pgfqpoint{0.000000in}{0.000000in}}%
\pgfpathlineto{\pgfqpoint{0.000000in}{-0.048611in}}%
\pgfusepath{stroke,fill}%
}%
\begin{pgfscope}%
\pgfsys@transformshift{0.700000in}{0.525000in}%
\pgfsys@useobject{currentmarker}{}%
\end{pgfscope}%
\end{pgfscope}%
\begin{pgfscope}%
\definecolor{textcolor}{rgb}{0.000000,0.000000,0.000000}%
\pgfsetstrokecolor{textcolor}%
\pgfsetfillcolor{textcolor}%
\pgftext[x=0.700000in,y=0.427778in,,top]{\color{textcolor}\rmfamily\fontsize{9.000000}{10.800000}\selectfont \(\displaystyle {10^{2}}\)}%
\end{pgfscope}%
\begin{pgfscope}%
\pgfpathrectangle{\pgfqpoint{0.536250in}{0.525000in}}{\pgfqpoint{1.310000in}{1.743750in}}%
\pgfusepath{clip}%
\pgfsetbuttcap%
\pgfsetroundjoin%
\pgfsetlinewidth{0.803000pt}%
\definecolor{currentstroke}{rgb}{0.752941,0.752941,0.752941}%
\pgfsetstrokecolor{currentstroke}%
\pgfsetdash{{2.960000pt}{1.280000pt}}{0.000000pt}%
\pgfpathmoveto{\pgfqpoint{1.027500in}{0.525000in}}%
\pgfpathlineto{\pgfqpoint{1.027500in}{2.268750in}}%
\pgfusepath{stroke}%
\end{pgfscope}%
\begin{pgfscope}%
\pgfsetbuttcap%
\pgfsetroundjoin%
\definecolor{currentfill}{rgb}{0.000000,0.000000,0.000000}%
\pgfsetfillcolor{currentfill}%
\pgfsetlinewidth{0.803000pt}%
\definecolor{currentstroke}{rgb}{0.000000,0.000000,0.000000}%
\pgfsetstrokecolor{currentstroke}%
\pgfsetdash{}{0pt}%
\pgfsys@defobject{currentmarker}{\pgfqpoint{0.000000in}{-0.048611in}}{\pgfqpoint{0.000000in}{0.000000in}}{%
\pgfpathmoveto{\pgfqpoint{0.000000in}{0.000000in}}%
\pgfpathlineto{\pgfqpoint{0.000000in}{-0.048611in}}%
\pgfusepath{stroke,fill}%
}%
\begin{pgfscope}%
\pgfsys@transformshift{1.027500in}{0.525000in}%
\pgfsys@useobject{currentmarker}{}%
\end{pgfscope}%
\end{pgfscope}%
\begin{pgfscope}%
\definecolor{textcolor}{rgb}{0.000000,0.000000,0.000000}%
\pgfsetstrokecolor{textcolor}%
\pgfsetfillcolor{textcolor}%
\pgftext[x=1.027500in,y=0.427778in,,top]{\color{textcolor}\rmfamily\fontsize{9.000000}{10.800000}\selectfont \(\displaystyle {10^{4}}\)}%
\end{pgfscope}%
\begin{pgfscope}%
\pgfpathrectangle{\pgfqpoint{0.536250in}{0.525000in}}{\pgfqpoint{1.310000in}{1.743750in}}%
\pgfusepath{clip}%
\pgfsetbuttcap%
\pgfsetroundjoin%
\pgfsetlinewidth{0.803000pt}%
\definecolor{currentstroke}{rgb}{0.752941,0.752941,0.752941}%
\pgfsetstrokecolor{currentstroke}%
\pgfsetdash{{2.960000pt}{1.280000pt}}{0.000000pt}%
\pgfpathmoveto{\pgfqpoint{1.355000in}{0.525000in}}%
\pgfpathlineto{\pgfqpoint{1.355000in}{2.268750in}}%
\pgfusepath{stroke}%
\end{pgfscope}%
\begin{pgfscope}%
\pgfsetbuttcap%
\pgfsetroundjoin%
\definecolor{currentfill}{rgb}{0.000000,0.000000,0.000000}%
\pgfsetfillcolor{currentfill}%
\pgfsetlinewidth{0.803000pt}%
\definecolor{currentstroke}{rgb}{0.000000,0.000000,0.000000}%
\pgfsetstrokecolor{currentstroke}%
\pgfsetdash{}{0pt}%
\pgfsys@defobject{currentmarker}{\pgfqpoint{0.000000in}{-0.048611in}}{\pgfqpoint{0.000000in}{0.000000in}}{%
\pgfpathmoveto{\pgfqpoint{0.000000in}{0.000000in}}%
\pgfpathlineto{\pgfqpoint{0.000000in}{-0.048611in}}%
\pgfusepath{stroke,fill}%
}%
\begin{pgfscope}%
\pgfsys@transformshift{1.355000in}{0.525000in}%
\pgfsys@useobject{currentmarker}{}%
\end{pgfscope}%
\end{pgfscope}%
\begin{pgfscope}%
\definecolor{textcolor}{rgb}{0.000000,0.000000,0.000000}%
\pgfsetstrokecolor{textcolor}%
\pgfsetfillcolor{textcolor}%
\pgftext[x=1.355000in,y=0.427778in,,top]{\color{textcolor}\rmfamily\fontsize{9.000000}{10.800000}\selectfont \(\displaystyle {10^{6}}\)}%
\end{pgfscope}%
\begin{pgfscope}%
\pgfpathrectangle{\pgfqpoint{0.536250in}{0.525000in}}{\pgfqpoint{1.310000in}{1.743750in}}%
\pgfusepath{clip}%
\pgfsetbuttcap%
\pgfsetroundjoin%
\pgfsetlinewidth{0.803000pt}%
\definecolor{currentstroke}{rgb}{0.752941,0.752941,0.752941}%
\pgfsetstrokecolor{currentstroke}%
\pgfsetdash{{2.960000pt}{1.280000pt}}{0.000000pt}%
\pgfpathmoveto{\pgfqpoint{1.682500in}{0.525000in}}%
\pgfpathlineto{\pgfqpoint{1.682500in}{2.268750in}}%
\pgfusepath{stroke}%
\end{pgfscope}%
\begin{pgfscope}%
\pgfsetbuttcap%
\pgfsetroundjoin%
\definecolor{currentfill}{rgb}{0.000000,0.000000,0.000000}%
\pgfsetfillcolor{currentfill}%
\pgfsetlinewidth{0.803000pt}%
\definecolor{currentstroke}{rgb}{0.000000,0.000000,0.000000}%
\pgfsetstrokecolor{currentstroke}%
\pgfsetdash{}{0pt}%
\pgfsys@defobject{currentmarker}{\pgfqpoint{0.000000in}{-0.048611in}}{\pgfqpoint{0.000000in}{0.000000in}}{%
\pgfpathmoveto{\pgfqpoint{0.000000in}{0.000000in}}%
\pgfpathlineto{\pgfqpoint{0.000000in}{-0.048611in}}%
\pgfusepath{stroke,fill}%
}%
\begin{pgfscope}%
\pgfsys@transformshift{1.682500in}{0.525000in}%
\pgfsys@useobject{currentmarker}{}%
\end{pgfscope}%
\end{pgfscope}%
\begin{pgfscope}%
\definecolor{textcolor}{rgb}{0.000000,0.000000,0.000000}%
\pgfsetstrokecolor{textcolor}%
\pgfsetfillcolor{textcolor}%
\pgftext[x=1.682500in,y=0.427778in,,top]{\color{textcolor}\rmfamily\fontsize{9.000000}{10.800000}\selectfont \(\displaystyle {10^{8}}\)}%
\end{pgfscope}%
\begin{pgfscope}%
\definecolor{textcolor}{rgb}{0.000000,0.000000,0.000000}%
\pgfsetstrokecolor{textcolor}%
\pgfsetfillcolor{textcolor}%
\pgftext[x=1.191250in,y=0.251251in,,top]{\color{textcolor}\rmfamily\fontsize{9.000000}{10.800000}\selectfont Momentum [MeV/c]}%
\end{pgfscope}%
\begin{pgfscope}%
\pgfpathrectangle{\pgfqpoint{0.536250in}{0.525000in}}{\pgfqpoint{1.310000in}{1.743750in}}%
\pgfusepath{clip}%
\pgfsetbuttcap%
\pgfsetroundjoin%
\pgfsetlinewidth{0.803000pt}%
\definecolor{currentstroke}{rgb}{0.752941,0.752941,0.752941}%
\pgfsetstrokecolor{currentstroke}%
\pgfsetdash{{2.960000pt}{1.280000pt}}{0.000000pt}%
\pgfpathmoveto{\pgfqpoint{0.536250in}{0.525000in}}%
\pgfpathlineto{\pgfqpoint{1.846250in}{0.525000in}}%
\pgfusepath{stroke}%
\end{pgfscope}%
\begin{pgfscope}%
\pgfsetbuttcap%
\pgfsetroundjoin%
\definecolor{currentfill}{rgb}{0.000000,0.000000,0.000000}%
\pgfsetfillcolor{currentfill}%
\pgfsetlinewidth{0.803000pt}%
\definecolor{currentstroke}{rgb}{0.000000,0.000000,0.000000}%
\pgfsetstrokecolor{currentstroke}%
\pgfsetdash{}{0pt}%
\pgfsys@defobject{currentmarker}{\pgfqpoint{-0.048611in}{0.000000in}}{\pgfqpoint{-0.000000in}{0.000000in}}{%
\pgfpathmoveto{\pgfqpoint{-0.000000in}{0.000000in}}%
\pgfpathlineto{\pgfqpoint{-0.048611in}{0.000000in}}%
\pgfusepath{stroke,fill}%
}%
\begin{pgfscope}%
\pgfsys@transformshift{0.536250in}{0.525000in}%
\pgfsys@useobject{currentmarker}{}%
\end{pgfscope}%
\end{pgfscope}%
\begin{pgfscope}%
\definecolor{textcolor}{rgb}{0.000000,0.000000,0.000000}%
\pgfsetstrokecolor{textcolor}%
\pgfsetfillcolor{textcolor}%
\pgftext[x=0.252687in, y=0.477515in, left, base]{\color{textcolor}\rmfamily\fontsize{9.000000}{10.800000}\selectfont \(\displaystyle {10^{0}}\)}%
\end{pgfscope}%
\begin{pgfscope}%
\pgfpathrectangle{\pgfqpoint{0.536250in}{0.525000in}}{\pgfqpoint{1.310000in}{1.743750in}}%
\pgfusepath{clip}%
\pgfsetbuttcap%
\pgfsetroundjoin%
\pgfsetlinewidth{0.803000pt}%
\definecolor{currentstroke}{rgb}{0.752941,0.752941,0.752941}%
\pgfsetstrokecolor{currentstroke}%
\pgfsetdash{{2.960000pt}{1.280000pt}}{0.000000pt}%
\pgfpathmoveto{\pgfqpoint{0.536250in}{1.396875in}}%
\pgfpathlineto{\pgfqpoint{1.846250in}{1.396875in}}%
\pgfusepath{stroke}%
\end{pgfscope}%
\begin{pgfscope}%
\pgfsetbuttcap%
\pgfsetroundjoin%
\definecolor{currentfill}{rgb}{0.000000,0.000000,0.000000}%
\pgfsetfillcolor{currentfill}%
\pgfsetlinewidth{0.803000pt}%
\definecolor{currentstroke}{rgb}{0.000000,0.000000,0.000000}%
\pgfsetstrokecolor{currentstroke}%
\pgfsetdash{}{0pt}%
\pgfsys@defobject{currentmarker}{\pgfqpoint{-0.048611in}{0.000000in}}{\pgfqpoint{-0.000000in}{0.000000in}}{%
\pgfpathmoveto{\pgfqpoint{-0.000000in}{0.000000in}}%
\pgfpathlineto{\pgfqpoint{-0.048611in}{0.000000in}}%
\pgfusepath{stroke,fill}%
}%
\begin{pgfscope}%
\pgfsys@transformshift{0.536250in}{1.396875in}%
\pgfsys@useobject{currentmarker}{}%
\end{pgfscope}%
\end{pgfscope}%
\begin{pgfscope}%
\definecolor{textcolor}{rgb}{0.000000,0.000000,0.000000}%
\pgfsetstrokecolor{textcolor}%
\pgfsetfillcolor{textcolor}%
\pgftext[x=0.252687in, y=1.349390in, left, base]{\color{textcolor}\rmfamily\fontsize{9.000000}{10.800000}\selectfont \(\displaystyle {10^{1}}\)}%
\end{pgfscope}%
\begin{pgfscope}%
\pgfpathrectangle{\pgfqpoint{0.536250in}{0.525000in}}{\pgfqpoint{1.310000in}{1.743750in}}%
\pgfusepath{clip}%
\pgfsetbuttcap%
\pgfsetroundjoin%
\pgfsetlinewidth{0.803000pt}%
\definecolor{currentstroke}{rgb}{0.752941,0.752941,0.752941}%
\pgfsetstrokecolor{currentstroke}%
\pgfsetdash{{2.960000pt}{1.280000pt}}{0.000000pt}%
\pgfpathmoveto{\pgfqpoint{0.536250in}{2.268750in}}%
\pgfpathlineto{\pgfqpoint{1.846250in}{2.268750in}}%
\pgfusepath{stroke}%
\end{pgfscope}%
\begin{pgfscope}%
\pgfsetbuttcap%
\pgfsetroundjoin%
\definecolor{currentfill}{rgb}{0.000000,0.000000,0.000000}%
\pgfsetfillcolor{currentfill}%
\pgfsetlinewidth{0.803000pt}%
\definecolor{currentstroke}{rgb}{0.000000,0.000000,0.000000}%
\pgfsetstrokecolor{currentstroke}%
\pgfsetdash{}{0pt}%
\pgfsys@defobject{currentmarker}{\pgfqpoint{-0.048611in}{0.000000in}}{\pgfqpoint{-0.000000in}{0.000000in}}{%
\pgfpathmoveto{\pgfqpoint{-0.000000in}{0.000000in}}%
\pgfpathlineto{\pgfqpoint{-0.048611in}{0.000000in}}%
\pgfusepath{stroke,fill}%
}%
\begin{pgfscope}%
\pgfsys@transformshift{0.536250in}{2.268750in}%
\pgfsys@useobject{currentmarker}{}%
\end{pgfscope}%
\end{pgfscope}%
\begin{pgfscope}%
\definecolor{textcolor}{rgb}{0.000000,0.000000,0.000000}%
\pgfsetstrokecolor{textcolor}%
\pgfsetfillcolor{textcolor}%
\pgftext[x=0.252687in, y=2.221265in, left, base]{\color{textcolor}\rmfamily\fontsize{9.000000}{10.800000}\selectfont \(\displaystyle {10^{2}}\)}%
\end{pgfscope}%
\begin{pgfscope}%
\pgfpathrectangle{\pgfqpoint{0.536250in}{0.525000in}}{\pgfqpoint{1.310000in}{1.743750in}}%
\pgfusepath{clip}%
\pgfsetbuttcap%
\pgfsetroundjoin%
\pgfsetlinewidth{0.803000pt}%
\definecolor{currentstroke}{rgb}{0.752941,0.752941,0.752941}%
\pgfsetstrokecolor{currentstroke}%
\pgfsetdash{{2.960000pt}{1.280000pt}}{0.000000pt}%
\pgfpathmoveto{\pgfqpoint{0.536250in}{0.787461in}}%
\pgfpathlineto{\pgfqpoint{1.846250in}{0.787461in}}%
\pgfusepath{stroke}%
\end{pgfscope}%
\begin{pgfscope}%
\pgfsetbuttcap%
\pgfsetroundjoin%
\definecolor{currentfill}{rgb}{0.000000,0.000000,0.000000}%
\pgfsetfillcolor{currentfill}%
\pgfsetlinewidth{0.602250pt}%
\definecolor{currentstroke}{rgb}{0.000000,0.000000,0.000000}%
\pgfsetstrokecolor{currentstroke}%
\pgfsetdash{}{0pt}%
\pgfsys@defobject{currentmarker}{\pgfqpoint{-0.027778in}{0.000000in}}{\pgfqpoint{-0.000000in}{0.000000in}}{%
\pgfpathmoveto{\pgfqpoint{-0.000000in}{0.000000in}}%
\pgfpathlineto{\pgfqpoint{-0.027778in}{0.000000in}}%
\pgfusepath{stroke,fill}%
}%
\begin{pgfscope}%
\pgfsys@transformshift{0.536250in}{0.787461in}%
\pgfsys@useobject{currentmarker}{}%
\end{pgfscope}%
\end{pgfscope}%
\begin{pgfscope}%
\pgfpathrectangle{\pgfqpoint{0.536250in}{0.525000in}}{\pgfqpoint{1.310000in}{1.743750in}}%
\pgfusepath{clip}%
\pgfsetbuttcap%
\pgfsetroundjoin%
\pgfsetlinewidth{0.803000pt}%
\definecolor{currentstroke}{rgb}{0.752941,0.752941,0.752941}%
\pgfsetstrokecolor{currentstroke}%
\pgfsetdash{{2.960000pt}{1.280000pt}}{0.000000pt}%
\pgfpathmoveto{\pgfqpoint{0.536250in}{0.940990in}}%
\pgfpathlineto{\pgfqpoint{1.846250in}{0.940990in}}%
\pgfusepath{stroke}%
\end{pgfscope}%
\begin{pgfscope}%
\pgfsetbuttcap%
\pgfsetroundjoin%
\definecolor{currentfill}{rgb}{0.000000,0.000000,0.000000}%
\pgfsetfillcolor{currentfill}%
\pgfsetlinewidth{0.602250pt}%
\definecolor{currentstroke}{rgb}{0.000000,0.000000,0.000000}%
\pgfsetstrokecolor{currentstroke}%
\pgfsetdash{}{0pt}%
\pgfsys@defobject{currentmarker}{\pgfqpoint{-0.027778in}{0.000000in}}{\pgfqpoint{-0.000000in}{0.000000in}}{%
\pgfpathmoveto{\pgfqpoint{-0.000000in}{0.000000in}}%
\pgfpathlineto{\pgfqpoint{-0.027778in}{0.000000in}}%
\pgfusepath{stroke,fill}%
}%
\begin{pgfscope}%
\pgfsys@transformshift{0.536250in}{0.940990in}%
\pgfsys@useobject{currentmarker}{}%
\end{pgfscope}%
\end{pgfscope}%
\begin{pgfscope}%
\pgfpathrectangle{\pgfqpoint{0.536250in}{0.525000in}}{\pgfqpoint{1.310000in}{1.743750in}}%
\pgfusepath{clip}%
\pgfsetbuttcap%
\pgfsetroundjoin%
\pgfsetlinewidth{0.803000pt}%
\definecolor{currentstroke}{rgb}{0.752941,0.752941,0.752941}%
\pgfsetstrokecolor{currentstroke}%
\pgfsetdash{{2.960000pt}{1.280000pt}}{0.000000pt}%
\pgfpathmoveto{\pgfqpoint{0.536250in}{1.049921in}}%
\pgfpathlineto{\pgfqpoint{1.846250in}{1.049921in}}%
\pgfusepath{stroke}%
\end{pgfscope}%
\begin{pgfscope}%
\pgfsetbuttcap%
\pgfsetroundjoin%
\definecolor{currentfill}{rgb}{0.000000,0.000000,0.000000}%
\pgfsetfillcolor{currentfill}%
\pgfsetlinewidth{0.602250pt}%
\definecolor{currentstroke}{rgb}{0.000000,0.000000,0.000000}%
\pgfsetstrokecolor{currentstroke}%
\pgfsetdash{}{0pt}%
\pgfsys@defobject{currentmarker}{\pgfqpoint{-0.027778in}{0.000000in}}{\pgfqpoint{-0.000000in}{0.000000in}}{%
\pgfpathmoveto{\pgfqpoint{-0.000000in}{0.000000in}}%
\pgfpathlineto{\pgfqpoint{-0.027778in}{0.000000in}}%
\pgfusepath{stroke,fill}%
}%
\begin{pgfscope}%
\pgfsys@transformshift{0.536250in}{1.049921in}%
\pgfsys@useobject{currentmarker}{}%
\end{pgfscope}%
\end{pgfscope}%
\begin{pgfscope}%
\pgfpathrectangle{\pgfqpoint{0.536250in}{0.525000in}}{\pgfqpoint{1.310000in}{1.743750in}}%
\pgfusepath{clip}%
\pgfsetbuttcap%
\pgfsetroundjoin%
\pgfsetlinewidth{0.803000pt}%
\definecolor{currentstroke}{rgb}{0.752941,0.752941,0.752941}%
\pgfsetstrokecolor{currentstroke}%
\pgfsetdash{{2.960000pt}{1.280000pt}}{0.000000pt}%
\pgfpathmoveto{\pgfqpoint{0.536250in}{1.134414in}}%
\pgfpathlineto{\pgfqpoint{1.846250in}{1.134414in}}%
\pgfusepath{stroke}%
\end{pgfscope}%
\begin{pgfscope}%
\pgfsetbuttcap%
\pgfsetroundjoin%
\definecolor{currentfill}{rgb}{0.000000,0.000000,0.000000}%
\pgfsetfillcolor{currentfill}%
\pgfsetlinewidth{0.602250pt}%
\definecolor{currentstroke}{rgb}{0.000000,0.000000,0.000000}%
\pgfsetstrokecolor{currentstroke}%
\pgfsetdash{}{0pt}%
\pgfsys@defobject{currentmarker}{\pgfqpoint{-0.027778in}{0.000000in}}{\pgfqpoint{-0.000000in}{0.000000in}}{%
\pgfpathmoveto{\pgfqpoint{-0.000000in}{0.000000in}}%
\pgfpathlineto{\pgfqpoint{-0.027778in}{0.000000in}}%
\pgfusepath{stroke,fill}%
}%
\begin{pgfscope}%
\pgfsys@transformshift{0.536250in}{1.134414in}%
\pgfsys@useobject{currentmarker}{}%
\end{pgfscope}%
\end{pgfscope}%
\begin{pgfscope}%
\pgfpathrectangle{\pgfqpoint{0.536250in}{0.525000in}}{\pgfqpoint{1.310000in}{1.743750in}}%
\pgfusepath{clip}%
\pgfsetbuttcap%
\pgfsetroundjoin%
\pgfsetlinewidth{0.803000pt}%
\definecolor{currentstroke}{rgb}{0.752941,0.752941,0.752941}%
\pgfsetstrokecolor{currentstroke}%
\pgfsetdash{{2.960000pt}{1.280000pt}}{0.000000pt}%
\pgfpathmoveto{\pgfqpoint{0.536250in}{1.203451in}}%
\pgfpathlineto{\pgfqpoint{1.846250in}{1.203451in}}%
\pgfusepath{stroke}%
\end{pgfscope}%
\begin{pgfscope}%
\pgfsetbuttcap%
\pgfsetroundjoin%
\definecolor{currentfill}{rgb}{0.000000,0.000000,0.000000}%
\pgfsetfillcolor{currentfill}%
\pgfsetlinewidth{0.602250pt}%
\definecolor{currentstroke}{rgb}{0.000000,0.000000,0.000000}%
\pgfsetstrokecolor{currentstroke}%
\pgfsetdash{}{0pt}%
\pgfsys@defobject{currentmarker}{\pgfqpoint{-0.027778in}{0.000000in}}{\pgfqpoint{-0.000000in}{0.000000in}}{%
\pgfpathmoveto{\pgfqpoint{-0.000000in}{0.000000in}}%
\pgfpathlineto{\pgfqpoint{-0.027778in}{0.000000in}}%
\pgfusepath{stroke,fill}%
}%
\begin{pgfscope}%
\pgfsys@transformshift{0.536250in}{1.203451in}%
\pgfsys@useobject{currentmarker}{}%
\end{pgfscope}%
\end{pgfscope}%
\begin{pgfscope}%
\pgfpathrectangle{\pgfqpoint{0.536250in}{0.525000in}}{\pgfqpoint{1.310000in}{1.743750in}}%
\pgfusepath{clip}%
\pgfsetbuttcap%
\pgfsetroundjoin%
\pgfsetlinewidth{0.803000pt}%
\definecolor{currentstroke}{rgb}{0.752941,0.752941,0.752941}%
\pgfsetstrokecolor{currentstroke}%
\pgfsetdash{{2.960000pt}{1.280000pt}}{0.000000pt}%
\pgfpathmoveto{\pgfqpoint{0.536250in}{1.261820in}}%
\pgfpathlineto{\pgfqpoint{1.846250in}{1.261820in}}%
\pgfusepath{stroke}%
\end{pgfscope}%
\begin{pgfscope}%
\pgfsetbuttcap%
\pgfsetroundjoin%
\definecolor{currentfill}{rgb}{0.000000,0.000000,0.000000}%
\pgfsetfillcolor{currentfill}%
\pgfsetlinewidth{0.602250pt}%
\definecolor{currentstroke}{rgb}{0.000000,0.000000,0.000000}%
\pgfsetstrokecolor{currentstroke}%
\pgfsetdash{}{0pt}%
\pgfsys@defobject{currentmarker}{\pgfqpoint{-0.027778in}{0.000000in}}{\pgfqpoint{-0.000000in}{0.000000in}}{%
\pgfpathmoveto{\pgfqpoint{-0.000000in}{0.000000in}}%
\pgfpathlineto{\pgfqpoint{-0.027778in}{0.000000in}}%
\pgfusepath{stroke,fill}%
}%
\begin{pgfscope}%
\pgfsys@transformshift{0.536250in}{1.261820in}%
\pgfsys@useobject{currentmarker}{}%
\end{pgfscope}%
\end{pgfscope}%
\begin{pgfscope}%
\pgfpathrectangle{\pgfqpoint{0.536250in}{0.525000in}}{\pgfqpoint{1.310000in}{1.743750in}}%
\pgfusepath{clip}%
\pgfsetbuttcap%
\pgfsetroundjoin%
\pgfsetlinewidth{0.803000pt}%
\definecolor{currentstroke}{rgb}{0.752941,0.752941,0.752941}%
\pgfsetstrokecolor{currentstroke}%
\pgfsetdash{{2.960000pt}{1.280000pt}}{0.000000pt}%
\pgfpathmoveto{\pgfqpoint{0.536250in}{1.312382in}}%
\pgfpathlineto{\pgfqpoint{1.846250in}{1.312382in}}%
\pgfusepath{stroke}%
\end{pgfscope}%
\begin{pgfscope}%
\pgfsetbuttcap%
\pgfsetroundjoin%
\definecolor{currentfill}{rgb}{0.000000,0.000000,0.000000}%
\pgfsetfillcolor{currentfill}%
\pgfsetlinewidth{0.602250pt}%
\definecolor{currentstroke}{rgb}{0.000000,0.000000,0.000000}%
\pgfsetstrokecolor{currentstroke}%
\pgfsetdash{}{0pt}%
\pgfsys@defobject{currentmarker}{\pgfqpoint{-0.027778in}{0.000000in}}{\pgfqpoint{-0.000000in}{0.000000in}}{%
\pgfpathmoveto{\pgfqpoint{-0.000000in}{0.000000in}}%
\pgfpathlineto{\pgfqpoint{-0.027778in}{0.000000in}}%
\pgfusepath{stroke,fill}%
}%
\begin{pgfscope}%
\pgfsys@transformshift{0.536250in}{1.312382in}%
\pgfsys@useobject{currentmarker}{}%
\end{pgfscope}%
\end{pgfscope}%
\begin{pgfscope}%
\pgfpathrectangle{\pgfqpoint{0.536250in}{0.525000in}}{\pgfqpoint{1.310000in}{1.743750in}}%
\pgfusepath{clip}%
\pgfsetbuttcap%
\pgfsetroundjoin%
\pgfsetlinewidth{0.803000pt}%
\definecolor{currentstroke}{rgb}{0.752941,0.752941,0.752941}%
\pgfsetstrokecolor{currentstroke}%
\pgfsetdash{{2.960000pt}{1.280000pt}}{0.000000pt}%
\pgfpathmoveto{\pgfqpoint{0.536250in}{1.356980in}}%
\pgfpathlineto{\pgfqpoint{1.846250in}{1.356980in}}%
\pgfusepath{stroke}%
\end{pgfscope}%
\begin{pgfscope}%
\pgfsetbuttcap%
\pgfsetroundjoin%
\definecolor{currentfill}{rgb}{0.000000,0.000000,0.000000}%
\pgfsetfillcolor{currentfill}%
\pgfsetlinewidth{0.602250pt}%
\definecolor{currentstroke}{rgb}{0.000000,0.000000,0.000000}%
\pgfsetstrokecolor{currentstroke}%
\pgfsetdash{}{0pt}%
\pgfsys@defobject{currentmarker}{\pgfqpoint{-0.027778in}{0.000000in}}{\pgfqpoint{-0.000000in}{0.000000in}}{%
\pgfpathmoveto{\pgfqpoint{-0.000000in}{0.000000in}}%
\pgfpathlineto{\pgfqpoint{-0.027778in}{0.000000in}}%
\pgfusepath{stroke,fill}%
}%
\begin{pgfscope}%
\pgfsys@transformshift{0.536250in}{1.356980in}%
\pgfsys@useobject{currentmarker}{}%
\end{pgfscope}%
\end{pgfscope}%
\begin{pgfscope}%
\pgfpathrectangle{\pgfqpoint{0.536250in}{0.525000in}}{\pgfqpoint{1.310000in}{1.743750in}}%
\pgfusepath{clip}%
\pgfsetbuttcap%
\pgfsetroundjoin%
\pgfsetlinewidth{0.803000pt}%
\definecolor{currentstroke}{rgb}{0.752941,0.752941,0.752941}%
\pgfsetstrokecolor{currentstroke}%
\pgfsetdash{{2.960000pt}{1.280000pt}}{0.000000pt}%
\pgfpathmoveto{\pgfqpoint{0.536250in}{1.659336in}}%
\pgfpathlineto{\pgfqpoint{1.846250in}{1.659336in}}%
\pgfusepath{stroke}%
\end{pgfscope}%
\begin{pgfscope}%
\pgfsetbuttcap%
\pgfsetroundjoin%
\definecolor{currentfill}{rgb}{0.000000,0.000000,0.000000}%
\pgfsetfillcolor{currentfill}%
\pgfsetlinewidth{0.602250pt}%
\definecolor{currentstroke}{rgb}{0.000000,0.000000,0.000000}%
\pgfsetstrokecolor{currentstroke}%
\pgfsetdash{}{0pt}%
\pgfsys@defobject{currentmarker}{\pgfqpoint{-0.027778in}{0.000000in}}{\pgfqpoint{-0.000000in}{0.000000in}}{%
\pgfpathmoveto{\pgfqpoint{-0.000000in}{0.000000in}}%
\pgfpathlineto{\pgfqpoint{-0.027778in}{0.000000in}}%
\pgfusepath{stroke,fill}%
}%
\begin{pgfscope}%
\pgfsys@transformshift{0.536250in}{1.659336in}%
\pgfsys@useobject{currentmarker}{}%
\end{pgfscope}%
\end{pgfscope}%
\begin{pgfscope}%
\pgfpathrectangle{\pgfqpoint{0.536250in}{0.525000in}}{\pgfqpoint{1.310000in}{1.743750in}}%
\pgfusepath{clip}%
\pgfsetbuttcap%
\pgfsetroundjoin%
\pgfsetlinewidth{0.803000pt}%
\definecolor{currentstroke}{rgb}{0.752941,0.752941,0.752941}%
\pgfsetstrokecolor{currentstroke}%
\pgfsetdash{{2.960000pt}{1.280000pt}}{0.000000pt}%
\pgfpathmoveto{\pgfqpoint{0.536250in}{1.812865in}}%
\pgfpathlineto{\pgfqpoint{1.846250in}{1.812865in}}%
\pgfusepath{stroke}%
\end{pgfscope}%
\begin{pgfscope}%
\pgfsetbuttcap%
\pgfsetroundjoin%
\definecolor{currentfill}{rgb}{0.000000,0.000000,0.000000}%
\pgfsetfillcolor{currentfill}%
\pgfsetlinewidth{0.602250pt}%
\definecolor{currentstroke}{rgb}{0.000000,0.000000,0.000000}%
\pgfsetstrokecolor{currentstroke}%
\pgfsetdash{}{0pt}%
\pgfsys@defobject{currentmarker}{\pgfqpoint{-0.027778in}{0.000000in}}{\pgfqpoint{-0.000000in}{0.000000in}}{%
\pgfpathmoveto{\pgfqpoint{-0.000000in}{0.000000in}}%
\pgfpathlineto{\pgfqpoint{-0.027778in}{0.000000in}}%
\pgfusepath{stroke,fill}%
}%
\begin{pgfscope}%
\pgfsys@transformshift{0.536250in}{1.812865in}%
\pgfsys@useobject{currentmarker}{}%
\end{pgfscope}%
\end{pgfscope}%
\begin{pgfscope}%
\pgfpathrectangle{\pgfqpoint{0.536250in}{0.525000in}}{\pgfqpoint{1.310000in}{1.743750in}}%
\pgfusepath{clip}%
\pgfsetbuttcap%
\pgfsetroundjoin%
\pgfsetlinewidth{0.803000pt}%
\definecolor{currentstroke}{rgb}{0.752941,0.752941,0.752941}%
\pgfsetstrokecolor{currentstroke}%
\pgfsetdash{{2.960000pt}{1.280000pt}}{0.000000pt}%
\pgfpathmoveto{\pgfqpoint{0.536250in}{1.921796in}}%
\pgfpathlineto{\pgfqpoint{1.846250in}{1.921796in}}%
\pgfusepath{stroke}%
\end{pgfscope}%
\begin{pgfscope}%
\pgfsetbuttcap%
\pgfsetroundjoin%
\definecolor{currentfill}{rgb}{0.000000,0.000000,0.000000}%
\pgfsetfillcolor{currentfill}%
\pgfsetlinewidth{0.602250pt}%
\definecolor{currentstroke}{rgb}{0.000000,0.000000,0.000000}%
\pgfsetstrokecolor{currentstroke}%
\pgfsetdash{}{0pt}%
\pgfsys@defobject{currentmarker}{\pgfqpoint{-0.027778in}{0.000000in}}{\pgfqpoint{-0.000000in}{0.000000in}}{%
\pgfpathmoveto{\pgfqpoint{-0.000000in}{0.000000in}}%
\pgfpathlineto{\pgfqpoint{-0.027778in}{0.000000in}}%
\pgfusepath{stroke,fill}%
}%
\begin{pgfscope}%
\pgfsys@transformshift{0.536250in}{1.921796in}%
\pgfsys@useobject{currentmarker}{}%
\end{pgfscope}%
\end{pgfscope}%
\begin{pgfscope}%
\pgfpathrectangle{\pgfqpoint{0.536250in}{0.525000in}}{\pgfqpoint{1.310000in}{1.743750in}}%
\pgfusepath{clip}%
\pgfsetbuttcap%
\pgfsetroundjoin%
\pgfsetlinewidth{0.803000pt}%
\definecolor{currentstroke}{rgb}{0.752941,0.752941,0.752941}%
\pgfsetstrokecolor{currentstroke}%
\pgfsetdash{{2.960000pt}{1.280000pt}}{0.000000pt}%
\pgfpathmoveto{\pgfqpoint{0.536250in}{2.006289in}}%
\pgfpathlineto{\pgfqpoint{1.846250in}{2.006289in}}%
\pgfusepath{stroke}%
\end{pgfscope}%
\begin{pgfscope}%
\pgfsetbuttcap%
\pgfsetroundjoin%
\definecolor{currentfill}{rgb}{0.000000,0.000000,0.000000}%
\pgfsetfillcolor{currentfill}%
\pgfsetlinewidth{0.602250pt}%
\definecolor{currentstroke}{rgb}{0.000000,0.000000,0.000000}%
\pgfsetstrokecolor{currentstroke}%
\pgfsetdash{}{0pt}%
\pgfsys@defobject{currentmarker}{\pgfqpoint{-0.027778in}{0.000000in}}{\pgfqpoint{-0.000000in}{0.000000in}}{%
\pgfpathmoveto{\pgfqpoint{-0.000000in}{0.000000in}}%
\pgfpathlineto{\pgfqpoint{-0.027778in}{0.000000in}}%
\pgfusepath{stroke,fill}%
}%
\begin{pgfscope}%
\pgfsys@transformshift{0.536250in}{2.006289in}%
\pgfsys@useobject{currentmarker}{}%
\end{pgfscope}%
\end{pgfscope}%
\begin{pgfscope}%
\pgfpathrectangle{\pgfqpoint{0.536250in}{0.525000in}}{\pgfqpoint{1.310000in}{1.743750in}}%
\pgfusepath{clip}%
\pgfsetbuttcap%
\pgfsetroundjoin%
\pgfsetlinewidth{0.803000pt}%
\definecolor{currentstroke}{rgb}{0.752941,0.752941,0.752941}%
\pgfsetstrokecolor{currentstroke}%
\pgfsetdash{{2.960000pt}{1.280000pt}}{0.000000pt}%
\pgfpathmoveto{\pgfqpoint{0.536250in}{2.075326in}}%
\pgfpathlineto{\pgfqpoint{1.846250in}{2.075326in}}%
\pgfusepath{stroke}%
\end{pgfscope}%
\begin{pgfscope}%
\pgfsetbuttcap%
\pgfsetroundjoin%
\definecolor{currentfill}{rgb}{0.000000,0.000000,0.000000}%
\pgfsetfillcolor{currentfill}%
\pgfsetlinewidth{0.602250pt}%
\definecolor{currentstroke}{rgb}{0.000000,0.000000,0.000000}%
\pgfsetstrokecolor{currentstroke}%
\pgfsetdash{}{0pt}%
\pgfsys@defobject{currentmarker}{\pgfqpoint{-0.027778in}{0.000000in}}{\pgfqpoint{-0.000000in}{0.000000in}}{%
\pgfpathmoveto{\pgfqpoint{-0.000000in}{0.000000in}}%
\pgfpathlineto{\pgfqpoint{-0.027778in}{0.000000in}}%
\pgfusepath{stroke,fill}%
}%
\begin{pgfscope}%
\pgfsys@transformshift{0.536250in}{2.075326in}%
\pgfsys@useobject{currentmarker}{}%
\end{pgfscope}%
\end{pgfscope}%
\begin{pgfscope}%
\pgfpathrectangle{\pgfqpoint{0.536250in}{0.525000in}}{\pgfqpoint{1.310000in}{1.743750in}}%
\pgfusepath{clip}%
\pgfsetbuttcap%
\pgfsetroundjoin%
\pgfsetlinewidth{0.803000pt}%
\definecolor{currentstroke}{rgb}{0.752941,0.752941,0.752941}%
\pgfsetstrokecolor{currentstroke}%
\pgfsetdash{{2.960000pt}{1.280000pt}}{0.000000pt}%
\pgfpathmoveto{\pgfqpoint{0.536250in}{2.133695in}}%
\pgfpathlineto{\pgfqpoint{1.846250in}{2.133695in}}%
\pgfusepath{stroke}%
\end{pgfscope}%
\begin{pgfscope}%
\pgfsetbuttcap%
\pgfsetroundjoin%
\definecolor{currentfill}{rgb}{0.000000,0.000000,0.000000}%
\pgfsetfillcolor{currentfill}%
\pgfsetlinewidth{0.602250pt}%
\definecolor{currentstroke}{rgb}{0.000000,0.000000,0.000000}%
\pgfsetstrokecolor{currentstroke}%
\pgfsetdash{}{0pt}%
\pgfsys@defobject{currentmarker}{\pgfqpoint{-0.027778in}{0.000000in}}{\pgfqpoint{-0.000000in}{0.000000in}}{%
\pgfpathmoveto{\pgfqpoint{-0.000000in}{0.000000in}}%
\pgfpathlineto{\pgfqpoint{-0.027778in}{0.000000in}}%
\pgfusepath{stroke,fill}%
}%
\begin{pgfscope}%
\pgfsys@transformshift{0.536250in}{2.133695in}%
\pgfsys@useobject{currentmarker}{}%
\end{pgfscope}%
\end{pgfscope}%
\begin{pgfscope}%
\pgfpathrectangle{\pgfqpoint{0.536250in}{0.525000in}}{\pgfqpoint{1.310000in}{1.743750in}}%
\pgfusepath{clip}%
\pgfsetbuttcap%
\pgfsetroundjoin%
\pgfsetlinewidth{0.803000pt}%
\definecolor{currentstroke}{rgb}{0.752941,0.752941,0.752941}%
\pgfsetstrokecolor{currentstroke}%
\pgfsetdash{{2.960000pt}{1.280000pt}}{0.000000pt}%
\pgfpathmoveto{\pgfqpoint{0.536250in}{2.184257in}}%
\pgfpathlineto{\pgfqpoint{1.846250in}{2.184257in}}%
\pgfusepath{stroke}%
\end{pgfscope}%
\begin{pgfscope}%
\pgfsetbuttcap%
\pgfsetroundjoin%
\definecolor{currentfill}{rgb}{0.000000,0.000000,0.000000}%
\pgfsetfillcolor{currentfill}%
\pgfsetlinewidth{0.602250pt}%
\definecolor{currentstroke}{rgb}{0.000000,0.000000,0.000000}%
\pgfsetstrokecolor{currentstroke}%
\pgfsetdash{}{0pt}%
\pgfsys@defobject{currentmarker}{\pgfqpoint{-0.027778in}{0.000000in}}{\pgfqpoint{-0.000000in}{0.000000in}}{%
\pgfpathmoveto{\pgfqpoint{-0.000000in}{0.000000in}}%
\pgfpathlineto{\pgfqpoint{-0.027778in}{0.000000in}}%
\pgfusepath{stroke,fill}%
}%
\begin{pgfscope}%
\pgfsys@transformshift{0.536250in}{2.184257in}%
\pgfsys@useobject{currentmarker}{}%
\end{pgfscope}%
\end{pgfscope}%
\begin{pgfscope}%
\pgfpathrectangle{\pgfqpoint{0.536250in}{0.525000in}}{\pgfqpoint{1.310000in}{1.743750in}}%
\pgfusepath{clip}%
\pgfsetbuttcap%
\pgfsetroundjoin%
\pgfsetlinewidth{0.803000pt}%
\definecolor{currentstroke}{rgb}{0.752941,0.752941,0.752941}%
\pgfsetstrokecolor{currentstroke}%
\pgfsetdash{{2.960000pt}{1.280000pt}}{0.000000pt}%
\pgfpathmoveto{\pgfqpoint{0.536250in}{2.228855in}}%
\pgfpathlineto{\pgfqpoint{1.846250in}{2.228855in}}%
\pgfusepath{stroke}%
\end{pgfscope}%
\begin{pgfscope}%
\pgfsetbuttcap%
\pgfsetroundjoin%
\definecolor{currentfill}{rgb}{0.000000,0.000000,0.000000}%
\pgfsetfillcolor{currentfill}%
\pgfsetlinewidth{0.602250pt}%
\definecolor{currentstroke}{rgb}{0.000000,0.000000,0.000000}%
\pgfsetstrokecolor{currentstroke}%
\pgfsetdash{}{0pt}%
\pgfsys@defobject{currentmarker}{\pgfqpoint{-0.027778in}{0.000000in}}{\pgfqpoint{-0.000000in}{0.000000in}}{%
\pgfpathmoveto{\pgfqpoint{-0.000000in}{0.000000in}}%
\pgfpathlineto{\pgfqpoint{-0.027778in}{0.000000in}}%
\pgfusepath{stroke,fill}%
}%
\begin{pgfscope}%
\pgfsys@transformshift{0.536250in}{2.228855in}%
\pgfsys@useobject{currentmarker}{}%
\end{pgfscope}%
\end{pgfscope}%
\begin{pgfscope}%
\definecolor{textcolor}{rgb}{0.000000,0.000000,0.000000}%
\pgfsetstrokecolor{textcolor}%
\pgfsetfillcolor{textcolor}%
\pgftext[x=0.197131in,y=1.396875in,,bottom,rotate=90.000000]{\color{textcolor}\rmfamily\fontsize{9.000000}{10.800000}\selectfont Ionization loss [MeV cm\(\displaystyle ^{2}\) g\(\displaystyle ^{-1}\)]}%
\end{pgfscope}%
\begin{pgfscope}%
\pgfpathrectangle{\pgfqpoint{0.536250in}{0.525000in}}{\pgfqpoint{1.310000in}{1.743750in}}%
\pgfusepath{clip}%
\pgfsetrectcap%
\pgfsetroundjoin%
\pgfsetlinewidth{1.003750pt}%
\definecolor{currentstroke}{rgb}{0.000000,0.000000,0.000000}%
\pgfsetstrokecolor{currentstroke}%
\pgfsetdash{}{0pt}%
\pgfpathmoveto{\pgfqpoint{0.569541in}{1.846883in}}%
\pgfpathlineto{\pgfqpoint{0.653180in}{1.175157in}}%
\pgfpathlineto{\pgfqpoint{0.672594in}{1.041498in}}%
\pgfpathlineto{\pgfqpoint{0.688518in}{0.946876in}}%
\pgfpathlineto{\pgfqpoint{0.700213in}{0.887968in}}%
\pgfpathlineto{\pgfqpoint{0.713556in}{0.832059in}}%
\pgfpathlineto{\pgfqpoint{0.724081in}{0.797180in}}%
\pgfpathlineto{\pgfqpoint{0.735484in}{0.767839in}}%
\pgfpathlineto{\pgfqpoint{0.740364in}{0.757736in}}%
\pgfpathlineto{\pgfqpoint{0.749080in}{0.743548in}}%
\pgfpathlineto{\pgfqpoint{0.756651in}{0.734075in}}%
\pgfpathlineto{\pgfqpoint{0.766459in}{0.725700in}}%
\pgfpathlineto{\pgfqpoint{0.774929in}{0.720990in}}%
\pgfpathlineto{\pgfqpoint{0.786946in}{0.718046in}}%
\pgfpathlineto{\pgfqpoint{0.797096in}{0.718046in}}%
\pgfpathlineto{\pgfqpoint{0.813670in}{0.721667in}}%
\pgfpathlineto{\pgfqpoint{0.832797in}{0.729028in}}%
\pgfpathlineto{\pgfqpoint{0.882463in}{0.754651in}}%
\pgfpathlineto{\pgfqpoint{0.944302in}{0.785753in}}%
\pgfpathlineto{\pgfqpoint{0.979684in}{0.801218in}}%
\pgfpathlineto{\pgfqpoint{1.020832in}{0.816952in}}%
\pgfpathlineto{\pgfqpoint{1.065695in}{0.831891in}}%
\pgfpathlineto{\pgfqpoint{1.105889in}{0.843497in}}%
\pgfpathlineto{\pgfqpoint{1.204275in}{0.867383in}}%
\pgfpathlineto{\pgfqpoint{1.846250in}{1.002269in}}%
\pgfpathlineto{\pgfqpoint{1.846250in}{1.002269in}}%
\pgfusepath{stroke}%
\end{pgfscope}%
\begin{pgfscope}%
\pgfpathrectangle{\pgfqpoint{0.536250in}{0.525000in}}{\pgfqpoint{1.310000in}{1.743750in}}%
\pgfusepath{clip}%
\pgfsetbuttcap%
\pgfsetroundjoin%
\pgfsetlinewidth{1.003750pt}%
\definecolor{currentstroke}{rgb}{0.000000,0.000000,0.000000}%
\pgfsetstrokecolor{currentstroke}%
\pgfsetdash{{3.700000pt}{1.600000pt}}{0.000000pt}%
\pgfpathmoveto{\pgfqpoint{0.791743in}{0.525000in}}%
\pgfpathlineto{\pgfqpoint{0.791743in}{2.268750in}}%
\pgfusepath{stroke}%
\end{pgfscope}%
\begin{pgfscope}%
\pgfsetrectcap%
\pgfsetmiterjoin%
\pgfsetlinewidth{1.003750pt}%
\definecolor{currentstroke}{rgb}{0.000000,0.000000,0.000000}%
\pgfsetstrokecolor{currentstroke}%
\pgfsetdash{}{0pt}%
\pgfpathmoveto{\pgfqpoint{0.536250in}{0.525000in}}%
\pgfpathlineto{\pgfqpoint{0.536250in}{2.268750in}}%
\pgfusepath{stroke}%
\end{pgfscope}%
\begin{pgfscope}%
\pgfsetrectcap%
\pgfsetmiterjoin%
\pgfsetlinewidth{1.003750pt}%
\definecolor{currentstroke}{rgb}{0.000000,0.000000,0.000000}%
\pgfsetstrokecolor{currentstroke}%
\pgfsetdash{}{0pt}%
\pgfpathmoveto{\pgfqpoint{1.846250in}{0.525000in}}%
\pgfpathlineto{\pgfqpoint{1.846250in}{2.268750in}}%
\pgfusepath{stroke}%
\end{pgfscope}%
\begin{pgfscope}%
\pgfsetrectcap%
\pgfsetmiterjoin%
\pgfsetlinewidth{1.003750pt}%
\definecolor{currentstroke}{rgb}{0.000000,0.000000,0.000000}%
\pgfsetstrokecolor{currentstroke}%
\pgfsetdash{}{0pt}%
\pgfpathmoveto{\pgfqpoint{0.536250in}{0.525000in}}%
\pgfpathlineto{\pgfqpoint{1.846250in}{0.525000in}}%
\pgfusepath{stroke}%
\end{pgfscope}%
\begin{pgfscope}%
\pgfsetrectcap%
\pgfsetmiterjoin%
\pgfsetlinewidth{1.003750pt}%
\definecolor{currentstroke}{rgb}{0.000000,0.000000,0.000000}%
\pgfsetstrokecolor{currentstroke}%
\pgfsetdash{}{0pt}%
\pgfpathmoveto{\pgfqpoint{0.536250in}{2.268750in}}%
\pgfpathlineto{\pgfqpoint{1.846250in}{2.268750in}}%
\pgfusepath{stroke}%
\end{pgfscope}%
\begin{pgfscope}%
\pgfsetbuttcap%
\pgfsetroundjoin%
\definecolor{currentfill}{rgb}{0.000000,0.000000,0.000000}%
\pgfsetfillcolor{currentfill}%
\pgfsetlinewidth{0.803000pt}%
\definecolor{currentstroke}{rgb}{0.000000,0.000000,0.000000}%
\pgfsetstrokecolor{currentstroke}%
\pgfsetdash{}{0pt}%
\pgfsys@defobject{currentmarker}{\pgfqpoint{0.000000in}{0.000000in}}{\pgfqpoint{0.000000in}{0.048611in}}{%
\pgfpathmoveto{\pgfqpoint{0.000000in}{0.000000in}}%
\pgfpathlineto{\pgfqpoint{0.000000in}{0.048611in}}%
\pgfusepath{stroke,fill}%
}%
\begin{pgfscope}%
\pgfsys@transformshift{0.703915in}{2.268750in}%
\pgfsys@useobject{currentmarker}{}%
\end{pgfscope}%
\end{pgfscope}%
\begin{pgfscope}%
\definecolor{textcolor}{rgb}{0.000000,0.000000,0.000000}%
\pgfsetstrokecolor{textcolor}%
\pgfsetfillcolor{textcolor}%
\pgftext[x=0.703915in,y=2.365972in,,bottom]{\color{textcolor}\rmfamily\fontsize{9.000000}{10.800000}\selectfont \(\displaystyle 10^{0}\)}%
\end{pgfscope}%
\begin{pgfscope}%
\pgfsetbuttcap%
\pgfsetroundjoin%
\definecolor{currentfill}{rgb}{0.000000,0.000000,0.000000}%
\pgfsetfillcolor{currentfill}%
\pgfsetlinewidth{0.803000pt}%
\definecolor{currentstroke}{rgb}{0.000000,0.000000,0.000000}%
\pgfsetstrokecolor{currentstroke}%
\pgfsetdash{}{0pt}%
\pgfsys@defobject{currentmarker}{\pgfqpoint{0.000000in}{0.000000in}}{\pgfqpoint{0.000000in}{0.048611in}}{%
\pgfpathmoveto{\pgfqpoint{0.000000in}{0.000000in}}%
\pgfpathlineto{\pgfqpoint{0.000000in}{0.048611in}}%
\pgfusepath{stroke,fill}%
}%
\begin{pgfscope}%
\pgfsys@transformshift{1.031415in}{2.268750in}%
\pgfsys@useobject{currentmarker}{}%
\end{pgfscope}%
\end{pgfscope}%
\begin{pgfscope}%
\definecolor{textcolor}{rgb}{0.000000,0.000000,0.000000}%
\pgfsetstrokecolor{textcolor}%
\pgfsetfillcolor{textcolor}%
\pgftext[x=1.031415in,y=2.365972in,,bottom]{\color{textcolor}\rmfamily\fontsize{9.000000}{10.800000}\selectfont \(\displaystyle 10^{2}\)}%
\end{pgfscope}%
\begin{pgfscope}%
\pgfsetbuttcap%
\pgfsetroundjoin%
\definecolor{currentfill}{rgb}{0.000000,0.000000,0.000000}%
\pgfsetfillcolor{currentfill}%
\pgfsetlinewidth{0.803000pt}%
\definecolor{currentstroke}{rgb}{0.000000,0.000000,0.000000}%
\pgfsetstrokecolor{currentstroke}%
\pgfsetdash{}{0pt}%
\pgfsys@defobject{currentmarker}{\pgfqpoint{0.000000in}{0.000000in}}{\pgfqpoint{0.000000in}{0.048611in}}{%
\pgfpathmoveto{\pgfqpoint{0.000000in}{0.000000in}}%
\pgfpathlineto{\pgfqpoint{0.000000in}{0.048611in}}%
\pgfusepath{stroke,fill}%
}%
\begin{pgfscope}%
\pgfsys@transformshift{1.358915in}{2.268750in}%
\pgfsys@useobject{currentmarker}{}%
\end{pgfscope}%
\end{pgfscope}%
\begin{pgfscope}%
\definecolor{textcolor}{rgb}{0.000000,0.000000,0.000000}%
\pgfsetstrokecolor{textcolor}%
\pgfsetfillcolor{textcolor}%
\pgftext[x=1.358915in,y=2.365972in,,bottom]{\color{textcolor}\rmfamily\fontsize{9.000000}{10.800000}\selectfont \(\displaystyle 10^{4}\)}%
\end{pgfscope}%
\begin{pgfscope}%
\pgfsetbuttcap%
\pgfsetroundjoin%
\definecolor{currentfill}{rgb}{0.000000,0.000000,0.000000}%
\pgfsetfillcolor{currentfill}%
\pgfsetlinewidth{0.803000pt}%
\definecolor{currentstroke}{rgb}{0.000000,0.000000,0.000000}%
\pgfsetstrokecolor{currentstroke}%
\pgfsetdash{}{0pt}%
\pgfsys@defobject{currentmarker}{\pgfqpoint{0.000000in}{0.000000in}}{\pgfqpoint{0.000000in}{0.048611in}}{%
\pgfpathmoveto{\pgfqpoint{0.000000in}{0.000000in}}%
\pgfpathlineto{\pgfqpoint{0.000000in}{0.048611in}}%
\pgfusepath{stroke,fill}%
}%
\begin{pgfscope}%
\pgfsys@transformshift{1.686415in}{2.268750in}%
\pgfsys@useobject{currentmarker}{}%
\end{pgfscope}%
\end{pgfscope}%
\begin{pgfscope}%
\definecolor{textcolor}{rgb}{0.000000,0.000000,0.000000}%
\pgfsetstrokecolor{textcolor}%
\pgfsetfillcolor{textcolor}%
\pgftext[x=1.686415in,y=2.365972in,,bottom]{\color{textcolor}\rmfamily\fontsize{9.000000}{10.800000}\selectfont \(\displaystyle 10^{6}\)}%
\end{pgfscope}%
\begin{pgfscope}%
\definecolor{textcolor}{rgb}{0.000000,0.000000,0.000000}%
\pgfsetstrokecolor{textcolor}%
\pgfsetfillcolor{textcolor}%
\pgftext[x=1.191250in,y=2.542499in,,base]{\color{textcolor}\rmfamily\fontsize{9.000000}{10.800000}\selectfont \(\displaystyle \beta\gamma\)}%
\end{pgfscope}%
\begin{pgfscope}%
\pgfsetrectcap%
\pgfsetmiterjoin%
\pgfsetlinewidth{1.003750pt}%
\definecolor{currentstroke}{rgb}{0.000000,0.000000,0.000000}%
\pgfsetstrokecolor{currentstroke}%
\pgfsetdash{}{0pt}%
\pgfpathmoveto{\pgfqpoint{0.536250in}{0.525000in}}%
\pgfpathlineto{\pgfqpoint{0.536250in}{2.268750in}}%
\pgfusepath{stroke}%
\end{pgfscope}%
\begin{pgfscope}%
\pgfsetrectcap%
\pgfsetmiterjoin%
\pgfsetlinewidth{1.003750pt}%
\definecolor{currentstroke}{rgb}{0.000000,0.000000,0.000000}%
\pgfsetstrokecolor{currentstroke}%
\pgfsetdash{}{0pt}%
\pgfpathmoveto{\pgfqpoint{1.846250in}{0.525000in}}%
\pgfpathlineto{\pgfqpoint{1.846250in}{2.268750in}}%
\pgfusepath{stroke}%
\end{pgfscope}%
\begin{pgfscope}%
\pgfsetrectcap%
\pgfsetmiterjoin%
\pgfsetlinewidth{1.003750pt}%
\definecolor{currentstroke}{rgb}{0.000000,0.000000,0.000000}%
\pgfsetstrokecolor{currentstroke}%
\pgfsetdash{}{0pt}%
\pgfpathmoveto{\pgfqpoint{0.536250in}{0.525000in}}%
\pgfpathlineto{\pgfqpoint{1.846250in}{0.525000in}}%
\pgfusepath{stroke}%
\end{pgfscope}%
\begin{pgfscope}%
\pgfsetrectcap%
\pgfsetmiterjoin%
\pgfsetlinewidth{1.003750pt}%
\definecolor{currentstroke}{rgb}{0.000000,0.000000,0.000000}%
\pgfsetstrokecolor{currentstroke}%
\pgfsetdash{}{0pt}%
\pgfpathmoveto{\pgfqpoint{0.536250in}{2.268750in}}%
\pgfpathlineto{\pgfqpoint{1.846250in}{2.268750in}}%
\pgfusepath{stroke}%
\end{pgfscope}%
\begin{pgfscope}%
\definecolor{textcolor}{rgb}{0.000000,0.000000,0.000000}%
\pgfsetstrokecolor{textcolor}%
\pgfsetfillcolor{textcolor}%
\pgftext[x=0.663672in,y=2.006289in,,]{\color{textcolor}\rmfamily\fontsize{7.497000}{8.996400}\selectfont \(\displaystyle \frac{1}{E}\)}%
\end{pgfscope}%
\begin{pgfscope}%
\definecolor{textcolor}{rgb}{0.000000,0.000000,0.000000}%
\pgfsetstrokecolor{textcolor}%
\pgfsetfillcolor{textcolor}%
\pgftext[x=1.355000in,y=2.006289in,,]{\color{textcolor}\rmfamily\fontsize{7.497000}{8.996400}\selectfont \(\displaystyle \log{E}\)}%
\end{pgfscope}%
\end{pgfpicture}%
\makeatother%
\endgroup%

  \caption{Muon ionization loss in silicon as a function of the particle momentum
    (adapted from~\cite{PDG}). The vertical dashed line indicates the minimum of
    the ionization (in this specific case at $p \sim 363$~MeV/c, or
    $\beta\gamma \sim 3.44$) separating the two regimes where the energy loss
    scales as $\nicefrac{1}{E}$ and $\log E$, respectively.}
  \label{fig:muon_ionization_loss}
\end{marginfigure}

Basically the energy loss per unit length is proportional to the $Z/A$ ratio of
the target material and the square of the charge $z$ of the incident particle.
When the latter is slow ($\beta \ll 1$) the energy losses decrease as
$1/\beta^2 \propto 1/E$ as $\beta$ increases. In the ultra-relativistic regime
($\beta \approx 1$) the losses increase as $\ln \gamma^2 \propto \ln E$, which
generally goes under the name of relativistic (or logarithmic) rise.
When plotted as a function of the momentum of the incident particle, the ionization
energy-loss curve has the typical shape shown in figure~\ref{fig:muon_ionization_loss},
with a relatively broad minimum at $\beta\gamma \sim 3$, where $dE/dx$ (normalized
to the density of the target) is of the order of 1--2~MeV~g$^{-1}$~cm$^{2}$.
A particle with the energy corresponding to this minimum ionization is customarily
called a \emph{minimum ionizing particle} (MIP).

We note for completeness that, since ionization losses are fundamentally due to
interactions between the incoming particle and the atomic electrons of the
medium, there are significant kinematic differences between the two cases of
heavy and light projectiles. In addition to that, quantum-mechanical effects due
to the fact that the projectile and the target are identical particles come
into play in case of electrons, but we won't insist any further on the subject.


\subsection{Radiation losses and critical energy}

At sufficiently high energy any charged particle passing through matter emits
electromagnetic radiation, most notably via \bremss\ in the field of the
atomic nuclei of the target material---although there are other physical processes
contributing, as explained in detail in~\cite{2001ADNDT..78..183G}.

\begin{marginfigure}
  %% Creator: Matplotlib, PGF backend
%%
%% To include the figure in your LaTeX document, write
%%   \input{<filename>.pgf}
%%
%% Make sure the required packages are loaded in your preamble
%%   \usepackage{pgf}
%%
%% Also ensure that all the required font packages are loaded; for instance,
%% the lmodern package is sometimes necessary when using math font.
%%   \usepackage{lmodern}
%%
%% Figures using additional raster images can only be included by \input if
%% they are in the same directory as the main LaTeX file. For loading figures
%% from other directories you can use the `import` package
%%   \usepackage{import}
%%
%% and then include the figures with
%%   \import{<path to file>}{<filename>.pgf}
%%
%% Matplotlib used the following preamble
%%   \usepackage{fontspec}
%%   \setmainfont{DejaVuSerif.ttf}[Path=\detokenize{/usr/share/matplotlib/mpl-data/fonts/ttf/}]
%%   \setsansfont{DejaVuSans.ttf}[Path=\detokenize{/usr/share/matplotlib/mpl-data/fonts/ttf/}]
%%   \setmonofont{DejaVuSansMono.ttf}[Path=\detokenize{/usr/share/matplotlib/mpl-data/fonts/ttf/}]
%%
\begingroup%
\makeatletter%
\begin{pgfpicture}%
\pgfpathrectangle{\pgfpointorigin}{\pgfqpoint{1.950000in}{2.750000in}}%
\pgfusepath{use as bounding box, clip}%
\begin{pgfscope}%
\pgfsetbuttcap%
\pgfsetmiterjoin%
\definecolor{currentfill}{rgb}{1.000000,1.000000,1.000000}%
\pgfsetfillcolor{currentfill}%
\pgfsetlinewidth{0.000000pt}%
\definecolor{currentstroke}{rgb}{1.000000,1.000000,1.000000}%
\pgfsetstrokecolor{currentstroke}%
\pgfsetdash{}{0pt}%
\pgfpathmoveto{\pgfqpoint{0.000000in}{0.000000in}}%
\pgfpathlineto{\pgfqpoint{1.950000in}{0.000000in}}%
\pgfpathlineto{\pgfqpoint{1.950000in}{2.750000in}}%
\pgfpathlineto{\pgfqpoint{0.000000in}{2.750000in}}%
\pgfpathlineto{\pgfqpoint{0.000000in}{0.000000in}}%
\pgfpathclose%
\pgfusepath{fill}%
\end{pgfscope}%
\begin{pgfscope}%
\pgfsetbuttcap%
\pgfsetmiterjoin%
\definecolor{currentfill}{rgb}{1.000000,1.000000,1.000000}%
\pgfsetfillcolor{currentfill}%
\pgfsetlinewidth{0.000000pt}%
\definecolor{currentstroke}{rgb}{0.000000,0.000000,0.000000}%
\pgfsetstrokecolor{currentstroke}%
\pgfsetstrokeopacity{0.000000}%
\pgfsetdash{}{0pt}%
\pgfpathmoveto{\pgfqpoint{0.536250in}{0.525000in}}%
\pgfpathlineto{\pgfqpoint{1.846250in}{0.525000in}}%
\pgfpathlineto{\pgfqpoint{1.846250in}{2.268750in}}%
\pgfpathlineto{\pgfqpoint{0.536250in}{2.268750in}}%
\pgfpathlineto{\pgfqpoint{0.536250in}{0.525000in}}%
\pgfpathclose%
\pgfusepath{fill}%
\end{pgfscope}%
\begin{pgfscope}%
\pgfpathrectangle{\pgfqpoint{0.536250in}{0.525000in}}{\pgfqpoint{1.310000in}{1.743750in}}%
\pgfusepath{clip}%
\pgfsetbuttcap%
\pgfsetroundjoin%
\pgfsetlinewidth{0.803000pt}%
\definecolor{currentstroke}{rgb}{0.752941,0.752941,0.752941}%
\pgfsetstrokecolor{currentstroke}%
\pgfsetdash{{2.960000pt}{1.280000pt}}{0.000000pt}%
\pgfpathmoveto{\pgfqpoint{0.700000in}{0.525000in}}%
\pgfpathlineto{\pgfqpoint{0.700000in}{2.268750in}}%
\pgfusepath{stroke}%
\end{pgfscope}%
\begin{pgfscope}%
\pgfsetbuttcap%
\pgfsetroundjoin%
\definecolor{currentfill}{rgb}{0.000000,0.000000,0.000000}%
\pgfsetfillcolor{currentfill}%
\pgfsetlinewidth{0.803000pt}%
\definecolor{currentstroke}{rgb}{0.000000,0.000000,0.000000}%
\pgfsetstrokecolor{currentstroke}%
\pgfsetdash{}{0pt}%
\pgfsys@defobject{currentmarker}{\pgfqpoint{0.000000in}{-0.048611in}}{\pgfqpoint{0.000000in}{0.000000in}}{%
\pgfpathmoveto{\pgfqpoint{0.000000in}{0.000000in}}%
\pgfpathlineto{\pgfqpoint{0.000000in}{-0.048611in}}%
\pgfusepath{stroke,fill}%
}%
\begin{pgfscope}%
\pgfsys@transformshift{0.700000in}{0.525000in}%
\pgfsys@useobject{currentmarker}{}%
\end{pgfscope}%
\end{pgfscope}%
\begin{pgfscope}%
\definecolor{textcolor}{rgb}{0.000000,0.000000,0.000000}%
\pgfsetstrokecolor{textcolor}%
\pgfsetfillcolor{textcolor}%
\pgftext[x=0.700000in,y=0.427778in,,top]{\color{textcolor}\rmfamily\fontsize{9.000000}{10.800000}\selectfont \(\displaystyle {10^{2}}\)}%
\end{pgfscope}%
\begin{pgfscope}%
\pgfpathrectangle{\pgfqpoint{0.536250in}{0.525000in}}{\pgfqpoint{1.310000in}{1.743750in}}%
\pgfusepath{clip}%
\pgfsetbuttcap%
\pgfsetroundjoin%
\pgfsetlinewidth{0.803000pt}%
\definecolor{currentstroke}{rgb}{0.752941,0.752941,0.752941}%
\pgfsetstrokecolor{currentstroke}%
\pgfsetdash{{2.960000pt}{1.280000pt}}{0.000000pt}%
\pgfpathmoveto{\pgfqpoint{1.027500in}{0.525000in}}%
\pgfpathlineto{\pgfqpoint{1.027500in}{2.268750in}}%
\pgfusepath{stroke}%
\end{pgfscope}%
\begin{pgfscope}%
\pgfsetbuttcap%
\pgfsetroundjoin%
\definecolor{currentfill}{rgb}{0.000000,0.000000,0.000000}%
\pgfsetfillcolor{currentfill}%
\pgfsetlinewidth{0.803000pt}%
\definecolor{currentstroke}{rgb}{0.000000,0.000000,0.000000}%
\pgfsetstrokecolor{currentstroke}%
\pgfsetdash{}{0pt}%
\pgfsys@defobject{currentmarker}{\pgfqpoint{0.000000in}{-0.048611in}}{\pgfqpoint{0.000000in}{0.000000in}}{%
\pgfpathmoveto{\pgfqpoint{0.000000in}{0.000000in}}%
\pgfpathlineto{\pgfqpoint{0.000000in}{-0.048611in}}%
\pgfusepath{stroke,fill}%
}%
\begin{pgfscope}%
\pgfsys@transformshift{1.027500in}{0.525000in}%
\pgfsys@useobject{currentmarker}{}%
\end{pgfscope}%
\end{pgfscope}%
\begin{pgfscope}%
\definecolor{textcolor}{rgb}{0.000000,0.000000,0.000000}%
\pgfsetstrokecolor{textcolor}%
\pgfsetfillcolor{textcolor}%
\pgftext[x=1.027500in,y=0.427778in,,top]{\color{textcolor}\rmfamily\fontsize{9.000000}{10.800000}\selectfont \(\displaystyle {10^{4}}\)}%
\end{pgfscope}%
\begin{pgfscope}%
\pgfpathrectangle{\pgfqpoint{0.536250in}{0.525000in}}{\pgfqpoint{1.310000in}{1.743750in}}%
\pgfusepath{clip}%
\pgfsetbuttcap%
\pgfsetroundjoin%
\pgfsetlinewidth{0.803000pt}%
\definecolor{currentstroke}{rgb}{0.752941,0.752941,0.752941}%
\pgfsetstrokecolor{currentstroke}%
\pgfsetdash{{2.960000pt}{1.280000pt}}{0.000000pt}%
\pgfpathmoveto{\pgfqpoint{1.355000in}{0.525000in}}%
\pgfpathlineto{\pgfqpoint{1.355000in}{2.268750in}}%
\pgfusepath{stroke}%
\end{pgfscope}%
\begin{pgfscope}%
\pgfsetbuttcap%
\pgfsetroundjoin%
\definecolor{currentfill}{rgb}{0.000000,0.000000,0.000000}%
\pgfsetfillcolor{currentfill}%
\pgfsetlinewidth{0.803000pt}%
\definecolor{currentstroke}{rgb}{0.000000,0.000000,0.000000}%
\pgfsetstrokecolor{currentstroke}%
\pgfsetdash{}{0pt}%
\pgfsys@defobject{currentmarker}{\pgfqpoint{0.000000in}{-0.048611in}}{\pgfqpoint{0.000000in}{0.000000in}}{%
\pgfpathmoveto{\pgfqpoint{0.000000in}{0.000000in}}%
\pgfpathlineto{\pgfqpoint{0.000000in}{-0.048611in}}%
\pgfusepath{stroke,fill}%
}%
\begin{pgfscope}%
\pgfsys@transformshift{1.355000in}{0.525000in}%
\pgfsys@useobject{currentmarker}{}%
\end{pgfscope}%
\end{pgfscope}%
\begin{pgfscope}%
\definecolor{textcolor}{rgb}{0.000000,0.000000,0.000000}%
\pgfsetstrokecolor{textcolor}%
\pgfsetfillcolor{textcolor}%
\pgftext[x=1.355000in,y=0.427778in,,top]{\color{textcolor}\rmfamily\fontsize{9.000000}{10.800000}\selectfont \(\displaystyle {10^{6}}\)}%
\end{pgfscope}%
\begin{pgfscope}%
\pgfpathrectangle{\pgfqpoint{0.536250in}{0.525000in}}{\pgfqpoint{1.310000in}{1.743750in}}%
\pgfusepath{clip}%
\pgfsetbuttcap%
\pgfsetroundjoin%
\pgfsetlinewidth{0.803000pt}%
\definecolor{currentstroke}{rgb}{0.752941,0.752941,0.752941}%
\pgfsetstrokecolor{currentstroke}%
\pgfsetdash{{2.960000pt}{1.280000pt}}{0.000000pt}%
\pgfpathmoveto{\pgfqpoint{1.682500in}{0.525000in}}%
\pgfpathlineto{\pgfqpoint{1.682500in}{2.268750in}}%
\pgfusepath{stroke}%
\end{pgfscope}%
\begin{pgfscope}%
\pgfsetbuttcap%
\pgfsetroundjoin%
\definecolor{currentfill}{rgb}{0.000000,0.000000,0.000000}%
\pgfsetfillcolor{currentfill}%
\pgfsetlinewidth{0.803000pt}%
\definecolor{currentstroke}{rgb}{0.000000,0.000000,0.000000}%
\pgfsetstrokecolor{currentstroke}%
\pgfsetdash{}{0pt}%
\pgfsys@defobject{currentmarker}{\pgfqpoint{0.000000in}{-0.048611in}}{\pgfqpoint{0.000000in}{0.000000in}}{%
\pgfpathmoveto{\pgfqpoint{0.000000in}{0.000000in}}%
\pgfpathlineto{\pgfqpoint{0.000000in}{-0.048611in}}%
\pgfusepath{stroke,fill}%
}%
\begin{pgfscope}%
\pgfsys@transformshift{1.682500in}{0.525000in}%
\pgfsys@useobject{currentmarker}{}%
\end{pgfscope}%
\end{pgfscope}%
\begin{pgfscope}%
\definecolor{textcolor}{rgb}{0.000000,0.000000,0.000000}%
\pgfsetstrokecolor{textcolor}%
\pgfsetfillcolor{textcolor}%
\pgftext[x=1.682500in,y=0.427778in,,top]{\color{textcolor}\rmfamily\fontsize{9.000000}{10.800000}\selectfont \(\displaystyle {10^{8}}\)}%
\end{pgfscope}%
\begin{pgfscope}%
\definecolor{textcolor}{rgb}{0.000000,0.000000,0.000000}%
\pgfsetstrokecolor{textcolor}%
\pgfsetfillcolor{textcolor}%
\pgftext[x=1.191250in,y=0.251251in,,top]{\color{textcolor}\rmfamily\fontsize{9.000000}{10.800000}\selectfont Momentum [MeV/c]}%
\end{pgfscope}%
\begin{pgfscope}%
\pgfpathrectangle{\pgfqpoint{0.536250in}{0.525000in}}{\pgfqpoint{1.310000in}{1.743750in}}%
\pgfusepath{clip}%
\pgfsetbuttcap%
\pgfsetroundjoin%
\pgfsetlinewidth{0.803000pt}%
\definecolor{currentstroke}{rgb}{0.752941,0.752941,0.752941}%
\pgfsetstrokecolor{currentstroke}%
\pgfsetdash{{2.960000pt}{1.280000pt}}{0.000000pt}%
\pgfpathmoveto{\pgfqpoint{0.536250in}{0.525000in}}%
\pgfpathlineto{\pgfqpoint{1.846250in}{0.525000in}}%
\pgfusepath{stroke}%
\end{pgfscope}%
\begin{pgfscope}%
\pgfsetbuttcap%
\pgfsetroundjoin%
\definecolor{currentfill}{rgb}{0.000000,0.000000,0.000000}%
\pgfsetfillcolor{currentfill}%
\pgfsetlinewidth{0.803000pt}%
\definecolor{currentstroke}{rgb}{0.000000,0.000000,0.000000}%
\pgfsetstrokecolor{currentstroke}%
\pgfsetdash{}{0pt}%
\pgfsys@defobject{currentmarker}{\pgfqpoint{-0.048611in}{0.000000in}}{\pgfqpoint{-0.000000in}{0.000000in}}{%
\pgfpathmoveto{\pgfqpoint{-0.000000in}{0.000000in}}%
\pgfpathlineto{\pgfqpoint{-0.048611in}{0.000000in}}%
\pgfusepath{stroke,fill}%
}%
\begin{pgfscope}%
\pgfsys@transformshift{0.536250in}{0.525000in}%
\pgfsys@useobject{currentmarker}{}%
\end{pgfscope}%
\end{pgfscope}%
\begin{pgfscope}%
\definecolor{textcolor}{rgb}{0.000000,0.000000,0.000000}%
\pgfsetstrokecolor{textcolor}%
\pgfsetfillcolor{textcolor}%
\pgftext[x=0.252687in, y=0.477515in, left, base]{\color{textcolor}\rmfamily\fontsize{9.000000}{10.800000}\selectfont \(\displaystyle {10^{0}}\)}%
\end{pgfscope}%
\begin{pgfscope}%
\pgfpathrectangle{\pgfqpoint{0.536250in}{0.525000in}}{\pgfqpoint{1.310000in}{1.743750in}}%
\pgfusepath{clip}%
\pgfsetbuttcap%
\pgfsetroundjoin%
\pgfsetlinewidth{0.803000pt}%
\definecolor{currentstroke}{rgb}{0.752941,0.752941,0.752941}%
\pgfsetstrokecolor{currentstroke}%
\pgfsetdash{{2.960000pt}{1.280000pt}}{0.000000pt}%
\pgfpathmoveto{\pgfqpoint{0.536250in}{1.396875in}}%
\pgfpathlineto{\pgfqpoint{1.846250in}{1.396875in}}%
\pgfusepath{stroke}%
\end{pgfscope}%
\begin{pgfscope}%
\pgfsetbuttcap%
\pgfsetroundjoin%
\definecolor{currentfill}{rgb}{0.000000,0.000000,0.000000}%
\pgfsetfillcolor{currentfill}%
\pgfsetlinewidth{0.803000pt}%
\definecolor{currentstroke}{rgb}{0.000000,0.000000,0.000000}%
\pgfsetstrokecolor{currentstroke}%
\pgfsetdash{}{0pt}%
\pgfsys@defobject{currentmarker}{\pgfqpoint{-0.048611in}{0.000000in}}{\pgfqpoint{-0.000000in}{0.000000in}}{%
\pgfpathmoveto{\pgfqpoint{-0.000000in}{0.000000in}}%
\pgfpathlineto{\pgfqpoint{-0.048611in}{0.000000in}}%
\pgfusepath{stroke,fill}%
}%
\begin{pgfscope}%
\pgfsys@transformshift{0.536250in}{1.396875in}%
\pgfsys@useobject{currentmarker}{}%
\end{pgfscope}%
\end{pgfscope}%
\begin{pgfscope}%
\definecolor{textcolor}{rgb}{0.000000,0.000000,0.000000}%
\pgfsetstrokecolor{textcolor}%
\pgfsetfillcolor{textcolor}%
\pgftext[x=0.252687in, y=1.349390in, left, base]{\color{textcolor}\rmfamily\fontsize{9.000000}{10.800000}\selectfont \(\displaystyle {10^{1}}\)}%
\end{pgfscope}%
\begin{pgfscope}%
\pgfpathrectangle{\pgfqpoint{0.536250in}{0.525000in}}{\pgfqpoint{1.310000in}{1.743750in}}%
\pgfusepath{clip}%
\pgfsetbuttcap%
\pgfsetroundjoin%
\pgfsetlinewidth{0.803000pt}%
\definecolor{currentstroke}{rgb}{0.752941,0.752941,0.752941}%
\pgfsetstrokecolor{currentstroke}%
\pgfsetdash{{2.960000pt}{1.280000pt}}{0.000000pt}%
\pgfpathmoveto{\pgfqpoint{0.536250in}{2.268750in}}%
\pgfpathlineto{\pgfqpoint{1.846250in}{2.268750in}}%
\pgfusepath{stroke}%
\end{pgfscope}%
\begin{pgfscope}%
\pgfsetbuttcap%
\pgfsetroundjoin%
\definecolor{currentfill}{rgb}{0.000000,0.000000,0.000000}%
\pgfsetfillcolor{currentfill}%
\pgfsetlinewidth{0.803000pt}%
\definecolor{currentstroke}{rgb}{0.000000,0.000000,0.000000}%
\pgfsetstrokecolor{currentstroke}%
\pgfsetdash{}{0pt}%
\pgfsys@defobject{currentmarker}{\pgfqpoint{-0.048611in}{0.000000in}}{\pgfqpoint{-0.000000in}{0.000000in}}{%
\pgfpathmoveto{\pgfqpoint{-0.000000in}{0.000000in}}%
\pgfpathlineto{\pgfqpoint{-0.048611in}{0.000000in}}%
\pgfusepath{stroke,fill}%
}%
\begin{pgfscope}%
\pgfsys@transformshift{0.536250in}{2.268750in}%
\pgfsys@useobject{currentmarker}{}%
\end{pgfscope}%
\end{pgfscope}%
\begin{pgfscope}%
\definecolor{textcolor}{rgb}{0.000000,0.000000,0.000000}%
\pgfsetstrokecolor{textcolor}%
\pgfsetfillcolor{textcolor}%
\pgftext[x=0.252687in, y=2.221265in, left, base]{\color{textcolor}\rmfamily\fontsize{9.000000}{10.800000}\selectfont \(\displaystyle {10^{2}}\)}%
\end{pgfscope}%
\begin{pgfscope}%
\pgfpathrectangle{\pgfqpoint{0.536250in}{0.525000in}}{\pgfqpoint{1.310000in}{1.743750in}}%
\pgfusepath{clip}%
\pgfsetbuttcap%
\pgfsetroundjoin%
\pgfsetlinewidth{0.803000pt}%
\definecolor{currentstroke}{rgb}{0.752941,0.752941,0.752941}%
\pgfsetstrokecolor{currentstroke}%
\pgfsetdash{{2.960000pt}{1.280000pt}}{0.000000pt}%
\pgfpathmoveto{\pgfqpoint{0.536250in}{0.787461in}}%
\pgfpathlineto{\pgfqpoint{1.846250in}{0.787461in}}%
\pgfusepath{stroke}%
\end{pgfscope}%
\begin{pgfscope}%
\pgfsetbuttcap%
\pgfsetroundjoin%
\definecolor{currentfill}{rgb}{0.000000,0.000000,0.000000}%
\pgfsetfillcolor{currentfill}%
\pgfsetlinewidth{0.602250pt}%
\definecolor{currentstroke}{rgb}{0.000000,0.000000,0.000000}%
\pgfsetstrokecolor{currentstroke}%
\pgfsetdash{}{0pt}%
\pgfsys@defobject{currentmarker}{\pgfqpoint{-0.027778in}{0.000000in}}{\pgfqpoint{-0.000000in}{0.000000in}}{%
\pgfpathmoveto{\pgfqpoint{-0.000000in}{0.000000in}}%
\pgfpathlineto{\pgfqpoint{-0.027778in}{0.000000in}}%
\pgfusepath{stroke,fill}%
}%
\begin{pgfscope}%
\pgfsys@transformshift{0.536250in}{0.787461in}%
\pgfsys@useobject{currentmarker}{}%
\end{pgfscope}%
\end{pgfscope}%
\begin{pgfscope}%
\pgfpathrectangle{\pgfqpoint{0.536250in}{0.525000in}}{\pgfqpoint{1.310000in}{1.743750in}}%
\pgfusepath{clip}%
\pgfsetbuttcap%
\pgfsetroundjoin%
\pgfsetlinewidth{0.803000pt}%
\definecolor{currentstroke}{rgb}{0.752941,0.752941,0.752941}%
\pgfsetstrokecolor{currentstroke}%
\pgfsetdash{{2.960000pt}{1.280000pt}}{0.000000pt}%
\pgfpathmoveto{\pgfqpoint{0.536250in}{0.940990in}}%
\pgfpathlineto{\pgfqpoint{1.846250in}{0.940990in}}%
\pgfusepath{stroke}%
\end{pgfscope}%
\begin{pgfscope}%
\pgfsetbuttcap%
\pgfsetroundjoin%
\definecolor{currentfill}{rgb}{0.000000,0.000000,0.000000}%
\pgfsetfillcolor{currentfill}%
\pgfsetlinewidth{0.602250pt}%
\definecolor{currentstroke}{rgb}{0.000000,0.000000,0.000000}%
\pgfsetstrokecolor{currentstroke}%
\pgfsetdash{}{0pt}%
\pgfsys@defobject{currentmarker}{\pgfqpoint{-0.027778in}{0.000000in}}{\pgfqpoint{-0.000000in}{0.000000in}}{%
\pgfpathmoveto{\pgfqpoint{-0.000000in}{0.000000in}}%
\pgfpathlineto{\pgfqpoint{-0.027778in}{0.000000in}}%
\pgfusepath{stroke,fill}%
}%
\begin{pgfscope}%
\pgfsys@transformshift{0.536250in}{0.940990in}%
\pgfsys@useobject{currentmarker}{}%
\end{pgfscope}%
\end{pgfscope}%
\begin{pgfscope}%
\pgfpathrectangle{\pgfqpoint{0.536250in}{0.525000in}}{\pgfqpoint{1.310000in}{1.743750in}}%
\pgfusepath{clip}%
\pgfsetbuttcap%
\pgfsetroundjoin%
\pgfsetlinewidth{0.803000pt}%
\definecolor{currentstroke}{rgb}{0.752941,0.752941,0.752941}%
\pgfsetstrokecolor{currentstroke}%
\pgfsetdash{{2.960000pt}{1.280000pt}}{0.000000pt}%
\pgfpathmoveto{\pgfqpoint{0.536250in}{1.049921in}}%
\pgfpathlineto{\pgfqpoint{1.846250in}{1.049921in}}%
\pgfusepath{stroke}%
\end{pgfscope}%
\begin{pgfscope}%
\pgfsetbuttcap%
\pgfsetroundjoin%
\definecolor{currentfill}{rgb}{0.000000,0.000000,0.000000}%
\pgfsetfillcolor{currentfill}%
\pgfsetlinewidth{0.602250pt}%
\definecolor{currentstroke}{rgb}{0.000000,0.000000,0.000000}%
\pgfsetstrokecolor{currentstroke}%
\pgfsetdash{}{0pt}%
\pgfsys@defobject{currentmarker}{\pgfqpoint{-0.027778in}{0.000000in}}{\pgfqpoint{-0.000000in}{0.000000in}}{%
\pgfpathmoveto{\pgfqpoint{-0.000000in}{0.000000in}}%
\pgfpathlineto{\pgfqpoint{-0.027778in}{0.000000in}}%
\pgfusepath{stroke,fill}%
}%
\begin{pgfscope}%
\pgfsys@transformshift{0.536250in}{1.049921in}%
\pgfsys@useobject{currentmarker}{}%
\end{pgfscope}%
\end{pgfscope}%
\begin{pgfscope}%
\pgfpathrectangle{\pgfqpoint{0.536250in}{0.525000in}}{\pgfqpoint{1.310000in}{1.743750in}}%
\pgfusepath{clip}%
\pgfsetbuttcap%
\pgfsetroundjoin%
\pgfsetlinewidth{0.803000pt}%
\definecolor{currentstroke}{rgb}{0.752941,0.752941,0.752941}%
\pgfsetstrokecolor{currentstroke}%
\pgfsetdash{{2.960000pt}{1.280000pt}}{0.000000pt}%
\pgfpathmoveto{\pgfqpoint{0.536250in}{1.134414in}}%
\pgfpathlineto{\pgfqpoint{1.846250in}{1.134414in}}%
\pgfusepath{stroke}%
\end{pgfscope}%
\begin{pgfscope}%
\pgfsetbuttcap%
\pgfsetroundjoin%
\definecolor{currentfill}{rgb}{0.000000,0.000000,0.000000}%
\pgfsetfillcolor{currentfill}%
\pgfsetlinewidth{0.602250pt}%
\definecolor{currentstroke}{rgb}{0.000000,0.000000,0.000000}%
\pgfsetstrokecolor{currentstroke}%
\pgfsetdash{}{0pt}%
\pgfsys@defobject{currentmarker}{\pgfqpoint{-0.027778in}{0.000000in}}{\pgfqpoint{-0.000000in}{0.000000in}}{%
\pgfpathmoveto{\pgfqpoint{-0.000000in}{0.000000in}}%
\pgfpathlineto{\pgfqpoint{-0.027778in}{0.000000in}}%
\pgfusepath{stroke,fill}%
}%
\begin{pgfscope}%
\pgfsys@transformshift{0.536250in}{1.134414in}%
\pgfsys@useobject{currentmarker}{}%
\end{pgfscope}%
\end{pgfscope}%
\begin{pgfscope}%
\pgfpathrectangle{\pgfqpoint{0.536250in}{0.525000in}}{\pgfqpoint{1.310000in}{1.743750in}}%
\pgfusepath{clip}%
\pgfsetbuttcap%
\pgfsetroundjoin%
\pgfsetlinewidth{0.803000pt}%
\definecolor{currentstroke}{rgb}{0.752941,0.752941,0.752941}%
\pgfsetstrokecolor{currentstroke}%
\pgfsetdash{{2.960000pt}{1.280000pt}}{0.000000pt}%
\pgfpathmoveto{\pgfqpoint{0.536250in}{1.203451in}}%
\pgfpathlineto{\pgfqpoint{1.846250in}{1.203451in}}%
\pgfusepath{stroke}%
\end{pgfscope}%
\begin{pgfscope}%
\pgfsetbuttcap%
\pgfsetroundjoin%
\definecolor{currentfill}{rgb}{0.000000,0.000000,0.000000}%
\pgfsetfillcolor{currentfill}%
\pgfsetlinewidth{0.602250pt}%
\definecolor{currentstroke}{rgb}{0.000000,0.000000,0.000000}%
\pgfsetstrokecolor{currentstroke}%
\pgfsetdash{}{0pt}%
\pgfsys@defobject{currentmarker}{\pgfqpoint{-0.027778in}{0.000000in}}{\pgfqpoint{-0.000000in}{0.000000in}}{%
\pgfpathmoveto{\pgfqpoint{-0.000000in}{0.000000in}}%
\pgfpathlineto{\pgfqpoint{-0.027778in}{0.000000in}}%
\pgfusepath{stroke,fill}%
}%
\begin{pgfscope}%
\pgfsys@transformshift{0.536250in}{1.203451in}%
\pgfsys@useobject{currentmarker}{}%
\end{pgfscope}%
\end{pgfscope}%
\begin{pgfscope}%
\pgfpathrectangle{\pgfqpoint{0.536250in}{0.525000in}}{\pgfqpoint{1.310000in}{1.743750in}}%
\pgfusepath{clip}%
\pgfsetbuttcap%
\pgfsetroundjoin%
\pgfsetlinewidth{0.803000pt}%
\definecolor{currentstroke}{rgb}{0.752941,0.752941,0.752941}%
\pgfsetstrokecolor{currentstroke}%
\pgfsetdash{{2.960000pt}{1.280000pt}}{0.000000pt}%
\pgfpathmoveto{\pgfqpoint{0.536250in}{1.261820in}}%
\pgfpathlineto{\pgfqpoint{1.846250in}{1.261820in}}%
\pgfusepath{stroke}%
\end{pgfscope}%
\begin{pgfscope}%
\pgfsetbuttcap%
\pgfsetroundjoin%
\definecolor{currentfill}{rgb}{0.000000,0.000000,0.000000}%
\pgfsetfillcolor{currentfill}%
\pgfsetlinewidth{0.602250pt}%
\definecolor{currentstroke}{rgb}{0.000000,0.000000,0.000000}%
\pgfsetstrokecolor{currentstroke}%
\pgfsetdash{}{0pt}%
\pgfsys@defobject{currentmarker}{\pgfqpoint{-0.027778in}{0.000000in}}{\pgfqpoint{-0.000000in}{0.000000in}}{%
\pgfpathmoveto{\pgfqpoint{-0.000000in}{0.000000in}}%
\pgfpathlineto{\pgfqpoint{-0.027778in}{0.000000in}}%
\pgfusepath{stroke,fill}%
}%
\begin{pgfscope}%
\pgfsys@transformshift{0.536250in}{1.261820in}%
\pgfsys@useobject{currentmarker}{}%
\end{pgfscope}%
\end{pgfscope}%
\begin{pgfscope}%
\pgfpathrectangle{\pgfqpoint{0.536250in}{0.525000in}}{\pgfqpoint{1.310000in}{1.743750in}}%
\pgfusepath{clip}%
\pgfsetbuttcap%
\pgfsetroundjoin%
\pgfsetlinewidth{0.803000pt}%
\definecolor{currentstroke}{rgb}{0.752941,0.752941,0.752941}%
\pgfsetstrokecolor{currentstroke}%
\pgfsetdash{{2.960000pt}{1.280000pt}}{0.000000pt}%
\pgfpathmoveto{\pgfqpoint{0.536250in}{1.312382in}}%
\pgfpathlineto{\pgfqpoint{1.846250in}{1.312382in}}%
\pgfusepath{stroke}%
\end{pgfscope}%
\begin{pgfscope}%
\pgfsetbuttcap%
\pgfsetroundjoin%
\definecolor{currentfill}{rgb}{0.000000,0.000000,0.000000}%
\pgfsetfillcolor{currentfill}%
\pgfsetlinewidth{0.602250pt}%
\definecolor{currentstroke}{rgb}{0.000000,0.000000,0.000000}%
\pgfsetstrokecolor{currentstroke}%
\pgfsetdash{}{0pt}%
\pgfsys@defobject{currentmarker}{\pgfqpoint{-0.027778in}{0.000000in}}{\pgfqpoint{-0.000000in}{0.000000in}}{%
\pgfpathmoveto{\pgfqpoint{-0.000000in}{0.000000in}}%
\pgfpathlineto{\pgfqpoint{-0.027778in}{0.000000in}}%
\pgfusepath{stroke,fill}%
}%
\begin{pgfscope}%
\pgfsys@transformshift{0.536250in}{1.312382in}%
\pgfsys@useobject{currentmarker}{}%
\end{pgfscope}%
\end{pgfscope}%
\begin{pgfscope}%
\pgfpathrectangle{\pgfqpoint{0.536250in}{0.525000in}}{\pgfqpoint{1.310000in}{1.743750in}}%
\pgfusepath{clip}%
\pgfsetbuttcap%
\pgfsetroundjoin%
\pgfsetlinewidth{0.803000pt}%
\definecolor{currentstroke}{rgb}{0.752941,0.752941,0.752941}%
\pgfsetstrokecolor{currentstroke}%
\pgfsetdash{{2.960000pt}{1.280000pt}}{0.000000pt}%
\pgfpathmoveto{\pgfqpoint{0.536250in}{1.356980in}}%
\pgfpathlineto{\pgfqpoint{1.846250in}{1.356980in}}%
\pgfusepath{stroke}%
\end{pgfscope}%
\begin{pgfscope}%
\pgfsetbuttcap%
\pgfsetroundjoin%
\definecolor{currentfill}{rgb}{0.000000,0.000000,0.000000}%
\pgfsetfillcolor{currentfill}%
\pgfsetlinewidth{0.602250pt}%
\definecolor{currentstroke}{rgb}{0.000000,0.000000,0.000000}%
\pgfsetstrokecolor{currentstroke}%
\pgfsetdash{}{0pt}%
\pgfsys@defobject{currentmarker}{\pgfqpoint{-0.027778in}{0.000000in}}{\pgfqpoint{-0.000000in}{0.000000in}}{%
\pgfpathmoveto{\pgfqpoint{-0.000000in}{0.000000in}}%
\pgfpathlineto{\pgfqpoint{-0.027778in}{0.000000in}}%
\pgfusepath{stroke,fill}%
}%
\begin{pgfscope}%
\pgfsys@transformshift{0.536250in}{1.356980in}%
\pgfsys@useobject{currentmarker}{}%
\end{pgfscope}%
\end{pgfscope}%
\begin{pgfscope}%
\pgfpathrectangle{\pgfqpoint{0.536250in}{0.525000in}}{\pgfqpoint{1.310000in}{1.743750in}}%
\pgfusepath{clip}%
\pgfsetbuttcap%
\pgfsetroundjoin%
\pgfsetlinewidth{0.803000pt}%
\definecolor{currentstroke}{rgb}{0.752941,0.752941,0.752941}%
\pgfsetstrokecolor{currentstroke}%
\pgfsetdash{{2.960000pt}{1.280000pt}}{0.000000pt}%
\pgfpathmoveto{\pgfqpoint{0.536250in}{1.659336in}}%
\pgfpathlineto{\pgfqpoint{1.846250in}{1.659336in}}%
\pgfusepath{stroke}%
\end{pgfscope}%
\begin{pgfscope}%
\pgfsetbuttcap%
\pgfsetroundjoin%
\definecolor{currentfill}{rgb}{0.000000,0.000000,0.000000}%
\pgfsetfillcolor{currentfill}%
\pgfsetlinewidth{0.602250pt}%
\definecolor{currentstroke}{rgb}{0.000000,0.000000,0.000000}%
\pgfsetstrokecolor{currentstroke}%
\pgfsetdash{}{0pt}%
\pgfsys@defobject{currentmarker}{\pgfqpoint{-0.027778in}{0.000000in}}{\pgfqpoint{-0.000000in}{0.000000in}}{%
\pgfpathmoveto{\pgfqpoint{-0.000000in}{0.000000in}}%
\pgfpathlineto{\pgfqpoint{-0.027778in}{0.000000in}}%
\pgfusepath{stroke,fill}%
}%
\begin{pgfscope}%
\pgfsys@transformshift{0.536250in}{1.659336in}%
\pgfsys@useobject{currentmarker}{}%
\end{pgfscope}%
\end{pgfscope}%
\begin{pgfscope}%
\pgfpathrectangle{\pgfqpoint{0.536250in}{0.525000in}}{\pgfqpoint{1.310000in}{1.743750in}}%
\pgfusepath{clip}%
\pgfsetbuttcap%
\pgfsetroundjoin%
\pgfsetlinewidth{0.803000pt}%
\definecolor{currentstroke}{rgb}{0.752941,0.752941,0.752941}%
\pgfsetstrokecolor{currentstroke}%
\pgfsetdash{{2.960000pt}{1.280000pt}}{0.000000pt}%
\pgfpathmoveto{\pgfqpoint{0.536250in}{1.812865in}}%
\pgfpathlineto{\pgfqpoint{1.846250in}{1.812865in}}%
\pgfusepath{stroke}%
\end{pgfscope}%
\begin{pgfscope}%
\pgfsetbuttcap%
\pgfsetroundjoin%
\definecolor{currentfill}{rgb}{0.000000,0.000000,0.000000}%
\pgfsetfillcolor{currentfill}%
\pgfsetlinewidth{0.602250pt}%
\definecolor{currentstroke}{rgb}{0.000000,0.000000,0.000000}%
\pgfsetstrokecolor{currentstroke}%
\pgfsetdash{}{0pt}%
\pgfsys@defobject{currentmarker}{\pgfqpoint{-0.027778in}{0.000000in}}{\pgfqpoint{-0.000000in}{0.000000in}}{%
\pgfpathmoveto{\pgfqpoint{-0.000000in}{0.000000in}}%
\pgfpathlineto{\pgfqpoint{-0.027778in}{0.000000in}}%
\pgfusepath{stroke,fill}%
}%
\begin{pgfscope}%
\pgfsys@transformshift{0.536250in}{1.812865in}%
\pgfsys@useobject{currentmarker}{}%
\end{pgfscope}%
\end{pgfscope}%
\begin{pgfscope}%
\pgfpathrectangle{\pgfqpoint{0.536250in}{0.525000in}}{\pgfqpoint{1.310000in}{1.743750in}}%
\pgfusepath{clip}%
\pgfsetbuttcap%
\pgfsetroundjoin%
\pgfsetlinewidth{0.803000pt}%
\definecolor{currentstroke}{rgb}{0.752941,0.752941,0.752941}%
\pgfsetstrokecolor{currentstroke}%
\pgfsetdash{{2.960000pt}{1.280000pt}}{0.000000pt}%
\pgfpathmoveto{\pgfqpoint{0.536250in}{1.921796in}}%
\pgfpathlineto{\pgfqpoint{1.846250in}{1.921796in}}%
\pgfusepath{stroke}%
\end{pgfscope}%
\begin{pgfscope}%
\pgfsetbuttcap%
\pgfsetroundjoin%
\definecolor{currentfill}{rgb}{0.000000,0.000000,0.000000}%
\pgfsetfillcolor{currentfill}%
\pgfsetlinewidth{0.602250pt}%
\definecolor{currentstroke}{rgb}{0.000000,0.000000,0.000000}%
\pgfsetstrokecolor{currentstroke}%
\pgfsetdash{}{0pt}%
\pgfsys@defobject{currentmarker}{\pgfqpoint{-0.027778in}{0.000000in}}{\pgfqpoint{-0.000000in}{0.000000in}}{%
\pgfpathmoveto{\pgfqpoint{-0.000000in}{0.000000in}}%
\pgfpathlineto{\pgfqpoint{-0.027778in}{0.000000in}}%
\pgfusepath{stroke,fill}%
}%
\begin{pgfscope}%
\pgfsys@transformshift{0.536250in}{1.921796in}%
\pgfsys@useobject{currentmarker}{}%
\end{pgfscope}%
\end{pgfscope}%
\begin{pgfscope}%
\pgfpathrectangle{\pgfqpoint{0.536250in}{0.525000in}}{\pgfqpoint{1.310000in}{1.743750in}}%
\pgfusepath{clip}%
\pgfsetbuttcap%
\pgfsetroundjoin%
\pgfsetlinewidth{0.803000pt}%
\definecolor{currentstroke}{rgb}{0.752941,0.752941,0.752941}%
\pgfsetstrokecolor{currentstroke}%
\pgfsetdash{{2.960000pt}{1.280000pt}}{0.000000pt}%
\pgfpathmoveto{\pgfqpoint{0.536250in}{2.006289in}}%
\pgfpathlineto{\pgfqpoint{1.846250in}{2.006289in}}%
\pgfusepath{stroke}%
\end{pgfscope}%
\begin{pgfscope}%
\pgfsetbuttcap%
\pgfsetroundjoin%
\definecolor{currentfill}{rgb}{0.000000,0.000000,0.000000}%
\pgfsetfillcolor{currentfill}%
\pgfsetlinewidth{0.602250pt}%
\definecolor{currentstroke}{rgb}{0.000000,0.000000,0.000000}%
\pgfsetstrokecolor{currentstroke}%
\pgfsetdash{}{0pt}%
\pgfsys@defobject{currentmarker}{\pgfqpoint{-0.027778in}{0.000000in}}{\pgfqpoint{-0.000000in}{0.000000in}}{%
\pgfpathmoveto{\pgfqpoint{-0.000000in}{0.000000in}}%
\pgfpathlineto{\pgfqpoint{-0.027778in}{0.000000in}}%
\pgfusepath{stroke,fill}%
}%
\begin{pgfscope}%
\pgfsys@transformshift{0.536250in}{2.006289in}%
\pgfsys@useobject{currentmarker}{}%
\end{pgfscope}%
\end{pgfscope}%
\begin{pgfscope}%
\pgfpathrectangle{\pgfqpoint{0.536250in}{0.525000in}}{\pgfqpoint{1.310000in}{1.743750in}}%
\pgfusepath{clip}%
\pgfsetbuttcap%
\pgfsetroundjoin%
\pgfsetlinewidth{0.803000pt}%
\definecolor{currentstroke}{rgb}{0.752941,0.752941,0.752941}%
\pgfsetstrokecolor{currentstroke}%
\pgfsetdash{{2.960000pt}{1.280000pt}}{0.000000pt}%
\pgfpathmoveto{\pgfqpoint{0.536250in}{2.075326in}}%
\pgfpathlineto{\pgfqpoint{1.846250in}{2.075326in}}%
\pgfusepath{stroke}%
\end{pgfscope}%
\begin{pgfscope}%
\pgfsetbuttcap%
\pgfsetroundjoin%
\definecolor{currentfill}{rgb}{0.000000,0.000000,0.000000}%
\pgfsetfillcolor{currentfill}%
\pgfsetlinewidth{0.602250pt}%
\definecolor{currentstroke}{rgb}{0.000000,0.000000,0.000000}%
\pgfsetstrokecolor{currentstroke}%
\pgfsetdash{}{0pt}%
\pgfsys@defobject{currentmarker}{\pgfqpoint{-0.027778in}{0.000000in}}{\pgfqpoint{-0.000000in}{0.000000in}}{%
\pgfpathmoveto{\pgfqpoint{-0.000000in}{0.000000in}}%
\pgfpathlineto{\pgfqpoint{-0.027778in}{0.000000in}}%
\pgfusepath{stroke,fill}%
}%
\begin{pgfscope}%
\pgfsys@transformshift{0.536250in}{2.075326in}%
\pgfsys@useobject{currentmarker}{}%
\end{pgfscope}%
\end{pgfscope}%
\begin{pgfscope}%
\pgfpathrectangle{\pgfqpoint{0.536250in}{0.525000in}}{\pgfqpoint{1.310000in}{1.743750in}}%
\pgfusepath{clip}%
\pgfsetbuttcap%
\pgfsetroundjoin%
\pgfsetlinewidth{0.803000pt}%
\definecolor{currentstroke}{rgb}{0.752941,0.752941,0.752941}%
\pgfsetstrokecolor{currentstroke}%
\pgfsetdash{{2.960000pt}{1.280000pt}}{0.000000pt}%
\pgfpathmoveto{\pgfqpoint{0.536250in}{2.133695in}}%
\pgfpathlineto{\pgfqpoint{1.846250in}{2.133695in}}%
\pgfusepath{stroke}%
\end{pgfscope}%
\begin{pgfscope}%
\pgfsetbuttcap%
\pgfsetroundjoin%
\definecolor{currentfill}{rgb}{0.000000,0.000000,0.000000}%
\pgfsetfillcolor{currentfill}%
\pgfsetlinewidth{0.602250pt}%
\definecolor{currentstroke}{rgb}{0.000000,0.000000,0.000000}%
\pgfsetstrokecolor{currentstroke}%
\pgfsetdash{}{0pt}%
\pgfsys@defobject{currentmarker}{\pgfqpoint{-0.027778in}{0.000000in}}{\pgfqpoint{-0.000000in}{0.000000in}}{%
\pgfpathmoveto{\pgfqpoint{-0.000000in}{0.000000in}}%
\pgfpathlineto{\pgfqpoint{-0.027778in}{0.000000in}}%
\pgfusepath{stroke,fill}%
}%
\begin{pgfscope}%
\pgfsys@transformshift{0.536250in}{2.133695in}%
\pgfsys@useobject{currentmarker}{}%
\end{pgfscope}%
\end{pgfscope}%
\begin{pgfscope}%
\pgfpathrectangle{\pgfqpoint{0.536250in}{0.525000in}}{\pgfqpoint{1.310000in}{1.743750in}}%
\pgfusepath{clip}%
\pgfsetbuttcap%
\pgfsetroundjoin%
\pgfsetlinewidth{0.803000pt}%
\definecolor{currentstroke}{rgb}{0.752941,0.752941,0.752941}%
\pgfsetstrokecolor{currentstroke}%
\pgfsetdash{{2.960000pt}{1.280000pt}}{0.000000pt}%
\pgfpathmoveto{\pgfqpoint{0.536250in}{2.184257in}}%
\pgfpathlineto{\pgfqpoint{1.846250in}{2.184257in}}%
\pgfusepath{stroke}%
\end{pgfscope}%
\begin{pgfscope}%
\pgfsetbuttcap%
\pgfsetroundjoin%
\definecolor{currentfill}{rgb}{0.000000,0.000000,0.000000}%
\pgfsetfillcolor{currentfill}%
\pgfsetlinewidth{0.602250pt}%
\definecolor{currentstroke}{rgb}{0.000000,0.000000,0.000000}%
\pgfsetstrokecolor{currentstroke}%
\pgfsetdash{}{0pt}%
\pgfsys@defobject{currentmarker}{\pgfqpoint{-0.027778in}{0.000000in}}{\pgfqpoint{-0.000000in}{0.000000in}}{%
\pgfpathmoveto{\pgfqpoint{-0.000000in}{0.000000in}}%
\pgfpathlineto{\pgfqpoint{-0.027778in}{0.000000in}}%
\pgfusepath{stroke,fill}%
}%
\begin{pgfscope}%
\pgfsys@transformshift{0.536250in}{2.184257in}%
\pgfsys@useobject{currentmarker}{}%
\end{pgfscope}%
\end{pgfscope}%
\begin{pgfscope}%
\pgfpathrectangle{\pgfqpoint{0.536250in}{0.525000in}}{\pgfqpoint{1.310000in}{1.743750in}}%
\pgfusepath{clip}%
\pgfsetbuttcap%
\pgfsetroundjoin%
\pgfsetlinewidth{0.803000pt}%
\definecolor{currentstroke}{rgb}{0.752941,0.752941,0.752941}%
\pgfsetstrokecolor{currentstroke}%
\pgfsetdash{{2.960000pt}{1.280000pt}}{0.000000pt}%
\pgfpathmoveto{\pgfqpoint{0.536250in}{2.228855in}}%
\pgfpathlineto{\pgfqpoint{1.846250in}{2.228855in}}%
\pgfusepath{stroke}%
\end{pgfscope}%
\begin{pgfscope}%
\pgfsetbuttcap%
\pgfsetroundjoin%
\definecolor{currentfill}{rgb}{0.000000,0.000000,0.000000}%
\pgfsetfillcolor{currentfill}%
\pgfsetlinewidth{0.602250pt}%
\definecolor{currentstroke}{rgb}{0.000000,0.000000,0.000000}%
\pgfsetstrokecolor{currentstroke}%
\pgfsetdash{}{0pt}%
\pgfsys@defobject{currentmarker}{\pgfqpoint{-0.027778in}{0.000000in}}{\pgfqpoint{-0.000000in}{0.000000in}}{%
\pgfpathmoveto{\pgfqpoint{-0.000000in}{0.000000in}}%
\pgfpathlineto{\pgfqpoint{-0.027778in}{0.000000in}}%
\pgfusepath{stroke,fill}%
}%
\begin{pgfscope}%
\pgfsys@transformshift{0.536250in}{2.228855in}%
\pgfsys@useobject{currentmarker}{}%
\end{pgfscope}%
\end{pgfscope}%
\begin{pgfscope}%
\definecolor{textcolor}{rgb}{0.000000,0.000000,0.000000}%
\pgfsetstrokecolor{textcolor}%
\pgfsetfillcolor{textcolor}%
\pgftext[x=0.197131in,y=1.396875in,,bottom,rotate=90.000000]{\color{textcolor}\rmfamily\fontsize{9.000000}{10.800000}\selectfont Stopping power [MeV cm\(\displaystyle ^{2}\) g\(\displaystyle ^{-1}\)]}%
\end{pgfscope}%
\begin{pgfscope}%
\pgfpathrectangle{\pgfqpoint{0.536250in}{0.525000in}}{\pgfqpoint{1.310000in}{1.743750in}}%
\pgfusepath{clip}%
\pgfsetbuttcap%
\pgfsetroundjoin%
\pgfsetlinewidth{1.003750pt}%
\definecolor{currentstroke}{rgb}{0.000000,0.000000,0.000000}%
\pgfsetstrokecolor{currentstroke}%
\pgfsetdash{{3.700000pt}{1.600000pt}}{0.000000pt}%
\pgfpathmoveto{\pgfqpoint{0.569541in}{1.846883in}}%
\pgfpathlineto{\pgfqpoint{0.653180in}{1.175157in}}%
\pgfpathlineto{\pgfqpoint{0.672594in}{1.041498in}}%
\pgfpathlineto{\pgfqpoint{0.688518in}{0.946876in}}%
\pgfpathlineto{\pgfqpoint{0.700213in}{0.887968in}}%
\pgfpathlineto{\pgfqpoint{0.713556in}{0.832059in}}%
\pgfpathlineto{\pgfqpoint{0.724081in}{0.797180in}}%
\pgfpathlineto{\pgfqpoint{0.735484in}{0.767839in}}%
\pgfpathlineto{\pgfqpoint{0.740364in}{0.757736in}}%
\pgfpathlineto{\pgfqpoint{0.749080in}{0.743548in}}%
\pgfpathlineto{\pgfqpoint{0.756651in}{0.734075in}}%
\pgfpathlineto{\pgfqpoint{0.766459in}{0.725700in}}%
\pgfpathlineto{\pgfqpoint{0.774929in}{0.720990in}}%
\pgfpathlineto{\pgfqpoint{0.786946in}{0.718046in}}%
\pgfpathlineto{\pgfqpoint{0.797096in}{0.718046in}}%
\pgfpathlineto{\pgfqpoint{0.813670in}{0.721667in}}%
\pgfpathlineto{\pgfqpoint{0.832797in}{0.729028in}}%
\pgfpathlineto{\pgfqpoint{0.882463in}{0.754651in}}%
\pgfpathlineto{\pgfqpoint{0.944302in}{0.785753in}}%
\pgfpathlineto{\pgfqpoint{0.979684in}{0.801218in}}%
\pgfpathlineto{\pgfqpoint{1.020832in}{0.816952in}}%
\pgfpathlineto{\pgfqpoint{1.065695in}{0.831891in}}%
\pgfpathlineto{\pgfqpoint{1.105889in}{0.843497in}}%
\pgfpathlineto{\pgfqpoint{1.204275in}{0.867383in}}%
\pgfpathlineto{\pgfqpoint{1.846250in}{1.002269in}}%
\pgfpathlineto{\pgfqpoint{1.846250in}{1.002269in}}%
\pgfusepath{stroke}%
\end{pgfscope}%
\begin{pgfscope}%
\pgfpathrectangle{\pgfqpoint{0.536250in}{0.525000in}}{\pgfqpoint{1.310000in}{1.743750in}}%
\pgfusepath{clip}%
\pgfsetbuttcap%
\pgfsetroundjoin%
\pgfsetlinewidth{1.003750pt}%
\definecolor{currentstroke}{rgb}{0.000000,0.000000,0.000000}%
\pgfsetstrokecolor{currentstroke}%
\pgfsetdash{{3.700000pt}{1.600000pt}}{0.000000pt}%
\pgfpathmoveto{\pgfqpoint{1.252631in}{0.515000in}}%
\pgfpathlineto{\pgfqpoint{1.329645in}{0.967316in}}%
\pgfpathlineto{\pgfqpoint{1.404294in}{1.391421in}}%
\pgfpathlineto{\pgfqpoint{1.531716in}{2.093680in}}%
\pgfpathlineto{\pgfqpoint{1.565700in}{2.278750in}}%
\pgfpathlineto{\pgfqpoint{1.565700in}{2.278750in}}%
\pgfusepath{stroke}%
\end{pgfscope}%
\begin{pgfscope}%
\pgfpathrectangle{\pgfqpoint{0.536250in}{0.525000in}}{\pgfqpoint{1.310000in}{1.743750in}}%
\pgfusepath{clip}%
\pgfsetrectcap%
\pgfsetroundjoin%
\pgfsetlinewidth{1.003750pt}%
\definecolor{currentstroke}{rgb}{0.000000,0.000000,0.000000}%
\pgfsetstrokecolor{currentstroke}%
\pgfsetdash{}{0pt}%
\pgfpathmoveto{\pgfqpoint{0.569541in}{1.846883in}}%
\pgfpathlineto{\pgfqpoint{0.653180in}{1.175157in}}%
\pgfpathlineto{\pgfqpoint{0.672594in}{1.041498in}}%
\pgfpathlineto{\pgfqpoint{0.688518in}{0.946876in}}%
\pgfpathlineto{\pgfqpoint{0.700213in}{0.887968in}}%
\pgfpathlineto{\pgfqpoint{0.713556in}{0.832227in}}%
\pgfpathlineto{\pgfqpoint{0.724081in}{0.797180in}}%
\pgfpathlineto{\pgfqpoint{0.735484in}{0.767839in}}%
\pgfpathlineto{\pgfqpoint{0.740364in}{0.757736in}}%
\pgfpathlineto{\pgfqpoint{0.749080in}{0.743548in}}%
\pgfpathlineto{\pgfqpoint{0.756651in}{0.734075in}}%
\pgfpathlineto{\pgfqpoint{0.766459in}{0.725700in}}%
\pgfpathlineto{\pgfqpoint{0.774929in}{0.720990in}}%
\pgfpathlineto{\pgfqpoint{0.786946in}{0.718046in}}%
\pgfpathlineto{\pgfqpoint{0.797096in}{0.718273in}}%
\pgfpathlineto{\pgfqpoint{0.813670in}{0.721667in}}%
\pgfpathlineto{\pgfqpoint{0.832797in}{0.729249in}}%
\pgfpathlineto{\pgfqpoint{0.856218in}{0.740774in}}%
\pgfpathlineto{\pgfqpoint{0.931811in}{0.780583in}}%
\pgfpathlineto{\pgfqpoint{0.992406in}{0.808451in}}%
\pgfpathlineto{\pgfqpoint{1.116814in}{0.862911in}}%
\pgfpathlineto{\pgfqpoint{1.142113in}{0.877517in}}%
\pgfpathlineto{\pgfqpoint{1.165996in}{0.894303in}}%
\pgfpathlineto{\pgfqpoint{1.183844in}{0.909555in}}%
\pgfpathlineto{\pgfqpoint{1.204275in}{0.930626in}}%
\pgfpathlineto{\pgfqpoint{1.215229in}{0.943757in}}%
\pgfpathlineto{\pgfqpoint{1.229028in}{0.962458in}}%
\pgfpathlineto{\pgfqpoint{1.240579in}{0.980279in}}%
\pgfpathlineto{\pgfqpoint{1.256441in}{1.007811in}}%
\pgfpathlineto{\pgfqpoint{1.280361in}{1.057790in}}%
\pgfpathlineto{\pgfqpoint{1.298229in}{1.102348in}}%
\pgfpathlineto{\pgfqpoint{1.318684in}{1.160529in}}%
\pgfpathlineto{\pgfqpoint{1.339140in}{1.226644in}}%
\pgfpathlineto{\pgfqpoint{1.355000in}{1.283526in}}%
\pgfpathlineto{\pgfqpoint{1.378928in}{1.376535in}}%
\pgfpathlineto{\pgfqpoint{1.404294in}{1.484986in}}%
\pgfpathlineto{\pgfqpoint{1.433129in}{1.616272in}}%
\pgfpathlineto{\pgfqpoint{1.469456in}{1.793044in}}%
\pgfpathlineto{\pgfqpoint{1.511257in}{2.005000in}}%
\pgfpathlineto{\pgfqpoint{1.563634in}{2.278750in}}%
\pgfpathlineto{\pgfqpoint{1.563634in}{2.278750in}}%
\pgfusepath{stroke}%
\end{pgfscope}%
\begin{pgfscope}%
\pgfsetrectcap%
\pgfsetmiterjoin%
\pgfsetlinewidth{1.003750pt}%
\definecolor{currentstroke}{rgb}{0.000000,0.000000,0.000000}%
\pgfsetstrokecolor{currentstroke}%
\pgfsetdash{}{0pt}%
\pgfpathmoveto{\pgfqpoint{0.536250in}{0.525000in}}%
\pgfpathlineto{\pgfqpoint{0.536250in}{2.268750in}}%
\pgfusepath{stroke}%
\end{pgfscope}%
\begin{pgfscope}%
\pgfsetrectcap%
\pgfsetmiterjoin%
\pgfsetlinewidth{1.003750pt}%
\definecolor{currentstroke}{rgb}{0.000000,0.000000,0.000000}%
\pgfsetstrokecolor{currentstroke}%
\pgfsetdash{}{0pt}%
\pgfpathmoveto{\pgfqpoint{1.846250in}{0.525000in}}%
\pgfpathlineto{\pgfqpoint{1.846250in}{2.268750in}}%
\pgfusepath{stroke}%
\end{pgfscope}%
\begin{pgfscope}%
\pgfsetrectcap%
\pgfsetmiterjoin%
\pgfsetlinewidth{1.003750pt}%
\definecolor{currentstroke}{rgb}{0.000000,0.000000,0.000000}%
\pgfsetstrokecolor{currentstroke}%
\pgfsetdash{}{0pt}%
\pgfpathmoveto{\pgfqpoint{0.536250in}{0.525000in}}%
\pgfpathlineto{\pgfqpoint{1.846250in}{0.525000in}}%
\pgfusepath{stroke}%
\end{pgfscope}%
\begin{pgfscope}%
\pgfsetrectcap%
\pgfsetmiterjoin%
\pgfsetlinewidth{1.003750pt}%
\definecolor{currentstroke}{rgb}{0.000000,0.000000,0.000000}%
\pgfsetstrokecolor{currentstroke}%
\pgfsetdash{}{0pt}%
\pgfpathmoveto{\pgfqpoint{0.536250in}{2.268750in}}%
\pgfpathlineto{\pgfqpoint{1.846250in}{2.268750in}}%
\pgfusepath{stroke}%
\end{pgfscope}%
\begin{pgfscope}%
\pgfsetbuttcap%
\pgfsetroundjoin%
\definecolor{currentfill}{rgb}{0.000000,0.000000,0.000000}%
\pgfsetfillcolor{currentfill}%
\pgfsetlinewidth{0.803000pt}%
\definecolor{currentstroke}{rgb}{0.000000,0.000000,0.000000}%
\pgfsetstrokecolor{currentstroke}%
\pgfsetdash{}{0pt}%
\pgfsys@defobject{currentmarker}{\pgfqpoint{0.000000in}{0.000000in}}{\pgfqpoint{0.000000in}{0.048611in}}{%
\pgfpathmoveto{\pgfqpoint{0.000000in}{0.000000in}}%
\pgfpathlineto{\pgfqpoint{0.000000in}{0.048611in}}%
\pgfusepath{stroke,fill}%
}%
\begin{pgfscope}%
\pgfsys@transformshift{0.703915in}{2.268750in}%
\pgfsys@useobject{currentmarker}{}%
\end{pgfscope}%
\end{pgfscope}%
\begin{pgfscope}%
\definecolor{textcolor}{rgb}{0.000000,0.000000,0.000000}%
\pgfsetstrokecolor{textcolor}%
\pgfsetfillcolor{textcolor}%
\pgftext[x=0.703915in,y=2.365972in,,bottom]{\color{textcolor}\rmfamily\fontsize{9.000000}{10.800000}\selectfont \(\displaystyle 10^{0}\)}%
\end{pgfscope}%
\begin{pgfscope}%
\pgfsetbuttcap%
\pgfsetroundjoin%
\definecolor{currentfill}{rgb}{0.000000,0.000000,0.000000}%
\pgfsetfillcolor{currentfill}%
\pgfsetlinewidth{0.803000pt}%
\definecolor{currentstroke}{rgb}{0.000000,0.000000,0.000000}%
\pgfsetstrokecolor{currentstroke}%
\pgfsetdash{}{0pt}%
\pgfsys@defobject{currentmarker}{\pgfqpoint{0.000000in}{0.000000in}}{\pgfqpoint{0.000000in}{0.048611in}}{%
\pgfpathmoveto{\pgfqpoint{0.000000in}{0.000000in}}%
\pgfpathlineto{\pgfqpoint{0.000000in}{0.048611in}}%
\pgfusepath{stroke,fill}%
}%
\begin{pgfscope}%
\pgfsys@transformshift{1.031415in}{2.268750in}%
\pgfsys@useobject{currentmarker}{}%
\end{pgfscope}%
\end{pgfscope}%
\begin{pgfscope}%
\definecolor{textcolor}{rgb}{0.000000,0.000000,0.000000}%
\pgfsetstrokecolor{textcolor}%
\pgfsetfillcolor{textcolor}%
\pgftext[x=1.031415in,y=2.365972in,,bottom]{\color{textcolor}\rmfamily\fontsize{9.000000}{10.800000}\selectfont \(\displaystyle 10^{2}\)}%
\end{pgfscope}%
\begin{pgfscope}%
\pgfsetbuttcap%
\pgfsetroundjoin%
\definecolor{currentfill}{rgb}{0.000000,0.000000,0.000000}%
\pgfsetfillcolor{currentfill}%
\pgfsetlinewidth{0.803000pt}%
\definecolor{currentstroke}{rgb}{0.000000,0.000000,0.000000}%
\pgfsetstrokecolor{currentstroke}%
\pgfsetdash{}{0pt}%
\pgfsys@defobject{currentmarker}{\pgfqpoint{0.000000in}{0.000000in}}{\pgfqpoint{0.000000in}{0.048611in}}{%
\pgfpathmoveto{\pgfqpoint{0.000000in}{0.000000in}}%
\pgfpathlineto{\pgfqpoint{0.000000in}{0.048611in}}%
\pgfusepath{stroke,fill}%
}%
\begin{pgfscope}%
\pgfsys@transformshift{1.358915in}{2.268750in}%
\pgfsys@useobject{currentmarker}{}%
\end{pgfscope}%
\end{pgfscope}%
\begin{pgfscope}%
\definecolor{textcolor}{rgb}{0.000000,0.000000,0.000000}%
\pgfsetstrokecolor{textcolor}%
\pgfsetfillcolor{textcolor}%
\pgftext[x=1.358915in,y=2.365972in,,bottom]{\color{textcolor}\rmfamily\fontsize{9.000000}{10.800000}\selectfont \(\displaystyle 10^{4}\)}%
\end{pgfscope}%
\begin{pgfscope}%
\pgfsetbuttcap%
\pgfsetroundjoin%
\definecolor{currentfill}{rgb}{0.000000,0.000000,0.000000}%
\pgfsetfillcolor{currentfill}%
\pgfsetlinewidth{0.803000pt}%
\definecolor{currentstroke}{rgb}{0.000000,0.000000,0.000000}%
\pgfsetstrokecolor{currentstroke}%
\pgfsetdash{}{0pt}%
\pgfsys@defobject{currentmarker}{\pgfqpoint{0.000000in}{0.000000in}}{\pgfqpoint{0.000000in}{0.048611in}}{%
\pgfpathmoveto{\pgfqpoint{0.000000in}{0.000000in}}%
\pgfpathlineto{\pgfqpoint{0.000000in}{0.048611in}}%
\pgfusepath{stroke,fill}%
}%
\begin{pgfscope}%
\pgfsys@transformshift{1.686415in}{2.268750in}%
\pgfsys@useobject{currentmarker}{}%
\end{pgfscope}%
\end{pgfscope}%
\begin{pgfscope}%
\definecolor{textcolor}{rgb}{0.000000,0.000000,0.000000}%
\pgfsetstrokecolor{textcolor}%
\pgfsetfillcolor{textcolor}%
\pgftext[x=1.686415in,y=2.365972in,,bottom]{\color{textcolor}\rmfamily\fontsize{9.000000}{10.800000}\selectfont \(\displaystyle 10^{6}\)}%
\end{pgfscope}%
\begin{pgfscope}%
\definecolor{textcolor}{rgb}{0.000000,0.000000,0.000000}%
\pgfsetstrokecolor{textcolor}%
\pgfsetfillcolor{textcolor}%
\pgftext[x=1.191250in,y=2.542499in,,base]{\color{textcolor}\rmfamily\fontsize{9.000000}{10.800000}\selectfont \(\displaystyle \beta\gamma\)}%
\end{pgfscope}%
\begin{pgfscope}%
\pgfsetrectcap%
\pgfsetmiterjoin%
\pgfsetlinewidth{1.003750pt}%
\definecolor{currentstroke}{rgb}{0.000000,0.000000,0.000000}%
\pgfsetstrokecolor{currentstroke}%
\pgfsetdash{}{0pt}%
\pgfpathmoveto{\pgfqpoint{0.536250in}{0.525000in}}%
\pgfpathlineto{\pgfqpoint{0.536250in}{2.268750in}}%
\pgfusepath{stroke}%
\end{pgfscope}%
\begin{pgfscope}%
\pgfsetrectcap%
\pgfsetmiterjoin%
\pgfsetlinewidth{1.003750pt}%
\definecolor{currentstroke}{rgb}{0.000000,0.000000,0.000000}%
\pgfsetstrokecolor{currentstroke}%
\pgfsetdash{}{0pt}%
\pgfpathmoveto{\pgfqpoint{1.846250in}{0.525000in}}%
\pgfpathlineto{\pgfqpoint{1.846250in}{2.268750in}}%
\pgfusepath{stroke}%
\end{pgfscope}%
\begin{pgfscope}%
\pgfsetrectcap%
\pgfsetmiterjoin%
\pgfsetlinewidth{1.003750pt}%
\definecolor{currentstroke}{rgb}{0.000000,0.000000,0.000000}%
\pgfsetstrokecolor{currentstroke}%
\pgfsetdash{}{0pt}%
\pgfpathmoveto{\pgfqpoint{0.536250in}{0.525000in}}%
\pgfpathlineto{\pgfqpoint{1.846250in}{0.525000in}}%
\pgfusepath{stroke}%
\end{pgfscope}%
\begin{pgfscope}%
\pgfsetrectcap%
\pgfsetmiterjoin%
\pgfsetlinewidth{1.003750pt}%
\definecolor{currentstroke}{rgb}{0.000000,0.000000,0.000000}%
\pgfsetstrokecolor{currentstroke}%
\pgfsetdash{}{0pt}%
\pgfpathmoveto{\pgfqpoint{0.536250in}{2.268750in}}%
\pgfpathlineto{\pgfqpoint{1.846250in}{2.268750in}}%
\pgfusepath{stroke}%
\end{pgfscope}%
\end{pgfpicture}%
\makeatother%
\endgroup%

  \caption{Muon stopping power in silicon as a function of the particle momentum
    (adapted from~\cite{PDG}). The two dashed lines indicate the ionization losses
    (also shown in figure~\ref{fig:muon_ionization_loss}) and the radiation
    losses. The energy at which the two cross each other is the critical energy,
    which is in this specific case is 580~GeV.}
  \label{fig:muon_stopping_power}
\end{marginfigure}

To first order radiation losses are proportional to the energy of the radiating
particle
\begin{align}\label{eq:rad_losses}
  -\left<\frac{dE}{dx}\right>_\text{rad} \!\!\!\!\!\! \propto \frac{Z^2}{m^2} E
\end{align}
and, since ionization losses only grow logarithmically, the two loss rates, when
seen as function of energy, are bound to cross each other at some point, as shown
in figure~\ref{fig:muon_stopping_power}. The energy $E_c$ at which this
happens is called \emph{critical energy}. Since radiation losses scale as $1/m^2$,
for any given material $E_c$ is different for different particles---and it is
much higher for heavier particles. For reference, the critical energy in silicon
is $\sim 40$~MeV for electrons, $\sim 580$~GeV for muons, and in the tens to
hundreds of TeV for protons. (We do note explicitly that, in any given material,
the critical energies for electrons and positrons are, in general, slightly different,
but this will not bear any real consequence in this context.)

\begin{marginfigure}
  %% Creator: Matplotlib, PGF backend
%%
%% To include the figure in your LaTeX document, write
%%   \input{<filename>.pgf}
%%
%% Make sure the required packages are loaded in your preamble
%%   \usepackage{pgf}
%%
%% Also ensure that all the required font packages are loaded; for instance,
%% the lmodern package is sometimes necessary when using math font.
%%   \usepackage{lmodern}
%%
%% Figures using additional raster images can only be included by \input if
%% they are in the same directory as the main LaTeX file. For loading figures
%% from other directories you can use the `import` package
%%   \usepackage{import}
%%
%% and then include the figures with
%%   \import{<path to file>}{<filename>.pgf}
%%
%% Matplotlib used the following preamble
%%   \usepackage{fontspec}
%%   \setmainfont{DejaVuSerif.ttf}[Path=\detokenize{/usr/share/matplotlib/mpl-data/fonts/ttf/}]
%%   \setsansfont{DejaVuSans.ttf}[Path=\detokenize{/usr/share/matplotlib/mpl-data/fonts/ttf/}]
%%   \setmonofont{DejaVuSansMono.ttf}[Path=\detokenize{/usr/share/matplotlib/mpl-data/fonts/ttf/}]
%%
\begingroup%
\makeatletter%
\begin{pgfpicture}%
\pgfpathrectangle{\pgfpointorigin}{\pgfqpoint{1.950000in}{2.500000in}}%
\pgfusepath{use as bounding box, clip}%
\begin{pgfscope}%
\pgfsetbuttcap%
\pgfsetmiterjoin%
\definecolor{currentfill}{rgb}{1.000000,1.000000,1.000000}%
\pgfsetfillcolor{currentfill}%
\pgfsetlinewidth{0.000000pt}%
\definecolor{currentstroke}{rgb}{1.000000,1.000000,1.000000}%
\pgfsetstrokecolor{currentstroke}%
\pgfsetdash{}{0pt}%
\pgfpathmoveto{\pgfqpoint{0.000000in}{0.000000in}}%
\pgfpathlineto{\pgfqpoint{1.950000in}{0.000000in}}%
\pgfpathlineto{\pgfqpoint{1.950000in}{2.500000in}}%
\pgfpathlineto{\pgfqpoint{0.000000in}{2.500000in}}%
\pgfpathlineto{\pgfqpoint{0.000000in}{0.000000in}}%
\pgfpathclose%
\pgfusepath{fill}%
\end{pgfscope}%
\begin{pgfscope}%
\pgfsetbuttcap%
\pgfsetmiterjoin%
\definecolor{currentfill}{rgb}{1.000000,1.000000,1.000000}%
\pgfsetfillcolor{currentfill}%
\pgfsetlinewidth{0.000000pt}%
\definecolor{currentstroke}{rgb}{0.000000,0.000000,0.000000}%
\pgfsetstrokecolor{currentstroke}%
\pgfsetstrokeopacity{0.000000}%
\pgfsetdash{}{0pt}%
\pgfpathmoveto{\pgfqpoint{0.536250in}{0.525000in}}%
\pgfpathlineto{\pgfqpoint{1.846250in}{0.525000in}}%
\pgfpathlineto{\pgfqpoint{1.846250in}{2.412500in}}%
\pgfpathlineto{\pgfqpoint{0.536250in}{2.412500in}}%
\pgfpathlineto{\pgfqpoint{0.536250in}{0.525000in}}%
\pgfpathclose%
\pgfusepath{fill}%
\end{pgfscope}%
\begin{pgfscope}%
\pgfpathrectangle{\pgfqpoint{0.536250in}{0.525000in}}{\pgfqpoint{1.310000in}{1.887500in}}%
\pgfusepath{clip}%
\pgfsetbuttcap%
\pgfsetroundjoin%
\pgfsetlinewidth{0.803000pt}%
\definecolor{currentstroke}{rgb}{0.752941,0.752941,0.752941}%
\pgfsetstrokecolor{currentstroke}%
\pgfsetdash{{2.960000pt}{1.280000pt}}{0.000000pt}%
\pgfpathmoveto{\pgfqpoint{0.536250in}{0.525000in}}%
\pgfpathlineto{\pgfqpoint{0.536250in}{2.412500in}}%
\pgfusepath{stroke}%
\end{pgfscope}%
\begin{pgfscope}%
\pgfsetbuttcap%
\pgfsetroundjoin%
\definecolor{currentfill}{rgb}{0.000000,0.000000,0.000000}%
\pgfsetfillcolor{currentfill}%
\pgfsetlinewidth{0.803000pt}%
\definecolor{currentstroke}{rgb}{0.000000,0.000000,0.000000}%
\pgfsetstrokecolor{currentstroke}%
\pgfsetdash{}{0pt}%
\pgfsys@defobject{currentmarker}{\pgfqpoint{0.000000in}{-0.048611in}}{\pgfqpoint{0.000000in}{0.000000in}}{%
\pgfpathmoveto{\pgfqpoint{0.000000in}{0.000000in}}%
\pgfpathlineto{\pgfqpoint{0.000000in}{-0.048611in}}%
\pgfusepath{stroke,fill}%
}%
\begin{pgfscope}%
\pgfsys@transformshift{0.536250in}{0.525000in}%
\pgfsys@useobject{currentmarker}{}%
\end{pgfscope}%
\end{pgfscope}%
\begin{pgfscope}%
\definecolor{textcolor}{rgb}{0.000000,0.000000,0.000000}%
\pgfsetstrokecolor{textcolor}%
\pgfsetfillcolor{textcolor}%
\pgftext[x=0.536250in,y=0.427778in,,top]{\color{textcolor}\rmfamily\fontsize{9.000000}{10.800000}\selectfont \(\displaystyle {10^{0}}\)}%
\end{pgfscope}%
\begin{pgfscope}%
\pgfpathrectangle{\pgfqpoint{0.536250in}{0.525000in}}{\pgfqpoint{1.310000in}{1.887500in}}%
\pgfusepath{clip}%
\pgfsetbuttcap%
\pgfsetroundjoin%
\pgfsetlinewidth{0.803000pt}%
\definecolor{currentstroke}{rgb}{0.752941,0.752941,0.752941}%
\pgfsetstrokecolor{currentstroke}%
\pgfsetdash{{2.960000pt}{1.280000pt}}{0.000000pt}%
\pgfpathmoveto{\pgfqpoint{1.192683in}{0.525000in}}%
\pgfpathlineto{\pgfqpoint{1.192683in}{2.412500in}}%
\pgfusepath{stroke}%
\end{pgfscope}%
\begin{pgfscope}%
\pgfsetbuttcap%
\pgfsetroundjoin%
\definecolor{currentfill}{rgb}{0.000000,0.000000,0.000000}%
\pgfsetfillcolor{currentfill}%
\pgfsetlinewidth{0.803000pt}%
\definecolor{currentstroke}{rgb}{0.000000,0.000000,0.000000}%
\pgfsetstrokecolor{currentstroke}%
\pgfsetdash{}{0pt}%
\pgfsys@defobject{currentmarker}{\pgfqpoint{0.000000in}{-0.048611in}}{\pgfqpoint{0.000000in}{0.000000in}}{%
\pgfpathmoveto{\pgfqpoint{0.000000in}{0.000000in}}%
\pgfpathlineto{\pgfqpoint{0.000000in}{-0.048611in}}%
\pgfusepath{stroke,fill}%
}%
\begin{pgfscope}%
\pgfsys@transformshift{1.192683in}{0.525000in}%
\pgfsys@useobject{currentmarker}{}%
\end{pgfscope}%
\end{pgfscope}%
\begin{pgfscope}%
\definecolor{textcolor}{rgb}{0.000000,0.000000,0.000000}%
\pgfsetstrokecolor{textcolor}%
\pgfsetfillcolor{textcolor}%
\pgftext[x=1.192683in,y=0.427778in,,top]{\color{textcolor}\rmfamily\fontsize{9.000000}{10.800000}\selectfont \(\displaystyle {10^{1}}\)}%
\end{pgfscope}%
\begin{pgfscope}%
\pgfpathrectangle{\pgfqpoint{0.536250in}{0.525000in}}{\pgfqpoint{1.310000in}{1.887500in}}%
\pgfusepath{clip}%
\pgfsetbuttcap%
\pgfsetroundjoin%
\pgfsetlinewidth{0.803000pt}%
\definecolor{currentstroke}{rgb}{0.752941,0.752941,0.752941}%
\pgfsetstrokecolor{currentstroke}%
\pgfsetdash{{2.960000pt}{1.280000pt}}{0.000000pt}%
\pgfpathmoveto{\pgfqpoint{0.733856in}{0.525000in}}%
\pgfpathlineto{\pgfqpoint{0.733856in}{2.412500in}}%
\pgfusepath{stroke}%
\end{pgfscope}%
\begin{pgfscope}%
\pgfsetbuttcap%
\pgfsetroundjoin%
\definecolor{currentfill}{rgb}{0.000000,0.000000,0.000000}%
\pgfsetfillcolor{currentfill}%
\pgfsetlinewidth{0.602250pt}%
\definecolor{currentstroke}{rgb}{0.000000,0.000000,0.000000}%
\pgfsetstrokecolor{currentstroke}%
\pgfsetdash{}{0pt}%
\pgfsys@defobject{currentmarker}{\pgfqpoint{0.000000in}{-0.027778in}}{\pgfqpoint{0.000000in}{0.000000in}}{%
\pgfpathmoveto{\pgfqpoint{0.000000in}{0.000000in}}%
\pgfpathlineto{\pgfqpoint{0.000000in}{-0.027778in}}%
\pgfusepath{stroke,fill}%
}%
\begin{pgfscope}%
\pgfsys@transformshift{0.733856in}{0.525000in}%
\pgfsys@useobject{currentmarker}{}%
\end{pgfscope}%
\end{pgfscope}%
\begin{pgfscope}%
\pgfpathrectangle{\pgfqpoint{0.536250in}{0.525000in}}{\pgfqpoint{1.310000in}{1.887500in}}%
\pgfusepath{clip}%
\pgfsetbuttcap%
\pgfsetroundjoin%
\pgfsetlinewidth{0.803000pt}%
\definecolor{currentstroke}{rgb}{0.752941,0.752941,0.752941}%
\pgfsetstrokecolor{currentstroke}%
\pgfsetdash{{2.960000pt}{1.280000pt}}{0.000000pt}%
\pgfpathmoveto{\pgfqpoint{0.849448in}{0.525000in}}%
\pgfpathlineto{\pgfqpoint{0.849448in}{2.412500in}}%
\pgfusepath{stroke}%
\end{pgfscope}%
\begin{pgfscope}%
\pgfsetbuttcap%
\pgfsetroundjoin%
\definecolor{currentfill}{rgb}{0.000000,0.000000,0.000000}%
\pgfsetfillcolor{currentfill}%
\pgfsetlinewidth{0.602250pt}%
\definecolor{currentstroke}{rgb}{0.000000,0.000000,0.000000}%
\pgfsetstrokecolor{currentstroke}%
\pgfsetdash{}{0pt}%
\pgfsys@defobject{currentmarker}{\pgfqpoint{0.000000in}{-0.027778in}}{\pgfqpoint{0.000000in}{0.000000in}}{%
\pgfpathmoveto{\pgfqpoint{0.000000in}{0.000000in}}%
\pgfpathlineto{\pgfqpoint{0.000000in}{-0.027778in}}%
\pgfusepath{stroke,fill}%
}%
\begin{pgfscope}%
\pgfsys@transformshift{0.849448in}{0.525000in}%
\pgfsys@useobject{currentmarker}{}%
\end{pgfscope}%
\end{pgfscope}%
\begin{pgfscope}%
\pgfpathrectangle{\pgfqpoint{0.536250in}{0.525000in}}{\pgfqpoint{1.310000in}{1.887500in}}%
\pgfusepath{clip}%
\pgfsetbuttcap%
\pgfsetroundjoin%
\pgfsetlinewidth{0.803000pt}%
\definecolor{currentstroke}{rgb}{0.752941,0.752941,0.752941}%
\pgfsetstrokecolor{currentstroke}%
\pgfsetdash{{2.960000pt}{1.280000pt}}{0.000000pt}%
\pgfpathmoveto{\pgfqpoint{0.931462in}{0.525000in}}%
\pgfpathlineto{\pgfqpoint{0.931462in}{2.412500in}}%
\pgfusepath{stroke}%
\end{pgfscope}%
\begin{pgfscope}%
\pgfsetbuttcap%
\pgfsetroundjoin%
\definecolor{currentfill}{rgb}{0.000000,0.000000,0.000000}%
\pgfsetfillcolor{currentfill}%
\pgfsetlinewidth{0.602250pt}%
\definecolor{currentstroke}{rgb}{0.000000,0.000000,0.000000}%
\pgfsetstrokecolor{currentstroke}%
\pgfsetdash{}{0pt}%
\pgfsys@defobject{currentmarker}{\pgfqpoint{0.000000in}{-0.027778in}}{\pgfqpoint{0.000000in}{0.000000in}}{%
\pgfpathmoveto{\pgfqpoint{0.000000in}{0.000000in}}%
\pgfpathlineto{\pgfqpoint{0.000000in}{-0.027778in}}%
\pgfusepath{stroke,fill}%
}%
\begin{pgfscope}%
\pgfsys@transformshift{0.931462in}{0.525000in}%
\pgfsys@useobject{currentmarker}{}%
\end{pgfscope}%
\end{pgfscope}%
\begin{pgfscope}%
\pgfpathrectangle{\pgfqpoint{0.536250in}{0.525000in}}{\pgfqpoint{1.310000in}{1.887500in}}%
\pgfusepath{clip}%
\pgfsetbuttcap%
\pgfsetroundjoin%
\pgfsetlinewidth{0.803000pt}%
\definecolor{currentstroke}{rgb}{0.752941,0.752941,0.752941}%
\pgfsetstrokecolor{currentstroke}%
\pgfsetdash{{2.960000pt}{1.280000pt}}{0.000000pt}%
\pgfpathmoveto{\pgfqpoint{0.995077in}{0.525000in}}%
\pgfpathlineto{\pgfqpoint{0.995077in}{2.412500in}}%
\pgfusepath{stroke}%
\end{pgfscope}%
\begin{pgfscope}%
\pgfsetbuttcap%
\pgfsetroundjoin%
\definecolor{currentfill}{rgb}{0.000000,0.000000,0.000000}%
\pgfsetfillcolor{currentfill}%
\pgfsetlinewidth{0.602250pt}%
\definecolor{currentstroke}{rgb}{0.000000,0.000000,0.000000}%
\pgfsetstrokecolor{currentstroke}%
\pgfsetdash{}{0pt}%
\pgfsys@defobject{currentmarker}{\pgfqpoint{0.000000in}{-0.027778in}}{\pgfqpoint{0.000000in}{0.000000in}}{%
\pgfpathmoveto{\pgfqpoint{0.000000in}{0.000000in}}%
\pgfpathlineto{\pgfqpoint{0.000000in}{-0.027778in}}%
\pgfusepath{stroke,fill}%
}%
\begin{pgfscope}%
\pgfsys@transformshift{0.995077in}{0.525000in}%
\pgfsys@useobject{currentmarker}{}%
\end{pgfscope}%
\end{pgfscope}%
\begin{pgfscope}%
\pgfpathrectangle{\pgfqpoint{0.536250in}{0.525000in}}{\pgfqpoint{1.310000in}{1.887500in}}%
\pgfusepath{clip}%
\pgfsetbuttcap%
\pgfsetroundjoin%
\pgfsetlinewidth{0.803000pt}%
\definecolor{currentstroke}{rgb}{0.752941,0.752941,0.752941}%
\pgfsetstrokecolor{currentstroke}%
\pgfsetdash{{2.960000pt}{1.280000pt}}{0.000000pt}%
\pgfpathmoveto{\pgfqpoint{1.047054in}{0.525000in}}%
\pgfpathlineto{\pgfqpoint{1.047054in}{2.412500in}}%
\pgfusepath{stroke}%
\end{pgfscope}%
\begin{pgfscope}%
\pgfsetbuttcap%
\pgfsetroundjoin%
\definecolor{currentfill}{rgb}{0.000000,0.000000,0.000000}%
\pgfsetfillcolor{currentfill}%
\pgfsetlinewidth{0.602250pt}%
\definecolor{currentstroke}{rgb}{0.000000,0.000000,0.000000}%
\pgfsetstrokecolor{currentstroke}%
\pgfsetdash{}{0pt}%
\pgfsys@defobject{currentmarker}{\pgfqpoint{0.000000in}{-0.027778in}}{\pgfqpoint{0.000000in}{0.000000in}}{%
\pgfpathmoveto{\pgfqpoint{0.000000in}{0.000000in}}%
\pgfpathlineto{\pgfqpoint{0.000000in}{-0.027778in}}%
\pgfusepath{stroke,fill}%
}%
\begin{pgfscope}%
\pgfsys@transformshift{1.047054in}{0.525000in}%
\pgfsys@useobject{currentmarker}{}%
\end{pgfscope}%
\end{pgfscope}%
\begin{pgfscope}%
\pgfpathrectangle{\pgfqpoint{0.536250in}{0.525000in}}{\pgfqpoint{1.310000in}{1.887500in}}%
\pgfusepath{clip}%
\pgfsetbuttcap%
\pgfsetroundjoin%
\pgfsetlinewidth{0.803000pt}%
\definecolor{currentstroke}{rgb}{0.752941,0.752941,0.752941}%
\pgfsetstrokecolor{currentstroke}%
\pgfsetdash{{2.960000pt}{1.280000pt}}{0.000000pt}%
\pgfpathmoveto{\pgfqpoint{1.091000in}{0.525000in}}%
\pgfpathlineto{\pgfqpoint{1.091000in}{2.412500in}}%
\pgfusepath{stroke}%
\end{pgfscope}%
\begin{pgfscope}%
\pgfsetbuttcap%
\pgfsetroundjoin%
\definecolor{currentfill}{rgb}{0.000000,0.000000,0.000000}%
\pgfsetfillcolor{currentfill}%
\pgfsetlinewidth{0.602250pt}%
\definecolor{currentstroke}{rgb}{0.000000,0.000000,0.000000}%
\pgfsetstrokecolor{currentstroke}%
\pgfsetdash{}{0pt}%
\pgfsys@defobject{currentmarker}{\pgfqpoint{0.000000in}{-0.027778in}}{\pgfqpoint{0.000000in}{0.000000in}}{%
\pgfpathmoveto{\pgfqpoint{0.000000in}{0.000000in}}%
\pgfpathlineto{\pgfqpoint{0.000000in}{-0.027778in}}%
\pgfusepath{stroke,fill}%
}%
\begin{pgfscope}%
\pgfsys@transformshift{1.091000in}{0.525000in}%
\pgfsys@useobject{currentmarker}{}%
\end{pgfscope}%
\end{pgfscope}%
\begin{pgfscope}%
\pgfpathrectangle{\pgfqpoint{0.536250in}{0.525000in}}{\pgfqpoint{1.310000in}{1.887500in}}%
\pgfusepath{clip}%
\pgfsetbuttcap%
\pgfsetroundjoin%
\pgfsetlinewidth{0.803000pt}%
\definecolor{currentstroke}{rgb}{0.752941,0.752941,0.752941}%
\pgfsetstrokecolor{currentstroke}%
\pgfsetdash{{2.960000pt}{1.280000pt}}{0.000000pt}%
\pgfpathmoveto{\pgfqpoint{1.129068in}{0.525000in}}%
\pgfpathlineto{\pgfqpoint{1.129068in}{2.412500in}}%
\pgfusepath{stroke}%
\end{pgfscope}%
\begin{pgfscope}%
\pgfsetbuttcap%
\pgfsetroundjoin%
\definecolor{currentfill}{rgb}{0.000000,0.000000,0.000000}%
\pgfsetfillcolor{currentfill}%
\pgfsetlinewidth{0.602250pt}%
\definecolor{currentstroke}{rgb}{0.000000,0.000000,0.000000}%
\pgfsetstrokecolor{currentstroke}%
\pgfsetdash{}{0pt}%
\pgfsys@defobject{currentmarker}{\pgfqpoint{0.000000in}{-0.027778in}}{\pgfqpoint{0.000000in}{0.000000in}}{%
\pgfpathmoveto{\pgfqpoint{0.000000in}{0.000000in}}%
\pgfpathlineto{\pgfqpoint{0.000000in}{-0.027778in}}%
\pgfusepath{stroke,fill}%
}%
\begin{pgfscope}%
\pgfsys@transformshift{1.129068in}{0.525000in}%
\pgfsys@useobject{currentmarker}{}%
\end{pgfscope}%
\end{pgfscope}%
\begin{pgfscope}%
\pgfpathrectangle{\pgfqpoint{0.536250in}{0.525000in}}{\pgfqpoint{1.310000in}{1.887500in}}%
\pgfusepath{clip}%
\pgfsetbuttcap%
\pgfsetroundjoin%
\pgfsetlinewidth{0.803000pt}%
\definecolor{currentstroke}{rgb}{0.752941,0.752941,0.752941}%
\pgfsetstrokecolor{currentstroke}%
\pgfsetdash{{2.960000pt}{1.280000pt}}{0.000000pt}%
\pgfpathmoveto{\pgfqpoint{1.162646in}{0.525000in}}%
\pgfpathlineto{\pgfqpoint{1.162646in}{2.412500in}}%
\pgfusepath{stroke}%
\end{pgfscope}%
\begin{pgfscope}%
\pgfsetbuttcap%
\pgfsetroundjoin%
\definecolor{currentfill}{rgb}{0.000000,0.000000,0.000000}%
\pgfsetfillcolor{currentfill}%
\pgfsetlinewidth{0.602250pt}%
\definecolor{currentstroke}{rgb}{0.000000,0.000000,0.000000}%
\pgfsetstrokecolor{currentstroke}%
\pgfsetdash{}{0pt}%
\pgfsys@defobject{currentmarker}{\pgfqpoint{0.000000in}{-0.027778in}}{\pgfqpoint{0.000000in}{0.000000in}}{%
\pgfpathmoveto{\pgfqpoint{0.000000in}{0.000000in}}%
\pgfpathlineto{\pgfqpoint{0.000000in}{-0.027778in}}%
\pgfusepath{stroke,fill}%
}%
\begin{pgfscope}%
\pgfsys@transformshift{1.162646in}{0.525000in}%
\pgfsys@useobject{currentmarker}{}%
\end{pgfscope}%
\end{pgfscope}%
\begin{pgfscope}%
\pgfpathrectangle{\pgfqpoint{0.536250in}{0.525000in}}{\pgfqpoint{1.310000in}{1.887500in}}%
\pgfusepath{clip}%
\pgfsetbuttcap%
\pgfsetroundjoin%
\pgfsetlinewidth{0.803000pt}%
\definecolor{currentstroke}{rgb}{0.752941,0.752941,0.752941}%
\pgfsetstrokecolor{currentstroke}%
\pgfsetdash{{2.960000pt}{1.280000pt}}{0.000000pt}%
\pgfpathmoveto{\pgfqpoint{1.390289in}{0.525000in}}%
\pgfpathlineto{\pgfqpoint{1.390289in}{2.412500in}}%
\pgfusepath{stroke}%
\end{pgfscope}%
\begin{pgfscope}%
\pgfsetbuttcap%
\pgfsetroundjoin%
\definecolor{currentfill}{rgb}{0.000000,0.000000,0.000000}%
\pgfsetfillcolor{currentfill}%
\pgfsetlinewidth{0.602250pt}%
\definecolor{currentstroke}{rgb}{0.000000,0.000000,0.000000}%
\pgfsetstrokecolor{currentstroke}%
\pgfsetdash{}{0pt}%
\pgfsys@defobject{currentmarker}{\pgfqpoint{0.000000in}{-0.027778in}}{\pgfqpoint{0.000000in}{0.000000in}}{%
\pgfpathmoveto{\pgfqpoint{0.000000in}{0.000000in}}%
\pgfpathlineto{\pgfqpoint{0.000000in}{-0.027778in}}%
\pgfusepath{stroke,fill}%
}%
\begin{pgfscope}%
\pgfsys@transformshift{1.390289in}{0.525000in}%
\pgfsys@useobject{currentmarker}{}%
\end{pgfscope}%
\end{pgfscope}%
\begin{pgfscope}%
\pgfpathrectangle{\pgfqpoint{0.536250in}{0.525000in}}{\pgfqpoint{1.310000in}{1.887500in}}%
\pgfusepath{clip}%
\pgfsetbuttcap%
\pgfsetroundjoin%
\pgfsetlinewidth{0.803000pt}%
\definecolor{currentstroke}{rgb}{0.752941,0.752941,0.752941}%
\pgfsetstrokecolor{currentstroke}%
\pgfsetdash{{2.960000pt}{1.280000pt}}{0.000000pt}%
\pgfpathmoveto{\pgfqpoint{1.505881in}{0.525000in}}%
\pgfpathlineto{\pgfqpoint{1.505881in}{2.412500in}}%
\pgfusepath{stroke}%
\end{pgfscope}%
\begin{pgfscope}%
\pgfsetbuttcap%
\pgfsetroundjoin%
\definecolor{currentfill}{rgb}{0.000000,0.000000,0.000000}%
\pgfsetfillcolor{currentfill}%
\pgfsetlinewidth{0.602250pt}%
\definecolor{currentstroke}{rgb}{0.000000,0.000000,0.000000}%
\pgfsetstrokecolor{currentstroke}%
\pgfsetdash{}{0pt}%
\pgfsys@defobject{currentmarker}{\pgfqpoint{0.000000in}{-0.027778in}}{\pgfqpoint{0.000000in}{0.000000in}}{%
\pgfpathmoveto{\pgfqpoint{0.000000in}{0.000000in}}%
\pgfpathlineto{\pgfqpoint{0.000000in}{-0.027778in}}%
\pgfusepath{stroke,fill}%
}%
\begin{pgfscope}%
\pgfsys@transformshift{1.505881in}{0.525000in}%
\pgfsys@useobject{currentmarker}{}%
\end{pgfscope}%
\end{pgfscope}%
\begin{pgfscope}%
\pgfpathrectangle{\pgfqpoint{0.536250in}{0.525000in}}{\pgfqpoint{1.310000in}{1.887500in}}%
\pgfusepath{clip}%
\pgfsetbuttcap%
\pgfsetroundjoin%
\pgfsetlinewidth{0.803000pt}%
\definecolor{currentstroke}{rgb}{0.752941,0.752941,0.752941}%
\pgfsetstrokecolor{currentstroke}%
\pgfsetdash{{2.960000pt}{1.280000pt}}{0.000000pt}%
\pgfpathmoveto{\pgfqpoint{1.587894in}{0.525000in}}%
\pgfpathlineto{\pgfqpoint{1.587894in}{2.412500in}}%
\pgfusepath{stroke}%
\end{pgfscope}%
\begin{pgfscope}%
\pgfsetbuttcap%
\pgfsetroundjoin%
\definecolor{currentfill}{rgb}{0.000000,0.000000,0.000000}%
\pgfsetfillcolor{currentfill}%
\pgfsetlinewidth{0.602250pt}%
\definecolor{currentstroke}{rgb}{0.000000,0.000000,0.000000}%
\pgfsetstrokecolor{currentstroke}%
\pgfsetdash{}{0pt}%
\pgfsys@defobject{currentmarker}{\pgfqpoint{0.000000in}{-0.027778in}}{\pgfqpoint{0.000000in}{0.000000in}}{%
\pgfpathmoveto{\pgfqpoint{0.000000in}{0.000000in}}%
\pgfpathlineto{\pgfqpoint{0.000000in}{-0.027778in}}%
\pgfusepath{stroke,fill}%
}%
\begin{pgfscope}%
\pgfsys@transformshift{1.587894in}{0.525000in}%
\pgfsys@useobject{currentmarker}{}%
\end{pgfscope}%
\end{pgfscope}%
\begin{pgfscope}%
\pgfpathrectangle{\pgfqpoint{0.536250in}{0.525000in}}{\pgfqpoint{1.310000in}{1.887500in}}%
\pgfusepath{clip}%
\pgfsetbuttcap%
\pgfsetroundjoin%
\pgfsetlinewidth{0.803000pt}%
\definecolor{currentstroke}{rgb}{0.752941,0.752941,0.752941}%
\pgfsetstrokecolor{currentstroke}%
\pgfsetdash{{2.960000pt}{1.280000pt}}{0.000000pt}%
\pgfpathmoveto{\pgfqpoint{1.651509in}{0.525000in}}%
\pgfpathlineto{\pgfqpoint{1.651509in}{2.412500in}}%
\pgfusepath{stroke}%
\end{pgfscope}%
\begin{pgfscope}%
\pgfsetbuttcap%
\pgfsetroundjoin%
\definecolor{currentfill}{rgb}{0.000000,0.000000,0.000000}%
\pgfsetfillcolor{currentfill}%
\pgfsetlinewidth{0.602250pt}%
\definecolor{currentstroke}{rgb}{0.000000,0.000000,0.000000}%
\pgfsetstrokecolor{currentstroke}%
\pgfsetdash{}{0pt}%
\pgfsys@defobject{currentmarker}{\pgfqpoint{0.000000in}{-0.027778in}}{\pgfqpoint{0.000000in}{0.000000in}}{%
\pgfpathmoveto{\pgfqpoint{0.000000in}{0.000000in}}%
\pgfpathlineto{\pgfqpoint{0.000000in}{-0.027778in}}%
\pgfusepath{stroke,fill}%
}%
\begin{pgfscope}%
\pgfsys@transformshift{1.651509in}{0.525000in}%
\pgfsys@useobject{currentmarker}{}%
\end{pgfscope}%
\end{pgfscope}%
\begin{pgfscope}%
\pgfpathrectangle{\pgfqpoint{0.536250in}{0.525000in}}{\pgfqpoint{1.310000in}{1.887500in}}%
\pgfusepath{clip}%
\pgfsetbuttcap%
\pgfsetroundjoin%
\pgfsetlinewidth{0.803000pt}%
\definecolor{currentstroke}{rgb}{0.752941,0.752941,0.752941}%
\pgfsetstrokecolor{currentstroke}%
\pgfsetdash{{2.960000pt}{1.280000pt}}{0.000000pt}%
\pgfpathmoveto{\pgfqpoint{1.703486in}{0.525000in}}%
\pgfpathlineto{\pgfqpoint{1.703486in}{2.412500in}}%
\pgfusepath{stroke}%
\end{pgfscope}%
\begin{pgfscope}%
\pgfsetbuttcap%
\pgfsetroundjoin%
\definecolor{currentfill}{rgb}{0.000000,0.000000,0.000000}%
\pgfsetfillcolor{currentfill}%
\pgfsetlinewidth{0.602250pt}%
\definecolor{currentstroke}{rgb}{0.000000,0.000000,0.000000}%
\pgfsetstrokecolor{currentstroke}%
\pgfsetdash{}{0pt}%
\pgfsys@defobject{currentmarker}{\pgfqpoint{0.000000in}{-0.027778in}}{\pgfqpoint{0.000000in}{0.000000in}}{%
\pgfpathmoveto{\pgfqpoint{0.000000in}{0.000000in}}%
\pgfpathlineto{\pgfqpoint{0.000000in}{-0.027778in}}%
\pgfusepath{stroke,fill}%
}%
\begin{pgfscope}%
\pgfsys@transformshift{1.703486in}{0.525000in}%
\pgfsys@useobject{currentmarker}{}%
\end{pgfscope}%
\end{pgfscope}%
\begin{pgfscope}%
\pgfpathrectangle{\pgfqpoint{0.536250in}{0.525000in}}{\pgfqpoint{1.310000in}{1.887500in}}%
\pgfusepath{clip}%
\pgfsetbuttcap%
\pgfsetroundjoin%
\pgfsetlinewidth{0.803000pt}%
\definecolor{currentstroke}{rgb}{0.752941,0.752941,0.752941}%
\pgfsetstrokecolor{currentstroke}%
\pgfsetdash{{2.960000pt}{1.280000pt}}{0.000000pt}%
\pgfpathmoveto{\pgfqpoint{1.747433in}{0.525000in}}%
\pgfpathlineto{\pgfqpoint{1.747433in}{2.412500in}}%
\pgfusepath{stroke}%
\end{pgfscope}%
\begin{pgfscope}%
\pgfsetbuttcap%
\pgfsetroundjoin%
\definecolor{currentfill}{rgb}{0.000000,0.000000,0.000000}%
\pgfsetfillcolor{currentfill}%
\pgfsetlinewidth{0.602250pt}%
\definecolor{currentstroke}{rgb}{0.000000,0.000000,0.000000}%
\pgfsetstrokecolor{currentstroke}%
\pgfsetdash{}{0pt}%
\pgfsys@defobject{currentmarker}{\pgfqpoint{0.000000in}{-0.027778in}}{\pgfqpoint{0.000000in}{0.000000in}}{%
\pgfpathmoveto{\pgfqpoint{0.000000in}{0.000000in}}%
\pgfpathlineto{\pgfqpoint{0.000000in}{-0.027778in}}%
\pgfusepath{stroke,fill}%
}%
\begin{pgfscope}%
\pgfsys@transformshift{1.747433in}{0.525000in}%
\pgfsys@useobject{currentmarker}{}%
\end{pgfscope}%
\end{pgfscope}%
\begin{pgfscope}%
\pgfpathrectangle{\pgfqpoint{0.536250in}{0.525000in}}{\pgfqpoint{1.310000in}{1.887500in}}%
\pgfusepath{clip}%
\pgfsetbuttcap%
\pgfsetroundjoin%
\pgfsetlinewidth{0.803000pt}%
\definecolor{currentstroke}{rgb}{0.752941,0.752941,0.752941}%
\pgfsetstrokecolor{currentstroke}%
\pgfsetdash{{2.960000pt}{1.280000pt}}{0.000000pt}%
\pgfpathmoveto{\pgfqpoint{1.785500in}{0.525000in}}%
\pgfpathlineto{\pgfqpoint{1.785500in}{2.412500in}}%
\pgfusepath{stroke}%
\end{pgfscope}%
\begin{pgfscope}%
\pgfsetbuttcap%
\pgfsetroundjoin%
\definecolor{currentfill}{rgb}{0.000000,0.000000,0.000000}%
\pgfsetfillcolor{currentfill}%
\pgfsetlinewidth{0.602250pt}%
\definecolor{currentstroke}{rgb}{0.000000,0.000000,0.000000}%
\pgfsetstrokecolor{currentstroke}%
\pgfsetdash{}{0pt}%
\pgfsys@defobject{currentmarker}{\pgfqpoint{0.000000in}{-0.027778in}}{\pgfqpoint{0.000000in}{0.000000in}}{%
\pgfpathmoveto{\pgfqpoint{0.000000in}{0.000000in}}%
\pgfpathlineto{\pgfqpoint{0.000000in}{-0.027778in}}%
\pgfusepath{stroke,fill}%
}%
\begin{pgfscope}%
\pgfsys@transformshift{1.785500in}{0.525000in}%
\pgfsys@useobject{currentmarker}{}%
\end{pgfscope}%
\end{pgfscope}%
\begin{pgfscope}%
\pgfpathrectangle{\pgfqpoint{0.536250in}{0.525000in}}{\pgfqpoint{1.310000in}{1.887500in}}%
\pgfusepath{clip}%
\pgfsetbuttcap%
\pgfsetroundjoin%
\pgfsetlinewidth{0.803000pt}%
\definecolor{currentstroke}{rgb}{0.752941,0.752941,0.752941}%
\pgfsetstrokecolor{currentstroke}%
\pgfsetdash{{2.960000pt}{1.280000pt}}{0.000000pt}%
\pgfpathmoveto{\pgfqpoint{1.819078in}{0.525000in}}%
\pgfpathlineto{\pgfqpoint{1.819078in}{2.412500in}}%
\pgfusepath{stroke}%
\end{pgfscope}%
\begin{pgfscope}%
\pgfsetbuttcap%
\pgfsetroundjoin%
\definecolor{currentfill}{rgb}{0.000000,0.000000,0.000000}%
\pgfsetfillcolor{currentfill}%
\pgfsetlinewidth{0.602250pt}%
\definecolor{currentstroke}{rgb}{0.000000,0.000000,0.000000}%
\pgfsetstrokecolor{currentstroke}%
\pgfsetdash{}{0pt}%
\pgfsys@defobject{currentmarker}{\pgfqpoint{0.000000in}{-0.027778in}}{\pgfqpoint{0.000000in}{0.000000in}}{%
\pgfpathmoveto{\pgfqpoint{0.000000in}{0.000000in}}%
\pgfpathlineto{\pgfqpoint{0.000000in}{-0.027778in}}%
\pgfusepath{stroke,fill}%
}%
\begin{pgfscope}%
\pgfsys@transformshift{1.819078in}{0.525000in}%
\pgfsys@useobject{currentmarker}{}%
\end{pgfscope}%
\end{pgfscope}%
\begin{pgfscope}%
\definecolor{textcolor}{rgb}{0.000000,0.000000,0.000000}%
\pgfsetstrokecolor{textcolor}%
\pgfsetfillcolor{textcolor}%
\pgftext[x=1.191250in,y=0.251251in,,top]{\color{textcolor}\rmfamily\fontsize{9.000000}{10.800000}\selectfont Atomic number}%
\end{pgfscope}%
\begin{pgfscope}%
\pgfpathrectangle{\pgfqpoint{0.536250in}{0.525000in}}{\pgfqpoint{1.310000in}{1.887500in}}%
\pgfusepath{clip}%
\pgfsetbuttcap%
\pgfsetroundjoin%
\pgfsetlinewidth{0.803000pt}%
\definecolor{currentstroke}{rgb}{0.752941,0.752941,0.752941}%
\pgfsetstrokecolor{currentstroke}%
\pgfsetdash{{2.960000pt}{1.280000pt}}{0.000000pt}%
\pgfpathmoveto{\pgfqpoint{0.536250in}{0.818312in}}%
\pgfpathlineto{\pgfqpoint{1.846250in}{0.818312in}}%
\pgfusepath{stroke}%
\end{pgfscope}%
\begin{pgfscope}%
\pgfsetbuttcap%
\pgfsetroundjoin%
\definecolor{currentfill}{rgb}{0.000000,0.000000,0.000000}%
\pgfsetfillcolor{currentfill}%
\pgfsetlinewidth{0.803000pt}%
\definecolor{currentstroke}{rgb}{0.000000,0.000000,0.000000}%
\pgfsetstrokecolor{currentstroke}%
\pgfsetdash{}{0pt}%
\pgfsys@defobject{currentmarker}{\pgfqpoint{-0.048611in}{0.000000in}}{\pgfqpoint{-0.000000in}{0.000000in}}{%
\pgfpathmoveto{\pgfqpoint{-0.000000in}{0.000000in}}%
\pgfpathlineto{\pgfqpoint{-0.048611in}{0.000000in}}%
\pgfusepath{stroke,fill}%
}%
\begin{pgfscope}%
\pgfsys@transformshift{0.536250in}{0.818312in}%
\pgfsys@useobject{currentmarker}{}%
\end{pgfscope}%
\end{pgfscope}%
\begin{pgfscope}%
\definecolor{textcolor}{rgb}{0.000000,0.000000,0.000000}%
\pgfsetstrokecolor{textcolor}%
\pgfsetfillcolor{textcolor}%
\pgftext[x=0.252687in, y=0.770827in, left, base]{\color{textcolor}\rmfamily\fontsize{9.000000}{10.800000}\selectfont \(\displaystyle {10^{1}}\)}%
\end{pgfscope}%
\begin{pgfscope}%
\pgfpathrectangle{\pgfqpoint{0.536250in}{0.525000in}}{\pgfqpoint{1.310000in}{1.887500in}}%
\pgfusepath{clip}%
\pgfsetbuttcap%
\pgfsetroundjoin%
\pgfsetlinewidth{0.803000pt}%
\definecolor{currentstroke}{rgb}{0.752941,0.752941,0.752941}%
\pgfsetstrokecolor{currentstroke}%
\pgfsetdash{{2.960000pt}{1.280000pt}}{0.000000pt}%
\pgfpathmoveto{\pgfqpoint{0.536250in}{1.780323in}}%
\pgfpathlineto{\pgfqpoint{1.846250in}{1.780323in}}%
\pgfusepath{stroke}%
\end{pgfscope}%
\begin{pgfscope}%
\pgfsetbuttcap%
\pgfsetroundjoin%
\definecolor{currentfill}{rgb}{0.000000,0.000000,0.000000}%
\pgfsetfillcolor{currentfill}%
\pgfsetlinewidth{0.803000pt}%
\definecolor{currentstroke}{rgb}{0.000000,0.000000,0.000000}%
\pgfsetstrokecolor{currentstroke}%
\pgfsetdash{}{0pt}%
\pgfsys@defobject{currentmarker}{\pgfqpoint{-0.048611in}{0.000000in}}{\pgfqpoint{-0.000000in}{0.000000in}}{%
\pgfpathmoveto{\pgfqpoint{-0.000000in}{0.000000in}}%
\pgfpathlineto{\pgfqpoint{-0.048611in}{0.000000in}}%
\pgfusepath{stroke,fill}%
}%
\begin{pgfscope}%
\pgfsys@transformshift{0.536250in}{1.780323in}%
\pgfsys@useobject{currentmarker}{}%
\end{pgfscope}%
\end{pgfscope}%
\begin{pgfscope}%
\definecolor{textcolor}{rgb}{0.000000,0.000000,0.000000}%
\pgfsetstrokecolor{textcolor}%
\pgfsetfillcolor{textcolor}%
\pgftext[x=0.252687in, y=1.732838in, left, base]{\color{textcolor}\rmfamily\fontsize{9.000000}{10.800000}\selectfont \(\displaystyle {10^{2}}\)}%
\end{pgfscope}%
\begin{pgfscope}%
\pgfpathrectangle{\pgfqpoint{0.536250in}{0.525000in}}{\pgfqpoint{1.310000in}{1.887500in}}%
\pgfusepath{clip}%
\pgfsetbuttcap%
\pgfsetroundjoin%
\pgfsetlinewidth{0.803000pt}%
\definecolor{currentstroke}{rgb}{0.752941,0.752941,0.752941}%
\pgfsetstrokecolor{currentstroke}%
\pgfsetdash{{2.960000pt}{1.280000pt}}{0.000000pt}%
\pgfpathmoveto{\pgfqpoint{0.536250in}{0.528718in}}%
\pgfpathlineto{\pgfqpoint{1.846250in}{0.528718in}}%
\pgfusepath{stroke}%
\end{pgfscope}%
\begin{pgfscope}%
\pgfsetbuttcap%
\pgfsetroundjoin%
\definecolor{currentfill}{rgb}{0.000000,0.000000,0.000000}%
\pgfsetfillcolor{currentfill}%
\pgfsetlinewidth{0.602250pt}%
\definecolor{currentstroke}{rgb}{0.000000,0.000000,0.000000}%
\pgfsetstrokecolor{currentstroke}%
\pgfsetdash{}{0pt}%
\pgfsys@defobject{currentmarker}{\pgfqpoint{-0.027778in}{0.000000in}}{\pgfqpoint{-0.000000in}{0.000000in}}{%
\pgfpathmoveto{\pgfqpoint{-0.000000in}{0.000000in}}%
\pgfpathlineto{\pgfqpoint{-0.027778in}{0.000000in}}%
\pgfusepath{stroke,fill}%
}%
\begin{pgfscope}%
\pgfsys@transformshift{0.536250in}{0.528718in}%
\pgfsys@useobject{currentmarker}{}%
\end{pgfscope}%
\end{pgfscope}%
\begin{pgfscope}%
\pgfpathrectangle{\pgfqpoint{0.536250in}{0.525000in}}{\pgfqpoint{1.310000in}{1.887500in}}%
\pgfusepath{clip}%
\pgfsetbuttcap%
\pgfsetroundjoin%
\pgfsetlinewidth{0.803000pt}%
\definecolor{currentstroke}{rgb}{0.752941,0.752941,0.752941}%
\pgfsetstrokecolor{currentstroke}%
\pgfsetdash{{2.960000pt}{1.280000pt}}{0.000000pt}%
\pgfpathmoveto{\pgfqpoint{0.536250in}{0.604891in}}%
\pgfpathlineto{\pgfqpoint{1.846250in}{0.604891in}}%
\pgfusepath{stroke}%
\end{pgfscope}%
\begin{pgfscope}%
\pgfsetbuttcap%
\pgfsetroundjoin%
\definecolor{currentfill}{rgb}{0.000000,0.000000,0.000000}%
\pgfsetfillcolor{currentfill}%
\pgfsetlinewidth{0.602250pt}%
\definecolor{currentstroke}{rgb}{0.000000,0.000000,0.000000}%
\pgfsetstrokecolor{currentstroke}%
\pgfsetdash{}{0pt}%
\pgfsys@defobject{currentmarker}{\pgfqpoint{-0.027778in}{0.000000in}}{\pgfqpoint{-0.000000in}{0.000000in}}{%
\pgfpathmoveto{\pgfqpoint{-0.000000in}{0.000000in}}%
\pgfpathlineto{\pgfqpoint{-0.027778in}{0.000000in}}%
\pgfusepath{stroke,fill}%
}%
\begin{pgfscope}%
\pgfsys@transformshift{0.536250in}{0.604891in}%
\pgfsys@useobject{currentmarker}{}%
\end{pgfscope}%
\end{pgfscope}%
\begin{pgfscope}%
\pgfpathrectangle{\pgfqpoint{0.536250in}{0.525000in}}{\pgfqpoint{1.310000in}{1.887500in}}%
\pgfusepath{clip}%
\pgfsetbuttcap%
\pgfsetroundjoin%
\pgfsetlinewidth{0.803000pt}%
\definecolor{currentstroke}{rgb}{0.752941,0.752941,0.752941}%
\pgfsetstrokecolor{currentstroke}%
\pgfsetdash{{2.960000pt}{1.280000pt}}{0.000000pt}%
\pgfpathmoveto{\pgfqpoint{0.536250in}{0.669295in}}%
\pgfpathlineto{\pgfqpoint{1.846250in}{0.669295in}}%
\pgfusepath{stroke}%
\end{pgfscope}%
\begin{pgfscope}%
\pgfsetbuttcap%
\pgfsetroundjoin%
\definecolor{currentfill}{rgb}{0.000000,0.000000,0.000000}%
\pgfsetfillcolor{currentfill}%
\pgfsetlinewidth{0.602250pt}%
\definecolor{currentstroke}{rgb}{0.000000,0.000000,0.000000}%
\pgfsetstrokecolor{currentstroke}%
\pgfsetdash{}{0pt}%
\pgfsys@defobject{currentmarker}{\pgfqpoint{-0.027778in}{0.000000in}}{\pgfqpoint{-0.000000in}{0.000000in}}{%
\pgfpathmoveto{\pgfqpoint{-0.000000in}{0.000000in}}%
\pgfpathlineto{\pgfqpoint{-0.027778in}{0.000000in}}%
\pgfusepath{stroke,fill}%
}%
\begin{pgfscope}%
\pgfsys@transformshift{0.536250in}{0.669295in}%
\pgfsys@useobject{currentmarker}{}%
\end{pgfscope}%
\end{pgfscope}%
\begin{pgfscope}%
\pgfpathrectangle{\pgfqpoint{0.536250in}{0.525000in}}{\pgfqpoint{1.310000in}{1.887500in}}%
\pgfusepath{clip}%
\pgfsetbuttcap%
\pgfsetroundjoin%
\pgfsetlinewidth{0.803000pt}%
\definecolor{currentstroke}{rgb}{0.752941,0.752941,0.752941}%
\pgfsetstrokecolor{currentstroke}%
\pgfsetdash{{2.960000pt}{1.280000pt}}{0.000000pt}%
\pgfpathmoveto{\pgfqpoint{0.536250in}{0.725084in}}%
\pgfpathlineto{\pgfqpoint{1.846250in}{0.725084in}}%
\pgfusepath{stroke}%
\end{pgfscope}%
\begin{pgfscope}%
\pgfsetbuttcap%
\pgfsetroundjoin%
\definecolor{currentfill}{rgb}{0.000000,0.000000,0.000000}%
\pgfsetfillcolor{currentfill}%
\pgfsetlinewidth{0.602250pt}%
\definecolor{currentstroke}{rgb}{0.000000,0.000000,0.000000}%
\pgfsetstrokecolor{currentstroke}%
\pgfsetdash{}{0pt}%
\pgfsys@defobject{currentmarker}{\pgfqpoint{-0.027778in}{0.000000in}}{\pgfqpoint{-0.000000in}{0.000000in}}{%
\pgfpathmoveto{\pgfqpoint{-0.000000in}{0.000000in}}%
\pgfpathlineto{\pgfqpoint{-0.027778in}{0.000000in}}%
\pgfusepath{stroke,fill}%
}%
\begin{pgfscope}%
\pgfsys@transformshift{0.536250in}{0.725084in}%
\pgfsys@useobject{currentmarker}{}%
\end{pgfscope}%
\end{pgfscope}%
\begin{pgfscope}%
\pgfpathrectangle{\pgfqpoint{0.536250in}{0.525000in}}{\pgfqpoint{1.310000in}{1.887500in}}%
\pgfusepath{clip}%
\pgfsetbuttcap%
\pgfsetroundjoin%
\pgfsetlinewidth{0.803000pt}%
\definecolor{currentstroke}{rgb}{0.752941,0.752941,0.752941}%
\pgfsetstrokecolor{currentstroke}%
\pgfsetdash{{2.960000pt}{1.280000pt}}{0.000000pt}%
\pgfpathmoveto{\pgfqpoint{0.536250in}{0.774293in}}%
\pgfpathlineto{\pgfqpoint{1.846250in}{0.774293in}}%
\pgfusepath{stroke}%
\end{pgfscope}%
\begin{pgfscope}%
\pgfsetbuttcap%
\pgfsetroundjoin%
\definecolor{currentfill}{rgb}{0.000000,0.000000,0.000000}%
\pgfsetfillcolor{currentfill}%
\pgfsetlinewidth{0.602250pt}%
\definecolor{currentstroke}{rgb}{0.000000,0.000000,0.000000}%
\pgfsetstrokecolor{currentstroke}%
\pgfsetdash{}{0pt}%
\pgfsys@defobject{currentmarker}{\pgfqpoint{-0.027778in}{0.000000in}}{\pgfqpoint{-0.000000in}{0.000000in}}{%
\pgfpathmoveto{\pgfqpoint{-0.000000in}{0.000000in}}%
\pgfpathlineto{\pgfqpoint{-0.027778in}{0.000000in}}%
\pgfusepath{stroke,fill}%
}%
\begin{pgfscope}%
\pgfsys@transformshift{0.536250in}{0.774293in}%
\pgfsys@useobject{currentmarker}{}%
\end{pgfscope}%
\end{pgfscope}%
\begin{pgfscope}%
\pgfpathrectangle{\pgfqpoint{0.536250in}{0.525000in}}{\pgfqpoint{1.310000in}{1.887500in}}%
\pgfusepath{clip}%
\pgfsetbuttcap%
\pgfsetroundjoin%
\pgfsetlinewidth{0.803000pt}%
\definecolor{currentstroke}{rgb}{0.752941,0.752941,0.752941}%
\pgfsetstrokecolor{currentstroke}%
\pgfsetdash{{2.960000pt}{1.280000pt}}{0.000000pt}%
\pgfpathmoveto{\pgfqpoint{0.536250in}{1.107906in}}%
\pgfpathlineto{\pgfqpoint{1.846250in}{1.107906in}}%
\pgfusepath{stroke}%
\end{pgfscope}%
\begin{pgfscope}%
\pgfsetbuttcap%
\pgfsetroundjoin%
\definecolor{currentfill}{rgb}{0.000000,0.000000,0.000000}%
\pgfsetfillcolor{currentfill}%
\pgfsetlinewidth{0.602250pt}%
\definecolor{currentstroke}{rgb}{0.000000,0.000000,0.000000}%
\pgfsetstrokecolor{currentstroke}%
\pgfsetdash{}{0pt}%
\pgfsys@defobject{currentmarker}{\pgfqpoint{-0.027778in}{0.000000in}}{\pgfqpoint{-0.000000in}{0.000000in}}{%
\pgfpathmoveto{\pgfqpoint{-0.000000in}{0.000000in}}%
\pgfpathlineto{\pgfqpoint{-0.027778in}{0.000000in}}%
\pgfusepath{stroke,fill}%
}%
\begin{pgfscope}%
\pgfsys@transformshift{0.536250in}{1.107906in}%
\pgfsys@useobject{currentmarker}{}%
\end{pgfscope}%
\end{pgfscope}%
\begin{pgfscope}%
\pgfpathrectangle{\pgfqpoint{0.536250in}{0.525000in}}{\pgfqpoint{1.310000in}{1.887500in}}%
\pgfusepath{clip}%
\pgfsetbuttcap%
\pgfsetroundjoin%
\pgfsetlinewidth{0.803000pt}%
\definecolor{currentstroke}{rgb}{0.752941,0.752941,0.752941}%
\pgfsetstrokecolor{currentstroke}%
\pgfsetdash{{2.960000pt}{1.280000pt}}{0.000000pt}%
\pgfpathmoveto{\pgfqpoint{0.536250in}{1.277308in}}%
\pgfpathlineto{\pgfqpoint{1.846250in}{1.277308in}}%
\pgfusepath{stroke}%
\end{pgfscope}%
\begin{pgfscope}%
\pgfsetbuttcap%
\pgfsetroundjoin%
\definecolor{currentfill}{rgb}{0.000000,0.000000,0.000000}%
\pgfsetfillcolor{currentfill}%
\pgfsetlinewidth{0.602250pt}%
\definecolor{currentstroke}{rgb}{0.000000,0.000000,0.000000}%
\pgfsetstrokecolor{currentstroke}%
\pgfsetdash{}{0pt}%
\pgfsys@defobject{currentmarker}{\pgfqpoint{-0.027778in}{0.000000in}}{\pgfqpoint{-0.000000in}{0.000000in}}{%
\pgfpathmoveto{\pgfqpoint{-0.000000in}{0.000000in}}%
\pgfpathlineto{\pgfqpoint{-0.027778in}{0.000000in}}%
\pgfusepath{stroke,fill}%
}%
\begin{pgfscope}%
\pgfsys@transformshift{0.536250in}{1.277308in}%
\pgfsys@useobject{currentmarker}{}%
\end{pgfscope}%
\end{pgfscope}%
\begin{pgfscope}%
\pgfpathrectangle{\pgfqpoint{0.536250in}{0.525000in}}{\pgfqpoint{1.310000in}{1.887500in}}%
\pgfusepath{clip}%
\pgfsetbuttcap%
\pgfsetroundjoin%
\pgfsetlinewidth{0.803000pt}%
\definecolor{currentstroke}{rgb}{0.752941,0.752941,0.752941}%
\pgfsetstrokecolor{currentstroke}%
\pgfsetdash{{2.960000pt}{1.280000pt}}{0.000000pt}%
\pgfpathmoveto{\pgfqpoint{0.536250in}{1.397501in}}%
\pgfpathlineto{\pgfqpoint{1.846250in}{1.397501in}}%
\pgfusepath{stroke}%
\end{pgfscope}%
\begin{pgfscope}%
\pgfsetbuttcap%
\pgfsetroundjoin%
\definecolor{currentfill}{rgb}{0.000000,0.000000,0.000000}%
\pgfsetfillcolor{currentfill}%
\pgfsetlinewidth{0.602250pt}%
\definecolor{currentstroke}{rgb}{0.000000,0.000000,0.000000}%
\pgfsetstrokecolor{currentstroke}%
\pgfsetdash{}{0pt}%
\pgfsys@defobject{currentmarker}{\pgfqpoint{-0.027778in}{0.000000in}}{\pgfqpoint{-0.000000in}{0.000000in}}{%
\pgfpathmoveto{\pgfqpoint{-0.000000in}{0.000000in}}%
\pgfpathlineto{\pgfqpoint{-0.027778in}{0.000000in}}%
\pgfusepath{stroke,fill}%
}%
\begin{pgfscope}%
\pgfsys@transformshift{0.536250in}{1.397501in}%
\pgfsys@useobject{currentmarker}{}%
\end{pgfscope}%
\end{pgfscope}%
\begin{pgfscope}%
\pgfpathrectangle{\pgfqpoint{0.536250in}{0.525000in}}{\pgfqpoint{1.310000in}{1.887500in}}%
\pgfusepath{clip}%
\pgfsetbuttcap%
\pgfsetroundjoin%
\pgfsetlinewidth{0.803000pt}%
\definecolor{currentstroke}{rgb}{0.752941,0.752941,0.752941}%
\pgfsetstrokecolor{currentstroke}%
\pgfsetdash{{2.960000pt}{1.280000pt}}{0.000000pt}%
\pgfpathmoveto{\pgfqpoint{0.536250in}{1.490729in}}%
\pgfpathlineto{\pgfqpoint{1.846250in}{1.490729in}}%
\pgfusepath{stroke}%
\end{pgfscope}%
\begin{pgfscope}%
\pgfsetbuttcap%
\pgfsetroundjoin%
\definecolor{currentfill}{rgb}{0.000000,0.000000,0.000000}%
\pgfsetfillcolor{currentfill}%
\pgfsetlinewidth{0.602250pt}%
\definecolor{currentstroke}{rgb}{0.000000,0.000000,0.000000}%
\pgfsetstrokecolor{currentstroke}%
\pgfsetdash{}{0pt}%
\pgfsys@defobject{currentmarker}{\pgfqpoint{-0.027778in}{0.000000in}}{\pgfqpoint{-0.000000in}{0.000000in}}{%
\pgfpathmoveto{\pgfqpoint{-0.000000in}{0.000000in}}%
\pgfpathlineto{\pgfqpoint{-0.027778in}{0.000000in}}%
\pgfusepath{stroke,fill}%
}%
\begin{pgfscope}%
\pgfsys@transformshift{0.536250in}{1.490729in}%
\pgfsys@useobject{currentmarker}{}%
\end{pgfscope}%
\end{pgfscope}%
\begin{pgfscope}%
\pgfpathrectangle{\pgfqpoint{0.536250in}{0.525000in}}{\pgfqpoint{1.310000in}{1.887500in}}%
\pgfusepath{clip}%
\pgfsetbuttcap%
\pgfsetroundjoin%
\pgfsetlinewidth{0.803000pt}%
\definecolor{currentstroke}{rgb}{0.752941,0.752941,0.752941}%
\pgfsetstrokecolor{currentstroke}%
\pgfsetdash{{2.960000pt}{1.280000pt}}{0.000000pt}%
\pgfpathmoveto{\pgfqpoint{0.536250in}{1.566902in}}%
\pgfpathlineto{\pgfqpoint{1.846250in}{1.566902in}}%
\pgfusepath{stroke}%
\end{pgfscope}%
\begin{pgfscope}%
\pgfsetbuttcap%
\pgfsetroundjoin%
\definecolor{currentfill}{rgb}{0.000000,0.000000,0.000000}%
\pgfsetfillcolor{currentfill}%
\pgfsetlinewidth{0.602250pt}%
\definecolor{currentstroke}{rgb}{0.000000,0.000000,0.000000}%
\pgfsetstrokecolor{currentstroke}%
\pgfsetdash{}{0pt}%
\pgfsys@defobject{currentmarker}{\pgfqpoint{-0.027778in}{0.000000in}}{\pgfqpoint{-0.000000in}{0.000000in}}{%
\pgfpathmoveto{\pgfqpoint{-0.000000in}{0.000000in}}%
\pgfpathlineto{\pgfqpoint{-0.027778in}{0.000000in}}%
\pgfusepath{stroke,fill}%
}%
\begin{pgfscope}%
\pgfsys@transformshift{0.536250in}{1.566902in}%
\pgfsys@useobject{currentmarker}{}%
\end{pgfscope}%
\end{pgfscope}%
\begin{pgfscope}%
\pgfpathrectangle{\pgfqpoint{0.536250in}{0.525000in}}{\pgfqpoint{1.310000in}{1.887500in}}%
\pgfusepath{clip}%
\pgfsetbuttcap%
\pgfsetroundjoin%
\pgfsetlinewidth{0.803000pt}%
\definecolor{currentstroke}{rgb}{0.752941,0.752941,0.752941}%
\pgfsetstrokecolor{currentstroke}%
\pgfsetdash{{2.960000pt}{1.280000pt}}{0.000000pt}%
\pgfpathmoveto{\pgfqpoint{0.536250in}{1.631306in}}%
\pgfpathlineto{\pgfqpoint{1.846250in}{1.631306in}}%
\pgfusepath{stroke}%
\end{pgfscope}%
\begin{pgfscope}%
\pgfsetbuttcap%
\pgfsetroundjoin%
\definecolor{currentfill}{rgb}{0.000000,0.000000,0.000000}%
\pgfsetfillcolor{currentfill}%
\pgfsetlinewidth{0.602250pt}%
\definecolor{currentstroke}{rgb}{0.000000,0.000000,0.000000}%
\pgfsetstrokecolor{currentstroke}%
\pgfsetdash{}{0pt}%
\pgfsys@defobject{currentmarker}{\pgfqpoint{-0.027778in}{0.000000in}}{\pgfqpoint{-0.000000in}{0.000000in}}{%
\pgfpathmoveto{\pgfqpoint{-0.000000in}{0.000000in}}%
\pgfpathlineto{\pgfqpoint{-0.027778in}{0.000000in}}%
\pgfusepath{stroke,fill}%
}%
\begin{pgfscope}%
\pgfsys@transformshift{0.536250in}{1.631306in}%
\pgfsys@useobject{currentmarker}{}%
\end{pgfscope}%
\end{pgfscope}%
\begin{pgfscope}%
\pgfpathrectangle{\pgfqpoint{0.536250in}{0.525000in}}{\pgfqpoint{1.310000in}{1.887500in}}%
\pgfusepath{clip}%
\pgfsetbuttcap%
\pgfsetroundjoin%
\pgfsetlinewidth{0.803000pt}%
\definecolor{currentstroke}{rgb}{0.752941,0.752941,0.752941}%
\pgfsetstrokecolor{currentstroke}%
\pgfsetdash{{2.960000pt}{1.280000pt}}{0.000000pt}%
\pgfpathmoveto{\pgfqpoint{0.536250in}{1.687095in}}%
\pgfpathlineto{\pgfqpoint{1.846250in}{1.687095in}}%
\pgfusepath{stroke}%
\end{pgfscope}%
\begin{pgfscope}%
\pgfsetbuttcap%
\pgfsetroundjoin%
\definecolor{currentfill}{rgb}{0.000000,0.000000,0.000000}%
\pgfsetfillcolor{currentfill}%
\pgfsetlinewidth{0.602250pt}%
\definecolor{currentstroke}{rgb}{0.000000,0.000000,0.000000}%
\pgfsetstrokecolor{currentstroke}%
\pgfsetdash{}{0pt}%
\pgfsys@defobject{currentmarker}{\pgfqpoint{-0.027778in}{0.000000in}}{\pgfqpoint{-0.000000in}{0.000000in}}{%
\pgfpathmoveto{\pgfqpoint{-0.000000in}{0.000000in}}%
\pgfpathlineto{\pgfqpoint{-0.027778in}{0.000000in}}%
\pgfusepath{stroke,fill}%
}%
\begin{pgfscope}%
\pgfsys@transformshift{0.536250in}{1.687095in}%
\pgfsys@useobject{currentmarker}{}%
\end{pgfscope}%
\end{pgfscope}%
\begin{pgfscope}%
\pgfpathrectangle{\pgfqpoint{0.536250in}{0.525000in}}{\pgfqpoint{1.310000in}{1.887500in}}%
\pgfusepath{clip}%
\pgfsetbuttcap%
\pgfsetroundjoin%
\pgfsetlinewidth{0.803000pt}%
\definecolor{currentstroke}{rgb}{0.752941,0.752941,0.752941}%
\pgfsetstrokecolor{currentstroke}%
\pgfsetdash{{2.960000pt}{1.280000pt}}{0.000000pt}%
\pgfpathmoveto{\pgfqpoint{0.536250in}{1.736304in}}%
\pgfpathlineto{\pgfqpoint{1.846250in}{1.736304in}}%
\pgfusepath{stroke}%
\end{pgfscope}%
\begin{pgfscope}%
\pgfsetbuttcap%
\pgfsetroundjoin%
\definecolor{currentfill}{rgb}{0.000000,0.000000,0.000000}%
\pgfsetfillcolor{currentfill}%
\pgfsetlinewidth{0.602250pt}%
\definecolor{currentstroke}{rgb}{0.000000,0.000000,0.000000}%
\pgfsetstrokecolor{currentstroke}%
\pgfsetdash{}{0pt}%
\pgfsys@defobject{currentmarker}{\pgfqpoint{-0.027778in}{0.000000in}}{\pgfqpoint{-0.000000in}{0.000000in}}{%
\pgfpathmoveto{\pgfqpoint{-0.000000in}{0.000000in}}%
\pgfpathlineto{\pgfqpoint{-0.027778in}{0.000000in}}%
\pgfusepath{stroke,fill}%
}%
\begin{pgfscope}%
\pgfsys@transformshift{0.536250in}{1.736304in}%
\pgfsys@useobject{currentmarker}{}%
\end{pgfscope}%
\end{pgfscope}%
\begin{pgfscope}%
\pgfpathrectangle{\pgfqpoint{0.536250in}{0.525000in}}{\pgfqpoint{1.310000in}{1.887500in}}%
\pgfusepath{clip}%
\pgfsetbuttcap%
\pgfsetroundjoin%
\pgfsetlinewidth{0.803000pt}%
\definecolor{currentstroke}{rgb}{0.752941,0.752941,0.752941}%
\pgfsetstrokecolor{currentstroke}%
\pgfsetdash{{2.960000pt}{1.280000pt}}{0.000000pt}%
\pgfpathmoveto{\pgfqpoint{0.536250in}{2.069918in}}%
\pgfpathlineto{\pgfqpoint{1.846250in}{2.069918in}}%
\pgfusepath{stroke}%
\end{pgfscope}%
\begin{pgfscope}%
\pgfsetbuttcap%
\pgfsetroundjoin%
\definecolor{currentfill}{rgb}{0.000000,0.000000,0.000000}%
\pgfsetfillcolor{currentfill}%
\pgfsetlinewidth{0.602250pt}%
\definecolor{currentstroke}{rgb}{0.000000,0.000000,0.000000}%
\pgfsetstrokecolor{currentstroke}%
\pgfsetdash{}{0pt}%
\pgfsys@defobject{currentmarker}{\pgfqpoint{-0.027778in}{0.000000in}}{\pgfqpoint{-0.000000in}{0.000000in}}{%
\pgfpathmoveto{\pgfqpoint{-0.000000in}{0.000000in}}%
\pgfpathlineto{\pgfqpoint{-0.027778in}{0.000000in}}%
\pgfusepath{stroke,fill}%
}%
\begin{pgfscope}%
\pgfsys@transformshift{0.536250in}{2.069918in}%
\pgfsys@useobject{currentmarker}{}%
\end{pgfscope}%
\end{pgfscope}%
\begin{pgfscope}%
\pgfpathrectangle{\pgfqpoint{0.536250in}{0.525000in}}{\pgfqpoint{1.310000in}{1.887500in}}%
\pgfusepath{clip}%
\pgfsetbuttcap%
\pgfsetroundjoin%
\pgfsetlinewidth{0.803000pt}%
\definecolor{currentstroke}{rgb}{0.752941,0.752941,0.752941}%
\pgfsetstrokecolor{currentstroke}%
\pgfsetdash{{2.960000pt}{1.280000pt}}{0.000000pt}%
\pgfpathmoveto{\pgfqpoint{0.536250in}{2.239319in}}%
\pgfpathlineto{\pgfqpoint{1.846250in}{2.239319in}}%
\pgfusepath{stroke}%
\end{pgfscope}%
\begin{pgfscope}%
\pgfsetbuttcap%
\pgfsetroundjoin%
\definecolor{currentfill}{rgb}{0.000000,0.000000,0.000000}%
\pgfsetfillcolor{currentfill}%
\pgfsetlinewidth{0.602250pt}%
\definecolor{currentstroke}{rgb}{0.000000,0.000000,0.000000}%
\pgfsetstrokecolor{currentstroke}%
\pgfsetdash{}{0pt}%
\pgfsys@defobject{currentmarker}{\pgfqpoint{-0.027778in}{0.000000in}}{\pgfqpoint{-0.000000in}{0.000000in}}{%
\pgfpathmoveto{\pgfqpoint{-0.000000in}{0.000000in}}%
\pgfpathlineto{\pgfqpoint{-0.027778in}{0.000000in}}%
\pgfusepath{stroke,fill}%
}%
\begin{pgfscope}%
\pgfsys@transformshift{0.536250in}{2.239319in}%
\pgfsys@useobject{currentmarker}{}%
\end{pgfscope}%
\end{pgfscope}%
\begin{pgfscope}%
\pgfpathrectangle{\pgfqpoint{0.536250in}{0.525000in}}{\pgfqpoint{1.310000in}{1.887500in}}%
\pgfusepath{clip}%
\pgfsetbuttcap%
\pgfsetroundjoin%
\pgfsetlinewidth{0.803000pt}%
\definecolor{currentstroke}{rgb}{0.752941,0.752941,0.752941}%
\pgfsetstrokecolor{currentstroke}%
\pgfsetdash{{2.960000pt}{1.280000pt}}{0.000000pt}%
\pgfpathmoveto{\pgfqpoint{0.536250in}{2.359512in}}%
\pgfpathlineto{\pgfqpoint{1.846250in}{2.359512in}}%
\pgfusepath{stroke}%
\end{pgfscope}%
\begin{pgfscope}%
\pgfsetbuttcap%
\pgfsetroundjoin%
\definecolor{currentfill}{rgb}{0.000000,0.000000,0.000000}%
\pgfsetfillcolor{currentfill}%
\pgfsetlinewidth{0.602250pt}%
\definecolor{currentstroke}{rgb}{0.000000,0.000000,0.000000}%
\pgfsetstrokecolor{currentstroke}%
\pgfsetdash{}{0pt}%
\pgfsys@defobject{currentmarker}{\pgfqpoint{-0.027778in}{0.000000in}}{\pgfqpoint{-0.000000in}{0.000000in}}{%
\pgfpathmoveto{\pgfqpoint{-0.000000in}{0.000000in}}%
\pgfpathlineto{\pgfqpoint{-0.027778in}{0.000000in}}%
\pgfusepath{stroke,fill}%
}%
\begin{pgfscope}%
\pgfsys@transformshift{0.536250in}{2.359512in}%
\pgfsys@useobject{currentmarker}{}%
\end{pgfscope}%
\end{pgfscope}%
\begin{pgfscope}%
\definecolor{textcolor}{rgb}{0.000000,0.000000,0.000000}%
\pgfsetstrokecolor{textcolor}%
\pgfsetfillcolor{textcolor}%
\pgftext[x=0.197131in,y=1.468750in,,bottom,rotate=90.000000]{\color{textcolor}\rmfamily\fontsize{9.000000}{10.800000}\selectfont Critical energy [MeV]}%
\end{pgfscope}%
\begin{pgfscope}%
\pgfpathrectangle{\pgfqpoint{0.536250in}{0.525000in}}{\pgfqpoint{1.310000in}{1.887500in}}%
\pgfusepath{clip}%
\pgfsetrectcap%
\pgfsetroundjoin%
\pgfsetlinewidth{1.003750pt}%
\definecolor{currentstroke}{rgb}{0.000000,0.000000,0.000000}%
\pgfsetstrokecolor{currentstroke}%
\pgfsetdash{}{0pt}%
\pgfpathmoveto{\pgfqpoint{0.536250in}{2.198877in}}%
\pgfpathlineto{\pgfqpoint{0.733856in}{2.044669in}}%
\pgfpathlineto{\pgfqpoint{0.849448in}{1.932287in}}%
\pgfpathlineto{\pgfqpoint{0.931462in}{1.843815in}}%
\pgfpathlineto{\pgfqpoint{0.995077in}{1.770843in}}%
\pgfpathlineto{\pgfqpoint{1.047054in}{1.708741in}}%
\pgfpathlineto{\pgfqpoint{1.091000in}{1.654687in}}%
\pgfpathlineto{\pgfqpoint{1.129068in}{1.606832in}}%
\pgfpathlineto{\pgfqpoint{1.162646in}{1.563900in}}%
\pgfpathlineto{\pgfqpoint{1.192683in}{1.524970in}}%
\pgfpathlineto{\pgfqpoint{1.219854in}{1.489362in}}%
\pgfpathlineto{\pgfqpoint{1.244660in}{1.456551in}}%
\pgfpathlineto{\pgfqpoint{1.267479in}{1.426130in}}%
\pgfpathlineto{\pgfqpoint{1.288606in}{1.397775in}}%
\pgfpathlineto{\pgfqpoint{1.308275in}{1.371222in}}%
\pgfpathlineto{\pgfqpoint{1.326674in}{1.346257in}}%
\pgfpathlineto{\pgfqpoint{1.343957in}{1.322699in}}%
\pgfpathlineto{\pgfqpoint{1.360252in}{1.300400in}}%
\pgfpathlineto{\pgfqpoint{1.375666in}{1.279230in}}%
\pgfpathlineto{\pgfqpoint{1.390289in}{1.259082in}}%
\pgfpathlineto{\pgfqpoint{1.404198in}{1.239861in}}%
\pgfpathlineto{\pgfqpoint{1.417460in}{1.221485in}}%
\pgfpathlineto{\pgfqpoint{1.430133in}{1.203884in}}%
\pgfpathlineto{\pgfqpoint{1.442266in}{1.186994in}}%
\pgfpathlineto{\pgfqpoint{1.453903in}{1.170760in}}%
\pgfpathlineto{\pgfqpoint{1.465085in}{1.155134in}}%
\pgfpathlineto{\pgfqpoint{1.475844in}{1.140071in}}%
\pgfpathlineto{\pgfqpoint{1.486212in}{1.125533in}}%
\pgfpathlineto{\pgfqpoint{1.496216in}{1.111483in}}%
\pgfpathlineto{\pgfqpoint{1.505881in}{1.097891in}}%
\pgfpathlineto{\pgfqpoint{1.515228in}{1.084726in}}%
\pgfpathlineto{\pgfqpoint{1.524280in}{1.071964in}}%
\pgfpathlineto{\pgfqpoint{1.533052in}{1.059581in}}%
\pgfpathlineto{\pgfqpoint{1.541563in}{1.047554in}}%
\pgfpathlineto{\pgfqpoint{1.549827in}{1.035863in}}%
\pgfpathlineto{\pgfqpoint{1.557858in}{1.024491in}}%
\pgfpathlineto{\pgfqpoint{1.565669in}{1.013419in}}%
\pgfpathlineto{\pgfqpoint{1.573271in}{1.002634in}}%
\pgfpathlineto{\pgfqpoint{1.580677in}{0.992120in}}%
\pgfpathlineto{\pgfqpoint{1.587894in}{0.981865in}}%
\pgfpathlineto{\pgfqpoint{1.594934in}{0.971855in}}%
\pgfpathlineto{\pgfqpoint{1.601804in}{0.962079in}}%
\pgfpathlineto{\pgfqpoint{1.608512in}{0.952527in}}%
\pgfpathlineto{\pgfqpoint{1.615066in}{0.943188in}}%
\pgfpathlineto{\pgfqpoint{1.621473in}{0.934054in}}%
\pgfpathlineto{\pgfqpoint{1.627738in}{0.925115in}}%
\pgfpathlineto{\pgfqpoint{1.633870in}{0.916363in}}%
\pgfpathlineto{\pgfqpoint{1.639872in}{0.907790in}}%
\pgfpathlineto{\pgfqpoint{1.645750in}{0.899391in}}%
\pgfpathlineto{\pgfqpoint{1.651509in}{0.891156in}}%
\pgfpathlineto{\pgfqpoint{1.657155in}{0.883081in}}%
\pgfpathlineto{\pgfqpoint{1.662691in}{0.875159in}}%
\pgfpathlineto{\pgfqpoint{1.668121in}{0.867384in}}%
\pgfpathlineto{\pgfqpoint{1.673450in}{0.859752in}}%
\pgfpathlineto{\pgfqpoint{1.678681in}{0.852256in}}%
\pgfpathlineto{\pgfqpoint{1.683818in}{0.844893in}}%
\pgfpathlineto{\pgfqpoint{1.688863in}{0.837657in}}%
\pgfpathlineto{\pgfqpoint{1.693822in}{0.830544in}}%
\pgfpathlineto{\pgfqpoint{1.698695in}{0.823550in}}%
\pgfpathlineto{\pgfqpoint{1.703486in}{0.816672in}}%
\pgfpathlineto{\pgfqpoint{1.708199in}{0.809904in}}%
\pgfpathlineto{\pgfqpoint{1.712834in}{0.803245in}}%
\pgfpathlineto{\pgfqpoint{1.717396in}{0.796690in}}%
\pgfpathlineto{\pgfqpoint{1.721885in}{0.790237in}}%
\pgfpathlineto{\pgfqpoint{1.726305in}{0.783881in}}%
\pgfpathlineto{\pgfqpoint{1.730658in}{0.777621in}}%
\pgfpathlineto{\pgfqpoint{1.734945in}{0.771453in}}%
\pgfpathlineto{\pgfqpoint{1.739169in}{0.765375in}}%
\pgfpathlineto{\pgfqpoint{1.743330in}{0.759384in}}%
\pgfpathlineto{\pgfqpoint{1.747433in}{0.753478in}}%
\pgfpathlineto{\pgfqpoint{1.751476in}{0.747654in}}%
\pgfpathlineto{\pgfqpoint{1.755464in}{0.741911in}}%
\pgfpathlineto{\pgfqpoint{1.759396in}{0.736245in}}%
\pgfpathlineto{\pgfqpoint{1.763275in}{0.730655in}}%
\pgfpathlineto{\pgfqpoint{1.767101in}{0.725138in}}%
\pgfpathlineto{\pgfqpoint{1.770877in}{0.719694in}}%
\pgfpathlineto{\pgfqpoint{1.774604in}{0.714320in}}%
\pgfpathlineto{\pgfqpoint{1.778283in}{0.709014in}}%
\pgfpathlineto{\pgfqpoint{1.781914in}{0.703774in}}%
\pgfpathlineto{\pgfqpoint{1.785500in}{0.698599in}}%
\pgfpathlineto{\pgfqpoint{1.789042in}{0.693488in}}%
\pgfpathlineto{\pgfqpoint{1.792540in}{0.688438in}}%
\pgfpathlineto{\pgfqpoint{1.795995in}{0.683449in}}%
\pgfpathlineto{\pgfqpoint{1.799410in}{0.678519in}}%
\pgfpathlineto{\pgfqpoint{1.802783in}{0.673646in}}%
\pgfpathlineto{\pgfqpoint{1.806118in}{0.668829in}}%
\pgfpathlineto{\pgfqpoint{1.809414in}{0.664067in}}%
\pgfpathlineto{\pgfqpoint{1.812672in}{0.659359in}}%
\pgfpathlineto{\pgfqpoint{1.815893in}{0.654704in}}%
\pgfpathlineto{\pgfqpoint{1.819078in}{0.650099in}}%
\pgfpathlineto{\pgfqpoint{1.822229in}{0.645545in}}%
\pgfpathlineto{\pgfqpoint{1.825344in}{0.641040in}}%
\pgfpathlineto{\pgfqpoint{1.828426in}{0.636583in}}%
\pgfpathlineto{\pgfqpoint{1.831475in}{0.632173in}}%
\pgfpathlineto{\pgfqpoint{1.834492in}{0.627809in}}%
\pgfpathlineto{\pgfqpoint{1.837477in}{0.623490in}}%
\pgfpathlineto{\pgfqpoint{1.840432in}{0.619216in}}%
\pgfpathlineto{\pgfqpoint{1.843356in}{0.614984in}}%
\pgfpathlineto{\pgfqpoint{1.846250in}{0.610795in}}%
\pgfusepath{stroke}%
\end{pgfscope}%
\begin{pgfscope}%
\pgfpathrectangle{\pgfqpoint{0.536250in}{0.525000in}}{\pgfqpoint{1.310000in}{1.887500in}}%
\pgfusepath{clip}%
\pgfsetbuttcap%
\pgfsetroundjoin%
\pgfsetlinewidth{1.003750pt}%
\definecolor{currentstroke}{rgb}{0.000000,0.000000,0.000000}%
\pgfsetstrokecolor{currentstroke}%
\pgfsetdash{{3.700000pt}{1.600000pt}}{0.000000pt}%
\pgfpathmoveto{\pgfqpoint{0.536250in}{2.326705in}}%
\pgfpathlineto{\pgfqpoint{0.733856in}{2.151540in}}%
\pgfpathlineto{\pgfqpoint{0.849448in}{2.028496in}}%
\pgfpathlineto{\pgfqpoint{0.931462in}{1.933565in}}%
\pgfpathlineto{\pgfqpoint{0.995077in}{1.856261in}}%
\pgfpathlineto{\pgfqpoint{1.047054in}{1.791052in}}%
\pgfpathlineto{\pgfqpoint{1.091000in}{1.734660in}}%
\pgfpathlineto{\pgfqpoint{1.129068in}{1.684982in}}%
\pgfpathlineto{\pgfqpoint{1.162646in}{1.640588in}}%
\pgfpathlineto{\pgfqpoint{1.192683in}{1.600462in}}%
\pgfpathlineto{\pgfqpoint{1.219854in}{1.563854in}}%
\pgfpathlineto{\pgfqpoint{1.244660in}{1.530196in}}%
\pgfpathlineto{\pgfqpoint{1.267479in}{1.499050in}}%
\pgfpathlineto{\pgfqpoint{1.288606in}{1.470065in}}%
\pgfpathlineto{\pgfqpoint{1.308275in}{1.442961in}}%
\pgfpathlineto{\pgfqpoint{1.326674in}{1.417509in}}%
\pgfpathlineto{\pgfqpoint{1.343957in}{1.393518in}}%
\pgfpathlineto{\pgfqpoint{1.360252in}{1.370831in}}%
\pgfpathlineto{\pgfqpoint{1.375666in}{1.349313in}}%
\pgfpathlineto{\pgfqpoint{1.390289in}{1.328848in}}%
\pgfpathlineto{\pgfqpoint{1.404198in}{1.309340in}}%
\pgfpathlineto{\pgfqpoint{1.417460in}{1.290702in}}%
\pgfpathlineto{\pgfqpoint{1.430133in}{1.272860in}}%
\pgfpathlineto{\pgfqpoint{1.442266in}{1.255749in}}%
\pgfpathlineto{\pgfqpoint{1.453903in}{1.239311in}}%
\pgfpathlineto{\pgfqpoint{1.465085in}{1.223495in}}%
\pgfpathlineto{\pgfqpoint{1.475844in}{1.208257in}}%
\pgfpathlineto{\pgfqpoint{1.486212in}{1.193554in}}%
\pgfpathlineto{\pgfqpoint{1.496216in}{1.179352in}}%
\pgfpathlineto{\pgfqpoint{1.505881in}{1.165616in}}%
\pgfpathlineto{\pgfqpoint{1.515228in}{1.152318in}}%
\pgfpathlineto{\pgfqpoint{1.524280in}{1.139430in}}%
\pgfpathlineto{\pgfqpoint{1.533052in}{1.126928in}}%
\pgfpathlineto{\pgfqpoint{1.541563in}{1.114789in}}%
\pgfpathlineto{\pgfqpoint{1.549827in}{1.102992in}}%
\pgfpathlineto{\pgfqpoint{1.557858in}{1.091520in}}%
\pgfpathlineto{\pgfqpoint{1.565669in}{1.080354in}}%
\pgfpathlineto{\pgfqpoint{1.573271in}{1.069479in}}%
\pgfpathlineto{\pgfqpoint{1.580677in}{1.058880in}}%
\pgfpathlineto{\pgfqpoint{1.587894in}{1.048543in}}%
\pgfpathlineto{\pgfqpoint{1.594934in}{1.038456in}}%
\pgfpathlineto{\pgfqpoint{1.601804in}{1.028606in}}%
\pgfpathlineto{\pgfqpoint{1.608512in}{1.018984in}}%
\pgfpathlineto{\pgfqpoint{1.615066in}{1.009578in}}%
\pgfpathlineto{\pgfqpoint{1.621473in}{1.000379in}}%
\pgfpathlineto{\pgfqpoint{1.627738in}{0.991378in}}%
\pgfpathlineto{\pgfqpoint{1.633870in}{0.982567in}}%
\pgfpathlineto{\pgfqpoint{1.639872in}{0.973938in}}%
\pgfpathlineto{\pgfqpoint{1.645750in}{0.965484in}}%
\pgfpathlineto{\pgfqpoint{1.651509in}{0.957198in}}%
\pgfpathlineto{\pgfqpoint{1.657155in}{0.949072in}}%
\pgfpathlineto{\pgfqpoint{1.662691in}{0.941102in}}%
\pgfpathlineto{\pgfqpoint{1.668121in}{0.933281in}}%
\pgfpathlineto{\pgfqpoint{1.673450in}{0.925603in}}%
\pgfpathlineto{\pgfqpoint{1.678681in}{0.918064in}}%
\pgfpathlineto{\pgfqpoint{1.683818in}{0.910659in}}%
\pgfpathlineto{\pgfqpoint{1.688863in}{0.903383in}}%
\pgfpathlineto{\pgfqpoint{1.693822in}{0.896231in}}%
\pgfpathlineto{\pgfqpoint{1.698695in}{0.889199in}}%
\pgfpathlineto{\pgfqpoint{1.703486in}{0.882284in}}%
\pgfpathlineto{\pgfqpoint{1.708199in}{0.875482in}}%
\pgfpathlineto{\pgfqpoint{1.712834in}{0.868788in}}%
\pgfpathlineto{\pgfqpoint{1.717396in}{0.862201in}}%
\pgfpathlineto{\pgfqpoint{1.721885in}{0.855715in}}%
\pgfpathlineto{\pgfqpoint{1.726305in}{0.849328in}}%
\pgfpathlineto{\pgfqpoint{1.730658in}{0.843038in}}%
\pgfpathlineto{\pgfqpoint{1.734945in}{0.836841in}}%
\pgfpathlineto{\pgfqpoint{1.739169in}{0.830735in}}%
\pgfpathlineto{\pgfqpoint{1.743330in}{0.824716in}}%
\pgfpathlineto{\pgfqpoint{1.747433in}{0.818783in}}%
\pgfpathlineto{\pgfqpoint{1.751476in}{0.812933in}}%
\pgfpathlineto{\pgfqpoint{1.755464in}{0.807164in}}%
\pgfpathlineto{\pgfqpoint{1.759396in}{0.801473in}}%
\pgfpathlineto{\pgfqpoint{1.763275in}{0.795859in}}%
\pgfpathlineto{\pgfqpoint{1.767101in}{0.790320in}}%
\pgfpathlineto{\pgfqpoint{1.770877in}{0.784852in}}%
\pgfpathlineto{\pgfqpoint{1.774604in}{0.779456in}}%
\pgfpathlineto{\pgfqpoint{1.778283in}{0.774128in}}%
\pgfpathlineto{\pgfqpoint{1.781914in}{0.768867in}}%
\pgfpathlineto{\pgfqpoint{1.785500in}{0.763672in}}%
\pgfpathlineto{\pgfqpoint{1.789042in}{0.758541in}}%
\pgfpathlineto{\pgfqpoint{1.792540in}{0.753472in}}%
\pgfpathlineto{\pgfqpoint{1.795995in}{0.748463in}}%
\pgfpathlineto{\pgfqpoint{1.799410in}{0.743514in}}%
\pgfpathlineto{\pgfqpoint{1.802783in}{0.738623in}}%
\pgfpathlineto{\pgfqpoint{1.806118in}{0.733788in}}%
\pgfpathlineto{\pgfqpoint{1.809414in}{0.729009in}}%
\pgfpathlineto{\pgfqpoint{1.812672in}{0.724284in}}%
\pgfpathlineto{\pgfqpoint{1.815893in}{0.719612in}}%
\pgfpathlineto{\pgfqpoint{1.819078in}{0.714991in}}%
\pgfpathlineto{\pgfqpoint{1.822229in}{0.710421in}}%
\pgfpathlineto{\pgfqpoint{1.825344in}{0.705900in}}%
\pgfpathlineto{\pgfqpoint{1.828426in}{0.701428in}}%
\pgfpathlineto{\pgfqpoint{1.831475in}{0.697003in}}%
\pgfpathlineto{\pgfqpoint{1.834492in}{0.692625in}}%
\pgfpathlineto{\pgfqpoint{1.837477in}{0.688291in}}%
\pgfpathlineto{\pgfqpoint{1.840432in}{0.684003in}}%
\pgfpathlineto{\pgfqpoint{1.843356in}{0.679758in}}%
\pgfpathlineto{\pgfqpoint{1.846250in}{0.675555in}}%
\pgfusepath{stroke}%
\end{pgfscope}%
\begin{pgfscope}%
\pgfsetrectcap%
\pgfsetmiterjoin%
\pgfsetlinewidth{1.003750pt}%
\definecolor{currentstroke}{rgb}{0.000000,0.000000,0.000000}%
\pgfsetstrokecolor{currentstroke}%
\pgfsetdash{}{0pt}%
\pgfpathmoveto{\pgfqpoint{0.536250in}{0.525000in}}%
\pgfpathlineto{\pgfqpoint{0.536250in}{2.412500in}}%
\pgfusepath{stroke}%
\end{pgfscope}%
\begin{pgfscope}%
\pgfsetrectcap%
\pgfsetmiterjoin%
\pgfsetlinewidth{1.003750pt}%
\definecolor{currentstroke}{rgb}{0.000000,0.000000,0.000000}%
\pgfsetstrokecolor{currentstroke}%
\pgfsetdash{}{0pt}%
\pgfpathmoveto{\pgfqpoint{1.846250in}{0.525000in}}%
\pgfpathlineto{\pgfqpoint{1.846250in}{2.412500in}}%
\pgfusepath{stroke}%
\end{pgfscope}%
\begin{pgfscope}%
\pgfsetrectcap%
\pgfsetmiterjoin%
\pgfsetlinewidth{1.003750pt}%
\definecolor{currentstroke}{rgb}{0.000000,0.000000,0.000000}%
\pgfsetstrokecolor{currentstroke}%
\pgfsetdash{}{0pt}%
\pgfpathmoveto{\pgfqpoint{0.536250in}{0.525000in}}%
\pgfpathlineto{\pgfqpoint{1.846250in}{0.525000in}}%
\pgfusepath{stroke}%
\end{pgfscope}%
\begin{pgfscope}%
\pgfsetrectcap%
\pgfsetmiterjoin%
\pgfsetlinewidth{1.003750pt}%
\definecolor{currentstroke}{rgb}{0.000000,0.000000,0.000000}%
\pgfsetstrokecolor{currentstroke}%
\pgfsetdash{}{0pt}%
\pgfpathmoveto{\pgfqpoint{0.536250in}{2.412500in}}%
\pgfpathlineto{\pgfqpoint{1.846250in}{2.412500in}}%
\pgfusepath{stroke}%
\end{pgfscope}%
\begin{pgfscope}%
\pgfsetbuttcap%
\pgfsetmiterjoin%
\definecolor{currentfill}{rgb}{1.000000,1.000000,1.000000}%
\pgfsetfillcolor{currentfill}%
\pgfsetfillopacity{0.800000}%
\pgfsetlinewidth{1.003750pt}%
\definecolor{currentstroke}{rgb}{0.800000,0.800000,0.800000}%
\pgfsetstrokecolor{currentstroke}%
\pgfsetstrokeopacity{0.800000}%
\pgfsetdash{}{0pt}%
\pgfpathmoveto{\pgfqpoint{0.973740in}{1.945557in}}%
\pgfpathlineto{\pgfqpoint{1.758750in}{1.945557in}}%
\pgfpathquadraticcurveto{\pgfqpoint{1.783750in}{1.945557in}}{\pgfqpoint{1.783750in}{1.970557in}}%
\pgfpathlineto{\pgfqpoint{1.783750in}{2.325000in}}%
\pgfpathquadraticcurveto{\pgfqpoint{1.783750in}{2.350000in}}{\pgfqpoint{1.758750in}{2.350000in}}%
\pgfpathlineto{\pgfqpoint{0.973740in}{2.350000in}}%
\pgfpathquadraticcurveto{\pgfqpoint{0.948740in}{2.350000in}}{\pgfqpoint{0.948740in}{2.325000in}}%
\pgfpathlineto{\pgfqpoint{0.948740in}{1.970557in}}%
\pgfpathquadraticcurveto{\pgfqpoint{0.948740in}{1.945557in}}{\pgfqpoint{0.973740in}{1.945557in}}%
\pgfpathlineto{\pgfqpoint{0.973740in}{1.945557in}}%
\pgfpathclose%
\pgfusepath{stroke,fill}%
\end{pgfscope}%
\begin{pgfscope}%
\pgfsetrectcap%
\pgfsetroundjoin%
\pgfsetlinewidth{1.003750pt}%
\definecolor{currentstroke}{rgb}{0.000000,0.000000,0.000000}%
\pgfsetstrokecolor{currentstroke}%
\pgfsetdash{}{0pt}%
\pgfpathmoveto{\pgfqpoint{0.998740in}{2.248779in}}%
\pgfpathlineto{\pgfqpoint{1.123740in}{2.248779in}}%
\pgfpathlineto{\pgfqpoint{1.248740in}{2.248779in}}%
\pgfusepath{stroke}%
\end{pgfscope}%
\begin{pgfscope}%
\definecolor{textcolor}{rgb}{0.000000,0.000000,0.000000}%
\pgfsetstrokecolor{textcolor}%
\pgfsetfillcolor{textcolor}%
\pgftext[x=1.348740in,y=2.205029in,left,base]{\color{textcolor}\rmfamily\fontsize{9.000000}{10.800000}\selectfont Solids}%
\end{pgfscope}%
\begin{pgfscope}%
\pgfsetbuttcap%
\pgfsetroundjoin%
\pgfsetlinewidth{1.003750pt}%
\definecolor{currentstroke}{rgb}{0.000000,0.000000,0.000000}%
\pgfsetstrokecolor{currentstroke}%
\pgfsetdash{{3.700000pt}{1.600000pt}}{0.000000pt}%
\pgfpathmoveto{\pgfqpoint{0.998740in}{2.065308in}}%
\pgfpathlineto{\pgfqpoint{1.123740in}{2.065308in}}%
\pgfpathlineto{\pgfqpoint{1.248740in}{2.065308in}}%
\pgfusepath{stroke}%
\end{pgfscope}%
\begin{pgfscope}%
\definecolor{textcolor}{rgb}{0.000000,0.000000,0.000000}%
\pgfsetstrokecolor{textcolor}%
\pgfsetfillcolor{textcolor}%
\pgftext[x=1.348740in,y=2.021558in,left,base]{\color{textcolor}\rmfamily\fontsize{9.000000}{10.800000}\selectfont Gases}%
\end{pgfscope}%
\end{pgfpicture}%
\makeatother%
\endgroup%

  \caption{Approximate critical energy~\eqref{eq:critical_energy} for elements as
    a function of the atomic number. (The solid line is for solids, while the
    dashed one is for gases.)}
  \label{fig:critical_energy}
\end{marginfigure}

To most practical purposes radiation losses are only relevant for electrons
(and positrons) and it is customary to refer to the critical energy for electrons
as \emph{the} critical energy. A popular empirical approximation for $E_c$ as a
function of the atomic number of the material is
\begin{align}\label{eq:critical_energy}
  E_c =
  \begin{cases}
    \displaystyle\frac{610}{Z + 1.24}~\text{MeV}\quad\text{for solids}\\[8pt]
    \displaystyle\frac{710}{Z + 0.92}~\text{MeV}\quad\text{for gases}
  \end{cases}
\end{align}
as shown in figure~\ref{fig:critical_energy}.


\subsection{The radiation length}%
\label{sec:radiation_length}

Above the critical energy, i.e., in the regime where radiation losses dominate,
equation~\eqref{eq:rad_losses} can readily integrated for a homogeneous material,
yielding the electron energy as a function of the distance $x$ traversed:
\begin{align}
  E(x) = E_0 \exp\left(-\frac{x}{X_0}\right).
\end{align}
The quantity $X_0$, representing the typical length over which an electron looses
all but $1/e$ of its energy due to \bremss, is called \emph{radiation length} and
is characteristic of the material. The radiation length is conveniently expressed
in g~cm$^{-2}$, i.e., factoring out the density of the medium, and a popular
parametrization is
\begin{align}
  X_0 = \frac{716~A}{Z(Z + 1) \ln\frac{287}{\sqrt{Z}}}~\text{g~cm}^{-2}.
\end{align}
Table~\ref{tab:exp_radlen} shows numerical values for a few materials of interest.

\begin{table}[htb!]
  \begin{tabular}{p{0.21\linewidth}p{0.21\linewidth}p{0.21\linewidth}%
      p{0.21\linewidth}}
    \hline
    Material & $X_0$~[g~cm$^{-2}$] & $\density$ [g~cm$^{-3}$] & $X_0$ [cm]\\
    \hline
    \hline
    Pb & 6.37 & 11.350 & 0.561\\
    BGO & 7.97 & 7.130 & 1.12\\
    CsI & 8.39 & 4.510 & 1.86\\
    W & 6.76 & 19.3 & 0.350\\
    C (graphite) & 42.70 & 2.210 & 19.3\\
    Si & 21.82 & 2.329 & 9.37\\
    Air & 36.62 & $1.2 \times 10^{-3}$ & 30,500\\
    \hline
  \end{tabular}
  \caption{Tabulated values of the radiation length for some materials of
    interest.}
  \label{tab:exp_radlen}
\end{table}

The radiation length is the natural scale for all the most relevant
electromagnetic phenomena we shall deal with in the following: multiple
scattering, electron \bremss, pair production and electromagnetic
showers. We shall customarily indicate with the letter $t = x/X_0$ any
distance measured in units of $X_0$.


\subsection{Multiple Coulomb scattering}%
\label{sec:inter_mcs}

A charged particle traversing a medium undergoes multiple Coulomb scatterings
on the atomic nuclei of the material. In the so-called gaussian approximation
the root mean square of the deviation angle projected in any of the planes
containing the incoming particle direction can be parameterized as
\begin{align}\label{eq:theta_ms}
  \theta^\text{rms}_\text{plane} =
  \frac{13.6~\text{MeV}}{\beta c p}z\sqrt{t} (1 + 0.038 \ln t)~\text{rad},
\end{align}
where $\beta c$, $p$ and $z$ are the velocity, momentum and charge number
of the incoming particle, and $t$ is the thickness of the traversed
material in units of radiation lengths.

\begin{marginfigure}
  %% Creator: Matplotlib, PGF backend
%%
%% To include the figure in your LaTeX document, write
%%   \input{<filename>.pgf}
%%
%% Make sure the required packages are loaded in your preamble
%%   \usepackage{pgf}
%%
%% Also ensure that all the required font packages are loaded; for instance,
%% the lmodern package is sometimes necessary when using math font.
%%   \usepackage{lmodern}
%%
%% Figures using additional raster images can only be included by \input if
%% they are in the same directory as the main LaTeX file. For loading figures
%% from other directories you can use the `import` package
%%   \usepackage{import}
%%
%% and then include the figures with
%%   \import{<path to file>}{<filename>.pgf}
%%
%% Matplotlib used the following preamble
%%   \usepackage{fontspec}
%%   \setmainfont{DejaVuSerif.ttf}[Path=\detokenize{/usr/share/matplotlib/mpl-data/fonts/ttf/}]
%%   \setsansfont{DejaVuSans.ttf}[Path=\detokenize{/usr/share/matplotlib/mpl-data/fonts/ttf/}]
%%   \setmonofont{DejaVuSansMono.ttf}[Path=\detokenize{/usr/share/matplotlib/mpl-data/fonts/ttf/}]
%%
\begingroup%
\makeatletter%
\begin{pgfpicture}%
\pgfpathrectangle{\pgfpointorigin}{\pgfqpoint{1.950000in}{2.500000in}}%
\pgfusepath{use as bounding box, clip}%
\begin{pgfscope}%
\pgfsetbuttcap%
\pgfsetmiterjoin%
\definecolor{currentfill}{rgb}{1.000000,1.000000,1.000000}%
\pgfsetfillcolor{currentfill}%
\pgfsetlinewidth{0.000000pt}%
\definecolor{currentstroke}{rgb}{1.000000,1.000000,1.000000}%
\pgfsetstrokecolor{currentstroke}%
\pgfsetdash{}{0pt}%
\pgfpathmoveto{\pgfqpoint{0.000000in}{0.000000in}}%
\pgfpathlineto{\pgfqpoint{1.950000in}{0.000000in}}%
\pgfpathlineto{\pgfqpoint{1.950000in}{2.500000in}}%
\pgfpathlineto{\pgfqpoint{0.000000in}{2.500000in}}%
\pgfpathlineto{\pgfqpoint{0.000000in}{0.000000in}}%
\pgfpathclose%
\pgfusepath{fill}%
\end{pgfscope}%
\begin{pgfscope}%
\pgfsetbuttcap%
\pgfsetmiterjoin%
\definecolor{currentfill}{rgb}{1.000000,1.000000,1.000000}%
\pgfsetfillcolor{currentfill}%
\pgfsetlinewidth{0.000000pt}%
\definecolor{currentstroke}{rgb}{0.000000,0.000000,0.000000}%
\pgfsetstrokecolor{currentstroke}%
\pgfsetstrokeopacity{0.000000}%
\pgfsetdash{}{0pt}%
\pgfpathmoveto{\pgfqpoint{0.536250in}{0.525000in}}%
\pgfpathlineto{\pgfqpoint{1.846250in}{0.525000in}}%
\pgfpathlineto{\pgfqpoint{1.846250in}{2.412500in}}%
\pgfpathlineto{\pgfqpoint{0.536250in}{2.412500in}}%
\pgfpathlineto{\pgfqpoint{0.536250in}{0.525000in}}%
\pgfpathclose%
\pgfusepath{fill}%
\end{pgfscope}%
\begin{pgfscope}%
\pgfpathrectangle{\pgfqpoint{0.536250in}{0.525000in}}{\pgfqpoint{1.310000in}{1.887500in}}%
\pgfusepath{clip}%
\pgfsetbuttcap%
\pgfsetroundjoin%
\pgfsetlinewidth{0.803000pt}%
\definecolor{currentstroke}{rgb}{0.752941,0.752941,0.752941}%
\pgfsetstrokecolor{currentstroke}%
\pgfsetdash{{2.960000pt}{1.280000pt}}{0.000000pt}%
\pgfpathmoveto{\pgfqpoint{0.536250in}{0.525000in}}%
\pgfpathlineto{\pgfqpoint{0.536250in}{2.412500in}}%
\pgfusepath{stroke}%
\end{pgfscope}%
\begin{pgfscope}%
\pgfsetbuttcap%
\pgfsetroundjoin%
\definecolor{currentfill}{rgb}{0.000000,0.000000,0.000000}%
\pgfsetfillcolor{currentfill}%
\pgfsetlinewidth{0.803000pt}%
\definecolor{currentstroke}{rgb}{0.000000,0.000000,0.000000}%
\pgfsetstrokecolor{currentstroke}%
\pgfsetdash{}{0pt}%
\pgfsys@defobject{currentmarker}{\pgfqpoint{0.000000in}{-0.048611in}}{\pgfqpoint{0.000000in}{0.000000in}}{%
\pgfpathmoveto{\pgfqpoint{0.000000in}{0.000000in}}%
\pgfpathlineto{\pgfqpoint{0.000000in}{-0.048611in}}%
\pgfusepath{stroke,fill}%
}%
\begin{pgfscope}%
\pgfsys@transformshift{0.536250in}{0.525000in}%
\pgfsys@useobject{currentmarker}{}%
\end{pgfscope}%
\end{pgfscope}%
\begin{pgfscope}%
\definecolor{textcolor}{rgb}{0.000000,0.000000,0.000000}%
\pgfsetstrokecolor{textcolor}%
\pgfsetfillcolor{textcolor}%
\pgftext[x=0.536250in,y=0.427778in,,top]{\color{textcolor}\rmfamily\fontsize{9.000000}{10.800000}\selectfont \(\displaystyle {10^{2}}\)}%
\end{pgfscope}%
\begin{pgfscope}%
\pgfpathrectangle{\pgfqpoint{0.536250in}{0.525000in}}{\pgfqpoint{1.310000in}{1.887500in}}%
\pgfusepath{clip}%
\pgfsetbuttcap%
\pgfsetroundjoin%
\pgfsetlinewidth{0.803000pt}%
\definecolor{currentstroke}{rgb}{0.752941,0.752941,0.752941}%
\pgfsetstrokecolor{currentstroke}%
\pgfsetdash{{2.960000pt}{1.280000pt}}{0.000000pt}%
\pgfpathmoveto{\pgfqpoint{0.863750in}{0.525000in}}%
\pgfpathlineto{\pgfqpoint{0.863750in}{2.412500in}}%
\pgfusepath{stroke}%
\end{pgfscope}%
\begin{pgfscope}%
\pgfsetbuttcap%
\pgfsetroundjoin%
\definecolor{currentfill}{rgb}{0.000000,0.000000,0.000000}%
\pgfsetfillcolor{currentfill}%
\pgfsetlinewidth{0.803000pt}%
\definecolor{currentstroke}{rgb}{0.000000,0.000000,0.000000}%
\pgfsetstrokecolor{currentstroke}%
\pgfsetdash{}{0pt}%
\pgfsys@defobject{currentmarker}{\pgfqpoint{0.000000in}{-0.048611in}}{\pgfqpoint{0.000000in}{0.000000in}}{%
\pgfpathmoveto{\pgfqpoint{0.000000in}{0.000000in}}%
\pgfpathlineto{\pgfqpoint{0.000000in}{-0.048611in}}%
\pgfusepath{stroke,fill}%
}%
\begin{pgfscope}%
\pgfsys@transformshift{0.863750in}{0.525000in}%
\pgfsys@useobject{currentmarker}{}%
\end{pgfscope}%
\end{pgfscope}%
\begin{pgfscope}%
\definecolor{textcolor}{rgb}{0.000000,0.000000,0.000000}%
\pgfsetstrokecolor{textcolor}%
\pgfsetfillcolor{textcolor}%
\pgftext[x=0.863750in,y=0.427778in,,top]{\color{textcolor}\rmfamily\fontsize{9.000000}{10.800000}\selectfont \(\displaystyle {10^{3}}\)}%
\end{pgfscope}%
\begin{pgfscope}%
\pgfpathrectangle{\pgfqpoint{0.536250in}{0.525000in}}{\pgfqpoint{1.310000in}{1.887500in}}%
\pgfusepath{clip}%
\pgfsetbuttcap%
\pgfsetroundjoin%
\pgfsetlinewidth{0.803000pt}%
\definecolor{currentstroke}{rgb}{0.752941,0.752941,0.752941}%
\pgfsetstrokecolor{currentstroke}%
\pgfsetdash{{2.960000pt}{1.280000pt}}{0.000000pt}%
\pgfpathmoveto{\pgfqpoint{1.191250in}{0.525000in}}%
\pgfpathlineto{\pgfqpoint{1.191250in}{2.412500in}}%
\pgfusepath{stroke}%
\end{pgfscope}%
\begin{pgfscope}%
\pgfsetbuttcap%
\pgfsetroundjoin%
\definecolor{currentfill}{rgb}{0.000000,0.000000,0.000000}%
\pgfsetfillcolor{currentfill}%
\pgfsetlinewidth{0.803000pt}%
\definecolor{currentstroke}{rgb}{0.000000,0.000000,0.000000}%
\pgfsetstrokecolor{currentstroke}%
\pgfsetdash{}{0pt}%
\pgfsys@defobject{currentmarker}{\pgfqpoint{0.000000in}{-0.048611in}}{\pgfqpoint{0.000000in}{0.000000in}}{%
\pgfpathmoveto{\pgfqpoint{0.000000in}{0.000000in}}%
\pgfpathlineto{\pgfqpoint{0.000000in}{-0.048611in}}%
\pgfusepath{stroke,fill}%
}%
\begin{pgfscope}%
\pgfsys@transformshift{1.191250in}{0.525000in}%
\pgfsys@useobject{currentmarker}{}%
\end{pgfscope}%
\end{pgfscope}%
\begin{pgfscope}%
\definecolor{textcolor}{rgb}{0.000000,0.000000,0.000000}%
\pgfsetstrokecolor{textcolor}%
\pgfsetfillcolor{textcolor}%
\pgftext[x=1.191250in,y=0.427778in,,top]{\color{textcolor}\rmfamily\fontsize{9.000000}{10.800000}\selectfont \(\displaystyle {10^{4}}\)}%
\end{pgfscope}%
\begin{pgfscope}%
\pgfpathrectangle{\pgfqpoint{0.536250in}{0.525000in}}{\pgfqpoint{1.310000in}{1.887500in}}%
\pgfusepath{clip}%
\pgfsetbuttcap%
\pgfsetroundjoin%
\pgfsetlinewidth{0.803000pt}%
\definecolor{currentstroke}{rgb}{0.752941,0.752941,0.752941}%
\pgfsetstrokecolor{currentstroke}%
\pgfsetdash{{2.960000pt}{1.280000pt}}{0.000000pt}%
\pgfpathmoveto{\pgfqpoint{1.518750in}{0.525000in}}%
\pgfpathlineto{\pgfqpoint{1.518750in}{2.412500in}}%
\pgfusepath{stroke}%
\end{pgfscope}%
\begin{pgfscope}%
\pgfsetbuttcap%
\pgfsetroundjoin%
\definecolor{currentfill}{rgb}{0.000000,0.000000,0.000000}%
\pgfsetfillcolor{currentfill}%
\pgfsetlinewidth{0.803000pt}%
\definecolor{currentstroke}{rgb}{0.000000,0.000000,0.000000}%
\pgfsetstrokecolor{currentstroke}%
\pgfsetdash{}{0pt}%
\pgfsys@defobject{currentmarker}{\pgfqpoint{0.000000in}{-0.048611in}}{\pgfqpoint{0.000000in}{0.000000in}}{%
\pgfpathmoveto{\pgfqpoint{0.000000in}{0.000000in}}%
\pgfpathlineto{\pgfqpoint{0.000000in}{-0.048611in}}%
\pgfusepath{stroke,fill}%
}%
\begin{pgfscope}%
\pgfsys@transformshift{1.518750in}{0.525000in}%
\pgfsys@useobject{currentmarker}{}%
\end{pgfscope}%
\end{pgfscope}%
\begin{pgfscope}%
\definecolor{textcolor}{rgb}{0.000000,0.000000,0.000000}%
\pgfsetstrokecolor{textcolor}%
\pgfsetfillcolor{textcolor}%
\pgftext[x=1.518750in,y=0.427778in,,top]{\color{textcolor}\rmfamily\fontsize{9.000000}{10.800000}\selectfont \(\displaystyle {10^{5}}\)}%
\end{pgfscope}%
\begin{pgfscope}%
\pgfpathrectangle{\pgfqpoint{0.536250in}{0.525000in}}{\pgfqpoint{1.310000in}{1.887500in}}%
\pgfusepath{clip}%
\pgfsetbuttcap%
\pgfsetroundjoin%
\pgfsetlinewidth{0.803000pt}%
\definecolor{currentstroke}{rgb}{0.752941,0.752941,0.752941}%
\pgfsetstrokecolor{currentstroke}%
\pgfsetdash{{2.960000pt}{1.280000pt}}{0.000000pt}%
\pgfpathmoveto{\pgfqpoint{1.846250in}{0.525000in}}%
\pgfpathlineto{\pgfqpoint{1.846250in}{2.412500in}}%
\pgfusepath{stroke}%
\end{pgfscope}%
\begin{pgfscope}%
\pgfsetbuttcap%
\pgfsetroundjoin%
\definecolor{currentfill}{rgb}{0.000000,0.000000,0.000000}%
\pgfsetfillcolor{currentfill}%
\pgfsetlinewidth{0.803000pt}%
\definecolor{currentstroke}{rgb}{0.000000,0.000000,0.000000}%
\pgfsetstrokecolor{currentstroke}%
\pgfsetdash{}{0pt}%
\pgfsys@defobject{currentmarker}{\pgfqpoint{0.000000in}{-0.048611in}}{\pgfqpoint{0.000000in}{0.000000in}}{%
\pgfpathmoveto{\pgfqpoint{0.000000in}{0.000000in}}%
\pgfpathlineto{\pgfqpoint{0.000000in}{-0.048611in}}%
\pgfusepath{stroke,fill}%
}%
\begin{pgfscope}%
\pgfsys@transformshift{1.846250in}{0.525000in}%
\pgfsys@useobject{currentmarker}{}%
\end{pgfscope}%
\end{pgfscope}%
\begin{pgfscope}%
\definecolor{textcolor}{rgb}{0.000000,0.000000,0.000000}%
\pgfsetstrokecolor{textcolor}%
\pgfsetfillcolor{textcolor}%
\pgftext[x=1.846250in,y=0.427778in,,top]{\color{textcolor}\rmfamily\fontsize{9.000000}{10.800000}\selectfont \(\displaystyle {10^{6}}\)}%
\end{pgfscope}%
\begin{pgfscope}%
\definecolor{textcolor}{rgb}{0.000000,0.000000,0.000000}%
\pgfsetstrokecolor{textcolor}%
\pgfsetfillcolor{textcolor}%
\pgftext[x=1.191250in,y=0.251251in,,top]{\color{textcolor}\rmfamily\fontsize{9.000000}{10.800000}\selectfont Momentum [MeV/c]}%
\end{pgfscope}%
\begin{pgfscope}%
\pgfpathrectangle{\pgfqpoint{0.536250in}{0.525000in}}{\pgfqpoint{1.310000in}{1.887500in}}%
\pgfusepath{clip}%
\pgfsetbuttcap%
\pgfsetroundjoin%
\pgfsetlinewidth{0.803000pt}%
\definecolor{currentstroke}{rgb}{0.752941,0.752941,0.752941}%
\pgfsetstrokecolor{currentstroke}%
\pgfsetdash{{2.960000pt}{1.280000pt}}{0.000000pt}%
\pgfpathmoveto{\pgfqpoint{0.536250in}{0.525000in}}%
\pgfpathlineto{\pgfqpoint{1.846250in}{0.525000in}}%
\pgfusepath{stroke}%
\end{pgfscope}%
\begin{pgfscope}%
\pgfsetbuttcap%
\pgfsetroundjoin%
\definecolor{currentfill}{rgb}{0.000000,0.000000,0.000000}%
\pgfsetfillcolor{currentfill}%
\pgfsetlinewidth{0.803000pt}%
\definecolor{currentstroke}{rgb}{0.000000,0.000000,0.000000}%
\pgfsetstrokecolor{currentstroke}%
\pgfsetdash{}{0pt}%
\pgfsys@defobject{currentmarker}{\pgfqpoint{-0.048611in}{0.000000in}}{\pgfqpoint{-0.000000in}{0.000000in}}{%
\pgfpathmoveto{\pgfqpoint{-0.000000in}{0.000000in}}%
\pgfpathlineto{\pgfqpoint{-0.048611in}{0.000000in}}%
\pgfusepath{stroke,fill}%
}%
\begin{pgfscope}%
\pgfsys@transformshift{0.536250in}{0.525000in}%
\pgfsys@useobject{currentmarker}{}%
\end{pgfscope}%
\end{pgfscope}%
\begin{pgfscope}%
\definecolor{textcolor}{rgb}{0.000000,0.000000,0.000000}%
\pgfsetstrokecolor{textcolor}%
\pgfsetfillcolor{textcolor}%
\pgftext[x=0.172441in, y=0.477515in, left, base]{\color{textcolor}\rmfamily\fontsize{9.000000}{10.800000}\selectfont \(\displaystyle {10^{-4}}\)}%
\end{pgfscope}%
\begin{pgfscope}%
\pgfpathrectangle{\pgfqpoint{0.536250in}{0.525000in}}{\pgfqpoint{1.310000in}{1.887500in}}%
\pgfusepath{clip}%
\pgfsetbuttcap%
\pgfsetroundjoin%
\pgfsetlinewidth{0.803000pt}%
\definecolor{currentstroke}{rgb}{0.752941,0.752941,0.752941}%
\pgfsetstrokecolor{currentstroke}%
\pgfsetdash{{2.960000pt}{1.280000pt}}{0.000000pt}%
\pgfpathmoveto{\pgfqpoint{0.536250in}{0.839583in}}%
\pgfpathlineto{\pgfqpoint{1.846250in}{0.839583in}}%
\pgfusepath{stroke}%
\end{pgfscope}%
\begin{pgfscope}%
\pgfsetbuttcap%
\pgfsetroundjoin%
\definecolor{currentfill}{rgb}{0.000000,0.000000,0.000000}%
\pgfsetfillcolor{currentfill}%
\pgfsetlinewidth{0.803000pt}%
\definecolor{currentstroke}{rgb}{0.000000,0.000000,0.000000}%
\pgfsetstrokecolor{currentstroke}%
\pgfsetdash{}{0pt}%
\pgfsys@defobject{currentmarker}{\pgfqpoint{-0.048611in}{0.000000in}}{\pgfqpoint{-0.000000in}{0.000000in}}{%
\pgfpathmoveto{\pgfqpoint{-0.000000in}{0.000000in}}%
\pgfpathlineto{\pgfqpoint{-0.048611in}{0.000000in}}%
\pgfusepath{stroke,fill}%
}%
\begin{pgfscope}%
\pgfsys@transformshift{0.536250in}{0.839583in}%
\pgfsys@useobject{currentmarker}{}%
\end{pgfscope}%
\end{pgfscope}%
\begin{pgfscope}%
\definecolor{textcolor}{rgb}{0.000000,0.000000,0.000000}%
\pgfsetstrokecolor{textcolor}%
\pgfsetfillcolor{textcolor}%
\pgftext[x=0.172441in, y=0.792098in, left, base]{\color{textcolor}\rmfamily\fontsize{9.000000}{10.800000}\selectfont \(\displaystyle {10^{-3}}\)}%
\end{pgfscope}%
\begin{pgfscope}%
\pgfpathrectangle{\pgfqpoint{0.536250in}{0.525000in}}{\pgfqpoint{1.310000in}{1.887500in}}%
\pgfusepath{clip}%
\pgfsetbuttcap%
\pgfsetroundjoin%
\pgfsetlinewidth{0.803000pt}%
\definecolor{currentstroke}{rgb}{0.752941,0.752941,0.752941}%
\pgfsetstrokecolor{currentstroke}%
\pgfsetdash{{2.960000pt}{1.280000pt}}{0.000000pt}%
\pgfpathmoveto{\pgfqpoint{0.536250in}{1.154167in}}%
\pgfpathlineto{\pgfqpoint{1.846250in}{1.154167in}}%
\pgfusepath{stroke}%
\end{pgfscope}%
\begin{pgfscope}%
\pgfsetbuttcap%
\pgfsetroundjoin%
\definecolor{currentfill}{rgb}{0.000000,0.000000,0.000000}%
\pgfsetfillcolor{currentfill}%
\pgfsetlinewidth{0.803000pt}%
\definecolor{currentstroke}{rgb}{0.000000,0.000000,0.000000}%
\pgfsetstrokecolor{currentstroke}%
\pgfsetdash{}{0pt}%
\pgfsys@defobject{currentmarker}{\pgfqpoint{-0.048611in}{0.000000in}}{\pgfqpoint{-0.000000in}{0.000000in}}{%
\pgfpathmoveto{\pgfqpoint{-0.000000in}{0.000000in}}%
\pgfpathlineto{\pgfqpoint{-0.048611in}{0.000000in}}%
\pgfusepath{stroke,fill}%
}%
\begin{pgfscope}%
\pgfsys@transformshift{0.536250in}{1.154167in}%
\pgfsys@useobject{currentmarker}{}%
\end{pgfscope}%
\end{pgfscope}%
\begin{pgfscope}%
\definecolor{textcolor}{rgb}{0.000000,0.000000,0.000000}%
\pgfsetstrokecolor{textcolor}%
\pgfsetfillcolor{textcolor}%
\pgftext[x=0.172441in, y=1.106681in, left, base]{\color{textcolor}\rmfamily\fontsize{9.000000}{10.800000}\selectfont \(\displaystyle {10^{-2}}\)}%
\end{pgfscope}%
\begin{pgfscope}%
\pgfpathrectangle{\pgfqpoint{0.536250in}{0.525000in}}{\pgfqpoint{1.310000in}{1.887500in}}%
\pgfusepath{clip}%
\pgfsetbuttcap%
\pgfsetroundjoin%
\pgfsetlinewidth{0.803000pt}%
\definecolor{currentstroke}{rgb}{0.752941,0.752941,0.752941}%
\pgfsetstrokecolor{currentstroke}%
\pgfsetdash{{2.960000pt}{1.280000pt}}{0.000000pt}%
\pgfpathmoveto{\pgfqpoint{0.536250in}{1.468750in}}%
\pgfpathlineto{\pgfqpoint{1.846250in}{1.468750in}}%
\pgfusepath{stroke}%
\end{pgfscope}%
\begin{pgfscope}%
\pgfsetbuttcap%
\pgfsetroundjoin%
\definecolor{currentfill}{rgb}{0.000000,0.000000,0.000000}%
\pgfsetfillcolor{currentfill}%
\pgfsetlinewidth{0.803000pt}%
\definecolor{currentstroke}{rgb}{0.000000,0.000000,0.000000}%
\pgfsetstrokecolor{currentstroke}%
\pgfsetdash{}{0pt}%
\pgfsys@defobject{currentmarker}{\pgfqpoint{-0.048611in}{0.000000in}}{\pgfqpoint{-0.000000in}{0.000000in}}{%
\pgfpathmoveto{\pgfqpoint{-0.000000in}{0.000000in}}%
\pgfpathlineto{\pgfqpoint{-0.048611in}{0.000000in}}%
\pgfusepath{stroke,fill}%
}%
\begin{pgfscope}%
\pgfsys@transformshift{0.536250in}{1.468750in}%
\pgfsys@useobject{currentmarker}{}%
\end{pgfscope}%
\end{pgfscope}%
\begin{pgfscope}%
\definecolor{textcolor}{rgb}{0.000000,0.000000,0.000000}%
\pgfsetstrokecolor{textcolor}%
\pgfsetfillcolor{textcolor}%
\pgftext[x=0.172441in, y=1.421265in, left, base]{\color{textcolor}\rmfamily\fontsize{9.000000}{10.800000}\selectfont \(\displaystyle {10^{-1}}\)}%
\end{pgfscope}%
\begin{pgfscope}%
\pgfpathrectangle{\pgfqpoint{0.536250in}{0.525000in}}{\pgfqpoint{1.310000in}{1.887500in}}%
\pgfusepath{clip}%
\pgfsetbuttcap%
\pgfsetroundjoin%
\pgfsetlinewidth{0.803000pt}%
\definecolor{currentstroke}{rgb}{0.752941,0.752941,0.752941}%
\pgfsetstrokecolor{currentstroke}%
\pgfsetdash{{2.960000pt}{1.280000pt}}{0.000000pt}%
\pgfpathmoveto{\pgfqpoint{0.536250in}{1.783333in}}%
\pgfpathlineto{\pgfqpoint{1.846250in}{1.783333in}}%
\pgfusepath{stroke}%
\end{pgfscope}%
\begin{pgfscope}%
\pgfsetbuttcap%
\pgfsetroundjoin%
\definecolor{currentfill}{rgb}{0.000000,0.000000,0.000000}%
\pgfsetfillcolor{currentfill}%
\pgfsetlinewidth{0.803000pt}%
\definecolor{currentstroke}{rgb}{0.000000,0.000000,0.000000}%
\pgfsetstrokecolor{currentstroke}%
\pgfsetdash{}{0pt}%
\pgfsys@defobject{currentmarker}{\pgfqpoint{-0.048611in}{0.000000in}}{\pgfqpoint{-0.000000in}{0.000000in}}{%
\pgfpathmoveto{\pgfqpoint{-0.000000in}{0.000000in}}%
\pgfpathlineto{\pgfqpoint{-0.048611in}{0.000000in}}%
\pgfusepath{stroke,fill}%
}%
\begin{pgfscope}%
\pgfsys@transformshift{0.536250in}{1.783333in}%
\pgfsys@useobject{currentmarker}{}%
\end{pgfscope}%
\end{pgfscope}%
\begin{pgfscope}%
\definecolor{textcolor}{rgb}{0.000000,0.000000,0.000000}%
\pgfsetstrokecolor{textcolor}%
\pgfsetfillcolor{textcolor}%
\pgftext[x=0.252687in, y=1.735848in, left, base]{\color{textcolor}\rmfamily\fontsize{9.000000}{10.800000}\selectfont \(\displaystyle {10^{0}}\)}%
\end{pgfscope}%
\begin{pgfscope}%
\pgfpathrectangle{\pgfqpoint{0.536250in}{0.525000in}}{\pgfqpoint{1.310000in}{1.887500in}}%
\pgfusepath{clip}%
\pgfsetbuttcap%
\pgfsetroundjoin%
\pgfsetlinewidth{0.803000pt}%
\definecolor{currentstroke}{rgb}{0.752941,0.752941,0.752941}%
\pgfsetstrokecolor{currentstroke}%
\pgfsetdash{{2.960000pt}{1.280000pt}}{0.000000pt}%
\pgfpathmoveto{\pgfqpoint{0.536250in}{2.097917in}}%
\pgfpathlineto{\pgfqpoint{1.846250in}{2.097917in}}%
\pgfusepath{stroke}%
\end{pgfscope}%
\begin{pgfscope}%
\pgfsetbuttcap%
\pgfsetroundjoin%
\definecolor{currentfill}{rgb}{0.000000,0.000000,0.000000}%
\pgfsetfillcolor{currentfill}%
\pgfsetlinewidth{0.803000pt}%
\definecolor{currentstroke}{rgb}{0.000000,0.000000,0.000000}%
\pgfsetstrokecolor{currentstroke}%
\pgfsetdash{}{0pt}%
\pgfsys@defobject{currentmarker}{\pgfqpoint{-0.048611in}{0.000000in}}{\pgfqpoint{-0.000000in}{0.000000in}}{%
\pgfpathmoveto{\pgfqpoint{-0.000000in}{0.000000in}}%
\pgfpathlineto{\pgfqpoint{-0.048611in}{0.000000in}}%
\pgfusepath{stroke,fill}%
}%
\begin{pgfscope}%
\pgfsys@transformshift{0.536250in}{2.097917in}%
\pgfsys@useobject{currentmarker}{}%
\end{pgfscope}%
\end{pgfscope}%
\begin{pgfscope}%
\definecolor{textcolor}{rgb}{0.000000,0.000000,0.000000}%
\pgfsetstrokecolor{textcolor}%
\pgfsetfillcolor{textcolor}%
\pgftext[x=0.252687in, y=2.050431in, left, base]{\color{textcolor}\rmfamily\fontsize{9.000000}{10.800000}\selectfont \(\displaystyle {10^{1}}\)}%
\end{pgfscope}%
\begin{pgfscope}%
\pgfpathrectangle{\pgfqpoint{0.536250in}{0.525000in}}{\pgfqpoint{1.310000in}{1.887500in}}%
\pgfusepath{clip}%
\pgfsetbuttcap%
\pgfsetroundjoin%
\pgfsetlinewidth{0.803000pt}%
\definecolor{currentstroke}{rgb}{0.752941,0.752941,0.752941}%
\pgfsetstrokecolor{currentstroke}%
\pgfsetdash{{2.960000pt}{1.280000pt}}{0.000000pt}%
\pgfpathmoveto{\pgfqpoint{0.536250in}{2.412500in}}%
\pgfpathlineto{\pgfqpoint{1.846250in}{2.412500in}}%
\pgfusepath{stroke}%
\end{pgfscope}%
\begin{pgfscope}%
\pgfsetbuttcap%
\pgfsetroundjoin%
\definecolor{currentfill}{rgb}{0.000000,0.000000,0.000000}%
\pgfsetfillcolor{currentfill}%
\pgfsetlinewidth{0.803000pt}%
\definecolor{currentstroke}{rgb}{0.000000,0.000000,0.000000}%
\pgfsetstrokecolor{currentstroke}%
\pgfsetdash{}{0pt}%
\pgfsys@defobject{currentmarker}{\pgfqpoint{-0.048611in}{0.000000in}}{\pgfqpoint{-0.000000in}{0.000000in}}{%
\pgfpathmoveto{\pgfqpoint{-0.000000in}{0.000000in}}%
\pgfpathlineto{\pgfqpoint{-0.048611in}{0.000000in}}%
\pgfusepath{stroke,fill}%
}%
\begin{pgfscope}%
\pgfsys@transformshift{0.536250in}{2.412500in}%
\pgfsys@useobject{currentmarker}{}%
\end{pgfscope}%
\end{pgfscope}%
\begin{pgfscope}%
\definecolor{textcolor}{rgb}{0.000000,0.000000,0.000000}%
\pgfsetstrokecolor{textcolor}%
\pgfsetfillcolor{textcolor}%
\pgftext[x=0.252687in, y=2.365015in, left, base]{\color{textcolor}\rmfamily\fontsize{9.000000}{10.800000}\selectfont \(\displaystyle {10^{2}}\)}%
\end{pgfscope}%
\begin{pgfscope}%
\pgfpathrectangle{\pgfqpoint{0.536250in}{0.525000in}}{\pgfqpoint{1.310000in}{1.887500in}}%
\pgfusepath{clip}%
\pgfsetbuttcap%
\pgfsetroundjoin%
\pgfsetlinewidth{0.803000pt}%
\definecolor{currentstroke}{rgb}{0.752941,0.752941,0.752941}%
\pgfsetstrokecolor{currentstroke}%
\pgfsetdash{{2.960000pt}{1.280000pt}}{0.000000pt}%
\pgfpathmoveto{\pgfqpoint{0.536250in}{0.619699in}}%
\pgfpathlineto{\pgfqpoint{1.846250in}{0.619699in}}%
\pgfusepath{stroke}%
\end{pgfscope}%
\begin{pgfscope}%
\pgfsetbuttcap%
\pgfsetroundjoin%
\definecolor{currentfill}{rgb}{0.000000,0.000000,0.000000}%
\pgfsetfillcolor{currentfill}%
\pgfsetlinewidth{0.602250pt}%
\definecolor{currentstroke}{rgb}{0.000000,0.000000,0.000000}%
\pgfsetstrokecolor{currentstroke}%
\pgfsetdash{}{0pt}%
\pgfsys@defobject{currentmarker}{\pgfqpoint{-0.027778in}{0.000000in}}{\pgfqpoint{-0.000000in}{0.000000in}}{%
\pgfpathmoveto{\pgfqpoint{-0.000000in}{0.000000in}}%
\pgfpathlineto{\pgfqpoint{-0.027778in}{0.000000in}}%
\pgfusepath{stroke,fill}%
}%
\begin{pgfscope}%
\pgfsys@transformshift{0.536250in}{0.619699in}%
\pgfsys@useobject{currentmarker}{}%
\end{pgfscope}%
\end{pgfscope}%
\begin{pgfscope}%
\pgfpathrectangle{\pgfqpoint{0.536250in}{0.525000in}}{\pgfqpoint{1.310000in}{1.887500in}}%
\pgfusepath{clip}%
\pgfsetbuttcap%
\pgfsetroundjoin%
\pgfsetlinewidth{0.803000pt}%
\definecolor{currentstroke}{rgb}{0.752941,0.752941,0.752941}%
\pgfsetstrokecolor{currentstroke}%
\pgfsetdash{{2.960000pt}{1.280000pt}}{0.000000pt}%
\pgfpathmoveto{\pgfqpoint{0.536250in}{0.675094in}}%
\pgfpathlineto{\pgfqpoint{1.846250in}{0.675094in}}%
\pgfusepath{stroke}%
\end{pgfscope}%
\begin{pgfscope}%
\pgfsetbuttcap%
\pgfsetroundjoin%
\definecolor{currentfill}{rgb}{0.000000,0.000000,0.000000}%
\pgfsetfillcolor{currentfill}%
\pgfsetlinewidth{0.602250pt}%
\definecolor{currentstroke}{rgb}{0.000000,0.000000,0.000000}%
\pgfsetstrokecolor{currentstroke}%
\pgfsetdash{}{0pt}%
\pgfsys@defobject{currentmarker}{\pgfqpoint{-0.027778in}{0.000000in}}{\pgfqpoint{-0.000000in}{0.000000in}}{%
\pgfpathmoveto{\pgfqpoint{-0.000000in}{0.000000in}}%
\pgfpathlineto{\pgfqpoint{-0.027778in}{0.000000in}}%
\pgfusepath{stroke,fill}%
}%
\begin{pgfscope}%
\pgfsys@transformshift{0.536250in}{0.675094in}%
\pgfsys@useobject{currentmarker}{}%
\end{pgfscope}%
\end{pgfscope}%
\begin{pgfscope}%
\pgfpathrectangle{\pgfqpoint{0.536250in}{0.525000in}}{\pgfqpoint{1.310000in}{1.887500in}}%
\pgfusepath{clip}%
\pgfsetbuttcap%
\pgfsetroundjoin%
\pgfsetlinewidth{0.803000pt}%
\definecolor{currentstroke}{rgb}{0.752941,0.752941,0.752941}%
\pgfsetstrokecolor{currentstroke}%
\pgfsetdash{{2.960000pt}{1.280000pt}}{0.000000pt}%
\pgfpathmoveto{\pgfqpoint{0.536250in}{0.714398in}}%
\pgfpathlineto{\pgfqpoint{1.846250in}{0.714398in}}%
\pgfusepath{stroke}%
\end{pgfscope}%
\begin{pgfscope}%
\pgfsetbuttcap%
\pgfsetroundjoin%
\definecolor{currentfill}{rgb}{0.000000,0.000000,0.000000}%
\pgfsetfillcolor{currentfill}%
\pgfsetlinewidth{0.602250pt}%
\definecolor{currentstroke}{rgb}{0.000000,0.000000,0.000000}%
\pgfsetstrokecolor{currentstroke}%
\pgfsetdash{}{0pt}%
\pgfsys@defobject{currentmarker}{\pgfqpoint{-0.027778in}{0.000000in}}{\pgfqpoint{-0.000000in}{0.000000in}}{%
\pgfpathmoveto{\pgfqpoint{-0.000000in}{0.000000in}}%
\pgfpathlineto{\pgfqpoint{-0.027778in}{0.000000in}}%
\pgfusepath{stroke,fill}%
}%
\begin{pgfscope}%
\pgfsys@transformshift{0.536250in}{0.714398in}%
\pgfsys@useobject{currentmarker}{}%
\end{pgfscope}%
\end{pgfscope}%
\begin{pgfscope}%
\pgfpathrectangle{\pgfqpoint{0.536250in}{0.525000in}}{\pgfqpoint{1.310000in}{1.887500in}}%
\pgfusepath{clip}%
\pgfsetbuttcap%
\pgfsetroundjoin%
\pgfsetlinewidth{0.803000pt}%
\definecolor{currentstroke}{rgb}{0.752941,0.752941,0.752941}%
\pgfsetstrokecolor{currentstroke}%
\pgfsetdash{{2.960000pt}{1.280000pt}}{0.000000pt}%
\pgfpathmoveto{\pgfqpoint{0.536250in}{0.744884in}}%
\pgfpathlineto{\pgfqpoint{1.846250in}{0.744884in}}%
\pgfusepath{stroke}%
\end{pgfscope}%
\begin{pgfscope}%
\pgfsetbuttcap%
\pgfsetroundjoin%
\definecolor{currentfill}{rgb}{0.000000,0.000000,0.000000}%
\pgfsetfillcolor{currentfill}%
\pgfsetlinewidth{0.602250pt}%
\definecolor{currentstroke}{rgb}{0.000000,0.000000,0.000000}%
\pgfsetstrokecolor{currentstroke}%
\pgfsetdash{}{0pt}%
\pgfsys@defobject{currentmarker}{\pgfqpoint{-0.027778in}{0.000000in}}{\pgfqpoint{-0.000000in}{0.000000in}}{%
\pgfpathmoveto{\pgfqpoint{-0.000000in}{0.000000in}}%
\pgfpathlineto{\pgfqpoint{-0.027778in}{0.000000in}}%
\pgfusepath{stroke,fill}%
}%
\begin{pgfscope}%
\pgfsys@transformshift{0.536250in}{0.744884in}%
\pgfsys@useobject{currentmarker}{}%
\end{pgfscope}%
\end{pgfscope}%
\begin{pgfscope}%
\pgfpathrectangle{\pgfqpoint{0.536250in}{0.525000in}}{\pgfqpoint{1.310000in}{1.887500in}}%
\pgfusepath{clip}%
\pgfsetbuttcap%
\pgfsetroundjoin%
\pgfsetlinewidth{0.803000pt}%
\definecolor{currentstroke}{rgb}{0.752941,0.752941,0.752941}%
\pgfsetstrokecolor{currentstroke}%
\pgfsetdash{{2.960000pt}{1.280000pt}}{0.000000pt}%
\pgfpathmoveto{\pgfqpoint{0.536250in}{0.769793in}}%
\pgfpathlineto{\pgfqpoint{1.846250in}{0.769793in}}%
\pgfusepath{stroke}%
\end{pgfscope}%
\begin{pgfscope}%
\pgfsetbuttcap%
\pgfsetroundjoin%
\definecolor{currentfill}{rgb}{0.000000,0.000000,0.000000}%
\pgfsetfillcolor{currentfill}%
\pgfsetlinewidth{0.602250pt}%
\definecolor{currentstroke}{rgb}{0.000000,0.000000,0.000000}%
\pgfsetstrokecolor{currentstroke}%
\pgfsetdash{}{0pt}%
\pgfsys@defobject{currentmarker}{\pgfqpoint{-0.027778in}{0.000000in}}{\pgfqpoint{-0.000000in}{0.000000in}}{%
\pgfpathmoveto{\pgfqpoint{-0.000000in}{0.000000in}}%
\pgfpathlineto{\pgfqpoint{-0.027778in}{0.000000in}}%
\pgfusepath{stroke,fill}%
}%
\begin{pgfscope}%
\pgfsys@transformshift{0.536250in}{0.769793in}%
\pgfsys@useobject{currentmarker}{}%
\end{pgfscope}%
\end{pgfscope}%
\begin{pgfscope}%
\pgfpathrectangle{\pgfqpoint{0.536250in}{0.525000in}}{\pgfqpoint{1.310000in}{1.887500in}}%
\pgfusepath{clip}%
\pgfsetbuttcap%
\pgfsetroundjoin%
\pgfsetlinewidth{0.803000pt}%
\definecolor{currentstroke}{rgb}{0.752941,0.752941,0.752941}%
\pgfsetstrokecolor{currentstroke}%
\pgfsetdash{{2.960000pt}{1.280000pt}}{0.000000pt}%
\pgfpathmoveto{\pgfqpoint{0.536250in}{0.790854in}}%
\pgfpathlineto{\pgfqpoint{1.846250in}{0.790854in}}%
\pgfusepath{stroke}%
\end{pgfscope}%
\begin{pgfscope}%
\pgfsetbuttcap%
\pgfsetroundjoin%
\definecolor{currentfill}{rgb}{0.000000,0.000000,0.000000}%
\pgfsetfillcolor{currentfill}%
\pgfsetlinewidth{0.602250pt}%
\definecolor{currentstroke}{rgb}{0.000000,0.000000,0.000000}%
\pgfsetstrokecolor{currentstroke}%
\pgfsetdash{}{0pt}%
\pgfsys@defobject{currentmarker}{\pgfqpoint{-0.027778in}{0.000000in}}{\pgfqpoint{-0.000000in}{0.000000in}}{%
\pgfpathmoveto{\pgfqpoint{-0.000000in}{0.000000in}}%
\pgfpathlineto{\pgfqpoint{-0.027778in}{0.000000in}}%
\pgfusepath{stroke,fill}%
}%
\begin{pgfscope}%
\pgfsys@transformshift{0.536250in}{0.790854in}%
\pgfsys@useobject{currentmarker}{}%
\end{pgfscope}%
\end{pgfscope}%
\begin{pgfscope}%
\pgfpathrectangle{\pgfqpoint{0.536250in}{0.525000in}}{\pgfqpoint{1.310000in}{1.887500in}}%
\pgfusepath{clip}%
\pgfsetbuttcap%
\pgfsetroundjoin%
\pgfsetlinewidth{0.803000pt}%
\definecolor{currentstroke}{rgb}{0.752941,0.752941,0.752941}%
\pgfsetstrokecolor{currentstroke}%
\pgfsetdash{{2.960000pt}{1.280000pt}}{0.000000pt}%
\pgfpathmoveto{\pgfqpoint{0.536250in}{0.809097in}}%
\pgfpathlineto{\pgfqpoint{1.846250in}{0.809097in}}%
\pgfusepath{stroke}%
\end{pgfscope}%
\begin{pgfscope}%
\pgfsetbuttcap%
\pgfsetroundjoin%
\definecolor{currentfill}{rgb}{0.000000,0.000000,0.000000}%
\pgfsetfillcolor{currentfill}%
\pgfsetlinewidth{0.602250pt}%
\definecolor{currentstroke}{rgb}{0.000000,0.000000,0.000000}%
\pgfsetstrokecolor{currentstroke}%
\pgfsetdash{}{0pt}%
\pgfsys@defobject{currentmarker}{\pgfqpoint{-0.027778in}{0.000000in}}{\pgfqpoint{-0.000000in}{0.000000in}}{%
\pgfpathmoveto{\pgfqpoint{-0.000000in}{0.000000in}}%
\pgfpathlineto{\pgfqpoint{-0.027778in}{0.000000in}}%
\pgfusepath{stroke,fill}%
}%
\begin{pgfscope}%
\pgfsys@transformshift{0.536250in}{0.809097in}%
\pgfsys@useobject{currentmarker}{}%
\end{pgfscope}%
\end{pgfscope}%
\begin{pgfscope}%
\pgfpathrectangle{\pgfqpoint{0.536250in}{0.525000in}}{\pgfqpoint{1.310000in}{1.887500in}}%
\pgfusepath{clip}%
\pgfsetbuttcap%
\pgfsetroundjoin%
\pgfsetlinewidth{0.803000pt}%
\definecolor{currentstroke}{rgb}{0.752941,0.752941,0.752941}%
\pgfsetstrokecolor{currentstroke}%
\pgfsetdash{{2.960000pt}{1.280000pt}}{0.000000pt}%
\pgfpathmoveto{\pgfqpoint{0.536250in}{0.825189in}}%
\pgfpathlineto{\pgfqpoint{1.846250in}{0.825189in}}%
\pgfusepath{stroke}%
\end{pgfscope}%
\begin{pgfscope}%
\pgfsetbuttcap%
\pgfsetroundjoin%
\definecolor{currentfill}{rgb}{0.000000,0.000000,0.000000}%
\pgfsetfillcolor{currentfill}%
\pgfsetlinewidth{0.602250pt}%
\definecolor{currentstroke}{rgb}{0.000000,0.000000,0.000000}%
\pgfsetstrokecolor{currentstroke}%
\pgfsetdash{}{0pt}%
\pgfsys@defobject{currentmarker}{\pgfqpoint{-0.027778in}{0.000000in}}{\pgfqpoint{-0.000000in}{0.000000in}}{%
\pgfpathmoveto{\pgfqpoint{-0.000000in}{0.000000in}}%
\pgfpathlineto{\pgfqpoint{-0.027778in}{0.000000in}}%
\pgfusepath{stroke,fill}%
}%
\begin{pgfscope}%
\pgfsys@transformshift{0.536250in}{0.825189in}%
\pgfsys@useobject{currentmarker}{}%
\end{pgfscope}%
\end{pgfscope}%
\begin{pgfscope}%
\pgfpathrectangle{\pgfqpoint{0.536250in}{0.525000in}}{\pgfqpoint{1.310000in}{1.887500in}}%
\pgfusepath{clip}%
\pgfsetbuttcap%
\pgfsetroundjoin%
\pgfsetlinewidth{0.803000pt}%
\definecolor{currentstroke}{rgb}{0.752941,0.752941,0.752941}%
\pgfsetstrokecolor{currentstroke}%
\pgfsetdash{{2.960000pt}{1.280000pt}}{0.000000pt}%
\pgfpathmoveto{\pgfqpoint{0.536250in}{0.934282in}}%
\pgfpathlineto{\pgfqpoint{1.846250in}{0.934282in}}%
\pgfusepath{stroke}%
\end{pgfscope}%
\begin{pgfscope}%
\pgfsetbuttcap%
\pgfsetroundjoin%
\definecolor{currentfill}{rgb}{0.000000,0.000000,0.000000}%
\pgfsetfillcolor{currentfill}%
\pgfsetlinewidth{0.602250pt}%
\definecolor{currentstroke}{rgb}{0.000000,0.000000,0.000000}%
\pgfsetstrokecolor{currentstroke}%
\pgfsetdash{}{0pt}%
\pgfsys@defobject{currentmarker}{\pgfqpoint{-0.027778in}{0.000000in}}{\pgfqpoint{-0.000000in}{0.000000in}}{%
\pgfpathmoveto{\pgfqpoint{-0.000000in}{0.000000in}}%
\pgfpathlineto{\pgfqpoint{-0.027778in}{0.000000in}}%
\pgfusepath{stroke,fill}%
}%
\begin{pgfscope}%
\pgfsys@transformshift{0.536250in}{0.934282in}%
\pgfsys@useobject{currentmarker}{}%
\end{pgfscope}%
\end{pgfscope}%
\begin{pgfscope}%
\pgfpathrectangle{\pgfqpoint{0.536250in}{0.525000in}}{\pgfqpoint{1.310000in}{1.887500in}}%
\pgfusepath{clip}%
\pgfsetbuttcap%
\pgfsetroundjoin%
\pgfsetlinewidth{0.803000pt}%
\definecolor{currentstroke}{rgb}{0.752941,0.752941,0.752941}%
\pgfsetstrokecolor{currentstroke}%
\pgfsetdash{{2.960000pt}{1.280000pt}}{0.000000pt}%
\pgfpathmoveto{\pgfqpoint{0.536250in}{0.989678in}}%
\pgfpathlineto{\pgfqpoint{1.846250in}{0.989678in}}%
\pgfusepath{stroke}%
\end{pgfscope}%
\begin{pgfscope}%
\pgfsetbuttcap%
\pgfsetroundjoin%
\definecolor{currentfill}{rgb}{0.000000,0.000000,0.000000}%
\pgfsetfillcolor{currentfill}%
\pgfsetlinewidth{0.602250pt}%
\definecolor{currentstroke}{rgb}{0.000000,0.000000,0.000000}%
\pgfsetstrokecolor{currentstroke}%
\pgfsetdash{}{0pt}%
\pgfsys@defobject{currentmarker}{\pgfqpoint{-0.027778in}{0.000000in}}{\pgfqpoint{-0.000000in}{0.000000in}}{%
\pgfpathmoveto{\pgfqpoint{-0.000000in}{0.000000in}}%
\pgfpathlineto{\pgfqpoint{-0.027778in}{0.000000in}}%
\pgfusepath{stroke,fill}%
}%
\begin{pgfscope}%
\pgfsys@transformshift{0.536250in}{0.989678in}%
\pgfsys@useobject{currentmarker}{}%
\end{pgfscope}%
\end{pgfscope}%
\begin{pgfscope}%
\pgfpathrectangle{\pgfqpoint{0.536250in}{0.525000in}}{\pgfqpoint{1.310000in}{1.887500in}}%
\pgfusepath{clip}%
\pgfsetbuttcap%
\pgfsetroundjoin%
\pgfsetlinewidth{0.803000pt}%
\definecolor{currentstroke}{rgb}{0.752941,0.752941,0.752941}%
\pgfsetstrokecolor{currentstroke}%
\pgfsetdash{{2.960000pt}{1.280000pt}}{0.000000pt}%
\pgfpathmoveto{\pgfqpoint{0.536250in}{1.028981in}}%
\pgfpathlineto{\pgfqpoint{1.846250in}{1.028981in}}%
\pgfusepath{stroke}%
\end{pgfscope}%
\begin{pgfscope}%
\pgfsetbuttcap%
\pgfsetroundjoin%
\definecolor{currentfill}{rgb}{0.000000,0.000000,0.000000}%
\pgfsetfillcolor{currentfill}%
\pgfsetlinewidth{0.602250pt}%
\definecolor{currentstroke}{rgb}{0.000000,0.000000,0.000000}%
\pgfsetstrokecolor{currentstroke}%
\pgfsetdash{}{0pt}%
\pgfsys@defobject{currentmarker}{\pgfqpoint{-0.027778in}{0.000000in}}{\pgfqpoint{-0.000000in}{0.000000in}}{%
\pgfpathmoveto{\pgfqpoint{-0.000000in}{0.000000in}}%
\pgfpathlineto{\pgfqpoint{-0.027778in}{0.000000in}}%
\pgfusepath{stroke,fill}%
}%
\begin{pgfscope}%
\pgfsys@transformshift{0.536250in}{1.028981in}%
\pgfsys@useobject{currentmarker}{}%
\end{pgfscope}%
\end{pgfscope}%
\begin{pgfscope}%
\pgfpathrectangle{\pgfqpoint{0.536250in}{0.525000in}}{\pgfqpoint{1.310000in}{1.887500in}}%
\pgfusepath{clip}%
\pgfsetbuttcap%
\pgfsetroundjoin%
\pgfsetlinewidth{0.803000pt}%
\definecolor{currentstroke}{rgb}{0.752941,0.752941,0.752941}%
\pgfsetstrokecolor{currentstroke}%
\pgfsetdash{{2.960000pt}{1.280000pt}}{0.000000pt}%
\pgfpathmoveto{\pgfqpoint{0.536250in}{1.059468in}}%
\pgfpathlineto{\pgfqpoint{1.846250in}{1.059468in}}%
\pgfusepath{stroke}%
\end{pgfscope}%
\begin{pgfscope}%
\pgfsetbuttcap%
\pgfsetroundjoin%
\definecolor{currentfill}{rgb}{0.000000,0.000000,0.000000}%
\pgfsetfillcolor{currentfill}%
\pgfsetlinewidth{0.602250pt}%
\definecolor{currentstroke}{rgb}{0.000000,0.000000,0.000000}%
\pgfsetstrokecolor{currentstroke}%
\pgfsetdash{}{0pt}%
\pgfsys@defobject{currentmarker}{\pgfqpoint{-0.027778in}{0.000000in}}{\pgfqpoint{-0.000000in}{0.000000in}}{%
\pgfpathmoveto{\pgfqpoint{-0.000000in}{0.000000in}}%
\pgfpathlineto{\pgfqpoint{-0.027778in}{0.000000in}}%
\pgfusepath{stroke,fill}%
}%
\begin{pgfscope}%
\pgfsys@transformshift{0.536250in}{1.059468in}%
\pgfsys@useobject{currentmarker}{}%
\end{pgfscope}%
\end{pgfscope}%
\begin{pgfscope}%
\pgfpathrectangle{\pgfqpoint{0.536250in}{0.525000in}}{\pgfqpoint{1.310000in}{1.887500in}}%
\pgfusepath{clip}%
\pgfsetbuttcap%
\pgfsetroundjoin%
\pgfsetlinewidth{0.803000pt}%
\definecolor{currentstroke}{rgb}{0.752941,0.752941,0.752941}%
\pgfsetstrokecolor{currentstroke}%
\pgfsetdash{{2.960000pt}{1.280000pt}}{0.000000pt}%
\pgfpathmoveto{\pgfqpoint{0.536250in}{1.084377in}}%
\pgfpathlineto{\pgfqpoint{1.846250in}{1.084377in}}%
\pgfusepath{stroke}%
\end{pgfscope}%
\begin{pgfscope}%
\pgfsetbuttcap%
\pgfsetroundjoin%
\definecolor{currentfill}{rgb}{0.000000,0.000000,0.000000}%
\pgfsetfillcolor{currentfill}%
\pgfsetlinewidth{0.602250pt}%
\definecolor{currentstroke}{rgb}{0.000000,0.000000,0.000000}%
\pgfsetstrokecolor{currentstroke}%
\pgfsetdash{}{0pt}%
\pgfsys@defobject{currentmarker}{\pgfqpoint{-0.027778in}{0.000000in}}{\pgfqpoint{-0.000000in}{0.000000in}}{%
\pgfpathmoveto{\pgfqpoint{-0.000000in}{0.000000in}}%
\pgfpathlineto{\pgfqpoint{-0.027778in}{0.000000in}}%
\pgfusepath{stroke,fill}%
}%
\begin{pgfscope}%
\pgfsys@transformshift{0.536250in}{1.084377in}%
\pgfsys@useobject{currentmarker}{}%
\end{pgfscope}%
\end{pgfscope}%
\begin{pgfscope}%
\pgfpathrectangle{\pgfqpoint{0.536250in}{0.525000in}}{\pgfqpoint{1.310000in}{1.887500in}}%
\pgfusepath{clip}%
\pgfsetbuttcap%
\pgfsetroundjoin%
\pgfsetlinewidth{0.803000pt}%
\definecolor{currentstroke}{rgb}{0.752941,0.752941,0.752941}%
\pgfsetstrokecolor{currentstroke}%
\pgfsetdash{{2.960000pt}{1.280000pt}}{0.000000pt}%
\pgfpathmoveto{\pgfqpoint{0.536250in}{1.105437in}}%
\pgfpathlineto{\pgfqpoint{1.846250in}{1.105437in}}%
\pgfusepath{stroke}%
\end{pgfscope}%
\begin{pgfscope}%
\pgfsetbuttcap%
\pgfsetroundjoin%
\definecolor{currentfill}{rgb}{0.000000,0.000000,0.000000}%
\pgfsetfillcolor{currentfill}%
\pgfsetlinewidth{0.602250pt}%
\definecolor{currentstroke}{rgb}{0.000000,0.000000,0.000000}%
\pgfsetstrokecolor{currentstroke}%
\pgfsetdash{}{0pt}%
\pgfsys@defobject{currentmarker}{\pgfqpoint{-0.027778in}{0.000000in}}{\pgfqpoint{-0.000000in}{0.000000in}}{%
\pgfpathmoveto{\pgfqpoint{-0.000000in}{0.000000in}}%
\pgfpathlineto{\pgfqpoint{-0.027778in}{0.000000in}}%
\pgfusepath{stroke,fill}%
}%
\begin{pgfscope}%
\pgfsys@transformshift{0.536250in}{1.105437in}%
\pgfsys@useobject{currentmarker}{}%
\end{pgfscope}%
\end{pgfscope}%
\begin{pgfscope}%
\pgfpathrectangle{\pgfqpoint{0.536250in}{0.525000in}}{\pgfqpoint{1.310000in}{1.887500in}}%
\pgfusepath{clip}%
\pgfsetbuttcap%
\pgfsetroundjoin%
\pgfsetlinewidth{0.803000pt}%
\definecolor{currentstroke}{rgb}{0.752941,0.752941,0.752941}%
\pgfsetstrokecolor{currentstroke}%
\pgfsetdash{{2.960000pt}{1.280000pt}}{0.000000pt}%
\pgfpathmoveto{\pgfqpoint{0.536250in}{1.123680in}}%
\pgfpathlineto{\pgfqpoint{1.846250in}{1.123680in}}%
\pgfusepath{stroke}%
\end{pgfscope}%
\begin{pgfscope}%
\pgfsetbuttcap%
\pgfsetroundjoin%
\definecolor{currentfill}{rgb}{0.000000,0.000000,0.000000}%
\pgfsetfillcolor{currentfill}%
\pgfsetlinewidth{0.602250pt}%
\definecolor{currentstroke}{rgb}{0.000000,0.000000,0.000000}%
\pgfsetstrokecolor{currentstroke}%
\pgfsetdash{}{0pt}%
\pgfsys@defobject{currentmarker}{\pgfqpoint{-0.027778in}{0.000000in}}{\pgfqpoint{-0.000000in}{0.000000in}}{%
\pgfpathmoveto{\pgfqpoint{-0.000000in}{0.000000in}}%
\pgfpathlineto{\pgfqpoint{-0.027778in}{0.000000in}}%
\pgfusepath{stroke,fill}%
}%
\begin{pgfscope}%
\pgfsys@transformshift{0.536250in}{1.123680in}%
\pgfsys@useobject{currentmarker}{}%
\end{pgfscope}%
\end{pgfscope}%
\begin{pgfscope}%
\pgfpathrectangle{\pgfqpoint{0.536250in}{0.525000in}}{\pgfqpoint{1.310000in}{1.887500in}}%
\pgfusepath{clip}%
\pgfsetbuttcap%
\pgfsetroundjoin%
\pgfsetlinewidth{0.803000pt}%
\definecolor{currentstroke}{rgb}{0.752941,0.752941,0.752941}%
\pgfsetstrokecolor{currentstroke}%
\pgfsetdash{{2.960000pt}{1.280000pt}}{0.000000pt}%
\pgfpathmoveto{\pgfqpoint{0.536250in}{1.139772in}}%
\pgfpathlineto{\pgfqpoint{1.846250in}{1.139772in}}%
\pgfusepath{stroke}%
\end{pgfscope}%
\begin{pgfscope}%
\pgfsetbuttcap%
\pgfsetroundjoin%
\definecolor{currentfill}{rgb}{0.000000,0.000000,0.000000}%
\pgfsetfillcolor{currentfill}%
\pgfsetlinewidth{0.602250pt}%
\definecolor{currentstroke}{rgb}{0.000000,0.000000,0.000000}%
\pgfsetstrokecolor{currentstroke}%
\pgfsetdash{}{0pt}%
\pgfsys@defobject{currentmarker}{\pgfqpoint{-0.027778in}{0.000000in}}{\pgfqpoint{-0.000000in}{0.000000in}}{%
\pgfpathmoveto{\pgfqpoint{-0.000000in}{0.000000in}}%
\pgfpathlineto{\pgfqpoint{-0.027778in}{0.000000in}}%
\pgfusepath{stroke,fill}%
}%
\begin{pgfscope}%
\pgfsys@transformshift{0.536250in}{1.139772in}%
\pgfsys@useobject{currentmarker}{}%
\end{pgfscope}%
\end{pgfscope}%
\begin{pgfscope}%
\pgfpathrectangle{\pgfqpoint{0.536250in}{0.525000in}}{\pgfqpoint{1.310000in}{1.887500in}}%
\pgfusepath{clip}%
\pgfsetbuttcap%
\pgfsetroundjoin%
\pgfsetlinewidth{0.803000pt}%
\definecolor{currentstroke}{rgb}{0.752941,0.752941,0.752941}%
\pgfsetstrokecolor{currentstroke}%
\pgfsetdash{{2.960000pt}{1.280000pt}}{0.000000pt}%
\pgfpathmoveto{\pgfqpoint{0.536250in}{1.248866in}}%
\pgfpathlineto{\pgfqpoint{1.846250in}{1.248866in}}%
\pgfusepath{stroke}%
\end{pgfscope}%
\begin{pgfscope}%
\pgfsetbuttcap%
\pgfsetroundjoin%
\definecolor{currentfill}{rgb}{0.000000,0.000000,0.000000}%
\pgfsetfillcolor{currentfill}%
\pgfsetlinewidth{0.602250pt}%
\definecolor{currentstroke}{rgb}{0.000000,0.000000,0.000000}%
\pgfsetstrokecolor{currentstroke}%
\pgfsetdash{}{0pt}%
\pgfsys@defobject{currentmarker}{\pgfqpoint{-0.027778in}{0.000000in}}{\pgfqpoint{-0.000000in}{0.000000in}}{%
\pgfpathmoveto{\pgfqpoint{-0.000000in}{0.000000in}}%
\pgfpathlineto{\pgfqpoint{-0.027778in}{0.000000in}}%
\pgfusepath{stroke,fill}%
}%
\begin{pgfscope}%
\pgfsys@transformshift{0.536250in}{1.248866in}%
\pgfsys@useobject{currentmarker}{}%
\end{pgfscope}%
\end{pgfscope}%
\begin{pgfscope}%
\pgfpathrectangle{\pgfqpoint{0.536250in}{0.525000in}}{\pgfqpoint{1.310000in}{1.887500in}}%
\pgfusepath{clip}%
\pgfsetbuttcap%
\pgfsetroundjoin%
\pgfsetlinewidth{0.803000pt}%
\definecolor{currentstroke}{rgb}{0.752941,0.752941,0.752941}%
\pgfsetstrokecolor{currentstroke}%
\pgfsetdash{{2.960000pt}{1.280000pt}}{0.000000pt}%
\pgfpathmoveto{\pgfqpoint{0.536250in}{1.304261in}}%
\pgfpathlineto{\pgfqpoint{1.846250in}{1.304261in}}%
\pgfusepath{stroke}%
\end{pgfscope}%
\begin{pgfscope}%
\pgfsetbuttcap%
\pgfsetroundjoin%
\definecolor{currentfill}{rgb}{0.000000,0.000000,0.000000}%
\pgfsetfillcolor{currentfill}%
\pgfsetlinewidth{0.602250pt}%
\definecolor{currentstroke}{rgb}{0.000000,0.000000,0.000000}%
\pgfsetstrokecolor{currentstroke}%
\pgfsetdash{}{0pt}%
\pgfsys@defobject{currentmarker}{\pgfqpoint{-0.027778in}{0.000000in}}{\pgfqpoint{-0.000000in}{0.000000in}}{%
\pgfpathmoveto{\pgfqpoint{-0.000000in}{0.000000in}}%
\pgfpathlineto{\pgfqpoint{-0.027778in}{0.000000in}}%
\pgfusepath{stroke,fill}%
}%
\begin{pgfscope}%
\pgfsys@transformshift{0.536250in}{1.304261in}%
\pgfsys@useobject{currentmarker}{}%
\end{pgfscope}%
\end{pgfscope}%
\begin{pgfscope}%
\pgfpathrectangle{\pgfqpoint{0.536250in}{0.525000in}}{\pgfqpoint{1.310000in}{1.887500in}}%
\pgfusepath{clip}%
\pgfsetbuttcap%
\pgfsetroundjoin%
\pgfsetlinewidth{0.803000pt}%
\definecolor{currentstroke}{rgb}{0.752941,0.752941,0.752941}%
\pgfsetstrokecolor{currentstroke}%
\pgfsetdash{{2.960000pt}{1.280000pt}}{0.000000pt}%
\pgfpathmoveto{\pgfqpoint{0.536250in}{1.343565in}}%
\pgfpathlineto{\pgfqpoint{1.846250in}{1.343565in}}%
\pgfusepath{stroke}%
\end{pgfscope}%
\begin{pgfscope}%
\pgfsetbuttcap%
\pgfsetroundjoin%
\definecolor{currentfill}{rgb}{0.000000,0.000000,0.000000}%
\pgfsetfillcolor{currentfill}%
\pgfsetlinewidth{0.602250pt}%
\definecolor{currentstroke}{rgb}{0.000000,0.000000,0.000000}%
\pgfsetstrokecolor{currentstroke}%
\pgfsetdash{}{0pt}%
\pgfsys@defobject{currentmarker}{\pgfqpoint{-0.027778in}{0.000000in}}{\pgfqpoint{-0.000000in}{0.000000in}}{%
\pgfpathmoveto{\pgfqpoint{-0.000000in}{0.000000in}}%
\pgfpathlineto{\pgfqpoint{-0.027778in}{0.000000in}}%
\pgfusepath{stroke,fill}%
}%
\begin{pgfscope}%
\pgfsys@transformshift{0.536250in}{1.343565in}%
\pgfsys@useobject{currentmarker}{}%
\end{pgfscope}%
\end{pgfscope}%
\begin{pgfscope}%
\pgfpathrectangle{\pgfqpoint{0.536250in}{0.525000in}}{\pgfqpoint{1.310000in}{1.887500in}}%
\pgfusepath{clip}%
\pgfsetbuttcap%
\pgfsetroundjoin%
\pgfsetlinewidth{0.803000pt}%
\definecolor{currentstroke}{rgb}{0.752941,0.752941,0.752941}%
\pgfsetstrokecolor{currentstroke}%
\pgfsetdash{{2.960000pt}{1.280000pt}}{0.000000pt}%
\pgfpathmoveto{\pgfqpoint{0.536250in}{1.374051in}}%
\pgfpathlineto{\pgfqpoint{1.846250in}{1.374051in}}%
\pgfusepath{stroke}%
\end{pgfscope}%
\begin{pgfscope}%
\pgfsetbuttcap%
\pgfsetroundjoin%
\definecolor{currentfill}{rgb}{0.000000,0.000000,0.000000}%
\pgfsetfillcolor{currentfill}%
\pgfsetlinewidth{0.602250pt}%
\definecolor{currentstroke}{rgb}{0.000000,0.000000,0.000000}%
\pgfsetstrokecolor{currentstroke}%
\pgfsetdash{}{0pt}%
\pgfsys@defobject{currentmarker}{\pgfqpoint{-0.027778in}{0.000000in}}{\pgfqpoint{-0.000000in}{0.000000in}}{%
\pgfpathmoveto{\pgfqpoint{-0.000000in}{0.000000in}}%
\pgfpathlineto{\pgfqpoint{-0.027778in}{0.000000in}}%
\pgfusepath{stroke,fill}%
}%
\begin{pgfscope}%
\pgfsys@transformshift{0.536250in}{1.374051in}%
\pgfsys@useobject{currentmarker}{}%
\end{pgfscope}%
\end{pgfscope}%
\begin{pgfscope}%
\pgfpathrectangle{\pgfqpoint{0.536250in}{0.525000in}}{\pgfqpoint{1.310000in}{1.887500in}}%
\pgfusepath{clip}%
\pgfsetbuttcap%
\pgfsetroundjoin%
\pgfsetlinewidth{0.803000pt}%
\definecolor{currentstroke}{rgb}{0.752941,0.752941,0.752941}%
\pgfsetstrokecolor{currentstroke}%
\pgfsetdash{{2.960000pt}{1.280000pt}}{0.000000pt}%
\pgfpathmoveto{\pgfqpoint{0.536250in}{1.398960in}}%
\pgfpathlineto{\pgfqpoint{1.846250in}{1.398960in}}%
\pgfusepath{stroke}%
\end{pgfscope}%
\begin{pgfscope}%
\pgfsetbuttcap%
\pgfsetroundjoin%
\definecolor{currentfill}{rgb}{0.000000,0.000000,0.000000}%
\pgfsetfillcolor{currentfill}%
\pgfsetlinewidth{0.602250pt}%
\definecolor{currentstroke}{rgb}{0.000000,0.000000,0.000000}%
\pgfsetstrokecolor{currentstroke}%
\pgfsetdash{}{0pt}%
\pgfsys@defobject{currentmarker}{\pgfqpoint{-0.027778in}{0.000000in}}{\pgfqpoint{-0.000000in}{0.000000in}}{%
\pgfpathmoveto{\pgfqpoint{-0.000000in}{0.000000in}}%
\pgfpathlineto{\pgfqpoint{-0.027778in}{0.000000in}}%
\pgfusepath{stroke,fill}%
}%
\begin{pgfscope}%
\pgfsys@transformshift{0.536250in}{1.398960in}%
\pgfsys@useobject{currentmarker}{}%
\end{pgfscope}%
\end{pgfscope}%
\begin{pgfscope}%
\pgfpathrectangle{\pgfqpoint{0.536250in}{0.525000in}}{\pgfqpoint{1.310000in}{1.887500in}}%
\pgfusepath{clip}%
\pgfsetbuttcap%
\pgfsetroundjoin%
\pgfsetlinewidth{0.803000pt}%
\definecolor{currentstroke}{rgb}{0.752941,0.752941,0.752941}%
\pgfsetstrokecolor{currentstroke}%
\pgfsetdash{{2.960000pt}{1.280000pt}}{0.000000pt}%
\pgfpathmoveto{\pgfqpoint{0.536250in}{1.420020in}}%
\pgfpathlineto{\pgfqpoint{1.846250in}{1.420020in}}%
\pgfusepath{stroke}%
\end{pgfscope}%
\begin{pgfscope}%
\pgfsetbuttcap%
\pgfsetroundjoin%
\definecolor{currentfill}{rgb}{0.000000,0.000000,0.000000}%
\pgfsetfillcolor{currentfill}%
\pgfsetlinewidth{0.602250pt}%
\definecolor{currentstroke}{rgb}{0.000000,0.000000,0.000000}%
\pgfsetstrokecolor{currentstroke}%
\pgfsetdash{}{0pt}%
\pgfsys@defobject{currentmarker}{\pgfqpoint{-0.027778in}{0.000000in}}{\pgfqpoint{-0.000000in}{0.000000in}}{%
\pgfpathmoveto{\pgfqpoint{-0.000000in}{0.000000in}}%
\pgfpathlineto{\pgfqpoint{-0.027778in}{0.000000in}}%
\pgfusepath{stroke,fill}%
}%
\begin{pgfscope}%
\pgfsys@transformshift{0.536250in}{1.420020in}%
\pgfsys@useobject{currentmarker}{}%
\end{pgfscope}%
\end{pgfscope}%
\begin{pgfscope}%
\pgfpathrectangle{\pgfqpoint{0.536250in}{0.525000in}}{\pgfqpoint{1.310000in}{1.887500in}}%
\pgfusepath{clip}%
\pgfsetbuttcap%
\pgfsetroundjoin%
\pgfsetlinewidth{0.803000pt}%
\definecolor{currentstroke}{rgb}{0.752941,0.752941,0.752941}%
\pgfsetstrokecolor{currentstroke}%
\pgfsetdash{{2.960000pt}{1.280000pt}}{0.000000pt}%
\pgfpathmoveto{\pgfqpoint{0.536250in}{1.438264in}}%
\pgfpathlineto{\pgfqpoint{1.846250in}{1.438264in}}%
\pgfusepath{stroke}%
\end{pgfscope}%
\begin{pgfscope}%
\pgfsetbuttcap%
\pgfsetroundjoin%
\definecolor{currentfill}{rgb}{0.000000,0.000000,0.000000}%
\pgfsetfillcolor{currentfill}%
\pgfsetlinewidth{0.602250pt}%
\definecolor{currentstroke}{rgb}{0.000000,0.000000,0.000000}%
\pgfsetstrokecolor{currentstroke}%
\pgfsetdash{}{0pt}%
\pgfsys@defobject{currentmarker}{\pgfqpoint{-0.027778in}{0.000000in}}{\pgfqpoint{-0.000000in}{0.000000in}}{%
\pgfpathmoveto{\pgfqpoint{-0.000000in}{0.000000in}}%
\pgfpathlineto{\pgfqpoint{-0.027778in}{0.000000in}}%
\pgfusepath{stroke,fill}%
}%
\begin{pgfscope}%
\pgfsys@transformshift{0.536250in}{1.438264in}%
\pgfsys@useobject{currentmarker}{}%
\end{pgfscope}%
\end{pgfscope}%
\begin{pgfscope}%
\pgfpathrectangle{\pgfqpoint{0.536250in}{0.525000in}}{\pgfqpoint{1.310000in}{1.887500in}}%
\pgfusepath{clip}%
\pgfsetbuttcap%
\pgfsetroundjoin%
\pgfsetlinewidth{0.803000pt}%
\definecolor{currentstroke}{rgb}{0.752941,0.752941,0.752941}%
\pgfsetstrokecolor{currentstroke}%
\pgfsetdash{{2.960000pt}{1.280000pt}}{0.000000pt}%
\pgfpathmoveto{\pgfqpoint{0.536250in}{1.454355in}}%
\pgfpathlineto{\pgfqpoint{1.846250in}{1.454355in}}%
\pgfusepath{stroke}%
\end{pgfscope}%
\begin{pgfscope}%
\pgfsetbuttcap%
\pgfsetroundjoin%
\definecolor{currentfill}{rgb}{0.000000,0.000000,0.000000}%
\pgfsetfillcolor{currentfill}%
\pgfsetlinewidth{0.602250pt}%
\definecolor{currentstroke}{rgb}{0.000000,0.000000,0.000000}%
\pgfsetstrokecolor{currentstroke}%
\pgfsetdash{}{0pt}%
\pgfsys@defobject{currentmarker}{\pgfqpoint{-0.027778in}{0.000000in}}{\pgfqpoint{-0.000000in}{0.000000in}}{%
\pgfpathmoveto{\pgfqpoint{-0.000000in}{0.000000in}}%
\pgfpathlineto{\pgfqpoint{-0.027778in}{0.000000in}}%
\pgfusepath{stroke,fill}%
}%
\begin{pgfscope}%
\pgfsys@transformshift{0.536250in}{1.454355in}%
\pgfsys@useobject{currentmarker}{}%
\end{pgfscope}%
\end{pgfscope}%
\begin{pgfscope}%
\pgfpathrectangle{\pgfqpoint{0.536250in}{0.525000in}}{\pgfqpoint{1.310000in}{1.887500in}}%
\pgfusepath{clip}%
\pgfsetbuttcap%
\pgfsetroundjoin%
\pgfsetlinewidth{0.803000pt}%
\definecolor{currentstroke}{rgb}{0.752941,0.752941,0.752941}%
\pgfsetstrokecolor{currentstroke}%
\pgfsetdash{{2.960000pt}{1.280000pt}}{0.000000pt}%
\pgfpathmoveto{\pgfqpoint{0.536250in}{1.563449in}}%
\pgfpathlineto{\pgfqpoint{1.846250in}{1.563449in}}%
\pgfusepath{stroke}%
\end{pgfscope}%
\begin{pgfscope}%
\pgfsetbuttcap%
\pgfsetroundjoin%
\definecolor{currentfill}{rgb}{0.000000,0.000000,0.000000}%
\pgfsetfillcolor{currentfill}%
\pgfsetlinewidth{0.602250pt}%
\definecolor{currentstroke}{rgb}{0.000000,0.000000,0.000000}%
\pgfsetstrokecolor{currentstroke}%
\pgfsetdash{}{0pt}%
\pgfsys@defobject{currentmarker}{\pgfqpoint{-0.027778in}{0.000000in}}{\pgfqpoint{-0.000000in}{0.000000in}}{%
\pgfpathmoveto{\pgfqpoint{-0.000000in}{0.000000in}}%
\pgfpathlineto{\pgfqpoint{-0.027778in}{0.000000in}}%
\pgfusepath{stroke,fill}%
}%
\begin{pgfscope}%
\pgfsys@transformshift{0.536250in}{1.563449in}%
\pgfsys@useobject{currentmarker}{}%
\end{pgfscope}%
\end{pgfscope}%
\begin{pgfscope}%
\pgfpathrectangle{\pgfqpoint{0.536250in}{0.525000in}}{\pgfqpoint{1.310000in}{1.887500in}}%
\pgfusepath{clip}%
\pgfsetbuttcap%
\pgfsetroundjoin%
\pgfsetlinewidth{0.803000pt}%
\definecolor{currentstroke}{rgb}{0.752941,0.752941,0.752941}%
\pgfsetstrokecolor{currentstroke}%
\pgfsetdash{{2.960000pt}{1.280000pt}}{0.000000pt}%
\pgfpathmoveto{\pgfqpoint{0.536250in}{1.618844in}}%
\pgfpathlineto{\pgfqpoint{1.846250in}{1.618844in}}%
\pgfusepath{stroke}%
\end{pgfscope}%
\begin{pgfscope}%
\pgfsetbuttcap%
\pgfsetroundjoin%
\definecolor{currentfill}{rgb}{0.000000,0.000000,0.000000}%
\pgfsetfillcolor{currentfill}%
\pgfsetlinewidth{0.602250pt}%
\definecolor{currentstroke}{rgb}{0.000000,0.000000,0.000000}%
\pgfsetstrokecolor{currentstroke}%
\pgfsetdash{}{0pt}%
\pgfsys@defobject{currentmarker}{\pgfqpoint{-0.027778in}{0.000000in}}{\pgfqpoint{-0.000000in}{0.000000in}}{%
\pgfpathmoveto{\pgfqpoint{-0.000000in}{0.000000in}}%
\pgfpathlineto{\pgfqpoint{-0.027778in}{0.000000in}}%
\pgfusepath{stroke,fill}%
}%
\begin{pgfscope}%
\pgfsys@transformshift{0.536250in}{1.618844in}%
\pgfsys@useobject{currentmarker}{}%
\end{pgfscope}%
\end{pgfscope}%
\begin{pgfscope}%
\pgfpathrectangle{\pgfqpoint{0.536250in}{0.525000in}}{\pgfqpoint{1.310000in}{1.887500in}}%
\pgfusepath{clip}%
\pgfsetbuttcap%
\pgfsetroundjoin%
\pgfsetlinewidth{0.803000pt}%
\definecolor{currentstroke}{rgb}{0.752941,0.752941,0.752941}%
\pgfsetstrokecolor{currentstroke}%
\pgfsetdash{{2.960000pt}{1.280000pt}}{0.000000pt}%
\pgfpathmoveto{\pgfqpoint{0.536250in}{1.658148in}}%
\pgfpathlineto{\pgfqpoint{1.846250in}{1.658148in}}%
\pgfusepath{stroke}%
\end{pgfscope}%
\begin{pgfscope}%
\pgfsetbuttcap%
\pgfsetroundjoin%
\definecolor{currentfill}{rgb}{0.000000,0.000000,0.000000}%
\pgfsetfillcolor{currentfill}%
\pgfsetlinewidth{0.602250pt}%
\definecolor{currentstroke}{rgb}{0.000000,0.000000,0.000000}%
\pgfsetstrokecolor{currentstroke}%
\pgfsetdash{}{0pt}%
\pgfsys@defobject{currentmarker}{\pgfqpoint{-0.027778in}{0.000000in}}{\pgfqpoint{-0.000000in}{0.000000in}}{%
\pgfpathmoveto{\pgfqpoint{-0.000000in}{0.000000in}}%
\pgfpathlineto{\pgfqpoint{-0.027778in}{0.000000in}}%
\pgfusepath{stroke,fill}%
}%
\begin{pgfscope}%
\pgfsys@transformshift{0.536250in}{1.658148in}%
\pgfsys@useobject{currentmarker}{}%
\end{pgfscope}%
\end{pgfscope}%
\begin{pgfscope}%
\pgfpathrectangle{\pgfqpoint{0.536250in}{0.525000in}}{\pgfqpoint{1.310000in}{1.887500in}}%
\pgfusepath{clip}%
\pgfsetbuttcap%
\pgfsetroundjoin%
\pgfsetlinewidth{0.803000pt}%
\definecolor{currentstroke}{rgb}{0.752941,0.752941,0.752941}%
\pgfsetstrokecolor{currentstroke}%
\pgfsetdash{{2.960000pt}{1.280000pt}}{0.000000pt}%
\pgfpathmoveto{\pgfqpoint{0.536250in}{1.688634in}}%
\pgfpathlineto{\pgfqpoint{1.846250in}{1.688634in}}%
\pgfusepath{stroke}%
\end{pgfscope}%
\begin{pgfscope}%
\pgfsetbuttcap%
\pgfsetroundjoin%
\definecolor{currentfill}{rgb}{0.000000,0.000000,0.000000}%
\pgfsetfillcolor{currentfill}%
\pgfsetlinewidth{0.602250pt}%
\definecolor{currentstroke}{rgb}{0.000000,0.000000,0.000000}%
\pgfsetstrokecolor{currentstroke}%
\pgfsetdash{}{0pt}%
\pgfsys@defobject{currentmarker}{\pgfqpoint{-0.027778in}{0.000000in}}{\pgfqpoint{-0.000000in}{0.000000in}}{%
\pgfpathmoveto{\pgfqpoint{-0.000000in}{0.000000in}}%
\pgfpathlineto{\pgfqpoint{-0.027778in}{0.000000in}}%
\pgfusepath{stroke,fill}%
}%
\begin{pgfscope}%
\pgfsys@transformshift{0.536250in}{1.688634in}%
\pgfsys@useobject{currentmarker}{}%
\end{pgfscope}%
\end{pgfscope}%
\begin{pgfscope}%
\pgfpathrectangle{\pgfqpoint{0.536250in}{0.525000in}}{\pgfqpoint{1.310000in}{1.887500in}}%
\pgfusepath{clip}%
\pgfsetbuttcap%
\pgfsetroundjoin%
\pgfsetlinewidth{0.803000pt}%
\definecolor{currentstroke}{rgb}{0.752941,0.752941,0.752941}%
\pgfsetstrokecolor{currentstroke}%
\pgfsetdash{{2.960000pt}{1.280000pt}}{0.000000pt}%
\pgfpathmoveto{\pgfqpoint{0.536250in}{1.713543in}}%
\pgfpathlineto{\pgfqpoint{1.846250in}{1.713543in}}%
\pgfusepath{stroke}%
\end{pgfscope}%
\begin{pgfscope}%
\pgfsetbuttcap%
\pgfsetroundjoin%
\definecolor{currentfill}{rgb}{0.000000,0.000000,0.000000}%
\pgfsetfillcolor{currentfill}%
\pgfsetlinewidth{0.602250pt}%
\definecolor{currentstroke}{rgb}{0.000000,0.000000,0.000000}%
\pgfsetstrokecolor{currentstroke}%
\pgfsetdash{}{0pt}%
\pgfsys@defobject{currentmarker}{\pgfqpoint{-0.027778in}{0.000000in}}{\pgfqpoint{-0.000000in}{0.000000in}}{%
\pgfpathmoveto{\pgfqpoint{-0.000000in}{0.000000in}}%
\pgfpathlineto{\pgfqpoint{-0.027778in}{0.000000in}}%
\pgfusepath{stroke,fill}%
}%
\begin{pgfscope}%
\pgfsys@transformshift{0.536250in}{1.713543in}%
\pgfsys@useobject{currentmarker}{}%
\end{pgfscope}%
\end{pgfscope}%
\begin{pgfscope}%
\pgfpathrectangle{\pgfqpoint{0.536250in}{0.525000in}}{\pgfqpoint{1.310000in}{1.887500in}}%
\pgfusepath{clip}%
\pgfsetbuttcap%
\pgfsetroundjoin%
\pgfsetlinewidth{0.803000pt}%
\definecolor{currentstroke}{rgb}{0.752941,0.752941,0.752941}%
\pgfsetstrokecolor{currentstroke}%
\pgfsetdash{{2.960000pt}{1.280000pt}}{0.000000pt}%
\pgfpathmoveto{\pgfqpoint{0.536250in}{1.734604in}}%
\pgfpathlineto{\pgfqpoint{1.846250in}{1.734604in}}%
\pgfusepath{stroke}%
\end{pgfscope}%
\begin{pgfscope}%
\pgfsetbuttcap%
\pgfsetroundjoin%
\definecolor{currentfill}{rgb}{0.000000,0.000000,0.000000}%
\pgfsetfillcolor{currentfill}%
\pgfsetlinewidth{0.602250pt}%
\definecolor{currentstroke}{rgb}{0.000000,0.000000,0.000000}%
\pgfsetstrokecolor{currentstroke}%
\pgfsetdash{}{0pt}%
\pgfsys@defobject{currentmarker}{\pgfqpoint{-0.027778in}{0.000000in}}{\pgfqpoint{-0.000000in}{0.000000in}}{%
\pgfpathmoveto{\pgfqpoint{-0.000000in}{0.000000in}}%
\pgfpathlineto{\pgfqpoint{-0.027778in}{0.000000in}}%
\pgfusepath{stroke,fill}%
}%
\begin{pgfscope}%
\pgfsys@transformshift{0.536250in}{1.734604in}%
\pgfsys@useobject{currentmarker}{}%
\end{pgfscope}%
\end{pgfscope}%
\begin{pgfscope}%
\pgfpathrectangle{\pgfqpoint{0.536250in}{0.525000in}}{\pgfqpoint{1.310000in}{1.887500in}}%
\pgfusepath{clip}%
\pgfsetbuttcap%
\pgfsetroundjoin%
\pgfsetlinewidth{0.803000pt}%
\definecolor{currentstroke}{rgb}{0.752941,0.752941,0.752941}%
\pgfsetstrokecolor{currentstroke}%
\pgfsetdash{{2.960000pt}{1.280000pt}}{0.000000pt}%
\pgfpathmoveto{\pgfqpoint{0.536250in}{1.752847in}}%
\pgfpathlineto{\pgfqpoint{1.846250in}{1.752847in}}%
\pgfusepath{stroke}%
\end{pgfscope}%
\begin{pgfscope}%
\pgfsetbuttcap%
\pgfsetroundjoin%
\definecolor{currentfill}{rgb}{0.000000,0.000000,0.000000}%
\pgfsetfillcolor{currentfill}%
\pgfsetlinewidth{0.602250pt}%
\definecolor{currentstroke}{rgb}{0.000000,0.000000,0.000000}%
\pgfsetstrokecolor{currentstroke}%
\pgfsetdash{}{0pt}%
\pgfsys@defobject{currentmarker}{\pgfqpoint{-0.027778in}{0.000000in}}{\pgfqpoint{-0.000000in}{0.000000in}}{%
\pgfpathmoveto{\pgfqpoint{-0.000000in}{0.000000in}}%
\pgfpathlineto{\pgfqpoint{-0.027778in}{0.000000in}}%
\pgfusepath{stroke,fill}%
}%
\begin{pgfscope}%
\pgfsys@transformshift{0.536250in}{1.752847in}%
\pgfsys@useobject{currentmarker}{}%
\end{pgfscope}%
\end{pgfscope}%
\begin{pgfscope}%
\pgfpathrectangle{\pgfqpoint{0.536250in}{0.525000in}}{\pgfqpoint{1.310000in}{1.887500in}}%
\pgfusepath{clip}%
\pgfsetbuttcap%
\pgfsetroundjoin%
\pgfsetlinewidth{0.803000pt}%
\definecolor{currentstroke}{rgb}{0.752941,0.752941,0.752941}%
\pgfsetstrokecolor{currentstroke}%
\pgfsetdash{{2.960000pt}{1.280000pt}}{0.000000pt}%
\pgfpathmoveto{\pgfqpoint{0.536250in}{1.768939in}}%
\pgfpathlineto{\pgfqpoint{1.846250in}{1.768939in}}%
\pgfusepath{stroke}%
\end{pgfscope}%
\begin{pgfscope}%
\pgfsetbuttcap%
\pgfsetroundjoin%
\definecolor{currentfill}{rgb}{0.000000,0.000000,0.000000}%
\pgfsetfillcolor{currentfill}%
\pgfsetlinewidth{0.602250pt}%
\definecolor{currentstroke}{rgb}{0.000000,0.000000,0.000000}%
\pgfsetstrokecolor{currentstroke}%
\pgfsetdash{}{0pt}%
\pgfsys@defobject{currentmarker}{\pgfqpoint{-0.027778in}{0.000000in}}{\pgfqpoint{-0.000000in}{0.000000in}}{%
\pgfpathmoveto{\pgfqpoint{-0.000000in}{0.000000in}}%
\pgfpathlineto{\pgfqpoint{-0.027778in}{0.000000in}}%
\pgfusepath{stroke,fill}%
}%
\begin{pgfscope}%
\pgfsys@transformshift{0.536250in}{1.768939in}%
\pgfsys@useobject{currentmarker}{}%
\end{pgfscope}%
\end{pgfscope}%
\begin{pgfscope}%
\pgfpathrectangle{\pgfqpoint{0.536250in}{0.525000in}}{\pgfqpoint{1.310000in}{1.887500in}}%
\pgfusepath{clip}%
\pgfsetbuttcap%
\pgfsetroundjoin%
\pgfsetlinewidth{0.803000pt}%
\definecolor{currentstroke}{rgb}{0.752941,0.752941,0.752941}%
\pgfsetstrokecolor{currentstroke}%
\pgfsetdash{{2.960000pt}{1.280000pt}}{0.000000pt}%
\pgfpathmoveto{\pgfqpoint{0.536250in}{1.878032in}}%
\pgfpathlineto{\pgfqpoint{1.846250in}{1.878032in}}%
\pgfusepath{stroke}%
\end{pgfscope}%
\begin{pgfscope}%
\pgfsetbuttcap%
\pgfsetroundjoin%
\definecolor{currentfill}{rgb}{0.000000,0.000000,0.000000}%
\pgfsetfillcolor{currentfill}%
\pgfsetlinewidth{0.602250pt}%
\definecolor{currentstroke}{rgb}{0.000000,0.000000,0.000000}%
\pgfsetstrokecolor{currentstroke}%
\pgfsetdash{}{0pt}%
\pgfsys@defobject{currentmarker}{\pgfqpoint{-0.027778in}{0.000000in}}{\pgfqpoint{-0.000000in}{0.000000in}}{%
\pgfpathmoveto{\pgfqpoint{-0.000000in}{0.000000in}}%
\pgfpathlineto{\pgfqpoint{-0.027778in}{0.000000in}}%
\pgfusepath{stroke,fill}%
}%
\begin{pgfscope}%
\pgfsys@transformshift{0.536250in}{1.878032in}%
\pgfsys@useobject{currentmarker}{}%
\end{pgfscope}%
\end{pgfscope}%
\begin{pgfscope}%
\pgfpathrectangle{\pgfqpoint{0.536250in}{0.525000in}}{\pgfqpoint{1.310000in}{1.887500in}}%
\pgfusepath{clip}%
\pgfsetbuttcap%
\pgfsetroundjoin%
\pgfsetlinewidth{0.803000pt}%
\definecolor{currentstroke}{rgb}{0.752941,0.752941,0.752941}%
\pgfsetstrokecolor{currentstroke}%
\pgfsetdash{{2.960000pt}{1.280000pt}}{0.000000pt}%
\pgfpathmoveto{\pgfqpoint{0.536250in}{1.933428in}}%
\pgfpathlineto{\pgfqpoint{1.846250in}{1.933428in}}%
\pgfusepath{stroke}%
\end{pgfscope}%
\begin{pgfscope}%
\pgfsetbuttcap%
\pgfsetroundjoin%
\definecolor{currentfill}{rgb}{0.000000,0.000000,0.000000}%
\pgfsetfillcolor{currentfill}%
\pgfsetlinewidth{0.602250pt}%
\definecolor{currentstroke}{rgb}{0.000000,0.000000,0.000000}%
\pgfsetstrokecolor{currentstroke}%
\pgfsetdash{}{0pt}%
\pgfsys@defobject{currentmarker}{\pgfqpoint{-0.027778in}{0.000000in}}{\pgfqpoint{-0.000000in}{0.000000in}}{%
\pgfpathmoveto{\pgfqpoint{-0.000000in}{0.000000in}}%
\pgfpathlineto{\pgfqpoint{-0.027778in}{0.000000in}}%
\pgfusepath{stroke,fill}%
}%
\begin{pgfscope}%
\pgfsys@transformshift{0.536250in}{1.933428in}%
\pgfsys@useobject{currentmarker}{}%
\end{pgfscope}%
\end{pgfscope}%
\begin{pgfscope}%
\pgfpathrectangle{\pgfqpoint{0.536250in}{0.525000in}}{\pgfqpoint{1.310000in}{1.887500in}}%
\pgfusepath{clip}%
\pgfsetbuttcap%
\pgfsetroundjoin%
\pgfsetlinewidth{0.803000pt}%
\definecolor{currentstroke}{rgb}{0.752941,0.752941,0.752941}%
\pgfsetstrokecolor{currentstroke}%
\pgfsetdash{{2.960000pt}{1.280000pt}}{0.000000pt}%
\pgfpathmoveto{\pgfqpoint{0.536250in}{1.972731in}}%
\pgfpathlineto{\pgfqpoint{1.846250in}{1.972731in}}%
\pgfusepath{stroke}%
\end{pgfscope}%
\begin{pgfscope}%
\pgfsetbuttcap%
\pgfsetroundjoin%
\definecolor{currentfill}{rgb}{0.000000,0.000000,0.000000}%
\pgfsetfillcolor{currentfill}%
\pgfsetlinewidth{0.602250pt}%
\definecolor{currentstroke}{rgb}{0.000000,0.000000,0.000000}%
\pgfsetstrokecolor{currentstroke}%
\pgfsetdash{}{0pt}%
\pgfsys@defobject{currentmarker}{\pgfqpoint{-0.027778in}{0.000000in}}{\pgfqpoint{-0.000000in}{0.000000in}}{%
\pgfpathmoveto{\pgfqpoint{-0.000000in}{0.000000in}}%
\pgfpathlineto{\pgfqpoint{-0.027778in}{0.000000in}}%
\pgfusepath{stroke,fill}%
}%
\begin{pgfscope}%
\pgfsys@transformshift{0.536250in}{1.972731in}%
\pgfsys@useobject{currentmarker}{}%
\end{pgfscope}%
\end{pgfscope}%
\begin{pgfscope}%
\pgfpathrectangle{\pgfqpoint{0.536250in}{0.525000in}}{\pgfqpoint{1.310000in}{1.887500in}}%
\pgfusepath{clip}%
\pgfsetbuttcap%
\pgfsetroundjoin%
\pgfsetlinewidth{0.803000pt}%
\definecolor{currentstroke}{rgb}{0.752941,0.752941,0.752941}%
\pgfsetstrokecolor{currentstroke}%
\pgfsetdash{{2.960000pt}{1.280000pt}}{0.000000pt}%
\pgfpathmoveto{\pgfqpoint{0.536250in}{2.003218in}}%
\pgfpathlineto{\pgfqpoint{1.846250in}{2.003218in}}%
\pgfusepath{stroke}%
\end{pgfscope}%
\begin{pgfscope}%
\pgfsetbuttcap%
\pgfsetroundjoin%
\definecolor{currentfill}{rgb}{0.000000,0.000000,0.000000}%
\pgfsetfillcolor{currentfill}%
\pgfsetlinewidth{0.602250pt}%
\definecolor{currentstroke}{rgb}{0.000000,0.000000,0.000000}%
\pgfsetstrokecolor{currentstroke}%
\pgfsetdash{}{0pt}%
\pgfsys@defobject{currentmarker}{\pgfqpoint{-0.027778in}{0.000000in}}{\pgfqpoint{-0.000000in}{0.000000in}}{%
\pgfpathmoveto{\pgfqpoint{-0.000000in}{0.000000in}}%
\pgfpathlineto{\pgfqpoint{-0.027778in}{0.000000in}}%
\pgfusepath{stroke,fill}%
}%
\begin{pgfscope}%
\pgfsys@transformshift{0.536250in}{2.003218in}%
\pgfsys@useobject{currentmarker}{}%
\end{pgfscope}%
\end{pgfscope}%
\begin{pgfscope}%
\pgfpathrectangle{\pgfqpoint{0.536250in}{0.525000in}}{\pgfqpoint{1.310000in}{1.887500in}}%
\pgfusepath{clip}%
\pgfsetbuttcap%
\pgfsetroundjoin%
\pgfsetlinewidth{0.803000pt}%
\definecolor{currentstroke}{rgb}{0.752941,0.752941,0.752941}%
\pgfsetstrokecolor{currentstroke}%
\pgfsetdash{{2.960000pt}{1.280000pt}}{0.000000pt}%
\pgfpathmoveto{\pgfqpoint{0.536250in}{2.028127in}}%
\pgfpathlineto{\pgfqpoint{1.846250in}{2.028127in}}%
\pgfusepath{stroke}%
\end{pgfscope}%
\begin{pgfscope}%
\pgfsetbuttcap%
\pgfsetroundjoin%
\definecolor{currentfill}{rgb}{0.000000,0.000000,0.000000}%
\pgfsetfillcolor{currentfill}%
\pgfsetlinewidth{0.602250pt}%
\definecolor{currentstroke}{rgb}{0.000000,0.000000,0.000000}%
\pgfsetstrokecolor{currentstroke}%
\pgfsetdash{}{0pt}%
\pgfsys@defobject{currentmarker}{\pgfqpoint{-0.027778in}{0.000000in}}{\pgfqpoint{-0.000000in}{0.000000in}}{%
\pgfpathmoveto{\pgfqpoint{-0.000000in}{0.000000in}}%
\pgfpathlineto{\pgfqpoint{-0.027778in}{0.000000in}}%
\pgfusepath{stroke,fill}%
}%
\begin{pgfscope}%
\pgfsys@transformshift{0.536250in}{2.028127in}%
\pgfsys@useobject{currentmarker}{}%
\end{pgfscope}%
\end{pgfscope}%
\begin{pgfscope}%
\pgfpathrectangle{\pgfqpoint{0.536250in}{0.525000in}}{\pgfqpoint{1.310000in}{1.887500in}}%
\pgfusepath{clip}%
\pgfsetbuttcap%
\pgfsetroundjoin%
\pgfsetlinewidth{0.803000pt}%
\definecolor{currentstroke}{rgb}{0.752941,0.752941,0.752941}%
\pgfsetstrokecolor{currentstroke}%
\pgfsetdash{{2.960000pt}{1.280000pt}}{0.000000pt}%
\pgfpathmoveto{\pgfqpoint{0.536250in}{2.049187in}}%
\pgfpathlineto{\pgfqpoint{1.846250in}{2.049187in}}%
\pgfusepath{stroke}%
\end{pgfscope}%
\begin{pgfscope}%
\pgfsetbuttcap%
\pgfsetroundjoin%
\definecolor{currentfill}{rgb}{0.000000,0.000000,0.000000}%
\pgfsetfillcolor{currentfill}%
\pgfsetlinewidth{0.602250pt}%
\definecolor{currentstroke}{rgb}{0.000000,0.000000,0.000000}%
\pgfsetstrokecolor{currentstroke}%
\pgfsetdash{}{0pt}%
\pgfsys@defobject{currentmarker}{\pgfqpoint{-0.027778in}{0.000000in}}{\pgfqpoint{-0.000000in}{0.000000in}}{%
\pgfpathmoveto{\pgfqpoint{-0.000000in}{0.000000in}}%
\pgfpathlineto{\pgfqpoint{-0.027778in}{0.000000in}}%
\pgfusepath{stroke,fill}%
}%
\begin{pgfscope}%
\pgfsys@transformshift{0.536250in}{2.049187in}%
\pgfsys@useobject{currentmarker}{}%
\end{pgfscope}%
\end{pgfscope}%
\begin{pgfscope}%
\pgfpathrectangle{\pgfqpoint{0.536250in}{0.525000in}}{\pgfqpoint{1.310000in}{1.887500in}}%
\pgfusepath{clip}%
\pgfsetbuttcap%
\pgfsetroundjoin%
\pgfsetlinewidth{0.803000pt}%
\definecolor{currentstroke}{rgb}{0.752941,0.752941,0.752941}%
\pgfsetstrokecolor{currentstroke}%
\pgfsetdash{{2.960000pt}{1.280000pt}}{0.000000pt}%
\pgfpathmoveto{\pgfqpoint{0.536250in}{2.067430in}}%
\pgfpathlineto{\pgfqpoint{1.846250in}{2.067430in}}%
\pgfusepath{stroke}%
\end{pgfscope}%
\begin{pgfscope}%
\pgfsetbuttcap%
\pgfsetroundjoin%
\definecolor{currentfill}{rgb}{0.000000,0.000000,0.000000}%
\pgfsetfillcolor{currentfill}%
\pgfsetlinewidth{0.602250pt}%
\definecolor{currentstroke}{rgb}{0.000000,0.000000,0.000000}%
\pgfsetstrokecolor{currentstroke}%
\pgfsetdash{}{0pt}%
\pgfsys@defobject{currentmarker}{\pgfqpoint{-0.027778in}{0.000000in}}{\pgfqpoint{-0.000000in}{0.000000in}}{%
\pgfpathmoveto{\pgfqpoint{-0.000000in}{0.000000in}}%
\pgfpathlineto{\pgfqpoint{-0.027778in}{0.000000in}}%
\pgfusepath{stroke,fill}%
}%
\begin{pgfscope}%
\pgfsys@transformshift{0.536250in}{2.067430in}%
\pgfsys@useobject{currentmarker}{}%
\end{pgfscope}%
\end{pgfscope}%
\begin{pgfscope}%
\pgfpathrectangle{\pgfqpoint{0.536250in}{0.525000in}}{\pgfqpoint{1.310000in}{1.887500in}}%
\pgfusepath{clip}%
\pgfsetbuttcap%
\pgfsetroundjoin%
\pgfsetlinewidth{0.803000pt}%
\definecolor{currentstroke}{rgb}{0.752941,0.752941,0.752941}%
\pgfsetstrokecolor{currentstroke}%
\pgfsetdash{{2.960000pt}{1.280000pt}}{0.000000pt}%
\pgfpathmoveto{\pgfqpoint{0.536250in}{2.083522in}}%
\pgfpathlineto{\pgfqpoint{1.846250in}{2.083522in}}%
\pgfusepath{stroke}%
\end{pgfscope}%
\begin{pgfscope}%
\pgfsetbuttcap%
\pgfsetroundjoin%
\definecolor{currentfill}{rgb}{0.000000,0.000000,0.000000}%
\pgfsetfillcolor{currentfill}%
\pgfsetlinewidth{0.602250pt}%
\definecolor{currentstroke}{rgb}{0.000000,0.000000,0.000000}%
\pgfsetstrokecolor{currentstroke}%
\pgfsetdash{}{0pt}%
\pgfsys@defobject{currentmarker}{\pgfqpoint{-0.027778in}{0.000000in}}{\pgfqpoint{-0.000000in}{0.000000in}}{%
\pgfpathmoveto{\pgfqpoint{-0.000000in}{0.000000in}}%
\pgfpathlineto{\pgfqpoint{-0.027778in}{0.000000in}}%
\pgfusepath{stroke,fill}%
}%
\begin{pgfscope}%
\pgfsys@transformshift{0.536250in}{2.083522in}%
\pgfsys@useobject{currentmarker}{}%
\end{pgfscope}%
\end{pgfscope}%
\begin{pgfscope}%
\pgfpathrectangle{\pgfqpoint{0.536250in}{0.525000in}}{\pgfqpoint{1.310000in}{1.887500in}}%
\pgfusepath{clip}%
\pgfsetbuttcap%
\pgfsetroundjoin%
\pgfsetlinewidth{0.803000pt}%
\definecolor{currentstroke}{rgb}{0.752941,0.752941,0.752941}%
\pgfsetstrokecolor{currentstroke}%
\pgfsetdash{{2.960000pt}{1.280000pt}}{0.000000pt}%
\pgfpathmoveto{\pgfqpoint{0.536250in}{2.192616in}}%
\pgfpathlineto{\pgfqpoint{1.846250in}{2.192616in}}%
\pgfusepath{stroke}%
\end{pgfscope}%
\begin{pgfscope}%
\pgfsetbuttcap%
\pgfsetroundjoin%
\definecolor{currentfill}{rgb}{0.000000,0.000000,0.000000}%
\pgfsetfillcolor{currentfill}%
\pgfsetlinewidth{0.602250pt}%
\definecolor{currentstroke}{rgb}{0.000000,0.000000,0.000000}%
\pgfsetstrokecolor{currentstroke}%
\pgfsetdash{}{0pt}%
\pgfsys@defobject{currentmarker}{\pgfqpoint{-0.027778in}{0.000000in}}{\pgfqpoint{-0.000000in}{0.000000in}}{%
\pgfpathmoveto{\pgfqpoint{-0.000000in}{0.000000in}}%
\pgfpathlineto{\pgfqpoint{-0.027778in}{0.000000in}}%
\pgfusepath{stroke,fill}%
}%
\begin{pgfscope}%
\pgfsys@transformshift{0.536250in}{2.192616in}%
\pgfsys@useobject{currentmarker}{}%
\end{pgfscope}%
\end{pgfscope}%
\begin{pgfscope}%
\pgfpathrectangle{\pgfqpoint{0.536250in}{0.525000in}}{\pgfqpoint{1.310000in}{1.887500in}}%
\pgfusepath{clip}%
\pgfsetbuttcap%
\pgfsetroundjoin%
\pgfsetlinewidth{0.803000pt}%
\definecolor{currentstroke}{rgb}{0.752941,0.752941,0.752941}%
\pgfsetstrokecolor{currentstroke}%
\pgfsetdash{{2.960000pt}{1.280000pt}}{0.000000pt}%
\pgfpathmoveto{\pgfqpoint{0.536250in}{2.248011in}}%
\pgfpathlineto{\pgfqpoint{1.846250in}{2.248011in}}%
\pgfusepath{stroke}%
\end{pgfscope}%
\begin{pgfscope}%
\pgfsetbuttcap%
\pgfsetroundjoin%
\definecolor{currentfill}{rgb}{0.000000,0.000000,0.000000}%
\pgfsetfillcolor{currentfill}%
\pgfsetlinewidth{0.602250pt}%
\definecolor{currentstroke}{rgb}{0.000000,0.000000,0.000000}%
\pgfsetstrokecolor{currentstroke}%
\pgfsetdash{}{0pt}%
\pgfsys@defobject{currentmarker}{\pgfqpoint{-0.027778in}{0.000000in}}{\pgfqpoint{-0.000000in}{0.000000in}}{%
\pgfpathmoveto{\pgfqpoint{-0.000000in}{0.000000in}}%
\pgfpathlineto{\pgfqpoint{-0.027778in}{0.000000in}}%
\pgfusepath{stroke,fill}%
}%
\begin{pgfscope}%
\pgfsys@transformshift{0.536250in}{2.248011in}%
\pgfsys@useobject{currentmarker}{}%
\end{pgfscope}%
\end{pgfscope}%
\begin{pgfscope}%
\pgfpathrectangle{\pgfqpoint{0.536250in}{0.525000in}}{\pgfqpoint{1.310000in}{1.887500in}}%
\pgfusepath{clip}%
\pgfsetbuttcap%
\pgfsetroundjoin%
\pgfsetlinewidth{0.803000pt}%
\definecolor{currentstroke}{rgb}{0.752941,0.752941,0.752941}%
\pgfsetstrokecolor{currentstroke}%
\pgfsetdash{{2.960000pt}{1.280000pt}}{0.000000pt}%
\pgfpathmoveto{\pgfqpoint{0.536250in}{2.287315in}}%
\pgfpathlineto{\pgfqpoint{1.846250in}{2.287315in}}%
\pgfusepath{stroke}%
\end{pgfscope}%
\begin{pgfscope}%
\pgfsetbuttcap%
\pgfsetroundjoin%
\definecolor{currentfill}{rgb}{0.000000,0.000000,0.000000}%
\pgfsetfillcolor{currentfill}%
\pgfsetlinewidth{0.602250pt}%
\definecolor{currentstroke}{rgb}{0.000000,0.000000,0.000000}%
\pgfsetstrokecolor{currentstroke}%
\pgfsetdash{}{0pt}%
\pgfsys@defobject{currentmarker}{\pgfqpoint{-0.027778in}{0.000000in}}{\pgfqpoint{-0.000000in}{0.000000in}}{%
\pgfpathmoveto{\pgfqpoint{-0.000000in}{0.000000in}}%
\pgfpathlineto{\pgfqpoint{-0.027778in}{0.000000in}}%
\pgfusepath{stroke,fill}%
}%
\begin{pgfscope}%
\pgfsys@transformshift{0.536250in}{2.287315in}%
\pgfsys@useobject{currentmarker}{}%
\end{pgfscope}%
\end{pgfscope}%
\begin{pgfscope}%
\pgfpathrectangle{\pgfqpoint{0.536250in}{0.525000in}}{\pgfqpoint{1.310000in}{1.887500in}}%
\pgfusepath{clip}%
\pgfsetbuttcap%
\pgfsetroundjoin%
\pgfsetlinewidth{0.803000pt}%
\definecolor{currentstroke}{rgb}{0.752941,0.752941,0.752941}%
\pgfsetstrokecolor{currentstroke}%
\pgfsetdash{{2.960000pt}{1.280000pt}}{0.000000pt}%
\pgfpathmoveto{\pgfqpoint{0.536250in}{2.317801in}}%
\pgfpathlineto{\pgfqpoint{1.846250in}{2.317801in}}%
\pgfusepath{stroke}%
\end{pgfscope}%
\begin{pgfscope}%
\pgfsetbuttcap%
\pgfsetroundjoin%
\definecolor{currentfill}{rgb}{0.000000,0.000000,0.000000}%
\pgfsetfillcolor{currentfill}%
\pgfsetlinewidth{0.602250pt}%
\definecolor{currentstroke}{rgb}{0.000000,0.000000,0.000000}%
\pgfsetstrokecolor{currentstroke}%
\pgfsetdash{}{0pt}%
\pgfsys@defobject{currentmarker}{\pgfqpoint{-0.027778in}{0.000000in}}{\pgfqpoint{-0.000000in}{0.000000in}}{%
\pgfpathmoveto{\pgfqpoint{-0.000000in}{0.000000in}}%
\pgfpathlineto{\pgfqpoint{-0.027778in}{0.000000in}}%
\pgfusepath{stroke,fill}%
}%
\begin{pgfscope}%
\pgfsys@transformshift{0.536250in}{2.317801in}%
\pgfsys@useobject{currentmarker}{}%
\end{pgfscope}%
\end{pgfscope}%
\begin{pgfscope}%
\pgfpathrectangle{\pgfqpoint{0.536250in}{0.525000in}}{\pgfqpoint{1.310000in}{1.887500in}}%
\pgfusepath{clip}%
\pgfsetbuttcap%
\pgfsetroundjoin%
\pgfsetlinewidth{0.803000pt}%
\definecolor{currentstroke}{rgb}{0.752941,0.752941,0.752941}%
\pgfsetstrokecolor{currentstroke}%
\pgfsetdash{{2.960000pt}{1.280000pt}}{0.000000pt}%
\pgfpathmoveto{\pgfqpoint{0.536250in}{2.342710in}}%
\pgfpathlineto{\pgfqpoint{1.846250in}{2.342710in}}%
\pgfusepath{stroke}%
\end{pgfscope}%
\begin{pgfscope}%
\pgfsetbuttcap%
\pgfsetroundjoin%
\definecolor{currentfill}{rgb}{0.000000,0.000000,0.000000}%
\pgfsetfillcolor{currentfill}%
\pgfsetlinewidth{0.602250pt}%
\definecolor{currentstroke}{rgb}{0.000000,0.000000,0.000000}%
\pgfsetstrokecolor{currentstroke}%
\pgfsetdash{}{0pt}%
\pgfsys@defobject{currentmarker}{\pgfqpoint{-0.027778in}{0.000000in}}{\pgfqpoint{-0.000000in}{0.000000in}}{%
\pgfpathmoveto{\pgfqpoint{-0.000000in}{0.000000in}}%
\pgfpathlineto{\pgfqpoint{-0.027778in}{0.000000in}}%
\pgfusepath{stroke,fill}%
}%
\begin{pgfscope}%
\pgfsys@transformshift{0.536250in}{2.342710in}%
\pgfsys@useobject{currentmarker}{}%
\end{pgfscope}%
\end{pgfscope}%
\begin{pgfscope}%
\pgfpathrectangle{\pgfqpoint{0.536250in}{0.525000in}}{\pgfqpoint{1.310000in}{1.887500in}}%
\pgfusepath{clip}%
\pgfsetbuttcap%
\pgfsetroundjoin%
\pgfsetlinewidth{0.803000pt}%
\definecolor{currentstroke}{rgb}{0.752941,0.752941,0.752941}%
\pgfsetstrokecolor{currentstroke}%
\pgfsetdash{{2.960000pt}{1.280000pt}}{0.000000pt}%
\pgfpathmoveto{\pgfqpoint{0.536250in}{2.363770in}}%
\pgfpathlineto{\pgfqpoint{1.846250in}{2.363770in}}%
\pgfusepath{stroke}%
\end{pgfscope}%
\begin{pgfscope}%
\pgfsetbuttcap%
\pgfsetroundjoin%
\definecolor{currentfill}{rgb}{0.000000,0.000000,0.000000}%
\pgfsetfillcolor{currentfill}%
\pgfsetlinewidth{0.602250pt}%
\definecolor{currentstroke}{rgb}{0.000000,0.000000,0.000000}%
\pgfsetstrokecolor{currentstroke}%
\pgfsetdash{}{0pt}%
\pgfsys@defobject{currentmarker}{\pgfqpoint{-0.027778in}{0.000000in}}{\pgfqpoint{-0.000000in}{0.000000in}}{%
\pgfpathmoveto{\pgfqpoint{-0.000000in}{0.000000in}}%
\pgfpathlineto{\pgfqpoint{-0.027778in}{0.000000in}}%
\pgfusepath{stroke,fill}%
}%
\begin{pgfscope}%
\pgfsys@transformshift{0.536250in}{2.363770in}%
\pgfsys@useobject{currentmarker}{}%
\end{pgfscope}%
\end{pgfscope}%
\begin{pgfscope}%
\pgfpathrectangle{\pgfqpoint{0.536250in}{0.525000in}}{\pgfqpoint{1.310000in}{1.887500in}}%
\pgfusepath{clip}%
\pgfsetbuttcap%
\pgfsetroundjoin%
\pgfsetlinewidth{0.803000pt}%
\definecolor{currentstroke}{rgb}{0.752941,0.752941,0.752941}%
\pgfsetstrokecolor{currentstroke}%
\pgfsetdash{{2.960000pt}{1.280000pt}}{0.000000pt}%
\pgfpathmoveto{\pgfqpoint{0.536250in}{2.382014in}}%
\pgfpathlineto{\pgfqpoint{1.846250in}{2.382014in}}%
\pgfusepath{stroke}%
\end{pgfscope}%
\begin{pgfscope}%
\pgfsetbuttcap%
\pgfsetroundjoin%
\definecolor{currentfill}{rgb}{0.000000,0.000000,0.000000}%
\pgfsetfillcolor{currentfill}%
\pgfsetlinewidth{0.602250pt}%
\definecolor{currentstroke}{rgb}{0.000000,0.000000,0.000000}%
\pgfsetstrokecolor{currentstroke}%
\pgfsetdash{}{0pt}%
\pgfsys@defobject{currentmarker}{\pgfqpoint{-0.027778in}{0.000000in}}{\pgfqpoint{-0.000000in}{0.000000in}}{%
\pgfpathmoveto{\pgfqpoint{-0.000000in}{0.000000in}}%
\pgfpathlineto{\pgfqpoint{-0.027778in}{0.000000in}}%
\pgfusepath{stroke,fill}%
}%
\begin{pgfscope}%
\pgfsys@transformshift{0.536250in}{2.382014in}%
\pgfsys@useobject{currentmarker}{}%
\end{pgfscope}%
\end{pgfscope}%
\begin{pgfscope}%
\pgfpathrectangle{\pgfqpoint{0.536250in}{0.525000in}}{\pgfqpoint{1.310000in}{1.887500in}}%
\pgfusepath{clip}%
\pgfsetbuttcap%
\pgfsetroundjoin%
\pgfsetlinewidth{0.803000pt}%
\definecolor{currentstroke}{rgb}{0.752941,0.752941,0.752941}%
\pgfsetstrokecolor{currentstroke}%
\pgfsetdash{{2.960000pt}{1.280000pt}}{0.000000pt}%
\pgfpathmoveto{\pgfqpoint{0.536250in}{2.398105in}}%
\pgfpathlineto{\pgfqpoint{1.846250in}{2.398105in}}%
\pgfusepath{stroke}%
\end{pgfscope}%
\begin{pgfscope}%
\pgfsetbuttcap%
\pgfsetroundjoin%
\definecolor{currentfill}{rgb}{0.000000,0.000000,0.000000}%
\pgfsetfillcolor{currentfill}%
\pgfsetlinewidth{0.602250pt}%
\definecolor{currentstroke}{rgb}{0.000000,0.000000,0.000000}%
\pgfsetstrokecolor{currentstroke}%
\pgfsetdash{}{0pt}%
\pgfsys@defobject{currentmarker}{\pgfqpoint{-0.027778in}{0.000000in}}{\pgfqpoint{-0.000000in}{0.000000in}}{%
\pgfpathmoveto{\pgfqpoint{-0.000000in}{0.000000in}}%
\pgfpathlineto{\pgfqpoint{-0.027778in}{0.000000in}}%
\pgfusepath{stroke,fill}%
}%
\begin{pgfscope}%
\pgfsys@transformshift{0.536250in}{2.398105in}%
\pgfsys@useobject{currentmarker}{}%
\end{pgfscope}%
\end{pgfscope}%
\begin{pgfscope}%
\definecolor{textcolor}{rgb}{0.000000,0.000000,0.000000}%
\pgfsetstrokecolor{textcolor}%
\pgfsetfillcolor{textcolor}%
\pgftext[x=0.116885in,y=1.468750in,,bottom,rotate=90.000000]{\color{textcolor}\rmfamily\fontsize{9.000000}{10.800000}\selectfont Multiple scattering angle [\(\displaystyle ^\circ\)]}%
\end{pgfscope}%
\begin{pgfscope}%
\pgfpathrectangle{\pgfqpoint{0.536250in}{0.525000in}}{\pgfqpoint{1.310000in}{1.887500in}}%
\pgfusepath{clip}%
\pgfsetrectcap%
\pgfsetroundjoin%
\pgfsetlinewidth{1.003750pt}%
\definecolor{currentstroke}{rgb}{0.000000,0.000000,0.000000}%
\pgfsetstrokecolor{currentstroke}%
\pgfsetdash{}{0pt}%
\pgfpathmoveto{\pgfqpoint{0.536250in}{1.894382in}}%
\pgfpathlineto{\pgfqpoint{1.194067in}{1.262165in}}%
\pgfpathlineto{\pgfqpoint{1.291955in}{1.168136in}}%
\pgfpathlineto{\pgfqpoint{1.349391in}{1.112964in}}%
\pgfpathlineto{\pgfqpoint{1.390192in}{1.073773in}}%
\pgfpathlineto{\pgfqpoint{1.421859in}{1.043354in}}%
\pgfpathlineto{\pgfqpoint{1.447744in}{1.018490in}}%
\pgfpathlineto{\pgfqpoint{1.469636in}{0.997461in}}%
\pgfpathlineto{\pgfqpoint{1.488603in}{0.979242in}}%
\pgfpathlineto{\pgfqpoint{1.505336in}{0.963169in}}%
\pgfpathlineto{\pgfqpoint{1.520306in}{0.948789in}}%
\pgfpathlineto{\pgfqpoint{1.533849in}{0.935780in}}%
\pgfpathlineto{\pgfqpoint{1.546214in}{0.923903in}}%
\pgfpathlineto{\pgfqpoint{1.557590in}{0.912976in}}%
\pgfpathlineto{\pgfqpoint{1.568123in}{0.902858in}}%
\pgfpathlineto{\pgfqpoint{1.577929in}{0.893439in}}%
\pgfpathlineto{\pgfqpoint{1.587103in}{0.884627in}}%
\pgfpathlineto{\pgfqpoint{1.595720in}{0.876350in}}%
\pgfpathlineto{\pgfqpoint{1.603845in}{0.868545in}}%
\pgfpathlineto{\pgfqpoint{1.611531in}{0.861162in}}%
\pgfpathlineto{\pgfqpoint{1.618823in}{0.854158in}}%
\pgfpathlineto{\pgfqpoint{1.625759in}{0.847495in}}%
\pgfpathlineto{\pgfqpoint{1.632373in}{0.841142in}}%
\pgfpathlineto{\pgfqpoint{1.638692in}{0.835072in}}%
\pgfpathlineto{\pgfqpoint{1.644743in}{0.829260in}}%
\pgfpathlineto{\pgfqpoint{1.650547in}{0.823685in}}%
\pgfpathlineto{\pgfqpoint{1.656123in}{0.818329in}}%
\pgfpathlineto{\pgfqpoint{1.661489in}{0.813174in}}%
\pgfpathlineto{\pgfqpoint{1.666660in}{0.808208in}}%
\pgfpathlineto{\pgfqpoint{1.671649in}{0.803415in}}%
\pgfpathlineto{\pgfqpoint{1.676469in}{0.798785in}}%
\pgfpathlineto{\pgfqpoint{1.681132in}{0.794307in}}%
\pgfpathlineto{\pgfqpoint{1.685646in}{0.789970in}}%
\pgfpathlineto{\pgfqpoint{1.690021in}{0.785768in}}%
\pgfpathlineto{\pgfqpoint{1.694266in}{0.781690in}}%
\pgfpathlineto{\pgfqpoint{1.698388in}{0.777731in}}%
\pgfpathlineto{\pgfqpoint{1.702393in}{0.773883in}}%
\pgfpathlineto{\pgfqpoint{1.706289in}{0.770141in}}%
\pgfpathlineto{\pgfqpoint{1.710081in}{0.766499in}}%
\pgfpathlineto{\pgfqpoint{1.713775in}{0.762951in}}%
\pgfpathlineto{\pgfqpoint{1.717375in}{0.759493in}}%
\pgfpathlineto{\pgfqpoint{1.720886in}{0.756120in}}%
\pgfpathlineto{\pgfqpoint{1.724313in}{0.752828in}}%
\pgfpathlineto{\pgfqpoint{1.727659in}{0.749614in}}%
\pgfpathlineto{\pgfqpoint{1.730928in}{0.746474in}}%
\pgfpathlineto{\pgfqpoint{1.734124in}{0.743405in}}%
\pgfpathlineto{\pgfqpoint{1.737249in}{0.740402in}}%
\pgfpathlineto{\pgfqpoint{1.740307in}{0.737465in}}%
\pgfpathlineto{\pgfqpoint{1.743301in}{0.734589in}}%
\pgfpathlineto{\pgfqpoint{1.746233in}{0.731773in}}%
\pgfpathlineto{\pgfqpoint{1.749106in}{0.729013in}}%
\pgfpathlineto{\pgfqpoint{1.751922in}{0.726308in}}%
\pgfpathlineto{\pgfqpoint{1.754683in}{0.723656in}}%
\pgfpathlineto{\pgfqpoint{1.757392in}{0.721054in}}%
\pgfpathlineto{\pgfqpoint{1.760050in}{0.718500in}}%
\pgfpathlineto{\pgfqpoint{1.762660in}{0.715994in}}%
\pgfpathlineto{\pgfqpoint{1.765222in}{0.713533in}}%
\pgfpathlineto{\pgfqpoint{1.767739in}{0.711115in}}%
\pgfpathlineto{\pgfqpoint{1.770212in}{0.708739in}}%
\pgfpathlineto{\pgfqpoint{1.772643in}{0.706404in}}%
\pgfpathlineto{\pgfqpoint{1.775033in}{0.704108in}}%
\pgfpathlineto{\pgfqpoint{1.777384in}{0.701850in}}%
\pgfpathlineto{\pgfqpoint{1.779696in}{0.699629in}}%
\pgfpathlineto{\pgfqpoint{1.781972in}{0.697444in}}%
\pgfpathlineto{\pgfqpoint{1.784211in}{0.695292in}}%
\pgfpathlineto{\pgfqpoint{1.786416in}{0.693175in}}%
\pgfpathlineto{\pgfqpoint{1.788587in}{0.691089in}}%
\pgfpathlineto{\pgfqpoint{1.790726in}{0.689035in}}%
\pgfpathlineto{\pgfqpoint{1.792833in}{0.687011in}}%
\pgfpathlineto{\pgfqpoint{1.794909in}{0.685017in}}%
\pgfpathlineto{\pgfqpoint{1.796955in}{0.683051in}}%
\pgfpathlineto{\pgfqpoint{1.798972in}{0.681114in}}%
\pgfpathlineto{\pgfqpoint{1.800961in}{0.679203in}}%
\pgfpathlineto{\pgfqpoint{1.802923in}{0.677319in}}%
\pgfpathlineto{\pgfqpoint{1.804858in}{0.675460in}}%
\pgfpathlineto{\pgfqpoint{1.806767in}{0.673627in}}%
\pgfpathlineto{\pgfqpoint{1.808650in}{0.671817in}}%
\pgfpathlineto{\pgfqpoint{1.810509in}{0.670032in}}%
\pgfpathlineto{\pgfqpoint{1.812344in}{0.668269in}}%
\pgfpathlineto{\pgfqpoint{1.814156in}{0.666529in}}%
\pgfpathlineto{\pgfqpoint{1.815945in}{0.664810in}}%
\pgfpathlineto{\pgfqpoint{1.817711in}{0.663113in}}%
\pgfpathlineto{\pgfqpoint{1.819456in}{0.661437in}}%
\pgfpathlineto{\pgfqpoint{1.821180in}{0.659781in}}%
\pgfpathlineto{\pgfqpoint{1.822883in}{0.658145in}}%
\pgfpathlineto{\pgfqpoint{1.824567in}{0.656529in}}%
\pgfpathlineto{\pgfqpoint{1.826230in}{0.654931in}}%
\pgfpathlineto{\pgfqpoint{1.827874in}{0.653352in}}%
\pgfpathlineto{\pgfqpoint{1.829499in}{0.651790in}}%
\pgfpathlineto{\pgfqpoint{1.831106in}{0.650247in}}%
\pgfpathlineto{\pgfqpoint{1.832695in}{0.648721in}}%
\pgfpathlineto{\pgfqpoint{1.834267in}{0.647211in}}%
\pgfpathlineto{\pgfqpoint{1.835821in}{0.645718in}}%
\pgfpathlineto{\pgfqpoint{1.837359in}{0.644241in}}%
\pgfpathlineto{\pgfqpoint{1.838880in}{0.642780in}}%
\pgfpathlineto{\pgfqpoint{1.840385in}{0.641335in}}%
\pgfpathlineto{\pgfqpoint{1.841874in}{0.639904in}}%
\pgfpathlineto{\pgfqpoint{1.843348in}{0.638488in}}%
\pgfpathlineto{\pgfqpoint{1.844806in}{0.637087in}}%
\pgfpathlineto{\pgfqpoint{1.846250in}{0.635700in}}%
\pgfusepath{stroke}%
\end{pgfscope}%
\begin{pgfscope}%
\pgfpathrectangle{\pgfqpoint{0.536250in}{0.525000in}}{\pgfqpoint{1.310000in}{1.887500in}}%
\pgfusepath{clip}%
\pgfsetbuttcap%
\pgfsetroundjoin%
\pgfsetlinewidth{1.003750pt}%
\definecolor{currentstroke}{rgb}{0.000000,0.000000,0.000000}%
\pgfsetstrokecolor{currentstroke}%
\pgfsetdash{{3.700000pt}{1.600000pt}}{0.000000pt}%
\pgfpathmoveto{\pgfqpoint{0.536250in}{2.053893in}}%
\pgfpathlineto{\pgfqpoint{1.194067in}{1.268173in}}%
\pgfpathlineto{\pgfqpoint{1.291955in}{1.171221in}}%
\pgfpathlineto{\pgfqpoint{1.349391in}{1.115040in}}%
\pgfpathlineto{\pgfqpoint{1.390192in}{1.075336in}}%
\pgfpathlineto{\pgfqpoint{1.421859in}{1.044608in}}%
\pgfpathlineto{\pgfqpoint{1.447744in}{1.019537in}}%
\pgfpathlineto{\pgfqpoint{1.469636in}{0.998360in}}%
\pgfpathlineto{\pgfqpoint{1.488603in}{0.980029in}}%
\pgfpathlineto{\pgfqpoint{1.505336in}{0.963869in}}%
\pgfpathlineto{\pgfqpoint{1.520306in}{0.949420in}}%
\pgfpathlineto{\pgfqpoint{1.533849in}{0.936354in}}%
\pgfpathlineto{\pgfqpoint{1.546214in}{0.924429in}}%
\pgfpathlineto{\pgfqpoint{1.557590in}{0.913462in}}%
\pgfpathlineto{\pgfqpoint{1.568123in}{0.903310in}}%
\pgfpathlineto{\pgfqpoint{1.577929in}{0.893860in}}%
\pgfpathlineto{\pgfqpoint{1.587103in}{0.885022in}}%
\pgfpathlineto{\pgfqpoint{1.595720in}{0.876721in}}%
\pgfpathlineto{\pgfqpoint{1.603845in}{0.868896in}}%
\pgfpathlineto{\pgfqpoint{1.611531in}{0.861495in}}%
\pgfpathlineto{\pgfqpoint{1.618823in}{0.854474in}}%
\pgfpathlineto{\pgfqpoint{1.625759in}{0.847796in}}%
\pgfpathlineto{\pgfqpoint{1.632373in}{0.841430in}}%
\pgfpathlineto{\pgfqpoint{1.638692in}{0.835347in}}%
\pgfpathlineto{\pgfqpoint{1.644743in}{0.829524in}}%
\pgfpathlineto{\pgfqpoint{1.650547in}{0.823938in}}%
\pgfpathlineto{\pgfqpoint{1.656123in}{0.818572in}}%
\pgfpathlineto{\pgfqpoint{1.661489in}{0.813409in}}%
\pgfpathlineto{\pgfqpoint{1.666660in}{0.808434in}}%
\pgfpathlineto{\pgfqpoint{1.671649in}{0.803633in}}%
\pgfpathlineto{\pgfqpoint{1.676469in}{0.798996in}}%
\pgfpathlineto{\pgfqpoint{1.681132in}{0.794511in}}%
\pgfpathlineto{\pgfqpoint{1.685646in}{0.790168in}}%
\pgfpathlineto{\pgfqpoint{1.690021in}{0.785959in}}%
\pgfpathlineto{\pgfqpoint{1.694266in}{0.781876in}}%
\pgfpathlineto{\pgfqpoint{1.698388in}{0.777912in}}%
\pgfpathlineto{\pgfqpoint{1.702393in}{0.774059in}}%
\pgfpathlineto{\pgfqpoint{1.706289in}{0.770312in}}%
\pgfpathlineto{\pgfqpoint{1.710081in}{0.766665in}}%
\pgfpathlineto{\pgfqpoint{1.713775in}{0.763113in}}%
\pgfpathlineto{\pgfqpoint{1.717375in}{0.759651in}}%
\pgfpathlineto{\pgfqpoint{1.720886in}{0.756274in}}%
\pgfpathlineto{\pgfqpoint{1.724313in}{0.752979in}}%
\pgfpathlineto{\pgfqpoint{1.727659in}{0.749762in}}%
\pgfpathlineto{\pgfqpoint{1.730928in}{0.746618in}}%
\pgfpathlineto{\pgfqpoint{1.734124in}{0.743545in}}%
\pgfpathlineto{\pgfqpoint{1.737249in}{0.740540in}}%
\pgfpathlineto{\pgfqpoint{1.740307in}{0.737600in}}%
\pgfpathlineto{\pgfqpoint{1.743301in}{0.734721in}}%
\pgfpathlineto{\pgfqpoint{1.746233in}{0.731902in}}%
\pgfpathlineto{\pgfqpoint{1.749106in}{0.729140in}}%
\pgfpathlineto{\pgfqpoint{1.751922in}{0.726432in}}%
\pgfpathlineto{\pgfqpoint{1.754683in}{0.723777in}}%
\pgfpathlineto{\pgfqpoint{1.757392in}{0.721173in}}%
\pgfpathlineto{\pgfqpoint{1.760050in}{0.718618in}}%
\pgfpathlineto{\pgfqpoint{1.762660in}{0.716109in}}%
\pgfpathlineto{\pgfqpoint{1.765222in}{0.713646in}}%
\pgfpathlineto{\pgfqpoint{1.767739in}{0.711226in}}%
\pgfpathlineto{\pgfqpoint{1.770212in}{0.708849in}}%
\pgfpathlineto{\pgfqpoint{1.772643in}{0.706512in}}%
\pgfpathlineto{\pgfqpoint{1.775033in}{0.704214in}}%
\pgfpathlineto{\pgfqpoint{1.777384in}{0.701954in}}%
\pgfpathlineto{\pgfqpoint{1.779696in}{0.699732in}}%
\pgfpathlineto{\pgfqpoint{1.781972in}{0.697544in}}%
\pgfpathlineto{\pgfqpoint{1.784211in}{0.695391in}}%
\pgfpathlineto{\pgfqpoint{1.786416in}{0.693272in}}%
\pgfpathlineto{\pgfqpoint{1.788587in}{0.691185in}}%
\pgfpathlineto{\pgfqpoint{1.790726in}{0.689129in}}%
\pgfpathlineto{\pgfqpoint{1.792833in}{0.687104in}}%
\pgfpathlineto{\pgfqpoint{1.794909in}{0.685109in}}%
\pgfpathlineto{\pgfqpoint{1.796955in}{0.683142in}}%
\pgfpathlineto{\pgfqpoint{1.798972in}{0.681203in}}%
\pgfpathlineto{\pgfqpoint{1.800961in}{0.679291in}}%
\pgfpathlineto{\pgfqpoint{1.802923in}{0.677406in}}%
\pgfpathlineto{\pgfqpoint{1.804858in}{0.675546in}}%
\pgfpathlineto{\pgfqpoint{1.806767in}{0.673711in}}%
\pgfpathlineto{\pgfqpoint{1.808650in}{0.671901in}}%
\pgfpathlineto{\pgfqpoint{1.810509in}{0.670114in}}%
\pgfpathlineto{\pgfqpoint{1.812344in}{0.668350in}}%
\pgfpathlineto{\pgfqpoint{1.814156in}{0.666609in}}%
\pgfpathlineto{\pgfqpoint{1.815945in}{0.664890in}}%
\pgfpathlineto{\pgfqpoint{1.817711in}{0.663192in}}%
\pgfpathlineto{\pgfqpoint{1.819456in}{0.661515in}}%
\pgfpathlineto{\pgfqpoint{1.821180in}{0.659858in}}%
\pgfpathlineto{\pgfqpoint{1.822883in}{0.658221in}}%
\pgfpathlineto{\pgfqpoint{1.824567in}{0.656603in}}%
\pgfpathlineto{\pgfqpoint{1.826230in}{0.655005in}}%
\pgfpathlineto{\pgfqpoint{1.827874in}{0.653425in}}%
\pgfpathlineto{\pgfqpoint{1.829499in}{0.651862in}}%
\pgfpathlineto{\pgfqpoint{1.831106in}{0.650318in}}%
\pgfpathlineto{\pgfqpoint{1.832695in}{0.648791in}}%
\pgfpathlineto{\pgfqpoint{1.834267in}{0.647281in}}%
\pgfpathlineto{\pgfqpoint{1.835821in}{0.645787in}}%
\pgfpathlineto{\pgfqpoint{1.837359in}{0.644309in}}%
\pgfpathlineto{\pgfqpoint{1.838880in}{0.642848in}}%
\pgfpathlineto{\pgfqpoint{1.840385in}{0.641401in}}%
\pgfpathlineto{\pgfqpoint{1.841874in}{0.639970in}}%
\pgfpathlineto{\pgfqpoint{1.843348in}{0.638554in}}%
\pgfpathlineto{\pgfqpoint{1.844806in}{0.637152in}}%
\pgfpathlineto{\pgfqpoint{1.846250in}{0.635764in}}%
\pgfusepath{stroke}%
\end{pgfscope}%
\begin{pgfscope}%
\pgfpathrectangle{\pgfqpoint{0.536250in}{0.525000in}}{\pgfqpoint{1.310000in}{1.887500in}}%
\pgfusepath{clip}%
\pgfsetrectcap%
\pgfsetroundjoin%
\pgfsetlinewidth{1.003750pt}%
\definecolor{currentstroke}{rgb}{0.000000,0.000000,0.000000}%
\pgfsetstrokecolor{currentstroke}%
\pgfsetdash{}{0pt}%
\pgfpathmoveto{\pgfqpoint{0.536250in}{1.723318in}}%
\pgfpathlineto{\pgfqpoint{1.194067in}{1.091102in}}%
\pgfpathlineto{\pgfqpoint{1.291955in}{0.997072in}}%
\pgfpathlineto{\pgfqpoint{1.349391in}{0.941901in}}%
\pgfpathlineto{\pgfqpoint{1.390192in}{0.902709in}}%
\pgfpathlineto{\pgfqpoint{1.421859in}{0.872290in}}%
\pgfpathlineto{\pgfqpoint{1.447744in}{0.847426in}}%
\pgfpathlineto{\pgfqpoint{1.469636in}{0.826398in}}%
\pgfpathlineto{\pgfqpoint{1.488603in}{0.808179in}}%
\pgfpathlineto{\pgfqpoint{1.505336in}{0.792106in}}%
\pgfpathlineto{\pgfqpoint{1.520306in}{0.777726in}}%
\pgfpathlineto{\pgfqpoint{1.533849in}{0.764717in}}%
\pgfpathlineto{\pgfqpoint{1.546214in}{0.752839in}}%
\pgfpathlineto{\pgfqpoint{1.557590in}{0.741912in}}%
\pgfpathlineto{\pgfqpoint{1.568123in}{0.731795in}}%
\pgfpathlineto{\pgfqpoint{1.577929in}{0.722376in}}%
\pgfpathlineto{\pgfqpoint{1.587103in}{0.713564in}}%
\pgfpathlineto{\pgfqpoint{1.595720in}{0.705286in}}%
\pgfpathlineto{\pgfqpoint{1.603845in}{0.697481in}}%
\pgfpathlineto{\pgfqpoint{1.611531in}{0.690099in}}%
\pgfpathlineto{\pgfqpoint{1.618823in}{0.683094in}}%
\pgfpathlineto{\pgfqpoint{1.625759in}{0.676432in}}%
\pgfpathlineto{\pgfqpoint{1.632373in}{0.670079in}}%
\pgfpathlineto{\pgfqpoint{1.638692in}{0.664009in}}%
\pgfpathlineto{\pgfqpoint{1.644743in}{0.658196in}}%
\pgfpathlineto{\pgfqpoint{1.650547in}{0.652622in}}%
\pgfpathlineto{\pgfqpoint{1.656123in}{0.647265in}}%
\pgfpathlineto{\pgfqpoint{1.661489in}{0.642111in}}%
\pgfpathlineto{\pgfqpoint{1.666660in}{0.637144in}}%
\pgfpathlineto{\pgfqpoint{1.671649in}{0.632352in}}%
\pgfpathlineto{\pgfqpoint{1.676469in}{0.627721in}}%
\pgfpathlineto{\pgfqpoint{1.681132in}{0.623243in}}%
\pgfpathlineto{\pgfqpoint{1.685646in}{0.618907in}}%
\pgfpathlineto{\pgfqpoint{1.690021in}{0.614704in}}%
\pgfpathlineto{\pgfqpoint{1.694266in}{0.610627in}}%
\pgfpathlineto{\pgfqpoint{1.698388in}{0.606668in}}%
\pgfpathlineto{\pgfqpoint{1.702393in}{0.602820in}}%
\pgfpathlineto{\pgfqpoint{1.706289in}{0.599078in}}%
\pgfpathlineto{\pgfqpoint{1.710081in}{0.595435in}}%
\pgfpathlineto{\pgfqpoint{1.713775in}{0.591887in}}%
\pgfpathlineto{\pgfqpoint{1.717375in}{0.588429in}}%
\pgfpathlineto{\pgfqpoint{1.720886in}{0.585056in}}%
\pgfpathlineto{\pgfqpoint{1.724313in}{0.581765in}}%
\pgfpathlineto{\pgfqpoint{1.727659in}{0.578551in}}%
\pgfpathlineto{\pgfqpoint{1.730928in}{0.575411in}}%
\pgfpathlineto{\pgfqpoint{1.734124in}{0.572341in}}%
\pgfpathlineto{\pgfqpoint{1.737249in}{0.569339in}}%
\pgfpathlineto{\pgfqpoint{1.740307in}{0.566401in}}%
\pgfpathlineto{\pgfqpoint{1.743301in}{0.563526in}}%
\pgfpathlineto{\pgfqpoint{1.746233in}{0.560709in}}%
\pgfpathlineto{\pgfqpoint{1.749106in}{0.557950in}}%
\pgfpathlineto{\pgfqpoint{1.751922in}{0.555245in}}%
\pgfpathlineto{\pgfqpoint{1.754683in}{0.552592in}}%
\pgfpathlineto{\pgfqpoint{1.757392in}{0.549990in}}%
\pgfpathlineto{\pgfqpoint{1.760050in}{0.547437in}}%
\pgfpathlineto{\pgfqpoint{1.762660in}{0.544931in}}%
\pgfpathlineto{\pgfqpoint{1.765222in}{0.542469in}}%
\pgfpathlineto{\pgfqpoint{1.767739in}{0.540052in}}%
\pgfpathlineto{\pgfqpoint{1.770212in}{0.537676in}}%
\pgfpathlineto{\pgfqpoint{1.772643in}{0.535341in}}%
\pgfpathlineto{\pgfqpoint{1.775033in}{0.533045in}}%
\pgfpathlineto{\pgfqpoint{1.777384in}{0.530787in}}%
\pgfpathlineto{\pgfqpoint{1.779696in}{0.528566in}}%
\pgfpathlineto{\pgfqpoint{1.781972in}{0.526380in}}%
\pgfpathlineto{\pgfqpoint{1.784211in}{0.524229in}}%
\pgfpathlineto{\pgfqpoint{1.786416in}{0.522111in}}%
\pgfpathlineto{\pgfqpoint{1.788587in}{0.520026in}}%
\pgfpathlineto{\pgfqpoint{1.790726in}{0.517971in}}%
\pgfpathlineto{\pgfqpoint{1.792833in}{0.515948in}}%
\pgfpathlineto{\pgfqpoint{1.793819in}{0.515000in}}%
\pgfusepath{stroke}%
\end{pgfscope}%
\begin{pgfscope}%
\pgfpathrectangle{\pgfqpoint{0.536250in}{0.525000in}}{\pgfqpoint{1.310000in}{1.887500in}}%
\pgfusepath{clip}%
\pgfsetbuttcap%
\pgfsetroundjoin%
\pgfsetlinewidth{1.003750pt}%
\definecolor{currentstroke}{rgb}{0.000000,0.000000,0.000000}%
\pgfsetstrokecolor{currentstroke}%
\pgfsetdash{{3.700000pt}{1.600000pt}}{0.000000pt}%
\pgfpathmoveto{\pgfqpoint{0.536250in}{1.882829in}}%
\pgfpathlineto{\pgfqpoint{1.194067in}{1.097110in}}%
\pgfpathlineto{\pgfqpoint{1.291955in}{1.000157in}}%
\pgfpathlineto{\pgfqpoint{1.349391in}{0.943976in}}%
\pgfpathlineto{\pgfqpoint{1.390192in}{0.904273in}}%
\pgfpathlineto{\pgfqpoint{1.421859in}{0.873545in}}%
\pgfpathlineto{\pgfqpoint{1.447744in}{0.848473in}}%
\pgfpathlineto{\pgfqpoint{1.469636in}{0.827297in}}%
\pgfpathlineto{\pgfqpoint{1.488603in}{0.808966in}}%
\pgfpathlineto{\pgfqpoint{1.505336in}{0.792806in}}%
\pgfpathlineto{\pgfqpoint{1.520306in}{0.778357in}}%
\pgfpathlineto{\pgfqpoint{1.533849in}{0.765290in}}%
\pgfpathlineto{\pgfqpoint{1.546214in}{0.753365in}}%
\pgfpathlineto{\pgfqpoint{1.557590in}{0.742398in}}%
\pgfpathlineto{\pgfqpoint{1.568123in}{0.732246in}}%
\pgfpathlineto{\pgfqpoint{1.577929in}{0.722797in}}%
\pgfpathlineto{\pgfqpoint{1.587103in}{0.713959in}}%
\pgfpathlineto{\pgfqpoint{1.595720in}{0.705658in}}%
\pgfpathlineto{\pgfqpoint{1.603845in}{0.697833in}}%
\pgfpathlineto{\pgfqpoint{1.611531in}{0.690431in}}%
\pgfpathlineto{\pgfqpoint{1.618823in}{0.683411in}}%
\pgfpathlineto{\pgfqpoint{1.625759in}{0.676733in}}%
\pgfpathlineto{\pgfqpoint{1.632373in}{0.670367in}}%
\pgfpathlineto{\pgfqpoint{1.638692in}{0.664284in}}%
\pgfpathlineto{\pgfqpoint{1.644743in}{0.658460in}}%
\pgfpathlineto{\pgfqpoint{1.650547in}{0.652875in}}%
\pgfpathlineto{\pgfqpoint{1.656123in}{0.647509in}}%
\pgfpathlineto{\pgfqpoint{1.661489in}{0.642345in}}%
\pgfpathlineto{\pgfqpoint{1.666660in}{0.637370in}}%
\pgfpathlineto{\pgfqpoint{1.671649in}{0.632570in}}%
\pgfpathlineto{\pgfqpoint{1.676469in}{0.627932in}}%
\pgfpathlineto{\pgfqpoint{1.681132in}{0.623447in}}%
\pgfpathlineto{\pgfqpoint{1.685646in}{0.619105in}}%
\pgfpathlineto{\pgfqpoint{1.690021in}{0.614896in}}%
\pgfpathlineto{\pgfqpoint{1.694266in}{0.610813in}}%
\pgfpathlineto{\pgfqpoint{1.698388in}{0.606848in}}%
\pgfpathlineto{\pgfqpoint{1.702393in}{0.602996in}}%
\pgfpathlineto{\pgfqpoint{1.706289in}{0.599249in}}%
\pgfpathlineto{\pgfqpoint{1.710081in}{0.595602in}}%
\pgfpathlineto{\pgfqpoint{1.713775in}{0.592050in}}%
\pgfpathlineto{\pgfqpoint{1.717375in}{0.588587in}}%
\pgfpathlineto{\pgfqpoint{1.720886in}{0.585211in}}%
\pgfpathlineto{\pgfqpoint{1.724313in}{0.581916in}}%
\pgfpathlineto{\pgfqpoint{1.727659in}{0.578698in}}%
\pgfpathlineto{\pgfqpoint{1.730928in}{0.575555in}}%
\pgfpathlineto{\pgfqpoint{1.734124in}{0.572482in}}%
\pgfpathlineto{\pgfqpoint{1.737249in}{0.569477in}}%
\pgfpathlineto{\pgfqpoint{1.740307in}{0.566536in}}%
\pgfpathlineto{\pgfqpoint{1.743301in}{0.563658in}}%
\pgfpathlineto{\pgfqpoint{1.746233in}{0.560838in}}%
\pgfpathlineto{\pgfqpoint{1.749106in}{0.558076in}}%
\pgfpathlineto{\pgfqpoint{1.751922in}{0.555369in}}%
\pgfpathlineto{\pgfqpoint{1.754683in}{0.552714in}}%
\pgfpathlineto{\pgfqpoint{1.757392in}{0.550110in}}%
\pgfpathlineto{\pgfqpoint{1.760050in}{0.547554in}}%
\pgfpathlineto{\pgfqpoint{1.762660in}{0.545046in}}%
\pgfpathlineto{\pgfqpoint{1.765222in}{0.542582in}}%
\pgfpathlineto{\pgfqpoint{1.767739in}{0.540163in}}%
\pgfpathlineto{\pgfqpoint{1.770212in}{0.537785in}}%
\pgfpathlineto{\pgfqpoint{1.772643in}{0.535448in}}%
\pgfpathlineto{\pgfqpoint{1.775033in}{0.533151in}}%
\pgfpathlineto{\pgfqpoint{1.777384in}{0.530891in}}%
\pgfpathlineto{\pgfqpoint{1.779696in}{0.528668in}}%
\pgfpathlineto{\pgfqpoint{1.781972in}{0.526481in}}%
\pgfpathlineto{\pgfqpoint{1.784211in}{0.524328in}}%
\pgfpathlineto{\pgfqpoint{1.786416in}{0.522209in}}%
\pgfpathlineto{\pgfqpoint{1.788587in}{0.520122in}}%
\pgfpathlineto{\pgfqpoint{1.790726in}{0.518066in}}%
\pgfpathlineto{\pgfqpoint{1.792833in}{0.516041in}}%
\pgfpathlineto{\pgfqpoint{1.793915in}{0.515000in}}%
\pgfusepath{stroke}%
\end{pgfscope}%
\begin{pgfscope}%
\pgfsetrectcap%
\pgfsetmiterjoin%
\pgfsetlinewidth{1.003750pt}%
\definecolor{currentstroke}{rgb}{0.000000,0.000000,0.000000}%
\pgfsetstrokecolor{currentstroke}%
\pgfsetdash{}{0pt}%
\pgfpathmoveto{\pgfqpoint{0.536250in}{0.525000in}}%
\pgfpathlineto{\pgfqpoint{0.536250in}{2.412500in}}%
\pgfusepath{stroke}%
\end{pgfscope}%
\begin{pgfscope}%
\pgfsetrectcap%
\pgfsetmiterjoin%
\pgfsetlinewidth{1.003750pt}%
\definecolor{currentstroke}{rgb}{0.000000,0.000000,0.000000}%
\pgfsetstrokecolor{currentstroke}%
\pgfsetdash{}{0pt}%
\pgfpathmoveto{\pgfqpoint{1.846250in}{0.525000in}}%
\pgfpathlineto{\pgfqpoint{1.846250in}{2.412500in}}%
\pgfusepath{stroke}%
\end{pgfscope}%
\begin{pgfscope}%
\pgfsetrectcap%
\pgfsetmiterjoin%
\pgfsetlinewidth{1.003750pt}%
\definecolor{currentstroke}{rgb}{0.000000,0.000000,0.000000}%
\pgfsetstrokecolor{currentstroke}%
\pgfsetdash{}{0pt}%
\pgfpathmoveto{\pgfqpoint{0.536250in}{0.525000in}}%
\pgfpathlineto{\pgfqpoint{1.846250in}{0.525000in}}%
\pgfusepath{stroke}%
\end{pgfscope}%
\begin{pgfscope}%
\pgfsetrectcap%
\pgfsetmiterjoin%
\pgfsetlinewidth{1.003750pt}%
\definecolor{currentstroke}{rgb}{0.000000,0.000000,0.000000}%
\pgfsetstrokecolor{currentstroke}%
\pgfsetdash{}{0pt}%
\pgfpathmoveto{\pgfqpoint{0.536250in}{2.412500in}}%
\pgfpathlineto{\pgfqpoint{1.846250in}{2.412500in}}%
\pgfusepath{stroke}%
\end{pgfscope}%
\begin{pgfscope}%
\definecolor{textcolor}{rgb}{0.000000,0.000000,0.000000}%
\pgfsetstrokecolor{textcolor}%
\pgfsetfillcolor{textcolor}%
\pgftext[x=1.191250in,y=1.374051in,left,base]{\color{textcolor}\rmfamily\fontsize{7.497000}{8.996400}\selectfont 10\% \(\displaystyle X_0\)}%
\end{pgfscope}%
\begin{pgfscope}%
\definecolor{textcolor}{rgb}{0.000000,0.000000,0.000000}%
\pgfsetstrokecolor{textcolor}%
\pgfsetfillcolor{textcolor}%
\pgftext[x=0.765163in,y=1.059468in,left,base]{\color{textcolor}\rmfamily\fontsize{7.497000}{8.996400}\selectfont 1\% \(\displaystyle X_0\)}%
\end{pgfscope}%
\begin{pgfscope}%
\pgfsetbuttcap%
\pgfsetmiterjoin%
\definecolor{currentfill}{rgb}{1.000000,1.000000,1.000000}%
\pgfsetfillcolor{currentfill}%
\pgfsetfillopacity{0.800000}%
\pgfsetlinewidth{1.003750pt}%
\definecolor{currentstroke}{rgb}{0.800000,0.800000,0.800000}%
\pgfsetstrokecolor{currentstroke}%
\pgfsetstrokeopacity{0.800000}%
\pgfsetdash{}{0pt}%
\pgfpathmoveto{\pgfqpoint{0.753648in}{1.945557in}}%
\pgfpathlineto{\pgfqpoint{1.758750in}{1.945557in}}%
\pgfpathquadraticcurveto{\pgfqpoint{1.783750in}{1.945557in}}{\pgfqpoint{1.783750in}{1.970557in}}%
\pgfpathlineto{\pgfqpoint{1.783750in}{2.325000in}}%
\pgfpathquadraticcurveto{\pgfqpoint{1.783750in}{2.350000in}}{\pgfqpoint{1.758750in}{2.350000in}}%
\pgfpathlineto{\pgfqpoint{0.753648in}{2.350000in}}%
\pgfpathquadraticcurveto{\pgfqpoint{0.728648in}{2.350000in}}{\pgfqpoint{0.728648in}{2.325000in}}%
\pgfpathlineto{\pgfqpoint{0.728648in}{1.970557in}}%
\pgfpathquadraticcurveto{\pgfqpoint{0.728648in}{1.945557in}}{\pgfqpoint{0.753648in}{1.945557in}}%
\pgfpathlineto{\pgfqpoint{0.753648in}{1.945557in}}%
\pgfpathclose%
\pgfusepath{stroke,fill}%
\end{pgfscope}%
\begin{pgfscope}%
\pgfsetrectcap%
\pgfsetroundjoin%
\pgfsetlinewidth{1.003750pt}%
\definecolor{currentstroke}{rgb}{0.000000,0.000000,0.000000}%
\pgfsetstrokecolor{currentstroke}%
\pgfsetdash{}{0pt}%
\pgfpathmoveto{\pgfqpoint{0.778648in}{2.248779in}}%
\pgfpathlineto{\pgfqpoint{0.903648in}{2.248779in}}%
\pgfpathlineto{\pgfqpoint{1.028648in}{2.248779in}}%
\pgfusepath{stroke}%
\end{pgfscope}%
\begin{pgfscope}%
\definecolor{textcolor}{rgb}{0.000000,0.000000,0.000000}%
\pgfsetstrokecolor{textcolor}%
\pgfsetfillcolor{textcolor}%
\pgftext[x=1.128648in,y=2.205029in,left,base]{\color{textcolor}\rmfamily\fontsize{9.000000}{10.800000}\selectfont Electrons}%
\end{pgfscope}%
\begin{pgfscope}%
\pgfsetbuttcap%
\pgfsetroundjoin%
\pgfsetlinewidth{1.003750pt}%
\definecolor{currentstroke}{rgb}{0.000000,0.000000,0.000000}%
\pgfsetstrokecolor{currentstroke}%
\pgfsetdash{{3.700000pt}{1.600000pt}}{0.000000pt}%
\pgfpathmoveto{\pgfqpoint{0.778648in}{2.065308in}}%
\pgfpathlineto{\pgfqpoint{0.903648in}{2.065308in}}%
\pgfpathlineto{\pgfqpoint{1.028648in}{2.065308in}}%
\pgfusepath{stroke}%
\end{pgfscope}%
\begin{pgfscope}%
\definecolor{textcolor}{rgb}{0.000000,0.000000,0.000000}%
\pgfsetstrokecolor{textcolor}%
\pgfsetfillcolor{textcolor}%
\pgftext[x=1.128648in,y=2.021558in,left,base]{\color{textcolor}\rmfamily\fontsize{9.000000}{10.800000}\selectfont Protons}%
\end{pgfscope}%
\end{pgfpicture}%
\makeatother%
\endgroup%

  \caption{Multiple scattering angle in the plane from~\eqref{eq:theta_ms} as a
    function of momentum for $t = 1\%~X_0$ and $t = 10\%~X_0$. The solid (dashed)
    line indicates the curve for electrons (protons).}
  \label{fig:multiple_scattering}
\end{marginfigure}

The deflection space angle being the sum in quadrature of the independent
deflections on two orthogonal planes, it is simply given by
\begin{align}
  \theta^\text{rms}_\text{space} = \sqrt{2}\theta^\text{rms}_\text{plane}.
\end{align}
Whether one or the other is more relevant depends on the problem at hand
(e.g., typically the pointing accuracy of a detector is parameterized in terms
of the space angle, but in a magnetic spectrometer the important figure is
really the deflection angle in the bending plane).


\section{Photons}%
\label{sec:inter_gammas}

High-energy photon interaction with matter involves essentially three distinct physical
processes, each dominant in a different energy range.

\begin{figure}[htbp!]
  %% Creator: Matplotlib, PGF backend
%%
%% To include the figure in your LaTeX document, write
%%   \input{<filename>.pgf}
%%
%% Make sure the required packages are loaded in your preamble
%%   \usepackage{pgf}
%%
%% Also ensure that all the required font packages are loaded; for instance,
%% the lmodern package is sometimes necessary when using math font.
%%   \usepackage{lmodern}
%%
%% Figures using additional raster images can only be included by \input if
%% they are in the same directory as the main LaTeX file. For loading figures
%% from other directories you can use the `import` package
%%   \usepackage{import}
%%
%% and then include the figures with
%%   \import{<path to file>}{<filename>.pgf}
%%
%% Matplotlib used the following preamble
%%   \usepackage{fontspec}
%%   \setmainfont{DejaVuSerif.ttf}[Path=\detokenize{/usr/share/matplotlib/mpl-data/fonts/ttf/}]
%%   \setsansfont{DejaVuSans.ttf}[Path=\detokenize{/usr/share/matplotlib/mpl-data/fonts/ttf/}]
%%   \setmonofont{DejaVuSansMono.ttf}[Path=\detokenize{/usr/share/matplotlib/mpl-data/fonts/ttf/}]
%%
\begingroup%
\makeatletter%
\begin{pgfpicture}%
\pgfpathrectangle{\pgfpointorigin}{\pgfqpoint{4.150000in}{3.500000in}}%
\pgfusepath{use as bounding box, clip}%
\begin{pgfscope}%
\pgfsetbuttcap%
\pgfsetmiterjoin%
\definecolor{currentfill}{rgb}{1.000000,1.000000,1.000000}%
\pgfsetfillcolor{currentfill}%
\pgfsetlinewidth{0.000000pt}%
\definecolor{currentstroke}{rgb}{1.000000,1.000000,1.000000}%
\pgfsetstrokecolor{currentstroke}%
\pgfsetdash{}{0pt}%
\pgfpathmoveto{\pgfqpoint{0.000000in}{0.000000in}}%
\pgfpathlineto{\pgfqpoint{4.150000in}{0.000000in}}%
\pgfpathlineto{\pgfqpoint{4.150000in}{3.500000in}}%
\pgfpathlineto{\pgfqpoint{0.000000in}{3.500000in}}%
\pgfpathlineto{\pgfqpoint{0.000000in}{0.000000in}}%
\pgfpathclose%
\pgfusepath{fill}%
\end{pgfscope}%
\begin{pgfscope}%
\pgfsetbuttcap%
\pgfsetmiterjoin%
\definecolor{currentfill}{rgb}{1.000000,1.000000,1.000000}%
\pgfsetfillcolor{currentfill}%
\pgfsetlinewidth{0.000000pt}%
\definecolor{currentstroke}{rgb}{0.000000,0.000000,0.000000}%
\pgfsetstrokecolor{currentstroke}%
\pgfsetstrokeopacity{0.000000}%
\pgfsetdash{}{0pt}%
\pgfpathmoveto{\pgfqpoint{0.726250in}{0.525000in}}%
\pgfpathlineto{\pgfqpoint{4.046250in}{0.525000in}}%
\pgfpathlineto{\pgfqpoint{4.046250in}{3.412500in}}%
\pgfpathlineto{\pgfqpoint{0.726250in}{3.412500in}}%
\pgfpathlineto{\pgfqpoint{0.726250in}{0.525000in}}%
\pgfpathclose%
\pgfusepath{fill}%
\end{pgfscope}%
\begin{pgfscope}%
\pgfpathrectangle{\pgfqpoint{0.726250in}{0.525000in}}{\pgfqpoint{3.320000in}{2.887500in}}%
\pgfusepath{clip}%
\pgfsetbuttcap%
\pgfsetroundjoin%
\pgfsetlinewidth{0.803000pt}%
\definecolor{currentstroke}{rgb}{0.752941,0.752941,0.752941}%
\pgfsetstrokecolor{currentstroke}%
\pgfsetdash{{2.960000pt}{1.280000pt}}{0.000000pt}%
\pgfpathmoveto{\pgfqpoint{0.726250in}{0.525000in}}%
\pgfpathlineto{\pgfqpoint{0.726250in}{3.412500in}}%
\pgfusepath{stroke}%
\end{pgfscope}%
\begin{pgfscope}%
\pgfsetbuttcap%
\pgfsetroundjoin%
\definecolor{currentfill}{rgb}{0.000000,0.000000,0.000000}%
\pgfsetfillcolor{currentfill}%
\pgfsetlinewidth{0.803000pt}%
\definecolor{currentstroke}{rgb}{0.000000,0.000000,0.000000}%
\pgfsetstrokecolor{currentstroke}%
\pgfsetdash{}{0pt}%
\pgfsys@defobject{currentmarker}{\pgfqpoint{0.000000in}{-0.048611in}}{\pgfqpoint{0.000000in}{0.000000in}}{%
\pgfpathmoveto{\pgfqpoint{0.000000in}{0.000000in}}%
\pgfpathlineto{\pgfqpoint{0.000000in}{-0.048611in}}%
\pgfusepath{stroke,fill}%
}%
\begin{pgfscope}%
\pgfsys@transformshift{0.726250in}{0.525000in}%
\pgfsys@useobject{currentmarker}{}%
\end{pgfscope}%
\end{pgfscope}%
\begin{pgfscope}%
\definecolor{textcolor}{rgb}{0.000000,0.000000,0.000000}%
\pgfsetstrokecolor{textcolor}%
\pgfsetfillcolor{textcolor}%
\pgftext[x=0.726250in,y=0.427778in,,top]{\color{textcolor}\rmfamily\fontsize{9.000000}{10.800000}\selectfont \(\displaystyle {10^{-3}}\)}%
\end{pgfscope}%
\begin{pgfscope}%
\pgfpathrectangle{\pgfqpoint{0.726250in}{0.525000in}}{\pgfqpoint{3.320000in}{2.887500in}}%
\pgfusepath{clip}%
\pgfsetbuttcap%
\pgfsetroundjoin%
\pgfsetlinewidth{0.803000pt}%
\definecolor{currentstroke}{rgb}{0.752941,0.752941,0.752941}%
\pgfsetstrokecolor{currentstroke}%
\pgfsetdash{{2.960000pt}{1.280000pt}}{0.000000pt}%
\pgfpathmoveto{\pgfqpoint{1.141250in}{0.525000in}}%
\pgfpathlineto{\pgfqpoint{1.141250in}{3.412500in}}%
\pgfusepath{stroke}%
\end{pgfscope}%
\begin{pgfscope}%
\pgfsetbuttcap%
\pgfsetroundjoin%
\definecolor{currentfill}{rgb}{0.000000,0.000000,0.000000}%
\pgfsetfillcolor{currentfill}%
\pgfsetlinewidth{0.803000pt}%
\definecolor{currentstroke}{rgb}{0.000000,0.000000,0.000000}%
\pgfsetstrokecolor{currentstroke}%
\pgfsetdash{}{0pt}%
\pgfsys@defobject{currentmarker}{\pgfqpoint{0.000000in}{-0.048611in}}{\pgfqpoint{0.000000in}{0.000000in}}{%
\pgfpathmoveto{\pgfqpoint{0.000000in}{0.000000in}}%
\pgfpathlineto{\pgfqpoint{0.000000in}{-0.048611in}}%
\pgfusepath{stroke,fill}%
}%
\begin{pgfscope}%
\pgfsys@transformshift{1.141250in}{0.525000in}%
\pgfsys@useobject{currentmarker}{}%
\end{pgfscope}%
\end{pgfscope}%
\begin{pgfscope}%
\definecolor{textcolor}{rgb}{0.000000,0.000000,0.000000}%
\pgfsetstrokecolor{textcolor}%
\pgfsetfillcolor{textcolor}%
\pgftext[x=1.141250in,y=0.427778in,,top]{\color{textcolor}\rmfamily\fontsize{9.000000}{10.800000}\selectfont \(\displaystyle {10^{-2}}\)}%
\end{pgfscope}%
\begin{pgfscope}%
\pgfpathrectangle{\pgfqpoint{0.726250in}{0.525000in}}{\pgfqpoint{3.320000in}{2.887500in}}%
\pgfusepath{clip}%
\pgfsetbuttcap%
\pgfsetroundjoin%
\pgfsetlinewidth{0.803000pt}%
\definecolor{currentstroke}{rgb}{0.752941,0.752941,0.752941}%
\pgfsetstrokecolor{currentstroke}%
\pgfsetdash{{2.960000pt}{1.280000pt}}{0.000000pt}%
\pgfpathmoveto{\pgfqpoint{1.556250in}{0.525000in}}%
\pgfpathlineto{\pgfqpoint{1.556250in}{3.412500in}}%
\pgfusepath{stroke}%
\end{pgfscope}%
\begin{pgfscope}%
\pgfsetbuttcap%
\pgfsetroundjoin%
\definecolor{currentfill}{rgb}{0.000000,0.000000,0.000000}%
\pgfsetfillcolor{currentfill}%
\pgfsetlinewidth{0.803000pt}%
\definecolor{currentstroke}{rgb}{0.000000,0.000000,0.000000}%
\pgfsetstrokecolor{currentstroke}%
\pgfsetdash{}{0pt}%
\pgfsys@defobject{currentmarker}{\pgfqpoint{0.000000in}{-0.048611in}}{\pgfqpoint{0.000000in}{0.000000in}}{%
\pgfpathmoveto{\pgfqpoint{0.000000in}{0.000000in}}%
\pgfpathlineto{\pgfqpoint{0.000000in}{-0.048611in}}%
\pgfusepath{stroke,fill}%
}%
\begin{pgfscope}%
\pgfsys@transformshift{1.556250in}{0.525000in}%
\pgfsys@useobject{currentmarker}{}%
\end{pgfscope}%
\end{pgfscope}%
\begin{pgfscope}%
\definecolor{textcolor}{rgb}{0.000000,0.000000,0.000000}%
\pgfsetstrokecolor{textcolor}%
\pgfsetfillcolor{textcolor}%
\pgftext[x=1.556250in,y=0.427778in,,top]{\color{textcolor}\rmfamily\fontsize{9.000000}{10.800000}\selectfont \(\displaystyle {10^{-1}}\)}%
\end{pgfscope}%
\begin{pgfscope}%
\pgfpathrectangle{\pgfqpoint{0.726250in}{0.525000in}}{\pgfqpoint{3.320000in}{2.887500in}}%
\pgfusepath{clip}%
\pgfsetbuttcap%
\pgfsetroundjoin%
\pgfsetlinewidth{0.803000pt}%
\definecolor{currentstroke}{rgb}{0.752941,0.752941,0.752941}%
\pgfsetstrokecolor{currentstroke}%
\pgfsetdash{{2.960000pt}{1.280000pt}}{0.000000pt}%
\pgfpathmoveto{\pgfqpoint{1.971250in}{0.525000in}}%
\pgfpathlineto{\pgfqpoint{1.971250in}{3.412500in}}%
\pgfusepath{stroke}%
\end{pgfscope}%
\begin{pgfscope}%
\pgfsetbuttcap%
\pgfsetroundjoin%
\definecolor{currentfill}{rgb}{0.000000,0.000000,0.000000}%
\pgfsetfillcolor{currentfill}%
\pgfsetlinewidth{0.803000pt}%
\definecolor{currentstroke}{rgb}{0.000000,0.000000,0.000000}%
\pgfsetstrokecolor{currentstroke}%
\pgfsetdash{}{0pt}%
\pgfsys@defobject{currentmarker}{\pgfqpoint{0.000000in}{-0.048611in}}{\pgfqpoint{0.000000in}{0.000000in}}{%
\pgfpathmoveto{\pgfqpoint{0.000000in}{0.000000in}}%
\pgfpathlineto{\pgfqpoint{0.000000in}{-0.048611in}}%
\pgfusepath{stroke,fill}%
}%
\begin{pgfscope}%
\pgfsys@transformshift{1.971250in}{0.525000in}%
\pgfsys@useobject{currentmarker}{}%
\end{pgfscope}%
\end{pgfscope}%
\begin{pgfscope}%
\definecolor{textcolor}{rgb}{0.000000,0.000000,0.000000}%
\pgfsetstrokecolor{textcolor}%
\pgfsetfillcolor{textcolor}%
\pgftext[x=1.971250in,y=0.427778in,,top]{\color{textcolor}\rmfamily\fontsize{9.000000}{10.800000}\selectfont \(\displaystyle {10^{0}}\)}%
\end{pgfscope}%
\begin{pgfscope}%
\pgfpathrectangle{\pgfqpoint{0.726250in}{0.525000in}}{\pgfqpoint{3.320000in}{2.887500in}}%
\pgfusepath{clip}%
\pgfsetbuttcap%
\pgfsetroundjoin%
\pgfsetlinewidth{0.803000pt}%
\definecolor{currentstroke}{rgb}{0.752941,0.752941,0.752941}%
\pgfsetstrokecolor{currentstroke}%
\pgfsetdash{{2.960000pt}{1.280000pt}}{0.000000pt}%
\pgfpathmoveto{\pgfqpoint{2.386250in}{0.525000in}}%
\pgfpathlineto{\pgfqpoint{2.386250in}{3.412500in}}%
\pgfusepath{stroke}%
\end{pgfscope}%
\begin{pgfscope}%
\pgfsetbuttcap%
\pgfsetroundjoin%
\definecolor{currentfill}{rgb}{0.000000,0.000000,0.000000}%
\pgfsetfillcolor{currentfill}%
\pgfsetlinewidth{0.803000pt}%
\definecolor{currentstroke}{rgb}{0.000000,0.000000,0.000000}%
\pgfsetstrokecolor{currentstroke}%
\pgfsetdash{}{0pt}%
\pgfsys@defobject{currentmarker}{\pgfqpoint{0.000000in}{-0.048611in}}{\pgfqpoint{0.000000in}{0.000000in}}{%
\pgfpathmoveto{\pgfqpoint{0.000000in}{0.000000in}}%
\pgfpathlineto{\pgfqpoint{0.000000in}{-0.048611in}}%
\pgfusepath{stroke,fill}%
}%
\begin{pgfscope}%
\pgfsys@transformshift{2.386250in}{0.525000in}%
\pgfsys@useobject{currentmarker}{}%
\end{pgfscope}%
\end{pgfscope}%
\begin{pgfscope}%
\definecolor{textcolor}{rgb}{0.000000,0.000000,0.000000}%
\pgfsetstrokecolor{textcolor}%
\pgfsetfillcolor{textcolor}%
\pgftext[x=2.386250in,y=0.427778in,,top]{\color{textcolor}\rmfamily\fontsize{9.000000}{10.800000}\selectfont \(\displaystyle {10^{1}}\)}%
\end{pgfscope}%
\begin{pgfscope}%
\pgfpathrectangle{\pgfqpoint{0.726250in}{0.525000in}}{\pgfqpoint{3.320000in}{2.887500in}}%
\pgfusepath{clip}%
\pgfsetbuttcap%
\pgfsetroundjoin%
\pgfsetlinewidth{0.803000pt}%
\definecolor{currentstroke}{rgb}{0.752941,0.752941,0.752941}%
\pgfsetstrokecolor{currentstroke}%
\pgfsetdash{{2.960000pt}{1.280000pt}}{0.000000pt}%
\pgfpathmoveto{\pgfqpoint{2.801250in}{0.525000in}}%
\pgfpathlineto{\pgfqpoint{2.801250in}{3.412500in}}%
\pgfusepath{stroke}%
\end{pgfscope}%
\begin{pgfscope}%
\pgfsetbuttcap%
\pgfsetroundjoin%
\definecolor{currentfill}{rgb}{0.000000,0.000000,0.000000}%
\pgfsetfillcolor{currentfill}%
\pgfsetlinewidth{0.803000pt}%
\definecolor{currentstroke}{rgb}{0.000000,0.000000,0.000000}%
\pgfsetstrokecolor{currentstroke}%
\pgfsetdash{}{0pt}%
\pgfsys@defobject{currentmarker}{\pgfqpoint{0.000000in}{-0.048611in}}{\pgfqpoint{0.000000in}{0.000000in}}{%
\pgfpathmoveto{\pgfqpoint{0.000000in}{0.000000in}}%
\pgfpathlineto{\pgfqpoint{0.000000in}{-0.048611in}}%
\pgfusepath{stroke,fill}%
}%
\begin{pgfscope}%
\pgfsys@transformshift{2.801250in}{0.525000in}%
\pgfsys@useobject{currentmarker}{}%
\end{pgfscope}%
\end{pgfscope}%
\begin{pgfscope}%
\definecolor{textcolor}{rgb}{0.000000,0.000000,0.000000}%
\pgfsetstrokecolor{textcolor}%
\pgfsetfillcolor{textcolor}%
\pgftext[x=2.801250in,y=0.427778in,,top]{\color{textcolor}\rmfamily\fontsize{9.000000}{10.800000}\selectfont \(\displaystyle {10^{2}}\)}%
\end{pgfscope}%
\begin{pgfscope}%
\pgfpathrectangle{\pgfqpoint{0.726250in}{0.525000in}}{\pgfqpoint{3.320000in}{2.887500in}}%
\pgfusepath{clip}%
\pgfsetbuttcap%
\pgfsetroundjoin%
\pgfsetlinewidth{0.803000pt}%
\definecolor{currentstroke}{rgb}{0.752941,0.752941,0.752941}%
\pgfsetstrokecolor{currentstroke}%
\pgfsetdash{{2.960000pt}{1.280000pt}}{0.000000pt}%
\pgfpathmoveto{\pgfqpoint{3.216250in}{0.525000in}}%
\pgfpathlineto{\pgfqpoint{3.216250in}{3.412500in}}%
\pgfusepath{stroke}%
\end{pgfscope}%
\begin{pgfscope}%
\pgfsetbuttcap%
\pgfsetroundjoin%
\definecolor{currentfill}{rgb}{0.000000,0.000000,0.000000}%
\pgfsetfillcolor{currentfill}%
\pgfsetlinewidth{0.803000pt}%
\definecolor{currentstroke}{rgb}{0.000000,0.000000,0.000000}%
\pgfsetstrokecolor{currentstroke}%
\pgfsetdash{}{0pt}%
\pgfsys@defobject{currentmarker}{\pgfqpoint{0.000000in}{-0.048611in}}{\pgfqpoint{0.000000in}{0.000000in}}{%
\pgfpathmoveto{\pgfqpoint{0.000000in}{0.000000in}}%
\pgfpathlineto{\pgfqpoint{0.000000in}{-0.048611in}}%
\pgfusepath{stroke,fill}%
}%
\begin{pgfscope}%
\pgfsys@transformshift{3.216250in}{0.525000in}%
\pgfsys@useobject{currentmarker}{}%
\end{pgfscope}%
\end{pgfscope}%
\begin{pgfscope}%
\definecolor{textcolor}{rgb}{0.000000,0.000000,0.000000}%
\pgfsetstrokecolor{textcolor}%
\pgfsetfillcolor{textcolor}%
\pgftext[x=3.216250in,y=0.427778in,,top]{\color{textcolor}\rmfamily\fontsize{9.000000}{10.800000}\selectfont \(\displaystyle {10^{3}}\)}%
\end{pgfscope}%
\begin{pgfscope}%
\pgfpathrectangle{\pgfqpoint{0.726250in}{0.525000in}}{\pgfqpoint{3.320000in}{2.887500in}}%
\pgfusepath{clip}%
\pgfsetbuttcap%
\pgfsetroundjoin%
\pgfsetlinewidth{0.803000pt}%
\definecolor{currentstroke}{rgb}{0.752941,0.752941,0.752941}%
\pgfsetstrokecolor{currentstroke}%
\pgfsetdash{{2.960000pt}{1.280000pt}}{0.000000pt}%
\pgfpathmoveto{\pgfqpoint{3.631250in}{0.525000in}}%
\pgfpathlineto{\pgfqpoint{3.631250in}{3.412500in}}%
\pgfusepath{stroke}%
\end{pgfscope}%
\begin{pgfscope}%
\pgfsetbuttcap%
\pgfsetroundjoin%
\definecolor{currentfill}{rgb}{0.000000,0.000000,0.000000}%
\pgfsetfillcolor{currentfill}%
\pgfsetlinewidth{0.803000pt}%
\definecolor{currentstroke}{rgb}{0.000000,0.000000,0.000000}%
\pgfsetstrokecolor{currentstroke}%
\pgfsetdash{}{0pt}%
\pgfsys@defobject{currentmarker}{\pgfqpoint{0.000000in}{-0.048611in}}{\pgfqpoint{0.000000in}{0.000000in}}{%
\pgfpathmoveto{\pgfqpoint{0.000000in}{0.000000in}}%
\pgfpathlineto{\pgfqpoint{0.000000in}{-0.048611in}}%
\pgfusepath{stroke,fill}%
}%
\begin{pgfscope}%
\pgfsys@transformshift{3.631250in}{0.525000in}%
\pgfsys@useobject{currentmarker}{}%
\end{pgfscope}%
\end{pgfscope}%
\begin{pgfscope}%
\definecolor{textcolor}{rgb}{0.000000,0.000000,0.000000}%
\pgfsetstrokecolor{textcolor}%
\pgfsetfillcolor{textcolor}%
\pgftext[x=3.631250in,y=0.427778in,,top]{\color{textcolor}\rmfamily\fontsize{9.000000}{10.800000}\selectfont \(\displaystyle {10^{4}}\)}%
\end{pgfscope}%
\begin{pgfscope}%
\pgfpathrectangle{\pgfqpoint{0.726250in}{0.525000in}}{\pgfqpoint{3.320000in}{2.887500in}}%
\pgfusepath{clip}%
\pgfsetbuttcap%
\pgfsetroundjoin%
\pgfsetlinewidth{0.803000pt}%
\definecolor{currentstroke}{rgb}{0.752941,0.752941,0.752941}%
\pgfsetstrokecolor{currentstroke}%
\pgfsetdash{{2.960000pt}{1.280000pt}}{0.000000pt}%
\pgfpathmoveto{\pgfqpoint{4.046250in}{0.525000in}}%
\pgfpathlineto{\pgfqpoint{4.046250in}{3.412500in}}%
\pgfusepath{stroke}%
\end{pgfscope}%
\begin{pgfscope}%
\pgfsetbuttcap%
\pgfsetroundjoin%
\definecolor{currentfill}{rgb}{0.000000,0.000000,0.000000}%
\pgfsetfillcolor{currentfill}%
\pgfsetlinewidth{0.803000pt}%
\definecolor{currentstroke}{rgb}{0.000000,0.000000,0.000000}%
\pgfsetstrokecolor{currentstroke}%
\pgfsetdash{}{0pt}%
\pgfsys@defobject{currentmarker}{\pgfqpoint{0.000000in}{-0.048611in}}{\pgfqpoint{0.000000in}{0.000000in}}{%
\pgfpathmoveto{\pgfqpoint{0.000000in}{0.000000in}}%
\pgfpathlineto{\pgfqpoint{0.000000in}{-0.048611in}}%
\pgfusepath{stroke,fill}%
}%
\begin{pgfscope}%
\pgfsys@transformshift{4.046250in}{0.525000in}%
\pgfsys@useobject{currentmarker}{}%
\end{pgfscope}%
\end{pgfscope}%
\begin{pgfscope}%
\definecolor{textcolor}{rgb}{0.000000,0.000000,0.000000}%
\pgfsetstrokecolor{textcolor}%
\pgfsetfillcolor{textcolor}%
\pgftext[x=4.046250in,y=0.427778in,,top]{\color{textcolor}\rmfamily\fontsize{9.000000}{10.800000}\selectfont \(\displaystyle {10^{5}}\)}%
\end{pgfscope}%
\begin{pgfscope}%
\definecolor{textcolor}{rgb}{0.000000,0.000000,0.000000}%
\pgfsetstrokecolor{textcolor}%
\pgfsetfillcolor{textcolor}%
\pgftext[x=2.386250in,y=0.251251in,,top]{\color{textcolor}\rmfamily\fontsize{9.000000}{10.800000}\selectfont Energy [MeV]}%
\end{pgfscope}%
\begin{pgfscope}%
\pgfpathrectangle{\pgfqpoint{0.726250in}{0.525000in}}{\pgfqpoint{3.320000in}{2.887500in}}%
\pgfusepath{clip}%
\pgfsetbuttcap%
\pgfsetroundjoin%
\pgfsetlinewidth{0.803000pt}%
\definecolor{currentstroke}{rgb}{0.752941,0.752941,0.752941}%
\pgfsetstrokecolor{currentstroke}%
\pgfsetdash{{2.960000pt}{1.280000pt}}{0.000000pt}%
\pgfpathmoveto{\pgfqpoint{0.726250in}{0.904444in}}%
\pgfpathlineto{\pgfqpoint{4.046250in}{0.904444in}}%
\pgfusepath{stroke}%
\end{pgfscope}%
\begin{pgfscope}%
\pgfsetbuttcap%
\pgfsetroundjoin%
\definecolor{currentfill}{rgb}{0.000000,0.000000,0.000000}%
\pgfsetfillcolor{currentfill}%
\pgfsetlinewidth{0.803000pt}%
\definecolor{currentstroke}{rgb}{0.000000,0.000000,0.000000}%
\pgfsetstrokecolor{currentstroke}%
\pgfsetdash{}{0pt}%
\pgfsys@defobject{currentmarker}{\pgfqpoint{-0.048611in}{0.000000in}}{\pgfqpoint{-0.000000in}{0.000000in}}{%
\pgfpathmoveto{\pgfqpoint{-0.000000in}{0.000000in}}%
\pgfpathlineto{\pgfqpoint{-0.048611in}{0.000000in}}%
\pgfusepath{stroke,fill}%
}%
\begin{pgfscope}%
\pgfsys@transformshift{0.726250in}{0.904444in}%
\pgfsys@useobject{currentmarker}{}%
\end{pgfscope}%
\end{pgfscope}%
\begin{pgfscope}%
\definecolor{textcolor}{rgb}{0.000000,0.000000,0.000000}%
\pgfsetstrokecolor{textcolor}%
\pgfsetfillcolor{textcolor}%
\pgftext[x=0.362441in, y=0.856958in, left, base]{\color{textcolor}\rmfamily\fontsize{9.000000}{10.800000}\selectfont \(\displaystyle {10^{-7}}\)}%
\end{pgfscope}%
\begin{pgfscope}%
\pgfpathrectangle{\pgfqpoint{0.726250in}{0.525000in}}{\pgfqpoint{3.320000in}{2.887500in}}%
\pgfusepath{clip}%
\pgfsetbuttcap%
\pgfsetroundjoin%
\pgfsetlinewidth{0.803000pt}%
\definecolor{currentstroke}{rgb}{0.752941,0.752941,0.752941}%
\pgfsetstrokecolor{currentstroke}%
\pgfsetdash{{2.960000pt}{1.280000pt}}{0.000000pt}%
\pgfpathmoveto{\pgfqpoint{0.726250in}{1.296907in}}%
\pgfpathlineto{\pgfqpoint{4.046250in}{1.296907in}}%
\pgfusepath{stroke}%
\end{pgfscope}%
\begin{pgfscope}%
\pgfsetbuttcap%
\pgfsetroundjoin%
\definecolor{currentfill}{rgb}{0.000000,0.000000,0.000000}%
\pgfsetfillcolor{currentfill}%
\pgfsetlinewidth{0.803000pt}%
\definecolor{currentstroke}{rgb}{0.000000,0.000000,0.000000}%
\pgfsetstrokecolor{currentstroke}%
\pgfsetdash{}{0pt}%
\pgfsys@defobject{currentmarker}{\pgfqpoint{-0.048611in}{0.000000in}}{\pgfqpoint{-0.000000in}{0.000000in}}{%
\pgfpathmoveto{\pgfqpoint{-0.000000in}{0.000000in}}%
\pgfpathlineto{\pgfqpoint{-0.048611in}{0.000000in}}%
\pgfusepath{stroke,fill}%
}%
\begin{pgfscope}%
\pgfsys@transformshift{0.726250in}{1.296907in}%
\pgfsys@useobject{currentmarker}{}%
\end{pgfscope}%
\end{pgfscope}%
\begin{pgfscope}%
\definecolor{textcolor}{rgb}{0.000000,0.000000,0.000000}%
\pgfsetstrokecolor{textcolor}%
\pgfsetfillcolor{textcolor}%
\pgftext[x=0.362441in, y=1.249421in, left, base]{\color{textcolor}\rmfamily\fontsize{9.000000}{10.800000}\selectfont \(\displaystyle {10^{-5}}\)}%
\end{pgfscope}%
\begin{pgfscope}%
\pgfpathrectangle{\pgfqpoint{0.726250in}{0.525000in}}{\pgfqpoint{3.320000in}{2.887500in}}%
\pgfusepath{clip}%
\pgfsetbuttcap%
\pgfsetroundjoin%
\pgfsetlinewidth{0.803000pt}%
\definecolor{currentstroke}{rgb}{0.752941,0.752941,0.752941}%
\pgfsetstrokecolor{currentstroke}%
\pgfsetdash{{2.960000pt}{1.280000pt}}{0.000000pt}%
\pgfpathmoveto{\pgfqpoint{0.726250in}{1.689369in}}%
\pgfpathlineto{\pgfqpoint{4.046250in}{1.689369in}}%
\pgfusepath{stroke}%
\end{pgfscope}%
\begin{pgfscope}%
\pgfsetbuttcap%
\pgfsetroundjoin%
\definecolor{currentfill}{rgb}{0.000000,0.000000,0.000000}%
\pgfsetfillcolor{currentfill}%
\pgfsetlinewidth{0.803000pt}%
\definecolor{currentstroke}{rgb}{0.000000,0.000000,0.000000}%
\pgfsetstrokecolor{currentstroke}%
\pgfsetdash{}{0pt}%
\pgfsys@defobject{currentmarker}{\pgfqpoint{-0.048611in}{0.000000in}}{\pgfqpoint{-0.000000in}{0.000000in}}{%
\pgfpathmoveto{\pgfqpoint{-0.000000in}{0.000000in}}%
\pgfpathlineto{\pgfqpoint{-0.048611in}{0.000000in}}%
\pgfusepath{stroke,fill}%
}%
\begin{pgfscope}%
\pgfsys@transformshift{0.726250in}{1.689369in}%
\pgfsys@useobject{currentmarker}{}%
\end{pgfscope}%
\end{pgfscope}%
\begin{pgfscope}%
\definecolor{textcolor}{rgb}{0.000000,0.000000,0.000000}%
\pgfsetstrokecolor{textcolor}%
\pgfsetfillcolor{textcolor}%
\pgftext[x=0.362441in, y=1.641884in, left, base]{\color{textcolor}\rmfamily\fontsize{9.000000}{10.800000}\selectfont \(\displaystyle {10^{-3}}\)}%
\end{pgfscope}%
\begin{pgfscope}%
\pgfpathrectangle{\pgfqpoint{0.726250in}{0.525000in}}{\pgfqpoint{3.320000in}{2.887500in}}%
\pgfusepath{clip}%
\pgfsetbuttcap%
\pgfsetroundjoin%
\pgfsetlinewidth{0.803000pt}%
\definecolor{currentstroke}{rgb}{0.752941,0.752941,0.752941}%
\pgfsetstrokecolor{currentstroke}%
\pgfsetdash{{2.960000pt}{1.280000pt}}{0.000000pt}%
\pgfpathmoveto{\pgfqpoint{0.726250in}{2.081832in}}%
\pgfpathlineto{\pgfqpoint{4.046250in}{2.081832in}}%
\pgfusepath{stroke}%
\end{pgfscope}%
\begin{pgfscope}%
\pgfsetbuttcap%
\pgfsetroundjoin%
\definecolor{currentfill}{rgb}{0.000000,0.000000,0.000000}%
\pgfsetfillcolor{currentfill}%
\pgfsetlinewidth{0.803000pt}%
\definecolor{currentstroke}{rgb}{0.000000,0.000000,0.000000}%
\pgfsetstrokecolor{currentstroke}%
\pgfsetdash{}{0pt}%
\pgfsys@defobject{currentmarker}{\pgfqpoint{-0.048611in}{0.000000in}}{\pgfqpoint{-0.000000in}{0.000000in}}{%
\pgfpathmoveto{\pgfqpoint{-0.000000in}{0.000000in}}%
\pgfpathlineto{\pgfqpoint{-0.048611in}{0.000000in}}%
\pgfusepath{stroke,fill}%
}%
\begin{pgfscope}%
\pgfsys@transformshift{0.726250in}{2.081832in}%
\pgfsys@useobject{currentmarker}{}%
\end{pgfscope}%
\end{pgfscope}%
\begin{pgfscope}%
\definecolor{textcolor}{rgb}{0.000000,0.000000,0.000000}%
\pgfsetstrokecolor{textcolor}%
\pgfsetfillcolor{textcolor}%
\pgftext[x=0.362441in, y=2.034347in, left, base]{\color{textcolor}\rmfamily\fontsize{9.000000}{10.800000}\selectfont \(\displaystyle {10^{-1}}\)}%
\end{pgfscope}%
\begin{pgfscope}%
\pgfpathrectangle{\pgfqpoint{0.726250in}{0.525000in}}{\pgfqpoint{3.320000in}{2.887500in}}%
\pgfusepath{clip}%
\pgfsetbuttcap%
\pgfsetroundjoin%
\pgfsetlinewidth{0.803000pt}%
\definecolor{currentstroke}{rgb}{0.752941,0.752941,0.752941}%
\pgfsetstrokecolor{currentstroke}%
\pgfsetdash{{2.960000pt}{1.280000pt}}{0.000000pt}%
\pgfpathmoveto{\pgfqpoint{0.726250in}{2.474295in}}%
\pgfpathlineto{\pgfqpoint{4.046250in}{2.474295in}}%
\pgfusepath{stroke}%
\end{pgfscope}%
\begin{pgfscope}%
\pgfsetbuttcap%
\pgfsetroundjoin%
\definecolor{currentfill}{rgb}{0.000000,0.000000,0.000000}%
\pgfsetfillcolor{currentfill}%
\pgfsetlinewidth{0.803000pt}%
\definecolor{currentstroke}{rgb}{0.000000,0.000000,0.000000}%
\pgfsetstrokecolor{currentstroke}%
\pgfsetdash{}{0pt}%
\pgfsys@defobject{currentmarker}{\pgfqpoint{-0.048611in}{0.000000in}}{\pgfqpoint{-0.000000in}{0.000000in}}{%
\pgfpathmoveto{\pgfqpoint{-0.000000in}{0.000000in}}%
\pgfpathlineto{\pgfqpoint{-0.048611in}{0.000000in}}%
\pgfusepath{stroke,fill}%
}%
\begin{pgfscope}%
\pgfsys@transformshift{0.726250in}{2.474295in}%
\pgfsys@useobject{currentmarker}{}%
\end{pgfscope}%
\end{pgfscope}%
\begin{pgfscope}%
\definecolor{textcolor}{rgb}{0.000000,0.000000,0.000000}%
\pgfsetstrokecolor{textcolor}%
\pgfsetfillcolor{textcolor}%
\pgftext[x=0.442687in, y=2.426810in, left, base]{\color{textcolor}\rmfamily\fontsize{9.000000}{10.800000}\selectfont \(\displaystyle {10^{1}}\)}%
\end{pgfscope}%
\begin{pgfscope}%
\pgfpathrectangle{\pgfqpoint{0.726250in}{0.525000in}}{\pgfqpoint{3.320000in}{2.887500in}}%
\pgfusepath{clip}%
\pgfsetbuttcap%
\pgfsetroundjoin%
\pgfsetlinewidth{0.803000pt}%
\definecolor{currentstroke}{rgb}{0.752941,0.752941,0.752941}%
\pgfsetstrokecolor{currentstroke}%
\pgfsetdash{{2.960000pt}{1.280000pt}}{0.000000pt}%
\pgfpathmoveto{\pgfqpoint{0.726250in}{2.866758in}}%
\pgfpathlineto{\pgfqpoint{4.046250in}{2.866758in}}%
\pgfusepath{stroke}%
\end{pgfscope}%
\begin{pgfscope}%
\pgfsetbuttcap%
\pgfsetroundjoin%
\definecolor{currentfill}{rgb}{0.000000,0.000000,0.000000}%
\pgfsetfillcolor{currentfill}%
\pgfsetlinewidth{0.803000pt}%
\definecolor{currentstroke}{rgb}{0.000000,0.000000,0.000000}%
\pgfsetstrokecolor{currentstroke}%
\pgfsetdash{}{0pt}%
\pgfsys@defobject{currentmarker}{\pgfqpoint{-0.048611in}{0.000000in}}{\pgfqpoint{-0.000000in}{0.000000in}}{%
\pgfpathmoveto{\pgfqpoint{-0.000000in}{0.000000in}}%
\pgfpathlineto{\pgfqpoint{-0.048611in}{0.000000in}}%
\pgfusepath{stroke,fill}%
}%
\begin{pgfscope}%
\pgfsys@transformshift{0.726250in}{2.866758in}%
\pgfsys@useobject{currentmarker}{}%
\end{pgfscope}%
\end{pgfscope}%
\begin{pgfscope}%
\definecolor{textcolor}{rgb}{0.000000,0.000000,0.000000}%
\pgfsetstrokecolor{textcolor}%
\pgfsetfillcolor{textcolor}%
\pgftext[x=0.442687in, y=2.819273in, left, base]{\color{textcolor}\rmfamily\fontsize{9.000000}{10.800000}\selectfont \(\displaystyle {10^{3}}\)}%
\end{pgfscope}%
\begin{pgfscope}%
\pgfpathrectangle{\pgfqpoint{0.726250in}{0.525000in}}{\pgfqpoint{3.320000in}{2.887500in}}%
\pgfusepath{clip}%
\pgfsetbuttcap%
\pgfsetroundjoin%
\pgfsetlinewidth{0.803000pt}%
\definecolor{currentstroke}{rgb}{0.752941,0.752941,0.752941}%
\pgfsetstrokecolor{currentstroke}%
\pgfsetdash{{2.960000pt}{1.280000pt}}{0.000000pt}%
\pgfpathmoveto{\pgfqpoint{0.726250in}{3.259221in}}%
\pgfpathlineto{\pgfqpoint{4.046250in}{3.259221in}}%
\pgfusepath{stroke}%
\end{pgfscope}%
\begin{pgfscope}%
\pgfsetbuttcap%
\pgfsetroundjoin%
\definecolor{currentfill}{rgb}{0.000000,0.000000,0.000000}%
\pgfsetfillcolor{currentfill}%
\pgfsetlinewidth{0.803000pt}%
\definecolor{currentstroke}{rgb}{0.000000,0.000000,0.000000}%
\pgfsetstrokecolor{currentstroke}%
\pgfsetdash{}{0pt}%
\pgfsys@defobject{currentmarker}{\pgfqpoint{-0.048611in}{0.000000in}}{\pgfqpoint{-0.000000in}{0.000000in}}{%
\pgfpathmoveto{\pgfqpoint{-0.000000in}{0.000000in}}%
\pgfpathlineto{\pgfqpoint{-0.048611in}{0.000000in}}%
\pgfusepath{stroke,fill}%
}%
\begin{pgfscope}%
\pgfsys@transformshift{0.726250in}{3.259221in}%
\pgfsys@useobject{currentmarker}{}%
\end{pgfscope}%
\end{pgfscope}%
\begin{pgfscope}%
\definecolor{textcolor}{rgb}{0.000000,0.000000,0.000000}%
\pgfsetstrokecolor{textcolor}%
\pgfsetfillcolor{textcolor}%
\pgftext[x=0.442687in, y=3.211736in, left, base]{\color{textcolor}\rmfamily\fontsize{9.000000}{10.800000}\selectfont \(\displaystyle {10^{5}}\)}%
\end{pgfscope}%
\begin{pgfscope}%
\definecolor{textcolor}{rgb}{0.000000,0.000000,0.000000}%
\pgfsetstrokecolor{textcolor}%
\pgfsetfillcolor{textcolor}%
\pgftext[x=0.306885in,y=1.968750in,,bottom,rotate=90.000000]{\color{textcolor}\rmfamily\fontsize{9.000000}{10.800000}\selectfont \(\displaystyle \sigma\) [barn/atom]}%
\end{pgfscope}%
\begin{pgfscope}%
\pgfpathrectangle{\pgfqpoint{0.726250in}{0.525000in}}{\pgfqpoint{3.320000in}{2.887500in}}%
\pgfusepath{clip}%
\pgfsetrectcap%
\pgfsetroundjoin%
\pgfsetlinewidth{1.003750pt}%
\definecolor{currentstroke}{rgb}{0.000000,0.000000,0.000000}%
\pgfsetstrokecolor{currentstroke}%
\pgfsetdash{}{0pt}%
\pgfpathmoveto{\pgfqpoint{0.726250in}{3.232655in}}%
\pgfpathlineto{\pgfqpoint{0.799328in}{3.141002in}}%
\pgfpathlineto{\pgfqpoint{0.851177in}{3.281250in}}%
\pgfpathlineto{\pgfqpoint{0.924255in}{3.192356in}}%
\pgfpathlineto{\pgfqpoint{0.976105in}{3.126698in}}%
\pgfpathlineto{\pgfqpoint{1.016323in}{3.074378in}}%
\pgfpathlineto{\pgfqpoint{1.049183in}{3.030806in}}%
\pgfpathlineto{\pgfqpoint{1.101032in}{2.960864in}}%
\pgfpathlineto{\pgfqpoint{1.141250in}{2.905743in}}%
\pgfpathlineto{\pgfqpoint{1.214328in}{2.804567in}}%
\pgfpathlineto{\pgfqpoint{1.266177in}{2.733005in}}%
\pgfpathlineto{\pgfqpoint{1.339255in}{2.636377in}}%
\pgfpathlineto{\pgfqpoint{1.391105in}{2.575247in}}%
\pgfpathlineto{\pgfqpoint{1.431323in}{2.535263in}}%
\pgfpathlineto{\pgfqpoint{1.464183in}{2.508588in}}%
\pgfpathlineto{\pgfqpoint{1.516032in}{2.477556in}}%
\pgfpathlineto{\pgfqpoint{1.556250in}{2.461041in}}%
\pgfpathlineto{\pgfqpoint{1.629328in}{2.440859in}}%
\pgfpathlineto{\pgfqpoint{1.681177in}{2.430021in}}%
\pgfpathlineto{\pgfqpoint{1.754255in}{2.415983in}}%
\pgfpathlineto{\pgfqpoint{1.806105in}{2.405933in}}%
\pgfpathlineto{\pgfqpoint{1.846323in}{2.397891in}}%
\pgfpathlineto{\pgfqpoint{1.879183in}{2.391095in}}%
\pgfpathlineto{\pgfqpoint{1.931032in}{2.379889in}}%
\pgfpathlineto{\pgfqpoint{1.971250in}{2.370734in}}%
\pgfpathlineto{\pgfqpoint{1.975172in}{2.369826in}}%
\pgfpathlineto{\pgfqpoint{2.011468in}{2.361206in}}%
\pgfpathlineto{\pgfqpoint{2.044328in}{2.353281in}}%
\pgfpathlineto{\pgfqpoint{2.096177in}{2.340869in}}%
\pgfpathlineto{\pgfqpoint{2.100100in}{2.339952in}}%
\pgfpathlineto{\pgfqpoint{2.169255in}{2.324058in}}%
\pgfpathlineto{\pgfqpoint{2.221105in}{2.313251in}}%
\pgfpathlineto{\pgfqpoint{2.261323in}{2.305756in}}%
\pgfpathlineto{\pgfqpoint{2.294183in}{2.300435in}}%
\pgfpathlineto{\pgfqpoint{2.321966in}{2.296538in}}%
\pgfpathlineto{\pgfqpoint{2.346032in}{2.293637in}}%
\pgfpathlineto{\pgfqpoint{2.367261in}{2.291470in}}%
\pgfpathlineto{\pgfqpoint{2.386250in}{2.289827in}}%
\pgfpathlineto{\pgfqpoint{2.403428in}{2.288559in}}%
\pgfpathlineto{\pgfqpoint{2.419110in}{2.287599in}}%
\pgfpathlineto{\pgfqpoint{2.433536in}{2.286867in}}%
\pgfpathlineto{\pgfqpoint{2.446893in}{2.286324in}}%
\pgfpathlineto{\pgfqpoint{2.459328in}{2.285948in}}%
\pgfpathlineto{\pgfqpoint{2.470960in}{2.285683in}}%
\pgfpathlineto{\pgfqpoint{2.492188in}{2.285428in}}%
\pgfpathlineto{\pgfqpoint{2.511177in}{2.285427in}}%
\pgfpathlineto{\pgfqpoint{2.528355in}{2.285606in}}%
\pgfpathlineto{\pgfqpoint{2.544038in}{2.285901in}}%
\pgfpathlineto{\pgfqpoint{2.558464in}{2.286267in}}%
\pgfpathlineto{\pgfqpoint{2.571821in}{2.286682in}}%
\pgfpathlineto{\pgfqpoint{2.584255in}{2.287132in}}%
\pgfpathlineto{\pgfqpoint{2.636105in}{2.289582in}}%
\pgfpathlineto{\pgfqpoint{2.676323in}{2.291895in}}%
\pgfpathlineto{\pgfqpoint{2.709183in}{2.293916in}}%
\pgfpathlineto{\pgfqpoint{2.761032in}{2.297232in}}%
\pgfpathlineto{\pgfqpoint{2.801250in}{2.299700in}}%
\pgfpathlineto{\pgfqpoint{2.874328in}{2.303952in}}%
\pgfpathlineto{\pgfqpoint{2.926177in}{2.306565in}}%
\pgfpathlineto{\pgfqpoint{2.999255in}{2.309783in}}%
\pgfpathlineto{\pgfqpoint{3.051105in}{2.311682in}}%
\pgfpathlineto{\pgfqpoint{3.091323in}{2.312943in}}%
\pgfpathlineto{\pgfqpoint{3.124183in}{2.313857in}}%
\pgfpathlineto{\pgfqpoint{3.176032in}{2.315125in}}%
\pgfpathlineto{\pgfqpoint{3.216250in}{2.315937in}}%
\pgfpathlineto{\pgfqpoint{3.289328in}{2.317107in}}%
\pgfpathlineto{\pgfqpoint{3.341177in}{2.317720in}}%
\pgfpathlineto{\pgfqpoint{3.414255in}{2.318440in}}%
\pgfpathlineto{\pgfqpoint{3.466105in}{2.318858in}}%
\pgfpathlineto{\pgfqpoint{3.506323in}{2.319066in}}%
\pgfpathlineto{\pgfqpoint{3.539183in}{2.319277in}}%
\pgfpathlineto{\pgfqpoint{3.591032in}{2.319484in}}%
\pgfpathlineto{\pgfqpoint{3.631250in}{2.319643in}}%
\pgfpathlineto{\pgfqpoint{3.704328in}{2.319803in}}%
\pgfpathlineto{\pgfqpoint{3.756177in}{2.319905in}}%
\pgfpathlineto{\pgfqpoint{3.829255in}{2.320065in}}%
\pgfpathlineto{\pgfqpoint{3.881105in}{2.320123in}}%
\pgfpathlineto{\pgfqpoint{3.921323in}{2.320123in}}%
\pgfpathlineto{\pgfqpoint{3.954183in}{2.320176in}}%
\pgfpathlineto{\pgfqpoint{4.006032in}{2.320177in}}%
\pgfpathlineto{\pgfqpoint{4.046250in}{2.320231in}}%
\pgfusepath{stroke}%
\end{pgfscope}%
\begin{pgfscope}%
\pgfpathrectangle{\pgfqpoint{0.726250in}{0.525000in}}{\pgfqpoint{3.320000in}{2.887500in}}%
\pgfusepath{clip}%
\pgfsetbuttcap%
\pgfsetroundjoin%
\pgfsetlinewidth{1.003750pt}%
\definecolor{currentstroke}{rgb}{0.000000,0.000000,0.000000}%
\pgfsetstrokecolor{currentstroke}%
\pgfsetdash{{3.700000pt}{1.600000pt}}{0.000000pt}%
\pgfpathmoveto{\pgfqpoint{0.726250in}{3.232517in}}%
\pgfpathlineto{\pgfqpoint{0.799328in}{3.140634in}}%
\pgfpathlineto{\pgfqpoint{0.851177in}{3.281186in}}%
\pgfpathlineto{\pgfqpoint{0.924255in}{3.192206in}}%
\pgfpathlineto{\pgfqpoint{0.976105in}{3.126422in}}%
\pgfpathlineto{\pgfqpoint{1.016323in}{3.073932in}}%
\pgfpathlineto{\pgfqpoint{1.049183in}{3.030148in}}%
\pgfpathlineto{\pgfqpoint{1.101032in}{2.959671in}}%
\pgfpathlineto{\pgfqpoint{1.141250in}{2.903887in}}%
\pgfpathlineto{\pgfqpoint{1.214328in}{2.800451in}}%
\pgfpathlineto{\pgfqpoint{1.266177in}{2.725540in}}%
\pgfpathlineto{\pgfqpoint{1.339255in}{2.618219in}}%
\pgfpathlineto{\pgfqpoint{1.391105in}{2.540945in}}%
\pgfpathlineto{\pgfqpoint{1.431323in}{2.480538in}}%
\pgfpathlineto{\pgfqpoint{1.464183in}{2.430889in}}%
\pgfpathlineto{\pgfqpoint{1.516032in}{2.352246in}}%
\pgfpathlineto{\pgfqpoint{1.556250in}{2.291079in}}%
\pgfpathlineto{\pgfqpoint{1.629328in}{2.180263in}}%
\pgfpathlineto{\pgfqpoint{1.681177in}{2.102604in}}%
\pgfpathlineto{\pgfqpoint{1.754255in}{1.996545in}}%
\pgfpathlineto{\pgfqpoint{1.806105in}{1.925121in}}%
\pgfpathlineto{\pgfqpoint{1.846323in}{1.872985in}}%
\pgfpathlineto{\pgfqpoint{1.879183in}{1.832978in}}%
\pgfpathlineto{\pgfqpoint{1.931032in}{1.774958in}}%
\pgfpathlineto{\pgfqpoint{1.971250in}{1.734440in}}%
\pgfpathlineto{\pgfqpoint{1.975172in}{1.729477in}}%
\pgfpathlineto{\pgfqpoint{2.011468in}{1.696322in}}%
\pgfpathlineto{\pgfqpoint{2.044328in}{1.668772in}}%
\pgfpathlineto{\pgfqpoint{2.096177in}{1.628559in}}%
\pgfpathlineto{\pgfqpoint{2.100100in}{1.625675in}}%
\pgfpathlineto{\pgfqpoint{2.169255in}{1.577943in}}%
\pgfpathlineto{\pgfqpoint{2.221105in}{1.545242in}}%
\pgfpathlineto{\pgfqpoint{2.261323in}{1.521263in}}%
\pgfpathlineto{\pgfqpoint{2.294183in}{1.502338in}}%
\pgfpathlineto{\pgfqpoint{2.321966in}{1.486797in}}%
\pgfpathlineto{\pgfqpoint{2.346032in}{1.473587in}}%
\pgfpathlineto{\pgfqpoint{2.367261in}{1.462106in}}%
\pgfpathlineto{\pgfqpoint{2.386250in}{1.451972in}}%
\pgfpathlineto{\pgfqpoint{2.403428in}{1.442899in}}%
\pgfpathlineto{\pgfqpoint{2.419110in}{1.434695in}}%
\pgfpathlineto{\pgfqpoint{2.433536in}{1.427182in}}%
\pgfpathlineto{\pgfqpoint{2.446893in}{1.420296in}}%
\pgfpathlineto{\pgfqpoint{2.459328in}{1.413913in}}%
\pgfpathlineto{\pgfqpoint{2.470960in}{1.407967in}}%
\pgfpathlineto{\pgfqpoint{2.492188in}{1.397170in}}%
\pgfpathlineto{\pgfqpoint{2.511177in}{1.387614in}}%
\pgfpathlineto{\pgfqpoint{2.528355in}{1.378990in}}%
\pgfpathlineto{\pgfqpoint{2.544038in}{1.371160in}}%
\pgfpathlineto{\pgfqpoint{2.558464in}{1.363984in}}%
\pgfpathlineto{\pgfqpoint{2.571821in}{1.357373in}}%
\pgfpathlineto{\pgfqpoint{2.584255in}{1.351247in}}%
\pgfpathlineto{\pgfqpoint{2.636105in}{1.325825in}}%
\pgfpathlineto{\pgfqpoint{2.676323in}{1.306260in}}%
\pgfpathlineto{\pgfqpoint{2.709183in}{1.290336in}}%
\pgfpathlineto{\pgfqpoint{2.761032in}{1.265358in}}%
\pgfpathlineto{\pgfqpoint{2.801250in}{1.246066in}}%
\pgfpathlineto{\pgfqpoint{2.874328in}{1.211131in}}%
\pgfpathlineto{\pgfqpoint{2.926177in}{1.186420in}}%
\pgfpathlineto{\pgfqpoint{2.999255in}{1.151709in}}%
\pgfpathlineto{\pgfqpoint{3.051105in}{1.127067in}}%
\pgfpathlineto{\pgfqpoint{3.091323in}{1.108019in}}%
\pgfpathlineto{\pgfqpoint{3.124183in}{1.092431in}}%
\pgfpathlineto{\pgfqpoint{3.176032in}{1.067871in}}%
\pgfpathlineto{\pgfqpoint{3.216250in}{1.048823in}}%
\pgfpathlineto{\pgfqpoint{3.289328in}{1.014221in}}%
\pgfpathlineto{\pgfqpoint{3.341177in}{0.989688in}}%
\pgfpathlineto{\pgfqpoint{3.414255in}{0.955102in}}%
\pgfpathlineto{\pgfqpoint{3.466105in}{0.930585in}}%
\pgfpathlineto{\pgfqpoint{3.506323in}{0.911553in}}%
\pgfpathlineto{\pgfqpoint{3.539183in}{0.896031in}}%
\pgfpathlineto{\pgfqpoint{3.591032in}{0.871514in}}%
\pgfpathlineto{\pgfqpoint{3.631250in}{0.852497in}}%
\pgfpathlineto{\pgfqpoint{3.704328in}{0.817942in}}%
\pgfpathlineto{\pgfqpoint{3.756177in}{0.793426in}}%
\pgfpathlineto{\pgfqpoint{3.829255in}{0.758871in}}%
\pgfpathlineto{\pgfqpoint{3.881105in}{0.734354in}}%
\pgfpathlineto{\pgfqpoint{3.921323in}{0.715322in}}%
\pgfpathlineto{\pgfqpoint{3.954183in}{0.699781in}}%
\pgfpathlineto{\pgfqpoint{4.006032in}{0.675270in}}%
\pgfpathlineto{\pgfqpoint{4.046250in}{0.656250in}}%
\pgfusepath{stroke}%
\end{pgfscope}%
\begin{pgfscope}%
\pgfpathrectangle{\pgfqpoint{0.726250in}{0.525000in}}{\pgfqpoint{3.320000in}{2.887500in}}%
\pgfusepath{clip}%
\pgfsetbuttcap%
\pgfsetroundjoin%
\pgfsetlinewidth{1.003750pt}%
\definecolor{currentstroke}{rgb}{0.000000,0.000000,0.000000}%
\pgfsetstrokecolor{currentstroke}%
\pgfsetdash{{3.700000pt}{1.600000pt}}{0.000000pt}%
\pgfpathmoveto{\pgfqpoint{0.726250in}{2.236510in}}%
\pgfpathlineto{\pgfqpoint{0.799328in}{2.287417in}}%
\pgfpathlineto{\pgfqpoint{0.851177in}{2.317047in}}%
\pgfpathlineto{\pgfqpoint{0.924255in}{2.349563in}}%
\pgfpathlineto{\pgfqpoint{0.976105in}{2.367647in}}%
\pgfpathlineto{\pgfqpoint{1.016323in}{2.380225in}}%
\pgfpathlineto{\pgfqpoint{1.049183in}{2.390091in}}%
\pgfpathlineto{\pgfqpoint{1.101032in}{2.404985in}}%
\pgfpathlineto{\pgfqpoint{1.141250in}{2.415564in}}%
\pgfpathlineto{\pgfqpoint{1.214328in}{2.430903in}}%
\pgfpathlineto{\pgfqpoint{1.266177in}{2.438080in}}%
\pgfpathlineto{\pgfqpoint{1.339255in}{2.443923in}}%
\pgfpathlineto{\pgfqpoint{1.391105in}{2.445765in}}%
\pgfpathlineto{\pgfqpoint{1.431323in}{2.445967in}}%
\pgfpathlineto{\pgfqpoint{1.464183in}{2.445323in}}%
\pgfpathlineto{\pgfqpoint{1.516032in}{2.442894in}}%
\pgfpathlineto{\pgfqpoint{1.556250in}{2.439885in}}%
\pgfpathlineto{\pgfqpoint{1.629328in}{2.432240in}}%
\pgfpathlineto{\pgfqpoint{1.681177in}{2.425337in}}%
\pgfpathlineto{\pgfqpoint{1.754255in}{2.413884in}}%
\pgfpathlineto{\pgfqpoint{1.806105in}{2.404697in}}%
\pgfpathlineto{\pgfqpoint{1.846323in}{2.397055in}}%
\pgfpathlineto{\pgfqpoint{1.879183in}{2.390479in}}%
\pgfpathlineto{\pgfqpoint{1.931032in}{2.379502in}}%
\pgfpathlineto{\pgfqpoint{1.971250in}{2.370460in}}%
\pgfpathlineto{\pgfqpoint{1.975172in}{2.369561in}}%
\pgfpathlineto{\pgfqpoint{2.011468in}{2.360957in}}%
\pgfpathlineto{\pgfqpoint{2.044328in}{2.352815in}}%
\pgfpathlineto{\pgfqpoint{2.096177in}{2.339323in}}%
\pgfpathlineto{\pgfqpoint{2.100100in}{2.338278in}}%
\pgfpathlineto{\pgfqpoint{2.169255in}{2.318861in}}%
\pgfpathlineto{\pgfqpoint{2.221105in}{2.303450in}}%
\pgfpathlineto{\pgfqpoint{2.261323in}{2.290933in}}%
\pgfpathlineto{\pgfqpoint{2.294183in}{2.280417in}}%
\pgfpathlineto{\pgfqpoint{2.321966in}{2.271328in}}%
\pgfpathlineto{\pgfqpoint{2.346032in}{2.263296in}}%
\pgfpathlineto{\pgfqpoint{2.367261in}{2.256121in}}%
\pgfpathlineto{\pgfqpoint{2.386250in}{2.249617in}}%
\pgfpathlineto{\pgfqpoint{2.403428in}{2.243679in}}%
\pgfpathlineto{\pgfqpoint{2.419110in}{2.238213in}}%
\pgfpathlineto{\pgfqpoint{2.433536in}{2.233127in}}%
\pgfpathlineto{\pgfqpoint{2.446893in}{2.228391in}}%
\pgfpathlineto{\pgfqpoint{2.459328in}{2.223974in}}%
\pgfpathlineto{\pgfqpoint{2.470960in}{2.219806in}}%
\pgfpathlineto{\pgfqpoint{2.492188in}{2.212145in}}%
\pgfpathlineto{\pgfqpoint{2.511177in}{2.205242in}}%
\pgfpathlineto{\pgfqpoint{2.528355in}{2.198947in}}%
\pgfpathlineto{\pgfqpoint{2.544038in}{2.193170in}}%
\pgfpathlineto{\pgfqpoint{2.558464in}{2.187813in}}%
\pgfpathlineto{\pgfqpoint{2.571821in}{2.182829in}}%
\pgfpathlineto{\pgfqpoint{2.584255in}{2.178170in}}%
\pgfpathlineto{\pgfqpoint{2.636105in}{2.158581in}}%
\pgfpathlineto{\pgfqpoint{2.676323in}{2.143216in}}%
\pgfpathlineto{\pgfqpoint{2.709183in}{2.130492in}}%
\pgfpathlineto{\pgfqpoint{2.761032in}{2.110263in}}%
\pgfpathlineto{\pgfqpoint{2.801250in}{2.094408in}}%
\pgfpathlineto{\pgfqpoint{2.874328in}{2.065459in}}%
\pgfpathlineto{\pgfqpoint{2.926177in}{2.044726in}}%
\pgfpathlineto{\pgfqpoint{2.999255in}{2.015395in}}%
\pgfpathlineto{\pgfqpoint{3.051105in}{1.994552in}}%
\pgfpathlineto{\pgfqpoint{3.091323in}{1.978457in}}%
\pgfpathlineto{\pgfqpoint{3.124183in}{1.965210in}}%
\pgfpathlineto{\pgfqpoint{3.176032in}{1.944160in}}%
\pgfpathlineto{\pgfqpoint{3.216250in}{1.927552in}}%
\pgfpathlineto{\pgfqpoint{3.289328in}{1.896917in}}%
\pgfpathlineto{\pgfqpoint{3.341177in}{1.875093in}}%
\pgfpathlineto{\pgfqpoint{3.414255in}{1.844130in}}%
\pgfpathlineto{\pgfqpoint{3.466105in}{1.822068in}}%
\pgfpathlineto{\pgfqpoint{3.506323in}{1.804895in}}%
\pgfpathlineto{\pgfqpoint{3.539183in}{1.790860in}}%
\pgfpathlineto{\pgfqpoint{3.591032in}{1.768609in}}%
\pgfpathlineto{\pgfqpoint{3.631250in}{1.751332in}}%
\pgfpathlineto{\pgfqpoint{3.704328in}{1.719851in}}%
\pgfpathlineto{\pgfqpoint{3.756177in}{1.697414in}}%
\pgfpathlineto{\pgfqpoint{3.829255in}{1.665779in}}%
\pgfpathlineto{\pgfqpoint{3.881105in}{1.643255in}}%
\pgfpathlineto{\pgfqpoint{3.921323in}{1.625747in}}%
\pgfpathlineto{\pgfqpoint{3.954183in}{1.611430in}}%
\pgfpathlineto{\pgfqpoint{4.006032in}{1.588785in}}%
\pgfpathlineto{\pgfqpoint{4.046250in}{1.571226in}}%
\pgfusepath{stroke}%
\end{pgfscope}%
\begin{pgfscope}%
\pgfpathrectangle{\pgfqpoint{0.726250in}{0.525000in}}{\pgfqpoint{3.320000in}{2.887500in}}%
\pgfusepath{clip}%
\pgfsetbuttcap%
\pgfsetroundjoin%
\pgfsetlinewidth{1.003750pt}%
\definecolor{currentstroke}{rgb}{0.000000,0.000000,0.000000}%
\pgfsetstrokecolor{currentstroke}%
\pgfsetdash{{3.700000pt}{1.600000pt}}{0.000000pt}%
\pgfpathmoveto{\pgfqpoint{2.011242in}{0.515000in}}%
\pgfpathlineto{\pgfqpoint{2.011468in}{1.731632in}}%
\pgfpathlineto{\pgfqpoint{2.044328in}{1.875832in}}%
\pgfpathlineto{\pgfqpoint{2.096177in}{1.992680in}}%
\pgfpathlineto{\pgfqpoint{2.100100in}{1.999058in}}%
\pgfpathlineto{\pgfqpoint{2.169255in}{2.082126in}}%
\pgfpathlineto{\pgfqpoint{2.221105in}{2.123734in}}%
\pgfpathlineto{\pgfqpoint{2.261323in}{2.149218in}}%
\pgfpathlineto{\pgfqpoint{2.294183in}{2.167064in}}%
\pgfpathlineto{\pgfqpoint{2.321966in}{2.180375in}}%
\pgfpathlineto{\pgfqpoint{2.346032in}{2.190854in}}%
\pgfpathlineto{\pgfqpoint{2.367261in}{2.199375in}}%
\pgfpathlineto{\pgfqpoint{2.386250in}{2.206469in}}%
\pgfpathlineto{\pgfqpoint{2.403428in}{2.212427in}}%
\pgfpathlineto{\pgfqpoint{2.419110in}{2.217580in}}%
\pgfpathlineto{\pgfqpoint{2.433536in}{2.222093in}}%
\pgfpathlineto{\pgfqpoint{2.446893in}{2.226084in}}%
\pgfpathlineto{\pgfqpoint{2.459328in}{2.229673in}}%
\pgfpathlineto{\pgfqpoint{2.470960in}{2.232903in}}%
\pgfpathlineto{\pgfqpoint{2.492188in}{2.238526in}}%
\pgfpathlineto{\pgfqpoint{2.511177in}{2.243257in}}%
\pgfpathlineto{\pgfqpoint{2.528355in}{2.247337in}}%
\pgfpathlineto{\pgfqpoint{2.544038in}{2.250890in}}%
\pgfpathlineto{\pgfqpoint{2.558464in}{2.254024in}}%
\pgfpathlineto{\pgfqpoint{2.571821in}{2.256812in}}%
\pgfpathlineto{\pgfqpoint{2.584255in}{2.259317in}}%
\pgfpathlineto{\pgfqpoint{2.636105in}{2.268952in}}%
\pgfpathlineto{\pgfqpoint{2.676323in}{2.275529in}}%
\pgfpathlineto{\pgfqpoint{2.709183in}{2.280369in}}%
\pgfpathlineto{\pgfqpoint{2.761032in}{2.287158in}}%
\pgfpathlineto{\pgfqpoint{2.801250in}{2.291670in}}%
\pgfpathlineto{\pgfqpoint{2.874328in}{2.298596in}}%
\pgfpathlineto{\pgfqpoint{2.926177in}{2.302524in}}%
\pgfpathlineto{\pgfqpoint{2.999255in}{2.307046in}}%
\pgfpathlineto{\pgfqpoint{3.051105in}{2.309594in}}%
\pgfpathlineto{\pgfqpoint{3.091323in}{2.311244in}}%
\pgfpathlineto{\pgfqpoint{3.124183in}{2.312419in}}%
\pgfpathlineto{\pgfqpoint{3.176032in}{2.314022in}}%
\pgfpathlineto{\pgfqpoint{3.216250in}{2.315038in}}%
\pgfpathlineto{\pgfqpoint{3.289328in}{2.316489in}}%
\pgfpathlineto{\pgfqpoint{3.341177in}{2.317246in}}%
\pgfpathlineto{\pgfqpoint{3.414255in}{2.318113in}}%
\pgfpathlineto{\pgfqpoint{3.466105in}{2.318607in}}%
\pgfpathlineto{\pgfqpoint{3.506323in}{2.318861in}}%
\pgfpathlineto{\pgfqpoint{3.539183in}{2.319104in}}%
\pgfpathlineto{\pgfqpoint{3.591032in}{2.319351in}}%
\pgfpathlineto{\pgfqpoint{3.631250in}{2.319534in}}%
\pgfpathlineto{\pgfqpoint{3.704328in}{2.319728in}}%
\pgfpathlineto{\pgfqpoint{3.756177in}{2.319848in}}%
\pgfpathlineto{\pgfqpoint{3.829255in}{2.320025in}}%
\pgfpathlineto{\pgfqpoint{3.881105in}{2.320093in}}%
\pgfpathlineto{\pgfqpoint{3.921323in}{2.320098in}}%
\pgfpathlineto{\pgfqpoint{3.954183in}{2.320155in}}%
\pgfpathlineto{\pgfqpoint{4.006032in}{2.320161in}}%
\pgfpathlineto{\pgfqpoint{4.046250in}{2.320218in}}%
\pgfusepath{stroke}%
\end{pgfscope}%
\begin{pgfscope}%
\pgfsetrectcap%
\pgfsetmiterjoin%
\pgfsetlinewidth{1.003750pt}%
\definecolor{currentstroke}{rgb}{0.000000,0.000000,0.000000}%
\pgfsetstrokecolor{currentstroke}%
\pgfsetdash{}{0pt}%
\pgfpathmoveto{\pgfqpoint{0.726250in}{0.525000in}}%
\pgfpathlineto{\pgfqpoint{0.726250in}{3.412500in}}%
\pgfusepath{stroke}%
\end{pgfscope}%
\begin{pgfscope}%
\pgfsetrectcap%
\pgfsetmiterjoin%
\pgfsetlinewidth{1.003750pt}%
\definecolor{currentstroke}{rgb}{0.000000,0.000000,0.000000}%
\pgfsetstrokecolor{currentstroke}%
\pgfsetdash{}{0pt}%
\pgfpathmoveto{\pgfqpoint{4.046250in}{0.525000in}}%
\pgfpathlineto{\pgfqpoint{4.046250in}{3.412500in}}%
\pgfusepath{stroke}%
\end{pgfscope}%
\begin{pgfscope}%
\pgfsetrectcap%
\pgfsetmiterjoin%
\pgfsetlinewidth{1.003750pt}%
\definecolor{currentstroke}{rgb}{0.000000,0.000000,0.000000}%
\pgfsetstrokecolor{currentstroke}%
\pgfsetdash{}{0pt}%
\pgfpathmoveto{\pgfqpoint{0.726250in}{0.525000in}}%
\pgfpathlineto{\pgfqpoint{4.046250in}{0.525000in}}%
\pgfusepath{stroke}%
\end{pgfscope}%
\begin{pgfscope}%
\pgfsetrectcap%
\pgfsetmiterjoin%
\pgfsetlinewidth{1.003750pt}%
\definecolor{currentstroke}{rgb}{0.000000,0.000000,0.000000}%
\pgfsetstrokecolor{currentstroke}%
\pgfsetdash{}{0pt}%
\pgfpathmoveto{\pgfqpoint{0.726250in}{3.412500in}}%
\pgfpathlineto{\pgfqpoint{4.046250in}{3.412500in}}%
\pgfusepath{stroke}%
\end{pgfscope}%
\begin{pgfscope}%
\definecolor{textcolor}{rgb}{0.000000,0.000000,0.000000}%
\pgfsetstrokecolor{textcolor}%
\pgfsetfillcolor{textcolor}%
\pgftext[x=1.141250in,y=2.925830in,left,base]{\color{textcolor}\rmfamily\fontsize{9.000000}{10.800000}\selectfont \(\displaystyle \sigma_{\mathrm{photoelectric}}\)}%
\end{pgfscope}%
\begin{pgfscope}%
\definecolor{textcolor}{rgb}{0.000000,0.000000,0.000000}%
\pgfsetstrokecolor{textcolor}%
\pgfsetfillcolor{textcolor}%
\pgftext[x=3.381395in,y=1.022587in,left,base]{\color{textcolor}\rmfamily\fontsize{9.000000}{10.800000}\selectfont \(\displaystyle \sigma_{\mathrm{photoelectric}}\)}%
\end{pgfscope}%
\begin{pgfscope}%
\definecolor{textcolor}{rgb}{0.000000,0.000000,0.000000}%
\pgfsetstrokecolor{textcolor}%
\pgfsetfillcolor{textcolor}%
\pgftext[x=0.851177in,y=2.234530in,left,base]{\color{textcolor}\rmfamily\fontsize{9.000000}{10.800000}\selectfont \(\displaystyle \sigma_{\mathrm{Compton}}\)}%
\end{pgfscope}%
\begin{pgfscope}%
\definecolor{textcolor}{rgb}{0.000000,0.000000,0.000000}%
\pgfsetstrokecolor{textcolor}%
\pgfsetfillcolor{textcolor}%
\pgftext[x=3.216250in,y=1.944672in,left,base]{\color{textcolor}\rmfamily\fontsize{9.000000}{10.800000}\selectfont \(\displaystyle \sigma_{\mathrm{Compton}}\)}%
\end{pgfscope}%
\begin{pgfscope}%
\definecolor{textcolor}{rgb}{0.000000,0.000000,0.000000}%
\pgfsetstrokecolor{textcolor}%
\pgfsetfillcolor{textcolor}%
\pgftext[x=2.044328in,y=0.963515in,left,base]{\color{textcolor}\rmfamily\fontsize{9.000000}{10.800000}\selectfont \(\displaystyle \sigma_{\mathrm{pair}}\)}%
\end{pgfscope}%
\begin{pgfscope}%
\definecolor{textcolor}{rgb}{0.000000,0.000000,0.000000}%
\pgfsetstrokecolor{textcolor}%
\pgfsetfillcolor{textcolor}%
\pgftext[x=3.631250in,y=2.396207in,left,base]{\color{textcolor}\rmfamily\fontsize{9.000000}{10.800000}\selectfont \(\displaystyle \sigma_{\mathrm{pair}}\)}%
\end{pgfscope}%
\end{pgfpicture}%
\makeatother%
\endgroup%

  \caption{Photon interaction cross section in silicon. The dashed lines indicate
    the three dominant process: photoelectric absorption, Compton scattering and
    pair production. The contribution from the coherent scattering on the atoms
    of the material, which is subdominant at all energies, is included in the total
    but not shown separately.}
  \label{fig:photon_xsec}
\end{figure}

At low energy\sidenote{If we are adamant in stating a number, here, we might say
that the photoelectric effect dominates below $\sim 10$~keV, but this is highly
dependent on the atomic number of the absorbing material.} the interaction process
with the largest cross section is the photoelectric effect, where the photon is
absorbed by an atom and an electron is emitted in its place. As it turns out, this
is a complicated process, preferentially involving the atomic electrons of the
inner orbitals, where the atom is left in an excited state and subsequently relaxes
via the emission of another electron (the so-called Auger electron) or a fluorescence
photon. The photoelectric cross section scales rather violently with both the
atomic number $Z$ of the target material and the photon energy $E$
\begin{align}
  \sigma_\text{photoelectric} \propto
  \begin{cases}
    \displaystyle\frac{Z^4}{E^3} \quad\text{at low energy}\\[8pt]
    \displaystyle\frac{Z^5}{E} \quad\text{at high energy}
  \end{cases}
\end{align}
with dramatic resonances (of the order of 10\%) in correspondence of the energies
of the atomic orbitals.

At intermediate energies (roughly speaking around a MeV) the dominant photon
interaction process is the Compton scattering off (quasi-free) atomic electrons
\begin{align}
  \gamma \; e^- \rightarrow \gamma \; e^-.
\end{align}
This is an inelastic process where the photon transfer part of its energy to the
electron and, in general, changes its direction. Since the atomic electrons are
seen incoherently, the total cross section
\begin{align}
  \sigma_\text{Compton} \propto Z
\end{align}
is proportional to the atomic number of the target material.

Very high-energy photons (above a few tens of MeV) interact with matter mainly by
pair production in the field of atomic nuclei (although pair production off
atomic electrons also play a subdominant role). Like the photoelectric effect, this
is a destructive process, where the photon disappear and (in this case) an
electron-positron pair materializes instead. From a kinematic point of view,
this is a threshold process, with a minimum energy of $2 m_e c^2 = 1.022$~MeV
in the case of pair production off nuclei and $4 m_e c^2 = 2.044$~MeV in the
case of electrons. The total cross section
\begin{align}
  \sigma_\text{pair} \propto Z(Z + 1) \approx Z^2
\end{align}
is asymptotically constant at high energy and proportional to the square of the
atomic number of the target material.


\subsection{Pair production}

\begin{marginfigure}
  %% Creator: Matplotlib, PGF backend
%%
%% To include the figure in your LaTeX document, write
%%   \input{<filename>.pgf}
%%
%% Make sure the required packages are loaded in your preamble
%%   \usepackage{pgf}
%%
%% Also ensure that all the required font packages are loaded; for instance,
%% the lmodern package is sometimes necessary when using math font.
%%   \usepackage{lmodern}
%%
%% Figures using additional raster images can only be included by \input if
%% they are in the same directory as the main LaTeX file. For loading figures
%% from other directories you can use the `import` package
%%   \usepackage{import}
%%
%% and then include the figures with
%%   \import{<path to file>}{<filename>.pgf}
%%
%% Matplotlib used the following preamble
%%   \usepackage{fontspec}
%%   \setmainfont{DejaVuSerif.ttf}[Path=\detokenize{/usr/share/matplotlib/mpl-data/fonts/ttf/}]
%%   \setsansfont{DejaVuSans.ttf}[Path=\detokenize{/usr/share/matplotlib/mpl-data/fonts/ttf/}]
%%   \setmonofont{DejaVuSansMono.ttf}[Path=\detokenize{/usr/share/matplotlib/mpl-data/fonts/ttf/}]
%%
\begingroup%
\makeatletter%
\begin{pgfpicture}%
\pgfpathrectangle{\pgfpointorigin}{\pgfqpoint{1.950000in}{3.500000in}}%
\pgfusepath{use as bounding box, clip}%
\begin{pgfscope}%
\pgfsetbuttcap%
\pgfsetmiterjoin%
\definecolor{currentfill}{rgb}{1.000000,1.000000,1.000000}%
\pgfsetfillcolor{currentfill}%
\pgfsetlinewidth{0.000000pt}%
\definecolor{currentstroke}{rgb}{1.000000,1.000000,1.000000}%
\pgfsetstrokecolor{currentstroke}%
\pgfsetdash{}{0pt}%
\pgfpathmoveto{\pgfqpoint{0.000000in}{0.000000in}}%
\pgfpathlineto{\pgfqpoint{1.950000in}{0.000000in}}%
\pgfpathlineto{\pgfqpoint{1.950000in}{3.500000in}}%
\pgfpathlineto{\pgfqpoint{0.000000in}{3.500000in}}%
\pgfpathlineto{\pgfqpoint{0.000000in}{0.000000in}}%
\pgfpathclose%
\pgfusepath{fill}%
\end{pgfscope}%
\begin{pgfscope}%
\pgfpathrectangle{\pgfqpoint{0.148125in}{0.175000in}}{\pgfqpoint{1.653750in}{3.150000in}}%
\pgfusepath{clip}%
\pgfsetrectcap%
\pgfsetroundjoin%
\pgfsetlinewidth{1.003750pt}%
\definecolor{currentstroke}{rgb}{0.000000,0.000000,0.000000}%
\pgfsetstrokecolor{currentstroke}%
\pgfsetdash{}{0pt}%
\pgfpathmoveto{\pgfqpoint{0.423750in}{3.128125in}}%
\pgfpathlineto{\pgfqpoint{0.975000in}{2.891875in}}%
\pgfusepath{stroke}%
\end{pgfscope}%
\begin{pgfscope}%
\pgfpathrectangle{\pgfqpoint{0.148125in}{0.175000in}}{\pgfqpoint{1.653750in}{3.150000in}}%
\pgfusepath{clip}%
\pgfsetrectcap%
\pgfsetroundjoin%
\pgfsetlinewidth{1.003750pt}%
\definecolor{currentstroke}{rgb}{0.000000,0.000000,0.000000}%
\pgfsetstrokecolor{currentstroke}%
\pgfsetdash{}{0pt}%
\pgfpathmoveto{\pgfqpoint{0.975000in}{2.891875in}}%
\pgfpathlineto{\pgfqpoint{0.975000in}{2.183125in}}%
\pgfusepath{stroke}%
\end{pgfscope}%
\begin{pgfscope}%
\pgfpathrectangle{\pgfqpoint{0.148125in}{0.175000in}}{\pgfqpoint{1.653750in}{3.150000in}}%
\pgfusepath{clip}%
\pgfsetrectcap%
\pgfsetroundjoin%
\pgfsetlinewidth{1.003750pt}%
\definecolor{currentstroke}{rgb}{0.000000,0.000000,0.000000}%
\pgfsetstrokecolor{currentstroke}%
\pgfsetdash{}{0pt}%
\pgfpathmoveto{\pgfqpoint{0.423750in}{1.946875in}}%
\pgfpathlineto{\pgfqpoint{0.434652in}{1.939780in}}%
\pgfpathlineto{\pgfqpoint{0.440729in}{1.937829in}}%
\pgfpathlineto{\pgfqpoint{0.445333in}{1.939316in}}%
\pgfpathlineto{\pgfqpoint{0.448419in}{1.944344in}}%
\pgfpathlineto{\pgfqpoint{0.451019in}{1.956620in}}%
\pgfpathlineto{\pgfqpoint{0.454328in}{1.973358in}}%
\pgfpathlineto{\pgfqpoint{0.457485in}{1.978220in}}%
\pgfpathlineto{\pgfqpoint{0.462166in}{1.979526in}}%
\pgfpathlineto{\pgfqpoint{0.468309in}{1.977422in}}%
\pgfpathlineto{\pgfqpoint{0.483010in}{1.967577in}}%
\pgfpathlineto{\pgfqpoint{0.493285in}{1.961945in}}%
\pgfpathlineto{\pgfqpoint{0.498581in}{1.961818in}}%
\pgfpathlineto{\pgfqpoint{0.502330in}{1.965299in}}%
\pgfpathlineto{\pgfqpoint{0.504702in}{1.971993in}}%
\pgfpathlineto{\pgfqpoint{0.511011in}{2.000073in}}%
\pgfpathlineto{\pgfqpoint{0.514991in}{2.003015in}}%
\pgfpathlineto{\pgfqpoint{0.520509in}{2.002369in}}%
\pgfpathlineto{\pgfqpoint{0.527293in}{1.998769in}}%
\pgfpathlineto{\pgfqpoint{0.548710in}{1.985483in}}%
\pgfpathlineto{\pgfqpoint{0.553930in}{1.985531in}}%
\pgfpathlineto{\pgfqpoint{0.557603in}{1.989189in}}%
\pgfpathlineto{\pgfqpoint{0.559918in}{1.996018in}}%
\pgfpathlineto{\pgfqpoint{0.566299in}{2.023930in}}%
\pgfpathlineto{\pgfqpoint{0.570356in}{2.026692in}}%
\pgfpathlineto{\pgfqpoint{0.575947in}{2.025876in}}%
\pgfpathlineto{\pgfqpoint{0.586485in}{2.019630in}}%
\pgfpathlineto{\pgfqpoint{0.601020in}{2.010172in}}%
\pgfpathlineto{\pgfqpoint{0.606894in}{2.008695in}}%
\pgfpathlineto{\pgfqpoint{0.611264in}{2.010729in}}%
\pgfpathlineto{\pgfqpoint{0.614142in}{2.016241in}}%
\pgfpathlineto{\pgfqpoint{0.616590in}{2.028872in}}%
\pgfpathlineto{\pgfqpoint{0.618910in}{2.041803in}}%
\pgfpathlineto{\pgfqpoint{0.621590in}{2.047778in}}%
\pgfpathlineto{\pgfqpoint{0.625725in}{2.050359in}}%
\pgfpathlineto{\pgfqpoint{0.631388in}{2.049374in}}%
\pgfpathlineto{\pgfqpoint{0.641986in}{2.042989in}}%
\pgfpathlineto{\pgfqpoint{0.656471in}{2.033649in}}%
\pgfpathlineto{\pgfqpoint{0.662275in}{2.032335in}}%
\pgfpathlineto{\pgfqpoint{0.666566in}{2.034551in}}%
\pgfpathlineto{\pgfqpoint{0.669378in}{2.040220in}}%
\pgfpathlineto{\pgfqpoint{0.671780in}{2.052957in}}%
\pgfpathlineto{\pgfqpoint{0.674140in}{2.065795in}}%
\pgfpathlineto{\pgfqpoint{0.676885in}{2.071618in}}%
\pgfpathlineto{\pgfqpoint{0.681098in}{2.074017in}}%
\pgfpathlineto{\pgfqpoint{0.686832in}{2.072866in}}%
\pgfpathlineto{\pgfqpoint{0.697487in}{2.066347in}}%
\pgfpathlineto{\pgfqpoint{0.711918in}{2.057134in}}%
\pgfpathlineto{\pgfqpoint{0.717652in}{2.055983in}}%
\pgfpathlineto{\pgfqpoint{0.721865in}{2.058382in}}%
\pgfpathlineto{\pgfqpoint{0.724610in}{2.064205in}}%
\pgfpathlineto{\pgfqpoint{0.726970in}{2.077043in}}%
\pgfpathlineto{\pgfqpoint{0.729372in}{2.089780in}}%
\pgfpathlineto{\pgfqpoint{0.732184in}{2.095449in}}%
\pgfpathlineto{\pgfqpoint{0.736475in}{2.097665in}}%
\pgfpathlineto{\pgfqpoint{0.742279in}{2.096351in}}%
\pgfpathlineto{\pgfqpoint{0.752989in}{2.089704in}}%
\pgfpathlineto{\pgfqpoint{0.767362in}{2.080626in}}%
\pgfpathlineto{\pgfqpoint{0.773025in}{2.079641in}}%
\pgfpathlineto{\pgfqpoint{0.777160in}{2.082222in}}%
\pgfpathlineto{\pgfqpoint{0.779840in}{2.088197in}}%
\pgfpathlineto{\pgfqpoint{0.782833in}{2.105671in}}%
\pgfpathlineto{\pgfqpoint{0.785877in}{2.116912in}}%
\pgfpathlineto{\pgfqpoint{0.789475in}{2.120747in}}%
\pgfpathlineto{\pgfqpoint{0.794619in}{2.120972in}}%
\pgfpathlineto{\pgfqpoint{0.801131in}{2.118006in}}%
\pgfpathlineto{\pgfqpoint{0.825781in}{2.103291in}}%
\pgfpathlineto{\pgfqpoint{0.830618in}{2.104234in}}%
\pgfpathlineto{\pgfqpoint{0.833919in}{2.108761in}}%
\pgfpathlineto{\pgfqpoint{0.835965in}{2.116215in}}%
\pgfpathlineto{\pgfqpoint{0.841147in}{2.140811in}}%
\pgfpathlineto{\pgfqpoint{0.844820in}{2.144469in}}%
\pgfpathlineto{\pgfqpoint{0.850040in}{2.144517in}}%
\pgfpathlineto{\pgfqpoint{0.856609in}{2.141418in}}%
\pgfpathlineto{\pgfqpoint{0.878241in}{2.127631in}}%
\pgfpathlineto{\pgfqpoint{0.883759in}{2.126985in}}%
\pgfpathlineto{\pgfqpoint{0.887739in}{2.129927in}}%
\pgfpathlineto{\pgfqpoint{0.890292in}{2.136198in}}%
\pgfpathlineto{\pgfqpoint{0.896420in}{2.164701in}}%
\pgfpathlineto{\pgfqpoint{0.900169in}{2.168182in}}%
\pgfpathlineto{\pgfqpoint{0.905465in}{2.168055in}}%
\pgfpathlineto{\pgfqpoint{0.912090in}{2.164825in}}%
\pgfpathlineto{\pgfqpoint{0.933676in}{2.151145in}}%
\pgfpathlineto{\pgfqpoint{0.939120in}{2.150670in}}%
\pgfpathlineto{\pgfqpoint{0.943023in}{2.153793in}}%
\pgfpathlineto{\pgfqpoint{0.945514in}{2.160208in}}%
\pgfpathlineto{\pgfqpoint{0.951697in}{2.188583in}}%
\pgfpathlineto{\pgfqpoint{0.955523in}{2.191885in}}%
\pgfpathlineto{\pgfqpoint{0.960893in}{2.191584in}}%
\pgfpathlineto{\pgfqpoint{0.967572in}{2.188227in}}%
\pgfpathlineto{\pgfqpoint{0.975000in}{2.183125in}}%
\pgfpathlineto{\pgfqpoint{0.975000in}{2.183125in}}%
\pgfusepath{stroke}%
\end{pgfscope}%
\begin{pgfscope}%
\pgfpathrectangle{\pgfqpoint{0.148125in}{0.175000in}}{\pgfqpoint{1.653750in}{3.150000in}}%
\pgfusepath{clip}%
\pgfsetrectcap%
\pgfsetroundjoin%
\pgfsetlinewidth{1.003750pt}%
\definecolor{currentstroke}{rgb}{0.000000,0.000000,0.000000}%
\pgfsetstrokecolor{currentstroke}%
\pgfsetdash{}{0pt}%
\pgfpathmoveto{\pgfqpoint{0.975000in}{2.183125in}}%
\pgfpathlineto{\pgfqpoint{1.526250in}{1.946875in}}%
\pgfusepath{stroke}%
\end{pgfscope}%
\begin{pgfscope}%
\pgfpathrectangle{\pgfqpoint{0.148125in}{0.175000in}}{\pgfqpoint{1.653750in}{3.150000in}}%
\pgfusepath{clip}%
\pgfsetrectcap%
\pgfsetroundjoin%
\pgfsetlinewidth{1.003750pt}%
\definecolor{currentstroke}{rgb}{0.000000,0.000000,0.000000}%
\pgfsetstrokecolor{currentstroke}%
\pgfsetdash{}{0pt}%
\pgfpathmoveto{\pgfqpoint{0.975000in}{2.891875in}}%
\pgfpathlineto{\pgfqpoint{0.985902in}{2.884780in}}%
\pgfpathlineto{\pgfqpoint{0.991979in}{2.882829in}}%
\pgfpathlineto{\pgfqpoint{0.996583in}{2.884316in}}%
\pgfpathlineto{\pgfqpoint{0.999669in}{2.889344in}}%
\pgfpathlineto{\pgfqpoint{1.002269in}{2.901620in}}%
\pgfpathlineto{\pgfqpoint{1.005578in}{2.918358in}}%
\pgfpathlineto{\pgfqpoint{1.008735in}{2.923220in}}%
\pgfpathlineto{\pgfqpoint{1.013416in}{2.924526in}}%
\pgfpathlineto{\pgfqpoint{1.019559in}{2.922422in}}%
\pgfpathlineto{\pgfqpoint{1.034260in}{2.912577in}}%
\pgfpathlineto{\pgfqpoint{1.044535in}{2.906945in}}%
\pgfpathlineto{\pgfqpoint{1.049831in}{2.906818in}}%
\pgfpathlineto{\pgfqpoint{1.053580in}{2.910299in}}%
\pgfpathlineto{\pgfqpoint{1.055952in}{2.916993in}}%
\pgfpathlineto{\pgfqpoint{1.062261in}{2.945073in}}%
\pgfpathlineto{\pgfqpoint{1.066241in}{2.948015in}}%
\pgfpathlineto{\pgfqpoint{1.071759in}{2.947369in}}%
\pgfpathlineto{\pgfqpoint{1.078543in}{2.943769in}}%
\pgfpathlineto{\pgfqpoint{1.099960in}{2.930483in}}%
\pgfpathlineto{\pgfqpoint{1.105180in}{2.930531in}}%
\pgfpathlineto{\pgfqpoint{1.108853in}{2.934189in}}%
\pgfpathlineto{\pgfqpoint{1.111168in}{2.941018in}}%
\pgfpathlineto{\pgfqpoint{1.117549in}{2.968930in}}%
\pgfpathlineto{\pgfqpoint{1.121606in}{2.971692in}}%
\pgfpathlineto{\pgfqpoint{1.127197in}{2.970876in}}%
\pgfpathlineto{\pgfqpoint{1.137735in}{2.964630in}}%
\pgfpathlineto{\pgfqpoint{1.152270in}{2.955172in}}%
\pgfpathlineto{\pgfqpoint{1.158144in}{2.953695in}}%
\pgfpathlineto{\pgfqpoint{1.162514in}{2.955729in}}%
\pgfpathlineto{\pgfqpoint{1.165392in}{2.961241in}}%
\pgfpathlineto{\pgfqpoint{1.167840in}{2.973872in}}%
\pgfpathlineto{\pgfqpoint{1.170160in}{2.986803in}}%
\pgfpathlineto{\pgfqpoint{1.172840in}{2.992778in}}%
\pgfpathlineto{\pgfqpoint{1.176975in}{2.995359in}}%
\pgfpathlineto{\pgfqpoint{1.182638in}{2.994374in}}%
\pgfpathlineto{\pgfqpoint{1.193236in}{2.987989in}}%
\pgfpathlineto{\pgfqpoint{1.207721in}{2.978649in}}%
\pgfpathlineto{\pgfqpoint{1.213525in}{2.977335in}}%
\pgfpathlineto{\pgfqpoint{1.217816in}{2.979551in}}%
\pgfpathlineto{\pgfqpoint{1.220628in}{2.985220in}}%
\pgfpathlineto{\pgfqpoint{1.223030in}{2.997957in}}%
\pgfpathlineto{\pgfqpoint{1.225390in}{3.010795in}}%
\pgfpathlineto{\pgfqpoint{1.228135in}{3.016618in}}%
\pgfpathlineto{\pgfqpoint{1.232348in}{3.019017in}}%
\pgfpathlineto{\pgfqpoint{1.238082in}{3.017866in}}%
\pgfpathlineto{\pgfqpoint{1.248737in}{3.011347in}}%
\pgfpathlineto{\pgfqpoint{1.263168in}{3.002134in}}%
\pgfpathlineto{\pgfqpoint{1.268902in}{3.000983in}}%
\pgfpathlineto{\pgfqpoint{1.273115in}{3.003382in}}%
\pgfpathlineto{\pgfqpoint{1.275860in}{3.009205in}}%
\pgfpathlineto{\pgfqpoint{1.278220in}{3.022043in}}%
\pgfpathlineto{\pgfqpoint{1.280622in}{3.034780in}}%
\pgfpathlineto{\pgfqpoint{1.283434in}{3.040449in}}%
\pgfpathlineto{\pgfqpoint{1.287725in}{3.042665in}}%
\pgfpathlineto{\pgfqpoint{1.293529in}{3.041351in}}%
\pgfpathlineto{\pgfqpoint{1.304239in}{3.034704in}}%
\pgfpathlineto{\pgfqpoint{1.318612in}{3.025626in}}%
\pgfpathlineto{\pgfqpoint{1.324275in}{3.024641in}}%
\pgfpathlineto{\pgfqpoint{1.328410in}{3.027222in}}%
\pgfpathlineto{\pgfqpoint{1.331090in}{3.033197in}}%
\pgfpathlineto{\pgfqpoint{1.334083in}{3.050671in}}%
\pgfpathlineto{\pgfqpoint{1.337127in}{3.061912in}}%
\pgfpathlineto{\pgfqpoint{1.340725in}{3.065747in}}%
\pgfpathlineto{\pgfqpoint{1.345869in}{3.065972in}}%
\pgfpathlineto{\pgfqpoint{1.352381in}{3.063006in}}%
\pgfpathlineto{\pgfqpoint{1.377031in}{3.048291in}}%
\pgfpathlineto{\pgfqpoint{1.381868in}{3.049234in}}%
\pgfpathlineto{\pgfqpoint{1.385169in}{3.053761in}}%
\pgfpathlineto{\pgfqpoint{1.387215in}{3.061215in}}%
\pgfpathlineto{\pgfqpoint{1.392397in}{3.085811in}}%
\pgfpathlineto{\pgfqpoint{1.396070in}{3.089469in}}%
\pgfpathlineto{\pgfqpoint{1.401290in}{3.089517in}}%
\pgfpathlineto{\pgfqpoint{1.407859in}{3.086418in}}%
\pgfpathlineto{\pgfqpoint{1.429491in}{3.072631in}}%
\pgfpathlineto{\pgfqpoint{1.435009in}{3.071985in}}%
\pgfpathlineto{\pgfqpoint{1.438989in}{3.074927in}}%
\pgfpathlineto{\pgfqpoint{1.441542in}{3.081198in}}%
\pgfpathlineto{\pgfqpoint{1.447670in}{3.109701in}}%
\pgfpathlineto{\pgfqpoint{1.451419in}{3.113182in}}%
\pgfpathlineto{\pgfqpoint{1.456715in}{3.113055in}}%
\pgfpathlineto{\pgfqpoint{1.463340in}{3.109825in}}%
\pgfpathlineto{\pgfqpoint{1.484926in}{3.096145in}}%
\pgfpathlineto{\pgfqpoint{1.490370in}{3.095670in}}%
\pgfpathlineto{\pgfqpoint{1.494273in}{3.098793in}}%
\pgfpathlineto{\pgfqpoint{1.496764in}{3.105208in}}%
\pgfpathlineto{\pgfqpoint{1.502947in}{3.133583in}}%
\pgfpathlineto{\pgfqpoint{1.506773in}{3.136885in}}%
\pgfpathlineto{\pgfqpoint{1.512143in}{3.136584in}}%
\pgfpathlineto{\pgfqpoint{1.518822in}{3.133227in}}%
\pgfpathlineto{\pgfqpoint{1.526250in}{3.128125in}}%
\pgfpathlineto{\pgfqpoint{1.526250in}{3.128125in}}%
\pgfusepath{stroke}%
\end{pgfscope}%
\begin{pgfscope}%
\pgfpathrectangle{\pgfqpoint{0.148125in}{0.175000in}}{\pgfqpoint{1.653750in}{3.150000in}}%
\pgfusepath{clip}%
\pgfsetrectcap%
\pgfsetroundjoin%
\pgfsetlinewidth{1.003750pt}%
\definecolor{currentstroke}{rgb}{0.000000,0.000000,0.000000}%
\pgfsetstrokecolor{currentstroke}%
\pgfsetdash{}{0pt}%
\pgfpathmoveto{\pgfqpoint{0.423750in}{0.371875in}}%
\pgfpathlineto{\pgfqpoint{0.434652in}{0.364780in}}%
\pgfpathlineto{\pgfqpoint{0.440729in}{0.362829in}}%
\pgfpathlineto{\pgfqpoint{0.445333in}{0.364316in}}%
\pgfpathlineto{\pgfqpoint{0.448419in}{0.369344in}}%
\pgfpathlineto{\pgfqpoint{0.451019in}{0.381620in}}%
\pgfpathlineto{\pgfqpoint{0.454328in}{0.398358in}}%
\pgfpathlineto{\pgfqpoint{0.457485in}{0.403220in}}%
\pgfpathlineto{\pgfqpoint{0.462166in}{0.404526in}}%
\pgfpathlineto{\pgfqpoint{0.468309in}{0.402422in}}%
\pgfpathlineto{\pgfqpoint{0.483010in}{0.392577in}}%
\pgfpathlineto{\pgfqpoint{0.493285in}{0.386945in}}%
\pgfpathlineto{\pgfqpoint{0.498581in}{0.386818in}}%
\pgfpathlineto{\pgfqpoint{0.502330in}{0.390299in}}%
\pgfpathlineto{\pgfqpoint{0.504702in}{0.396993in}}%
\pgfpathlineto{\pgfqpoint{0.511011in}{0.425073in}}%
\pgfpathlineto{\pgfqpoint{0.514991in}{0.428015in}}%
\pgfpathlineto{\pgfqpoint{0.520509in}{0.427369in}}%
\pgfpathlineto{\pgfqpoint{0.527293in}{0.423769in}}%
\pgfpathlineto{\pgfqpoint{0.548710in}{0.410483in}}%
\pgfpathlineto{\pgfqpoint{0.553930in}{0.410531in}}%
\pgfpathlineto{\pgfqpoint{0.557603in}{0.414189in}}%
\pgfpathlineto{\pgfqpoint{0.559918in}{0.421018in}}%
\pgfpathlineto{\pgfqpoint{0.566299in}{0.448930in}}%
\pgfpathlineto{\pgfqpoint{0.570356in}{0.451692in}}%
\pgfpathlineto{\pgfqpoint{0.575947in}{0.450876in}}%
\pgfpathlineto{\pgfqpoint{0.586485in}{0.444630in}}%
\pgfpathlineto{\pgfqpoint{0.601020in}{0.435172in}}%
\pgfpathlineto{\pgfqpoint{0.606894in}{0.433695in}}%
\pgfpathlineto{\pgfqpoint{0.611264in}{0.435729in}}%
\pgfpathlineto{\pgfqpoint{0.614142in}{0.441241in}}%
\pgfpathlineto{\pgfqpoint{0.616590in}{0.453872in}}%
\pgfpathlineto{\pgfqpoint{0.618910in}{0.466803in}}%
\pgfpathlineto{\pgfqpoint{0.621590in}{0.472778in}}%
\pgfpathlineto{\pgfqpoint{0.625725in}{0.475359in}}%
\pgfpathlineto{\pgfqpoint{0.631388in}{0.474374in}}%
\pgfpathlineto{\pgfqpoint{0.641986in}{0.467989in}}%
\pgfpathlineto{\pgfqpoint{0.656471in}{0.458649in}}%
\pgfpathlineto{\pgfqpoint{0.662275in}{0.457335in}}%
\pgfpathlineto{\pgfqpoint{0.666566in}{0.459551in}}%
\pgfpathlineto{\pgfqpoint{0.669378in}{0.465220in}}%
\pgfpathlineto{\pgfqpoint{0.671780in}{0.477957in}}%
\pgfpathlineto{\pgfqpoint{0.674140in}{0.490795in}}%
\pgfpathlineto{\pgfqpoint{0.676885in}{0.496618in}}%
\pgfpathlineto{\pgfqpoint{0.681098in}{0.499017in}}%
\pgfpathlineto{\pgfqpoint{0.686832in}{0.497866in}}%
\pgfpathlineto{\pgfqpoint{0.697487in}{0.491347in}}%
\pgfpathlineto{\pgfqpoint{0.711918in}{0.482134in}}%
\pgfpathlineto{\pgfqpoint{0.717652in}{0.480983in}}%
\pgfpathlineto{\pgfqpoint{0.721865in}{0.483382in}}%
\pgfpathlineto{\pgfqpoint{0.724610in}{0.489205in}}%
\pgfpathlineto{\pgfqpoint{0.726970in}{0.502043in}}%
\pgfpathlineto{\pgfqpoint{0.729372in}{0.514780in}}%
\pgfpathlineto{\pgfqpoint{0.732184in}{0.520449in}}%
\pgfpathlineto{\pgfqpoint{0.736475in}{0.522665in}}%
\pgfpathlineto{\pgfqpoint{0.742279in}{0.521351in}}%
\pgfpathlineto{\pgfqpoint{0.752989in}{0.514704in}}%
\pgfpathlineto{\pgfqpoint{0.767362in}{0.505626in}}%
\pgfpathlineto{\pgfqpoint{0.773025in}{0.504641in}}%
\pgfpathlineto{\pgfqpoint{0.777160in}{0.507222in}}%
\pgfpathlineto{\pgfqpoint{0.779840in}{0.513197in}}%
\pgfpathlineto{\pgfqpoint{0.782833in}{0.530671in}}%
\pgfpathlineto{\pgfqpoint{0.785877in}{0.541912in}}%
\pgfpathlineto{\pgfqpoint{0.789475in}{0.545747in}}%
\pgfpathlineto{\pgfqpoint{0.794619in}{0.545972in}}%
\pgfpathlineto{\pgfqpoint{0.801131in}{0.543006in}}%
\pgfpathlineto{\pgfqpoint{0.825781in}{0.528291in}}%
\pgfpathlineto{\pgfqpoint{0.830618in}{0.529234in}}%
\pgfpathlineto{\pgfqpoint{0.833919in}{0.533761in}}%
\pgfpathlineto{\pgfqpoint{0.835965in}{0.541215in}}%
\pgfpathlineto{\pgfqpoint{0.841147in}{0.565811in}}%
\pgfpathlineto{\pgfqpoint{0.844820in}{0.569469in}}%
\pgfpathlineto{\pgfqpoint{0.850040in}{0.569517in}}%
\pgfpathlineto{\pgfqpoint{0.856609in}{0.566418in}}%
\pgfpathlineto{\pgfqpoint{0.878241in}{0.552631in}}%
\pgfpathlineto{\pgfqpoint{0.883759in}{0.551985in}}%
\pgfpathlineto{\pgfqpoint{0.887739in}{0.554927in}}%
\pgfpathlineto{\pgfqpoint{0.890292in}{0.561198in}}%
\pgfpathlineto{\pgfqpoint{0.896420in}{0.589701in}}%
\pgfpathlineto{\pgfqpoint{0.900169in}{0.593182in}}%
\pgfpathlineto{\pgfqpoint{0.905465in}{0.593055in}}%
\pgfpathlineto{\pgfqpoint{0.912090in}{0.589825in}}%
\pgfpathlineto{\pgfqpoint{0.933676in}{0.576145in}}%
\pgfpathlineto{\pgfqpoint{0.939120in}{0.575670in}}%
\pgfpathlineto{\pgfqpoint{0.943023in}{0.578793in}}%
\pgfpathlineto{\pgfqpoint{0.945514in}{0.585208in}}%
\pgfpathlineto{\pgfqpoint{0.951697in}{0.613583in}}%
\pgfpathlineto{\pgfqpoint{0.955523in}{0.616885in}}%
\pgfpathlineto{\pgfqpoint{0.960893in}{0.616584in}}%
\pgfpathlineto{\pgfqpoint{0.967572in}{0.613227in}}%
\pgfpathlineto{\pgfqpoint{0.975000in}{0.608125in}}%
\pgfpathlineto{\pgfqpoint{0.975000in}{0.608125in}}%
\pgfusepath{stroke}%
\end{pgfscope}%
\begin{pgfscope}%
\pgfpathrectangle{\pgfqpoint{0.148125in}{0.175000in}}{\pgfqpoint{1.653750in}{3.150000in}}%
\pgfusepath{clip}%
\pgfsetrectcap%
\pgfsetroundjoin%
\pgfsetlinewidth{1.003750pt}%
\definecolor{currentstroke}{rgb}{0.000000,0.000000,0.000000}%
\pgfsetstrokecolor{currentstroke}%
\pgfsetdash{}{0pt}%
\pgfpathmoveto{\pgfqpoint{0.975000in}{0.608125in}}%
\pgfpathlineto{\pgfqpoint{0.975000in}{1.316875in}}%
\pgfusepath{stroke}%
\end{pgfscope}%
\begin{pgfscope}%
\pgfpathrectangle{\pgfqpoint{0.148125in}{0.175000in}}{\pgfqpoint{1.653750in}{3.150000in}}%
\pgfusepath{clip}%
\pgfsetrectcap%
\pgfsetroundjoin%
\pgfsetlinewidth{1.003750pt}%
\definecolor{currentstroke}{rgb}{0.000000,0.000000,0.000000}%
\pgfsetstrokecolor{currentstroke}%
\pgfsetdash{}{0pt}%
\pgfpathmoveto{\pgfqpoint{0.423750in}{1.553125in}}%
\pgfpathlineto{\pgfqpoint{0.426131in}{1.540337in}}%
\pgfpathlineto{\pgfqpoint{0.428909in}{1.534591in}}%
\pgfpathlineto{\pgfqpoint{0.433161in}{1.532283in}}%
\pgfpathlineto{\pgfqpoint{0.438930in}{1.533516in}}%
\pgfpathlineto{\pgfqpoint{0.449613in}{1.540099in}}%
\pgfpathlineto{\pgfqpoint{0.464015in}{1.549246in}}%
\pgfpathlineto{\pgfqpoint{0.469714in}{1.550314in}}%
\pgfpathlineto{\pgfqpoint{0.473888in}{1.547824in}}%
\pgfpathlineto{\pgfqpoint{0.476600in}{1.541925in}}%
\pgfpathlineto{\pgfqpoint{0.479610in}{1.524489in}}%
\pgfpathlineto{\pgfqpoint{0.482618in}{1.513165in}}%
\pgfpathlineto{\pgfqpoint{0.486178in}{1.509243in}}%
\pgfpathlineto{\pgfqpoint{0.491285in}{1.508929in}}%
\pgfpathlineto{\pgfqpoint{0.497768in}{1.511827in}}%
\pgfpathlineto{\pgfqpoint{0.522453in}{1.526624in}}%
\pgfpathlineto{\pgfqpoint{0.527328in}{1.525771in}}%
\pgfpathlineto{\pgfqpoint{0.530666in}{1.521330in}}%
\pgfpathlineto{\pgfqpoint{0.532737in}{1.513934in}}%
\pgfpathlineto{\pgfqpoint{0.537886in}{1.489263in}}%
\pgfpathlineto{\pgfqpoint{0.541522in}{1.485516in}}%
\pgfpathlineto{\pgfqpoint{0.546704in}{1.485379in}}%
\pgfpathlineto{\pgfqpoint{0.553245in}{1.488413in}}%
\pgfpathlineto{\pgfqpoint{0.577858in}{1.503041in}}%
\pgfpathlineto{\pgfqpoint{0.582656in}{1.502008in}}%
\pgfpathlineto{\pgfqpoint{0.585921in}{1.497397in}}%
\pgfpathlineto{\pgfqpoint{0.588665in}{1.485458in}}%
\pgfpathlineto{\pgfqpoint{0.591841in}{1.468410in}}%
\pgfpathlineto{\pgfqpoint{0.594822in}{1.463137in}}%
\pgfpathlineto{\pgfqpoint{0.599309in}{1.461377in}}%
\pgfpathlineto{\pgfqpoint{0.605285in}{1.463093in}}%
\pgfpathlineto{\pgfqpoint{0.616121in}{1.470032in}}%
\pgfpathlineto{\pgfqpoint{0.627084in}{1.477270in}}%
\pgfpathlineto{\pgfqpoint{0.633259in}{1.479450in}}%
\pgfpathlineto{\pgfqpoint{0.637980in}{1.478235in}}%
\pgfpathlineto{\pgfqpoint{0.641172in}{1.473456in}}%
\pgfpathlineto{\pgfqpoint{0.643857in}{1.461378in}}%
\pgfpathlineto{\pgfqpoint{0.647084in}{1.444448in}}%
\pgfpathlineto{\pgfqpoint{0.650135in}{1.439338in}}%
\pgfpathlineto{\pgfqpoint{0.654699in}{1.437759in}}%
\pgfpathlineto{\pgfqpoint{0.660743in}{1.439633in}}%
\pgfpathlineto{\pgfqpoint{0.671624in}{1.446677in}}%
\pgfpathlineto{\pgfqpoint{0.682547in}{1.453822in}}%
\pgfpathlineto{\pgfqpoint{0.688657in}{1.455850in}}%
\pgfpathlineto{\pgfqpoint{0.693299in}{1.454453in}}%
\pgfpathlineto{\pgfqpoint{0.696421in}{1.449508in}}%
\pgfpathlineto{\pgfqpoint{0.699049in}{1.437296in}}%
\pgfpathlineto{\pgfqpoint{0.702329in}{1.420492in}}%
\pgfpathlineto{\pgfqpoint{0.705451in}{1.415547in}}%
\pgfpathlineto{\pgfqpoint{0.710093in}{1.414150in}}%
\pgfpathlineto{\pgfqpoint{0.716203in}{1.416178in}}%
\pgfpathlineto{\pgfqpoint{0.727126in}{1.423323in}}%
\pgfpathlineto{\pgfqpoint{0.738007in}{1.430367in}}%
\pgfpathlineto{\pgfqpoint{0.744051in}{1.432241in}}%
\pgfpathlineto{\pgfqpoint{0.748615in}{1.430662in}}%
\pgfpathlineto{\pgfqpoint{0.751666in}{1.425552in}}%
\pgfpathlineto{\pgfqpoint{0.754240in}{1.413213in}}%
\pgfpathlineto{\pgfqpoint{0.757578in}{1.396544in}}%
\pgfpathlineto{\pgfqpoint{0.760770in}{1.391765in}}%
\pgfpathlineto{\pgfqpoint{0.765491in}{1.390550in}}%
\pgfpathlineto{\pgfqpoint{0.771666in}{1.392730in}}%
\pgfpathlineto{\pgfqpoint{0.786384in}{1.402614in}}%
\pgfpathlineto{\pgfqpoint{0.796623in}{1.408162in}}%
\pgfpathlineto{\pgfqpoint{0.801881in}{1.408201in}}%
\pgfpathlineto{\pgfqpoint{0.805592in}{1.404632in}}%
\pgfpathlineto{\pgfqpoint{0.807935in}{1.397870in}}%
\pgfpathlineto{\pgfqpoint{0.814280in}{1.369873in}}%
\pgfpathlineto{\pgfqpoint{0.818298in}{1.367020in}}%
\pgfpathlineto{\pgfqpoint{0.823852in}{1.367752in}}%
\pgfpathlineto{\pgfqpoint{0.830662in}{1.371411in}}%
\pgfpathlineto{\pgfqpoint{0.852046in}{1.384621in}}%
\pgfpathlineto{\pgfqpoint{0.857228in}{1.384484in}}%
\pgfpathlineto{\pgfqpoint{0.860864in}{1.380737in}}%
\pgfpathlineto{\pgfqpoint{0.863150in}{1.373842in}}%
\pgfpathlineto{\pgfqpoint{0.868084in}{1.348670in}}%
\pgfpathlineto{\pgfqpoint{0.871422in}{1.344229in}}%
\pgfpathlineto{\pgfqpoint{0.876297in}{1.343376in}}%
\pgfpathlineto{\pgfqpoint{0.882599in}{1.345852in}}%
\pgfpathlineto{\pgfqpoint{0.907465in}{1.361071in}}%
\pgfpathlineto{\pgfqpoint{0.912572in}{1.360757in}}%
\pgfpathlineto{\pgfqpoint{0.916132in}{1.356835in}}%
\pgfpathlineto{\pgfqpoint{0.918362in}{1.349810in}}%
\pgfpathlineto{\pgfqpoint{0.923342in}{1.324744in}}%
\pgfpathlineto{\pgfqpoint{0.926753in}{1.320474in}}%
\pgfpathlineto{\pgfqpoint{0.931706in}{1.319802in}}%
\pgfpathlineto{\pgfqpoint{0.938070in}{1.322421in}}%
\pgfpathlineto{\pgfqpoint{0.962881in}{1.337514in}}%
\pgfpathlineto{\pgfqpoint{0.967911in}{1.337021in}}%
\pgfpathlineto{\pgfqpoint{0.971396in}{1.332924in}}%
\pgfpathlineto{\pgfqpoint{0.973572in}{1.325772in}}%
\pgfpathlineto{\pgfqpoint{0.975000in}{1.316875in}}%
\pgfpathlineto{\pgfqpoint{0.975000in}{1.316875in}}%
\pgfusepath{stroke}%
\end{pgfscope}%
\begin{pgfscope}%
\pgfpathrectangle{\pgfqpoint{0.148125in}{0.175000in}}{\pgfqpoint{1.653750in}{3.150000in}}%
\pgfusepath{clip}%
\pgfsetrectcap%
\pgfsetroundjoin%
\pgfsetlinewidth{1.003750pt}%
\definecolor{currentstroke}{rgb}{0.000000,0.000000,0.000000}%
\pgfsetstrokecolor{currentstroke}%
\pgfsetdash{}{0pt}%
\pgfpathmoveto{\pgfqpoint{0.975000in}{1.316875in}}%
\pgfpathlineto{\pgfqpoint{1.526250in}{1.553125in}}%
\pgfusepath{stroke}%
\end{pgfscope}%
\begin{pgfscope}%
\pgfpathrectangle{\pgfqpoint{0.148125in}{0.175000in}}{\pgfqpoint{1.653750in}{3.150000in}}%
\pgfusepath{clip}%
\pgfsetrectcap%
\pgfsetroundjoin%
\pgfsetlinewidth{1.003750pt}%
\definecolor{currentstroke}{rgb}{0.000000,0.000000,0.000000}%
\pgfsetstrokecolor{currentstroke}%
\pgfsetdash{}{0pt}%
\pgfpathmoveto{\pgfqpoint{0.975000in}{0.608125in}}%
\pgfpathlineto{\pgfqpoint{1.526250in}{0.371875in}}%
\pgfusepath{stroke}%
\end{pgfscope}%
\begin{pgfscope}%
\definecolor{textcolor}{rgb}{0.000000,0.000000,0.000000}%
\pgfsetstrokecolor{textcolor}%
\pgfsetfillcolor{textcolor}%
\pgftext[x=0.266250in,y=3.128125in,left,]{\color{textcolor}\rmfamily\fontsize{9.000000}{10.800000}\selectfont \(\displaystyle e^-\)}%
\end{pgfscope}%
\begin{pgfscope}%
\definecolor{textcolor}{rgb}{0.000000,0.000000,0.000000}%
\pgfsetstrokecolor{textcolor}%
\pgfsetfillcolor{textcolor}%
\pgftext[x=1.565625in,y=1.946875in,left,]{\color{textcolor}\rmfamily\fontsize{9.000000}{10.800000}\selectfont \(\displaystyle e^-\)}%
\end{pgfscope}%
\begin{pgfscope}%
\definecolor{textcolor}{rgb}{0.000000,0.000000,0.000000}%
\pgfsetstrokecolor{textcolor}%
\pgfsetfillcolor{textcolor}%
\pgftext[x=1.565625in,y=3.128125in,left,]{\color{textcolor}\rmfamily\fontsize{9.000000}{10.800000}\selectfont \(\displaystyle \gamma\)}%
\end{pgfscope}%
\begin{pgfscope}%
\definecolor{textcolor}{rgb}{0.000000,0.000000,0.000000}%
\pgfsetstrokecolor{textcolor}%
\pgfsetfillcolor{textcolor}%
\pgftext[x=0.423750in,y=2.065000in,,base]{\color{textcolor}\rmfamily\fontsize{9.000000}{10.800000}\selectfont \(\displaystyle N\)}%
\end{pgfscope}%
\begin{pgfscope}%
\definecolor{textcolor}{rgb}{0.000000,0.000000,0.000000}%
\pgfsetstrokecolor{textcolor}%
\pgfsetfillcolor{textcolor}%
\pgftext[x=0.266250in,y=1.553125in,left,]{\color{textcolor}\rmfamily\fontsize{9.000000}{10.800000}\selectfont \(\displaystyle \gamma\)}%
\end{pgfscope}%
\begin{pgfscope}%
\definecolor{textcolor}{rgb}{0.000000,0.000000,0.000000}%
\pgfsetstrokecolor{textcolor}%
\pgfsetfillcolor{textcolor}%
\pgftext[x=1.565625in,y=1.553125in,left,]{\color{textcolor}\rmfamily\fontsize{9.000000}{10.800000}\selectfont \(\displaystyle e^+\)}%
\end{pgfscope}%
\begin{pgfscope}%
\definecolor{textcolor}{rgb}{0.000000,0.000000,0.000000}%
\pgfsetstrokecolor{textcolor}%
\pgfsetfillcolor{textcolor}%
\pgftext[x=1.565625in,y=0.371875in,left,]{\color{textcolor}\rmfamily\fontsize{9.000000}{10.800000}\selectfont \(\displaystyle e^-\)}%
\end{pgfscope}%
\begin{pgfscope}%
\definecolor{textcolor}{rgb}{0.000000,0.000000,0.000000}%
\pgfsetstrokecolor{textcolor}%
\pgfsetfillcolor{textcolor}%
\pgftext[x=0.423750in,y=0.490000in,,base]{\color{textcolor}\rmfamily\fontsize{9.000000}{10.800000}\selectfont \(\displaystyle N\)}%
\end{pgfscope}%
\begin{pgfscope}%
\pgfpathrectangle{\pgfqpoint{0.148125in}{0.175000in}}{\pgfqpoint{1.653750in}{3.150000in}}%
\pgfusepath{clip}%
\pgfsetbuttcap%
\pgfsetmiterjoin%
\definecolor{currentfill}{rgb}{1.000000,1.000000,1.000000}%
\pgfsetfillcolor{currentfill}%
\pgfsetlinewidth{1.003750pt}%
\definecolor{currentstroke}{rgb}{0.000000,0.000000,0.000000}%
\pgfsetstrokecolor{currentstroke}%
\pgfsetdash{}{0pt}%
\pgfpathmoveto{\pgfqpoint{0.423750in}{1.868125in}}%
\pgfpathcurveto{\pgfqpoint{0.444635in}{1.868125in}}{\pgfqpoint{0.464667in}{1.876423in}}{\pgfqpoint{0.479435in}{1.891190in}}%
\pgfpathcurveto{\pgfqpoint{0.494202in}{1.905958in}}{\pgfqpoint{0.502500in}{1.925990in}}{\pgfqpoint{0.502500in}{1.946875in}}%
\pgfpathcurveto{\pgfqpoint{0.502500in}{1.967760in}}{\pgfqpoint{0.494202in}{1.987792in}}{\pgfqpoint{0.479435in}{2.002560in}}%
\pgfpathcurveto{\pgfqpoint{0.464667in}{2.017327in}}{\pgfqpoint{0.444635in}{2.025625in}}{\pgfqpoint{0.423750in}{2.025625in}}%
\pgfpathcurveto{\pgfqpoint{0.402865in}{2.025625in}}{\pgfqpoint{0.382833in}{2.017327in}}{\pgfqpoint{0.368065in}{2.002560in}}%
\pgfpathcurveto{\pgfqpoint{0.353298in}{1.987792in}}{\pgfqpoint{0.345000in}{1.967760in}}{\pgfqpoint{0.345000in}{1.946875in}}%
\pgfpathcurveto{\pgfqpoint{0.345000in}{1.925990in}}{\pgfqpoint{0.353298in}{1.905958in}}{\pgfqpoint{0.368065in}{1.891190in}}%
\pgfpathcurveto{\pgfqpoint{0.382833in}{1.876423in}}{\pgfqpoint{0.402865in}{1.868125in}}{\pgfqpoint{0.423750in}{1.868125in}}%
\pgfpathlineto{\pgfqpoint{0.423750in}{1.868125in}}%
\pgfpathclose%
\pgfusepath{stroke,fill}%
\end{pgfscope}%
\begin{pgfscope}%
\pgfpathrectangle{\pgfqpoint{0.148125in}{0.175000in}}{\pgfqpoint{1.653750in}{3.150000in}}%
\pgfusepath{clip}%
\pgfsetbuttcap%
\pgfsetmiterjoin%
\definecolor{currentfill}{rgb}{1.000000,1.000000,1.000000}%
\pgfsetfillcolor{currentfill}%
\pgfsetlinewidth{1.003750pt}%
\definecolor{currentstroke}{rgb}{0.000000,0.000000,0.000000}%
\pgfsetstrokecolor{currentstroke}%
\pgfsetdash{}{0pt}%
\pgfpathmoveto{\pgfqpoint{0.423750in}{0.293125in}}%
\pgfpathcurveto{\pgfqpoint{0.444635in}{0.293125in}}{\pgfqpoint{0.464667in}{0.301423in}}{\pgfqpoint{0.479435in}{0.316190in}}%
\pgfpathcurveto{\pgfqpoint{0.494202in}{0.330958in}}{\pgfqpoint{0.502500in}{0.350990in}}{\pgfqpoint{0.502500in}{0.371875in}}%
\pgfpathcurveto{\pgfqpoint{0.502500in}{0.392760in}}{\pgfqpoint{0.494202in}{0.412792in}}{\pgfqpoint{0.479435in}{0.427560in}}%
\pgfpathcurveto{\pgfqpoint{0.464667in}{0.442327in}}{\pgfqpoint{0.444635in}{0.450625in}}{\pgfqpoint{0.423750in}{0.450625in}}%
\pgfpathcurveto{\pgfqpoint{0.402865in}{0.450625in}}{\pgfqpoint{0.382833in}{0.442327in}}{\pgfqpoint{0.368065in}{0.427560in}}%
\pgfpathcurveto{\pgfqpoint{0.353298in}{0.412792in}}{\pgfqpoint{0.345000in}{0.392760in}}{\pgfqpoint{0.345000in}{0.371875in}}%
\pgfpathcurveto{\pgfqpoint{0.345000in}{0.350990in}}{\pgfqpoint{0.353298in}{0.330958in}}{\pgfqpoint{0.368065in}{0.316190in}}%
\pgfpathcurveto{\pgfqpoint{0.382833in}{0.301423in}}{\pgfqpoint{0.402865in}{0.293125in}}{\pgfqpoint{0.423750in}{0.293125in}}%
\pgfpathlineto{\pgfqpoint{0.423750in}{0.293125in}}%
\pgfpathclose%
\pgfusepath{stroke,fill}%
\end{pgfscope}%
\begin{pgfscope}%
\pgfpathrectangle{\pgfqpoint{0.148125in}{0.175000in}}{\pgfqpoint{1.653750in}{3.150000in}}%
\pgfusepath{clip}%
\pgfsetbuttcap%
\pgfsetroundjoin%
\definecolor{currentfill}{rgb}{0.000000,0.000000,0.000000}%
\pgfsetfillcolor{currentfill}%
\pgfsetlinewidth{1.003750pt}%
\definecolor{currentstroke}{rgb}{0.000000,0.000000,0.000000}%
\pgfsetstrokecolor{currentstroke}%
\pgfsetdash{}{0pt}%
\pgfsys@defobject{currentmarker}{\pgfqpoint{-0.060141in}{-0.060141in}}{\pgfqpoint{0.060141in}{0.060141in}}{%
\pgfpathmoveto{\pgfqpoint{-0.060141in}{-0.060141in}}%
\pgfpathlineto{\pgfqpoint{0.060141in}{0.060141in}}%
\pgfpathmoveto{\pgfqpoint{-0.060141in}{0.060141in}}%
\pgfpathlineto{\pgfqpoint{0.060141in}{-0.060141in}}%
\pgfusepath{stroke,fill}%
}%
\begin{pgfscope}%
\pgfsys@transformshift{0.423750in}{1.946875in}%
\pgfsys@useobject{currentmarker}{}%
\end{pgfscope}%
\end{pgfscope}%
\begin{pgfscope}%
\pgfpathrectangle{\pgfqpoint{0.148125in}{0.175000in}}{\pgfqpoint{1.653750in}{3.150000in}}%
\pgfusepath{clip}%
\pgfsetbuttcap%
\pgfsetroundjoin%
\definecolor{currentfill}{rgb}{0.000000,0.000000,0.000000}%
\pgfsetfillcolor{currentfill}%
\pgfsetlinewidth{1.003750pt}%
\definecolor{currentstroke}{rgb}{0.000000,0.000000,0.000000}%
\pgfsetstrokecolor{currentstroke}%
\pgfsetdash{}{0pt}%
\pgfsys@defobject{currentmarker}{\pgfqpoint{-0.060141in}{-0.060141in}}{\pgfqpoint{0.060141in}{0.060141in}}{%
\pgfpathmoveto{\pgfqpoint{-0.060141in}{-0.060141in}}%
\pgfpathlineto{\pgfqpoint{0.060141in}{0.060141in}}%
\pgfpathmoveto{\pgfqpoint{-0.060141in}{0.060141in}}%
\pgfpathlineto{\pgfqpoint{0.060141in}{-0.060141in}}%
\pgfusepath{stroke,fill}%
}%
\begin{pgfscope}%
\pgfsys@transformshift{0.423750in}{0.371875in}%
\pgfsys@useobject{currentmarker}{}%
\end{pgfscope}%
\end{pgfscope}%
\end{pgfpicture}%
\makeatother%
\endgroup%

  \caption{Feynman diagram at tree level for the electron \bremss\ (top) and pair
  production by gamma rays in the field of an atomic nucleus. The two only differ
  for the exchange of the top external legs.}
  \label{fig:em_shower_feynman}
\end{marginfigure}

As it turns out, the processes of electron \bremss\ and pair production are intimately
related to each other, as the underlying tree-level Feynman diagrams for the two
cases tranform into each other via the swap of a single external leg. More specifically,
the scattering matrix element for the two processes is identical, and the only
different is the phase-space term, due to the fact that the kinematic of the final
state is different. As a result, the mean free path for pair production for a
high-energy photon is related to the radiation length of the material
\begin{align}\label{eq:lambda_pair}
  \lambda_\text{pair} = \frac{9}{7}X_0
\end{align}
(i.e., it is $9/7$ of the scale length over which a high-energy electron
looses all but $1/e$ of its energy).

It should be noted that, while pair production is a destructive process
(the gamma-ray no longer exists afterwards), \bremss\ only degrades the
electron energy. As we shall see in a second, these two physical processes are
at the base of the development of electromagnetic showers.


\section{Electromagnetic showers}%
\label{sec:em_showers}

As explained in the previous section, high-energy electrons and photons
produce in matter secondary photons by \bremss\ and electron-positron
pairs by pair production. These secondaries, in turn, can produce other
particles with progressively lower energy and start an
\emph{electromagnetic shower} (or \emph{cascade}).
The process continue until the average energy of the electron component falls
below the critical energy of the material---at which point the rest of the
energy is released via ionization.

\begin{marginfigure}
  %% Creator: Matplotlib, PGF backend
%%
%% To include the figure in your LaTeX document, write
%%   \input{<filename>.pgf}
%%
%% Make sure the required packages are loaded in your preamble
%%   \usepackage{pgf}
%%
%% Also ensure that all the required font packages are loaded; for instance,
%% the lmodern package is sometimes necessary when using math font.
%%   \usepackage{lmodern}
%%
%% Figures using additional raster images can only be included by \input if
%% they are in the same directory as the main LaTeX file. For loading figures
%% from other directories you can use the `import` package
%%   \usepackage{import}
%%
%% and then include the figures with
%%   \import{<path to file>}{<filename>.pgf}
%%
%% Matplotlib used the following preamble
%%   \usepackage{fontspec}
%%   \setmainfont{DejaVuSerif.ttf}[Path=\detokenize{/usr/share/matplotlib/mpl-data/fonts/ttf/}]
%%   \setsansfont{DejaVuSans.ttf}[Path=\detokenize{/usr/share/matplotlib/mpl-data/fonts/ttf/}]
%%   \setmonofont{DejaVuSansMono.ttf}[Path=\detokenize{/usr/share/matplotlib/mpl-data/fonts/ttf/}]
%%
\begingroup%
\makeatletter%
\begin{pgfpicture}%
\pgfpathrectangle{\pgfpointorigin}{\pgfqpoint{1.950000in}{3.500000in}}%
\pgfusepath{use as bounding box, clip}%
\begin{pgfscope}%
\pgfsetbuttcap%
\pgfsetmiterjoin%
\definecolor{currentfill}{rgb}{1.000000,1.000000,1.000000}%
\pgfsetfillcolor{currentfill}%
\pgfsetlinewidth{0.000000pt}%
\definecolor{currentstroke}{rgb}{1.000000,1.000000,1.000000}%
\pgfsetstrokecolor{currentstroke}%
\pgfsetdash{}{0pt}%
\pgfpathmoveto{\pgfqpoint{0.000000in}{0.000000in}}%
\pgfpathlineto{\pgfqpoint{1.950000in}{0.000000in}}%
\pgfpathlineto{\pgfqpoint{1.950000in}{3.500000in}}%
\pgfpathlineto{\pgfqpoint{0.000000in}{3.500000in}}%
\pgfpathlineto{\pgfqpoint{0.000000in}{0.000000in}}%
\pgfpathclose%
\pgfusepath{fill}%
\end{pgfscope}%
\begin{pgfscope}%
\pgfpathrectangle{\pgfqpoint{0.148125in}{0.175000in}}{\pgfqpoint{1.653750in}{3.150000in}}%
\pgfusepath{clip}%
\pgfsetrectcap%
\pgfsetroundjoin%
\pgfsetlinewidth{1.003750pt}%
\definecolor{currentstroke}{rgb}{0.000000,0.000000,0.000000}%
\pgfsetstrokecolor{currentstroke}%
\pgfsetdash{}{0pt}%
\pgfpathmoveto{\pgfqpoint{0.975000in}{3.325000in}}%
\pgfpathlineto{\pgfqpoint{0.975000in}{2.734375in}}%
\pgfusepath{stroke}%
\end{pgfscope}%
\begin{pgfscope}%
\pgfpathrectangle{\pgfqpoint{0.148125in}{0.175000in}}{\pgfqpoint{1.653750in}{3.150000in}}%
\pgfusepath{clip}%
\pgfsetrectcap%
\pgfsetroundjoin%
\pgfsetlinewidth{1.003750pt}%
\definecolor{currentstroke}{rgb}{0.000000,0.000000,0.000000}%
\pgfsetstrokecolor{currentstroke}%
\pgfsetdash{}{0pt}%
\pgfpathmoveto{\pgfqpoint{0.975000in}{2.734375in}}%
\pgfpathlineto{\pgfqpoint{0.959923in}{2.734148in}}%
\pgfpathlineto{\pgfqpoint{0.954010in}{2.731238in}}%
\pgfpathlineto{\pgfqpoint{0.952802in}{2.728617in}}%
\pgfpathlineto{\pgfqpoint{0.953717in}{2.721154in}}%
\pgfpathlineto{\pgfqpoint{0.964527in}{2.693390in}}%
\pgfpathlineto{\pgfqpoint{0.964195in}{2.690185in}}%
\pgfpathlineto{\pgfqpoint{0.962689in}{2.687763in}}%
\pgfpathlineto{\pgfqpoint{0.960000in}{2.686129in}}%
\pgfpathlineto{\pgfqpoint{0.951579in}{2.684891in}}%
\pgfpathlineto{\pgfqpoint{0.926135in}{2.684723in}}%
\pgfpathlineto{\pgfqpoint{0.922887in}{2.683462in}}%
\pgfpathlineto{\pgfqpoint{0.920783in}{2.681438in}}%
\pgfpathlineto{\pgfqpoint{0.919872in}{2.678620in}}%
\pgfpathlineto{\pgfqpoint{0.921279in}{2.670829in}}%
\pgfpathlineto{\pgfqpoint{0.931737in}{2.643300in}}%
\pgfpathlineto{\pgfqpoint{0.931118in}{2.640286in}}%
\pgfpathlineto{\pgfqpoint{0.929313in}{2.638064in}}%
\pgfpathlineto{\pgfqpoint{0.922328in}{2.635868in}}%
\pgfpathlineto{\pgfqpoint{0.906608in}{2.636069in}}%
\pgfpathlineto{\pgfqpoint{0.896422in}{2.636007in}}%
\pgfpathlineto{\pgfqpoint{0.889437in}{2.633811in}}%
\pgfpathlineto{\pgfqpoint{0.887632in}{2.631589in}}%
\pgfpathlineto{\pgfqpoint{0.887013in}{2.628575in}}%
\pgfpathlineto{\pgfqpoint{0.888887in}{2.620474in}}%
\pgfpathlineto{\pgfqpoint{0.898658in}{2.596828in}}%
\pgfpathlineto{\pgfqpoint{0.897967in}{2.590437in}}%
\pgfpathlineto{\pgfqpoint{0.895863in}{2.588413in}}%
\pgfpathlineto{\pgfqpoint{0.888374in}{2.586553in}}%
\pgfpathlineto{\pgfqpoint{0.858750in}{2.585746in}}%
\pgfpathlineto{\pgfqpoint{0.856061in}{2.584112in}}%
\pgfpathlineto{\pgfqpoint{0.854555in}{2.581690in}}%
\pgfpathlineto{\pgfqpoint{0.854223in}{2.578485in}}%
\pgfpathlineto{\pgfqpoint{0.856535in}{2.570091in}}%
\pgfpathlineto{\pgfqpoint{0.865999in}{2.546650in}}%
\pgfpathlineto{\pgfqpoint{0.865948in}{2.543258in}}%
\pgfpathlineto{\pgfqpoint{0.864740in}{2.540637in}}%
\pgfpathlineto{\pgfqpoint{0.862342in}{2.538810in}}%
\pgfpathlineto{\pgfqpoint{0.854373in}{2.537271in}}%
\pgfpathlineto{\pgfqpoint{0.825158in}{2.536190in}}%
\pgfpathlineto{\pgfqpoint{0.822760in}{2.534363in}}%
\pgfpathlineto{\pgfqpoint{0.821552in}{2.531742in}}%
\pgfpathlineto{\pgfqpoint{0.822467in}{2.524279in}}%
\pgfpathlineto{\pgfqpoint{0.833277in}{2.496515in}}%
\pgfpathlineto{\pgfqpoint{0.832945in}{2.493310in}}%
\pgfpathlineto{\pgfqpoint{0.831439in}{2.490888in}}%
\pgfpathlineto{\pgfqpoint{0.828750in}{2.489254in}}%
\pgfpathlineto{\pgfqpoint{0.820329in}{2.488016in}}%
\pgfpathlineto{\pgfqpoint{0.794885in}{2.487848in}}%
\pgfpathlineto{\pgfqpoint{0.791637in}{2.486587in}}%
\pgfpathlineto{\pgfqpoint{0.789533in}{2.484563in}}%
\pgfpathlineto{\pgfqpoint{0.788622in}{2.481745in}}%
\pgfpathlineto{\pgfqpoint{0.790029in}{2.473954in}}%
\pgfpathlineto{\pgfqpoint{0.800487in}{2.446425in}}%
\pgfpathlineto{\pgfqpoint{0.799868in}{2.443411in}}%
\pgfpathlineto{\pgfqpoint{0.798063in}{2.441189in}}%
\pgfpathlineto{\pgfqpoint{0.791078in}{2.438993in}}%
\pgfpathlineto{\pgfqpoint{0.775358in}{2.439194in}}%
\pgfpathlineto{\pgfqpoint{0.765172in}{2.439132in}}%
\pgfpathlineto{\pgfqpoint{0.758187in}{2.436936in}}%
\pgfpathlineto{\pgfqpoint{0.756382in}{2.434714in}}%
\pgfpathlineto{\pgfqpoint{0.755763in}{2.431700in}}%
\pgfpathlineto{\pgfqpoint{0.757637in}{2.423599in}}%
\pgfpathlineto{\pgfqpoint{0.767408in}{2.399953in}}%
\pgfpathlineto{\pgfqpoint{0.766717in}{2.393562in}}%
\pgfpathlineto{\pgfqpoint{0.764613in}{2.391538in}}%
\pgfpathlineto{\pgfqpoint{0.757124in}{2.389678in}}%
\pgfpathlineto{\pgfqpoint{0.727500in}{2.388871in}}%
\pgfpathlineto{\pgfqpoint{0.724811in}{2.387237in}}%
\pgfpathlineto{\pgfqpoint{0.723305in}{2.384815in}}%
\pgfpathlineto{\pgfqpoint{0.722973in}{2.381610in}}%
\pgfpathlineto{\pgfqpoint{0.725285in}{2.373216in}}%
\pgfpathlineto{\pgfqpoint{0.734749in}{2.349775in}}%
\pgfpathlineto{\pgfqpoint{0.734698in}{2.346383in}}%
\pgfpathlineto{\pgfqpoint{0.733490in}{2.343762in}}%
\pgfpathlineto{\pgfqpoint{0.731092in}{2.341935in}}%
\pgfpathlineto{\pgfqpoint{0.723123in}{2.340396in}}%
\pgfpathlineto{\pgfqpoint{0.693908in}{2.339315in}}%
\pgfpathlineto{\pgfqpoint{0.691510in}{2.337488in}}%
\pgfpathlineto{\pgfqpoint{0.690302in}{2.334867in}}%
\pgfpathlineto{\pgfqpoint{0.691217in}{2.327404in}}%
\pgfpathlineto{\pgfqpoint{0.702027in}{2.299640in}}%
\pgfpathlineto{\pgfqpoint{0.701695in}{2.296435in}}%
\pgfpathlineto{\pgfqpoint{0.700189in}{2.294013in}}%
\pgfpathlineto{\pgfqpoint{0.697500in}{2.292379in}}%
\pgfpathlineto{\pgfqpoint{0.689079in}{2.291141in}}%
\pgfpathlineto{\pgfqpoint{0.663635in}{2.290973in}}%
\pgfpathlineto{\pgfqpoint{0.660387in}{2.289712in}}%
\pgfpathlineto{\pgfqpoint{0.658283in}{2.287688in}}%
\pgfpathlineto{\pgfqpoint{0.657372in}{2.284870in}}%
\pgfpathlineto{\pgfqpoint{0.658779in}{2.277079in}}%
\pgfpathlineto{\pgfqpoint{0.669237in}{2.249550in}}%
\pgfpathlineto{\pgfqpoint{0.668618in}{2.246536in}}%
\pgfpathlineto{\pgfqpoint{0.666813in}{2.244314in}}%
\pgfpathlineto{\pgfqpoint{0.659828in}{2.242118in}}%
\pgfpathlineto{\pgfqpoint{0.644108in}{2.242319in}}%
\pgfpathlineto{\pgfqpoint{0.633922in}{2.242257in}}%
\pgfpathlineto{\pgfqpoint{0.626937in}{2.240061in}}%
\pgfpathlineto{\pgfqpoint{0.625132in}{2.237839in}}%
\pgfpathlineto{\pgfqpoint{0.624513in}{2.234825in}}%
\pgfpathlineto{\pgfqpoint{0.626387in}{2.226724in}}%
\pgfpathlineto{\pgfqpoint{0.636158in}{2.203078in}}%
\pgfpathlineto{\pgfqpoint{0.635467in}{2.196687in}}%
\pgfpathlineto{\pgfqpoint{0.633363in}{2.194663in}}%
\pgfpathlineto{\pgfqpoint{0.625874in}{2.192803in}}%
\pgfpathlineto{\pgfqpoint{0.596250in}{2.191996in}}%
\pgfpathlineto{\pgfqpoint{0.593561in}{2.190362in}}%
\pgfpathlineto{\pgfqpoint{0.592055in}{2.187940in}}%
\pgfpathlineto{\pgfqpoint{0.591723in}{2.184735in}}%
\pgfpathlineto{\pgfqpoint{0.594035in}{2.176341in}}%
\pgfpathlineto{\pgfqpoint{0.603499in}{2.152900in}}%
\pgfpathlineto{\pgfqpoint{0.603448in}{2.149508in}}%
\pgfpathlineto{\pgfqpoint{0.602240in}{2.146887in}}%
\pgfpathlineto{\pgfqpoint{0.599842in}{2.145060in}}%
\pgfpathlineto{\pgfqpoint{0.591873in}{2.143521in}}%
\pgfpathlineto{\pgfqpoint{0.581250in}{2.143750in}}%
\pgfpathlineto{\pgfqpoint{0.581250in}{2.143750in}}%
\pgfusepath{stroke}%
\end{pgfscope}%
\begin{pgfscope}%
\pgfpathrectangle{\pgfqpoint{0.148125in}{0.175000in}}{\pgfqpoint{1.653750in}{3.150000in}}%
\pgfusepath{clip}%
\pgfsetrectcap%
\pgfsetroundjoin%
\pgfsetlinewidth{1.003750pt}%
\definecolor{currentstroke}{rgb}{0.000000,0.000000,0.000000}%
\pgfsetstrokecolor{currentstroke}%
\pgfsetdash{}{0pt}%
\pgfpathmoveto{\pgfqpoint{0.975000in}{2.734375in}}%
\pgfpathlineto{\pgfqpoint{1.368750in}{2.143750in}}%
\pgfusepath{stroke}%
\end{pgfscope}%
\begin{pgfscope}%
\pgfpathrectangle{\pgfqpoint{0.148125in}{0.175000in}}{\pgfqpoint{1.653750in}{3.150000in}}%
\pgfusepath{clip}%
\pgfsetbuttcap%
\pgfsetroundjoin%
\pgfsetlinewidth{1.003750pt}%
\definecolor{currentstroke}{rgb}{0.827451,0.827451,0.827451}%
\pgfsetstrokecolor{currentstroke}%
\pgfsetdash{{3.700000pt}{1.600000pt}}{0.000000pt}%
\pgfpathmoveto{\pgfqpoint{0.148125in}{2.734375in}}%
\pgfpathlineto{\pgfqpoint{1.801875in}{2.734375in}}%
\pgfusepath{stroke}%
\end{pgfscope}%
\begin{pgfscope}%
\pgfpathrectangle{\pgfqpoint{0.148125in}{0.175000in}}{\pgfqpoint{1.653750in}{3.150000in}}%
\pgfusepath{clip}%
\pgfsetrectcap%
\pgfsetroundjoin%
\pgfsetlinewidth{1.003750pt}%
\definecolor{currentstroke}{rgb}{0.000000,0.000000,0.000000}%
\pgfsetstrokecolor{currentstroke}%
\pgfsetdash{}{0pt}%
\pgfpathmoveto{\pgfqpoint{0.581250in}{2.143750in}}%
\pgfpathlineto{\pgfqpoint{0.384375in}{1.553125in}}%
\pgfusepath{stroke}%
\end{pgfscope}%
\begin{pgfscope}%
\pgfpathrectangle{\pgfqpoint{0.148125in}{0.175000in}}{\pgfqpoint{1.653750in}{3.150000in}}%
\pgfusepath{clip}%
\pgfsetrectcap%
\pgfsetroundjoin%
\pgfsetlinewidth{1.003750pt}%
\definecolor{currentstroke}{rgb}{0.000000,0.000000,0.000000}%
\pgfsetstrokecolor{currentstroke}%
\pgfsetdash{}{0pt}%
\pgfpathmoveto{\pgfqpoint{0.581250in}{2.143750in}}%
\pgfpathlineto{\pgfqpoint{0.778125in}{1.553125in}}%
\pgfusepath{stroke}%
\end{pgfscope}%
\begin{pgfscope}%
\pgfpathrectangle{\pgfqpoint{0.148125in}{0.175000in}}{\pgfqpoint{1.653750in}{3.150000in}}%
\pgfusepath{clip}%
\pgfsetrectcap%
\pgfsetroundjoin%
\pgfsetlinewidth{1.003750pt}%
\definecolor{currentstroke}{rgb}{0.000000,0.000000,0.000000}%
\pgfsetstrokecolor{currentstroke}%
\pgfsetdash{}{0pt}%
\pgfpathmoveto{\pgfqpoint{1.368750in}{2.143750in}}%
\pgfpathlineto{\pgfqpoint{1.355325in}{2.140318in}}%
\pgfpathlineto{\pgfqpoint{1.350104in}{2.136788in}}%
\pgfpathlineto{\pgfqpoint{1.349131in}{2.134476in}}%
\pgfpathlineto{\pgfqpoint{1.350528in}{2.128740in}}%
\pgfpathlineto{\pgfqpoint{1.355498in}{2.121812in}}%
\pgfpathlineto{\pgfqpoint{1.369637in}{2.103921in}}%
\pgfpathlineto{\pgfqpoint{1.370250in}{2.098446in}}%
\pgfpathlineto{\pgfqpoint{1.368862in}{2.096273in}}%
\pgfpathlineto{\pgfqpoint{1.362921in}{2.092982in}}%
\pgfpathlineto{\pgfqpoint{1.344253in}{2.088663in}}%
\pgfpathlineto{\pgfqpoint{1.336183in}{2.086082in}}%
\pgfpathlineto{\pgfqpoint{1.331720in}{2.082298in}}%
\pgfpathlineto{\pgfqpoint{1.331164in}{2.079848in}}%
\pgfpathlineto{\pgfqpoint{1.333314in}{2.073860in}}%
\pgfpathlineto{\pgfqpoint{1.342056in}{2.063040in}}%
\pgfpathlineto{\pgfqpoint{1.350675in}{2.052260in}}%
\pgfpathlineto{\pgfqpoint{1.352640in}{2.046334in}}%
\pgfpathlineto{\pgfqpoint{1.351979in}{2.043919in}}%
\pgfpathlineto{\pgfqpoint{1.347324in}{2.040199in}}%
\pgfpathlineto{\pgfqpoint{1.339130in}{2.037659in}}%
\pgfpathlineto{\pgfqpoint{1.320537in}{2.033315in}}%
\pgfpathlineto{\pgfqpoint{1.314772in}{2.029966in}}%
\pgfpathlineto{\pgfqpoint{1.313487in}{2.027758in}}%
\pgfpathlineto{\pgfqpoint{1.313346in}{2.025170in}}%
\pgfpathlineto{\pgfqpoint{1.316211in}{2.018944in}}%
\pgfpathlineto{\pgfqpoint{1.334571in}{1.994375in}}%
\pgfpathlineto{\pgfqpoint{1.334635in}{1.991718in}}%
\pgfpathlineto{\pgfqpoint{1.333557in}{1.989442in}}%
\pgfpathlineto{\pgfqpoint{1.328153in}{1.985972in}}%
\pgfpathlineto{\pgfqpoint{1.314608in}{1.982580in}}%
\pgfpathlineto{\pgfqpoint{1.301310in}{1.979107in}}%
\pgfpathlineto{\pgfqpoint{1.296275in}{1.975514in}}%
\pgfpathlineto{\pgfqpoint{1.295406in}{1.973168in}}%
\pgfpathlineto{\pgfqpoint{1.296995in}{1.967367in}}%
\pgfpathlineto{\pgfqpoint{1.302086in}{1.960399in}}%
\pgfpathlineto{\pgfqpoint{1.314313in}{1.945781in}}%
\pgfpathlineto{\pgfqpoint{1.316826in}{1.939672in}}%
\pgfpathlineto{\pgfqpoint{1.316478in}{1.937153in}}%
\pgfpathlineto{\pgfqpoint{1.312406in}{1.933239in}}%
\pgfpathlineto{\pgfqpoint{1.304596in}{1.930571in}}%
\pgfpathlineto{\pgfqpoint{1.282197in}{1.924860in}}%
\pgfpathlineto{\pgfqpoint{1.277929in}{1.921012in}}%
\pgfpathlineto{\pgfqpoint{1.277477in}{1.918527in}}%
\pgfpathlineto{\pgfqpoint{1.279809in}{1.912478in}}%
\pgfpathlineto{\pgfqpoint{1.288668in}{1.901619in}}%
\pgfpathlineto{\pgfqpoint{1.297156in}{1.890883in}}%
\pgfpathlineto{\pgfqpoint{1.298934in}{1.885019in}}%
\pgfpathlineto{\pgfqpoint{1.298169in}{1.882638in}}%
\pgfpathlineto{\pgfqpoint{1.293324in}{1.878983in}}%
\pgfpathlineto{\pgfqpoint{1.280159in}{1.875464in}}%
\pgfpathlineto{\pgfqpoint{1.266501in}{1.872110in}}%
\pgfpathlineto{\pgfqpoint{1.260915in}{1.868701in}}%
\pgfpathlineto{\pgfqpoint{1.259734in}{1.866459in}}%
\pgfpathlineto{\pgfqpoint{1.260742in}{1.860852in}}%
\pgfpathlineto{\pgfqpoint{1.265456in}{1.854010in}}%
\pgfpathlineto{\pgfqpoint{1.279883in}{1.836023in}}%
\pgfpathlineto{\pgfqpoint{1.280891in}{1.830416in}}%
\pgfpathlineto{\pgfqpoint{1.279710in}{1.828174in}}%
\pgfpathlineto{\pgfqpoint{1.274124in}{1.824765in}}%
\pgfpathlineto{\pgfqpoint{1.260466in}{1.821411in}}%
\pgfpathlineto{\pgfqpoint{1.247301in}{1.817892in}}%
\pgfpathlineto{\pgfqpoint{1.242456in}{1.814237in}}%
\pgfpathlineto{\pgfqpoint{1.241691in}{1.811856in}}%
\pgfpathlineto{\pgfqpoint{1.243469in}{1.805992in}}%
\pgfpathlineto{\pgfqpoint{1.251957in}{1.795256in}}%
\pgfpathlineto{\pgfqpoint{1.260816in}{1.784397in}}%
\pgfpathlineto{\pgfqpoint{1.263148in}{1.778348in}}%
\pgfpathlineto{\pgfqpoint{1.262696in}{1.775863in}}%
\pgfpathlineto{\pgfqpoint{1.258428in}{1.772015in}}%
\pgfpathlineto{\pgfqpoint{1.250486in}{1.769391in}}%
\pgfpathlineto{\pgfqpoint{1.228219in}{1.763636in}}%
\pgfpathlineto{\pgfqpoint{1.224147in}{1.759722in}}%
\pgfpathlineto{\pgfqpoint{1.223799in}{1.757203in}}%
\pgfpathlineto{\pgfqpoint{1.226312in}{1.751094in}}%
\pgfpathlineto{\pgfqpoint{1.235279in}{1.740198in}}%
\pgfpathlineto{\pgfqpoint{1.243630in}{1.729508in}}%
\pgfpathlineto{\pgfqpoint{1.245219in}{1.723707in}}%
\pgfpathlineto{\pgfqpoint{1.244350in}{1.721361in}}%
\pgfpathlineto{\pgfqpoint{1.239315in}{1.717768in}}%
\pgfpathlineto{\pgfqpoint{1.226017in}{1.714295in}}%
\pgfpathlineto{\pgfqpoint{1.212472in}{1.710903in}}%
\pgfpathlineto{\pgfqpoint{1.207068in}{1.707433in}}%
\pgfpathlineto{\pgfqpoint{1.205990in}{1.705157in}}%
\pgfpathlineto{\pgfqpoint{1.207193in}{1.699485in}}%
\pgfpathlineto{\pgfqpoint{1.212037in}{1.692599in}}%
\pgfpathlineto{\pgfqpoint{1.226327in}{1.674658in}}%
\pgfpathlineto{\pgfqpoint{1.227138in}{1.669117in}}%
\pgfpathlineto{\pgfqpoint{1.225853in}{1.666909in}}%
\pgfpathlineto{\pgfqpoint{1.220088in}{1.663560in}}%
\pgfpathlineto{\pgfqpoint{1.206325in}{1.660241in}}%
\pgfpathlineto{\pgfqpoint{1.193301in}{1.656676in}}%
\pgfpathlineto{\pgfqpoint{1.188646in}{1.652956in}}%
\pgfpathlineto{\pgfqpoint{1.187985in}{1.650541in}}%
\pgfpathlineto{\pgfqpoint{1.189950in}{1.644615in}}%
\pgfpathlineto{\pgfqpoint{1.198569in}{1.633835in}}%
\pgfpathlineto{\pgfqpoint{1.207311in}{1.623015in}}%
\pgfpathlineto{\pgfqpoint{1.209461in}{1.617027in}}%
\pgfpathlineto{\pgfqpoint{1.208905in}{1.614577in}}%
\pgfpathlineto{\pgfqpoint{1.204442in}{1.610793in}}%
\pgfpathlineto{\pgfqpoint{1.196372in}{1.608212in}}%
\pgfpathlineto{\pgfqpoint{1.174250in}{1.602409in}}%
\pgfpathlineto{\pgfqpoint{1.170375in}{1.598429in}}%
\pgfpathlineto{\pgfqpoint{1.170130in}{1.595875in}}%
\pgfpathlineto{\pgfqpoint{1.172821in}{1.589707in}}%
\pgfpathlineto{\pgfqpoint{1.181890in}{1.578778in}}%
\pgfpathlineto{\pgfqpoint{1.190097in}{1.568135in}}%
\pgfpathlineto{\pgfqpoint{1.191494in}{1.562399in}}%
\pgfpathlineto{\pgfqpoint{1.190521in}{1.560087in}}%
\pgfpathlineto{\pgfqpoint{1.185300in}{1.556557in}}%
\pgfpathlineto{\pgfqpoint{1.171875in}{1.553125in}}%
\pgfpathlineto{\pgfqpoint{1.171875in}{1.553125in}}%
\pgfusepath{stroke}%
\end{pgfscope}%
\begin{pgfscope}%
\pgfpathrectangle{\pgfqpoint{0.148125in}{0.175000in}}{\pgfqpoint{1.653750in}{3.150000in}}%
\pgfusepath{clip}%
\pgfsetrectcap%
\pgfsetroundjoin%
\pgfsetlinewidth{1.003750pt}%
\definecolor{currentstroke}{rgb}{0.000000,0.000000,0.000000}%
\pgfsetstrokecolor{currentstroke}%
\pgfsetdash{}{0pt}%
\pgfpathmoveto{\pgfqpoint{1.368750in}{2.143750in}}%
\pgfpathlineto{\pgfqpoint{1.565625in}{1.553125in}}%
\pgfusepath{stroke}%
\end{pgfscope}%
\begin{pgfscope}%
\pgfpathrectangle{\pgfqpoint{0.148125in}{0.175000in}}{\pgfqpoint{1.653750in}{3.150000in}}%
\pgfusepath{clip}%
\pgfsetbuttcap%
\pgfsetroundjoin%
\pgfsetlinewidth{1.003750pt}%
\definecolor{currentstroke}{rgb}{0.827451,0.827451,0.827451}%
\pgfsetstrokecolor{currentstroke}%
\pgfsetdash{{3.700000pt}{1.600000pt}}{0.000000pt}%
\pgfpathmoveto{\pgfqpoint{0.148125in}{2.143750in}}%
\pgfpathlineto{\pgfqpoint{1.801875in}{2.143750in}}%
\pgfusepath{stroke}%
\end{pgfscope}%
\begin{pgfscope}%
\pgfpathrectangle{\pgfqpoint{0.148125in}{0.175000in}}{\pgfqpoint{1.653750in}{3.150000in}}%
\pgfusepath{clip}%
\pgfsetrectcap%
\pgfsetroundjoin%
\pgfsetlinewidth{1.003750pt}%
\definecolor{currentstroke}{rgb}{0.000000,0.000000,0.000000}%
\pgfsetstrokecolor{currentstroke}%
\pgfsetdash{}{0pt}%
\pgfpathmoveto{\pgfqpoint{0.384375in}{1.553125in}}%
\pgfpathlineto{\pgfqpoint{0.372480in}{1.547865in}}%
\pgfpathlineto{\pgfqpoint{0.367529in}{1.543839in}}%
\pgfpathlineto{\pgfqpoint{0.366263in}{1.539172in}}%
\pgfpathlineto{\pgfqpoint{0.368792in}{1.533843in}}%
\pgfpathlineto{\pgfqpoint{0.377660in}{1.524967in}}%
\pgfpathlineto{\pgfqpoint{0.389880in}{1.513064in}}%
\pgfpathlineto{\pgfqpoint{0.392232in}{1.507766in}}%
\pgfpathlineto{\pgfqpoint{0.390771in}{1.503133in}}%
\pgfpathlineto{\pgfqpoint{0.385656in}{1.499136in}}%
\pgfpathlineto{\pgfqpoint{0.369384in}{1.492194in}}%
\pgfpathlineto{\pgfqpoint{0.359058in}{1.486660in}}%
\pgfpathlineto{\pgfqpoint{0.356062in}{1.482294in}}%
\pgfpathlineto{\pgfqpoint{0.356932in}{1.477255in}}%
\pgfpathlineto{\pgfqpoint{0.361247in}{1.471615in}}%
\pgfpathlineto{\pgfqpoint{0.381334in}{1.451010in}}%
\pgfpathlineto{\pgfqpoint{0.381628in}{1.446071in}}%
\pgfpathlineto{\pgfqpoint{0.378076in}{1.441802in}}%
\pgfpathlineto{\pgfqpoint{0.371357in}{1.438084in}}%
\pgfpathlineto{\pgfqpoint{0.348536in}{1.427394in}}%
\pgfpathlineto{\pgfqpoint{0.345728in}{1.422996in}}%
\pgfpathlineto{\pgfqpoint{0.346789in}{1.417923in}}%
\pgfpathlineto{\pgfqpoint{0.351248in}{1.412259in}}%
\pgfpathlineto{\pgfqpoint{0.371155in}{1.391685in}}%
\pgfpathlineto{\pgfqpoint{0.371255in}{1.386780in}}%
\pgfpathlineto{\pgfqpoint{0.367521in}{1.382542in}}%
\pgfpathlineto{\pgfqpoint{0.356536in}{1.377123in}}%
\pgfpathlineto{\pgfqpoint{0.340679in}{1.370109in}}%
\pgfpathlineto{\pgfqpoint{0.336235in}{1.365995in}}%
\pgfpathlineto{\pgfqpoint{0.335556in}{1.361225in}}%
\pgfpathlineto{\pgfqpoint{0.338604in}{1.355807in}}%
\pgfpathlineto{\pgfqpoint{0.347853in}{1.346864in}}%
\pgfpathlineto{\pgfqpoint{0.357422in}{1.337866in}}%
\pgfpathlineto{\pgfqpoint{0.360966in}{1.332361in}}%
\pgfpathlineto{\pgfqpoint{0.360872in}{1.327490in}}%
\pgfpathlineto{\pgfqpoint{0.356958in}{1.323284in}}%
\pgfpathlineto{\pgfqpoint{0.345823in}{1.317891in}}%
\pgfpathlineto{\pgfqpoint{0.330093in}{1.310854in}}%
\pgfpathlineto{\pgfqpoint{0.325823in}{1.306710in}}%
\pgfpathlineto{\pgfqpoint{0.325339in}{1.301906in}}%
\pgfpathlineto{\pgfqpoint{0.328555in}{1.296459in}}%
\pgfpathlineto{\pgfqpoint{0.337918in}{1.287496in}}%
\pgfpathlineto{\pgfqpoint{0.347387in}{1.278516in}}%
\pgfpathlineto{\pgfqpoint{0.350769in}{1.273039in}}%
\pgfpathlineto{\pgfqpoint{0.350480in}{1.268202in}}%
\pgfpathlineto{\pgfqpoint{0.346387in}{1.264026in}}%
\pgfpathlineto{\pgfqpoint{0.335110in}{1.258658in}}%
\pgfpathlineto{\pgfqpoint{0.319514in}{1.251599in}}%
\pgfpathlineto{\pgfqpoint{0.315421in}{1.247423in}}%
\pgfpathlineto{\pgfqpoint{0.315132in}{1.242586in}}%
\pgfpathlineto{\pgfqpoint{0.318514in}{1.237109in}}%
\pgfpathlineto{\pgfqpoint{0.327983in}{1.228129in}}%
\pgfpathlineto{\pgfqpoint{0.337346in}{1.219166in}}%
\pgfpathlineto{\pgfqpoint{0.340562in}{1.213719in}}%
\pgfpathlineto{\pgfqpoint{0.340078in}{1.208915in}}%
\pgfpathlineto{\pgfqpoint{0.335808in}{1.204771in}}%
\pgfpathlineto{\pgfqpoint{0.324394in}{1.199427in}}%
\pgfpathlineto{\pgfqpoint{0.308943in}{1.192341in}}%
\pgfpathlineto{\pgfqpoint{0.305029in}{1.188135in}}%
\pgfpathlineto{\pgfqpoint{0.304935in}{1.183264in}}%
\pgfpathlineto{\pgfqpoint{0.308479in}{1.177759in}}%
\pgfpathlineto{\pgfqpoint{0.321488in}{1.165718in}}%
\pgfpathlineto{\pgfqpoint{0.329247in}{1.157035in}}%
\pgfpathlineto{\pgfqpoint{0.330497in}{1.151930in}}%
\pgfpathlineto{\pgfqpoint{0.327878in}{1.147498in}}%
\pgfpathlineto{\pgfqpoint{0.321839in}{1.143661in}}%
\pgfpathlineto{\pgfqpoint{0.296058in}{1.131044in}}%
\pgfpathlineto{\pgfqpoint{0.294208in}{1.126478in}}%
\pgfpathlineto{\pgfqpoint{0.296200in}{1.121243in}}%
\pgfpathlineto{\pgfqpoint{0.301330in}{1.115462in}}%
\pgfpathlineto{\pgfqpoint{0.319112in}{1.097702in}}%
\pgfpathlineto{\pgfqpoint{0.320173in}{1.092629in}}%
\pgfpathlineto{\pgfqpoint{0.317365in}{1.088231in}}%
\pgfpathlineto{\pgfqpoint{0.311183in}{1.084419in}}%
\pgfpathlineto{\pgfqpoint{0.287825in}{1.073823in}}%
\pgfpathlineto{\pgfqpoint{0.284273in}{1.069554in}}%
\pgfpathlineto{\pgfqpoint{0.284567in}{1.064615in}}%
\pgfpathlineto{\pgfqpoint{0.288428in}{1.059055in}}%
\pgfpathlineto{\pgfqpoint{0.308969in}{1.038370in}}%
\pgfpathlineto{\pgfqpoint{0.309839in}{1.033331in}}%
\pgfpathlineto{\pgfqpoint{0.306843in}{1.028965in}}%
\pgfpathlineto{\pgfqpoint{0.300521in}{1.025178in}}%
\pgfpathlineto{\pgfqpoint{0.277278in}{1.014562in}}%
\pgfpathlineto{\pgfqpoint{0.273910in}{1.010261in}}%
\pgfpathlineto{\pgfqpoint{0.274397in}{1.005288in}}%
\pgfpathlineto{\pgfqpoint{0.278412in}{0.999701in}}%
\pgfpathlineto{\pgfqpoint{0.298816in}{0.979040in}}%
\pgfpathlineto{\pgfqpoint{0.299496in}{0.974034in}}%
\pgfpathlineto{\pgfqpoint{0.296313in}{0.969701in}}%
\pgfpathlineto{\pgfqpoint{0.289855in}{0.965938in}}%
\pgfpathlineto{\pgfqpoint{0.281526in}{0.962500in}}%
\pgfpathlineto{\pgfqpoint{0.281526in}{0.962500in}}%
\pgfusepath{stroke}%
\end{pgfscope}%
\begin{pgfscope}%
\pgfpathrectangle{\pgfqpoint{0.148125in}{0.175000in}}{\pgfqpoint{1.653750in}{3.150000in}}%
\pgfusepath{clip}%
\pgfsetrectcap%
\pgfsetroundjoin%
\pgfsetlinewidth{1.003750pt}%
\definecolor{currentstroke}{rgb}{0.000000,0.000000,0.000000}%
\pgfsetstrokecolor{currentstroke}%
\pgfsetdash{}{0pt}%
\pgfpathmoveto{\pgfqpoint{0.384375in}{1.553125in}}%
\pgfpathlineto{\pgfqpoint{0.487224in}{0.962500in}}%
\pgfusepath{stroke}%
\end{pgfscope}%
\begin{pgfscope}%
\pgfpathrectangle{\pgfqpoint{0.148125in}{0.175000in}}{\pgfqpoint{1.653750in}{3.150000in}}%
\pgfusepath{clip}%
\pgfsetrectcap%
\pgfsetroundjoin%
\pgfsetlinewidth{1.003750pt}%
\definecolor{currentstroke}{rgb}{0.000000,0.000000,0.000000}%
\pgfsetstrokecolor{currentstroke}%
\pgfsetdash{}{0pt}%
\pgfpathmoveto{\pgfqpoint{0.778125in}{1.553125in}}%
\pgfpathlineto{\pgfqpoint{0.766230in}{1.547865in}}%
\pgfpathlineto{\pgfqpoint{0.761279in}{1.543839in}}%
\pgfpathlineto{\pgfqpoint{0.760013in}{1.539172in}}%
\pgfpathlineto{\pgfqpoint{0.762542in}{1.533843in}}%
\pgfpathlineto{\pgfqpoint{0.771410in}{1.524967in}}%
\pgfpathlineto{\pgfqpoint{0.783630in}{1.513064in}}%
\pgfpathlineto{\pgfqpoint{0.785982in}{1.507766in}}%
\pgfpathlineto{\pgfqpoint{0.784521in}{1.503133in}}%
\pgfpathlineto{\pgfqpoint{0.779406in}{1.499136in}}%
\pgfpathlineto{\pgfqpoint{0.763134in}{1.492194in}}%
\pgfpathlineto{\pgfqpoint{0.752808in}{1.486660in}}%
\pgfpathlineto{\pgfqpoint{0.749812in}{1.482294in}}%
\pgfpathlineto{\pgfqpoint{0.750682in}{1.477255in}}%
\pgfpathlineto{\pgfqpoint{0.754997in}{1.471615in}}%
\pgfpathlineto{\pgfqpoint{0.775084in}{1.451010in}}%
\pgfpathlineto{\pgfqpoint{0.775378in}{1.446071in}}%
\pgfpathlineto{\pgfqpoint{0.771826in}{1.441802in}}%
\pgfpathlineto{\pgfqpoint{0.765107in}{1.438084in}}%
\pgfpathlineto{\pgfqpoint{0.742286in}{1.427394in}}%
\pgfpathlineto{\pgfqpoint{0.739478in}{1.422996in}}%
\pgfpathlineto{\pgfqpoint{0.740539in}{1.417923in}}%
\pgfpathlineto{\pgfqpoint{0.744998in}{1.412259in}}%
\pgfpathlineto{\pgfqpoint{0.764905in}{1.391685in}}%
\pgfpathlineto{\pgfqpoint{0.765005in}{1.386780in}}%
\pgfpathlineto{\pgfqpoint{0.761271in}{1.382542in}}%
\pgfpathlineto{\pgfqpoint{0.750286in}{1.377123in}}%
\pgfpathlineto{\pgfqpoint{0.734429in}{1.370109in}}%
\pgfpathlineto{\pgfqpoint{0.729985in}{1.365995in}}%
\pgfpathlineto{\pgfqpoint{0.729306in}{1.361225in}}%
\pgfpathlineto{\pgfqpoint{0.732354in}{1.355807in}}%
\pgfpathlineto{\pgfqpoint{0.741603in}{1.346864in}}%
\pgfpathlineto{\pgfqpoint{0.751172in}{1.337866in}}%
\pgfpathlineto{\pgfqpoint{0.754716in}{1.332361in}}%
\pgfpathlineto{\pgfqpoint{0.754622in}{1.327490in}}%
\pgfpathlineto{\pgfqpoint{0.750708in}{1.323284in}}%
\pgfpathlineto{\pgfqpoint{0.739573in}{1.317891in}}%
\pgfpathlineto{\pgfqpoint{0.723843in}{1.310854in}}%
\pgfpathlineto{\pgfqpoint{0.719573in}{1.306710in}}%
\pgfpathlineto{\pgfqpoint{0.719089in}{1.301906in}}%
\pgfpathlineto{\pgfqpoint{0.722305in}{1.296459in}}%
\pgfpathlineto{\pgfqpoint{0.731668in}{1.287496in}}%
\pgfpathlineto{\pgfqpoint{0.741137in}{1.278516in}}%
\pgfpathlineto{\pgfqpoint{0.744519in}{1.273039in}}%
\pgfpathlineto{\pgfqpoint{0.744230in}{1.268202in}}%
\pgfpathlineto{\pgfqpoint{0.740137in}{1.264026in}}%
\pgfpathlineto{\pgfqpoint{0.728860in}{1.258658in}}%
\pgfpathlineto{\pgfqpoint{0.713264in}{1.251599in}}%
\pgfpathlineto{\pgfqpoint{0.709171in}{1.247423in}}%
\pgfpathlineto{\pgfqpoint{0.708882in}{1.242586in}}%
\pgfpathlineto{\pgfqpoint{0.712264in}{1.237109in}}%
\pgfpathlineto{\pgfqpoint{0.721733in}{1.228129in}}%
\pgfpathlineto{\pgfqpoint{0.731096in}{1.219166in}}%
\pgfpathlineto{\pgfqpoint{0.734312in}{1.213719in}}%
\pgfpathlineto{\pgfqpoint{0.733828in}{1.208915in}}%
\pgfpathlineto{\pgfqpoint{0.729558in}{1.204771in}}%
\pgfpathlineto{\pgfqpoint{0.718144in}{1.199427in}}%
\pgfpathlineto{\pgfqpoint{0.702693in}{1.192341in}}%
\pgfpathlineto{\pgfqpoint{0.698779in}{1.188135in}}%
\pgfpathlineto{\pgfqpoint{0.698685in}{1.183264in}}%
\pgfpathlineto{\pgfqpoint{0.702229in}{1.177759in}}%
\pgfpathlineto{\pgfqpoint{0.715238in}{1.165718in}}%
\pgfpathlineto{\pgfqpoint{0.722997in}{1.157035in}}%
\pgfpathlineto{\pgfqpoint{0.724247in}{1.151930in}}%
\pgfpathlineto{\pgfqpoint{0.721628in}{1.147498in}}%
\pgfpathlineto{\pgfqpoint{0.715589in}{1.143661in}}%
\pgfpathlineto{\pgfqpoint{0.689808in}{1.131044in}}%
\pgfpathlineto{\pgfqpoint{0.687958in}{1.126478in}}%
\pgfpathlineto{\pgfqpoint{0.689950in}{1.121243in}}%
\pgfpathlineto{\pgfqpoint{0.695080in}{1.115462in}}%
\pgfpathlineto{\pgfqpoint{0.712862in}{1.097702in}}%
\pgfpathlineto{\pgfqpoint{0.713923in}{1.092629in}}%
\pgfpathlineto{\pgfqpoint{0.711115in}{1.088231in}}%
\pgfpathlineto{\pgfqpoint{0.704933in}{1.084419in}}%
\pgfpathlineto{\pgfqpoint{0.681575in}{1.073823in}}%
\pgfpathlineto{\pgfqpoint{0.678023in}{1.069554in}}%
\pgfpathlineto{\pgfqpoint{0.678317in}{1.064615in}}%
\pgfpathlineto{\pgfqpoint{0.682178in}{1.059055in}}%
\pgfpathlineto{\pgfqpoint{0.702719in}{1.038370in}}%
\pgfpathlineto{\pgfqpoint{0.703589in}{1.033331in}}%
\pgfpathlineto{\pgfqpoint{0.700593in}{1.028965in}}%
\pgfpathlineto{\pgfqpoint{0.694271in}{1.025178in}}%
\pgfpathlineto{\pgfqpoint{0.671028in}{1.014562in}}%
\pgfpathlineto{\pgfqpoint{0.667660in}{1.010261in}}%
\pgfpathlineto{\pgfqpoint{0.668147in}{1.005288in}}%
\pgfpathlineto{\pgfqpoint{0.672162in}{0.999701in}}%
\pgfpathlineto{\pgfqpoint{0.692566in}{0.979040in}}%
\pgfpathlineto{\pgfqpoint{0.693246in}{0.974034in}}%
\pgfpathlineto{\pgfqpoint{0.690063in}{0.969701in}}%
\pgfpathlineto{\pgfqpoint{0.683605in}{0.965938in}}%
\pgfpathlineto{\pgfqpoint{0.675276in}{0.962500in}}%
\pgfpathlineto{\pgfqpoint{0.675276in}{0.962500in}}%
\pgfusepath{stroke}%
\end{pgfscope}%
\begin{pgfscope}%
\pgfpathrectangle{\pgfqpoint{0.148125in}{0.175000in}}{\pgfqpoint{1.653750in}{3.150000in}}%
\pgfusepath{clip}%
\pgfsetrectcap%
\pgfsetroundjoin%
\pgfsetlinewidth{1.003750pt}%
\definecolor{currentstroke}{rgb}{0.000000,0.000000,0.000000}%
\pgfsetstrokecolor{currentstroke}%
\pgfsetdash{}{0pt}%
\pgfpathmoveto{\pgfqpoint{0.778125in}{1.553125in}}%
\pgfpathlineto{\pgfqpoint{0.880974in}{0.962500in}}%
\pgfusepath{stroke}%
\end{pgfscope}%
\begin{pgfscope}%
\pgfpathrectangle{\pgfqpoint{0.148125in}{0.175000in}}{\pgfqpoint{1.653750in}{3.150000in}}%
\pgfusepath{clip}%
\pgfsetrectcap%
\pgfsetroundjoin%
\pgfsetlinewidth{1.003750pt}%
\definecolor{currentstroke}{rgb}{0.000000,0.000000,0.000000}%
\pgfsetstrokecolor{currentstroke}%
\pgfsetdash{}{0pt}%
\pgfpathmoveto{\pgfqpoint{1.171875in}{1.553125in}}%
\pgfpathlineto{\pgfqpoint{1.069026in}{0.962500in}}%
\pgfusepath{stroke}%
\end{pgfscope}%
\begin{pgfscope}%
\pgfpathrectangle{\pgfqpoint{0.148125in}{0.175000in}}{\pgfqpoint{1.653750in}{3.150000in}}%
\pgfusepath{clip}%
\pgfsetrectcap%
\pgfsetroundjoin%
\pgfsetlinewidth{1.003750pt}%
\definecolor{currentstroke}{rgb}{0.000000,0.000000,0.000000}%
\pgfsetstrokecolor{currentstroke}%
\pgfsetdash{}{0pt}%
\pgfpathmoveto{\pgfqpoint{1.171875in}{1.553125in}}%
\pgfpathlineto{\pgfqpoint{1.274724in}{0.962500in}}%
\pgfusepath{stroke}%
\end{pgfscope}%
\begin{pgfscope}%
\pgfpathrectangle{\pgfqpoint{0.148125in}{0.175000in}}{\pgfqpoint{1.653750in}{3.150000in}}%
\pgfusepath{clip}%
\pgfsetrectcap%
\pgfsetroundjoin%
\pgfsetlinewidth{1.003750pt}%
\definecolor{currentstroke}{rgb}{0.000000,0.000000,0.000000}%
\pgfsetstrokecolor{currentstroke}%
\pgfsetdash{}{0pt}%
\pgfpathmoveto{\pgfqpoint{1.565625in}{1.553125in}}%
\pgfpathlineto{\pgfqpoint{1.553730in}{1.547865in}}%
\pgfpathlineto{\pgfqpoint{1.548779in}{1.543839in}}%
\pgfpathlineto{\pgfqpoint{1.547513in}{1.539172in}}%
\pgfpathlineto{\pgfqpoint{1.550042in}{1.533843in}}%
\pgfpathlineto{\pgfqpoint{1.558910in}{1.524967in}}%
\pgfpathlineto{\pgfqpoint{1.571130in}{1.513064in}}%
\pgfpathlineto{\pgfqpoint{1.573482in}{1.507766in}}%
\pgfpathlineto{\pgfqpoint{1.572021in}{1.503133in}}%
\pgfpathlineto{\pgfqpoint{1.566906in}{1.499136in}}%
\pgfpathlineto{\pgfqpoint{1.550634in}{1.492194in}}%
\pgfpathlineto{\pgfqpoint{1.540308in}{1.486660in}}%
\pgfpathlineto{\pgfqpoint{1.537312in}{1.482294in}}%
\pgfpathlineto{\pgfqpoint{1.538182in}{1.477255in}}%
\pgfpathlineto{\pgfqpoint{1.542497in}{1.471615in}}%
\pgfpathlineto{\pgfqpoint{1.562584in}{1.451010in}}%
\pgfpathlineto{\pgfqpoint{1.562878in}{1.446071in}}%
\pgfpathlineto{\pgfqpoint{1.559326in}{1.441802in}}%
\pgfpathlineto{\pgfqpoint{1.552607in}{1.438084in}}%
\pgfpathlineto{\pgfqpoint{1.529786in}{1.427394in}}%
\pgfpathlineto{\pgfqpoint{1.526978in}{1.422996in}}%
\pgfpathlineto{\pgfqpoint{1.528039in}{1.417923in}}%
\pgfpathlineto{\pgfqpoint{1.532498in}{1.412259in}}%
\pgfpathlineto{\pgfqpoint{1.552405in}{1.391685in}}%
\pgfpathlineto{\pgfqpoint{1.552505in}{1.386780in}}%
\pgfpathlineto{\pgfqpoint{1.548771in}{1.382542in}}%
\pgfpathlineto{\pgfqpoint{1.537786in}{1.377123in}}%
\pgfpathlineto{\pgfqpoint{1.521929in}{1.370109in}}%
\pgfpathlineto{\pgfqpoint{1.517485in}{1.365995in}}%
\pgfpathlineto{\pgfqpoint{1.516806in}{1.361225in}}%
\pgfpathlineto{\pgfqpoint{1.519854in}{1.355807in}}%
\pgfpathlineto{\pgfqpoint{1.529103in}{1.346864in}}%
\pgfpathlineto{\pgfqpoint{1.538672in}{1.337866in}}%
\pgfpathlineto{\pgfqpoint{1.542216in}{1.332361in}}%
\pgfpathlineto{\pgfqpoint{1.542122in}{1.327490in}}%
\pgfpathlineto{\pgfqpoint{1.538208in}{1.323284in}}%
\pgfpathlineto{\pgfqpoint{1.527073in}{1.317891in}}%
\pgfpathlineto{\pgfqpoint{1.511343in}{1.310854in}}%
\pgfpathlineto{\pgfqpoint{1.507073in}{1.306710in}}%
\pgfpathlineto{\pgfqpoint{1.506589in}{1.301906in}}%
\pgfpathlineto{\pgfqpoint{1.509805in}{1.296459in}}%
\pgfpathlineto{\pgfqpoint{1.519168in}{1.287496in}}%
\pgfpathlineto{\pgfqpoint{1.528637in}{1.278516in}}%
\pgfpathlineto{\pgfqpoint{1.532019in}{1.273039in}}%
\pgfpathlineto{\pgfqpoint{1.531730in}{1.268202in}}%
\pgfpathlineto{\pgfqpoint{1.527637in}{1.264026in}}%
\pgfpathlineto{\pgfqpoint{1.516360in}{1.258658in}}%
\pgfpathlineto{\pgfqpoint{1.500764in}{1.251599in}}%
\pgfpathlineto{\pgfqpoint{1.496671in}{1.247423in}}%
\pgfpathlineto{\pgfqpoint{1.496382in}{1.242586in}}%
\pgfpathlineto{\pgfqpoint{1.499764in}{1.237109in}}%
\pgfpathlineto{\pgfqpoint{1.509233in}{1.228129in}}%
\pgfpathlineto{\pgfqpoint{1.518596in}{1.219166in}}%
\pgfpathlineto{\pgfqpoint{1.521812in}{1.213719in}}%
\pgfpathlineto{\pgfqpoint{1.521328in}{1.208915in}}%
\pgfpathlineto{\pgfqpoint{1.517058in}{1.204771in}}%
\pgfpathlineto{\pgfqpoint{1.505644in}{1.199427in}}%
\pgfpathlineto{\pgfqpoint{1.490193in}{1.192341in}}%
\pgfpathlineto{\pgfqpoint{1.486279in}{1.188135in}}%
\pgfpathlineto{\pgfqpoint{1.486185in}{1.183264in}}%
\pgfpathlineto{\pgfqpoint{1.489729in}{1.177759in}}%
\pgfpathlineto{\pgfqpoint{1.502738in}{1.165718in}}%
\pgfpathlineto{\pgfqpoint{1.510497in}{1.157035in}}%
\pgfpathlineto{\pgfqpoint{1.511747in}{1.151930in}}%
\pgfpathlineto{\pgfqpoint{1.509128in}{1.147498in}}%
\pgfpathlineto{\pgfqpoint{1.503089in}{1.143661in}}%
\pgfpathlineto{\pgfqpoint{1.477308in}{1.131044in}}%
\pgfpathlineto{\pgfqpoint{1.475458in}{1.126478in}}%
\pgfpathlineto{\pgfqpoint{1.477450in}{1.121243in}}%
\pgfpathlineto{\pgfqpoint{1.482580in}{1.115462in}}%
\pgfpathlineto{\pgfqpoint{1.500362in}{1.097702in}}%
\pgfpathlineto{\pgfqpoint{1.501423in}{1.092629in}}%
\pgfpathlineto{\pgfqpoint{1.498615in}{1.088231in}}%
\pgfpathlineto{\pgfqpoint{1.492433in}{1.084419in}}%
\pgfpathlineto{\pgfqpoint{1.469075in}{1.073823in}}%
\pgfpathlineto{\pgfqpoint{1.465523in}{1.069554in}}%
\pgfpathlineto{\pgfqpoint{1.465817in}{1.064615in}}%
\pgfpathlineto{\pgfqpoint{1.469678in}{1.059055in}}%
\pgfpathlineto{\pgfqpoint{1.490219in}{1.038370in}}%
\pgfpathlineto{\pgfqpoint{1.491089in}{1.033331in}}%
\pgfpathlineto{\pgfqpoint{1.488093in}{1.028965in}}%
\pgfpathlineto{\pgfqpoint{1.481771in}{1.025178in}}%
\pgfpathlineto{\pgfqpoint{1.458528in}{1.014562in}}%
\pgfpathlineto{\pgfqpoint{1.455160in}{1.010261in}}%
\pgfpathlineto{\pgfqpoint{1.455647in}{1.005288in}}%
\pgfpathlineto{\pgfqpoint{1.459662in}{0.999701in}}%
\pgfpathlineto{\pgfqpoint{1.480066in}{0.979040in}}%
\pgfpathlineto{\pgfqpoint{1.480746in}{0.974034in}}%
\pgfpathlineto{\pgfqpoint{1.477563in}{0.969701in}}%
\pgfpathlineto{\pgfqpoint{1.471105in}{0.965938in}}%
\pgfpathlineto{\pgfqpoint{1.462776in}{0.962500in}}%
\pgfpathlineto{\pgfqpoint{1.462776in}{0.962500in}}%
\pgfusepath{stroke}%
\end{pgfscope}%
\begin{pgfscope}%
\pgfpathrectangle{\pgfqpoint{0.148125in}{0.175000in}}{\pgfqpoint{1.653750in}{3.150000in}}%
\pgfusepath{clip}%
\pgfsetrectcap%
\pgfsetroundjoin%
\pgfsetlinewidth{1.003750pt}%
\definecolor{currentstroke}{rgb}{0.000000,0.000000,0.000000}%
\pgfsetstrokecolor{currentstroke}%
\pgfsetdash{}{0pt}%
\pgfpathmoveto{\pgfqpoint{1.565625in}{1.553125in}}%
\pgfpathlineto{\pgfqpoint{1.668474in}{0.962500in}}%
\pgfusepath{stroke}%
\end{pgfscope}%
\begin{pgfscope}%
\pgfpathrectangle{\pgfqpoint{0.148125in}{0.175000in}}{\pgfqpoint{1.653750in}{3.150000in}}%
\pgfusepath{clip}%
\pgfsetbuttcap%
\pgfsetroundjoin%
\pgfsetlinewidth{1.003750pt}%
\definecolor{currentstroke}{rgb}{0.827451,0.827451,0.827451}%
\pgfsetstrokecolor{currentstroke}%
\pgfsetdash{{3.700000pt}{1.600000pt}}{0.000000pt}%
\pgfpathmoveto{\pgfqpoint{0.148125in}{1.553125in}}%
\pgfpathlineto{\pgfqpoint{1.801875in}{1.553125in}}%
\pgfusepath{stroke}%
\end{pgfscope}%
\begin{pgfscope}%
\pgfpathrectangle{\pgfqpoint{0.148125in}{0.175000in}}{\pgfqpoint{1.653750in}{3.150000in}}%
\pgfusepath{clip}%
\pgfsetrectcap%
\pgfsetroundjoin%
\pgfsetlinewidth{1.003750pt}%
\definecolor{currentstroke}{rgb}{0.000000,0.000000,0.000000}%
\pgfsetstrokecolor{currentstroke}%
\pgfsetdash{}{0pt}%
\pgfpathmoveto{\pgfqpoint{0.281526in}{0.962500in}}%
\pgfpathlineto{\pgfqpoint{0.217978in}{0.371875in}}%
\pgfusepath{stroke}%
\end{pgfscope}%
\begin{pgfscope}%
\pgfpathrectangle{\pgfqpoint{0.148125in}{0.175000in}}{\pgfqpoint{1.653750in}{3.150000in}}%
\pgfusepath{clip}%
\pgfsetrectcap%
\pgfsetroundjoin%
\pgfsetlinewidth{1.003750pt}%
\definecolor{currentstroke}{rgb}{0.000000,0.000000,0.000000}%
\pgfsetstrokecolor{currentstroke}%
\pgfsetdash{}{0pt}%
\pgfpathmoveto{\pgfqpoint{0.281526in}{0.962500in}}%
\pgfpathlineto{\pgfqpoint{0.345073in}{0.371875in}}%
\pgfusepath{stroke}%
\end{pgfscope}%
\begin{pgfscope}%
\pgfpathrectangle{\pgfqpoint{0.148125in}{0.175000in}}{\pgfqpoint{1.653750in}{3.150000in}}%
\pgfusepath{clip}%
\pgfsetrectcap%
\pgfsetroundjoin%
\pgfsetlinewidth{1.003750pt}%
\definecolor{currentstroke}{rgb}{0.000000,0.000000,0.000000}%
\pgfsetstrokecolor{currentstroke}%
\pgfsetdash{}{0pt}%
\pgfpathmoveto{\pgfqpoint{0.487224in}{0.962500in}}%
\pgfpathlineto{\pgfqpoint{0.475704in}{0.956541in}}%
\pgfpathlineto{\pgfqpoint{0.471031in}{0.952245in}}%
\pgfpathlineto{\pgfqpoint{0.470076in}{0.947549in}}%
\pgfpathlineto{\pgfqpoint{0.472952in}{0.942441in}}%
\pgfpathlineto{\pgfqpoint{0.482387in}{0.934227in}}%
\pgfpathlineto{\pgfqpoint{0.495366in}{0.923233in}}%
\pgfpathlineto{\pgfqpoint{0.498063in}{0.918144in}}%
\pgfpathlineto{\pgfqpoint{0.496912in}{0.913469in}}%
\pgfpathlineto{\pgfqpoint{0.492073in}{0.909190in}}%
\pgfpathlineto{\pgfqpoint{0.476297in}{0.901290in}}%
\pgfpathlineto{\pgfqpoint{0.466361in}{0.895161in}}%
\pgfpathlineto{\pgfqpoint{0.463661in}{0.890652in}}%
\pgfpathlineto{\pgfqpoint{0.464863in}{0.885724in}}%
\pgfpathlineto{\pgfqpoint{0.469540in}{0.880422in}}%
\pgfpathlineto{\pgfqpoint{0.490944in}{0.861323in}}%
\pgfpathlineto{\pgfqpoint{0.491564in}{0.856457in}}%
\pgfpathlineto{\pgfqpoint{0.488302in}{0.852009in}}%
\pgfpathlineto{\pgfqpoint{0.481845in}{0.847905in}}%
\pgfpathlineto{\pgfqpoint{0.459783in}{0.835882in}}%
\pgfpathlineto{\pgfqpoint{0.457273in}{0.831354in}}%
\pgfpathlineto{\pgfqpoint{0.458667in}{0.826405in}}%
\pgfpathlineto{\pgfqpoint{0.463491in}{0.821087in}}%
\pgfpathlineto{\pgfqpoint{0.484712in}{0.802007in}}%
\pgfpathlineto{\pgfqpoint{0.485137in}{0.797163in}}%
\pgfpathlineto{\pgfqpoint{0.481691in}{0.792735in}}%
\pgfpathlineto{\pgfqpoint{0.471090in}{0.786677in}}%
\pgfpathlineto{\pgfqpoint{0.455733in}{0.778731in}}%
\pgfpathlineto{\pgfqpoint{0.451572in}{0.774380in}}%
\pgfpathlineto{\pgfqpoint{0.451210in}{0.769620in}}%
\pgfpathlineto{\pgfqpoint{0.454609in}{0.764456in}}%
\pgfpathlineto{\pgfqpoint{0.464428in}{0.756201in}}%
\pgfpathlineto{\pgfqpoint{0.474571in}{0.747911in}}%
\pgfpathlineto{\pgfqpoint{0.478471in}{0.742693in}}%
\pgfpathlineto{\pgfqpoint{0.478699in}{0.737869in}}%
\pgfpathlineto{\pgfqpoint{0.475072in}{0.733461in}}%
\pgfpathlineto{\pgfqpoint{0.464320in}{0.727419in}}%
\pgfpathlineto{\pgfqpoint{0.449091in}{0.719460in}}%
\pgfpathlineto{\pgfqpoint{0.445105in}{0.715090in}}%
\pgfpathlineto{\pgfqpoint{0.444940in}{0.710309in}}%
\pgfpathlineto{\pgfqpoint{0.448509in}{0.705126in}}%
\pgfpathlineto{\pgfqpoint{0.458443in}{0.696859in}}%
\pgfpathlineto{\pgfqpoint{0.468485in}{0.688580in}}%
\pgfpathlineto{\pgfqpoint{0.472220in}{0.683379in}}%
\pgfpathlineto{\pgfqpoint{0.472252in}{0.678577in}}%
\pgfpathlineto{\pgfqpoint{0.468445in}{0.674187in}}%
\pgfpathlineto{\pgfqpoint{0.457548in}{0.668161in}}%
\pgfpathlineto{\pgfqpoint{0.442456in}{0.660188in}}%
\pgfpathlineto{\pgfqpoint{0.438648in}{0.655798in}}%
\pgfpathlineto{\pgfqpoint{0.438680in}{0.650996in}}%
\pgfpathlineto{\pgfqpoint{0.442416in}{0.645795in}}%
\pgfpathlineto{\pgfqpoint{0.452458in}{0.637516in}}%
\pgfpathlineto{\pgfqpoint{0.462392in}{0.629249in}}%
\pgfpathlineto{\pgfqpoint{0.465960in}{0.624066in}}%
\pgfpathlineto{\pgfqpoint{0.465795in}{0.619285in}}%
\pgfpathlineto{\pgfqpoint{0.461810in}{0.614915in}}%
\pgfpathlineto{\pgfqpoint{0.450776in}{0.608904in}}%
\pgfpathlineto{\pgfqpoint{0.435829in}{0.600914in}}%
\pgfpathlineto{\pgfqpoint{0.432201in}{0.596506in}}%
\pgfpathlineto{\pgfqpoint{0.432429in}{0.591682in}}%
\pgfpathlineto{\pgfqpoint{0.436330in}{0.586464in}}%
\pgfpathlineto{\pgfqpoint{0.450105in}{0.575384in}}%
\pgfpathlineto{\pgfqpoint{0.458421in}{0.567291in}}%
\pgfpathlineto{\pgfqpoint{0.460005in}{0.562321in}}%
\pgfpathlineto{\pgfqpoint{0.457686in}{0.557772in}}%
\pgfpathlineto{\pgfqpoint{0.451915in}{0.553594in}}%
\pgfpathlineto{\pgfqpoint{0.427027in}{0.539476in}}%
\pgfpathlineto{\pgfqpoint{0.425484in}{0.534843in}}%
\pgfpathlineto{\pgfqpoint{0.427817in}{0.529793in}}%
\pgfpathlineto{\pgfqpoint{0.433317in}{0.524402in}}%
\pgfpathlineto{\pgfqpoint{0.452234in}{0.507970in}}%
\pgfpathlineto{\pgfqpoint{0.453628in}{0.503021in}}%
\pgfpathlineto{\pgfqpoint{0.451117in}{0.498493in}}%
\pgfpathlineto{\pgfqpoint{0.445201in}{0.494330in}}%
\pgfpathlineto{\pgfqpoint{0.422598in}{0.482366in}}%
\pgfpathlineto{\pgfqpoint{0.419336in}{0.477918in}}%
\pgfpathlineto{\pgfqpoint{0.419956in}{0.473052in}}%
\pgfpathlineto{\pgfqpoint{0.424177in}{0.467799in}}%
\pgfpathlineto{\pgfqpoint{0.446038in}{0.448651in}}%
\pgfpathlineto{\pgfqpoint{0.447240in}{0.443723in}}%
\pgfpathlineto{\pgfqpoint{0.444539in}{0.439214in}}%
\pgfpathlineto{\pgfqpoint{0.438482in}{0.435067in}}%
\pgfpathlineto{\pgfqpoint{0.415995in}{0.423090in}}%
\pgfpathlineto{\pgfqpoint{0.412919in}{0.418622in}}%
\pgfpathlineto{\pgfqpoint{0.413734in}{0.413735in}}%
\pgfpathlineto{\pgfqpoint{0.418109in}{0.408466in}}%
\pgfpathlineto{\pgfqpoint{0.439833in}{0.389332in}}%
\pgfpathlineto{\pgfqpoint{0.440842in}{0.384425in}}%
\pgfpathlineto{\pgfqpoint{0.437953in}{0.379937in}}%
\pgfpathlineto{\pgfqpoint{0.431759in}{0.375804in}}%
\pgfpathlineto{\pgfqpoint{0.423677in}{0.371875in}}%
\pgfpathlineto{\pgfqpoint{0.423677in}{0.371875in}}%
\pgfusepath{stroke}%
\end{pgfscope}%
\begin{pgfscope}%
\pgfpathrectangle{\pgfqpoint{0.148125in}{0.175000in}}{\pgfqpoint{1.653750in}{3.150000in}}%
\pgfusepath{clip}%
\pgfsetrectcap%
\pgfsetroundjoin%
\pgfsetlinewidth{1.003750pt}%
\definecolor{currentstroke}{rgb}{0.000000,0.000000,0.000000}%
\pgfsetstrokecolor{currentstroke}%
\pgfsetdash{}{0pt}%
\pgfpathmoveto{\pgfqpoint{0.487224in}{0.962500in}}%
\pgfpathlineto{\pgfqpoint{0.550772in}{0.371875in}}%
\pgfusepath{stroke}%
\end{pgfscope}%
\begin{pgfscope}%
\pgfpathrectangle{\pgfqpoint{0.148125in}{0.175000in}}{\pgfqpoint{1.653750in}{3.150000in}}%
\pgfusepath{clip}%
\pgfsetrectcap%
\pgfsetroundjoin%
\pgfsetlinewidth{1.003750pt}%
\definecolor{currentstroke}{rgb}{0.000000,0.000000,0.000000}%
\pgfsetstrokecolor{currentstroke}%
\pgfsetdash{}{0pt}%
\pgfpathmoveto{\pgfqpoint{0.675276in}{0.962500in}}%
\pgfpathlineto{\pgfqpoint{0.611728in}{0.371875in}}%
\pgfusepath{stroke}%
\end{pgfscope}%
\begin{pgfscope}%
\pgfpathrectangle{\pgfqpoint{0.148125in}{0.175000in}}{\pgfqpoint{1.653750in}{3.150000in}}%
\pgfusepath{clip}%
\pgfsetrectcap%
\pgfsetroundjoin%
\pgfsetlinewidth{1.003750pt}%
\definecolor{currentstroke}{rgb}{0.000000,0.000000,0.000000}%
\pgfsetstrokecolor{currentstroke}%
\pgfsetdash{}{0pt}%
\pgfpathmoveto{\pgfqpoint{0.675276in}{0.962500in}}%
\pgfpathlineto{\pgfqpoint{0.738823in}{0.371875in}}%
\pgfusepath{stroke}%
\end{pgfscope}%
\begin{pgfscope}%
\pgfpathrectangle{\pgfqpoint{0.148125in}{0.175000in}}{\pgfqpoint{1.653750in}{3.150000in}}%
\pgfusepath{clip}%
\pgfsetrectcap%
\pgfsetroundjoin%
\pgfsetlinewidth{1.003750pt}%
\definecolor{currentstroke}{rgb}{0.000000,0.000000,0.000000}%
\pgfsetstrokecolor{currentstroke}%
\pgfsetdash{}{0pt}%
\pgfpathmoveto{\pgfqpoint{0.880974in}{0.962500in}}%
\pgfpathlineto{\pgfqpoint{0.869454in}{0.956541in}}%
\pgfpathlineto{\pgfqpoint{0.864781in}{0.952245in}}%
\pgfpathlineto{\pgfqpoint{0.863826in}{0.947549in}}%
\pgfpathlineto{\pgfqpoint{0.866702in}{0.942441in}}%
\pgfpathlineto{\pgfqpoint{0.876137in}{0.934227in}}%
\pgfpathlineto{\pgfqpoint{0.889116in}{0.923233in}}%
\pgfpathlineto{\pgfqpoint{0.891813in}{0.918144in}}%
\pgfpathlineto{\pgfqpoint{0.890662in}{0.913469in}}%
\pgfpathlineto{\pgfqpoint{0.885823in}{0.909190in}}%
\pgfpathlineto{\pgfqpoint{0.870047in}{0.901290in}}%
\pgfpathlineto{\pgfqpoint{0.860111in}{0.895161in}}%
\pgfpathlineto{\pgfqpoint{0.857411in}{0.890652in}}%
\pgfpathlineto{\pgfqpoint{0.858613in}{0.885724in}}%
\pgfpathlineto{\pgfqpoint{0.863290in}{0.880422in}}%
\pgfpathlineto{\pgfqpoint{0.884694in}{0.861323in}}%
\pgfpathlineto{\pgfqpoint{0.885314in}{0.856457in}}%
\pgfpathlineto{\pgfqpoint{0.882052in}{0.852009in}}%
\pgfpathlineto{\pgfqpoint{0.875595in}{0.847905in}}%
\pgfpathlineto{\pgfqpoint{0.853533in}{0.835882in}}%
\pgfpathlineto{\pgfqpoint{0.851023in}{0.831354in}}%
\pgfpathlineto{\pgfqpoint{0.852417in}{0.826405in}}%
\pgfpathlineto{\pgfqpoint{0.857241in}{0.821087in}}%
\pgfpathlineto{\pgfqpoint{0.878462in}{0.802007in}}%
\pgfpathlineto{\pgfqpoint{0.878887in}{0.797163in}}%
\pgfpathlineto{\pgfqpoint{0.875441in}{0.792735in}}%
\pgfpathlineto{\pgfqpoint{0.864840in}{0.786677in}}%
\pgfpathlineto{\pgfqpoint{0.849483in}{0.778731in}}%
\pgfpathlineto{\pgfqpoint{0.845322in}{0.774380in}}%
\pgfpathlineto{\pgfqpoint{0.844960in}{0.769620in}}%
\pgfpathlineto{\pgfqpoint{0.848359in}{0.764456in}}%
\pgfpathlineto{\pgfqpoint{0.858178in}{0.756201in}}%
\pgfpathlineto{\pgfqpoint{0.868321in}{0.747911in}}%
\pgfpathlineto{\pgfqpoint{0.872221in}{0.742693in}}%
\pgfpathlineto{\pgfqpoint{0.872449in}{0.737869in}}%
\pgfpathlineto{\pgfqpoint{0.868822in}{0.733461in}}%
\pgfpathlineto{\pgfqpoint{0.858070in}{0.727419in}}%
\pgfpathlineto{\pgfqpoint{0.842841in}{0.719460in}}%
\pgfpathlineto{\pgfqpoint{0.838855in}{0.715090in}}%
\pgfpathlineto{\pgfqpoint{0.838690in}{0.710309in}}%
\pgfpathlineto{\pgfqpoint{0.842259in}{0.705126in}}%
\pgfpathlineto{\pgfqpoint{0.852193in}{0.696859in}}%
\pgfpathlineto{\pgfqpoint{0.862235in}{0.688580in}}%
\pgfpathlineto{\pgfqpoint{0.865970in}{0.683379in}}%
\pgfpathlineto{\pgfqpoint{0.866002in}{0.678577in}}%
\pgfpathlineto{\pgfqpoint{0.862195in}{0.674187in}}%
\pgfpathlineto{\pgfqpoint{0.851298in}{0.668161in}}%
\pgfpathlineto{\pgfqpoint{0.836206in}{0.660188in}}%
\pgfpathlineto{\pgfqpoint{0.832398in}{0.655798in}}%
\pgfpathlineto{\pgfqpoint{0.832430in}{0.650996in}}%
\pgfpathlineto{\pgfqpoint{0.836166in}{0.645795in}}%
\pgfpathlineto{\pgfqpoint{0.846208in}{0.637516in}}%
\pgfpathlineto{\pgfqpoint{0.856142in}{0.629249in}}%
\pgfpathlineto{\pgfqpoint{0.859710in}{0.624066in}}%
\pgfpathlineto{\pgfqpoint{0.859545in}{0.619285in}}%
\pgfpathlineto{\pgfqpoint{0.855560in}{0.614915in}}%
\pgfpathlineto{\pgfqpoint{0.844526in}{0.608904in}}%
\pgfpathlineto{\pgfqpoint{0.829579in}{0.600914in}}%
\pgfpathlineto{\pgfqpoint{0.825951in}{0.596506in}}%
\pgfpathlineto{\pgfqpoint{0.826179in}{0.591682in}}%
\pgfpathlineto{\pgfqpoint{0.830080in}{0.586464in}}%
\pgfpathlineto{\pgfqpoint{0.843855in}{0.575384in}}%
\pgfpathlineto{\pgfqpoint{0.852171in}{0.567291in}}%
\pgfpathlineto{\pgfqpoint{0.853755in}{0.562321in}}%
\pgfpathlineto{\pgfqpoint{0.851436in}{0.557772in}}%
\pgfpathlineto{\pgfqpoint{0.845665in}{0.553594in}}%
\pgfpathlineto{\pgfqpoint{0.820777in}{0.539476in}}%
\pgfpathlineto{\pgfqpoint{0.819234in}{0.534843in}}%
\pgfpathlineto{\pgfqpoint{0.821567in}{0.529793in}}%
\pgfpathlineto{\pgfqpoint{0.827067in}{0.524402in}}%
\pgfpathlineto{\pgfqpoint{0.845984in}{0.507970in}}%
\pgfpathlineto{\pgfqpoint{0.847378in}{0.503021in}}%
\pgfpathlineto{\pgfqpoint{0.844867in}{0.498493in}}%
\pgfpathlineto{\pgfqpoint{0.838951in}{0.494330in}}%
\pgfpathlineto{\pgfqpoint{0.816348in}{0.482366in}}%
\pgfpathlineto{\pgfqpoint{0.813086in}{0.477918in}}%
\pgfpathlineto{\pgfqpoint{0.813706in}{0.473052in}}%
\pgfpathlineto{\pgfqpoint{0.817927in}{0.467799in}}%
\pgfpathlineto{\pgfqpoint{0.839788in}{0.448651in}}%
\pgfpathlineto{\pgfqpoint{0.840990in}{0.443723in}}%
\pgfpathlineto{\pgfqpoint{0.838289in}{0.439214in}}%
\pgfpathlineto{\pgfqpoint{0.832232in}{0.435067in}}%
\pgfpathlineto{\pgfqpoint{0.809745in}{0.423090in}}%
\pgfpathlineto{\pgfqpoint{0.806669in}{0.418622in}}%
\pgfpathlineto{\pgfqpoint{0.807484in}{0.413735in}}%
\pgfpathlineto{\pgfqpoint{0.811859in}{0.408466in}}%
\pgfpathlineto{\pgfqpoint{0.833583in}{0.389332in}}%
\pgfpathlineto{\pgfqpoint{0.834592in}{0.384425in}}%
\pgfpathlineto{\pgfqpoint{0.831703in}{0.379937in}}%
\pgfpathlineto{\pgfqpoint{0.825509in}{0.375804in}}%
\pgfpathlineto{\pgfqpoint{0.817427in}{0.371875in}}%
\pgfpathlineto{\pgfqpoint{0.817427in}{0.371875in}}%
\pgfusepath{stroke}%
\end{pgfscope}%
\begin{pgfscope}%
\pgfpathrectangle{\pgfqpoint{0.148125in}{0.175000in}}{\pgfqpoint{1.653750in}{3.150000in}}%
\pgfusepath{clip}%
\pgfsetrectcap%
\pgfsetroundjoin%
\pgfsetlinewidth{1.003750pt}%
\definecolor{currentstroke}{rgb}{0.000000,0.000000,0.000000}%
\pgfsetstrokecolor{currentstroke}%
\pgfsetdash{}{0pt}%
\pgfpathmoveto{\pgfqpoint{0.880974in}{0.962500in}}%
\pgfpathlineto{\pgfqpoint{0.944522in}{0.371875in}}%
\pgfusepath{stroke}%
\end{pgfscope}%
\begin{pgfscope}%
\pgfpathrectangle{\pgfqpoint{0.148125in}{0.175000in}}{\pgfqpoint{1.653750in}{3.150000in}}%
\pgfusepath{clip}%
\pgfsetrectcap%
\pgfsetroundjoin%
\pgfsetlinewidth{1.003750pt}%
\definecolor{currentstroke}{rgb}{0.000000,0.000000,0.000000}%
\pgfsetstrokecolor{currentstroke}%
\pgfsetdash{}{0pt}%
\pgfpathmoveto{\pgfqpoint{1.069026in}{0.962500in}}%
\pgfpathlineto{\pgfqpoint{1.057506in}{0.956541in}}%
\pgfpathlineto{\pgfqpoint{1.052833in}{0.952245in}}%
\pgfpathlineto{\pgfqpoint{1.051878in}{0.947549in}}%
\pgfpathlineto{\pgfqpoint{1.054754in}{0.942441in}}%
\pgfpathlineto{\pgfqpoint{1.064189in}{0.934227in}}%
\pgfpathlineto{\pgfqpoint{1.077168in}{0.923233in}}%
\pgfpathlineto{\pgfqpoint{1.079865in}{0.918144in}}%
\pgfpathlineto{\pgfqpoint{1.078714in}{0.913469in}}%
\pgfpathlineto{\pgfqpoint{1.073875in}{0.909190in}}%
\pgfpathlineto{\pgfqpoint{1.058099in}{0.901290in}}%
\pgfpathlineto{\pgfqpoint{1.048163in}{0.895161in}}%
\pgfpathlineto{\pgfqpoint{1.045463in}{0.890652in}}%
\pgfpathlineto{\pgfqpoint{1.046665in}{0.885724in}}%
\pgfpathlineto{\pgfqpoint{1.051342in}{0.880422in}}%
\pgfpathlineto{\pgfqpoint{1.072746in}{0.861323in}}%
\pgfpathlineto{\pgfqpoint{1.073366in}{0.856457in}}%
\pgfpathlineto{\pgfqpoint{1.070104in}{0.852009in}}%
\pgfpathlineto{\pgfqpoint{1.063646in}{0.847905in}}%
\pgfpathlineto{\pgfqpoint{1.041585in}{0.835882in}}%
\pgfpathlineto{\pgfqpoint{1.039075in}{0.831354in}}%
\pgfpathlineto{\pgfqpoint{1.040469in}{0.826405in}}%
\pgfpathlineto{\pgfqpoint{1.045292in}{0.821087in}}%
\pgfpathlineto{\pgfqpoint{1.066514in}{0.802007in}}%
\pgfpathlineto{\pgfqpoint{1.066939in}{0.797163in}}%
\pgfpathlineto{\pgfqpoint{1.063493in}{0.792735in}}%
\pgfpathlineto{\pgfqpoint{1.052892in}{0.786677in}}%
\pgfpathlineto{\pgfqpoint{1.037535in}{0.778731in}}%
\pgfpathlineto{\pgfqpoint{1.033374in}{0.774380in}}%
\pgfpathlineto{\pgfqpoint{1.033012in}{0.769620in}}%
\pgfpathlineto{\pgfqpoint{1.036411in}{0.764456in}}%
\pgfpathlineto{\pgfqpoint{1.046230in}{0.756201in}}%
\pgfpathlineto{\pgfqpoint{1.056373in}{0.747911in}}%
\pgfpathlineto{\pgfqpoint{1.060273in}{0.742693in}}%
\pgfpathlineto{\pgfqpoint{1.060501in}{0.737869in}}%
\pgfpathlineto{\pgfqpoint{1.056874in}{0.733461in}}%
\pgfpathlineto{\pgfqpoint{1.046122in}{0.727419in}}%
\pgfpathlineto{\pgfqpoint{1.030893in}{0.719460in}}%
\pgfpathlineto{\pgfqpoint{1.026907in}{0.715090in}}%
\pgfpathlineto{\pgfqpoint{1.026742in}{0.710309in}}%
\pgfpathlineto{\pgfqpoint{1.030311in}{0.705126in}}%
\pgfpathlineto{\pgfqpoint{1.040245in}{0.696859in}}%
\pgfpathlineto{\pgfqpoint{1.050287in}{0.688580in}}%
\pgfpathlineto{\pgfqpoint{1.054022in}{0.683379in}}%
\pgfpathlineto{\pgfqpoint{1.054054in}{0.678577in}}%
\pgfpathlineto{\pgfqpoint{1.050247in}{0.674187in}}%
\pgfpathlineto{\pgfqpoint{1.039350in}{0.668161in}}%
\pgfpathlineto{\pgfqpoint{1.024258in}{0.660188in}}%
\pgfpathlineto{\pgfqpoint{1.020450in}{0.655798in}}%
\pgfpathlineto{\pgfqpoint{1.020482in}{0.650996in}}%
\pgfpathlineto{\pgfqpoint{1.024218in}{0.645795in}}%
\pgfpathlineto{\pgfqpoint{1.034260in}{0.637516in}}%
\pgfpathlineto{\pgfqpoint{1.044193in}{0.629249in}}%
\pgfpathlineto{\pgfqpoint{1.047762in}{0.624066in}}%
\pgfpathlineto{\pgfqpoint{1.047597in}{0.619285in}}%
\pgfpathlineto{\pgfqpoint{1.043612in}{0.614915in}}%
\pgfpathlineto{\pgfqpoint{1.032577in}{0.608904in}}%
\pgfpathlineto{\pgfqpoint{1.017631in}{0.600914in}}%
\pgfpathlineto{\pgfqpoint{1.014003in}{0.596506in}}%
\pgfpathlineto{\pgfqpoint{1.014231in}{0.591682in}}%
\pgfpathlineto{\pgfqpoint{1.018132in}{0.586464in}}%
\pgfpathlineto{\pgfqpoint{1.031907in}{0.575384in}}%
\pgfpathlineto{\pgfqpoint{1.040223in}{0.567291in}}%
\pgfpathlineto{\pgfqpoint{1.041807in}{0.562321in}}%
\pgfpathlineto{\pgfqpoint{1.039488in}{0.557772in}}%
\pgfpathlineto{\pgfqpoint{1.033716in}{0.553594in}}%
\pgfpathlineto{\pgfqpoint{1.008829in}{0.539476in}}%
\pgfpathlineto{\pgfqpoint{1.007286in}{0.534843in}}%
\pgfpathlineto{\pgfqpoint{1.009619in}{0.529793in}}%
\pgfpathlineto{\pgfqpoint{1.015119in}{0.524402in}}%
\pgfpathlineto{\pgfqpoint{1.034036in}{0.507970in}}%
\pgfpathlineto{\pgfqpoint{1.035430in}{0.503021in}}%
\pgfpathlineto{\pgfqpoint{1.032919in}{0.498493in}}%
\pgfpathlineto{\pgfqpoint{1.027003in}{0.494330in}}%
\pgfpathlineto{\pgfqpoint{1.004400in}{0.482366in}}%
\pgfpathlineto{\pgfqpoint{1.001138in}{0.477918in}}%
\pgfpathlineto{\pgfqpoint{1.001758in}{0.473052in}}%
\pgfpathlineto{\pgfqpoint{1.005979in}{0.467799in}}%
\pgfpathlineto{\pgfqpoint{1.027840in}{0.448651in}}%
\pgfpathlineto{\pgfqpoint{1.029042in}{0.443723in}}%
\pgfpathlineto{\pgfqpoint{1.026341in}{0.439214in}}%
\pgfpathlineto{\pgfqpoint{1.020284in}{0.435067in}}%
\pgfpathlineto{\pgfqpoint{0.997797in}{0.423090in}}%
\pgfpathlineto{\pgfqpoint{0.994720in}{0.418622in}}%
\pgfpathlineto{\pgfqpoint{0.995535in}{0.413735in}}%
\pgfpathlineto{\pgfqpoint{0.999911in}{0.408466in}}%
\pgfpathlineto{\pgfqpoint{1.021635in}{0.389332in}}%
\pgfpathlineto{\pgfqpoint{1.022644in}{0.384425in}}%
\pgfpathlineto{\pgfqpoint{1.019755in}{0.379937in}}%
\pgfpathlineto{\pgfqpoint{1.013561in}{0.375804in}}%
\pgfpathlineto{\pgfqpoint{1.005478in}{0.371875in}}%
\pgfpathlineto{\pgfqpoint{1.005478in}{0.371875in}}%
\pgfusepath{stroke}%
\end{pgfscope}%
\begin{pgfscope}%
\pgfpathrectangle{\pgfqpoint{0.148125in}{0.175000in}}{\pgfqpoint{1.653750in}{3.150000in}}%
\pgfusepath{clip}%
\pgfsetrectcap%
\pgfsetroundjoin%
\pgfsetlinewidth{1.003750pt}%
\definecolor{currentstroke}{rgb}{0.000000,0.000000,0.000000}%
\pgfsetstrokecolor{currentstroke}%
\pgfsetdash{}{0pt}%
\pgfpathmoveto{\pgfqpoint{1.069026in}{0.962500in}}%
\pgfpathlineto{\pgfqpoint{1.132573in}{0.371875in}}%
\pgfusepath{stroke}%
\end{pgfscope}%
\begin{pgfscope}%
\pgfpathrectangle{\pgfqpoint{0.148125in}{0.175000in}}{\pgfqpoint{1.653750in}{3.150000in}}%
\pgfusepath{clip}%
\pgfsetrectcap%
\pgfsetroundjoin%
\pgfsetlinewidth{1.003750pt}%
\definecolor{currentstroke}{rgb}{0.000000,0.000000,0.000000}%
\pgfsetstrokecolor{currentstroke}%
\pgfsetdash{}{0pt}%
\pgfpathmoveto{\pgfqpoint{1.274724in}{0.962500in}}%
\pgfpathlineto{\pgfqpoint{1.263204in}{0.956541in}}%
\pgfpathlineto{\pgfqpoint{1.258531in}{0.952245in}}%
\pgfpathlineto{\pgfqpoint{1.257576in}{0.947549in}}%
\pgfpathlineto{\pgfqpoint{1.260452in}{0.942441in}}%
\pgfpathlineto{\pgfqpoint{1.269887in}{0.934227in}}%
\pgfpathlineto{\pgfqpoint{1.282866in}{0.923233in}}%
\pgfpathlineto{\pgfqpoint{1.285563in}{0.918144in}}%
\pgfpathlineto{\pgfqpoint{1.284412in}{0.913469in}}%
\pgfpathlineto{\pgfqpoint{1.279573in}{0.909190in}}%
\pgfpathlineto{\pgfqpoint{1.263797in}{0.901290in}}%
\pgfpathlineto{\pgfqpoint{1.253861in}{0.895161in}}%
\pgfpathlineto{\pgfqpoint{1.251161in}{0.890652in}}%
\pgfpathlineto{\pgfqpoint{1.252363in}{0.885724in}}%
\pgfpathlineto{\pgfqpoint{1.257040in}{0.880422in}}%
\pgfpathlineto{\pgfqpoint{1.278444in}{0.861323in}}%
\pgfpathlineto{\pgfqpoint{1.279064in}{0.856457in}}%
\pgfpathlineto{\pgfqpoint{1.275802in}{0.852009in}}%
\pgfpathlineto{\pgfqpoint{1.269345in}{0.847905in}}%
\pgfpathlineto{\pgfqpoint{1.247283in}{0.835882in}}%
\pgfpathlineto{\pgfqpoint{1.244773in}{0.831354in}}%
\pgfpathlineto{\pgfqpoint{1.246167in}{0.826405in}}%
\pgfpathlineto{\pgfqpoint{1.250991in}{0.821087in}}%
\pgfpathlineto{\pgfqpoint{1.272212in}{0.802007in}}%
\pgfpathlineto{\pgfqpoint{1.272637in}{0.797163in}}%
\pgfpathlineto{\pgfqpoint{1.269191in}{0.792735in}}%
\pgfpathlineto{\pgfqpoint{1.258590in}{0.786677in}}%
\pgfpathlineto{\pgfqpoint{1.243233in}{0.778731in}}%
\pgfpathlineto{\pgfqpoint{1.239072in}{0.774380in}}%
\pgfpathlineto{\pgfqpoint{1.238710in}{0.769620in}}%
\pgfpathlineto{\pgfqpoint{1.242109in}{0.764456in}}%
\pgfpathlineto{\pgfqpoint{1.251928in}{0.756201in}}%
\pgfpathlineto{\pgfqpoint{1.262071in}{0.747911in}}%
\pgfpathlineto{\pgfqpoint{1.265971in}{0.742693in}}%
\pgfpathlineto{\pgfqpoint{1.266199in}{0.737869in}}%
\pgfpathlineto{\pgfqpoint{1.262572in}{0.733461in}}%
\pgfpathlineto{\pgfqpoint{1.251820in}{0.727419in}}%
\pgfpathlineto{\pgfqpoint{1.236591in}{0.719460in}}%
\pgfpathlineto{\pgfqpoint{1.232605in}{0.715090in}}%
\pgfpathlineto{\pgfqpoint{1.232440in}{0.710309in}}%
\pgfpathlineto{\pgfqpoint{1.236009in}{0.705126in}}%
\pgfpathlineto{\pgfqpoint{1.245943in}{0.696859in}}%
\pgfpathlineto{\pgfqpoint{1.255985in}{0.688580in}}%
\pgfpathlineto{\pgfqpoint{1.259720in}{0.683379in}}%
\pgfpathlineto{\pgfqpoint{1.259752in}{0.678577in}}%
\pgfpathlineto{\pgfqpoint{1.255945in}{0.674187in}}%
\pgfpathlineto{\pgfqpoint{1.245048in}{0.668161in}}%
\pgfpathlineto{\pgfqpoint{1.229956in}{0.660188in}}%
\pgfpathlineto{\pgfqpoint{1.226148in}{0.655798in}}%
\pgfpathlineto{\pgfqpoint{1.226180in}{0.650996in}}%
\pgfpathlineto{\pgfqpoint{1.229916in}{0.645795in}}%
\pgfpathlineto{\pgfqpoint{1.239958in}{0.637516in}}%
\pgfpathlineto{\pgfqpoint{1.249892in}{0.629249in}}%
\pgfpathlineto{\pgfqpoint{1.253460in}{0.624066in}}%
\pgfpathlineto{\pgfqpoint{1.253295in}{0.619285in}}%
\pgfpathlineto{\pgfqpoint{1.249310in}{0.614915in}}%
\pgfpathlineto{\pgfqpoint{1.238276in}{0.608904in}}%
\pgfpathlineto{\pgfqpoint{1.223329in}{0.600914in}}%
\pgfpathlineto{\pgfqpoint{1.219701in}{0.596506in}}%
\pgfpathlineto{\pgfqpoint{1.219929in}{0.591682in}}%
\pgfpathlineto{\pgfqpoint{1.223830in}{0.586464in}}%
\pgfpathlineto{\pgfqpoint{1.237605in}{0.575384in}}%
\pgfpathlineto{\pgfqpoint{1.245921in}{0.567291in}}%
\pgfpathlineto{\pgfqpoint{1.247505in}{0.562321in}}%
\pgfpathlineto{\pgfqpoint{1.245186in}{0.557772in}}%
\pgfpathlineto{\pgfqpoint{1.239415in}{0.553594in}}%
\pgfpathlineto{\pgfqpoint{1.214527in}{0.539476in}}%
\pgfpathlineto{\pgfqpoint{1.212984in}{0.534843in}}%
\pgfpathlineto{\pgfqpoint{1.215317in}{0.529793in}}%
\pgfpathlineto{\pgfqpoint{1.220817in}{0.524402in}}%
\pgfpathlineto{\pgfqpoint{1.239734in}{0.507970in}}%
\pgfpathlineto{\pgfqpoint{1.241128in}{0.503021in}}%
\pgfpathlineto{\pgfqpoint{1.238617in}{0.498493in}}%
\pgfpathlineto{\pgfqpoint{1.232701in}{0.494330in}}%
\pgfpathlineto{\pgfqpoint{1.210098in}{0.482366in}}%
\pgfpathlineto{\pgfqpoint{1.206836in}{0.477918in}}%
\pgfpathlineto{\pgfqpoint{1.207456in}{0.473052in}}%
\pgfpathlineto{\pgfqpoint{1.211677in}{0.467799in}}%
\pgfpathlineto{\pgfqpoint{1.233538in}{0.448651in}}%
\pgfpathlineto{\pgfqpoint{1.234740in}{0.443723in}}%
\pgfpathlineto{\pgfqpoint{1.232039in}{0.439214in}}%
\pgfpathlineto{\pgfqpoint{1.225982in}{0.435067in}}%
\pgfpathlineto{\pgfqpoint{1.203495in}{0.423090in}}%
\pgfpathlineto{\pgfqpoint{1.200419in}{0.418622in}}%
\pgfpathlineto{\pgfqpoint{1.201234in}{0.413735in}}%
\pgfpathlineto{\pgfqpoint{1.205609in}{0.408466in}}%
\pgfpathlineto{\pgfqpoint{1.227333in}{0.389332in}}%
\pgfpathlineto{\pgfqpoint{1.228342in}{0.384425in}}%
\pgfpathlineto{\pgfqpoint{1.225453in}{0.379937in}}%
\pgfpathlineto{\pgfqpoint{1.219259in}{0.375804in}}%
\pgfpathlineto{\pgfqpoint{1.211177in}{0.371875in}}%
\pgfpathlineto{\pgfqpoint{1.211177in}{0.371875in}}%
\pgfusepath{stroke}%
\end{pgfscope}%
\begin{pgfscope}%
\pgfpathrectangle{\pgfqpoint{0.148125in}{0.175000in}}{\pgfqpoint{1.653750in}{3.150000in}}%
\pgfusepath{clip}%
\pgfsetrectcap%
\pgfsetroundjoin%
\pgfsetlinewidth{1.003750pt}%
\definecolor{currentstroke}{rgb}{0.000000,0.000000,0.000000}%
\pgfsetstrokecolor{currentstroke}%
\pgfsetdash{}{0pt}%
\pgfpathmoveto{\pgfqpoint{1.274724in}{0.962500in}}%
\pgfpathlineto{\pgfqpoint{1.338272in}{0.371875in}}%
\pgfusepath{stroke}%
\end{pgfscope}%
\begin{pgfscope}%
\pgfpathrectangle{\pgfqpoint{0.148125in}{0.175000in}}{\pgfqpoint{1.653750in}{3.150000in}}%
\pgfusepath{clip}%
\pgfsetrectcap%
\pgfsetroundjoin%
\pgfsetlinewidth{1.003750pt}%
\definecolor{currentstroke}{rgb}{0.000000,0.000000,0.000000}%
\pgfsetstrokecolor{currentstroke}%
\pgfsetdash{}{0pt}%
\pgfpathmoveto{\pgfqpoint{1.462776in}{0.962500in}}%
\pgfpathlineto{\pgfqpoint{1.399228in}{0.371875in}}%
\pgfusepath{stroke}%
\end{pgfscope}%
\begin{pgfscope}%
\pgfpathrectangle{\pgfqpoint{0.148125in}{0.175000in}}{\pgfqpoint{1.653750in}{3.150000in}}%
\pgfusepath{clip}%
\pgfsetrectcap%
\pgfsetroundjoin%
\pgfsetlinewidth{1.003750pt}%
\definecolor{currentstroke}{rgb}{0.000000,0.000000,0.000000}%
\pgfsetstrokecolor{currentstroke}%
\pgfsetdash{}{0pt}%
\pgfpathmoveto{\pgfqpoint{1.462776in}{0.962500in}}%
\pgfpathlineto{\pgfqpoint{1.526323in}{0.371875in}}%
\pgfusepath{stroke}%
\end{pgfscope}%
\begin{pgfscope}%
\pgfpathrectangle{\pgfqpoint{0.148125in}{0.175000in}}{\pgfqpoint{1.653750in}{3.150000in}}%
\pgfusepath{clip}%
\pgfsetrectcap%
\pgfsetroundjoin%
\pgfsetlinewidth{1.003750pt}%
\definecolor{currentstroke}{rgb}{0.000000,0.000000,0.000000}%
\pgfsetstrokecolor{currentstroke}%
\pgfsetdash{}{0pt}%
\pgfpathmoveto{\pgfqpoint{1.668474in}{0.962500in}}%
\pgfpathlineto{\pgfqpoint{1.656954in}{0.956541in}}%
\pgfpathlineto{\pgfqpoint{1.652281in}{0.952245in}}%
\pgfpathlineto{\pgfqpoint{1.651326in}{0.947549in}}%
\pgfpathlineto{\pgfqpoint{1.654202in}{0.942441in}}%
\pgfpathlineto{\pgfqpoint{1.663637in}{0.934227in}}%
\pgfpathlineto{\pgfqpoint{1.676616in}{0.923233in}}%
\pgfpathlineto{\pgfqpoint{1.679313in}{0.918144in}}%
\pgfpathlineto{\pgfqpoint{1.678162in}{0.913469in}}%
\pgfpathlineto{\pgfqpoint{1.673323in}{0.909190in}}%
\pgfpathlineto{\pgfqpoint{1.657547in}{0.901290in}}%
\pgfpathlineto{\pgfqpoint{1.647611in}{0.895161in}}%
\pgfpathlineto{\pgfqpoint{1.644911in}{0.890652in}}%
\pgfpathlineto{\pgfqpoint{1.646113in}{0.885724in}}%
\pgfpathlineto{\pgfqpoint{1.650790in}{0.880422in}}%
\pgfpathlineto{\pgfqpoint{1.672194in}{0.861323in}}%
\pgfpathlineto{\pgfqpoint{1.672814in}{0.856457in}}%
\pgfpathlineto{\pgfqpoint{1.669552in}{0.852009in}}%
\pgfpathlineto{\pgfqpoint{1.663095in}{0.847905in}}%
\pgfpathlineto{\pgfqpoint{1.641033in}{0.835882in}}%
\pgfpathlineto{\pgfqpoint{1.638523in}{0.831354in}}%
\pgfpathlineto{\pgfqpoint{1.639917in}{0.826405in}}%
\pgfpathlineto{\pgfqpoint{1.644741in}{0.821087in}}%
\pgfpathlineto{\pgfqpoint{1.665962in}{0.802007in}}%
\pgfpathlineto{\pgfqpoint{1.666387in}{0.797163in}}%
\pgfpathlineto{\pgfqpoint{1.662941in}{0.792735in}}%
\pgfpathlineto{\pgfqpoint{1.652340in}{0.786677in}}%
\pgfpathlineto{\pgfqpoint{1.636983in}{0.778731in}}%
\pgfpathlineto{\pgfqpoint{1.632822in}{0.774380in}}%
\pgfpathlineto{\pgfqpoint{1.632460in}{0.769620in}}%
\pgfpathlineto{\pgfqpoint{1.635859in}{0.764456in}}%
\pgfpathlineto{\pgfqpoint{1.645678in}{0.756201in}}%
\pgfpathlineto{\pgfqpoint{1.655821in}{0.747911in}}%
\pgfpathlineto{\pgfqpoint{1.659721in}{0.742693in}}%
\pgfpathlineto{\pgfqpoint{1.659949in}{0.737869in}}%
\pgfpathlineto{\pgfqpoint{1.656322in}{0.733461in}}%
\pgfpathlineto{\pgfqpoint{1.645570in}{0.727419in}}%
\pgfpathlineto{\pgfqpoint{1.630341in}{0.719460in}}%
\pgfpathlineto{\pgfqpoint{1.626355in}{0.715090in}}%
\pgfpathlineto{\pgfqpoint{1.626190in}{0.710309in}}%
\pgfpathlineto{\pgfqpoint{1.629759in}{0.705126in}}%
\pgfpathlineto{\pgfqpoint{1.639693in}{0.696859in}}%
\pgfpathlineto{\pgfqpoint{1.649735in}{0.688580in}}%
\pgfpathlineto{\pgfqpoint{1.653470in}{0.683379in}}%
\pgfpathlineto{\pgfqpoint{1.653502in}{0.678577in}}%
\pgfpathlineto{\pgfqpoint{1.649695in}{0.674187in}}%
\pgfpathlineto{\pgfqpoint{1.638798in}{0.668161in}}%
\pgfpathlineto{\pgfqpoint{1.623706in}{0.660188in}}%
\pgfpathlineto{\pgfqpoint{1.619898in}{0.655798in}}%
\pgfpathlineto{\pgfqpoint{1.619930in}{0.650996in}}%
\pgfpathlineto{\pgfqpoint{1.623666in}{0.645795in}}%
\pgfpathlineto{\pgfqpoint{1.633708in}{0.637516in}}%
\pgfpathlineto{\pgfqpoint{1.643642in}{0.629249in}}%
\pgfpathlineto{\pgfqpoint{1.647210in}{0.624066in}}%
\pgfpathlineto{\pgfqpoint{1.647045in}{0.619285in}}%
\pgfpathlineto{\pgfqpoint{1.643060in}{0.614915in}}%
\pgfpathlineto{\pgfqpoint{1.632026in}{0.608904in}}%
\pgfpathlineto{\pgfqpoint{1.617079in}{0.600914in}}%
\pgfpathlineto{\pgfqpoint{1.613451in}{0.596506in}}%
\pgfpathlineto{\pgfqpoint{1.613679in}{0.591682in}}%
\pgfpathlineto{\pgfqpoint{1.617580in}{0.586464in}}%
\pgfpathlineto{\pgfqpoint{1.631355in}{0.575384in}}%
\pgfpathlineto{\pgfqpoint{1.639671in}{0.567291in}}%
\pgfpathlineto{\pgfqpoint{1.641255in}{0.562321in}}%
\pgfpathlineto{\pgfqpoint{1.638936in}{0.557772in}}%
\pgfpathlineto{\pgfqpoint{1.633165in}{0.553594in}}%
\pgfpathlineto{\pgfqpoint{1.608277in}{0.539476in}}%
\pgfpathlineto{\pgfqpoint{1.606734in}{0.534843in}}%
\pgfpathlineto{\pgfqpoint{1.609067in}{0.529793in}}%
\pgfpathlineto{\pgfqpoint{1.614567in}{0.524402in}}%
\pgfpathlineto{\pgfqpoint{1.633484in}{0.507970in}}%
\pgfpathlineto{\pgfqpoint{1.634878in}{0.503021in}}%
\pgfpathlineto{\pgfqpoint{1.632367in}{0.498493in}}%
\pgfpathlineto{\pgfqpoint{1.626451in}{0.494330in}}%
\pgfpathlineto{\pgfqpoint{1.603848in}{0.482366in}}%
\pgfpathlineto{\pgfqpoint{1.600586in}{0.477918in}}%
\pgfpathlineto{\pgfqpoint{1.601206in}{0.473052in}}%
\pgfpathlineto{\pgfqpoint{1.605427in}{0.467799in}}%
\pgfpathlineto{\pgfqpoint{1.627288in}{0.448651in}}%
\pgfpathlineto{\pgfqpoint{1.628490in}{0.443723in}}%
\pgfpathlineto{\pgfqpoint{1.625789in}{0.439214in}}%
\pgfpathlineto{\pgfqpoint{1.619732in}{0.435067in}}%
\pgfpathlineto{\pgfqpoint{1.597245in}{0.423090in}}%
\pgfpathlineto{\pgfqpoint{1.594169in}{0.418622in}}%
\pgfpathlineto{\pgfqpoint{1.594984in}{0.413735in}}%
\pgfpathlineto{\pgfqpoint{1.599359in}{0.408466in}}%
\pgfpathlineto{\pgfqpoint{1.621083in}{0.389332in}}%
\pgfpathlineto{\pgfqpoint{1.622092in}{0.384425in}}%
\pgfpathlineto{\pgfqpoint{1.619203in}{0.379937in}}%
\pgfpathlineto{\pgfqpoint{1.613009in}{0.375804in}}%
\pgfpathlineto{\pgfqpoint{1.604927in}{0.371875in}}%
\pgfpathlineto{\pgfqpoint{1.604927in}{0.371875in}}%
\pgfusepath{stroke}%
\end{pgfscope}%
\begin{pgfscope}%
\pgfpathrectangle{\pgfqpoint{0.148125in}{0.175000in}}{\pgfqpoint{1.653750in}{3.150000in}}%
\pgfusepath{clip}%
\pgfsetrectcap%
\pgfsetroundjoin%
\pgfsetlinewidth{1.003750pt}%
\definecolor{currentstroke}{rgb}{0.000000,0.000000,0.000000}%
\pgfsetstrokecolor{currentstroke}%
\pgfsetdash{}{0pt}%
\pgfpathmoveto{\pgfqpoint{1.668474in}{0.962500in}}%
\pgfpathlineto{\pgfqpoint{1.732022in}{0.371875in}}%
\pgfusepath{stroke}%
\end{pgfscope}%
\begin{pgfscope}%
\pgfpathrectangle{\pgfqpoint{0.148125in}{0.175000in}}{\pgfqpoint{1.653750in}{3.150000in}}%
\pgfusepath{clip}%
\pgfsetbuttcap%
\pgfsetroundjoin%
\pgfsetlinewidth{1.003750pt}%
\definecolor{currentstroke}{rgb}{0.827451,0.827451,0.827451}%
\pgfsetstrokecolor{currentstroke}%
\pgfsetdash{{3.700000pt}{1.600000pt}}{0.000000pt}%
\pgfpathmoveto{\pgfqpoint{0.148125in}{0.962500in}}%
\pgfpathlineto{\pgfqpoint{1.801875in}{0.962500in}}%
\pgfusepath{stroke}%
\end{pgfscope}%
\begin{pgfscope}%
\definecolor{textcolor}{rgb}{0.000000,0.000000,0.000000}%
\pgfsetstrokecolor{textcolor}%
\pgfsetfillcolor{textcolor}%
\pgftext[x=1.014375in,y=3.029688in,left,]{\color{textcolor}\rmfamily\fontsize{9.000000}{10.800000}\selectfont \(\displaystyle e^{-}\)}%
\end{pgfscope}%
\begin{pgfscope}%
\definecolor{textcolor}{rgb}{0.000000,0.000000,0.000000}%
\pgfsetstrokecolor{textcolor}%
\pgfsetfillcolor{textcolor}%
\pgftext[x=1.683750in,y=2.750125in,left,base]{\color{textcolor}\rmfamily\fontsize{9.000000}{10.800000}\selectfont \(\displaystyle t = 1\)}%
\end{pgfscope}%
\begin{pgfscope}%
\definecolor{textcolor}{rgb}{0.000000,0.000000,0.000000}%
\pgfsetstrokecolor{textcolor}%
\pgfsetfillcolor{textcolor}%
\pgftext[x=0.620625in,y=2.439062in,left,]{\color{textcolor}\rmfamily\fontsize{9.000000}{10.800000}\selectfont \(\displaystyle \gamma\)}%
\end{pgfscope}%
\begin{pgfscope}%
\definecolor{textcolor}{rgb}{0.000000,0.000000,0.000000}%
\pgfsetstrokecolor{textcolor}%
\pgfsetfillcolor{textcolor}%
\pgftext[x=1.683750in,y=2.159500in,left,base]{\color{textcolor}\rmfamily\fontsize{9.000000}{10.800000}\selectfont \(\displaystyle t = 2\)}%
\end{pgfscope}%
\begin{pgfscope}%
\definecolor{textcolor}{rgb}{0.000000,0.000000,0.000000}%
\pgfsetstrokecolor{textcolor}%
\pgfsetfillcolor{textcolor}%
\pgftext[x=0.285938in,y=1.848437in,left,]{\color{textcolor}\rmfamily\fontsize{9.000000}{10.800000}\selectfont \(\displaystyle e^+\)}%
\end{pgfscope}%
\begin{pgfscope}%
\definecolor{textcolor}{rgb}{0.000000,0.000000,0.000000}%
\pgfsetstrokecolor{textcolor}%
\pgfsetfillcolor{textcolor}%
\pgftext[x=0.738750in,y=1.848437in,left,]{\color{textcolor}\rmfamily\fontsize{9.000000}{10.800000}\selectfont \(\displaystyle e^-\)}%
\end{pgfscope}%
\begin{pgfscope}%
\definecolor{textcolor}{rgb}{0.000000,0.000000,0.000000}%
\pgfsetstrokecolor{textcolor}%
\pgfsetfillcolor{textcolor}%
\pgftext[x=1.683750in,y=1.568875in,left,base]{\color{textcolor}\rmfamily\fontsize{9.000000}{10.800000}\selectfont \(\displaystyle t = 3\)}%
\end{pgfscope}%
\begin{pgfscope}%
\definecolor{textcolor}{rgb}{0.000000,0.000000,0.000000}%
\pgfsetstrokecolor{textcolor}%
\pgfsetfillcolor{textcolor}%
\pgftext[x=1.683750in,y=0.978250in,left,base]{\color{textcolor}\rmfamily\fontsize{9.000000}{10.800000}\selectfont \(\displaystyle t = 4\)}%
\end{pgfscope}%
\end{pgfpicture}%
\makeatother%
\endgroup%

  \caption{Sketch of the development of an electron-induced electromagnetic shower
  in the simplest possible toy model.}
  \label{fig:em_shower_naive}
\end{marginfigure}

A simple toy model for an electron-initiated shower is schematically represented
in figure~\ref{fig:em_shower_naive}, where after $t$ radiation lengths of material,
the cascade has developed in $2^t$ particles (a mix of electrons, positrons and
photons, roughly in equal proportions) with an average energy of $\nicefrac{E_0}{2^t}$.
The process continue until the latter reach the critical energy in the medium,
at which point radiation losses become subdominant and the remaining energy
degradation proceed by ionization.
It goes without saying that this naive schematization is neglecting many of the
salient features of the development of a real shower: the intrinsic stochasticity
of the process, the the $\nicefrac{9}{7}$ term in~\eqref{eq:lambda_pair}, the fact
that the electron can---and does---radiate multiple photons of different energy
per radiation length, and the ionization losses, which are at play even when the
radiation losses dominate. And yet, rough as this model is, it is able to reproduce
one of the main features of electromagnetic showers, namely the fact that the
position of the shower maximum scales logarithmically with the energy. In fact the
stop condition reads
\begin{align}
  \frac{E_0}{2^{t_\text{max}}} \approx E_c
  \quad\text{or}\quad
  t_\text{max} \approx \ln \left(\frac{E_0}{E_c}\right).
\end{align}

\begin{marginfigure}
  %% Creator: Matplotlib, PGF backend
%%
%% To include the figure in your LaTeX document, write
%%   \input{<filename>.pgf}
%%
%% Make sure the required packages are loaded in your preamble
%%   \usepackage{pgf}
%%
%% Also ensure that all the required font packages are loaded; for instance,
%% the lmodern package is sometimes necessary when using math font.
%%   \usepackage{lmodern}
%%
%% Figures using additional raster images can only be included by \input if
%% they are in the same directory as the main LaTeX file. For loading figures
%% from other directories you can use the `import` package
%%   \usepackage{import}
%%
%% and then include the figures with
%%   \import{<path to file>}{<filename>.pgf}
%%
%% Matplotlib used the following preamble
%%   \usepackage{fontspec}
%%   \setmainfont{DejaVuSerif.ttf}[Path=\detokenize{/usr/share/matplotlib/mpl-data/fonts/ttf/}]
%%   \setsansfont{DejaVuSans.ttf}[Path=\detokenize{/usr/share/matplotlib/mpl-data/fonts/ttf/}]
%%   \setmonofont{DejaVuSansMono.ttf}[Path=\detokenize{/usr/share/matplotlib/mpl-data/fonts/ttf/}]
%%
\begingroup%
\makeatletter%
\begin{pgfpicture}%
\pgfpathrectangle{\pgfpointorigin}{\pgfqpoint{1.950000in}{2.250000in}}%
\pgfusepath{use as bounding box, clip}%
\begin{pgfscope}%
\pgfsetbuttcap%
\pgfsetmiterjoin%
\definecolor{currentfill}{rgb}{1.000000,1.000000,1.000000}%
\pgfsetfillcolor{currentfill}%
\pgfsetlinewidth{0.000000pt}%
\definecolor{currentstroke}{rgb}{1.000000,1.000000,1.000000}%
\pgfsetstrokecolor{currentstroke}%
\pgfsetdash{}{0pt}%
\pgfpathmoveto{\pgfqpoint{0.000000in}{0.000000in}}%
\pgfpathlineto{\pgfqpoint{1.950000in}{0.000000in}}%
\pgfpathlineto{\pgfqpoint{1.950000in}{2.250000in}}%
\pgfpathlineto{\pgfqpoint{0.000000in}{2.250000in}}%
\pgfpathlineto{\pgfqpoint{0.000000in}{0.000000in}}%
\pgfpathclose%
\pgfusepath{fill}%
\end{pgfscope}%
\begin{pgfscope}%
\pgfsetbuttcap%
\pgfsetmiterjoin%
\definecolor{currentfill}{rgb}{1.000000,1.000000,1.000000}%
\pgfsetfillcolor{currentfill}%
\pgfsetlinewidth{0.000000pt}%
\definecolor{currentstroke}{rgb}{0.000000,0.000000,0.000000}%
\pgfsetstrokecolor{currentstroke}%
\pgfsetstrokeopacity{0.000000}%
\pgfsetdash{}{0pt}%
\pgfpathmoveto{\pgfqpoint{0.726250in}{0.525000in}}%
\pgfpathlineto{\pgfqpoint{1.846250in}{0.525000in}}%
\pgfpathlineto{\pgfqpoint{1.846250in}{2.162500in}}%
\pgfpathlineto{\pgfqpoint{0.726250in}{2.162500in}}%
\pgfpathlineto{\pgfqpoint{0.726250in}{0.525000in}}%
\pgfpathclose%
\pgfusepath{fill}%
\end{pgfscope}%
\begin{pgfscope}%
\pgfpathrectangle{\pgfqpoint{0.726250in}{0.525000in}}{\pgfqpoint{1.120000in}{1.637500in}}%
\pgfusepath{clip}%
\pgfsetbuttcap%
\pgfsetroundjoin%
\pgfsetlinewidth{0.803000pt}%
\definecolor{currentstroke}{rgb}{0.752941,0.752941,0.752941}%
\pgfsetstrokecolor{currentstroke}%
\pgfsetdash{{2.960000pt}{1.280000pt}}{0.000000pt}%
\pgfpathmoveto{\pgfqpoint{0.726250in}{0.525000in}}%
\pgfpathlineto{\pgfqpoint{0.726250in}{2.162500in}}%
\pgfusepath{stroke}%
\end{pgfscope}%
\begin{pgfscope}%
\pgfsetbuttcap%
\pgfsetroundjoin%
\definecolor{currentfill}{rgb}{0.000000,0.000000,0.000000}%
\pgfsetfillcolor{currentfill}%
\pgfsetlinewidth{0.803000pt}%
\definecolor{currentstroke}{rgb}{0.000000,0.000000,0.000000}%
\pgfsetstrokecolor{currentstroke}%
\pgfsetdash{}{0pt}%
\pgfsys@defobject{currentmarker}{\pgfqpoint{0.000000in}{-0.048611in}}{\pgfqpoint{0.000000in}{0.000000in}}{%
\pgfpathmoveto{\pgfqpoint{0.000000in}{0.000000in}}%
\pgfpathlineto{\pgfqpoint{0.000000in}{-0.048611in}}%
\pgfusepath{stroke,fill}%
}%
\begin{pgfscope}%
\pgfsys@transformshift{0.726250in}{0.525000in}%
\pgfsys@useobject{currentmarker}{}%
\end{pgfscope}%
\end{pgfscope}%
\begin{pgfscope}%
\definecolor{textcolor}{rgb}{0.000000,0.000000,0.000000}%
\pgfsetstrokecolor{textcolor}%
\pgfsetfillcolor{textcolor}%
\pgftext[x=0.726250in,y=0.427778in,,top]{\color{textcolor}\rmfamily\fontsize{9.000000}{10.800000}\selectfont 0}%
\end{pgfscope}%
\begin{pgfscope}%
\pgfpathrectangle{\pgfqpoint{0.726250in}{0.525000in}}{\pgfqpoint{1.120000in}{1.637500in}}%
\pgfusepath{clip}%
\pgfsetbuttcap%
\pgfsetroundjoin%
\pgfsetlinewidth{0.803000pt}%
\definecolor{currentstroke}{rgb}{0.752941,0.752941,0.752941}%
\pgfsetstrokecolor{currentstroke}%
\pgfsetdash{{2.960000pt}{1.280000pt}}{0.000000pt}%
\pgfpathmoveto{\pgfqpoint{0.912917in}{0.525000in}}%
\pgfpathlineto{\pgfqpoint{0.912917in}{2.162500in}}%
\pgfusepath{stroke}%
\end{pgfscope}%
\begin{pgfscope}%
\pgfsetbuttcap%
\pgfsetroundjoin%
\definecolor{currentfill}{rgb}{0.000000,0.000000,0.000000}%
\pgfsetfillcolor{currentfill}%
\pgfsetlinewidth{0.803000pt}%
\definecolor{currentstroke}{rgb}{0.000000,0.000000,0.000000}%
\pgfsetstrokecolor{currentstroke}%
\pgfsetdash{}{0pt}%
\pgfsys@defobject{currentmarker}{\pgfqpoint{0.000000in}{-0.048611in}}{\pgfqpoint{0.000000in}{0.000000in}}{%
\pgfpathmoveto{\pgfqpoint{0.000000in}{0.000000in}}%
\pgfpathlineto{\pgfqpoint{0.000000in}{-0.048611in}}%
\pgfusepath{stroke,fill}%
}%
\begin{pgfscope}%
\pgfsys@transformshift{0.912917in}{0.525000in}%
\pgfsys@useobject{currentmarker}{}%
\end{pgfscope}%
\end{pgfscope}%
\begin{pgfscope}%
\definecolor{textcolor}{rgb}{0.000000,0.000000,0.000000}%
\pgfsetstrokecolor{textcolor}%
\pgfsetfillcolor{textcolor}%
\pgftext[x=0.912917in,y=0.427778in,,top]{\color{textcolor}\rmfamily\fontsize{9.000000}{10.800000}\selectfont 5}%
\end{pgfscope}%
\begin{pgfscope}%
\pgfpathrectangle{\pgfqpoint{0.726250in}{0.525000in}}{\pgfqpoint{1.120000in}{1.637500in}}%
\pgfusepath{clip}%
\pgfsetbuttcap%
\pgfsetroundjoin%
\pgfsetlinewidth{0.803000pt}%
\definecolor{currentstroke}{rgb}{0.752941,0.752941,0.752941}%
\pgfsetstrokecolor{currentstroke}%
\pgfsetdash{{2.960000pt}{1.280000pt}}{0.000000pt}%
\pgfpathmoveto{\pgfqpoint{1.099583in}{0.525000in}}%
\pgfpathlineto{\pgfqpoint{1.099583in}{2.162500in}}%
\pgfusepath{stroke}%
\end{pgfscope}%
\begin{pgfscope}%
\pgfsetbuttcap%
\pgfsetroundjoin%
\definecolor{currentfill}{rgb}{0.000000,0.000000,0.000000}%
\pgfsetfillcolor{currentfill}%
\pgfsetlinewidth{0.803000pt}%
\definecolor{currentstroke}{rgb}{0.000000,0.000000,0.000000}%
\pgfsetstrokecolor{currentstroke}%
\pgfsetdash{}{0pt}%
\pgfsys@defobject{currentmarker}{\pgfqpoint{0.000000in}{-0.048611in}}{\pgfqpoint{0.000000in}{0.000000in}}{%
\pgfpathmoveto{\pgfqpoint{0.000000in}{0.000000in}}%
\pgfpathlineto{\pgfqpoint{0.000000in}{-0.048611in}}%
\pgfusepath{stroke,fill}%
}%
\begin{pgfscope}%
\pgfsys@transformshift{1.099583in}{0.525000in}%
\pgfsys@useobject{currentmarker}{}%
\end{pgfscope}%
\end{pgfscope}%
\begin{pgfscope}%
\definecolor{textcolor}{rgb}{0.000000,0.000000,0.000000}%
\pgfsetstrokecolor{textcolor}%
\pgfsetfillcolor{textcolor}%
\pgftext[x=1.099583in,y=0.427778in,,top]{\color{textcolor}\rmfamily\fontsize{9.000000}{10.800000}\selectfont 10}%
\end{pgfscope}%
\begin{pgfscope}%
\pgfpathrectangle{\pgfqpoint{0.726250in}{0.525000in}}{\pgfqpoint{1.120000in}{1.637500in}}%
\pgfusepath{clip}%
\pgfsetbuttcap%
\pgfsetroundjoin%
\pgfsetlinewidth{0.803000pt}%
\definecolor{currentstroke}{rgb}{0.752941,0.752941,0.752941}%
\pgfsetstrokecolor{currentstroke}%
\pgfsetdash{{2.960000pt}{1.280000pt}}{0.000000pt}%
\pgfpathmoveto{\pgfqpoint{1.286250in}{0.525000in}}%
\pgfpathlineto{\pgfqpoint{1.286250in}{2.162500in}}%
\pgfusepath{stroke}%
\end{pgfscope}%
\begin{pgfscope}%
\pgfsetbuttcap%
\pgfsetroundjoin%
\definecolor{currentfill}{rgb}{0.000000,0.000000,0.000000}%
\pgfsetfillcolor{currentfill}%
\pgfsetlinewidth{0.803000pt}%
\definecolor{currentstroke}{rgb}{0.000000,0.000000,0.000000}%
\pgfsetstrokecolor{currentstroke}%
\pgfsetdash{}{0pt}%
\pgfsys@defobject{currentmarker}{\pgfqpoint{0.000000in}{-0.048611in}}{\pgfqpoint{0.000000in}{0.000000in}}{%
\pgfpathmoveto{\pgfqpoint{0.000000in}{0.000000in}}%
\pgfpathlineto{\pgfqpoint{0.000000in}{-0.048611in}}%
\pgfusepath{stroke,fill}%
}%
\begin{pgfscope}%
\pgfsys@transformshift{1.286250in}{0.525000in}%
\pgfsys@useobject{currentmarker}{}%
\end{pgfscope}%
\end{pgfscope}%
\begin{pgfscope}%
\definecolor{textcolor}{rgb}{0.000000,0.000000,0.000000}%
\pgfsetstrokecolor{textcolor}%
\pgfsetfillcolor{textcolor}%
\pgftext[x=1.286250in,y=0.427778in,,top]{\color{textcolor}\rmfamily\fontsize{9.000000}{10.800000}\selectfont 15}%
\end{pgfscope}%
\begin{pgfscope}%
\pgfpathrectangle{\pgfqpoint{0.726250in}{0.525000in}}{\pgfqpoint{1.120000in}{1.637500in}}%
\pgfusepath{clip}%
\pgfsetbuttcap%
\pgfsetroundjoin%
\pgfsetlinewidth{0.803000pt}%
\definecolor{currentstroke}{rgb}{0.752941,0.752941,0.752941}%
\pgfsetstrokecolor{currentstroke}%
\pgfsetdash{{2.960000pt}{1.280000pt}}{0.000000pt}%
\pgfpathmoveto{\pgfqpoint{1.472917in}{0.525000in}}%
\pgfpathlineto{\pgfqpoint{1.472917in}{2.162500in}}%
\pgfusepath{stroke}%
\end{pgfscope}%
\begin{pgfscope}%
\pgfsetbuttcap%
\pgfsetroundjoin%
\definecolor{currentfill}{rgb}{0.000000,0.000000,0.000000}%
\pgfsetfillcolor{currentfill}%
\pgfsetlinewidth{0.803000pt}%
\definecolor{currentstroke}{rgb}{0.000000,0.000000,0.000000}%
\pgfsetstrokecolor{currentstroke}%
\pgfsetdash{}{0pt}%
\pgfsys@defobject{currentmarker}{\pgfqpoint{0.000000in}{-0.048611in}}{\pgfqpoint{0.000000in}{0.000000in}}{%
\pgfpathmoveto{\pgfqpoint{0.000000in}{0.000000in}}%
\pgfpathlineto{\pgfqpoint{0.000000in}{-0.048611in}}%
\pgfusepath{stroke,fill}%
}%
\begin{pgfscope}%
\pgfsys@transformshift{1.472917in}{0.525000in}%
\pgfsys@useobject{currentmarker}{}%
\end{pgfscope}%
\end{pgfscope}%
\begin{pgfscope}%
\definecolor{textcolor}{rgb}{0.000000,0.000000,0.000000}%
\pgfsetstrokecolor{textcolor}%
\pgfsetfillcolor{textcolor}%
\pgftext[x=1.472917in,y=0.427778in,,top]{\color{textcolor}\rmfamily\fontsize{9.000000}{10.800000}\selectfont 20}%
\end{pgfscope}%
\begin{pgfscope}%
\pgfpathrectangle{\pgfqpoint{0.726250in}{0.525000in}}{\pgfqpoint{1.120000in}{1.637500in}}%
\pgfusepath{clip}%
\pgfsetbuttcap%
\pgfsetroundjoin%
\pgfsetlinewidth{0.803000pt}%
\definecolor{currentstroke}{rgb}{0.752941,0.752941,0.752941}%
\pgfsetstrokecolor{currentstroke}%
\pgfsetdash{{2.960000pt}{1.280000pt}}{0.000000pt}%
\pgfpathmoveto{\pgfqpoint{1.659583in}{0.525000in}}%
\pgfpathlineto{\pgfqpoint{1.659583in}{2.162500in}}%
\pgfusepath{stroke}%
\end{pgfscope}%
\begin{pgfscope}%
\pgfsetbuttcap%
\pgfsetroundjoin%
\definecolor{currentfill}{rgb}{0.000000,0.000000,0.000000}%
\pgfsetfillcolor{currentfill}%
\pgfsetlinewidth{0.803000pt}%
\definecolor{currentstroke}{rgb}{0.000000,0.000000,0.000000}%
\pgfsetstrokecolor{currentstroke}%
\pgfsetdash{}{0pt}%
\pgfsys@defobject{currentmarker}{\pgfqpoint{0.000000in}{-0.048611in}}{\pgfqpoint{0.000000in}{0.000000in}}{%
\pgfpathmoveto{\pgfqpoint{0.000000in}{0.000000in}}%
\pgfpathlineto{\pgfqpoint{0.000000in}{-0.048611in}}%
\pgfusepath{stroke,fill}%
}%
\begin{pgfscope}%
\pgfsys@transformshift{1.659583in}{0.525000in}%
\pgfsys@useobject{currentmarker}{}%
\end{pgfscope}%
\end{pgfscope}%
\begin{pgfscope}%
\definecolor{textcolor}{rgb}{0.000000,0.000000,0.000000}%
\pgfsetstrokecolor{textcolor}%
\pgfsetfillcolor{textcolor}%
\pgftext[x=1.659583in,y=0.427778in,,top]{\color{textcolor}\rmfamily\fontsize{9.000000}{10.800000}\selectfont 25}%
\end{pgfscope}%
\begin{pgfscope}%
\pgfpathrectangle{\pgfqpoint{0.726250in}{0.525000in}}{\pgfqpoint{1.120000in}{1.637500in}}%
\pgfusepath{clip}%
\pgfsetbuttcap%
\pgfsetroundjoin%
\pgfsetlinewidth{0.803000pt}%
\definecolor{currentstroke}{rgb}{0.752941,0.752941,0.752941}%
\pgfsetstrokecolor{currentstroke}%
\pgfsetdash{{2.960000pt}{1.280000pt}}{0.000000pt}%
\pgfpathmoveto{\pgfqpoint{1.846250in}{0.525000in}}%
\pgfpathlineto{\pgfqpoint{1.846250in}{2.162500in}}%
\pgfusepath{stroke}%
\end{pgfscope}%
\begin{pgfscope}%
\pgfsetbuttcap%
\pgfsetroundjoin%
\definecolor{currentfill}{rgb}{0.000000,0.000000,0.000000}%
\pgfsetfillcolor{currentfill}%
\pgfsetlinewidth{0.803000pt}%
\definecolor{currentstroke}{rgb}{0.000000,0.000000,0.000000}%
\pgfsetstrokecolor{currentstroke}%
\pgfsetdash{}{0pt}%
\pgfsys@defobject{currentmarker}{\pgfqpoint{0.000000in}{-0.048611in}}{\pgfqpoint{0.000000in}{0.000000in}}{%
\pgfpathmoveto{\pgfqpoint{0.000000in}{0.000000in}}%
\pgfpathlineto{\pgfqpoint{0.000000in}{-0.048611in}}%
\pgfusepath{stroke,fill}%
}%
\begin{pgfscope}%
\pgfsys@transformshift{1.846250in}{0.525000in}%
\pgfsys@useobject{currentmarker}{}%
\end{pgfscope}%
\end{pgfscope}%
\begin{pgfscope}%
\definecolor{textcolor}{rgb}{0.000000,0.000000,0.000000}%
\pgfsetstrokecolor{textcolor}%
\pgfsetfillcolor{textcolor}%
\pgftext[x=1.846250in,y=0.427778in,,top]{\color{textcolor}\rmfamily\fontsize{9.000000}{10.800000}\selectfont 30}%
\end{pgfscope}%
\begin{pgfscope}%
\definecolor{textcolor}{rgb}{0.000000,0.000000,0.000000}%
\pgfsetstrokecolor{textcolor}%
\pgfsetfillcolor{textcolor}%
\pgftext[x=1.286250in,y=0.251251in,,top]{\color{textcolor}\rmfamily\fontsize{9.000000}{10.800000}\selectfont \(\displaystyle t\) [\(\displaystyle X_0\)]}%
\end{pgfscope}%
\begin{pgfscope}%
\pgfpathrectangle{\pgfqpoint{0.726250in}{0.525000in}}{\pgfqpoint{1.120000in}{1.637500in}}%
\pgfusepath{clip}%
\pgfsetbuttcap%
\pgfsetroundjoin%
\pgfsetlinewidth{0.803000pt}%
\definecolor{currentstroke}{rgb}{0.752941,0.752941,0.752941}%
\pgfsetstrokecolor{currentstroke}%
\pgfsetdash{{2.960000pt}{1.280000pt}}{0.000000pt}%
\pgfpathmoveto{\pgfqpoint{0.726250in}{0.525000in}}%
\pgfpathlineto{\pgfqpoint{1.846250in}{0.525000in}}%
\pgfusepath{stroke}%
\end{pgfscope}%
\begin{pgfscope}%
\pgfsetbuttcap%
\pgfsetroundjoin%
\definecolor{currentfill}{rgb}{0.000000,0.000000,0.000000}%
\pgfsetfillcolor{currentfill}%
\pgfsetlinewidth{0.803000pt}%
\definecolor{currentstroke}{rgb}{0.000000,0.000000,0.000000}%
\pgfsetstrokecolor{currentstroke}%
\pgfsetdash{}{0pt}%
\pgfsys@defobject{currentmarker}{\pgfqpoint{-0.048611in}{0.000000in}}{\pgfqpoint{-0.000000in}{0.000000in}}{%
\pgfpathmoveto{\pgfqpoint{-0.000000in}{0.000000in}}%
\pgfpathlineto{\pgfqpoint{-0.048611in}{0.000000in}}%
\pgfusepath{stroke,fill}%
}%
\begin{pgfscope}%
\pgfsys@transformshift{0.726250in}{0.525000in}%
\pgfsys@useobject{currentmarker}{}%
\end{pgfscope}%
\end{pgfscope}%
\begin{pgfscope}%
\definecolor{textcolor}{rgb}{0.000000,0.000000,0.000000}%
\pgfsetstrokecolor{textcolor}%
\pgfsetfillcolor{textcolor}%
\pgftext[x=0.350708in, y=0.477515in, left, base]{\color{textcolor}\rmfamily\fontsize{9.000000}{10.800000}\selectfont 0.00}%
\end{pgfscope}%
\begin{pgfscope}%
\pgfpathrectangle{\pgfqpoint{0.726250in}{0.525000in}}{\pgfqpoint{1.120000in}{1.637500in}}%
\pgfusepath{clip}%
\pgfsetbuttcap%
\pgfsetroundjoin%
\pgfsetlinewidth{0.803000pt}%
\definecolor{currentstroke}{rgb}{0.752941,0.752941,0.752941}%
\pgfsetstrokecolor{currentstroke}%
\pgfsetdash{{2.960000pt}{1.280000pt}}{0.000000pt}%
\pgfpathmoveto{\pgfqpoint{0.726250in}{0.758929in}}%
\pgfpathlineto{\pgfqpoint{1.846250in}{0.758929in}}%
\pgfusepath{stroke}%
\end{pgfscope}%
\begin{pgfscope}%
\pgfsetbuttcap%
\pgfsetroundjoin%
\definecolor{currentfill}{rgb}{0.000000,0.000000,0.000000}%
\pgfsetfillcolor{currentfill}%
\pgfsetlinewidth{0.803000pt}%
\definecolor{currentstroke}{rgb}{0.000000,0.000000,0.000000}%
\pgfsetstrokecolor{currentstroke}%
\pgfsetdash{}{0pt}%
\pgfsys@defobject{currentmarker}{\pgfqpoint{-0.048611in}{0.000000in}}{\pgfqpoint{-0.000000in}{0.000000in}}{%
\pgfpathmoveto{\pgfqpoint{-0.000000in}{0.000000in}}%
\pgfpathlineto{\pgfqpoint{-0.048611in}{0.000000in}}%
\pgfusepath{stroke,fill}%
}%
\begin{pgfscope}%
\pgfsys@transformshift{0.726250in}{0.758929in}%
\pgfsys@useobject{currentmarker}{}%
\end{pgfscope}%
\end{pgfscope}%
\begin{pgfscope}%
\definecolor{textcolor}{rgb}{0.000000,0.000000,0.000000}%
\pgfsetstrokecolor{textcolor}%
\pgfsetfillcolor{textcolor}%
\pgftext[x=0.350708in, y=0.711443in, left, base]{\color{textcolor}\rmfamily\fontsize{9.000000}{10.800000}\selectfont 0.02}%
\end{pgfscope}%
\begin{pgfscope}%
\pgfpathrectangle{\pgfqpoint{0.726250in}{0.525000in}}{\pgfqpoint{1.120000in}{1.637500in}}%
\pgfusepath{clip}%
\pgfsetbuttcap%
\pgfsetroundjoin%
\pgfsetlinewidth{0.803000pt}%
\definecolor{currentstroke}{rgb}{0.752941,0.752941,0.752941}%
\pgfsetstrokecolor{currentstroke}%
\pgfsetdash{{2.960000pt}{1.280000pt}}{0.000000pt}%
\pgfpathmoveto{\pgfqpoint{0.726250in}{0.992857in}}%
\pgfpathlineto{\pgfqpoint{1.846250in}{0.992857in}}%
\pgfusepath{stroke}%
\end{pgfscope}%
\begin{pgfscope}%
\pgfsetbuttcap%
\pgfsetroundjoin%
\definecolor{currentfill}{rgb}{0.000000,0.000000,0.000000}%
\pgfsetfillcolor{currentfill}%
\pgfsetlinewidth{0.803000pt}%
\definecolor{currentstroke}{rgb}{0.000000,0.000000,0.000000}%
\pgfsetstrokecolor{currentstroke}%
\pgfsetdash{}{0pt}%
\pgfsys@defobject{currentmarker}{\pgfqpoint{-0.048611in}{0.000000in}}{\pgfqpoint{-0.000000in}{0.000000in}}{%
\pgfpathmoveto{\pgfqpoint{-0.000000in}{0.000000in}}%
\pgfpathlineto{\pgfqpoint{-0.048611in}{0.000000in}}%
\pgfusepath{stroke,fill}%
}%
\begin{pgfscope}%
\pgfsys@transformshift{0.726250in}{0.992857in}%
\pgfsys@useobject{currentmarker}{}%
\end{pgfscope}%
\end{pgfscope}%
\begin{pgfscope}%
\definecolor{textcolor}{rgb}{0.000000,0.000000,0.000000}%
\pgfsetstrokecolor{textcolor}%
\pgfsetfillcolor{textcolor}%
\pgftext[x=0.350708in, y=0.945372in, left, base]{\color{textcolor}\rmfamily\fontsize{9.000000}{10.800000}\selectfont 0.04}%
\end{pgfscope}%
\begin{pgfscope}%
\pgfpathrectangle{\pgfqpoint{0.726250in}{0.525000in}}{\pgfqpoint{1.120000in}{1.637500in}}%
\pgfusepath{clip}%
\pgfsetbuttcap%
\pgfsetroundjoin%
\pgfsetlinewidth{0.803000pt}%
\definecolor{currentstroke}{rgb}{0.752941,0.752941,0.752941}%
\pgfsetstrokecolor{currentstroke}%
\pgfsetdash{{2.960000pt}{1.280000pt}}{0.000000pt}%
\pgfpathmoveto{\pgfqpoint{0.726250in}{1.226786in}}%
\pgfpathlineto{\pgfqpoint{1.846250in}{1.226786in}}%
\pgfusepath{stroke}%
\end{pgfscope}%
\begin{pgfscope}%
\pgfsetbuttcap%
\pgfsetroundjoin%
\definecolor{currentfill}{rgb}{0.000000,0.000000,0.000000}%
\pgfsetfillcolor{currentfill}%
\pgfsetlinewidth{0.803000pt}%
\definecolor{currentstroke}{rgb}{0.000000,0.000000,0.000000}%
\pgfsetstrokecolor{currentstroke}%
\pgfsetdash{}{0pt}%
\pgfsys@defobject{currentmarker}{\pgfqpoint{-0.048611in}{0.000000in}}{\pgfqpoint{-0.000000in}{0.000000in}}{%
\pgfpathmoveto{\pgfqpoint{-0.000000in}{0.000000in}}%
\pgfpathlineto{\pgfqpoint{-0.048611in}{0.000000in}}%
\pgfusepath{stroke,fill}%
}%
\begin{pgfscope}%
\pgfsys@transformshift{0.726250in}{1.226786in}%
\pgfsys@useobject{currentmarker}{}%
\end{pgfscope}%
\end{pgfscope}%
\begin{pgfscope}%
\definecolor{textcolor}{rgb}{0.000000,0.000000,0.000000}%
\pgfsetstrokecolor{textcolor}%
\pgfsetfillcolor{textcolor}%
\pgftext[x=0.350708in, y=1.179300in, left, base]{\color{textcolor}\rmfamily\fontsize{9.000000}{10.800000}\selectfont 0.06}%
\end{pgfscope}%
\begin{pgfscope}%
\pgfpathrectangle{\pgfqpoint{0.726250in}{0.525000in}}{\pgfqpoint{1.120000in}{1.637500in}}%
\pgfusepath{clip}%
\pgfsetbuttcap%
\pgfsetroundjoin%
\pgfsetlinewidth{0.803000pt}%
\definecolor{currentstroke}{rgb}{0.752941,0.752941,0.752941}%
\pgfsetstrokecolor{currentstroke}%
\pgfsetdash{{2.960000pt}{1.280000pt}}{0.000000pt}%
\pgfpathmoveto{\pgfqpoint{0.726250in}{1.460714in}}%
\pgfpathlineto{\pgfqpoint{1.846250in}{1.460714in}}%
\pgfusepath{stroke}%
\end{pgfscope}%
\begin{pgfscope}%
\pgfsetbuttcap%
\pgfsetroundjoin%
\definecolor{currentfill}{rgb}{0.000000,0.000000,0.000000}%
\pgfsetfillcolor{currentfill}%
\pgfsetlinewidth{0.803000pt}%
\definecolor{currentstroke}{rgb}{0.000000,0.000000,0.000000}%
\pgfsetstrokecolor{currentstroke}%
\pgfsetdash{}{0pt}%
\pgfsys@defobject{currentmarker}{\pgfqpoint{-0.048611in}{0.000000in}}{\pgfqpoint{-0.000000in}{0.000000in}}{%
\pgfpathmoveto{\pgfqpoint{-0.000000in}{0.000000in}}%
\pgfpathlineto{\pgfqpoint{-0.048611in}{0.000000in}}%
\pgfusepath{stroke,fill}%
}%
\begin{pgfscope}%
\pgfsys@transformshift{0.726250in}{1.460714in}%
\pgfsys@useobject{currentmarker}{}%
\end{pgfscope}%
\end{pgfscope}%
\begin{pgfscope}%
\definecolor{textcolor}{rgb}{0.000000,0.000000,0.000000}%
\pgfsetstrokecolor{textcolor}%
\pgfsetfillcolor{textcolor}%
\pgftext[x=0.350708in, y=1.413229in, left, base]{\color{textcolor}\rmfamily\fontsize{9.000000}{10.800000}\selectfont 0.08}%
\end{pgfscope}%
\begin{pgfscope}%
\pgfpathrectangle{\pgfqpoint{0.726250in}{0.525000in}}{\pgfqpoint{1.120000in}{1.637500in}}%
\pgfusepath{clip}%
\pgfsetbuttcap%
\pgfsetroundjoin%
\pgfsetlinewidth{0.803000pt}%
\definecolor{currentstroke}{rgb}{0.752941,0.752941,0.752941}%
\pgfsetstrokecolor{currentstroke}%
\pgfsetdash{{2.960000pt}{1.280000pt}}{0.000000pt}%
\pgfpathmoveto{\pgfqpoint{0.726250in}{1.694643in}}%
\pgfpathlineto{\pgfqpoint{1.846250in}{1.694643in}}%
\pgfusepath{stroke}%
\end{pgfscope}%
\begin{pgfscope}%
\pgfsetbuttcap%
\pgfsetroundjoin%
\definecolor{currentfill}{rgb}{0.000000,0.000000,0.000000}%
\pgfsetfillcolor{currentfill}%
\pgfsetlinewidth{0.803000pt}%
\definecolor{currentstroke}{rgb}{0.000000,0.000000,0.000000}%
\pgfsetstrokecolor{currentstroke}%
\pgfsetdash{}{0pt}%
\pgfsys@defobject{currentmarker}{\pgfqpoint{-0.048611in}{0.000000in}}{\pgfqpoint{-0.000000in}{0.000000in}}{%
\pgfpathmoveto{\pgfqpoint{-0.000000in}{0.000000in}}%
\pgfpathlineto{\pgfqpoint{-0.048611in}{0.000000in}}%
\pgfusepath{stroke,fill}%
}%
\begin{pgfscope}%
\pgfsys@transformshift{0.726250in}{1.694643in}%
\pgfsys@useobject{currentmarker}{}%
\end{pgfscope}%
\end{pgfscope}%
\begin{pgfscope}%
\definecolor{textcolor}{rgb}{0.000000,0.000000,0.000000}%
\pgfsetstrokecolor{textcolor}%
\pgfsetfillcolor{textcolor}%
\pgftext[x=0.350708in, y=1.647158in, left, base]{\color{textcolor}\rmfamily\fontsize{9.000000}{10.800000}\selectfont 0.10}%
\end{pgfscope}%
\begin{pgfscope}%
\pgfpathrectangle{\pgfqpoint{0.726250in}{0.525000in}}{\pgfqpoint{1.120000in}{1.637500in}}%
\pgfusepath{clip}%
\pgfsetbuttcap%
\pgfsetroundjoin%
\pgfsetlinewidth{0.803000pt}%
\definecolor{currentstroke}{rgb}{0.752941,0.752941,0.752941}%
\pgfsetstrokecolor{currentstroke}%
\pgfsetdash{{2.960000pt}{1.280000pt}}{0.000000pt}%
\pgfpathmoveto{\pgfqpoint{0.726250in}{1.928571in}}%
\pgfpathlineto{\pgfqpoint{1.846250in}{1.928571in}}%
\pgfusepath{stroke}%
\end{pgfscope}%
\begin{pgfscope}%
\pgfsetbuttcap%
\pgfsetroundjoin%
\definecolor{currentfill}{rgb}{0.000000,0.000000,0.000000}%
\pgfsetfillcolor{currentfill}%
\pgfsetlinewidth{0.803000pt}%
\definecolor{currentstroke}{rgb}{0.000000,0.000000,0.000000}%
\pgfsetstrokecolor{currentstroke}%
\pgfsetdash{}{0pt}%
\pgfsys@defobject{currentmarker}{\pgfqpoint{-0.048611in}{0.000000in}}{\pgfqpoint{-0.000000in}{0.000000in}}{%
\pgfpathmoveto{\pgfqpoint{-0.000000in}{0.000000in}}%
\pgfpathlineto{\pgfqpoint{-0.048611in}{0.000000in}}%
\pgfusepath{stroke,fill}%
}%
\begin{pgfscope}%
\pgfsys@transformshift{0.726250in}{1.928571in}%
\pgfsys@useobject{currentmarker}{}%
\end{pgfscope}%
\end{pgfscope}%
\begin{pgfscope}%
\definecolor{textcolor}{rgb}{0.000000,0.000000,0.000000}%
\pgfsetstrokecolor{textcolor}%
\pgfsetfillcolor{textcolor}%
\pgftext[x=0.350708in, y=1.881086in, left, base]{\color{textcolor}\rmfamily\fontsize{9.000000}{10.800000}\selectfont 0.12}%
\end{pgfscope}%
\begin{pgfscope}%
\pgfpathrectangle{\pgfqpoint{0.726250in}{0.525000in}}{\pgfqpoint{1.120000in}{1.637500in}}%
\pgfusepath{clip}%
\pgfsetbuttcap%
\pgfsetroundjoin%
\pgfsetlinewidth{0.803000pt}%
\definecolor{currentstroke}{rgb}{0.752941,0.752941,0.752941}%
\pgfsetstrokecolor{currentstroke}%
\pgfsetdash{{2.960000pt}{1.280000pt}}{0.000000pt}%
\pgfpathmoveto{\pgfqpoint{0.726250in}{2.162500in}}%
\pgfpathlineto{\pgfqpoint{1.846250in}{2.162500in}}%
\pgfusepath{stroke}%
\end{pgfscope}%
\begin{pgfscope}%
\pgfsetbuttcap%
\pgfsetroundjoin%
\definecolor{currentfill}{rgb}{0.000000,0.000000,0.000000}%
\pgfsetfillcolor{currentfill}%
\pgfsetlinewidth{0.803000pt}%
\definecolor{currentstroke}{rgb}{0.000000,0.000000,0.000000}%
\pgfsetstrokecolor{currentstroke}%
\pgfsetdash{}{0pt}%
\pgfsys@defobject{currentmarker}{\pgfqpoint{-0.048611in}{0.000000in}}{\pgfqpoint{-0.000000in}{0.000000in}}{%
\pgfpathmoveto{\pgfqpoint{-0.000000in}{0.000000in}}%
\pgfpathlineto{\pgfqpoint{-0.048611in}{0.000000in}}%
\pgfusepath{stroke,fill}%
}%
\begin{pgfscope}%
\pgfsys@transformshift{0.726250in}{2.162500in}%
\pgfsys@useobject{currentmarker}{}%
\end{pgfscope}%
\end{pgfscope}%
\begin{pgfscope}%
\definecolor{textcolor}{rgb}{0.000000,0.000000,0.000000}%
\pgfsetstrokecolor{textcolor}%
\pgfsetfillcolor{textcolor}%
\pgftext[x=0.350708in, y=2.115015in, left, base]{\color{textcolor}\rmfamily\fontsize{9.000000}{10.800000}\selectfont 0.14}%
\end{pgfscope}%
\begin{pgfscope}%
\definecolor{textcolor}{rgb}{0.000000,0.000000,0.000000}%
\pgfsetstrokecolor{textcolor}%
\pgfsetfillcolor{textcolor}%
\pgftext[x=0.295152in,y=1.343750in,,bottom,rotate=90.000000]{\color{textcolor}\rmfamily\fontsize{9.000000}{10.800000}\selectfont \(\displaystyle \frac{1}{E_0}\frac{dE}{dt}\) [GeV \(\displaystyle X_0^{-1}\)]}%
\end{pgfscope}%
\begin{pgfscope}%
\pgfpathrectangle{\pgfqpoint{0.726250in}{0.525000in}}{\pgfqpoint{1.120000in}{1.637500in}}%
\pgfusepath{clip}%
\pgfsetrectcap%
\pgfsetroundjoin%
\pgfsetlinewidth{1.003750pt}%
\definecolor{currentstroke}{rgb}{0.000000,0.000000,0.000000}%
\pgfsetstrokecolor{currentstroke}%
\pgfsetdash{}{0pt}%
\pgfpathmoveto{\pgfqpoint{0.726250in}{0.525000in}}%
\pgfpathlineto{\pgfqpoint{0.737563in}{0.525033in}}%
\pgfpathlineto{\pgfqpoint{0.748876in}{0.525582in}}%
\pgfpathlineto{\pgfqpoint{0.760189in}{0.527917in}}%
\pgfpathlineto{\pgfqpoint{0.771503in}{0.533764in}}%
\pgfpathlineto{\pgfqpoint{0.782816in}{0.544883in}}%
\pgfpathlineto{\pgfqpoint{0.794129in}{0.562768in}}%
\pgfpathlineto{\pgfqpoint{0.805442in}{0.588468in}}%
\pgfpathlineto{\pgfqpoint{0.816755in}{0.622505in}}%
\pgfpathlineto{\pgfqpoint{0.828068in}{0.664877in}}%
\pgfpathlineto{\pgfqpoint{0.839381in}{0.715108in}}%
\pgfpathlineto{\pgfqpoint{0.850694in}{0.772322in}}%
\pgfpathlineto{\pgfqpoint{0.862008in}{0.835348in}}%
\pgfpathlineto{\pgfqpoint{0.873321in}{0.902806in}}%
\pgfpathlineto{\pgfqpoint{0.884634in}{0.973204in}}%
\pgfpathlineto{\pgfqpoint{0.895947in}{1.045020in}}%
\pgfpathlineto{\pgfqpoint{0.907260in}{1.116763in}}%
\pgfpathlineto{\pgfqpoint{0.918573in}{1.187036in}}%
\pgfpathlineto{\pgfqpoint{0.929886in}{1.254570in}}%
\pgfpathlineto{\pgfqpoint{0.941199in}{1.318254in}}%
\pgfpathlineto{\pgfqpoint{0.952513in}{1.377151in}}%
\pgfpathlineto{\pgfqpoint{0.963826in}{1.430507in}}%
\pgfpathlineto{\pgfqpoint{0.975139in}{1.477749in}}%
\pgfpathlineto{\pgfqpoint{0.986452in}{1.518476in}}%
\pgfpathlineto{\pgfqpoint{0.997765in}{1.552452in}}%
\pgfpathlineto{\pgfqpoint{1.009078in}{1.579589in}}%
\pgfpathlineto{\pgfqpoint{1.020391in}{1.599928in}}%
\pgfpathlineto{\pgfqpoint{1.031705in}{1.613626in}}%
\pgfpathlineto{\pgfqpoint{1.043018in}{1.620934in}}%
\pgfpathlineto{\pgfqpoint{1.054331in}{1.622183in}}%
\pgfpathlineto{\pgfqpoint{1.065644in}{1.617763in}}%
\pgfpathlineto{\pgfqpoint{1.076957in}{1.608110in}}%
\pgfpathlineto{\pgfqpoint{1.088270in}{1.593691in}}%
\pgfpathlineto{\pgfqpoint{1.099583in}{1.574991in}}%
\pgfpathlineto{\pgfqpoint{1.110896in}{1.552504in}}%
\pgfpathlineto{\pgfqpoint{1.122210in}{1.526720in}}%
\pgfpathlineto{\pgfqpoint{1.133523in}{1.498121in}}%
\pgfpathlineto{\pgfqpoint{1.144836in}{1.467168in}}%
\pgfpathlineto{\pgfqpoint{1.156149in}{1.434305in}}%
\pgfpathlineto{\pgfqpoint{1.167462in}{1.399947in}}%
\pgfpathlineto{\pgfqpoint{1.178775in}{1.364483in}}%
\pgfpathlineto{\pgfqpoint{1.190088in}{1.328268in}}%
\pgfpathlineto{\pgfqpoint{1.201402in}{1.291629in}}%
\pgfpathlineto{\pgfqpoint{1.212715in}{1.254858in}}%
\pgfpathlineto{\pgfqpoint{1.224028in}{1.218217in}}%
\pgfpathlineto{\pgfqpoint{1.235341in}{1.181938in}}%
\pgfpathlineto{\pgfqpoint{1.246654in}{1.146221in}}%
\pgfpathlineto{\pgfqpoint{1.257967in}{1.111239in}}%
\pgfpathlineto{\pgfqpoint{1.269280in}{1.077139in}}%
\pgfpathlineto{\pgfqpoint{1.280593in}{1.044041in}}%
\pgfpathlineto{\pgfqpoint{1.291907in}{1.012043in}}%
\pgfpathlineto{\pgfqpoint{1.303220in}{0.981223in}}%
\pgfpathlineto{\pgfqpoint{1.314533in}{0.951640in}}%
\pgfpathlineto{\pgfqpoint{1.325846in}{0.923333in}}%
\pgfpathlineto{\pgfqpoint{1.337159in}{0.896329in}}%
\pgfpathlineto{\pgfqpoint{1.348472in}{0.870639in}}%
\pgfpathlineto{\pgfqpoint{1.359785in}{0.846264in}}%
\pgfpathlineto{\pgfqpoint{1.371098in}{0.823194in}}%
\pgfpathlineto{\pgfqpoint{1.382412in}{0.801411in}}%
\pgfpathlineto{\pgfqpoint{1.393725in}{0.780887in}}%
\pgfpathlineto{\pgfqpoint{1.405038in}{0.761592in}}%
\pgfpathlineto{\pgfqpoint{1.416351in}{0.743487in}}%
\pgfpathlineto{\pgfqpoint{1.427664in}{0.726532in}}%
\pgfpathlineto{\pgfqpoint{1.438977in}{0.710682in}}%
\pgfpathlineto{\pgfqpoint{1.450290in}{0.695892in}}%
\pgfpathlineto{\pgfqpoint{1.461604in}{0.682112in}}%
\pgfpathlineto{\pgfqpoint{1.472917in}{0.669295in}}%
\pgfpathlineto{\pgfqpoint{1.484230in}{0.657392in}}%
\pgfpathlineto{\pgfqpoint{1.495543in}{0.646352in}}%
\pgfpathlineto{\pgfqpoint{1.506856in}{0.636129in}}%
\pgfpathlineto{\pgfqpoint{1.518169in}{0.626674in}}%
\pgfpathlineto{\pgfqpoint{1.529482in}{0.617941in}}%
\pgfpathlineto{\pgfqpoint{1.540795in}{0.609884in}}%
\pgfpathlineto{\pgfqpoint{1.552109in}{0.602461in}}%
\pgfpathlineto{\pgfqpoint{1.563422in}{0.595630in}}%
\pgfpathlineto{\pgfqpoint{1.574735in}{0.589349in}}%
\pgfpathlineto{\pgfqpoint{1.586048in}{0.583582in}}%
\pgfpathlineto{\pgfqpoint{1.597361in}{0.578292in}}%
\pgfpathlineto{\pgfqpoint{1.608674in}{0.573444in}}%
\pgfpathlineto{\pgfqpoint{1.619987in}{0.569005in}}%
\pgfpathlineto{\pgfqpoint{1.631301in}{0.564945in}}%
\pgfpathlineto{\pgfqpoint{1.642614in}{0.561235in}}%
\pgfpathlineto{\pgfqpoint{1.653927in}{0.557848in}}%
\pgfpathlineto{\pgfqpoint{1.665240in}{0.554758in}}%
\pgfpathlineto{\pgfqpoint{1.676553in}{0.551942in}}%
\pgfpathlineto{\pgfqpoint{1.687866in}{0.549377in}}%
\pgfpathlineto{\pgfqpoint{1.699179in}{0.547044in}}%
\pgfpathlineto{\pgfqpoint{1.710492in}{0.544922in}}%
\pgfpathlineto{\pgfqpoint{1.721806in}{0.542993in}}%
\pgfpathlineto{\pgfqpoint{1.733119in}{0.541243in}}%
\pgfpathlineto{\pgfqpoint{1.744432in}{0.539654in}}%
\pgfpathlineto{\pgfqpoint{1.755745in}{0.538214in}}%
\pgfpathlineto{\pgfqpoint{1.767058in}{0.536909in}}%
\pgfpathlineto{\pgfqpoint{1.778371in}{0.535728in}}%
\pgfpathlineto{\pgfqpoint{1.789684in}{0.534659in}}%
\pgfpathlineto{\pgfqpoint{1.800997in}{0.533692in}}%
\pgfpathlineto{\pgfqpoint{1.812311in}{0.532818in}}%
\pgfpathlineto{\pgfqpoint{1.823624in}{0.532028in}}%
\pgfpathlineto{\pgfqpoint{1.834937in}{0.531316in}}%
\pgfpathlineto{\pgfqpoint{1.846250in}{0.530673in}}%
\pgfusepath{stroke}%
\end{pgfscope}%
\begin{pgfscope}%
\pgfpathrectangle{\pgfqpoint{0.726250in}{0.525000in}}{\pgfqpoint{1.120000in}{1.637500in}}%
\pgfusepath{clip}%
\pgfsetbuttcap%
\pgfsetroundjoin%
\pgfsetlinewidth{1.003750pt}%
\definecolor{currentstroke}{rgb}{0.000000,0.000000,0.000000}%
\pgfsetstrokecolor{currentstroke}%
\pgfsetdash{{3.700000pt}{1.600000pt}}{0.000000pt}%
\pgfpathmoveto{\pgfqpoint{0.726250in}{0.525000in}}%
\pgfpathlineto{\pgfqpoint{0.737563in}{0.525006in}}%
\pgfpathlineto{\pgfqpoint{0.748876in}{0.525142in}}%
\pgfpathlineto{\pgfqpoint{0.760189in}{0.525870in}}%
\pgfpathlineto{\pgfqpoint{0.771503in}{0.528019in}}%
\pgfpathlineto{\pgfqpoint{0.782816in}{0.532658in}}%
\pgfpathlineto{\pgfqpoint{0.794129in}{0.540936in}}%
\pgfpathlineto{\pgfqpoint{0.805442in}{0.553925in}}%
\pgfpathlineto{\pgfqpoint{0.816755in}{0.572506in}}%
\pgfpathlineto{\pgfqpoint{0.828068in}{0.597284in}}%
\pgfpathlineto{\pgfqpoint{0.839381in}{0.628556in}}%
\pgfpathlineto{\pgfqpoint{0.850694in}{0.666298in}}%
\pgfpathlineto{\pgfqpoint{0.862008in}{0.710189in}}%
\pgfpathlineto{\pgfqpoint{0.873321in}{0.759647in}}%
\pgfpathlineto{\pgfqpoint{0.884634in}{0.813879in}}%
\pgfpathlineto{\pgfqpoint{0.895947in}{0.871929in}}%
\pgfpathlineto{\pgfqpoint{0.907260in}{0.932740in}}%
\pgfpathlineto{\pgfqpoint{0.918573in}{0.995199in}}%
\pgfpathlineto{\pgfqpoint{0.929886in}{1.058186in}}%
\pgfpathlineto{\pgfqpoint{0.941199in}{1.120613in}}%
\pgfpathlineto{\pgfqpoint{0.952513in}{1.181458in}}%
\pgfpathlineto{\pgfqpoint{0.963826in}{1.239787in}}%
\pgfpathlineto{\pgfqpoint{0.975139in}{1.294777in}}%
\pgfpathlineto{\pgfqpoint{0.986452in}{1.345723in}}%
\pgfpathlineto{\pgfqpoint{0.997765in}{1.392047in}}%
\pgfpathlineto{\pgfqpoint{1.009078in}{1.433298in}}%
\pgfpathlineto{\pgfqpoint{1.020391in}{1.469150in}}%
\pgfpathlineto{\pgfqpoint{1.031705in}{1.499396in}}%
\pgfpathlineto{\pgfqpoint{1.043018in}{1.523939in}}%
\pgfpathlineto{\pgfqpoint{1.054331in}{1.542779in}}%
\pgfpathlineto{\pgfqpoint{1.065644in}{1.556008in}}%
\pgfpathlineto{\pgfqpoint{1.076957in}{1.563792in}}%
\pgfpathlineto{\pgfqpoint{1.088270in}{1.566363in}}%
\pgfpathlineto{\pgfqpoint{1.099583in}{1.564006in}}%
\pgfpathlineto{\pgfqpoint{1.110896in}{1.557044in}}%
\pgfpathlineto{\pgfqpoint{1.122210in}{1.545836in}}%
\pgfpathlineto{\pgfqpoint{1.133523in}{1.530757in}}%
\pgfpathlineto{\pgfqpoint{1.144836in}{1.512199in}}%
\pgfpathlineto{\pgfqpoint{1.156149in}{1.490554in}}%
\pgfpathlineto{\pgfqpoint{1.167462in}{1.466217in}}%
\pgfpathlineto{\pgfqpoint{1.178775in}{1.439570in}}%
\pgfpathlineto{\pgfqpoint{1.190088in}{1.410988in}}%
\pgfpathlineto{\pgfqpoint{1.201402in}{1.380825in}}%
\pgfpathlineto{\pgfqpoint{1.212715in}{1.349419in}}%
\pgfpathlineto{\pgfqpoint{1.224028in}{1.317084in}}%
\pgfpathlineto{\pgfqpoint{1.235341in}{1.284112in}}%
\pgfpathlineto{\pgfqpoint{1.246654in}{1.250772in}}%
\pgfpathlineto{\pgfqpoint{1.257967in}{1.217308in}}%
\pgfpathlineto{\pgfqpoint{1.269280in}{1.183937in}}%
\pgfpathlineto{\pgfqpoint{1.280593in}{1.150856in}}%
\pgfpathlineto{\pgfqpoint{1.291907in}{1.118236in}}%
\pgfpathlineto{\pgfqpoint{1.303220in}{1.086226in}}%
\pgfpathlineto{\pgfqpoint{1.314533in}{1.054954in}}%
\pgfpathlineto{\pgfqpoint{1.325846in}{1.024528in}}%
\pgfpathlineto{\pgfqpoint{1.337159in}{0.995035in}}%
\pgfpathlineto{\pgfqpoint{1.348472in}{0.966549in}}%
\pgfpathlineto{\pgfqpoint{1.359785in}{0.939125in}}%
\pgfpathlineto{\pgfqpoint{1.371098in}{0.912804in}}%
\pgfpathlineto{\pgfqpoint{1.382412in}{0.887613in}}%
\pgfpathlineto{\pgfqpoint{1.393725in}{0.863571in}}%
\pgfpathlineto{\pgfqpoint{1.405038in}{0.840682in}}%
\pgfpathlineto{\pgfqpoint{1.416351in}{0.818945in}}%
\pgfpathlineto{\pgfqpoint{1.427664in}{0.798347in}}%
\pgfpathlineto{\pgfqpoint{1.438977in}{0.778873in}}%
\pgfpathlineto{\pgfqpoint{1.450290in}{0.760497in}}%
\pgfpathlineto{\pgfqpoint{1.461604in}{0.743194in}}%
\pgfpathlineto{\pgfqpoint{1.472917in}{0.726929in}}%
\pgfpathlineto{\pgfqpoint{1.484230in}{0.711670in}}%
\pgfpathlineto{\pgfqpoint{1.495543in}{0.697376in}}%
\pgfpathlineto{\pgfqpoint{1.506856in}{0.684011in}}%
\pgfpathlineto{\pgfqpoint{1.518169in}{0.671532in}}%
\pgfpathlineto{\pgfqpoint{1.529482in}{0.659899in}}%
\pgfpathlineto{\pgfqpoint{1.540795in}{0.649070in}}%
\pgfpathlineto{\pgfqpoint{1.552109in}{0.639004in}}%
\pgfpathlineto{\pgfqpoint{1.563422in}{0.629659in}}%
\pgfpathlineto{\pgfqpoint{1.574735in}{0.620995in}}%
\pgfpathlineto{\pgfqpoint{1.586048in}{0.612973in}}%
\pgfpathlineto{\pgfqpoint{1.597361in}{0.605553in}}%
\pgfpathlineto{\pgfqpoint{1.608674in}{0.598698in}}%
\pgfpathlineto{\pgfqpoint{1.619987in}{0.592374in}}%
\pgfpathlineto{\pgfqpoint{1.631301in}{0.586544in}}%
\pgfpathlineto{\pgfqpoint{1.642614in}{0.581176in}}%
\pgfpathlineto{\pgfqpoint{1.653927in}{0.576238in}}%
\pgfpathlineto{\pgfqpoint{1.665240in}{0.571701in}}%
\pgfpathlineto{\pgfqpoint{1.676553in}{0.567535in}}%
\pgfpathlineto{\pgfqpoint{1.687866in}{0.563714in}}%
\pgfpathlineto{\pgfqpoint{1.699179in}{0.560213in}}%
\pgfpathlineto{\pgfqpoint{1.710492in}{0.557008in}}%
\pgfpathlineto{\pgfqpoint{1.721806in}{0.554076in}}%
\pgfpathlineto{\pgfqpoint{1.733119in}{0.551395in}}%
\pgfpathlineto{\pgfqpoint{1.744432in}{0.548948in}}%
\pgfpathlineto{\pgfqpoint{1.755745in}{0.546714in}}%
\pgfpathlineto{\pgfqpoint{1.767058in}{0.544677in}}%
\pgfpathlineto{\pgfqpoint{1.778371in}{0.542821in}}%
\pgfpathlineto{\pgfqpoint{1.789684in}{0.541131in}}%
\pgfpathlineto{\pgfqpoint{1.800997in}{0.539593in}}%
\pgfpathlineto{\pgfqpoint{1.812311in}{0.538195in}}%
\pgfpathlineto{\pgfqpoint{1.823624in}{0.536924in}}%
\pgfpathlineto{\pgfqpoint{1.834937in}{0.535770in}}%
\pgfpathlineto{\pgfqpoint{1.846250in}{0.534723in}}%
\pgfusepath{stroke}%
\end{pgfscope}%
\begin{pgfscope}%
\pgfsetrectcap%
\pgfsetmiterjoin%
\pgfsetlinewidth{1.003750pt}%
\definecolor{currentstroke}{rgb}{0.000000,0.000000,0.000000}%
\pgfsetstrokecolor{currentstroke}%
\pgfsetdash{}{0pt}%
\pgfpathmoveto{\pgfqpoint{0.726250in}{0.525000in}}%
\pgfpathlineto{\pgfqpoint{0.726250in}{2.162500in}}%
\pgfusepath{stroke}%
\end{pgfscope}%
\begin{pgfscope}%
\pgfsetrectcap%
\pgfsetmiterjoin%
\pgfsetlinewidth{1.003750pt}%
\definecolor{currentstroke}{rgb}{0.000000,0.000000,0.000000}%
\pgfsetstrokecolor{currentstroke}%
\pgfsetdash{}{0pt}%
\pgfpathmoveto{\pgfqpoint{1.846250in}{0.525000in}}%
\pgfpathlineto{\pgfqpoint{1.846250in}{2.162500in}}%
\pgfusepath{stroke}%
\end{pgfscope}%
\begin{pgfscope}%
\pgfsetrectcap%
\pgfsetmiterjoin%
\pgfsetlinewidth{1.003750pt}%
\definecolor{currentstroke}{rgb}{0.000000,0.000000,0.000000}%
\pgfsetstrokecolor{currentstroke}%
\pgfsetdash{}{0pt}%
\pgfpathmoveto{\pgfqpoint{0.726250in}{0.525000in}}%
\pgfpathlineto{\pgfqpoint{1.846250in}{0.525000in}}%
\pgfusepath{stroke}%
\end{pgfscope}%
\begin{pgfscope}%
\pgfsetrectcap%
\pgfsetmiterjoin%
\pgfsetlinewidth{1.003750pt}%
\definecolor{currentstroke}{rgb}{0.000000,0.000000,0.000000}%
\pgfsetstrokecolor{currentstroke}%
\pgfsetdash{}{0pt}%
\pgfpathmoveto{\pgfqpoint{0.726250in}{2.162500in}}%
\pgfpathlineto{\pgfqpoint{1.846250in}{2.162500in}}%
\pgfusepath{stroke}%
\end{pgfscope}%
\begin{pgfscope}%
\pgfsetbuttcap%
\pgfsetmiterjoin%
\definecolor{currentfill}{rgb}{1.000000,1.000000,1.000000}%
\pgfsetfillcolor{currentfill}%
\pgfsetfillopacity{0.800000}%
\pgfsetlinewidth{1.003750pt}%
\definecolor{currentstroke}{rgb}{0.800000,0.800000,0.800000}%
\pgfsetstrokecolor{currentstroke}%
\pgfsetstrokeopacity{0.800000}%
\pgfsetdash{}{0pt}%
\pgfpathmoveto{\pgfqpoint{0.753648in}{1.695557in}}%
\pgfpathlineto{\pgfqpoint{1.758750in}{1.695557in}}%
\pgfpathquadraticcurveto{\pgfqpoint{1.783750in}{1.695557in}}{\pgfqpoint{1.783750in}{1.720557in}}%
\pgfpathlineto{\pgfqpoint{1.783750in}{2.075000in}}%
\pgfpathquadraticcurveto{\pgfqpoint{1.783750in}{2.100000in}}{\pgfqpoint{1.758750in}{2.100000in}}%
\pgfpathlineto{\pgfqpoint{0.753648in}{2.100000in}}%
\pgfpathquadraticcurveto{\pgfqpoint{0.728648in}{2.100000in}}{\pgfqpoint{0.728648in}{2.075000in}}%
\pgfpathlineto{\pgfqpoint{0.728648in}{1.720557in}}%
\pgfpathquadraticcurveto{\pgfqpoint{0.728648in}{1.695557in}}{\pgfqpoint{0.753648in}{1.695557in}}%
\pgfpathlineto{\pgfqpoint{0.753648in}{1.695557in}}%
\pgfpathclose%
\pgfusepath{stroke,fill}%
\end{pgfscope}%
\begin{pgfscope}%
\pgfsetrectcap%
\pgfsetroundjoin%
\pgfsetlinewidth{1.003750pt}%
\definecolor{currentstroke}{rgb}{0.000000,0.000000,0.000000}%
\pgfsetstrokecolor{currentstroke}%
\pgfsetdash{}{0pt}%
\pgfpathmoveto{\pgfqpoint{0.778648in}{1.998779in}}%
\pgfpathlineto{\pgfqpoint{0.903648in}{1.998779in}}%
\pgfpathlineto{\pgfqpoint{1.028647in}{1.998779in}}%
\pgfusepath{stroke}%
\end{pgfscope}%
\begin{pgfscope}%
\definecolor{textcolor}{rgb}{0.000000,0.000000,0.000000}%
\pgfsetstrokecolor{textcolor}%
\pgfsetfillcolor{textcolor}%
\pgftext[x=1.128648in,y=1.955029in,left,base]{\color{textcolor}\rmfamily\fontsize{9.000000}{10.800000}\selectfont Electrons}%
\end{pgfscope}%
\begin{pgfscope}%
\pgfsetbuttcap%
\pgfsetroundjoin%
\pgfsetlinewidth{1.003750pt}%
\definecolor{currentstroke}{rgb}{0.000000,0.000000,0.000000}%
\pgfsetstrokecolor{currentstroke}%
\pgfsetdash{{3.700000pt}{1.600000pt}}{0.000000pt}%
\pgfpathmoveto{\pgfqpoint{0.778648in}{1.815308in}}%
\pgfpathlineto{\pgfqpoint{0.903648in}{1.815308in}}%
\pgfpathlineto{\pgfqpoint{1.028647in}{1.815308in}}%
\pgfusepath{stroke}%
\end{pgfscope}%
\begin{pgfscope}%
\definecolor{textcolor}{rgb}{0.000000,0.000000,0.000000}%
\pgfsetstrokecolor{textcolor}%
\pgfsetfillcolor{textcolor}%
\pgftext[x=1.128648in,y=1.771558in,left,base]{\color{textcolor}\rmfamily\fontsize{9.000000}{10.800000}\selectfont Photons}%
\end{pgfscope}%
\end{pgfpicture}%
\makeatother%
\endgroup%

  \caption{Average longitudinal profile of the shower generated by a
     $100$~GeV electron in a homogeneous slab of BGO.}
  \label{fig:shower_longitudinal_profile}
\end{marginfigure}

At a slightly higher level of sophistication, the longitudinal profile of an
electromagnetic shower can be effectively described as
\begin{align}\label{eq:long_profile}
  \frac{dE}{dt} = E_0 b \frac{(bt)^{a-1} e^{-bt}}{\Gamma(a)},
\end{align}
where $a$ and $b$ are parameters related to the nature of the incident particle
(electron or photon) and to the characteristics of the medium ($b = 0.5$
is a reasonable approximation in many cases of practical interest).
The position of the shower maximum occurs at
\begin{align}\label{eq:shower_max}
  t_\text{max} = \frac{(a - 1)}{b} \approx
  \ln \left(\frac{E_0}{E_c}\right) + t_0
\end{align}
where $t_0 = -0.5$ for electrons and $t_0 = 0.5$ for photons. For the sake
of clarity, the way one typically uses these relation is to plug $E_0$, $E_c$
and $t_0$ in \eqref{eq:shower_max} to find $a$ (assuming $b = 0.5$) and then
use~\eqref{eq:long_profile} to describe the longitudinal profile of the shower.

Figure~\ref{fig:shower_longitudinal_profile} shows the average shower profile for
$100$~GeV in BGO. With a critical energy of $10.1$~MeV, the position of the shower
maximum is located, according to equation~\eqref{eq:shower_max}, at $8.7~X_0$.

We mention, in passing, that the transverse development of an electromagnetic
shower, mainly due to the multiple scattering of electrons and positrons away
from the shower axis, scales with the so-called the Moli\`ere radius, that can be
empirically parameterized as
\begin{align}
  R_M \approx X_0 \frac{E_s}{E_c}
  \quad\text{where}\quad
  E_s = 21~\text{MeV}.
\end{align}


\section{Hadronic showers}%
\label{sec:had_showers}

While the development of electromagnetic showers is determined, as we have seen,
by two well-understood QED processes, the energy degradation of hadrons proceeds
through both strong and electromagnetic interactions in the medium. The complexity
of the hadronic and nuclear processes produce a multitude of phenomena that make
hadronic showers intrinsically more complicated than the electromagnetic ones.
In other words for hadronic showers there is no such a thing as the toy model
illustrated in figure~\ref{fig:em_shower_naive}.

As a matter of fact hadronic showers consist in general of two distinctly different
components: (i) an electromagnetic component due to $\pi^0$ and $\eta$ generated
in the absorption process and decaying into photons (which in turn develop
electromagnetic showers) before they have a chance to undergo a new strong interaction;
and (ii) a non-electromagnetic component, which combines essentially everything
else that takes place in the absorption process. Most importantly, the two components
evolve with different length scales: the radiation length $X_0$ for the first and
the nuclear interaction length $\lambda_I$ for the second.

Interestingly enough, one can gain some useful insight into the latter by simple
semi-quantitative arguments based on dimensional analysis. Since the volume of the
nucleus is proportional to the mass number $A$, its radius will be
\begin{align*}
  r \approx r_0 A^\frac{1}{3}
  \quad\text{where}\quad
  r_0 \approx 10^{-15}~\text{m} = 1~\text{fm}.
\end{align*}
Now, since nuclear interactions are governed by the strong force, which is
characterized by a short range, the nuclear cross section approaches the geometrical
limit
\begin{align*}
  \sigma_N \propto r^2 \propto A^\frac{2}{3},
\end{align*}
and the corresponding interaction length (measured in~cm) can be written in terms
of $\sigma_N$ and the number density $n$ of the nuclei\sidenote{Mind that, since
$\sigma_N$ is the cross section of the nucleus, $n$ is  the number density of the
\emph{nuclei} and not that of the \emph{nucleons}. As such, it scales with the
theromidinamic properties of the medium, but does not depend on the mass number
$A$.} as
\begin{align*}
  \lambda_I = \frac{1}{n \sigma_N} = \frac{A}{N_A \density \sigma_N}\propto
  \begin{cases}
    \displaystyle A^{-\frac{2}{3}} \quad\text{in~cm}\\
    \displaystyle A^\frac{1}{3} \quad\text{in~g~cm}^{-2}.
  \end{cases}
\end{align*}

A useful parameterization for $\lambda_I$, that can be used for actual calculations,
is in fact
\begin{align}\label{eq:param_intlen}
  \lambda_I = 37.8~A^{0.312}~\text{g~cm}^{-2}
\end{align}
and the tabulated values for some relevant materials are shown in
table~\ref{tab:exp_intlen}.

%(which is at least on order of magnitude larger for $Z > 30$, as
%shown in figure~\ref{fig:rad_int_len}) .

% \begin{figure}
%   \includegraphics[width=\linewidth]{rad_int_len}
%   \caption{Approximate dependence of the radiation length $X_0$ and the
%     nuclear interaction length $\lambda_I$ on the atomic number $Z$ of the
%     material. The parameterizations used are those in equations
%     \eqref{eq:param_radlen} and \eqref{eq:param_intlen}, with the additional
%     assumption $A = 2Z$.}
%   \label{fig:rad_int_len}
% \end{figure}

\begin{table}[htb!]
  \begin{tabular}{p{0.23\linewidth}p{0.21\linewidth}p{0.21\linewidth}%
      p{0.21\linewidth}}
    \hline
    Material & $\lambda_I$~[g~cm$^{-2}$] & $\density$ [g~cm$^{-3}$] &
    $\lambda_I$ [cm]\\
    \hline
    \hline
    Pb & 199.6 & 11.350 & 17.6 \\
    BGO & 159.1 & 7.130 & 22.3 \\
    CsI & 171.5 & 4.510 & 38.0\\
    W & 191.9 & 19.3 & 9.94\\
    C (graphite) & 85.8 & 2.210 & 38.8\\
    Si & 108.4 & 2.329 & 46.5\\
    Air & 90.1 & $1.2 \times 10^{-3}$ & 75,000\\
    \hline
  \end{tabular}
  \caption{Tabulated values of the nuclear interaction length for some materials
    of interest.}
  \label{tab:exp_intlen}
\end{table}

The fractional non-electromagnetic component of a hadronic shower $F_h$ (as
opposed to electromagnetic component $F_e = 1 - F_h$) decreases with energy and
is generally~\cite{PDG} parameterized as
\begin{align}
  F_h(E) = \left( \frac{E}{E_0} \right)^{k-1},
\end{align}
where typical values are $E_0 \approx 1$~GeV and $k \approx 0.8$ (roughly
speaking, the fraction of energy the non-electromagnetic component accounts
for is of the order of $50\%$ at $100$~GeV and $30\%$ at $1$~TeV).

In broad terms, hadronic showers tend to start developing at relatively large
depths in the material and they are typically larger and more irregular when
compared with electromagnetic showers. All these differences are customarily used
in modern space-based imaging calorimeters for particle identification.


\section{\Cheren\ radiation}%
\label{sec:cherenkov_rad}

\begin{marginfigure}
  %% Creator: Matplotlib, PGF backend
%%
%% To include the figure in your LaTeX document, write
%%   \input{<filename>.pgf}
%%
%% Make sure the required packages are loaded in your preamble
%%   \usepackage{pgf}
%%
%% Also ensure that all the required font packages are loaded; for instance,
%% the lmodern package is sometimes necessary when using math font.
%%   \usepackage{lmodern}
%%
%% Figures using additional raster images can only be included by \input if
%% they are in the same directory as the main LaTeX file. For loading figures
%% from other directories you can use the `import` package
%%   \usepackage{import}
%%
%% and then include the figures with
%%   \import{<path to file>}{<filename>.pgf}
%%
%% Matplotlib used the following preamble
%%   \usepackage{fontspec}
%%   \setmainfont{DejaVuSerif.ttf}[Path=\detokenize{/usr/share/matplotlib/mpl-data/fonts/ttf/}]
%%   \setsansfont{DejaVuSans.ttf}[Path=\detokenize{/usr/share/matplotlib/mpl-data/fonts/ttf/}]
%%   \setmonofont{DejaVuSansMono.ttf}[Path=\detokenize{/usr/share/matplotlib/mpl-data/fonts/ttf/}]
%%
\begingroup%
\makeatletter%
\begin{pgfpicture}%
\pgfpathrectangle{\pgfpointorigin}{\pgfqpoint{1.950000in}{5.000000in}}%
\pgfusepath{use as bounding box, clip}%
\begin{pgfscope}%
\pgfsetbuttcap%
\pgfsetmiterjoin%
\definecolor{currentfill}{rgb}{1.000000,1.000000,1.000000}%
\pgfsetfillcolor{currentfill}%
\pgfsetlinewidth{0.000000pt}%
\definecolor{currentstroke}{rgb}{1.000000,1.000000,1.000000}%
\pgfsetstrokecolor{currentstroke}%
\pgfsetdash{}{0pt}%
\pgfpathmoveto{\pgfqpoint{0.000000in}{0.000000in}}%
\pgfpathlineto{\pgfqpoint{1.950000in}{0.000000in}}%
\pgfpathlineto{\pgfqpoint{1.950000in}{5.000000in}}%
\pgfpathlineto{\pgfqpoint{0.000000in}{5.000000in}}%
\pgfpathlineto{\pgfqpoint{0.000000in}{0.000000in}}%
\pgfpathclose%
\pgfusepath{fill}%
\end{pgfscope}%
\begin{pgfscope}%
\pgfpathrectangle{\pgfqpoint{0.141667in}{0.000000in}}{\pgfqpoint{1.666667in}{5.000000in}}%
\pgfusepath{clip}%
\pgfsetbuttcap%
\pgfsetmiterjoin%
\pgfsetlinewidth{1.003750pt}%
\definecolor{currentstroke}{rgb}{0.501961,0.501961,0.501961}%
\pgfsetstrokecolor{currentstroke}%
\pgfsetdash{}{0pt}%
\pgfpathmoveto{\pgfqpoint{1.475000in}{4.166667in}}%
\pgfpathcurveto{\pgfqpoint{1.475000in}{4.166667in}}{\pgfqpoint{1.475000in}{4.166667in}}{\pgfqpoint{1.475000in}{4.166667in}}%
\pgfpathcurveto{\pgfqpoint{1.475000in}{4.166667in}}{\pgfqpoint{1.475000in}{4.166667in}}{\pgfqpoint{1.475000in}{4.166667in}}%
\pgfpathcurveto{\pgfqpoint{1.475000in}{4.166667in}}{\pgfqpoint{1.475000in}{4.166667in}}{\pgfqpoint{1.475000in}{4.166667in}}%
\pgfpathcurveto{\pgfqpoint{1.475000in}{4.166667in}}{\pgfqpoint{1.475000in}{4.166667in}}{\pgfqpoint{1.475000in}{4.166667in}}%
\pgfpathcurveto{\pgfqpoint{1.475000in}{4.166667in}}{\pgfqpoint{1.475000in}{4.166667in}}{\pgfqpoint{1.475000in}{4.166667in}}%
\pgfpathcurveto{\pgfqpoint{1.475000in}{4.166667in}}{\pgfqpoint{1.475000in}{4.166667in}}{\pgfqpoint{1.475000in}{4.166667in}}%
\pgfpathcurveto{\pgfqpoint{1.475000in}{4.166667in}}{\pgfqpoint{1.475000in}{4.166667in}}{\pgfqpoint{1.475000in}{4.166667in}}%
\pgfpathcurveto{\pgfqpoint{1.475000in}{4.166667in}}{\pgfqpoint{1.475000in}{4.166667in}}{\pgfqpoint{1.475000in}{4.166667in}}%
\pgfpathlineto{\pgfqpoint{1.475000in}{4.166667in}}%
\pgfpathclose%
\pgfusepath{stroke}%
\end{pgfscope}%
\begin{pgfscope}%
\pgfpathrectangle{\pgfqpoint{0.141667in}{0.000000in}}{\pgfqpoint{1.666667in}{5.000000in}}%
\pgfusepath{clip}%
\pgfsetbuttcap%
\pgfsetmiterjoin%
\pgfsetlinewidth{1.003750pt}%
\definecolor{currentstroke}{rgb}{0.501961,0.501961,0.501961}%
\pgfsetstrokecolor{currentstroke}%
\pgfsetdash{}{0pt}%
\pgfpathmoveto{\pgfqpoint{1.417857in}{4.071429in}}%
\pgfpathcurveto{\pgfqpoint{1.443115in}{4.071429in}}{\pgfqpoint{1.467341in}{4.081463in}}{\pgfqpoint{1.485201in}{4.099323in}}%
\pgfpathcurveto{\pgfqpoint{1.503060in}{4.117183in}}{\pgfqpoint{1.513095in}{4.141409in}}{\pgfqpoint{1.513095in}{4.166667in}}%
\pgfpathcurveto{\pgfqpoint{1.513095in}{4.191924in}}{\pgfqpoint{1.503060in}{4.216150in}}{\pgfqpoint{1.485201in}{4.234010in}}%
\pgfpathcurveto{\pgfqpoint{1.467341in}{4.251870in}}{\pgfqpoint{1.443115in}{4.261905in}}{\pgfqpoint{1.417857in}{4.261905in}}%
\pgfpathcurveto{\pgfqpoint{1.392600in}{4.261905in}}{\pgfqpoint{1.368373in}{4.251870in}}{\pgfqpoint{1.350514in}{4.234010in}}%
\pgfpathcurveto{\pgfqpoint{1.332654in}{4.216150in}}{\pgfqpoint{1.322619in}{4.191924in}}{\pgfqpoint{1.322619in}{4.166667in}}%
\pgfpathcurveto{\pgfqpoint{1.322619in}{4.141409in}}{\pgfqpoint{1.332654in}{4.117183in}}{\pgfqpoint{1.350514in}{4.099323in}}%
\pgfpathcurveto{\pgfqpoint{1.368373in}{4.081463in}}{\pgfqpoint{1.392600in}{4.071429in}}{\pgfqpoint{1.417857in}{4.071429in}}%
\pgfpathlineto{\pgfqpoint{1.417857in}{4.071429in}}%
\pgfpathclose%
\pgfusepath{stroke}%
\end{pgfscope}%
\begin{pgfscope}%
\pgfpathrectangle{\pgfqpoint{0.141667in}{0.000000in}}{\pgfqpoint{1.666667in}{5.000000in}}%
\pgfusepath{clip}%
\pgfsetbuttcap%
\pgfsetmiterjoin%
\pgfsetlinewidth{1.003750pt}%
\definecolor{currentstroke}{rgb}{0.501961,0.501961,0.501961}%
\pgfsetstrokecolor{currentstroke}%
\pgfsetdash{}{0pt}%
\pgfpathmoveto{\pgfqpoint{1.360714in}{3.976190in}}%
\pgfpathcurveto{\pgfqpoint{1.411229in}{3.976190in}}{\pgfqpoint{1.459682in}{3.996260in}}{\pgfqpoint{1.495401in}{4.031980in}}%
\pgfpathcurveto{\pgfqpoint{1.531121in}{4.067699in}}{\pgfqpoint{1.551190in}{4.116152in}}{\pgfqpoint{1.551190in}{4.166667in}}%
\pgfpathcurveto{\pgfqpoint{1.551190in}{4.217182in}}{\pgfqpoint{1.531121in}{4.265634in}}{\pgfqpoint{1.495401in}{4.301354in}}%
\pgfpathcurveto{\pgfqpoint{1.459682in}{4.337073in}}{\pgfqpoint{1.411229in}{4.357143in}}{\pgfqpoint{1.360714in}{4.357143in}}%
\pgfpathcurveto{\pgfqpoint{1.310199in}{4.357143in}}{\pgfqpoint{1.261747in}{4.337073in}}{\pgfqpoint{1.226027in}{4.301354in}}%
\pgfpathcurveto{\pgfqpoint{1.190308in}{4.265634in}}{\pgfqpoint{1.170238in}{4.217182in}}{\pgfqpoint{1.170238in}{4.166667in}}%
\pgfpathcurveto{\pgfqpoint{1.170238in}{4.116152in}}{\pgfqpoint{1.190308in}{4.067699in}}{\pgfqpoint{1.226027in}{4.031980in}}%
\pgfpathcurveto{\pgfqpoint{1.261747in}{3.996260in}}{\pgfqpoint{1.310199in}{3.976190in}}{\pgfqpoint{1.360714in}{3.976190in}}%
\pgfpathlineto{\pgfqpoint{1.360714in}{3.976190in}}%
\pgfpathclose%
\pgfusepath{stroke}%
\end{pgfscope}%
\begin{pgfscope}%
\pgfpathrectangle{\pgfqpoint{0.141667in}{0.000000in}}{\pgfqpoint{1.666667in}{5.000000in}}%
\pgfusepath{clip}%
\pgfsetbuttcap%
\pgfsetmiterjoin%
\pgfsetlinewidth{1.003750pt}%
\definecolor{currentstroke}{rgb}{0.501961,0.501961,0.501961}%
\pgfsetstrokecolor{currentstroke}%
\pgfsetdash{}{0pt}%
\pgfpathmoveto{\pgfqpoint{1.303571in}{3.880952in}}%
\pgfpathcurveto{\pgfqpoint{1.379344in}{3.880952in}}{\pgfqpoint{1.452023in}{3.911057in}}{\pgfqpoint{1.505602in}{3.964636in}}%
\pgfpathcurveto{\pgfqpoint{1.559181in}{4.018215in}}{\pgfqpoint{1.589286in}{4.090894in}}{\pgfqpoint{1.589286in}{4.166667in}}%
\pgfpathcurveto{\pgfqpoint{1.589286in}{4.242439in}}{\pgfqpoint{1.559181in}{4.315118in}}{\pgfqpoint{1.505602in}{4.368697in}}%
\pgfpathcurveto{\pgfqpoint{1.452023in}{4.422276in}}{\pgfqpoint{1.379344in}{4.452381in}}{\pgfqpoint{1.303571in}{4.452381in}}%
\pgfpathcurveto{\pgfqpoint{1.227799in}{4.452381in}}{\pgfqpoint{1.155120in}{4.422276in}}{\pgfqpoint{1.101541in}{4.368697in}}%
\pgfpathcurveto{\pgfqpoint{1.047962in}{4.315118in}}{\pgfqpoint{1.017857in}{4.242439in}}{\pgfqpoint{1.017857in}{4.166667in}}%
\pgfpathcurveto{\pgfqpoint{1.017857in}{4.090894in}}{\pgfqpoint{1.047962in}{4.018215in}}{\pgfqpoint{1.101541in}{3.964636in}}%
\pgfpathcurveto{\pgfqpoint{1.155120in}{3.911057in}}{\pgfqpoint{1.227799in}{3.880952in}}{\pgfqpoint{1.303571in}{3.880952in}}%
\pgfpathlineto{\pgfqpoint{1.303571in}{3.880952in}}%
\pgfpathclose%
\pgfusepath{stroke}%
\end{pgfscope}%
\begin{pgfscope}%
\pgfpathrectangle{\pgfqpoint{0.141667in}{0.000000in}}{\pgfqpoint{1.666667in}{5.000000in}}%
\pgfusepath{clip}%
\pgfsetbuttcap%
\pgfsetmiterjoin%
\pgfsetlinewidth{1.003750pt}%
\definecolor{currentstroke}{rgb}{0.501961,0.501961,0.501961}%
\pgfsetstrokecolor{currentstroke}%
\pgfsetdash{}{0pt}%
\pgfpathmoveto{\pgfqpoint{1.246429in}{3.785714in}}%
\pgfpathcurveto{\pgfqpoint{1.347458in}{3.785714in}}{\pgfqpoint{1.444364in}{3.825854in}}{\pgfqpoint{1.515803in}{3.897293in}}%
\pgfpathcurveto{\pgfqpoint{1.587241in}{3.968731in}}{\pgfqpoint{1.627381in}{4.065637in}}{\pgfqpoint{1.627381in}{4.166667in}}%
\pgfpathcurveto{\pgfqpoint{1.627381in}{4.267696in}}{\pgfqpoint{1.587241in}{4.364602in}}{\pgfqpoint{1.515803in}{4.436041in}}%
\pgfpathcurveto{\pgfqpoint{1.444364in}{4.507480in}}{\pgfqpoint{1.347458in}{4.547619in}}{\pgfqpoint{1.246429in}{4.547619in}}%
\pgfpathcurveto{\pgfqpoint{1.145399in}{4.547619in}}{\pgfqpoint{1.048493in}{4.507480in}}{\pgfqpoint{0.977055in}{4.436041in}}%
\pgfpathcurveto{\pgfqpoint{0.905616in}{4.364602in}}{\pgfqpoint{0.865476in}{4.267696in}}{\pgfqpoint{0.865476in}{4.166667in}}%
\pgfpathcurveto{\pgfqpoint{0.865476in}{4.065637in}}{\pgfqpoint{0.905616in}{3.968731in}}{\pgfqpoint{0.977055in}{3.897293in}}%
\pgfpathcurveto{\pgfqpoint{1.048493in}{3.825854in}}{\pgfqpoint{1.145399in}{3.785714in}}{\pgfqpoint{1.246429in}{3.785714in}}%
\pgfpathlineto{\pgfqpoint{1.246429in}{3.785714in}}%
\pgfpathclose%
\pgfusepath{stroke}%
\end{pgfscope}%
\begin{pgfscope}%
\pgfpathrectangle{\pgfqpoint{0.141667in}{0.000000in}}{\pgfqpoint{1.666667in}{5.000000in}}%
\pgfusepath{clip}%
\pgfsetbuttcap%
\pgfsetmiterjoin%
\pgfsetlinewidth{1.003750pt}%
\definecolor{currentstroke}{rgb}{0.501961,0.501961,0.501961}%
\pgfsetstrokecolor{currentstroke}%
\pgfsetdash{}{0pt}%
\pgfpathmoveto{\pgfqpoint{1.189286in}{3.690476in}}%
\pgfpathcurveto{\pgfqpoint{1.315573in}{3.690476in}}{\pgfqpoint{1.436705in}{3.740651in}}{\pgfqpoint{1.526003in}{3.829949in}}%
\pgfpathcurveto{\pgfqpoint{1.615302in}{3.919248in}}{\pgfqpoint{1.665476in}{4.040379in}}{\pgfqpoint{1.665476in}{4.166667in}}%
\pgfpathcurveto{\pgfqpoint{1.665476in}{4.292954in}}{\pgfqpoint{1.615302in}{4.414086in}}{\pgfqpoint{1.526003in}{4.503384in}}%
\pgfpathcurveto{\pgfqpoint{1.436705in}{4.592683in}}{\pgfqpoint{1.315573in}{4.642857in}}{\pgfqpoint{1.189286in}{4.642857in}}%
\pgfpathcurveto{\pgfqpoint{1.062999in}{4.642857in}}{\pgfqpoint{0.941867in}{4.592683in}}{\pgfqpoint{0.852568in}{4.503384in}}%
\pgfpathcurveto{\pgfqpoint{0.763270in}{4.414086in}}{\pgfqpoint{0.713095in}{4.292954in}}{\pgfqpoint{0.713095in}{4.166667in}}%
\pgfpathcurveto{\pgfqpoint{0.713095in}{4.040379in}}{\pgfqpoint{0.763270in}{3.919248in}}{\pgfqpoint{0.852568in}{3.829949in}}%
\pgfpathcurveto{\pgfqpoint{0.941867in}{3.740651in}}{\pgfqpoint{1.062999in}{3.690476in}}{\pgfqpoint{1.189286in}{3.690476in}}%
\pgfpathlineto{\pgfqpoint{1.189286in}{3.690476in}}%
\pgfpathclose%
\pgfusepath{stroke}%
\end{pgfscope}%
\begin{pgfscope}%
\pgfpathrectangle{\pgfqpoint{0.141667in}{0.000000in}}{\pgfqpoint{1.666667in}{5.000000in}}%
\pgfusepath{clip}%
\pgfsetbuttcap%
\pgfsetmiterjoin%
\pgfsetlinewidth{1.003750pt}%
\definecolor{currentstroke}{rgb}{0.501961,0.501961,0.501961}%
\pgfsetstrokecolor{currentstroke}%
\pgfsetdash{}{0pt}%
\pgfpathmoveto{\pgfqpoint{1.132143in}{3.595238in}}%
\pgfpathcurveto{\pgfqpoint{1.283687in}{3.595238in}}{\pgfqpoint{1.429046in}{3.655447in}}{\pgfqpoint{1.536204in}{3.762606in}}%
\pgfpathcurveto{\pgfqpoint{1.643362in}{3.869764in}}{\pgfqpoint{1.703571in}{4.015122in}}{\pgfqpoint{1.703571in}{4.166667in}}%
\pgfpathcurveto{\pgfqpoint{1.703571in}{4.318211in}}{\pgfqpoint{1.643362in}{4.463569in}}{\pgfqpoint{1.536204in}{4.570728in}}%
\pgfpathcurveto{\pgfqpoint{1.429046in}{4.677886in}}{\pgfqpoint{1.283687in}{4.738095in}}{\pgfqpoint{1.132143in}{4.738095in}}%
\pgfpathcurveto{\pgfqpoint{0.980598in}{4.738095in}}{\pgfqpoint{0.835240in}{4.677886in}}{\pgfqpoint{0.728082in}{4.570728in}}%
\pgfpathcurveto{\pgfqpoint{0.620924in}{4.463569in}}{\pgfqpoint{0.560714in}{4.318211in}}{\pgfqpoint{0.560714in}{4.166667in}}%
\pgfpathcurveto{\pgfqpoint{0.560714in}{4.015122in}}{\pgfqpoint{0.620924in}{3.869764in}}{\pgfqpoint{0.728082in}{3.762606in}}%
\pgfpathcurveto{\pgfqpoint{0.835240in}{3.655447in}}{\pgfqpoint{0.980598in}{3.595238in}}{\pgfqpoint{1.132143in}{3.595238in}}%
\pgfpathlineto{\pgfqpoint{1.132143in}{3.595238in}}%
\pgfpathclose%
\pgfusepath{stroke}%
\end{pgfscope}%
\begin{pgfscope}%
\pgfpathrectangle{\pgfqpoint{0.141667in}{0.000000in}}{\pgfqpoint{1.666667in}{5.000000in}}%
\pgfusepath{clip}%
\pgfsetbuttcap%
\pgfsetmiterjoin%
\pgfsetlinewidth{1.003750pt}%
\definecolor{currentstroke}{rgb}{0.501961,0.501961,0.501961}%
\pgfsetstrokecolor{currentstroke}%
\pgfsetdash{}{0pt}%
\pgfpathmoveto{\pgfqpoint{1.075000in}{3.500000in}}%
\pgfpathcurveto{\pgfqpoint{1.251802in}{3.500000in}}{\pgfqpoint{1.421387in}{3.570244in}}{\pgfqpoint{1.546405in}{3.695262in}}%
\pgfpathcurveto{\pgfqpoint{1.671422in}{3.820280in}}{\pgfqpoint{1.741667in}{3.989865in}}{\pgfqpoint{1.741667in}{4.166667in}}%
\pgfpathcurveto{\pgfqpoint{1.741667in}{4.343469in}}{\pgfqpoint{1.671422in}{4.513053in}}{\pgfqpoint{1.546405in}{4.638071in}}%
\pgfpathcurveto{\pgfqpoint{1.421387in}{4.763089in}}{\pgfqpoint{1.251802in}{4.833333in}}{\pgfqpoint{1.075000in}{4.833333in}}%
\pgfpathcurveto{\pgfqpoint{0.898198in}{4.833333in}}{\pgfqpoint{0.728613in}{4.763089in}}{\pgfqpoint{0.603595in}{4.638071in}}%
\pgfpathcurveto{\pgfqpoint{0.478578in}{4.513053in}}{\pgfqpoint{0.408333in}{4.343469in}}{\pgfqpoint{0.408333in}{4.166667in}}%
\pgfpathcurveto{\pgfqpoint{0.408333in}{3.989865in}}{\pgfqpoint{0.478578in}{3.820280in}}{\pgfqpoint{0.603595in}{3.695262in}}%
\pgfpathcurveto{\pgfqpoint{0.728613in}{3.570244in}}{\pgfqpoint{0.898198in}{3.500000in}}{\pgfqpoint{1.075000in}{3.500000in}}%
\pgfpathlineto{\pgfqpoint{1.075000in}{3.500000in}}%
\pgfpathclose%
\pgfusepath{stroke}%
\end{pgfscope}%
\begin{pgfscope}%
\pgfpathrectangle{\pgfqpoint{0.141667in}{0.000000in}}{\pgfqpoint{1.666667in}{5.000000in}}%
\pgfusepath{clip}%
\pgfsetbuttcap%
\pgfsetmiterjoin%
\pgfsetlinewidth{1.003750pt}%
\definecolor{currentstroke}{rgb}{0.501961,0.501961,0.501961}%
\pgfsetstrokecolor{currentstroke}%
\pgfsetdash{}{0pt}%
\pgfpathmoveto{\pgfqpoint{1.725000in}{2.500000in}}%
\pgfpathcurveto{\pgfqpoint{1.725000in}{2.500000in}}{\pgfqpoint{1.725000in}{2.500000in}}{\pgfqpoint{1.725000in}{2.500000in}}%
\pgfpathcurveto{\pgfqpoint{1.725000in}{2.500000in}}{\pgfqpoint{1.725000in}{2.500000in}}{\pgfqpoint{1.725000in}{2.500000in}}%
\pgfpathcurveto{\pgfqpoint{1.725000in}{2.500000in}}{\pgfqpoint{1.725000in}{2.500000in}}{\pgfqpoint{1.725000in}{2.500000in}}%
\pgfpathcurveto{\pgfqpoint{1.725000in}{2.500000in}}{\pgfqpoint{1.725000in}{2.500000in}}{\pgfqpoint{1.725000in}{2.500000in}}%
\pgfpathcurveto{\pgfqpoint{1.725000in}{2.500000in}}{\pgfqpoint{1.725000in}{2.500000in}}{\pgfqpoint{1.725000in}{2.500000in}}%
\pgfpathcurveto{\pgfqpoint{1.725000in}{2.500000in}}{\pgfqpoint{1.725000in}{2.500000in}}{\pgfqpoint{1.725000in}{2.500000in}}%
\pgfpathcurveto{\pgfqpoint{1.725000in}{2.500000in}}{\pgfqpoint{1.725000in}{2.500000in}}{\pgfqpoint{1.725000in}{2.500000in}}%
\pgfpathcurveto{\pgfqpoint{1.725000in}{2.500000in}}{\pgfqpoint{1.725000in}{2.500000in}}{\pgfqpoint{1.725000in}{2.500000in}}%
\pgfpathlineto{\pgfqpoint{1.725000in}{2.500000in}}%
\pgfpathclose%
\pgfusepath{stroke}%
\end{pgfscope}%
\begin{pgfscope}%
\pgfpathrectangle{\pgfqpoint{0.141667in}{0.000000in}}{\pgfqpoint{1.666667in}{5.000000in}}%
\pgfusepath{clip}%
\pgfsetbuttcap%
\pgfsetmiterjoin%
\pgfsetlinewidth{1.003750pt}%
\definecolor{currentstroke}{rgb}{0.501961,0.501961,0.501961}%
\pgfsetstrokecolor{currentstroke}%
\pgfsetdash{}{0pt}%
\pgfpathmoveto{\pgfqpoint{1.629762in}{2.404762in}}%
\pgfpathcurveto{\pgfqpoint{1.655019in}{2.404762in}}{\pgfqpoint{1.679246in}{2.414797in}}{\pgfqpoint{1.697105in}{2.432656in}}%
\pgfpathcurveto{\pgfqpoint{1.714965in}{2.450516in}}{\pgfqpoint{1.725000in}{2.474743in}}{\pgfqpoint{1.725000in}{2.500000in}}%
\pgfpathcurveto{\pgfqpoint{1.725000in}{2.525257in}}{\pgfqpoint{1.714965in}{2.549484in}}{\pgfqpoint{1.697105in}{2.567344in}}%
\pgfpathcurveto{\pgfqpoint{1.679246in}{2.585203in}}{\pgfqpoint{1.655019in}{2.595238in}}{\pgfqpoint{1.629762in}{2.595238in}}%
\pgfpathcurveto{\pgfqpoint{1.604504in}{2.595238in}}{\pgfqpoint{1.580278in}{2.585203in}}{\pgfqpoint{1.562418in}{2.567344in}}%
\pgfpathcurveto{\pgfqpoint{1.544559in}{2.549484in}}{\pgfqpoint{1.534524in}{2.525257in}}{\pgfqpoint{1.534524in}{2.500000in}}%
\pgfpathcurveto{\pgfqpoint{1.534524in}{2.474743in}}{\pgfqpoint{1.544559in}{2.450516in}}{\pgfqpoint{1.562418in}{2.432656in}}%
\pgfpathcurveto{\pgfqpoint{1.580278in}{2.414797in}}{\pgfqpoint{1.604504in}{2.404762in}}{\pgfqpoint{1.629762in}{2.404762in}}%
\pgfpathlineto{\pgfqpoint{1.629762in}{2.404762in}}%
\pgfpathclose%
\pgfusepath{stroke}%
\end{pgfscope}%
\begin{pgfscope}%
\pgfpathrectangle{\pgfqpoint{0.141667in}{0.000000in}}{\pgfqpoint{1.666667in}{5.000000in}}%
\pgfusepath{clip}%
\pgfsetbuttcap%
\pgfsetmiterjoin%
\pgfsetlinewidth{1.003750pt}%
\definecolor{currentstroke}{rgb}{0.501961,0.501961,0.501961}%
\pgfsetstrokecolor{currentstroke}%
\pgfsetdash{}{0pt}%
\pgfpathmoveto{\pgfqpoint{1.534524in}{2.309524in}}%
\pgfpathcurveto{\pgfqpoint{1.585039in}{2.309524in}}{\pgfqpoint{1.633491in}{2.329594in}}{\pgfqpoint{1.669211in}{2.365313in}}%
\pgfpathcurveto{\pgfqpoint{1.704930in}{2.401032in}}{\pgfqpoint{1.725000in}{2.449485in}}{\pgfqpoint{1.725000in}{2.500000in}}%
\pgfpathcurveto{\pgfqpoint{1.725000in}{2.550515in}}{\pgfqpoint{1.704930in}{2.598968in}}{\pgfqpoint{1.669211in}{2.634687in}}%
\pgfpathcurveto{\pgfqpoint{1.633491in}{2.670406in}}{\pgfqpoint{1.585039in}{2.690476in}}{\pgfqpoint{1.534524in}{2.690476in}}%
\pgfpathcurveto{\pgfqpoint{1.484009in}{2.690476in}}{\pgfqpoint{1.435556in}{2.670406in}}{\pgfqpoint{1.399837in}{2.634687in}}%
\pgfpathcurveto{\pgfqpoint{1.364117in}{2.598968in}}{\pgfqpoint{1.344048in}{2.550515in}}{\pgfqpoint{1.344048in}{2.500000in}}%
\pgfpathcurveto{\pgfqpoint{1.344048in}{2.449485in}}{\pgfqpoint{1.364117in}{2.401032in}}{\pgfqpoint{1.399837in}{2.365313in}}%
\pgfpathcurveto{\pgfqpoint{1.435556in}{2.329594in}}{\pgfqpoint{1.484009in}{2.309524in}}{\pgfqpoint{1.534524in}{2.309524in}}%
\pgfpathlineto{\pgfqpoint{1.534524in}{2.309524in}}%
\pgfpathclose%
\pgfusepath{stroke}%
\end{pgfscope}%
\begin{pgfscope}%
\pgfpathrectangle{\pgfqpoint{0.141667in}{0.000000in}}{\pgfqpoint{1.666667in}{5.000000in}}%
\pgfusepath{clip}%
\pgfsetbuttcap%
\pgfsetmiterjoin%
\pgfsetlinewidth{1.003750pt}%
\definecolor{currentstroke}{rgb}{0.501961,0.501961,0.501961}%
\pgfsetstrokecolor{currentstroke}%
\pgfsetdash{}{0pt}%
\pgfpathmoveto{\pgfqpoint{1.439286in}{2.214286in}}%
\pgfpathcurveto{\pgfqpoint{1.515058in}{2.214286in}}{\pgfqpoint{1.587737in}{2.244390in}}{\pgfqpoint{1.641316in}{2.297969in}}%
\pgfpathcurveto{\pgfqpoint{1.694895in}{2.351549in}}{\pgfqpoint{1.725000in}{2.424228in}}{\pgfqpoint{1.725000in}{2.500000in}}%
\pgfpathcurveto{\pgfqpoint{1.725000in}{2.575772in}}{\pgfqpoint{1.694895in}{2.648451in}}{\pgfqpoint{1.641316in}{2.702031in}}%
\pgfpathcurveto{\pgfqpoint{1.587737in}{2.755610in}}{\pgfqpoint{1.515058in}{2.785714in}}{\pgfqpoint{1.439286in}{2.785714in}}%
\pgfpathcurveto{\pgfqpoint{1.363513in}{2.785714in}}{\pgfqpoint{1.290834in}{2.755610in}}{\pgfqpoint{1.237255in}{2.702031in}}%
\pgfpathcurveto{\pgfqpoint{1.183676in}{2.648451in}}{\pgfqpoint{1.153571in}{2.575772in}}{\pgfqpoint{1.153571in}{2.500000in}}%
\pgfpathcurveto{\pgfqpoint{1.153571in}{2.424228in}}{\pgfqpoint{1.183676in}{2.351549in}}{\pgfqpoint{1.237255in}{2.297969in}}%
\pgfpathcurveto{\pgfqpoint{1.290834in}{2.244390in}}{\pgfqpoint{1.363513in}{2.214286in}}{\pgfqpoint{1.439286in}{2.214286in}}%
\pgfpathlineto{\pgfqpoint{1.439286in}{2.214286in}}%
\pgfpathclose%
\pgfusepath{stroke}%
\end{pgfscope}%
\begin{pgfscope}%
\pgfpathrectangle{\pgfqpoint{0.141667in}{0.000000in}}{\pgfqpoint{1.666667in}{5.000000in}}%
\pgfusepath{clip}%
\pgfsetbuttcap%
\pgfsetmiterjoin%
\pgfsetlinewidth{1.003750pt}%
\definecolor{currentstroke}{rgb}{0.501961,0.501961,0.501961}%
\pgfsetstrokecolor{currentstroke}%
\pgfsetdash{}{0pt}%
\pgfpathmoveto{\pgfqpoint{1.344048in}{2.119048in}}%
\pgfpathcurveto{\pgfqpoint{1.445077in}{2.119048in}}{\pgfqpoint{1.541983in}{2.159187in}}{\pgfqpoint{1.613422in}{2.230626in}}%
\pgfpathcurveto{\pgfqpoint{1.684860in}{2.302065in}}{\pgfqpoint{1.725000in}{2.398970in}}{\pgfqpoint{1.725000in}{2.500000in}}%
\pgfpathcurveto{\pgfqpoint{1.725000in}{2.601030in}}{\pgfqpoint{1.684860in}{2.697935in}}{\pgfqpoint{1.613422in}{2.769374in}}%
\pgfpathcurveto{\pgfqpoint{1.541983in}{2.840813in}}{\pgfqpoint{1.445077in}{2.880952in}}{\pgfqpoint{1.344048in}{2.880952in}}%
\pgfpathcurveto{\pgfqpoint{1.243018in}{2.880952in}}{\pgfqpoint{1.146112in}{2.840813in}}{\pgfqpoint{1.074674in}{2.769374in}}%
\pgfpathcurveto{\pgfqpoint{1.003235in}{2.697935in}}{\pgfqpoint{0.963095in}{2.601030in}}{\pgfqpoint{0.963095in}{2.500000in}}%
\pgfpathcurveto{\pgfqpoint{0.963095in}{2.398970in}}{\pgfqpoint{1.003235in}{2.302065in}}{\pgfqpoint{1.074674in}{2.230626in}}%
\pgfpathcurveto{\pgfqpoint{1.146112in}{2.159187in}}{\pgfqpoint{1.243018in}{2.119048in}}{\pgfqpoint{1.344048in}{2.119048in}}%
\pgfpathlineto{\pgfqpoint{1.344048in}{2.119048in}}%
\pgfpathclose%
\pgfusepath{stroke}%
\end{pgfscope}%
\begin{pgfscope}%
\pgfpathrectangle{\pgfqpoint{0.141667in}{0.000000in}}{\pgfqpoint{1.666667in}{5.000000in}}%
\pgfusepath{clip}%
\pgfsetbuttcap%
\pgfsetmiterjoin%
\pgfsetlinewidth{1.003750pt}%
\definecolor{currentstroke}{rgb}{0.501961,0.501961,0.501961}%
\pgfsetstrokecolor{currentstroke}%
\pgfsetdash{}{0pt}%
\pgfpathmoveto{\pgfqpoint{1.248810in}{2.023810in}}%
\pgfpathcurveto{\pgfqpoint{1.375097in}{2.023810in}}{\pgfqpoint{1.496229in}{2.073984in}}{\pgfqpoint{1.585527in}{2.163282in}}%
\pgfpathcurveto{\pgfqpoint{1.674826in}{2.252581in}}{\pgfqpoint{1.725000in}{2.373713in}}{\pgfqpoint{1.725000in}{2.500000in}}%
\pgfpathcurveto{\pgfqpoint{1.725000in}{2.626287in}}{\pgfqpoint{1.674826in}{2.747419in}}{\pgfqpoint{1.585527in}{2.836718in}}%
\pgfpathcurveto{\pgfqpoint{1.496229in}{2.926016in}}{\pgfqpoint{1.375097in}{2.976190in}}{\pgfqpoint{1.248810in}{2.976190in}}%
\pgfpathcurveto{\pgfqpoint{1.122522in}{2.976190in}}{\pgfqpoint{1.001391in}{2.926016in}}{\pgfqpoint{0.912092in}{2.836718in}}%
\pgfpathcurveto{\pgfqpoint{0.822793in}{2.747419in}}{\pgfqpoint{0.772619in}{2.626287in}}{\pgfqpoint{0.772619in}{2.500000in}}%
\pgfpathcurveto{\pgfqpoint{0.772619in}{2.373713in}}{\pgfqpoint{0.822793in}{2.252581in}}{\pgfqpoint{0.912092in}{2.163282in}}%
\pgfpathcurveto{\pgfqpoint{1.001391in}{2.073984in}}{\pgfqpoint{1.122522in}{2.023810in}}{\pgfqpoint{1.248810in}{2.023810in}}%
\pgfpathlineto{\pgfqpoint{1.248810in}{2.023810in}}%
\pgfpathclose%
\pgfusepath{stroke}%
\end{pgfscope}%
\begin{pgfscope}%
\pgfpathrectangle{\pgfqpoint{0.141667in}{0.000000in}}{\pgfqpoint{1.666667in}{5.000000in}}%
\pgfusepath{clip}%
\pgfsetbuttcap%
\pgfsetmiterjoin%
\pgfsetlinewidth{1.003750pt}%
\definecolor{currentstroke}{rgb}{0.501961,0.501961,0.501961}%
\pgfsetstrokecolor{currentstroke}%
\pgfsetdash{}{0pt}%
\pgfpathmoveto{\pgfqpoint{1.153571in}{1.928571in}}%
\pgfpathcurveto{\pgfqpoint{1.305116in}{1.928571in}}{\pgfqpoint{1.450474in}{1.988781in}}{\pgfqpoint{1.557632in}{2.095939in}}%
\pgfpathcurveto{\pgfqpoint{1.664791in}{2.203097in}}{\pgfqpoint{1.725000in}{2.348455in}}{\pgfqpoint{1.725000in}{2.500000in}}%
\pgfpathcurveto{\pgfqpoint{1.725000in}{2.651545in}}{\pgfqpoint{1.664791in}{2.796903in}}{\pgfqpoint{1.557632in}{2.904061in}}%
\pgfpathcurveto{\pgfqpoint{1.450474in}{3.011219in}}{\pgfqpoint{1.305116in}{3.071429in}}{\pgfqpoint{1.153571in}{3.071429in}}%
\pgfpathcurveto{\pgfqpoint{1.002027in}{3.071429in}}{\pgfqpoint{0.856669in}{3.011219in}}{\pgfqpoint{0.749510in}{2.904061in}}%
\pgfpathcurveto{\pgfqpoint{0.642352in}{2.796903in}}{\pgfqpoint{0.582143in}{2.651545in}}{\pgfqpoint{0.582143in}{2.500000in}}%
\pgfpathcurveto{\pgfqpoint{0.582143in}{2.348455in}}{\pgfqpoint{0.642352in}{2.203097in}}{\pgfqpoint{0.749510in}{2.095939in}}%
\pgfpathcurveto{\pgfqpoint{0.856669in}{1.988781in}}{\pgfqpoint{1.002027in}{1.928571in}}{\pgfqpoint{1.153571in}{1.928571in}}%
\pgfpathlineto{\pgfqpoint{1.153571in}{1.928571in}}%
\pgfpathclose%
\pgfusepath{stroke}%
\end{pgfscope}%
\begin{pgfscope}%
\pgfpathrectangle{\pgfqpoint{0.141667in}{0.000000in}}{\pgfqpoint{1.666667in}{5.000000in}}%
\pgfusepath{clip}%
\pgfsetbuttcap%
\pgfsetmiterjoin%
\pgfsetlinewidth{1.003750pt}%
\definecolor{currentstroke}{rgb}{0.501961,0.501961,0.501961}%
\pgfsetstrokecolor{currentstroke}%
\pgfsetdash{}{0pt}%
\pgfpathmoveto{\pgfqpoint{1.058333in}{1.833333in}}%
\pgfpathcurveto{\pgfqpoint{1.235135in}{1.833333in}}{\pgfqpoint{1.404720in}{1.903578in}}{\pgfqpoint{1.529738in}{2.028595in}}%
\pgfpathcurveto{\pgfqpoint{1.654756in}{2.153613in}}{\pgfqpoint{1.725000in}{2.323198in}}{\pgfqpoint{1.725000in}{2.500000in}}%
\pgfpathcurveto{\pgfqpoint{1.725000in}{2.676802in}}{\pgfqpoint{1.654756in}{2.846387in}}{\pgfqpoint{1.529738in}{2.971405in}}%
\pgfpathcurveto{\pgfqpoint{1.404720in}{3.096422in}}{\pgfqpoint{1.235135in}{3.166667in}}{\pgfqpoint{1.058333in}{3.166667in}}%
\pgfpathcurveto{\pgfqpoint{0.881531in}{3.166667in}}{\pgfqpoint{0.711947in}{3.096422in}}{\pgfqpoint{0.586929in}{2.971405in}}%
\pgfpathcurveto{\pgfqpoint{0.461911in}{2.846387in}}{\pgfqpoint{0.391667in}{2.676802in}}{\pgfqpoint{0.391667in}{2.500000in}}%
\pgfpathcurveto{\pgfqpoint{0.391667in}{2.323198in}}{\pgfqpoint{0.461911in}{2.153613in}}{\pgfqpoint{0.586929in}{2.028595in}}%
\pgfpathcurveto{\pgfqpoint{0.711947in}{1.903578in}}{\pgfqpoint{0.881531in}{1.833333in}}{\pgfqpoint{1.058333in}{1.833333in}}%
\pgfpathlineto{\pgfqpoint{1.058333in}{1.833333in}}%
\pgfpathclose%
\pgfusepath{stroke}%
\end{pgfscope}%
\begin{pgfscope}%
\pgfpathrectangle{\pgfqpoint{0.141667in}{0.000000in}}{\pgfqpoint{1.666667in}{5.000000in}}%
\pgfusepath{clip}%
\pgfsetbuttcap%
\pgfsetmiterjoin%
\pgfsetlinewidth{1.003750pt}%
\definecolor{currentstroke}{rgb}{0.501961,0.501961,0.501961}%
\pgfsetstrokecolor{currentstroke}%
\pgfsetdash{}{0pt}%
\pgfpathmoveto{\pgfqpoint{1.725000in}{0.833333in}}%
\pgfpathcurveto{\pgfqpoint{1.725000in}{0.833333in}}{\pgfqpoint{1.725000in}{0.833333in}}{\pgfqpoint{1.725000in}{0.833333in}}%
\pgfpathcurveto{\pgfqpoint{1.725000in}{0.833333in}}{\pgfqpoint{1.725000in}{0.833333in}}{\pgfqpoint{1.725000in}{0.833333in}}%
\pgfpathcurveto{\pgfqpoint{1.725000in}{0.833333in}}{\pgfqpoint{1.725000in}{0.833333in}}{\pgfqpoint{1.725000in}{0.833333in}}%
\pgfpathcurveto{\pgfqpoint{1.725000in}{0.833333in}}{\pgfqpoint{1.725000in}{0.833333in}}{\pgfqpoint{1.725000in}{0.833333in}}%
\pgfpathcurveto{\pgfqpoint{1.725000in}{0.833333in}}{\pgfqpoint{1.725000in}{0.833333in}}{\pgfqpoint{1.725000in}{0.833333in}}%
\pgfpathcurveto{\pgfqpoint{1.725000in}{0.833333in}}{\pgfqpoint{1.725000in}{0.833333in}}{\pgfqpoint{1.725000in}{0.833333in}}%
\pgfpathcurveto{\pgfqpoint{1.725000in}{0.833333in}}{\pgfqpoint{1.725000in}{0.833333in}}{\pgfqpoint{1.725000in}{0.833333in}}%
\pgfpathcurveto{\pgfqpoint{1.725000in}{0.833333in}}{\pgfqpoint{1.725000in}{0.833333in}}{\pgfqpoint{1.725000in}{0.833333in}}%
\pgfpathlineto{\pgfqpoint{1.725000in}{0.833333in}}%
\pgfpathclose%
\pgfusepath{stroke}%
\end{pgfscope}%
\begin{pgfscope}%
\pgfpathrectangle{\pgfqpoint{0.141667in}{0.000000in}}{\pgfqpoint{1.666667in}{5.000000in}}%
\pgfusepath{clip}%
\pgfsetbuttcap%
\pgfsetmiterjoin%
\pgfsetlinewidth{1.003750pt}%
\definecolor{currentstroke}{rgb}{0.501961,0.501961,0.501961}%
\pgfsetstrokecolor{currentstroke}%
\pgfsetdash{}{0pt}%
\pgfpathmoveto{\pgfqpoint{1.610714in}{0.738095in}}%
\pgfpathcurveto{\pgfqpoint{1.635972in}{0.738095in}}{\pgfqpoint{1.660198in}{0.748130in}}{\pgfqpoint{1.678058in}{0.765990in}}%
\pgfpathcurveto{\pgfqpoint{1.695917in}{0.783850in}}{\pgfqpoint{1.705952in}{0.808076in}}{\pgfqpoint{1.705952in}{0.833333in}}%
\pgfpathcurveto{\pgfqpoint{1.705952in}{0.858591in}}{\pgfqpoint{1.695917in}{0.882817in}}{\pgfqpoint{1.678058in}{0.900677in}}%
\pgfpathcurveto{\pgfqpoint{1.660198in}{0.918537in}}{\pgfqpoint{1.635972in}{0.928571in}}{\pgfqpoint{1.610714in}{0.928571in}}%
\pgfpathcurveto{\pgfqpoint{1.585457in}{0.928571in}}{\pgfqpoint{1.561230in}{0.918537in}}{\pgfqpoint{1.543371in}{0.900677in}}%
\pgfpathcurveto{\pgfqpoint{1.525511in}{0.882817in}}{\pgfqpoint{1.515476in}{0.858591in}}{\pgfqpoint{1.515476in}{0.833333in}}%
\pgfpathcurveto{\pgfqpoint{1.515476in}{0.808076in}}{\pgfqpoint{1.525511in}{0.783850in}}{\pgfqpoint{1.543371in}{0.765990in}}%
\pgfpathcurveto{\pgfqpoint{1.561230in}{0.748130in}}{\pgfqpoint{1.585457in}{0.738095in}}{\pgfqpoint{1.610714in}{0.738095in}}%
\pgfpathlineto{\pgfqpoint{1.610714in}{0.738095in}}%
\pgfpathclose%
\pgfusepath{stroke}%
\end{pgfscope}%
\begin{pgfscope}%
\pgfpathrectangle{\pgfqpoint{0.141667in}{0.000000in}}{\pgfqpoint{1.666667in}{5.000000in}}%
\pgfusepath{clip}%
\pgfsetbuttcap%
\pgfsetmiterjoin%
\pgfsetlinewidth{1.003750pt}%
\definecolor{currentstroke}{rgb}{0.501961,0.501961,0.501961}%
\pgfsetstrokecolor{currentstroke}%
\pgfsetdash{}{0pt}%
\pgfpathmoveto{\pgfqpoint{1.496429in}{0.642857in}}%
\pgfpathcurveto{\pgfqpoint{1.546943in}{0.642857in}}{\pgfqpoint{1.595396in}{0.662927in}}{\pgfqpoint{1.631116in}{0.698646in}}%
\pgfpathcurveto{\pgfqpoint{1.666835in}{0.734366in}}{\pgfqpoint{1.686905in}{0.782818in}}{\pgfqpoint{1.686905in}{0.833333in}}%
\pgfpathcurveto{\pgfqpoint{1.686905in}{0.883848in}}{\pgfqpoint{1.666835in}{0.932301in}}{\pgfqpoint{1.631116in}{0.968020in}}%
\pgfpathcurveto{\pgfqpoint{1.595396in}{1.003740in}}{\pgfqpoint{1.546943in}{1.023810in}}{\pgfqpoint{1.496429in}{1.023810in}}%
\pgfpathcurveto{\pgfqpoint{1.445914in}{1.023810in}}{\pgfqpoint{1.397461in}{1.003740in}}{\pgfqpoint{1.361742in}{0.968020in}}%
\pgfpathcurveto{\pgfqpoint{1.326022in}{0.932301in}}{\pgfqpoint{1.305952in}{0.883848in}}{\pgfqpoint{1.305952in}{0.833333in}}%
\pgfpathcurveto{\pgfqpoint{1.305952in}{0.782818in}}{\pgfqpoint{1.326022in}{0.734366in}}{\pgfqpoint{1.361742in}{0.698646in}}%
\pgfpathcurveto{\pgfqpoint{1.397461in}{0.662927in}}{\pgfqpoint{1.445914in}{0.642857in}}{\pgfqpoint{1.496429in}{0.642857in}}%
\pgfpathlineto{\pgfqpoint{1.496429in}{0.642857in}}%
\pgfpathclose%
\pgfusepath{stroke}%
\end{pgfscope}%
\begin{pgfscope}%
\pgfpathrectangle{\pgfqpoint{0.141667in}{0.000000in}}{\pgfqpoint{1.666667in}{5.000000in}}%
\pgfusepath{clip}%
\pgfsetbuttcap%
\pgfsetmiterjoin%
\pgfsetlinewidth{1.003750pt}%
\definecolor{currentstroke}{rgb}{0.501961,0.501961,0.501961}%
\pgfsetstrokecolor{currentstroke}%
\pgfsetdash{}{0pt}%
\pgfpathmoveto{\pgfqpoint{1.382143in}{0.547619in}}%
\pgfpathcurveto{\pgfqpoint{1.457915in}{0.547619in}}{\pgfqpoint{1.530594in}{0.577724in}}{\pgfqpoint{1.584173in}{0.631303in}}%
\pgfpathcurveto{\pgfqpoint{1.637752in}{0.684882in}}{\pgfqpoint{1.667857in}{0.757561in}}{\pgfqpoint{1.667857in}{0.833333in}}%
\pgfpathcurveto{\pgfqpoint{1.667857in}{0.909106in}}{\pgfqpoint{1.637752in}{0.981785in}}{\pgfqpoint{1.584173in}{1.035364in}}%
\pgfpathcurveto{\pgfqpoint{1.530594in}{1.088943in}}{\pgfqpoint{1.457915in}{1.119048in}}{\pgfqpoint{1.382143in}{1.119048in}}%
\pgfpathcurveto{\pgfqpoint{1.306371in}{1.119048in}}{\pgfqpoint{1.233691in}{1.088943in}}{\pgfqpoint{1.180112in}{1.035364in}}%
\pgfpathcurveto{\pgfqpoint{1.126533in}{0.981785in}}{\pgfqpoint{1.096429in}{0.909106in}}{\pgfqpoint{1.096429in}{0.833333in}}%
\pgfpathcurveto{\pgfqpoint{1.096429in}{0.757561in}}{\pgfqpoint{1.126533in}{0.684882in}}{\pgfqpoint{1.180112in}{0.631303in}}%
\pgfpathcurveto{\pgfqpoint{1.233691in}{0.577724in}}{\pgfqpoint{1.306371in}{0.547619in}}{\pgfqpoint{1.382143in}{0.547619in}}%
\pgfpathlineto{\pgfqpoint{1.382143in}{0.547619in}}%
\pgfpathclose%
\pgfusepath{stroke}%
\end{pgfscope}%
\begin{pgfscope}%
\pgfpathrectangle{\pgfqpoint{0.141667in}{0.000000in}}{\pgfqpoint{1.666667in}{5.000000in}}%
\pgfusepath{clip}%
\pgfsetbuttcap%
\pgfsetmiterjoin%
\pgfsetlinewidth{1.003750pt}%
\definecolor{currentstroke}{rgb}{0.501961,0.501961,0.501961}%
\pgfsetstrokecolor{currentstroke}%
\pgfsetdash{}{0pt}%
\pgfpathmoveto{\pgfqpoint{1.267857in}{0.452381in}}%
\pgfpathcurveto{\pgfqpoint{1.368887in}{0.452381in}}{\pgfqpoint{1.465792in}{0.492520in}}{\pgfqpoint{1.537231in}{0.563959in}}%
\pgfpathcurveto{\pgfqpoint{1.608670in}{0.635398in}}{\pgfqpoint{1.648810in}{0.732304in}}{\pgfqpoint{1.648810in}{0.833333in}}%
\pgfpathcurveto{\pgfqpoint{1.648810in}{0.934363in}}{\pgfqpoint{1.608670in}{1.031269in}}{\pgfqpoint{1.537231in}{1.102707in}}%
\pgfpathcurveto{\pgfqpoint{1.465792in}{1.174146in}}{\pgfqpoint{1.368887in}{1.214286in}}{\pgfqpoint{1.267857in}{1.214286in}}%
\pgfpathcurveto{\pgfqpoint{1.166827in}{1.214286in}}{\pgfqpoint{1.069922in}{1.174146in}}{\pgfqpoint{0.998483in}{1.102707in}}%
\pgfpathcurveto{\pgfqpoint{0.927044in}{1.031269in}}{\pgfqpoint{0.886905in}{0.934363in}}{\pgfqpoint{0.886905in}{0.833333in}}%
\pgfpathcurveto{\pgfqpoint{0.886905in}{0.732304in}}{\pgfqpoint{0.927044in}{0.635398in}}{\pgfqpoint{0.998483in}{0.563959in}}%
\pgfpathcurveto{\pgfqpoint{1.069922in}{0.492520in}}{\pgfqpoint{1.166827in}{0.452381in}}{\pgfqpoint{1.267857in}{0.452381in}}%
\pgfpathlineto{\pgfqpoint{1.267857in}{0.452381in}}%
\pgfpathclose%
\pgfusepath{stroke}%
\end{pgfscope}%
\begin{pgfscope}%
\pgfpathrectangle{\pgfqpoint{0.141667in}{0.000000in}}{\pgfqpoint{1.666667in}{5.000000in}}%
\pgfusepath{clip}%
\pgfsetbuttcap%
\pgfsetmiterjoin%
\pgfsetlinewidth{1.003750pt}%
\definecolor{currentstroke}{rgb}{0.501961,0.501961,0.501961}%
\pgfsetstrokecolor{currentstroke}%
\pgfsetdash{}{0pt}%
\pgfpathmoveto{\pgfqpoint{1.153571in}{0.357143in}}%
\pgfpathcurveto{\pgfqpoint{1.279859in}{0.357143in}}{\pgfqpoint{1.400990in}{0.407317in}}{\pgfqpoint{1.490289in}{0.496616in}}%
\pgfpathcurveto{\pgfqpoint{1.579587in}{0.585914in}}{\pgfqpoint{1.629762in}{0.707046in}}{\pgfqpoint{1.629762in}{0.833333in}}%
\pgfpathcurveto{\pgfqpoint{1.629762in}{0.959621in}}{\pgfqpoint{1.579587in}{1.080752in}}{\pgfqpoint{1.490289in}{1.170051in}}%
\pgfpathcurveto{\pgfqpoint{1.400990in}{1.259349in}}{\pgfqpoint{1.279859in}{1.309524in}}{\pgfqpoint{1.153571in}{1.309524in}}%
\pgfpathcurveto{\pgfqpoint{1.027284in}{1.309524in}}{\pgfqpoint{0.906152in}{1.259349in}}{\pgfqpoint{0.816854in}{1.170051in}}%
\pgfpathcurveto{\pgfqpoint{0.727555in}{1.080752in}}{\pgfqpoint{0.677381in}{0.959621in}}{\pgfqpoint{0.677381in}{0.833333in}}%
\pgfpathcurveto{\pgfqpoint{0.677381in}{0.707046in}}{\pgfqpoint{0.727555in}{0.585914in}}{\pgfqpoint{0.816854in}{0.496616in}}%
\pgfpathcurveto{\pgfqpoint{0.906152in}{0.407317in}}{\pgfqpoint{1.027284in}{0.357143in}}{\pgfqpoint{1.153571in}{0.357143in}}%
\pgfpathlineto{\pgfqpoint{1.153571in}{0.357143in}}%
\pgfpathclose%
\pgfusepath{stroke}%
\end{pgfscope}%
\begin{pgfscope}%
\pgfpathrectangle{\pgfqpoint{0.141667in}{0.000000in}}{\pgfqpoint{1.666667in}{5.000000in}}%
\pgfusepath{clip}%
\pgfsetbuttcap%
\pgfsetmiterjoin%
\pgfsetlinewidth{1.003750pt}%
\definecolor{currentstroke}{rgb}{0.501961,0.501961,0.501961}%
\pgfsetstrokecolor{currentstroke}%
\pgfsetdash{}{0pt}%
\pgfpathmoveto{\pgfqpoint{1.039286in}{0.261905in}}%
\pgfpathcurveto{\pgfqpoint{1.190830in}{0.261905in}}{\pgfqpoint{1.336188in}{0.322114in}}{\pgfqpoint{1.443347in}{0.429272in}}%
\pgfpathcurveto{\pgfqpoint{1.550505in}{0.536431in}}{\pgfqpoint{1.610714in}{0.681789in}}{\pgfqpoint{1.610714in}{0.833333in}}%
\pgfpathcurveto{\pgfqpoint{1.610714in}{0.984878in}}{\pgfqpoint{1.550505in}{1.130236in}}{\pgfqpoint{1.443347in}{1.237394in}}%
\pgfpathcurveto{\pgfqpoint{1.336188in}{1.344553in}}{\pgfqpoint{1.190830in}{1.404762in}}{\pgfqpoint{1.039286in}{1.404762in}}%
\pgfpathcurveto{\pgfqpoint{0.887741in}{1.404762in}}{\pgfqpoint{0.742383in}{1.344553in}}{\pgfqpoint{0.635225in}{1.237394in}}%
\pgfpathcurveto{\pgfqpoint{0.528066in}{1.130236in}}{\pgfqpoint{0.467857in}{0.984878in}}{\pgfqpoint{0.467857in}{0.833333in}}%
\pgfpathcurveto{\pgfqpoint{0.467857in}{0.681789in}}{\pgfqpoint{0.528066in}{0.536431in}}{\pgfqpoint{0.635225in}{0.429272in}}%
\pgfpathcurveto{\pgfqpoint{0.742383in}{0.322114in}}{\pgfqpoint{0.887741in}{0.261905in}}{\pgfqpoint{1.039286in}{0.261905in}}%
\pgfpathlineto{\pgfqpoint{1.039286in}{0.261905in}}%
\pgfpathclose%
\pgfusepath{stroke}%
\end{pgfscope}%
\begin{pgfscope}%
\pgfpathrectangle{\pgfqpoint{0.141667in}{0.000000in}}{\pgfqpoint{1.666667in}{5.000000in}}%
\pgfusepath{clip}%
\pgfsetbuttcap%
\pgfsetmiterjoin%
\pgfsetlinewidth{1.003750pt}%
\definecolor{currentstroke}{rgb}{0.501961,0.501961,0.501961}%
\pgfsetstrokecolor{currentstroke}%
\pgfsetdash{}{0pt}%
\pgfpathmoveto{\pgfqpoint{0.925000in}{0.166667in}}%
\pgfpathcurveto{\pgfqpoint{1.101802in}{0.166667in}}{\pgfqpoint{1.271387in}{0.236911in}}{\pgfqpoint{1.396405in}{0.361929in}}%
\pgfpathcurveto{\pgfqpoint{1.521422in}{0.486947in}}{\pgfqpoint{1.591667in}{0.656531in}}{\pgfqpoint{1.591667in}{0.833333in}}%
\pgfpathcurveto{\pgfqpoint{1.591667in}{1.010135in}}{\pgfqpoint{1.521422in}{1.179720in}}{\pgfqpoint{1.396405in}{1.304738in}}%
\pgfpathcurveto{\pgfqpoint{1.271387in}{1.429756in}}{\pgfqpoint{1.101802in}{1.500000in}}{\pgfqpoint{0.925000in}{1.500000in}}%
\pgfpathcurveto{\pgfqpoint{0.748198in}{1.500000in}}{\pgfqpoint{0.578613in}{1.429756in}}{\pgfqpoint{0.453595in}{1.304738in}}%
\pgfpathcurveto{\pgfqpoint{0.328578in}{1.179720in}}{\pgfqpoint{0.258333in}{1.010135in}}{\pgfqpoint{0.258333in}{0.833333in}}%
\pgfpathcurveto{\pgfqpoint{0.258333in}{0.656531in}}{\pgfqpoint{0.328578in}{0.486947in}}{\pgfqpoint{0.453595in}{0.361929in}}%
\pgfpathcurveto{\pgfqpoint{0.578613in}{0.236911in}}{\pgfqpoint{0.748198in}{0.166667in}}{\pgfqpoint{0.925000in}{0.166667in}}%
\pgfpathlineto{\pgfqpoint{0.925000in}{0.166667in}}%
\pgfpathclose%
\pgfusepath{stroke}%
\end{pgfscope}%
\begin{pgfscope}%
\pgfpathrectangle{\pgfqpoint{0.141667in}{0.000000in}}{\pgfqpoint{1.666667in}{5.000000in}}%
\pgfusepath{clip}%
\pgfsetbuttcap%
\pgfsetroundjoin%
\pgfsetlinewidth{1.003750pt}%
\definecolor{currentstroke}{rgb}{0.000000,0.000000,0.000000}%
\pgfsetstrokecolor{currentstroke}%
\pgfsetdash{{3.700000pt}{1.600000pt}}{0.000000pt}%
\pgfpathmoveto{\pgfqpoint{0.141667in}{4.166667in}}%
\pgfpathlineto{\pgfqpoint{1.808333in}{4.166667in}}%
\pgfusepath{stroke}%
\end{pgfscope}%
\begin{pgfscope}%
\pgfpathrectangle{\pgfqpoint{0.141667in}{0.000000in}}{\pgfqpoint{1.666667in}{5.000000in}}%
\pgfusepath{clip}%
\pgfsetbuttcap%
\pgfsetroundjoin%
\definecolor{currentfill}{rgb}{0.000000,0.000000,0.000000}%
\pgfsetfillcolor{currentfill}%
\pgfsetlinewidth{1.003750pt}%
\definecolor{currentstroke}{rgb}{0.000000,0.000000,0.000000}%
\pgfsetstrokecolor{currentstroke}%
\pgfsetdash{}{0pt}%
\pgfsys@defobject{currentmarker}{\pgfqpoint{-0.041667in}{-0.041667in}}{\pgfqpoint{0.041667in}{0.041667in}}{%
\pgfpathmoveto{\pgfqpoint{0.000000in}{-0.041667in}}%
\pgfpathcurveto{\pgfqpoint{0.011050in}{-0.041667in}}{\pgfqpoint{0.021649in}{-0.037276in}}{\pgfqpoint{0.029463in}{-0.029463in}}%
\pgfpathcurveto{\pgfqpoint{0.037276in}{-0.021649in}}{\pgfqpoint{0.041667in}{-0.011050in}}{\pgfqpoint{0.041667in}{0.000000in}}%
\pgfpathcurveto{\pgfqpoint{0.041667in}{0.011050in}}{\pgfqpoint{0.037276in}{0.021649in}}{\pgfqpoint{0.029463in}{0.029463in}}%
\pgfpathcurveto{\pgfqpoint{0.021649in}{0.037276in}}{\pgfqpoint{0.011050in}{0.041667in}}{\pgfqpoint{0.000000in}{0.041667in}}%
\pgfpathcurveto{\pgfqpoint{-0.011050in}{0.041667in}}{\pgfqpoint{-0.021649in}{0.037276in}}{\pgfqpoint{-0.029463in}{0.029463in}}%
\pgfpathcurveto{\pgfqpoint{-0.037276in}{0.021649in}}{\pgfqpoint{-0.041667in}{0.011050in}}{\pgfqpoint{-0.041667in}{0.000000in}}%
\pgfpathcurveto{\pgfqpoint{-0.041667in}{-0.011050in}}{\pgfqpoint{-0.037276in}{-0.021649in}}{\pgfqpoint{-0.029463in}{-0.029463in}}%
\pgfpathcurveto{\pgfqpoint{-0.021649in}{-0.037276in}}{\pgfqpoint{-0.011050in}{-0.041667in}}{\pgfqpoint{0.000000in}{-0.041667in}}%
\pgfpathlineto{\pgfqpoint{0.000000in}{-0.041667in}}%
\pgfpathclose%
\pgfusepath{stroke,fill}%
}%
\begin{pgfscope}%
\pgfsys@transformshift{1.475000in}{4.166667in}%
\pgfsys@useobject{currentmarker}{}%
\end{pgfscope}%
\end{pgfscope}%
\begin{pgfscope}%
\pgfpathrectangle{\pgfqpoint{0.141667in}{0.000000in}}{\pgfqpoint{1.666667in}{5.000000in}}%
\pgfusepath{clip}%
\pgfsetbuttcap%
\pgfsetroundjoin%
\pgfsetlinewidth{1.003750pt}%
\definecolor{currentstroke}{rgb}{0.000000,0.000000,0.000000}%
\pgfsetstrokecolor{currentstroke}%
\pgfsetdash{{3.700000pt}{1.600000pt}}{0.000000pt}%
\pgfpathmoveto{\pgfqpoint{0.141667in}{2.500000in}}%
\pgfpathlineto{\pgfqpoint{1.808333in}{2.500000in}}%
\pgfusepath{stroke}%
\end{pgfscope}%
\begin{pgfscope}%
\pgfpathrectangle{\pgfqpoint{0.141667in}{0.000000in}}{\pgfqpoint{1.666667in}{5.000000in}}%
\pgfusepath{clip}%
\pgfsetbuttcap%
\pgfsetroundjoin%
\definecolor{currentfill}{rgb}{0.000000,0.000000,0.000000}%
\pgfsetfillcolor{currentfill}%
\pgfsetlinewidth{1.003750pt}%
\definecolor{currentstroke}{rgb}{0.000000,0.000000,0.000000}%
\pgfsetstrokecolor{currentstroke}%
\pgfsetdash{}{0pt}%
\pgfsys@defobject{currentmarker}{\pgfqpoint{-0.041667in}{-0.041667in}}{\pgfqpoint{0.041667in}{0.041667in}}{%
\pgfpathmoveto{\pgfqpoint{0.000000in}{-0.041667in}}%
\pgfpathcurveto{\pgfqpoint{0.011050in}{-0.041667in}}{\pgfqpoint{0.021649in}{-0.037276in}}{\pgfqpoint{0.029463in}{-0.029463in}}%
\pgfpathcurveto{\pgfqpoint{0.037276in}{-0.021649in}}{\pgfqpoint{0.041667in}{-0.011050in}}{\pgfqpoint{0.041667in}{0.000000in}}%
\pgfpathcurveto{\pgfqpoint{0.041667in}{0.011050in}}{\pgfqpoint{0.037276in}{0.021649in}}{\pgfqpoint{0.029463in}{0.029463in}}%
\pgfpathcurveto{\pgfqpoint{0.021649in}{0.037276in}}{\pgfqpoint{0.011050in}{0.041667in}}{\pgfqpoint{0.000000in}{0.041667in}}%
\pgfpathcurveto{\pgfqpoint{-0.011050in}{0.041667in}}{\pgfqpoint{-0.021649in}{0.037276in}}{\pgfqpoint{-0.029463in}{0.029463in}}%
\pgfpathcurveto{\pgfqpoint{-0.037276in}{0.021649in}}{\pgfqpoint{-0.041667in}{0.011050in}}{\pgfqpoint{-0.041667in}{0.000000in}}%
\pgfpathcurveto{\pgfqpoint{-0.041667in}{-0.011050in}}{\pgfqpoint{-0.037276in}{-0.021649in}}{\pgfqpoint{-0.029463in}{-0.029463in}}%
\pgfpathcurveto{\pgfqpoint{-0.021649in}{-0.037276in}}{\pgfqpoint{-0.011050in}{-0.041667in}}{\pgfqpoint{0.000000in}{-0.041667in}}%
\pgfpathlineto{\pgfqpoint{0.000000in}{-0.041667in}}%
\pgfpathclose%
\pgfusepath{stroke,fill}%
}%
\begin{pgfscope}%
\pgfsys@transformshift{1.725000in}{2.500000in}%
\pgfsys@useobject{currentmarker}{}%
\end{pgfscope}%
\end{pgfscope}%
\begin{pgfscope}%
\pgfpathrectangle{\pgfqpoint{0.141667in}{0.000000in}}{\pgfqpoint{1.666667in}{5.000000in}}%
\pgfusepath{clip}%
\pgfsetrectcap%
\pgfsetroundjoin%
\pgfsetlinewidth{1.003750pt}%
\definecolor{currentstroke}{rgb}{0.000000,0.000000,0.000000}%
\pgfsetstrokecolor{currentstroke}%
\pgfsetdash{}{0pt}%
\pgfpathmoveto{\pgfqpoint{1.725000in}{1.700000in}}%
\pgfpathlineto{\pgfqpoint{1.725000in}{3.300000in}}%
\pgfusepath{stroke}%
\end{pgfscope}%
\begin{pgfscope}%
\pgfpathrectangle{\pgfqpoint{0.141667in}{0.000000in}}{\pgfqpoint{1.666667in}{5.000000in}}%
\pgfusepath{clip}%
\pgfsetbuttcap%
\pgfsetroundjoin%
\pgfsetlinewidth{1.003750pt}%
\definecolor{currentstroke}{rgb}{0.000000,0.000000,0.000000}%
\pgfsetstrokecolor{currentstroke}%
\pgfsetdash{{3.700000pt}{1.600000pt}}{0.000000pt}%
\pgfpathmoveto{\pgfqpoint{0.141667in}{0.833333in}}%
\pgfpathlineto{\pgfqpoint{1.808333in}{0.833333in}}%
\pgfusepath{stroke}%
\end{pgfscope}%
\begin{pgfscope}%
\pgfpathrectangle{\pgfqpoint{0.141667in}{0.000000in}}{\pgfqpoint{1.666667in}{5.000000in}}%
\pgfusepath{clip}%
\pgfsetbuttcap%
\pgfsetroundjoin%
\definecolor{currentfill}{rgb}{0.000000,0.000000,0.000000}%
\pgfsetfillcolor{currentfill}%
\pgfsetlinewidth{1.003750pt}%
\definecolor{currentstroke}{rgb}{0.000000,0.000000,0.000000}%
\pgfsetstrokecolor{currentstroke}%
\pgfsetdash{}{0pt}%
\pgfsys@defobject{currentmarker}{\pgfqpoint{-0.041667in}{-0.041667in}}{\pgfqpoint{0.041667in}{0.041667in}}{%
\pgfpathmoveto{\pgfqpoint{0.000000in}{-0.041667in}}%
\pgfpathcurveto{\pgfqpoint{0.011050in}{-0.041667in}}{\pgfqpoint{0.021649in}{-0.037276in}}{\pgfqpoint{0.029463in}{-0.029463in}}%
\pgfpathcurveto{\pgfqpoint{0.037276in}{-0.021649in}}{\pgfqpoint{0.041667in}{-0.011050in}}{\pgfqpoint{0.041667in}{0.000000in}}%
\pgfpathcurveto{\pgfqpoint{0.041667in}{0.011050in}}{\pgfqpoint{0.037276in}{0.021649in}}{\pgfqpoint{0.029463in}{0.029463in}}%
\pgfpathcurveto{\pgfqpoint{0.021649in}{0.037276in}}{\pgfqpoint{0.011050in}{0.041667in}}{\pgfqpoint{0.000000in}{0.041667in}}%
\pgfpathcurveto{\pgfqpoint{-0.011050in}{0.041667in}}{\pgfqpoint{-0.021649in}{0.037276in}}{\pgfqpoint{-0.029463in}{0.029463in}}%
\pgfpathcurveto{\pgfqpoint{-0.037276in}{0.021649in}}{\pgfqpoint{-0.041667in}{0.011050in}}{\pgfqpoint{-0.041667in}{0.000000in}}%
\pgfpathcurveto{\pgfqpoint{-0.041667in}{-0.011050in}}{\pgfqpoint{-0.037276in}{-0.021649in}}{\pgfqpoint{-0.029463in}{-0.029463in}}%
\pgfpathcurveto{\pgfqpoint{-0.021649in}{-0.037276in}}{\pgfqpoint{-0.011050in}{-0.041667in}}{\pgfqpoint{0.000000in}{-0.041667in}}%
\pgfpathlineto{\pgfqpoint{0.000000in}{-0.041667in}}%
\pgfpathclose%
\pgfusepath{stroke,fill}%
}%
\begin{pgfscope}%
\pgfsys@transformshift{1.725000in}{0.833333in}%
\pgfsys@useobject{currentmarker}{}%
\end{pgfscope}%
\end{pgfscope}%
\begin{pgfscope}%
\pgfpathrectangle{\pgfqpoint{0.141667in}{0.000000in}}{\pgfqpoint{1.666667in}{5.000000in}}%
\pgfusepath{clip}%
\pgfsetrectcap%
\pgfsetroundjoin%
\pgfsetlinewidth{1.003750pt}%
\definecolor{currentstroke}{rgb}{0.000000,0.000000,0.000000}%
\pgfsetstrokecolor{currentstroke}%
\pgfsetdash{}{0pt}%
\pgfpathmoveto{\pgfqpoint{1.725000in}{0.833333in}}%
\pgfpathlineto{\pgfqpoint{1.194340in}{1.633333in}}%
\pgfusepath{stroke}%
\end{pgfscope}%
\begin{pgfscope}%
\pgfpathrectangle{\pgfqpoint{0.141667in}{0.000000in}}{\pgfqpoint{1.666667in}{5.000000in}}%
\pgfusepath{clip}%
\pgfsetrectcap%
\pgfsetroundjoin%
\pgfsetlinewidth{1.003750pt}%
\definecolor{currentstroke}{rgb}{0.000000,0.000000,0.000000}%
\pgfsetstrokecolor{currentstroke}%
\pgfsetdash{}{0pt}%
\pgfpathmoveto{\pgfqpoint{1.725000in}{0.833333in}}%
\pgfpathlineto{\pgfqpoint{1.194340in}{0.033333in}}%
\pgfusepath{stroke}%
\end{pgfscope}%
\begin{pgfscope}%
\definecolor{textcolor}{rgb}{0.000000,0.000000,0.000000}%
\pgfsetstrokecolor{textcolor}%
\pgfsetfillcolor{textcolor}%
\pgftext[x=0.183333in,y=4.833333in,left,base]{\color{textcolor}\rmfamily\fontsize{9.000000}{10.800000}\selectfont \(\displaystyle \beta < \frac{1}{n}\)}%
\end{pgfscope}%
\begin{pgfscope}%
\definecolor{textcolor}{rgb}{0.000000,0.000000,0.000000}%
\pgfsetstrokecolor{textcolor}%
\pgfsetfillcolor{textcolor}%
\pgftext[x=0.183333in,y=3.166667in,left,base]{\color{textcolor}\rmfamily\fontsize{9.000000}{10.800000}\selectfont \(\displaystyle \beta = \frac{1}{n}\)}%
\end{pgfscope}%
\begin{pgfscope}%
\definecolor{textcolor}{rgb}{0.000000,0.000000,0.000000}%
\pgfsetstrokecolor{textcolor}%
\pgfsetfillcolor{textcolor}%
\pgftext[x=0.183333in,y=1.500000in,left,base]{\color{textcolor}\rmfamily\fontsize{9.000000}{10.800000}\selectfont \(\displaystyle \beta > \frac{1}{n}\)}%
\end{pgfscope}%
\end{pgfpicture}%
\makeatother%
\endgroup%

  \caption{Sketch of the \Cheren\ effect. When a particle moves at a speed that
  exceeds the speed of the light in the medium, the radiation adds up coherently
  to generate a wavefront.}
  \label{fig:cherenkov_effect}
\end{marginfigure}

\Cheren\ radiation is emitted when a charged particle moves in a medium at a speed
greater than the speed of light \emph{in that medium}
\begin{align}
  \beta > \frac{1}{n}.
\end{align}
(Here $\beta$ refers to the incident particle and $n$ is the index of
refraction of the material.) In the ideal case of a non-dispersive medium, the
Cherencov wave front form an acute angle with respect to the particle velocity
given by
\begin{align}
  \cos\theta_c = \frac{1}{n\beta}
\end{align}
as sketched in figure~\ref{fig:cherenkov_effect}.

The (double differential) spectrum of the photons produced per unit path length
and wavelength is given by
\begin{align}\label{eq:cherenkov_loss}
  \frac{d^2N}{dxd\lambda} = \frac{2\pi\alpha z^2}{\lambda^2}
  \left(1 -  \frac{1}{\beta^2 n^2(\lambda)}\right) =
  \frac{2\pi\alpha z^2}{\lambda^2}
  \left(1 - \frac{1 + \beta^2\gamma^2}{\beta^2\gamma^2n^2(\lambda)}\right),
\end{align}
where $z$ is the charge of the projectile, and the index of refraction $n$
is evaluated at the generic photon wavelength $\lambda$.
Equation~\eqref{eq:cherenkov_loss} cannot be readily translated into a detector
signal as, in practice, one has to convolve it with the response of the
transducer and integrate over the photon wavelengths of interest.
