\chapter{Massive Neutrinos}

The electron neutrino was was first hypothesized, although under a different name,
by W. Pauli in 1930 in order to save the energy conservation law in $\beta$ decay,
incorporated by Fermi a few years later into its effective theory of the process,
and detected experimentally\sidenote{To be precise, that was the antineutrino.} in 1956.
As the two hevier leptons (first the $\mu$ and then the $\tau$) were discovered, it
was postulated that they would come with their fellow neutrinos, and in fact $\nu_\mu$
was observed in 1962, and $\nu_\tau$ in 2000. We now have a fairly precise understading
of the three generations (or \emph{flavors}) of neutrino, their properties and
mixing, although a number of open problems is still object of active investigation.

In this section we shall use a hybrid approach, where we shall partially follow
the historical development of our understanding of the neutrinos, staring from the
saga of the solar neutrinos, which has its own special place in the history of science
as an endevour undertaken by a relatively small number of key people when that particular
subject of study was not fashionable and the very idea that the Sun could provide
quantitative information into the basic costituents of the matter was met with skepticism.
We shall interleave the history of the solar neutrino problem with the basic treatment
of two-flavor neutrino oscillation and neutrino propagation in matter (also know as
MSW effect), cover the atmospheric neutrinos, and the try and put everything into the
big picture of our current understading of three-flavor mixing, with a final outlook
to future developments and open problems.

% Interesting review on all neutrino measurements by Lino Miramonti (Milan)
% https://www0.mi.infn.it/~miramont/menu/attivita_didattica/Fisica_delle_Particelle_4_Astroparticelle/lec_oscillations.pdf
% and also
% https://wwwth.mpp.mpg.de/members/raffelt/Vorlesungen/WS2009/notes/notes13.pdf


\section{Solar neutrinos}

%https://www.fe.infn.it/radioactivity/fiorentini/materials/courses/astronuc_en/5.%20Nuclear%20reactions%20in%20the%20sun.pdf

Solar neutrinos, produced in the core of the Sun via various nuclear fusion process,
make up for the vast majority of the neutrinos reaching the Earth and, since, unlike
photons, they can largely travel unperturned through the Sun, they provide crucial
information about the inner functioning of our star---as well as about the neutrinos
themselves, as we shall see in a second. The saga of the \emph{solar neutrino problem}
is a long and torned one, and one where the detailed study of a mundane astrophysical
object such as the Sun proved to be carrier of a breakthrough related to the fundamental
constituents of matter, developing at the same time what is known as the standard
solar model~\cite{2003nipb.conf...15B}.

If we accept that the main mechanism generating energy in the Sun is the fusion
of $H$ into $He$---which is in agreement with the state of the art of our
understanding---then, by first principles, we have to assume that the basic reaction
by which the Sun shines is the conversion of \emph{four} protons into an $\alpha$
particle
\begin{align*}
  4p + 2 e^- \rightarrow \ce{^4He} + 2\nu_e + Q_\text{heat}
\end{align*}
which, in virtue of the conservation of the leptonic number, proceeds with the
creation of exactly two neutrinos for each reaction\sidenote{What we are doing here is
essentially to transform two out of four protons in neutrons, which is the inverse
of the neutron beta decay
\begin{align*}
  n \rightarrow p + e^- + \overline{\nu}_e
\end{align*}
and is bound to produce two neutrinos.}.
The difference between the sum of the masses in the initial state and that in the
final state is in this case $Q = 26.73$~MeV, and we can rephrase this by saying
that that we have an energy release of $26.73$~MeV associated to each pair of
neutrino being produced. These two simple facts bears important consequences,
because the \emph{total} neutrino flux at Earth can be related to the solar
constant\sidenote{In reality the situation is more complicted, as the proton fusion
can proceed via several different avenues, and a precise estimate requires an accurate
modeling of the Sun interior, but it is amusing to know that this simple calculation
is correct to the 10\% level.} (that is, the electromagnetic energy flux per unit
area at a distance $d = 1$~au from the Sun) by
\begin{align}
  \Phi_\nu = \frac{2 L_\odot}{4\pi d^2 Q_\text{heat}}.
\end{align}
If we now assume that the majority of the energy from the nuclear fusion goes into
heat, i.e., $Q_\text{heat} \sim Q$ we arrive to the astonishing figure of
$\Phi_\nu \approx 6.5 \times 10^{10}$~cm$^{-2}$~s$^{-1}$. In any given second, any
1~cm$^2$ surface orthogonal to the direction to the Sun is traversed by some
65~billion neutrinos. In the remaining of this section we shall elaborate on this
basic facts, and the reader is referred to~\cite{2021ARNPS..71..491O} for a more
in-depth, forward-looking review of neutrino production in the Sun.


\subsection{Neutrino production in nuclear fusion}

The archetypal neutrino production process in the Sun is the fusion of two protons
into a deuton (a stable hydrogen isotope consisting into a bound state of a
proton and a neutron), which takes place via the weak interaction, with the production
of a positron and a neutrino
\begin{align}\label{eq:deuton_fusion}
  p + p \rightarrow \ce{^2H} + e^+ + \nu_e + 0.423~\text{MeV}.
\end{align}

\begin{figure}[!htbp]
  %% Creator: Matplotlib, PGF backend
%%
%% To include the figure in your LaTeX document, write
%%   \input{<filename>.pgf}
%%
%% Make sure the required packages are loaded in your preamble
%%   \usepackage{pgf}
%%
%% Also ensure that all the required font packages are loaded; for instance,
%% the lmodern package is sometimes necessary when using math font.
%%   \usepackage{lmodern}
%%
%% Figures using additional raster images can only be included by \input if
%% they are in the same directory as the main LaTeX file. For loading figures
%% from other directories you can use the `import` package
%%   \usepackage{import}
%%
%% and then include the figures with
%%   \import{<path to file>}{<filename>.pgf}
%%
%% Matplotlib used the following preamble
%%   \usepackage{fontspec}
%%   \setmainfont{DejaVuSerif.ttf}[Path=\detokenize{/usr/share/matplotlib/mpl-data/fonts/ttf/}]
%%   \setsansfont{DejaVuSans.ttf}[Path=\detokenize{/usr/share/matplotlib/mpl-data/fonts/ttf/}]
%%   \setmonofont{DejaVuSansMono.ttf}[Path=\detokenize{/usr/share/matplotlib/mpl-data/fonts/ttf/}]
%%
\begingroup%
\makeatletter%
\begin{pgfpicture}%
\pgfpathrectangle{\pgfpointorigin}{\pgfqpoint{4.150000in}{3.500000in}}%
\pgfusepath{use as bounding box, clip}%
\begin{pgfscope}%
\pgfsetbuttcap%
\pgfsetmiterjoin%
\definecolor{currentfill}{rgb}{1.000000,1.000000,1.000000}%
\pgfsetfillcolor{currentfill}%
\pgfsetlinewidth{0.000000pt}%
\definecolor{currentstroke}{rgb}{1.000000,1.000000,1.000000}%
\pgfsetstrokecolor{currentstroke}%
\pgfsetdash{}{0pt}%
\pgfpathmoveto{\pgfqpoint{0.000000in}{0.000000in}}%
\pgfpathlineto{\pgfqpoint{4.150000in}{0.000000in}}%
\pgfpathlineto{\pgfqpoint{4.150000in}{3.500000in}}%
\pgfpathlineto{\pgfqpoint{0.000000in}{3.500000in}}%
\pgfpathlineto{\pgfqpoint{0.000000in}{0.000000in}}%
\pgfpathclose%
\pgfusepath{fill}%
\end{pgfscope}%
\begin{pgfscope}%
\pgfsetbuttcap%
\pgfsetmiterjoin%
\definecolor{currentfill}{rgb}{1.000000,1.000000,1.000000}%
\pgfsetfillcolor{currentfill}%
\pgfsetlinewidth{0.000000pt}%
\definecolor{currentstroke}{rgb}{0.000000,0.000000,0.000000}%
\pgfsetstrokecolor{currentstroke}%
\pgfsetstrokeopacity{0.000000}%
\pgfsetdash{}{0pt}%
\pgfpathmoveto{\pgfqpoint{0.726250in}{0.525000in}}%
\pgfpathlineto{\pgfqpoint{4.046250in}{0.525000in}}%
\pgfpathlineto{\pgfqpoint{4.046250in}{3.412500in}}%
\pgfpathlineto{\pgfqpoint{0.726250in}{3.412500in}}%
\pgfpathlineto{\pgfqpoint{0.726250in}{0.525000in}}%
\pgfpathclose%
\pgfusepath{fill}%
\end{pgfscope}%
\begin{pgfscope}%
\pgfpathrectangle{\pgfqpoint{0.726250in}{0.525000in}}{\pgfqpoint{3.320000in}{2.887500in}}%
\pgfusepath{clip}%
\pgfsetbuttcap%
\pgfsetroundjoin%
\pgfsetlinewidth{0.803000pt}%
\definecolor{currentstroke}{rgb}{0.752941,0.752941,0.752941}%
\pgfsetstrokecolor{currentstroke}%
\pgfsetdash{{2.960000pt}{1.280000pt}}{0.000000pt}%
\pgfpathmoveto{\pgfqpoint{0.726250in}{0.525000in}}%
\pgfpathlineto{\pgfqpoint{0.726250in}{3.412500in}}%
\pgfusepath{stroke}%
\end{pgfscope}%
\begin{pgfscope}%
\pgfsetbuttcap%
\pgfsetroundjoin%
\definecolor{currentfill}{rgb}{0.000000,0.000000,0.000000}%
\pgfsetfillcolor{currentfill}%
\pgfsetlinewidth{0.803000pt}%
\definecolor{currentstroke}{rgb}{0.000000,0.000000,0.000000}%
\pgfsetstrokecolor{currentstroke}%
\pgfsetdash{}{0pt}%
\pgfsys@defobject{currentmarker}{\pgfqpoint{0.000000in}{-0.048611in}}{\pgfqpoint{0.000000in}{0.000000in}}{%
\pgfpathmoveto{\pgfqpoint{0.000000in}{0.000000in}}%
\pgfpathlineto{\pgfqpoint{0.000000in}{-0.048611in}}%
\pgfusepath{stroke,fill}%
}%
\begin{pgfscope}%
\pgfsys@transformshift{0.726250in}{0.525000in}%
\pgfsys@useobject{currentmarker}{}%
\end{pgfscope}%
\end{pgfscope}%
\begin{pgfscope}%
\definecolor{textcolor}{rgb}{0.000000,0.000000,0.000000}%
\pgfsetstrokecolor{textcolor}%
\pgfsetfillcolor{textcolor}%
\pgftext[x=0.726250in,y=0.427778in,,top]{\color{textcolor}\rmfamily\fontsize{9.000000}{10.800000}\selectfont \(\displaystyle {10^{0}}\)}%
\end{pgfscope}%
\begin{pgfscope}%
\pgfpathrectangle{\pgfqpoint{0.726250in}{0.525000in}}{\pgfqpoint{3.320000in}{2.887500in}}%
\pgfusepath{clip}%
\pgfsetbuttcap%
\pgfsetroundjoin%
\pgfsetlinewidth{0.803000pt}%
\definecolor{currentstroke}{rgb}{0.752941,0.752941,0.752941}%
\pgfsetstrokecolor{currentstroke}%
\pgfsetdash{{2.960000pt}{1.280000pt}}{0.000000pt}%
\pgfpathmoveto{\pgfqpoint{2.123217in}{0.525000in}}%
\pgfpathlineto{\pgfqpoint{2.123217in}{3.412500in}}%
\pgfusepath{stroke}%
\end{pgfscope}%
\begin{pgfscope}%
\pgfsetbuttcap%
\pgfsetroundjoin%
\definecolor{currentfill}{rgb}{0.000000,0.000000,0.000000}%
\pgfsetfillcolor{currentfill}%
\pgfsetlinewidth{0.803000pt}%
\definecolor{currentstroke}{rgb}{0.000000,0.000000,0.000000}%
\pgfsetstrokecolor{currentstroke}%
\pgfsetdash{}{0pt}%
\pgfsys@defobject{currentmarker}{\pgfqpoint{0.000000in}{-0.048611in}}{\pgfqpoint{0.000000in}{0.000000in}}{%
\pgfpathmoveto{\pgfqpoint{0.000000in}{0.000000in}}%
\pgfpathlineto{\pgfqpoint{0.000000in}{-0.048611in}}%
\pgfusepath{stroke,fill}%
}%
\begin{pgfscope}%
\pgfsys@transformshift{2.123217in}{0.525000in}%
\pgfsys@useobject{currentmarker}{}%
\end{pgfscope}%
\end{pgfscope}%
\begin{pgfscope}%
\definecolor{textcolor}{rgb}{0.000000,0.000000,0.000000}%
\pgfsetstrokecolor{textcolor}%
\pgfsetfillcolor{textcolor}%
\pgftext[x=2.123217in,y=0.427778in,,top]{\color{textcolor}\rmfamily\fontsize{9.000000}{10.800000}\selectfont \(\displaystyle {10^{1}}\)}%
\end{pgfscope}%
\begin{pgfscope}%
\pgfpathrectangle{\pgfqpoint{0.726250in}{0.525000in}}{\pgfqpoint{3.320000in}{2.887500in}}%
\pgfusepath{clip}%
\pgfsetbuttcap%
\pgfsetroundjoin%
\pgfsetlinewidth{0.803000pt}%
\definecolor{currentstroke}{rgb}{0.752941,0.752941,0.752941}%
\pgfsetstrokecolor{currentstroke}%
\pgfsetdash{{2.960000pt}{1.280000pt}}{0.000000pt}%
\pgfpathmoveto{\pgfqpoint{3.520184in}{0.525000in}}%
\pgfpathlineto{\pgfqpoint{3.520184in}{3.412500in}}%
\pgfusepath{stroke}%
\end{pgfscope}%
\begin{pgfscope}%
\pgfsetbuttcap%
\pgfsetroundjoin%
\definecolor{currentfill}{rgb}{0.000000,0.000000,0.000000}%
\pgfsetfillcolor{currentfill}%
\pgfsetlinewidth{0.803000pt}%
\definecolor{currentstroke}{rgb}{0.000000,0.000000,0.000000}%
\pgfsetstrokecolor{currentstroke}%
\pgfsetdash{}{0pt}%
\pgfsys@defobject{currentmarker}{\pgfqpoint{0.000000in}{-0.048611in}}{\pgfqpoint{0.000000in}{0.000000in}}{%
\pgfpathmoveto{\pgfqpoint{0.000000in}{0.000000in}}%
\pgfpathlineto{\pgfqpoint{0.000000in}{-0.048611in}}%
\pgfusepath{stroke,fill}%
}%
\begin{pgfscope}%
\pgfsys@transformshift{3.520184in}{0.525000in}%
\pgfsys@useobject{currentmarker}{}%
\end{pgfscope}%
\end{pgfscope}%
\begin{pgfscope}%
\definecolor{textcolor}{rgb}{0.000000,0.000000,0.000000}%
\pgfsetstrokecolor{textcolor}%
\pgfsetfillcolor{textcolor}%
\pgftext[x=3.520184in,y=0.427778in,,top]{\color{textcolor}\rmfamily\fontsize{9.000000}{10.800000}\selectfont \(\displaystyle {10^{2}}\)}%
\end{pgfscope}%
\begin{pgfscope}%
\pgfpathrectangle{\pgfqpoint{0.726250in}{0.525000in}}{\pgfqpoint{3.320000in}{2.887500in}}%
\pgfusepath{clip}%
\pgfsetbuttcap%
\pgfsetroundjoin%
\pgfsetlinewidth{0.803000pt}%
\definecolor{currentstroke}{rgb}{0.752941,0.752941,0.752941}%
\pgfsetstrokecolor{currentstroke}%
\pgfsetdash{{2.960000pt}{1.280000pt}}{0.000000pt}%
\pgfpathmoveto{\pgfqpoint{1.146779in}{0.525000in}}%
\pgfpathlineto{\pgfqpoint{1.146779in}{3.412500in}}%
\pgfusepath{stroke}%
\end{pgfscope}%
\begin{pgfscope}%
\pgfsetbuttcap%
\pgfsetroundjoin%
\definecolor{currentfill}{rgb}{0.000000,0.000000,0.000000}%
\pgfsetfillcolor{currentfill}%
\pgfsetlinewidth{0.602250pt}%
\definecolor{currentstroke}{rgb}{0.000000,0.000000,0.000000}%
\pgfsetstrokecolor{currentstroke}%
\pgfsetdash{}{0pt}%
\pgfsys@defobject{currentmarker}{\pgfqpoint{0.000000in}{-0.027778in}}{\pgfqpoint{0.000000in}{0.000000in}}{%
\pgfpathmoveto{\pgfqpoint{0.000000in}{0.000000in}}%
\pgfpathlineto{\pgfqpoint{0.000000in}{-0.027778in}}%
\pgfusepath{stroke,fill}%
}%
\begin{pgfscope}%
\pgfsys@transformshift{1.146779in}{0.525000in}%
\pgfsys@useobject{currentmarker}{}%
\end{pgfscope}%
\end{pgfscope}%
\begin{pgfscope}%
\pgfpathrectangle{\pgfqpoint{0.726250in}{0.525000in}}{\pgfqpoint{3.320000in}{2.887500in}}%
\pgfusepath{clip}%
\pgfsetbuttcap%
\pgfsetroundjoin%
\pgfsetlinewidth{0.803000pt}%
\definecolor{currentstroke}{rgb}{0.752941,0.752941,0.752941}%
\pgfsetstrokecolor{currentstroke}%
\pgfsetdash{{2.960000pt}{1.280000pt}}{0.000000pt}%
\pgfpathmoveto{\pgfqpoint{1.392773in}{0.525000in}}%
\pgfpathlineto{\pgfqpoint{1.392773in}{3.412500in}}%
\pgfusepath{stroke}%
\end{pgfscope}%
\begin{pgfscope}%
\pgfsetbuttcap%
\pgfsetroundjoin%
\definecolor{currentfill}{rgb}{0.000000,0.000000,0.000000}%
\pgfsetfillcolor{currentfill}%
\pgfsetlinewidth{0.602250pt}%
\definecolor{currentstroke}{rgb}{0.000000,0.000000,0.000000}%
\pgfsetstrokecolor{currentstroke}%
\pgfsetdash{}{0pt}%
\pgfsys@defobject{currentmarker}{\pgfqpoint{0.000000in}{-0.027778in}}{\pgfqpoint{0.000000in}{0.000000in}}{%
\pgfpathmoveto{\pgfqpoint{0.000000in}{0.000000in}}%
\pgfpathlineto{\pgfqpoint{0.000000in}{-0.027778in}}%
\pgfusepath{stroke,fill}%
}%
\begin{pgfscope}%
\pgfsys@transformshift{1.392773in}{0.525000in}%
\pgfsys@useobject{currentmarker}{}%
\end{pgfscope}%
\end{pgfscope}%
\begin{pgfscope}%
\pgfpathrectangle{\pgfqpoint{0.726250in}{0.525000in}}{\pgfqpoint{3.320000in}{2.887500in}}%
\pgfusepath{clip}%
\pgfsetbuttcap%
\pgfsetroundjoin%
\pgfsetlinewidth{0.803000pt}%
\definecolor{currentstroke}{rgb}{0.752941,0.752941,0.752941}%
\pgfsetstrokecolor{currentstroke}%
\pgfsetdash{{2.960000pt}{1.280000pt}}{0.000000pt}%
\pgfpathmoveto{\pgfqpoint{1.567308in}{0.525000in}}%
\pgfpathlineto{\pgfqpoint{1.567308in}{3.412500in}}%
\pgfusepath{stroke}%
\end{pgfscope}%
\begin{pgfscope}%
\pgfsetbuttcap%
\pgfsetroundjoin%
\definecolor{currentfill}{rgb}{0.000000,0.000000,0.000000}%
\pgfsetfillcolor{currentfill}%
\pgfsetlinewidth{0.602250pt}%
\definecolor{currentstroke}{rgb}{0.000000,0.000000,0.000000}%
\pgfsetstrokecolor{currentstroke}%
\pgfsetdash{}{0pt}%
\pgfsys@defobject{currentmarker}{\pgfqpoint{0.000000in}{-0.027778in}}{\pgfqpoint{0.000000in}{0.000000in}}{%
\pgfpathmoveto{\pgfqpoint{0.000000in}{0.000000in}}%
\pgfpathlineto{\pgfqpoint{0.000000in}{-0.027778in}}%
\pgfusepath{stroke,fill}%
}%
\begin{pgfscope}%
\pgfsys@transformshift{1.567308in}{0.525000in}%
\pgfsys@useobject{currentmarker}{}%
\end{pgfscope}%
\end{pgfscope}%
\begin{pgfscope}%
\pgfpathrectangle{\pgfqpoint{0.726250in}{0.525000in}}{\pgfqpoint{3.320000in}{2.887500in}}%
\pgfusepath{clip}%
\pgfsetbuttcap%
\pgfsetroundjoin%
\pgfsetlinewidth{0.803000pt}%
\definecolor{currentstroke}{rgb}{0.752941,0.752941,0.752941}%
\pgfsetstrokecolor{currentstroke}%
\pgfsetdash{{2.960000pt}{1.280000pt}}{0.000000pt}%
\pgfpathmoveto{\pgfqpoint{1.702688in}{0.525000in}}%
\pgfpathlineto{\pgfqpoint{1.702688in}{3.412500in}}%
\pgfusepath{stroke}%
\end{pgfscope}%
\begin{pgfscope}%
\pgfsetbuttcap%
\pgfsetroundjoin%
\definecolor{currentfill}{rgb}{0.000000,0.000000,0.000000}%
\pgfsetfillcolor{currentfill}%
\pgfsetlinewidth{0.602250pt}%
\definecolor{currentstroke}{rgb}{0.000000,0.000000,0.000000}%
\pgfsetstrokecolor{currentstroke}%
\pgfsetdash{}{0pt}%
\pgfsys@defobject{currentmarker}{\pgfqpoint{0.000000in}{-0.027778in}}{\pgfqpoint{0.000000in}{0.000000in}}{%
\pgfpathmoveto{\pgfqpoint{0.000000in}{0.000000in}}%
\pgfpathlineto{\pgfqpoint{0.000000in}{-0.027778in}}%
\pgfusepath{stroke,fill}%
}%
\begin{pgfscope}%
\pgfsys@transformshift{1.702688in}{0.525000in}%
\pgfsys@useobject{currentmarker}{}%
\end{pgfscope}%
\end{pgfscope}%
\begin{pgfscope}%
\pgfpathrectangle{\pgfqpoint{0.726250in}{0.525000in}}{\pgfqpoint{3.320000in}{2.887500in}}%
\pgfusepath{clip}%
\pgfsetbuttcap%
\pgfsetroundjoin%
\pgfsetlinewidth{0.803000pt}%
\definecolor{currentstroke}{rgb}{0.752941,0.752941,0.752941}%
\pgfsetstrokecolor{currentstroke}%
\pgfsetdash{{2.960000pt}{1.280000pt}}{0.000000pt}%
\pgfpathmoveto{\pgfqpoint{1.813302in}{0.525000in}}%
\pgfpathlineto{\pgfqpoint{1.813302in}{3.412500in}}%
\pgfusepath{stroke}%
\end{pgfscope}%
\begin{pgfscope}%
\pgfsetbuttcap%
\pgfsetroundjoin%
\definecolor{currentfill}{rgb}{0.000000,0.000000,0.000000}%
\pgfsetfillcolor{currentfill}%
\pgfsetlinewidth{0.602250pt}%
\definecolor{currentstroke}{rgb}{0.000000,0.000000,0.000000}%
\pgfsetstrokecolor{currentstroke}%
\pgfsetdash{}{0pt}%
\pgfsys@defobject{currentmarker}{\pgfqpoint{0.000000in}{-0.027778in}}{\pgfqpoint{0.000000in}{0.000000in}}{%
\pgfpathmoveto{\pgfqpoint{0.000000in}{0.000000in}}%
\pgfpathlineto{\pgfqpoint{0.000000in}{-0.027778in}}%
\pgfusepath{stroke,fill}%
}%
\begin{pgfscope}%
\pgfsys@transformshift{1.813302in}{0.525000in}%
\pgfsys@useobject{currentmarker}{}%
\end{pgfscope}%
\end{pgfscope}%
\begin{pgfscope}%
\pgfpathrectangle{\pgfqpoint{0.726250in}{0.525000in}}{\pgfqpoint{3.320000in}{2.887500in}}%
\pgfusepath{clip}%
\pgfsetbuttcap%
\pgfsetroundjoin%
\pgfsetlinewidth{0.803000pt}%
\definecolor{currentstroke}{rgb}{0.752941,0.752941,0.752941}%
\pgfsetstrokecolor{currentstroke}%
\pgfsetdash{{2.960000pt}{1.280000pt}}{0.000000pt}%
\pgfpathmoveto{\pgfqpoint{1.906824in}{0.525000in}}%
\pgfpathlineto{\pgfqpoint{1.906824in}{3.412500in}}%
\pgfusepath{stroke}%
\end{pgfscope}%
\begin{pgfscope}%
\pgfsetbuttcap%
\pgfsetroundjoin%
\definecolor{currentfill}{rgb}{0.000000,0.000000,0.000000}%
\pgfsetfillcolor{currentfill}%
\pgfsetlinewidth{0.602250pt}%
\definecolor{currentstroke}{rgb}{0.000000,0.000000,0.000000}%
\pgfsetstrokecolor{currentstroke}%
\pgfsetdash{}{0pt}%
\pgfsys@defobject{currentmarker}{\pgfqpoint{0.000000in}{-0.027778in}}{\pgfqpoint{0.000000in}{0.000000in}}{%
\pgfpathmoveto{\pgfqpoint{0.000000in}{0.000000in}}%
\pgfpathlineto{\pgfqpoint{0.000000in}{-0.027778in}}%
\pgfusepath{stroke,fill}%
}%
\begin{pgfscope}%
\pgfsys@transformshift{1.906824in}{0.525000in}%
\pgfsys@useobject{currentmarker}{}%
\end{pgfscope}%
\end{pgfscope}%
\begin{pgfscope}%
\pgfpathrectangle{\pgfqpoint{0.726250in}{0.525000in}}{\pgfqpoint{3.320000in}{2.887500in}}%
\pgfusepath{clip}%
\pgfsetbuttcap%
\pgfsetroundjoin%
\pgfsetlinewidth{0.803000pt}%
\definecolor{currentstroke}{rgb}{0.752941,0.752941,0.752941}%
\pgfsetstrokecolor{currentstroke}%
\pgfsetdash{{2.960000pt}{1.280000pt}}{0.000000pt}%
\pgfpathmoveto{\pgfqpoint{1.987837in}{0.525000in}}%
\pgfpathlineto{\pgfqpoint{1.987837in}{3.412500in}}%
\pgfusepath{stroke}%
\end{pgfscope}%
\begin{pgfscope}%
\pgfsetbuttcap%
\pgfsetroundjoin%
\definecolor{currentfill}{rgb}{0.000000,0.000000,0.000000}%
\pgfsetfillcolor{currentfill}%
\pgfsetlinewidth{0.602250pt}%
\definecolor{currentstroke}{rgb}{0.000000,0.000000,0.000000}%
\pgfsetstrokecolor{currentstroke}%
\pgfsetdash{}{0pt}%
\pgfsys@defobject{currentmarker}{\pgfqpoint{0.000000in}{-0.027778in}}{\pgfqpoint{0.000000in}{0.000000in}}{%
\pgfpathmoveto{\pgfqpoint{0.000000in}{0.000000in}}%
\pgfpathlineto{\pgfqpoint{0.000000in}{-0.027778in}}%
\pgfusepath{stroke,fill}%
}%
\begin{pgfscope}%
\pgfsys@transformshift{1.987837in}{0.525000in}%
\pgfsys@useobject{currentmarker}{}%
\end{pgfscope}%
\end{pgfscope}%
\begin{pgfscope}%
\pgfpathrectangle{\pgfqpoint{0.726250in}{0.525000in}}{\pgfqpoint{3.320000in}{2.887500in}}%
\pgfusepath{clip}%
\pgfsetbuttcap%
\pgfsetroundjoin%
\pgfsetlinewidth{0.803000pt}%
\definecolor{currentstroke}{rgb}{0.752941,0.752941,0.752941}%
\pgfsetstrokecolor{currentstroke}%
\pgfsetdash{{2.960000pt}{1.280000pt}}{0.000000pt}%
\pgfpathmoveto{\pgfqpoint{2.059295in}{0.525000in}}%
\pgfpathlineto{\pgfqpoint{2.059295in}{3.412500in}}%
\pgfusepath{stroke}%
\end{pgfscope}%
\begin{pgfscope}%
\pgfsetbuttcap%
\pgfsetroundjoin%
\definecolor{currentfill}{rgb}{0.000000,0.000000,0.000000}%
\pgfsetfillcolor{currentfill}%
\pgfsetlinewidth{0.602250pt}%
\definecolor{currentstroke}{rgb}{0.000000,0.000000,0.000000}%
\pgfsetstrokecolor{currentstroke}%
\pgfsetdash{}{0pt}%
\pgfsys@defobject{currentmarker}{\pgfqpoint{0.000000in}{-0.027778in}}{\pgfqpoint{0.000000in}{0.000000in}}{%
\pgfpathmoveto{\pgfqpoint{0.000000in}{0.000000in}}%
\pgfpathlineto{\pgfqpoint{0.000000in}{-0.027778in}}%
\pgfusepath{stroke,fill}%
}%
\begin{pgfscope}%
\pgfsys@transformshift{2.059295in}{0.525000in}%
\pgfsys@useobject{currentmarker}{}%
\end{pgfscope}%
\end{pgfscope}%
\begin{pgfscope}%
\pgfpathrectangle{\pgfqpoint{0.726250in}{0.525000in}}{\pgfqpoint{3.320000in}{2.887500in}}%
\pgfusepath{clip}%
\pgfsetbuttcap%
\pgfsetroundjoin%
\pgfsetlinewidth{0.803000pt}%
\definecolor{currentstroke}{rgb}{0.752941,0.752941,0.752941}%
\pgfsetstrokecolor{currentstroke}%
\pgfsetdash{{2.960000pt}{1.280000pt}}{0.000000pt}%
\pgfpathmoveto{\pgfqpoint{2.543746in}{0.525000in}}%
\pgfpathlineto{\pgfqpoint{2.543746in}{3.412500in}}%
\pgfusepath{stroke}%
\end{pgfscope}%
\begin{pgfscope}%
\pgfsetbuttcap%
\pgfsetroundjoin%
\definecolor{currentfill}{rgb}{0.000000,0.000000,0.000000}%
\pgfsetfillcolor{currentfill}%
\pgfsetlinewidth{0.602250pt}%
\definecolor{currentstroke}{rgb}{0.000000,0.000000,0.000000}%
\pgfsetstrokecolor{currentstroke}%
\pgfsetdash{}{0pt}%
\pgfsys@defobject{currentmarker}{\pgfqpoint{0.000000in}{-0.027778in}}{\pgfqpoint{0.000000in}{0.000000in}}{%
\pgfpathmoveto{\pgfqpoint{0.000000in}{0.000000in}}%
\pgfpathlineto{\pgfqpoint{0.000000in}{-0.027778in}}%
\pgfusepath{stroke,fill}%
}%
\begin{pgfscope}%
\pgfsys@transformshift{2.543746in}{0.525000in}%
\pgfsys@useobject{currentmarker}{}%
\end{pgfscope}%
\end{pgfscope}%
\begin{pgfscope}%
\pgfpathrectangle{\pgfqpoint{0.726250in}{0.525000in}}{\pgfqpoint{3.320000in}{2.887500in}}%
\pgfusepath{clip}%
\pgfsetbuttcap%
\pgfsetroundjoin%
\pgfsetlinewidth{0.803000pt}%
\definecolor{currentstroke}{rgb}{0.752941,0.752941,0.752941}%
\pgfsetstrokecolor{currentstroke}%
\pgfsetdash{{2.960000pt}{1.280000pt}}{0.000000pt}%
\pgfpathmoveto{\pgfqpoint{2.789740in}{0.525000in}}%
\pgfpathlineto{\pgfqpoint{2.789740in}{3.412500in}}%
\pgfusepath{stroke}%
\end{pgfscope}%
\begin{pgfscope}%
\pgfsetbuttcap%
\pgfsetroundjoin%
\definecolor{currentfill}{rgb}{0.000000,0.000000,0.000000}%
\pgfsetfillcolor{currentfill}%
\pgfsetlinewidth{0.602250pt}%
\definecolor{currentstroke}{rgb}{0.000000,0.000000,0.000000}%
\pgfsetstrokecolor{currentstroke}%
\pgfsetdash{}{0pt}%
\pgfsys@defobject{currentmarker}{\pgfqpoint{0.000000in}{-0.027778in}}{\pgfqpoint{0.000000in}{0.000000in}}{%
\pgfpathmoveto{\pgfqpoint{0.000000in}{0.000000in}}%
\pgfpathlineto{\pgfqpoint{0.000000in}{-0.027778in}}%
\pgfusepath{stroke,fill}%
}%
\begin{pgfscope}%
\pgfsys@transformshift{2.789740in}{0.525000in}%
\pgfsys@useobject{currentmarker}{}%
\end{pgfscope}%
\end{pgfscope}%
\begin{pgfscope}%
\pgfpathrectangle{\pgfqpoint{0.726250in}{0.525000in}}{\pgfqpoint{3.320000in}{2.887500in}}%
\pgfusepath{clip}%
\pgfsetbuttcap%
\pgfsetroundjoin%
\pgfsetlinewidth{0.803000pt}%
\definecolor{currentstroke}{rgb}{0.752941,0.752941,0.752941}%
\pgfsetstrokecolor{currentstroke}%
\pgfsetdash{{2.960000pt}{1.280000pt}}{0.000000pt}%
\pgfpathmoveto{\pgfqpoint{2.964275in}{0.525000in}}%
\pgfpathlineto{\pgfqpoint{2.964275in}{3.412500in}}%
\pgfusepath{stroke}%
\end{pgfscope}%
\begin{pgfscope}%
\pgfsetbuttcap%
\pgfsetroundjoin%
\definecolor{currentfill}{rgb}{0.000000,0.000000,0.000000}%
\pgfsetfillcolor{currentfill}%
\pgfsetlinewidth{0.602250pt}%
\definecolor{currentstroke}{rgb}{0.000000,0.000000,0.000000}%
\pgfsetstrokecolor{currentstroke}%
\pgfsetdash{}{0pt}%
\pgfsys@defobject{currentmarker}{\pgfqpoint{0.000000in}{-0.027778in}}{\pgfqpoint{0.000000in}{0.000000in}}{%
\pgfpathmoveto{\pgfqpoint{0.000000in}{0.000000in}}%
\pgfpathlineto{\pgfqpoint{0.000000in}{-0.027778in}}%
\pgfusepath{stroke,fill}%
}%
\begin{pgfscope}%
\pgfsys@transformshift{2.964275in}{0.525000in}%
\pgfsys@useobject{currentmarker}{}%
\end{pgfscope}%
\end{pgfscope}%
\begin{pgfscope}%
\pgfpathrectangle{\pgfqpoint{0.726250in}{0.525000in}}{\pgfqpoint{3.320000in}{2.887500in}}%
\pgfusepath{clip}%
\pgfsetbuttcap%
\pgfsetroundjoin%
\pgfsetlinewidth{0.803000pt}%
\definecolor{currentstroke}{rgb}{0.752941,0.752941,0.752941}%
\pgfsetstrokecolor{currentstroke}%
\pgfsetdash{{2.960000pt}{1.280000pt}}{0.000000pt}%
\pgfpathmoveto{\pgfqpoint{3.099655in}{0.525000in}}%
\pgfpathlineto{\pgfqpoint{3.099655in}{3.412500in}}%
\pgfusepath{stroke}%
\end{pgfscope}%
\begin{pgfscope}%
\pgfsetbuttcap%
\pgfsetroundjoin%
\definecolor{currentfill}{rgb}{0.000000,0.000000,0.000000}%
\pgfsetfillcolor{currentfill}%
\pgfsetlinewidth{0.602250pt}%
\definecolor{currentstroke}{rgb}{0.000000,0.000000,0.000000}%
\pgfsetstrokecolor{currentstroke}%
\pgfsetdash{}{0pt}%
\pgfsys@defobject{currentmarker}{\pgfqpoint{0.000000in}{-0.027778in}}{\pgfqpoint{0.000000in}{0.000000in}}{%
\pgfpathmoveto{\pgfqpoint{0.000000in}{0.000000in}}%
\pgfpathlineto{\pgfqpoint{0.000000in}{-0.027778in}}%
\pgfusepath{stroke,fill}%
}%
\begin{pgfscope}%
\pgfsys@transformshift{3.099655in}{0.525000in}%
\pgfsys@useobject{currentmarker}{}%
\end{pgfscope}%
\end{pgfscope}%
\begin{pgfscope}%
\pgfpathrectangle{\pgfqpoint{0.726250in}{0.525000in}}{\pgfqpoint{3.320000in}{2.887500in}}%
\pgfusepath{clip}%
\pgfsetbuttcap%
\pgfsetroundjoin%
\pgfsetlinewidth{0.803000pt}%
\definecolor{currentstroke}{rgb}{0.752941,0.752941,0.752941}%
\pgfsetstrokecolor{currentstroke}%
\pgfsetdash{{2.960000pt}{1.280000pt}}{0.000000pt}%
\pgfpathmoveto{\pgfqpoint{3.210269in}{0.525000in}}%
\pgfpathlineto{\pgfqpoint{3.210269in}{3.412500in}}%
\pgfusepath{stroke}%
\end{pgfscope}%
\begin{pgfscope}%
\pgfsetbuttcap%
\pgfsetroundjoin%
\definecolor{currentfill}{rgb}{0.000000,0.000000,0.000000}%
\pgfsetfillcolor{currentfill}%
\pgfsetlinewidth{0.602250pt}%
\definecolor{currentstroke}{rgb}{0.000000,0.000000,0.000000}%
\pgfsetstrokecolor{currentstroke}%
\pgfsetdash{}{0pt}%
\pgfsys@defobject{currentmarker}{\pgfqpoint{0.000000in}{-0.027778in}}{\pgfqpoint{0.000000in}{0.000000in}}{%
\pgfpathmoveto{\pgfqpoint{0.000000in}{0.000000in}}%
\pgfpathlineto{\pgfqpoint{0.000000in}{-0.027778in}}%
\pgfusepath{stroke,fill}%
}%
\begin{pgfscope}%
\pgfsys@transformshift{3.210269in}{0.525000in}%
\pgfsys@useobject{currentmarker}{}%
\end{pgfscope}%
\end{pgfscope}%
\begin{pgfscope}%
\pgfpathrectangle{\pgfqpoint{0.726250in}{0.525000in}}{\pgfqpoint{3.320000in}{2.887500in}}%
\pgfusepath{clip}%
\pgfsetbuttcap%
\pgfsetroundjoin%
\pgfsetlinewidth{0.803000pt}%
\definecolor{currentstroke}{rgb}{0.752941,0.752941,0.752941}%
\pgfsetstrokecolor{currentstroke}%
\pgfsetdash{{2.960000pt}{1.280000pt}}{0.000000pt}%
\pgfpathmoveto{\pgfqpoint{3.303791in}{0.525000in}}%
\pgfpathlineto{\pgfqpoint{3.303791in}{3.412500in}}%
\pgfusepath{stroke}%
\end{pgfscope}%
\begin{pgfscope}%
\pgfsetbuttcap%
\pgfsetroundjoin%
\definecolor{currentfill}{rgb}{0.000000,0.000000,0.000000}%
\pgfsetfillcolor{currentfill}%
\pgfsetlinewidth{0.602250pt}%
\definecolor{currentstroke}{rgb}{0.000000,0.000000,0.000000}%
\pgfsetstrokecolor{currentstroke}%
\pgfsetdash{}{0pt}%
\pgfsys@defobject{currentmarker}{\pgfqpoint{0.000000in}{-0.027778in}}{\pgfqpoint{0.000000in}{0.000000in}}{%
\pgfpathmoveto{\pgfqpoint{0.000000in}{0.000000in}}%
\pgfpathlineto{\pgfqpoint{0.000000in}{-0.027778in}}%
\pgfusepath{stroke,fill}%
}%
\begin{pgfscope}%
\pgfsys@transformshift{3.303791in}{0.525000in}%
\pgfsys@useobject{currentmarker}{}%
\end{pgfscope}%
\end{pgfscope}%
\begin{pgfscope}%
\pgfpathrectangle{\pgfqpoint{0.726250in}{0.525000in}}{\pgfqpoint{3.320000in}{2.887500in}}%
\pgfusepath{clip}%
\pgfsetbuttcap%
\pgfsetroundjoin%
\pgfsetlinewidth{0.803000pt}%
\definecolor{currentstroke}{rgb}{0.752941,0.752941,0.752941}%
\pgfsetstrokecolor{currentstroke}%
\pgfsetdash{{2.960000pt}{1.280000pt}}{0.000000pt}%
\pgfpathmoveto{\pgfqpoint{3.384804in}{0.525000in}}%
\pgfpathlineto{\pgfqpoint{3.384804in}{3.412500in}}%
\pgfusepath{stroke}%
\end{pgfscope}%
\begin{pgfscope}%
\pgfsetbuttcap%
\pgfsetroundjoin%
\definecolor{currentfill}{rgb}{0.000000,0.000000,0.000000}%
\pgfsetfillcolor{currentfill}%
\pgfsetlinewidth{0.602250pt}%
\definecolor{currentstroke}{rgb}{0.000000,0.000000,0.000000}%
\pgfsetstrokecolor{currentstroke}%
\pgfsetdash{}{0pt}%
\pgfsys@defobject{currentmarker}{\pgfqpoint{0.000000in}{-0.027778in}}{\pgfqpoint{0.000000in}{0.000000in}}{%
\pgfpathmoveto{\pgfqpoint{0.000000in}{0.000000in}}%
\pgfpathlineto{\pgfqpoint{0.000000in}{-0.027778in}}%
\pgfusepath{stroke,fill}%
}%
\begin{pgfscope}%
\pgfsys@transformshift{3.384804in}{0.525000in}%
\pgfsys@useobject{currentmarker}{}%
\end{pgfscope}%
\end{pgfscope}%
\begin{pgfscope}%
\pgfpathrectangle{\pgfqpoint{0.726250in}{0.525000in}}{\pgfqpoint{3.320000in}{2.887500in}}%
\pgfusepath{clip}%
\pgfsetbuttcap%
\pgfsetroundjoin%
\pgfsetlinewidth{0.803000pt}%
\definecolor{currentstroke}{rgb}{0.752941,0.752941,0.752941}%
\pgfsetstrokecolor{currentstroke}%
\pgfsetdash{{2.960000pt}{1.280000pt}}{0.000000pt}%
\pgfpathmoveto{\pgfqpoint{3.456263in}{0.525000in}}%
\pgfpathlineto{\pgfqpoint{3.456263in}{3.412500in}}%
\pgfusepath{stroke}%
\end{pgfscope}%
\begin{pgfscope}%
\pgfsetbuttcap%
\pgfsetroundjoin%
\definecolor{currentfill}{rgb}{0.000000,0.000000,0.000000}%
\pgfsetfillcolor{currentfill}%
\pgfsetlinewidth{0.602250pt}%
\definecolor{currentstroke}{rgb}{0.000000,0.000000,0.000000}%
\pgfsetstrokecolor{currentstroke}%
\pgfsetdash{}{0pt}%
\pgfsys@defobject{currentmarker}{\pgfqpoint{0.000000in}{-0.027778in}}{\pgfqpoint{0.000000in}{0.000000in}}{%
\pgfpathmoveto{\pgfqpoint{0.000000in}{0.000000in}}%
\pgfpathlineto{\pgfqpoint{0.000000in}{-0.027778in}}%
\pgfusepath{stroke,fill}%
}%
\begin{pgfscope}%
\pgfsys@transformshift{3.456263in}{0.525000in}%
\pgfsys@useobject{currentmarker}{}%
\end{pgfscope}%
\end{pgfscope}%
\begin{pgfscope}%
\pgfpathrectangle{\pgfqpoint{0.726250in}{0.525000in}}{\pgfqpoint{3.320000in}{2.887500in}}%
\pgfusepath{clip}%
\pgfsetbuttcap%
\pgfsetroundjoin%
\pgfsetlinewidth{0.803000pt}%
\definecolor{currentstroke}{rgb}{0.752941,0.752941,0.752941}%
\pgfsetstrokecolor{currentstroke}%
\pgfsetdash{{2.960000pt}{1.280000pt}}{0.000000pt}%
\pgfpathmoveto{\pgfqpoint{3.940713in}{0.525000in}}%
\pgfpathlineto{\pgfqpoint{3.940713in}{3.412500in}}%
\pgfusepath{stroke}%
\end{pgfscope}%
\begin{pgfscope}%
\pgfsetbuttcap%
\pgfsetroundjoin%
\definecolor{currentfill}{rgb}{0.000000,0.000000,0.000000}%
\pgfsetfillcolor{currentfill}%
\pgfsetlinewidth{0.602250pt}%
\definecolor{currentstroke}{rgb}{0.000000,0.000000,0.000000}%
\pgfsetstrokecolor{currentstroke}%
\pgfsetdash{}{0pt}%
\pgfsys@defobject{currentmarker}{\pgfqpoint{0.000000in}{-0.027778in}}{\pgfqpoint{0.000000in}{0.000000in}}{%
\pgfpathmoveto{\pgfqpoint{0.000000in}{0.000000in}}%
\pgfpathlineto{\pgfqpoint{0.000000in}{-0.027778in}}%
\pgfusepath{stroke,fill}%
}%
\begin{pgfscope}%
\pgfsys@transformshift{3.940713in}{0.525000in}%
\pgfsys@useobject{currentmarker}{}%
\end{pgfscope}%
\end{pgfscope}%
\begin{pgfscope}%
\definecolor{textcolor}{rgb}{0.000000,0.000000,0.000000}%
\pgfsetstrokecolor{textcolor}%
\pgfsetfillcolor{textcolor}%
\pgftext[x=2.386250in,y=0.251251in,,top]{\color{textcolor}\rmfamily\fontsize{9.000000}{10.800000}\selectfont Mass number}%
\end{pgfscope}%
\begin{pgfscope}%
\pgfpathrectangle{\pgfqpoint{0.726250in}{0.525000in}}{\pgfqpoint{3.320000in}{2.887500in}}%
\pgfusepath{clip}%
\pgfsetbuttcap%
\pgfsetroundjoin%
\pgfsetlinewidth{0.803000pt}%
\definecolor{currentstroke}{rgb}{0.752941,0.752941,0.752941}%
\pgfsetstrokecolor{currentstroke}%
\pgfsetdash{{2.960000pt}{1.280000pt}}{0.000000pt}%
\pgfpathmoveto{\pgfqpoint{0.726250in}{0.525000in}}%
\pgfpathlineto{\pgfqpoint{4.046250in}{0.525000in}}%
\pgfusepath{stroke}%
\end{pgfscope}%
\begin{pgfscope}%
\pgfsetbuttcap%
\pgfsetroundjoin%
\definecolor{currentfill}{rgb}{0.000000,0.000000,0.000000}%
\pgfsetfillcolor{currentfill}%
\pgfsetlinewidth{0.803000pt}%
\definecolor{currentstroke}{rgb}{0.000000,0.000000,0.000000}%
\pgfsetstrokecolor{currentstroke}%
\pgfsetdash{}{0pt}%
\pgfsys@defobject{currentmarker}{\pgfqpoint{-0.048611in}{0.000000in}}{\pgfqpoint{-0.000000in}{0.000000in}}{%
\pgfpathmoveto{\pgfqpoint{-0.000000in}{0.000000in}}%
\pgfpathlineto{\pgfqpoint{-0.048611in}{0.000000in}}%
\pgfusepath{stroke,fill}%
}%
\begin{pgfscope}%
\pgfsys@transformshift{0.726250in}{0.525000in}%
\pgfsys@useobject{currentmarker}{}%
\end{pgfscope}%
\end{pgfscope}%
\begin{pgfscope}%
\definecolor{textcolor}{rgb}{0.000000,0.000000,0.000000}%
\pgfsetstrokecolor{textcolor}%
\pgfsetfillcolor{textcolor}%
\pgftext[x=0.549499in, y=0.477515in, left, base]{\color{textcolor}\rmfamily\fontsize{9.000000}{10.800000}\selectfont 0}%
\end{pgfscope}%
\begin{pgfscope}%
\pgfpathrectangle{\pgfqpoint{0.726250in}{0.525000in}}{\pgfqpoint{3.320000in}{2.887500in}}%
\pgfusepath{clip}%
\pgfsetbuttcap%
\pgfsetroundjoin%
\pgfsetlinewidth{0.803000pt}%
\definecolor{currentstroke}{rgb}{0.752941,0.752941,0.752941}%
\pgfsetstrokecolor{currentstroke}%
\pgfsetdash{{2.960000pt}{1.280000pt}}{0.000000pt}%
\pgfpathmoveto{\pgfqpoint{0.726250in}{1.102500in}}%
\pgfpathlineto{\pgfqpoint{4.046250in}{1.102500in}}%
\pgfusepath{stroke}%
\end{pgfscope}%
\begin{pgfscope}%
\pgfsetbuttcap%
\pgfsetroundjoin%
\definecolor{currentfill}{rgb}{0.000000,0.000000,0.000000}%
\pgfsetfillcolor{currentfill}%
\pgfsetlinewidth{0.803000pt}%
\definecolor{currentstroke}{rgb}{0.000000,0.000000,0.000000}%
\pgfsetstrokecolor{currentstroke}%
\pgfsetdash{}{0pt}%
\pgfsys@defobject{currentmarker}{\pgfqpoint{-0.048611in}{0.000000in}}{\pgfqpoint{-0.000000in}{0.000000in}}{%
\pgfpathmoveto{\pgfqpoint{-0.000000in}{0.000000in}}%
\pgfpathlineto{\pgfqpoint{-0.048611in}{0.000000in}}%
\pgfusepath{stroke,fill}%
}%
\begin{pgfscope}%
\pgfsys@transformshift{0.726250in}{1.102500in}%
\pgfsys@useobject{currentmarker}{}%
\end{pgfscope}%
\end{pgfscope}%
\begin{pgfscope}%
\definecolor{textcolor}{rgb}{0.000000,0.000000,0.000000}%
\pgfsetstrokecolor{textcolor}%
\pgfsetfillcolor{textcolor}%
\pgftext[x=0.549499in, y=1.055015in, left, base]{\color{textcolor}\rmfamily\fontsize{9.000000}{10.800000}\selectfont 2}%
\end{pgfscope}%
\begin{pgfscope}%
\pgfpathrectangle{\pgfqpoint{0.726250in}{0.525000in}}{\pgfqpoint{3.320000in}{2.887500in}}%
\pgfusepath{clip}%
\pgfsetbuttcap%
\pgfsetroundjoin%
\pgfsetlinewidth{0.803000pt}%
\definecolor{currentstroke}{rgb}{0.752941,0.752941,0.752941}%
\pgfsetstrokecolor{currentstroke}%
\pgfsetdash{{2.960000pt}{1.280000pt}}{0.000000pt}%
\pgfpathmoveto{\pgfqpoint{0.726250in}{1.680000in}}%
\pgfpathlineto{\pgfqpoint{4.046250in}{1.680000in}}%
\pgfusepath{stroke}%
\end{pgfscope}%
\begin{pgfscope}%
\pgfsetbuttcap%
\pgfsetroundjoin%
\definecolor{currentfill}{rgb}{0.000000,0.000000,0.000000}%
\pgfsetfillcolor{currentfill}%
\pgfsetlinewidth{0.803000pt}%
\definecolor{currentstroke}{rgb}{0.000000,0.000000,0.000000}%
\pgfsetstrokecolor{currentstroke}%
\pgfsetdash{}{0pt}%
\pgfsys@defobject{currentmarker}{\pgfqpoint{-0.048611in}{0.000000in}}{\pgfqpoint{-0.000000in}{0.000000in}}{%
\pgfpathmoveto{\pgfqpoint{-0.000000in}{0.000000in}}%
\pgfpathlineto{\pgfqpoint{-0.048611in}{0.000000in}}%
\pgfusepath{stroke,fill}%
}%
\begin{pgfscope}%
\pgfsys@transformshift{0.726250in}{1.680000in}%
\pgfsys@useobject{currentmarker}{}%
\end{pgfscope}%
\end{pgfscope}%
\begin{pgfscope}%
\definecolor{textcolor}{rgb}{0.000000,0.000000,0.000000}%
\pgfsetstrokecolor{textcolor}%
\pgfsetfillcolor{textcolor}%
\pgftext[x=0.549499in, y=1.632515in, left, base]{\color{textcolor}\rmfamily\fontsize{9.000000}{10.800000}\selectfont 4}%
\end{pgfscope}%
\begin{pgfscope}%
\pgfpathrectangle{\pgfqpoint{0.726250in}{0.525000in}}{\pgfqpoint{3.320000in}{2.887500in}}%
\pgfusepath{clip}%
\pgfsetbuttcap%
\pgfsetroundjoin%
\pgfsetlinewidth{0.803000pt}%
\definecolor{currentstroke}{rgb}{0.752941,0.752941,0.752941}%
\pgfsetstrokecolor{currentstroke}%
\pgfsetdash{{2.960000pt}{1.280000pt}}{0.000000pt}%
\pgfpathmoveto{\pgfqpoint{0.726250in}{2.257500in}}%
\pgfpathlineto{\pgfqpoint{4.046250in}{2.257500in}}%
\pgfusepath{stroke}%
\end{pgfscope}%
\begin{pgfscope}%
\pgfsetbuttcap%
\pgfsetroundjoin%
\definecolor{currentfill}{rgb}{0.000000,0.000000,0.000000}%
\pgfsetfillcolor{currentfill}%
\pgfsetlinewidth{0.803000pt}%
\definecolor{currentstroke}{rgb}{0.000000,0.000000,0.000000}%
\pgfsetstrokecolor{currentstroke}%
\pgfsetdash{}{0pt}%
\pgfsys@defobject{currentmarker}{\pgfqpoint{-0.048611in}{0.000000in}}{\pgfqpoint{-0.000000in}{0.000000in}}{%
\pgfpathmoveto{\pgfqpoint{-0.000000in}{0.000000in}}%
\pgfpathlineto{\pgfqpoint{-0.048611in}{0.000000in}}%
\pgfusepath{stroke,fill}%
}%
\begin{pgfscope}%
\pgfsys@transformshift{0.726250in}{2.257500in}%
\pgfsys@useobject{currentmarker}{}%
\end{pgfscope}%
\end{pgfscope}%
\begin{pgfscope}%
\definecolor{textcolor}{rgb}{0.000000,0.000000,0.000000}%
\pgfsetstrokecolor{textcolor}%
\pgfsetfillcolor{textcolor}%
\pgftext[x=0.549499in, y=2.210015in, left, base]{\color{textcolor}\rmfamily\fontsize{9.000000}{10.800000}\selectfont 6}%
\end{pgfscope}%
\begin{pgfscope}%
\pgfpathrectangle{\pgfqpoint{0.726250in}{0.525000in}}{\pgfqpoint{3.320000in}{2.887500in}}%
\pgfusepath{clip}%
\pgfsetbuttcap%
\pgfsetroundjoin%
\pgfsetlinewidth{0.803000pt}%
\definecolor{currentstroke}{rgb}{0.752941,0.752941,0.752941}%
\pgfsetstrokecolor{currentstroke}%
\pgfsetdash{{2.960000pt}{1.280000pt}}{0.000000pt}%
\pgfpathmoveto{\pgfqpoint{0.726250in}{2.835000in}}%
\pgfpathlineto{\pgfqpoint{4.046250in}{2.835000in}}%
\pgfusepath{stroke}%
\end{pgfscope}%
\begin{pgfscope}%
\pgfsetbuttcap%
\pgfsetroundjoin%
\definecolor{currentfill}{rgb}{0.000000,0.000000,0.000000}%
\pgfsetfillcolor{currentfill}%
\pgfsetlinewidth{0.803000pt}%
\definecolor{currentstroke}{rgb}{0.000000,0.000000,0.000000}%
\pgfsetstrokecolor{currentstroke}%
\pgfsetdash{}{0pt}%
\pgfsys@defobject{currentmarker}{\pgfqpoint{-0.048611in}{0.000000in}}{\pgfqpoint{-0.000000in}{0.000000in}}{%
\pgfpathmoveto{\pgfqpoint{-0.000000in}{0.000000in}}%
\pgfpathlineto{\pgfqpoint{-0.048611in}{0.000000in}}%
\pgfusepath{stroke,fill}%
}%
\begin{pgfscope}%
\pgfsys@transformshift{0.726250in}{2.835000in}%
\pgfsys@useobject{currentmarker}{}%
\end{pgfscope}%
\end{pgfscope}%
\begin{pgfscope}%
\definecolor{textcolor}{rgb}{0.000000,0.000000,0.000000}%
\pgfsetstrokecolor{textcolor}%
\pgfsetfillcolor{textcolor}%
\pgftext[x=0.549499in, y=2.787515in, left, base]{\color{textcolor}\rmfamily\fontsize{9.000000}{10.800000}\selectfont 8}%
\end{pgfscope}%
\begin{pgfscope}%
\pgfpathrectangle{\pgfqpoint{0.726250in}{0.525000in}}{\pgfqpoint{3.320000in}{2.887500in}}%
\pgfusepath{clip}%
\pgfsetbuttcap%
\pgfsetroundjoin%
\pgfsetlinewidth{0.803000pt}%
\definecolor{currentstroke}{rgb}{0.752941,0.752941,0.752941}%
\pgfsetstrokecolor{currentstroke}%
\pgfsetdash{{2.960000pt}{1.280000pt}}{0.000000pt}%
\pgfpathmoveto{\pgfqpoint{0.726250in}{3.412500in}}%
\pgfpathlineto{\pgfqpoint{4.046250in}{3.412500in}}%
\pgfusepath{stroke}%
\end{pgfscope}%
\begin{pgfscope}%
\pgfsetbuttcap%
\pgfsetroundjoin%
\definecolor{currentfill}{rgb}{0.000000,0.000000,0.000000}%
\pgfsetfillcolor{currentfill}%
\pgfsetlinewidth{0.803000pt}%
\definecolor{currentstroke}{rgb}{0.000000,0.000000,0.000000}%
\pgfsetstrokecolor{currentstroke}%
\pgfsetdash{}{0pt}%
\pgfsys@defobject{currentmarker}{\pgfqpoint{-0.048611in}{0.000000in}}{\pgfqpoint{-0.000000in}{0.000000in}}{%
\pgfpathmoveto{\pgfqpoint{-0.000000in}{0.000000in}}%
\pgfpathlineto{\pgfqpoint{-0.048611in}{0.000000in}}%
\pgfusepath{stroke,fill}%
}%
\begin{pgfscope}%
\pgfsys@transformshift{0.726250in}{3.412500in}%
\pgfsys@useobject{currentmarker}{}%
\end{pgfscope}%
\end{pgfscope}%
\begin{pgfscope}%
\definecolor{textcolor}{rgb}{0.000000,0.000000,0.000000}%
\pgfsetstrokecolor{textcolor}%
\pgfsetfillcolor{textcolor}%
\pgftext[x=0.469970in, y=3.365015in, left, base]{\color{textcolor}\rmfamily\fontsize{9.000000}{10.800000}\selectfont 10}%
\end{pgfscope}%
\begin{pgfscope}%
\definecolor{textcolor}{rgb}{0.000000,0.000000,0.000000}%
\pgfsetstrokecolor{textcolor}%
\pgfsetfillcolor{textcolor}%
\pgftext[x=0.414415in,y=1.968750in,,bottom,rotate=90.000000]{\color{textcolor}\rmfamily\fontsize{9.000000}{10.800000}\selectfont Binding enegry per nucleon [MeV]}%
\end{pgfscope}%
\begin{pgfscope}%
\pgfpathrectangle{\pgfqpoint{0.726250in}{0.525000in}}{\pgfqpoint{3.320000in}{2.887500in}}%
\pgfusepath{clip}%
\pgfsetrectcap%
\pgfsetroundjoin%
\pgfsetlinewidth{1.003750pt}%
\definecolor{currentstroke}{rgb}{0.000000,0.000000,0.000000}%
\pgfsetstrokecolor{currentstroke}%
\pgfsetdash{}{0pt}%
\pgfpathmoveto{\pgfqpoint{1.146779in}{0.846172in}}%
\pgfpathlineto{\pgfqpoint{1.392773in}{1.267861in}}%
\pgfpathlineto{\pgfqpoint{1.567308in}{2.567593in}}%
\pgfpathlineto{\pgfqpoint{1.813302in}{2.064711in}}%
\pgfpathlineto{\pgfqpoint{1.906824in}{2.143860in}}%
\pgfpathlineto{\pgfqpoint{2.059295in}{2.391096in}}%
\pgfpathlineto{\pgfqpoint{2.123217in}{2.394680in}}%
\pgfpathlineto{\pgfqpoint{2.181041in}{2.525383in}}%
\pgfpathlineto{\pgfqpoint{2.233831in}{2.742642in}}%
\pgfpathlineto{\pgfqpoint{2.282392in}{2.681919in}}%
\pgfpathlineto{\pgfqpoint{2.327353in}{2.683584in}}%
\pgfpathlineto{\pgfqpoint{2.369211in}{2.748219in}}%
\pgfpathlineto{\pgfqpoint{2.408366in}{2.828130in}}%
\pgfpathlineto{\pgfqpoint{2.445147in}{2.763023in}}%
\pgfpathlineto{\pgfqpoint{2.512627in}{2.771192in}}%
\pgfpathlineto{\pgfqpoint{2.543746in}{2.844310in}}%
\pgfpathlineto{\pgfqpoint{2.573347in}{2.826832in}}%
\pgfpathlineto{\pgfqpoint{2.601570in}{2.858234in}}%
\pgfpathlineto{\pgfqpoint{2.628539in}{2.867194in}}%
\pgfpathlineto{\pgfqpoint{2.654360in}{2.910280in}}%
\pgfpathlineto{\pgfqpoint{2.679126in}{2.899536in}}%
\pgfpathlineto{\pgfqpoint{2.702921in}{2.931405in}}%
\pgfpathlineto{\pgfqpoint{2.725818in}{2.930736in}}%
\pgfpathlineto{\pgfqpoint{2.747882in}{2.964286in}}%
\pgfpathlineto{\pgfqpoint{2.769172in}{2.964544in}}%
\pgfpathlineto{\pgfqpoint{2.789740in}{2.985339in}}%
\pgfpathlineto{\pgfqpoint{2.809633in}{2.973937in}}%
\pgfpathlineto{\pgfqpoint{2.828895in}{2.977391in}}%
\pgfpathlineto{\pgfqpoint{2.847564in}{2.978691in}}%
\pgfpathlineto{\pgfqpoint{2.865676in}{3.003485in}}%
\pgfpathlineto{\pgfqpoint{2.883262in}{2.985231in}}%
\pgfpathlineto{\pgfqpoint{2.900354in}{3.001144in}}%
\pgfpathlineto{\pgfqpoint{2.900354in}{2.985124in}}%
\pgfpathlineto{\pgfqpoint{2.933156in}{3.012374in}}%
\pgfpathlineto{\pgfqpoint{2.948915in}{2.995841in}}%
\pgfpathlineto{\pgfqpoint{2.964275in}{3.006881in}}%
\pgfpathlineto{\pgfqpoint{2.964275in}{2.990374in}}%
\pgfpathlineto{\pgfqpoint{2.964275in}{2.994189in}}%
\pgfpathlineto{\pgfqpoint{2.979256in}{3.001341in}}%
\pgfpathlineto{\pgfqpoint{2.993876in}{3.013033in}}%
\pgfpathlineto{\pgfqpoint{3.008152in}{3.008442in}}%
\pgfpathlineto{\pgfqpoint{3.022099in}{3.025049in}}%
\pgfpathlineto{\pgfqpoint{3.035734in}{3.013719in}}%
\pgfpathlineto{\pgfqpoint{3.049068in}{3.028169in}}%
\pgfpathlineto{\pgfqpoint{3.049068in}{3.024553in}}%
\pgfpathlineto{\pgfqpoint{3.074889in}{3.027507in}}%
\pgfpathlineto{\pgfqpoint{3.074889in}{3.043770in}}%
\pgfpathlineto{\pgfqpoint{3.087398in}{3.040348in}}%
\pgfpathlineto{\pgfqpoint{3.099655in}{3.053215in}}%
\pgfpathlineto{\pgfqpoint{3.099655in}{3.035942in}}%
\pgfpathlineto{\pgfqpoint{3.099655in}{3.037419in}}%
\pgfpathlineto{\pgfqpoint{3.111669in}{3.049277in}}%
\pgfpathlineto{\pgfqpoint{3.123450in}{3.059068in}}%
\pgfpathlineto{\pgfqpoint{3.135007in}{3.054511in}}%
\pgfpathlineto{\pgfqpoint{3.146347in}{3.059638in}}%
\pgfpathlineto{\pgfqpoint{3.146347in}{3.047631in}}%
\pgfpathlineto{\pgfqpoint{3.168411in}{3.063215in}}%
\pgfpathlineto{\pgfqpoint{3.179150in}{3.057419in}}%
\pgfpathlineto{\pgfqpoint{3.189701in}{3.063763in}}%
\pgfpathlineto{\pgfqpoint{3.189701in}{3.046383in}}%
\pgfpathlineto{\pgfqpoint{3.200072in}{3.056771in}}%
\pgfpathlineto{\pgfqpoint{3.210269in}{3.060449in}}%
\pgfpathlineto{\pgfqpoint{3.220297in}{3.055902in}}%
\pgfpathlineto{\pgfqpoint{3.230162in}{3.064428in}}%
\pgfpathlineto{\pgfqpoint{3.239870in}{3.052181in}}%
\pgfpathlineto{\pgfqpoint{3.249424in}{3.059493in}}%
\pgfpathlineto{\pgfqpoint{3.249424in}{3.047493in}}%
\pgfpathlineto{\pgfqpoint{3.258830in}{3.053612in}}%
\pgfpathlineto{\pgfqpoint{3.268093in}{3.054344in}}%
\pgfpathlineto{\pgfqpoint{3.277217in}{3.046987in}}%
\pgfpathlineto{\pgfqpoint{3.286205in}{3.053203in}}%
\pgfpathlineto{\pgfqpoint{3.295062in}{3.044222in}}%
\pgfpathlineto{\pgfqpoint{3.303791in}{3.045732in}}%
\pgfpathlineto{\pgfqpoint{3.303791in}{3.043392in}}%
\pgfpathlineto{\pgfqpoint{3.312397in}{3.042208in}}%
\pgfpathlineto{\pgfqpoint{3.320883in}{3.046292in}}%
\pgfpathlineto{\pgfqpoint{3.329251in}{3.038583in}}%
\pgfpathlineto{\pgfqpoint{3.337505in}{3.044402in}}%
\pgfpathlineto{\pgfqpoint{3.337505in}{3.033578in}}%
\pgfpathlineto{\pgfqpoint{3.345649in}{3.037378in}}%
\pgfpathlineto{\pgfqpoint{3.353685in}{3.038637in}}%
\pgfpathlineto{\pgfqpoint{3.353685in}{3.040439in}}%
\pgfpathlineto{\pgfqpoint{3.361616in}{3.035592in}}%
\pgfpathlineto{\pgfqpoint{3.369444in}{3.042267in}}%
\pgfpathlineto{\pgfqpoint{3.369444in}{3.025933in}}%
\pgfpathlineto{\pgfqpoint{3.384804in}{3.040248in}}%
\pgfpathlineto{\pgfqpoint{3.384804in}{3.035084in}}%
\pgfpathlineto{\pgfqpoint{3.392341in}{3.035955in}}%
\pgfpathlineto{\pgfqpoint{3.399785in}{3.035161in}}%
\pgfpathlineto{\pgfqpoint{3.399785in}{3.040208in}}%
\pgfpathlineto{\pgfqpoint{3.407139in}{3.035892in}}%
\pgfpathlineto{\pgfqpoint{3.414405in}{3.042163in}}%
\pgfpathlineto{\pgfqpoint{3.414405in}{3.030632in}}%
\pgfpathlineto{\pgfqpoint{3.421585in}{3.036386in}}%
\pgfpathlineto{\pgfqpoint{3.428681in}{3.040599in}}%
\pgfpathlineto{\pgfqpoint{3.428681in}{3.039567in}}%
\pgfpathlineto{\pgfqpoint{3.435695in}{3.040297in}}%
\pgfpathlineto{\pgfqpoint{3.435695in}{3.038637in}}%
\pgfpathlineto{\pgfqpoint{3.442628in}{3.046537in}}%
\pgfpathlineto{\pgfqpoint{3.449484in}{3.041171in}}%
\pgfpathlineto{\pgfqpoint{3.456263in}{3.040004in}}%
\pgfpathlineto{\pgfqpoint{3.462967in}{3.035195in}}%
\pgfpathlineto{\pgfqpoint{3.469597in}{3.035011in}}%
\pgfpathlineto{\pgfqpoint{3.469597in}{3.024920in}}%
\pgfpathlineto{\pgfqpoint{3.482645in}{3.027539in}}%
\pgfpathlineto{\pgfqpoint{3.482645in}{3.026249in}}%
\pgfpathlineto{\pgfqpoint{3.495418in}{3.018451in}}%
\pgfpathlineto{\pgfqpoint{3.495418in}{3.023839in}}%
\pgfpathlineto{\pgfqpoint{3.495418in}{3.010968in}}%
\pgfpathlineto{\pgfqpoint{3.501705in}{3.018383in}}%
\pgfpathlineto{\pgfqpoint{3.507927in}{3.018405in}}%
\pgfpathlineto{\pgfqpoint{3.507927in}{3.014116in}}%
\pgfpathlineto{\pgfqpoint{3.514087in}{3.010766in}}%
\pgfpathlineto{\pgfqpoint{3.520184in}{3.009596in}}%
\pgfpathlineto{\pgfqpoint{3.520184in}{3.013840in}}%
\pgfpathlineto{\pgfqpoint{3.526221in}{3.008644in}}%
\pgfpathlineto{\pgfqpoint{3.532199in}{3.010395in}}%
\pgfpathlineto{\pgfqpoint{3.532199in}{3.002558in}}%
\pgfpathlineto{\pgfqpoint{3.543979in}{3.004612in}}%
\pgfpathlineto{\pgfqpoint{3.543979in}{3.003875in}}%
\pgfpathlineto{\pgfqpoint{3.549785in}{2.999775in}}%
\pgfpathlineto{\pgfqpoint{3.555536in}{3.002473in}}%
\pgfpathlineto{\pgfqpoint{3.555536in}{2.990650in}}%
\pgfpathlineto{\pgfqpoint{3.566876in}{2.998728in}}%
\pgfpathlineto{\pgfqpoint{3.566876in}{2.993818in}}%
\pgfpathlineto{\pgfqpoint{3.578009in}{2.992993in}}%
\pgfpathlineto{\pgfqpoint{3.578009in}{2.994181in}}%
\pgfpathlineto{\pgfqpoint{3.583499in}{2.990082in}}%
\pgfpathlineto{\pgfqpoint{3.588940in}{2.992291in}}%
\pgfpathlineto{\pgfqpoint{3.588940in}{2.983307in}}%
\pgfpathlineto{\pgfqpoint{3.594333in}{2.987168in}}%
\pgfpathlineto{\pgfqpoint{3.594333in}{2.985996in}}%
\pgfpathlineto{\pgfqpoint{3.599679in}{2.988475in}}%
\pgfpathlineto{\pgfqpoint{3.599679in}{2.985891in}}%
\pgfpathlineto{\pgfqpoint{3.610230in}{2.982941in}}%
\pgfpathlineto{\pgfqpoint{3.610230in}{2.986050in}}%
\pgfpathlineto{\pgfqpoint{3.615438in}{2.982150in}}%
\pgfpathlineto{\pgfqpoint{3.620601in}{2.984149in}}%
\pgfpathlineto{\pgfqpoint{3.625721in}{2.979216in}}%
\pgfpathlineto{\pgfqpoint{3.630798in}{2.980671in}}%
\pgfpathlineto{\pgfqpoint{3.630798in}{2.972730in}}%
\pgfpathlineto{\pgfqpoint{3.640826in}{2.975880in}}%
\pgfpathlineto{\pgfqpoint{3.640826in}{2.973060in}}%
\pgfpathlineto{\pgfqpoint{3.645779in}{2.971382in}}%
\pgfpathlineto{\pgfqpoint{3.645779in}{2.969424in}}%
\pgfpathlineto{\pgfqpoint{3.650691in}{2.969962in}}%
\pgfpathlineto{\pgfqpoint{3.650691in}{2.971657in}}%
\pgfpathlineto{\pgfqpoint{3.650691in}{2.961361in}}%
\pgfpathlineto{\pgfqpoint{3.655564in}{2.967258in}}%
\pgfpathlineto{\pgfqpoint{3.660399in}{2.968760in}}%
\pgfpathlineto{\pgfqpoint{3.660399in}{2.963071in}}%
\pgfpathlineto{\pgfqpoint{3.669953in}{2.964578in}}%
\pgfpathlineto{\pgfqpoint{3.669953in}{2.963003in}}%
\pgfpathlineto{\pgfqpoint{3.674675in}{2.959564in}}%
\pgfpathlineto{\pgfqpoint{3.679359in}{2.959256in}}%
\pgfpathlineto{\pgfqpoint{3.679359in}{2.961395in}}%
\pgfpathlineto{\pgfqpoint{3.679359in}{2.952092in}}%
\pgfpathlineto{\pgfqpoint{3.684009in}{2.957354in}}%
\pgfpathlineto{\pgfqpoint{3.688622in}{2.958476in}}%
\pgfpathlineto{\pgfqpoint{3.688622in}{2.953207in}}%
\pgfpathlineto{\pgfqpoint{3.697746in}{2.954456in}}%
\pgfpathlineto{\pgfqpoint{3.697746in}{2.952860in}}%
\pgfpathlineto{\pgfqpoint{3.702256in}{2.949788in}}%
\pgfpathlineto{\pgfqpoint{3.706734in}{2.949400in}}%
\pgfpathlineto{\pgfqpoint{3.706734in}{2.951296in}}%
\pgfpathlineto{\pgfqpoint{3.706734in}{2.942924in}}%
\pgfpathlineto{\pgfqpoint{3.711178in}{2.948141in}}%
\pgfpathlineto{\pgfqpoint{3.715591in}{2.948601in}}%
\pgfpathlineto{\pgfqpoint{3.715591in}{2.943305in}}%
\pgfpathlineto{\pgfqpoint{3.715591in}{2.943871in}}%
\pgfpathlineto{\pgfqpoint{3.724320in}{2.943658in}}%
\pgfpathlineto{\pgfqpoint{3.728639in}{2.937213in}}%
\pgfpathlineto{\pgfqpoint{3.732926in}{2.934916in}}%
\pgfpathlineto{\pgfqpoint{3.737184in}{2.930429in}}%
\pgfpathlineto{\pgfqpoint{3.741412in}{2.929399in}}%
\pgfpathlineto{\pgfqpoint{3.741412in}{2.922688in}}%
\pgfpathlineto{\pgfqpoint{3.745610in}{2.924278in}}%
\pgfpathlineto{\pgfqpoint{3.749780in}{2.922807in}}%
\pgfpathlineto{\pgfqpoint{3.753921in}{2.916016in}}%
\pgfpathlineto{\pgfqpoint{3.758034in}{2.915035in}}%
\pgfpathlineto{\pgfqpoint{3.758034in}{2.915744in}}%
\pgfpathlineto{\pgfqpoint{3.766178in}{2.907066in}}%
\pgfpathlineto{\pgfqpoint{3.766178in}{2.910544in}}%
\pgfpathlineto{\pgfqpoint{3.770209in}{2.904098in}}%
\pgfpathlineto{\pgfqpoint{3.774214in}{2.905474in}}%
\pgfpathlineto{\pgfqpoint{3.774214in}{2.902395in}}%
\pgfpathlineto{\pgfqpoint{3.782145in}{2.899911in}}%
\pgfpathlineto{\pgfqpoint{3.786071in}{2.896577in}}%
\pgfpathlineto{\pgfqpoint{3.789973in}{2.897175in}}%
\pgfpathlineto{\pgfqpoint{3.789973in}{2.890566in}}%
\pgfpathlineto{\pgfqpoint{3.793850in}{2.893763in}}%
\pgfpathlineto{\pgfqpoint{3.797702in}{2.893276in}}%
\pgfpathlineto{\pgfqpoint{3.797702in}{2.889900in}}%
\pgfpathlineto{\pgfqpoint{3.801530in}{2.889517in}}%
\pgfpathlineto{\pgfqpoint{3.812870in}{2.885085in}}%
\pgfpathlineto{\pgfqpoint{3.812870in}{2.879004in}}%
\pgfpathlineto{\pgfqpoint{3.816603in}{2.881715in}}%
\pgfpathlineto{\pgfqpoint{3.820314in}{2.880828in}}%
\pgfpathlineto{\pgfqpoint{3.820314in}{2.878029in}}%
\pgfpathlineto{\pgfqpoint{3.827668in}{2.875987in}}%
\pgfpathlineto{\pgfqpoint{3.831312in}{2.873039in}}%
\pgfpathlineto{\pgfqpoint{3.834934in}{2.872419in}}%
\pgfpathlineto{\pgfqpoint{3.834934in}{2.867307in}}%
\pgfpathlineto{\pgfqpoint{3.838535in}{2.868053in}}%
\pgfpathlineto{\pgfqpoint{3.842114in}{2.867326in}}%
\pgfpathlineto{\pgfqpoint{3.842114in}{2.865784in}}%
\pgfpathlineto{\pgfqpoint{3.845672in}{2.863264in}}%
\pgfpathlineto{\pgfqpoint{3.849210in}{2.863133in}}%
\pgfpathlineto{\pgfqpoint{3.852727in}{2.860245in}}%
\pgfpathlineto{\pgfqpoint{3.856224in}{2.859211in}}%
\pgfpathlineto{\pgfqpoint{3.856224in}{2.854789in}}%
\pgfpathlineto{\pgfqpoint{3.859701in}{2.854965in}}%
\pgfpathlineto{\pgfqpoint{3.863157in}{2.853505in}}%
\pgfpathlineto{\pgfqpoint{3.863157in}{2.852042in}}%
\pgfpathlineto{\pgfqpoint{3.863157in}{2.852718in}}%
\pgfpathlineto{\pgfqpoint{3.866595in}{2.849968in}}%
\pgfpathlineto{\pgfqpoint{3.870013in}{2.849277in}}%
\pgfpathlineto{\pgfqpoint{3.873412in}{2.846131in}}%
\pgfpathlineto{\pgfqpoint{3.876792in}{2.845087in}}%
\pgfpathlineto{\pgfqpoint{3.876792in}{2.842347in}}%
\pgfpathlineto{\pgfqpoint{3.883496in}{2.840287in}}%
\pgfpathlineto{\pgfqpoint{3.886820in}{2.837403in}}%
\pgfpathlineto{\pgfqpoint{3.890126in}{2.836466in}}%
\pgfpathlineto{\pgfqpoint{3.890126in}{2.831731in}}%
\pgfpathlineto{\pgfqpoint{3.893414in}{2.832404in}}%
\pgfpathlineto{\pgfqpoint{3.896685in}{2.831709in}}%
\pgfpathlineto{\pgfqpoint{3.896685in}{2.830043in}}%
\pgfpathlineto{\pgfqpoint{3.903174in}{2.827454in}}%
\pgfpathlineto{\pgfqpoint{3.906392in}{2.824317in}}%
\pgfpathlineto{\pgfqpoint{3.909594in}{2.824058in}}%
\pgfpathlineto{\pgfqpoint{3.909594in}{2.819550in}}%
\pgfpathlineto{\pgfqpoint{3.915947in}{2.820137in}}%
\pgfpathlineto{\pgfqpoint{3.915947in}{2.818395in}}%
\pgfpathlineto{\pgfqpoint{3.928456in}{2.813785in}}%
\pgfpathlineto{\pgfqpoint{3.928456in}{2.810274in}}%
\pgfpathlineto{\pgfqpoint{3.934616in}{2.810211in}}%
\pgfpathlineto{\pgfqpoint{3.934616in}{2.809461in}}%
\pgfpathlineto{\pgfqpoint{3.937672in}{2.807649in}}%
\pgfpathlineto{\pgfqpoint{3.940713in}{2.807827in}}%
\pgfpathlineto{\pgfqpoint{3.943739in}{2.805421in}}%
\pgfpathlineto{\pgfqpoint{3.946750in}{2.805216in}}%
\pgfpathlineto{\pgfqpoint{3.949746in}{2.802098in}}%
\pgfpathlineto{\pgfqpoint{3.952728in}{2.801951in}}%
\pgfpathlineto{\pgfqpoint{3.952728in}{2.800331in}}%
\pgfpathlineto{\pgfqpoint{3.958647in}{2.799011in}}%
\pgfpathlineto{\pgfqpoint{3.964508in}{2.796727in}}%
\pgfpathlineto{\pgfqpoint{3.967418in}{2.791106in}}%
\pgfpathlineto{\pgfqpoint{4.028138in}{2.724821in}}%
\pgfpathlineto{\pgfqpoint{4.030759in}{2.723841in}}%
\pgfpathlineto{\pgfqpoint{4.046250in}{2.710874in}}%
\pgfpathlineto{\pgfqpoint{4.046250in}{2.710874in}}%
\pgfusepath{stroke}%
\end{pgfscope}%
\begin{pgfscope}%
\pgfpathrectangle{\pgfqpoint{0.726250in}{0.525000in}}{\pgfqpoint{3.320000in}{2.887500in}}%
\pgfusepath{clip}%
\pgfsetbuttcap%
\pgfsetroundjoin%
\definecolor{currentfill}{rgb}{0.000000,0.000000,0.000000}%
\pgfsetfillcolor{currentfill}%
\pgfsetlinewidth{1.003750pt}%
\definecolor{currentstroke}{rgb}{0.000000,0.000000,0.000000}%
\pgfsetstrokecolor{currentstroke}%
\pgfsetdash{}{0pt}%
\pgfsys@defobject{currentmarker}{\pgfqpoint{-0.020833in}{-0.020833in}}{\pgfqpoint{0.020833in}{0.020833in}}{%
\pgfpathmoveto{\pgfqpoint{0.000000in}{-0.020833in}}%
\pgfpathcurveto{\pgfqpoint{0.005525in}{-0.020833in}}{\pgfqpoint{0.010825in}{-0.018638in}}{\pgfqpoint{0.014731in}{-0.014731in}}%
\pgfpathcurveto{\pgfqpoint{0.018638in}{-0.010825in}}{\pgfqpoint{0.020833in}{-0.005525in}}{\pgfqpoint{0.020833in}{0.000000in}}%
\pgfpathcurveto{\pgfqpoint{0.020833in}{0.005525in}}{\pgfqpoint{0.018638in}{0.010825in}}{\pgfqpoint{0.014731in}{0.014731in}}%
\pgfpathcurveto{\pgfqpoint{0.010825in}{0.018638in}}{\pgfqpoint{0.005525in}{0.020833in}}{\pgfqpoint{0.000000in}{0.020833in}}%
\pgfpathcurveto{\pgfqpoint{-0.005525in}{0.020833in}}{\pgfqpoint{-0.010825in}{0.018638in}}{\pgfqpoint{-0.014731in}{0.014731in}}%
\pgfpathcurveto{\pgfqpoint{-0.018638in}{0.010825in}}{\pgfqpoint{-0.020833in}{0.005525in}}{\pgfqpoint{-0.020833in}{0.000000in}}%
\pgfpathcurveto{\pgfqpoint{-0.020833in}{-0.005525in}}{\pgfqpoint{-0.018638in}{-0.010825in}}{\pgfqpoint{-0.014731in}{-0.014731in}}%
\pgfpathcurveto{\pgfqpoint{-0.010825in}{-0.018638in}}{\pgfqpoint{-0.005525in}{-0.020833in}}{\pgfqpoint{0.000000in}{-0.020833in}}%
\pgfpathlineto{\pgfqpoint{0.000000in}{-0.020833in}}%
\pgfpathclose%
\pgfusepath{stroke,fill}%
}%
\begin{pgfscope}%
\pgfsys@transformshift{1.146779in}{0.846172in}%
\pgfsys@useobject{currentmarker}{}%
\end{pgfscope}%
\end{pgfscope}%
\begin{pgfscope}%
\pgfpathrectangle{\pgfqpoint{0.726250in}{0.525000in}}{\pgfqpoint{3.320000in}{2.887500in}}%
\pgfusepath{clip}%
\pgfsetbuttcap%
\pgfsetroundjoin%
\definecolor{currentfill}{rgb}{0.000000,0.000000,0.000000}%
\pgfsetfillcolor{currentfill}%
\pgfsetlinewidth{1.003750pt}%
\definecolor{currentstroke}{rgb}{0.000000,0.000000,0.000000}%
\pgfsetstrokecolor{currentstroke}%
\pgfsetdash{}{0pt}%
\pgfsys@defobject{currentmarker}{\pgfqpoint{-0.020833in}{-0.020833in}}{\pgfqpoint{0.020833in}{0.020833in}}{%
\pgfpathmoveto{\pgfqpoint{0.000000in}{-0.020833in}}%
\pgfpathcurveto{\pgfqpoint{0.005525in}{-0.020833in}}{\pgfqpoint{0.010825in}{-0.018638in}}{\pgfqpoint{0.014731in}{-0.014731in}}%
\pgfpathcurveto{\pgfqpoint{0.018638in}{-0.010825in}}{\pgfqpoint{0.020833in}{-0.005525in}}{\pgfqpoint{0.020833in}{0.000000in}}%
\pgfpathcurveto{\pgfqpoint{0.020833in}{0.005525in}}{\pgfqpoint{0.018638in}{0.010825in}}{\pgfqpoint{0.014731in}{0.014731in}}%
\pgfpathcurveto{\pgfqpoint{0.010825in}{0.018638in}}{\pgfqpoint{0.005525in}{0.020833in}}{\pgfqpoint{0.000000in}{0.020833in}}%
\pgfpathcurveto{\pgfqpoint{-0.005525in}{0.020833in}}{\pgfqpoint{-0.010825in}{0.018638in}}{\pgfqpoint{-0.014731in}{0.014731in}}%
\pgfpathcurveto{\pgfqpoint{-0.018638in}{0.010825in}}{\pgfqpoint{-0.020833in}{0.005525in}}{\pgfqpoint{-0.020833in}{0.000000in}}%
\pgfpathcurveto{\pgfqpoint{-0.020833in}{-0.005525in}}{\pgfqpoint{-0.018638in}{-0.010825in}}{\pgfqpoint{-0.014731in}{-0.014731in}}%
\pgfpathcurveto{\pgfqpoint{-0.010825in}{-0.018638in}}{\pgfqpoint{-0.005525in}{-0.020833in}}{\pgfqpoint{0.000000in}{-0.020833in}}%
\pgfpathlineto{\pgfqpoint{0.000000in}{-0.020833in}}%
\pgfpathclose%
\pgfusepath{stroke,fill}%
}%
\begin{pgfscope}%
\pgfsys@transformshift{1.392773in}{1.267861in}%
\pgfsys@useobject{currentmarker}{}%
\end{pgfscope}%
\end{pgfscope}%
\begin{pgfscope}%
\pgfpathrectangle{\pgfqpoint{0.726250in}{0.525000in}}{\pgfqpoint{3.320000in}{2.887500in}}%
\pgfusepath{clip}%
\pgfsetbuttcap%
\pgfsetroundjoin%
\definecolor{currentfill}{rgb}{0.000000,0.000000,0.000000}%
\pgfsetfillcolor{currentfill}%
\pgfsetlinewidth{1.003750pt}%
\definecolor{currentstroke}{rgb}{0.000000,0.000000,0.000000}%
\pgfsetstrokecolor{currentstroke}%
\pgfsetdash{}{0pt}%
\pgfsys@defobject{currentmarker}{\pgfqpoint{-0.020833in}{-0.020833in}}{\pgfqpoint{0.020833in}{0.020833in}}{%
\pgfpathmoveto{\pgfqpoint{0.000000in}{-0.020833in}}%
\pgfpathcurveto{\pgfqpoint{0.005525in}{-0.020833in}}{\pgfqpoint{0.010825in}{-0.018638in}}{\pgfqpoint{0.014731in}{-0.014731in}}%
\pgfpathcurveto{\pgfqpoint{0.018638in}{-0.010825in}}{\pgfqpoint{0.020833in}{-0.005525in}}{\pgfqpoint{0.020833in}{0.000000in}}%
\pgfpathcurveto{\pgfqpoint{0.020833in}{0.005525in}}{\pgfqpoint{0.018638in}{0.010825in}}{\pgfqpoint{0.014731in}{0.014731in}}%
\pgfpathcurveto{\pgfqpoint{0.010825in}{0.018638in}}{\pgfqpoint{0.005525in}{0.020833in}}{\pgfqpoint{0.000000in}{0.020833in}}%
\pgfpathcurveto{\pgfqpoint{-0.005525in}{0.020833in}}{\pgfqpoint{-0.010825in}{0.018638in}}{\pgfqpoint{-0.014731in}{0.014731in}}%
\pgfpathcurveto{\pgfqpoint{-0.018638in}{0.010825in}}{\pgfqpoint{-0.020833in}{0.005525in}}{\pgfqpoint{-0.020833in}{0.000000in}}%
\pgfpathcurveto{\pgfqpoint{-0.020833in}{-0.005525in}}{\pgfqpoint{-0.018638in}{-0.010825in}}{\pgfqpoint{-0.014731in}{-0.014731in}}%
\pgfpathcurveto{\pgfqpoint{-0.010825in}{-0.018638in}}{\pgfqpoint{-0.005525in}{-0.020833in}}{\pgfqpoint{0.000000in}{-0.020833in}}%
\pgfpathlineto{\pgfqpoint{0.000000in}{-0.020833in}}%
\pgfpathclose%
\pgfusepath{stroke,fill}%
}%
\begin{pgfscope}%
\pgfsys@transformshift{1.567308in}{2.567593in}%
\pgfsys@useobject{currentmarker}{}%
\end{pgfscope}%
\end{pgfscope}%
\begin{pgfscope}%
\pgfpathrectangle{\pgfqpoint{0.726250in}{0.525000in}}{\pgfqpoint{3.320000in}{2.887500in}}%
\pgfusepath{clip}%
\pgfsetbuttcap%
\pgfsetroundjoin%
\definecolor{currentfill}{rgb}{0.000000,0.000000,0.000000}%
\pgfsetfillcolor{currentfill}%
\pgfsetlinewidth{1.003750pt}%
\definecolor{currentstroke}{rgb}{0.000000,0.000000,0.000000}%
\pgfsetstrokecolor{currentstroke}%
\pgfsetdash{}{0pt}%
\pgfsys@defobject{currentmarker}{\pgfqpoint{-0.020833in}{-0.020833in}}{\pgfqpoint{0.020833in}{0.020833in}}{%
\pgfpathmoveto{\pgfqpoint{0.000000in}{-0.020833in}}%
\pgfpathcurveto{\pgfqpoint{0.005525in}{-0.020833in}}{\pgfqpoint{0.010825in}{-0.018638in}}{\pgfqpoint{0.014731in}{-0.014731in}}%
\pgfpathcurveto{\pgfqpoint{0.018638in}{-0.010825in}}{\pgfqpoint{0.020833in}{-0.005525in}}{\pgfqpoint{0.020833in}{0.000000in}}%
\pgfpathcurveto{\pgfqpoint{0.020833in}{0.005525in}}{\pgfqpoint{0.018638in}{0.010825in}}{\pgfqpoint{0.014731in}{0.014731in}}%
\pgfpathcurveto{\pgfqpoint{0.010825in}{0.018638in}}{\pgfqpoint{0.005525in}{0.020833in}}{\pgfqpoint{0.000000in}{0.020833in}}%
\pgfpathcurveto{\pgfqpoint{-0.005525in}{0.020833in}}{\pgfqpoint{-0.010825in}{0.018638in}}{\pgfqpoint{-0.014731in}{0.014731in}}%
\pgfpathcurveto{\pgfqpoint{-0.018638in}{0.010825in}}{\pgfqpoint{-0.020833in}{0.005525in}}{\pgfqpoint{-0.020833in}{0.000000in}}%
\pgfpathcurveto{\pgfqpoint{-0.020833in}{-0.005525in}}{\pgfqpoint{-0.018638in}{-0.010825in}}{\pgfqpoint{-0.014731in}{-0.014731in}}%
\pgfpathcurveto{\pgfqpoint{-0.010825in}{-0.018638in}}{\pgfqpoint{-0.005525in}{-0.020833in}}{\pgfqpoint{0.000000in}{-0.020833in}}%
\pgfpathlineto{\pgfqpoint{0.000000in}{-0.020833in}}%
\pgfpathclose%
\pgfusepath{stroke,fill}%
}%
\begin{pgfscope}%
\pgfsys@transformshift{1.813302in}{2.064711in}%
\pgfsys@useobject{currentmarker}{}%
\end{pgfscope}%
\end{pgfscope}%
\begin{pgfscope}%
\pgfpathrectangle{\pgfqpoint{0.726250in}{0.525000in}}{\pgfqpoint{3.320000in}{2.887500in}}%
\pgfusepath{clip}%
\pgfsetbuttcap%
\pgfsetroundjoin%
\definecolor{currentfill}{rgb}{0.000000,0.000000,0.000000}%
\pgfsetfillcolor{currentfill}%
\pgfsetlinewidth{1.003750pt}%
\definecolor{currentstroke}{rgb}{0.000000,0.000000,0.000000}%
\pgfsetstrokecolor{currentstroke}%
\pgfsetdash{}{0pt}%
\pgfsys@defobject{currentmarker}{\pgfqpoint{-0.020833in}{-0.020833in}}{\pgfqpoint{0.020833in}{0.020833in}}{%
\pgfpathmoveto{\pgfqpoint{0.000000in}{-0.020833in}}%
\pgfpathcurveto{\pgfqpoint{0.005525in}{-0.020833in}}{\pgfqpoint{0.010825in}{-0.018638in}}{\pgfqpoint{0.014731in}{-0.014731in}}%
\pgfpathcurveto{\pgfqpoint{0.018638in}{-0.010825in}}{\pgfqpoint{0.020833in}{-0.005525in}}{\pgfqpoint{0.020833in}{0.000000in}}%
\pgfpathcurveto{\pgfqpoint{0.020833in}{0.005525in}}{\pgfqpoint{0.018638in}{0.010825in}}{\pgfqpoint{0.014731in}{0.014731in}}%
\pgfpathcurveto{\pgfqpoint{0.010825in}{0.018638in}}{\pgfqpoint{0.005525in}{0.020833in}}{\pgfqpoint{0.000000in}{0.020833in}}%
\pgfpathcurveto{\pgfqpoint{-0.005525in}{0.020833in}}{\pgfqpoint{-0.010825in}{0.018638in}}{\pgfqpoint{-0.014731in}{0.014731in}}%
\pgfpathcurveto{\pgfqpoint{-0.018638in}{0.010825in}}{\pgfqpoint{-0.020833in}{0.005525in}}{\pgfqpoint{-0.020833in}{0.000000in}}%
\pgfpathcurveto{\pgfqpoint{-0.020833in}{-0.005525in}}{\pgfqpoint{-0.018638in}{-0.010825in}}{\pgfqpoint{-0.014731in}{-0.014731in}}%
\pgfpathcurveto{\pgfqpoint{-0.010825in}{-0.018638in}}{\pgfqpoint{-0.005525in}{-0.020833in}}{\pgfqpoint{0.000000in}{-0.020833in}}%
\pgfpathlineto{\pgfqpoint{0.000000in}{-0.020833in}}%
\pgfpathclose%
\pgfusepath{stroke,fill}%
}%
\begin{pgfscope}%
\pgfsys@transformshift{1.906824in}{2.143860in}%
\pgfsys@useobject{currentmarker}{}%
\end{pgfscope}%
\end{pgfscope}%
\begin{pgfscope}%
\pgfpathrectangle{\pgfqpoint{0.726250in}{0.525000in}}{\pgfqpoint{3.320000in}{2.887500in}}%
\pgfusepath{clip}%
\pgfsetbuttcap%
\pgfsetroundjoin%
\definecolor{currentfill}{rgb}{0.000000,0.000000,0.000000}%
\pgfsetfillcolor{currentfill}%
\pgfsetlinewidth{1.003750pt}%
\definecolor{currentstroke}{rgb}{0.000000,0.000000,0.000000}%
\pgfsetstrokecolor{currentstroke}%
\pgfsetdash{}{0pt}%
\pgfsys@defobject{currentmarker}{\pgfqpoint{-0.020833in}{-0.020833in}}{\pgfqpoint{0.020833in}{0.020833in}}{%
\pgfpathmoveto{\pgfqpoint{0.000000in}{-0.020833in}}%
\pgfpathcurveto{\pgfqpoint{0.005525in}{-0.020833in}}{\pgfqpoint{0.010825in}{-0.018638in}}{\pgfqpoint{0.014731in}{-0.014731in}}%
\pgfpathcurveto{\pgfqpoint{0.018638in}{-0.010825in}}{\pgfqpoint{0.020833in}{-0.005525in}}{\pgfqpoint{0.020833in}{0.000000in}}%
\pgfpathcurveto{\pgfqpoint{0.020833in}{0.005525in}}{\pgfqpoint{0.018638in}{0.010825in}}{\pgfqpoint{0.014731in}{0.014731in}}%
\pgfpathcurveto{\pgfqpoint{0.010825in}{0.018638in}}{\pgfqpoint{0.005525in}{0.020833in}}{\pgfqpoint{0.000000in}{0.020833in}}%
\pgfpathcurveto{\pgfqpoint{-0.005525in}{0.020833in}}{\pgfqpoint{-0.010825in}{0.018638in}}{\pgfqpoint{-0.014731in}{0.014731in}}%
\pgfpathcurveto{\pgfqpoint{-0.018638in}{0.010825in}}{\pgfqpoint{-0.020833in}{0.005525in}}{\pgfqpoint{-0.020833in}{0.000000in}}%
\pgfpathcurveto{\pgfqpoint{-0.020833in}{-0.005525in}}{\pgfqpoint{-0.018638in}{-0.010825in}}{\pgfqpoint{-0.014731in}{-0.014731in}}%
\pgfpathcurveto{\pgfqpoint{-0.010825in}{-0.018638in}}{\pgfqpoint{-0.005525in}{-0.020833in}}{\pgfqpoint{0.000000in}{-0.020833in}}%
\pgfpathlineto{\pgfqpoint{0.000000in}{-0.020833in}}%
\pgfpathclose%
\pgfusepath{stroke,fill}%
}%
\begin{pgfscope}%
\pgfsys@transformshift{2.059295in}{2.391096in}%
\pgfsys@useobject{currentmarker}{}%
\end{pgfscope}%
\end{pgfscope}%
\begin{pgfscope}%
\pgfpathrectangle{\pgfqpoint{0.726250in}{0.525000in}}{\pgfqpoint{3.320000in}{2.887500in}}%
\pgfusepath{clip}%
\pgfsetbuttcap%
\pgfsetroundjoin%
\definecolor{currentfill}{rgb}{0.000000,0.000000,0.000000}%
\pgfsetfillcolor{currentfill}%
\pgfsetlinewidth{1.003750pt}%
\definecolor{currentstroke}{rgb}{0.000000,0.000000,0.000000}%
\pgfsetstrokecolor{currentstroke}%
\pgfsetdash{}{0pt}%
\pgfsys@defobject{currentmarker}{\pgfqpoint{-0.020833in}{-0.020833in}}{\pgfqpoint{0.020833in}{0.020833in}}{%
\pgfpathmoveto{\pgfqpoint{0.000000in}{-0.020833in}}%
\pgfpathcurveto{\pgfqpoint{0.005525in}{-0.020833in}}{\pgfqpoint{0.010825in}{-0.018638in}}{\pgfqpoint{0.014731in}{-0.014731in}}%
\pgfpathcurveto{\pgfqpoint{0.018638in}{-0.010825in}}{\pgfqpoint{0.020833in}{-0.005525in}}{\pgfqpoint{0.020833in}{0.000000in}}%
\pgfpathcurveto{\pgfqpoint{0.020833in}{0.005525in}}{\pgfqpoint{0.018638in}{0.010825in}}{\pgfqpoint{0.014731in}{0.014731in}}%
\pgfpathcurveto{\pgfqpoint{0.010825in}{0.018638in}}{\pgfqpoint{0.005525in}{0.020833in}}{\pgfqpoint{0.000000in}{0.020833in}}%
\pgfpathcurveto{\pgfqpoint{-0.005525in}{0.020833in}}{\pgfqpoint{-0.010825in}{0.018638in}}{\pgfqpoint{-0.014731in}{0.014731in}}%
\pgfpathcurveto{\pgfqpoint{-0.018638in}{0.010825in}}{\pgfqpoint{-0.020833in}{0.005525in}}{\pgfqpoint{-0.020833in}{0.000000in}}%
\pgfpathcurveto{\pgfqpoint{-0.020833in}{-0.005525in}}{\pgfqpoint{-0.018638in}{-0.010825in}}{\pgfqpoint{-0.014731in}{-0.014731in}}%
\pgfpathcurveto{\pgfqpoint{-0.010825in}{-0.018638in}}{\pgfqpoint{-0.005525in}{-0.020833in}}{\pgfqpoint{0.000000in}{-0.020833in}}%
\pgfpathlineto{\pgfqpoint{0.000000in}{-0.020833in}}%
\pgfpathclose%
\pgfusepath{stroke,fill}%
}%
\begin{pgfscope}%
\pgfsys@transformshift{2.181041in}{2.525383in}%
\pgfsys@useobject{currentmarker}{}%
\end{pgfscope}%
\end{pgfscope}%
\begin{pgfscope}%
\pgfpathrectangle{\pgfqpoint{0.726250in}{0.525000in}}{\pgfqpoint{3.320000in}{2.887500in}}%
\pgfusepath{clip}%
\pgfsetbuttcap%
\pgfsetroundjoin%
\definecolor{currentfill}{rgb}{0.000000,0.000000,0.000000}%
\pgfsetfillcolor{currentfill}%
\pgfsetlinewidth{1.003750pt}%
\definecolor{currentstroke}{rgb}{0.000000,0.000000,0.000000}%
\pgfsetstrokecolor{currentstroke}%
\pgfsetdash{}{0pt}%
\pgfsys@defobject{currentmarker}{\pgfqpoint{-0.020833in}{-0.020833in}}{\pgfqpoint{0.020833in}{0.020833in}}{%
\pgfpathmoveto{\pgfqpoint{0.000000in}{-0.020833in}}%
\pgfpathcurveto{\pgfqpoint{0.005525in}{-0.020833in}}{\pgfqpoint{0.010825in}{-0.018638in}}{\pgfqpoint{0.014731in}{-0.014731in}}%
\pgfpathcurveto{\pgfqpoint{0.018638in}{-0.010825in}}{\pgfqpoint{0.020833in}{-0.005525in}}{\pgfqpoint{0.020833in}{0.000000in}}%
\pgfpathcurveto{\pgfqpoint{0.020833in}{0.005525in}}{\pgfqpoint{0.018638in}{0.010825in}}{\pgfqpoint{0.014731in}{0.014731in}}%
\pgfpathcurveto{\pgfqpoint{0.010825in}{0.018638in}}{\pgfqpoint{0.005525in}{0.020833in}}{\pgfqpoint{0.000000in}{0.020833in}}%
\pgfpathcurveto{\pgfqpoint{-0.005525in}{0.020833in}}{\pgfqpoint{-0.010825in}{0.018638in}}{\pgfqpoint{-0.014731in}{0.014731in}}%
\pgfpathcurveto{\pgfqpoint{-0.018638in}{0.010825in}}{\pgfqpoint{-0.020833in}{0.005525in}}{\pgfqpoint{-0.020833in}{0.000000in}}%
\pgfpathcurveto{\pgfqpoint{-0.020833in}{-0.005525in}}{\pgfqpoint{-0.018638in}{-0.010825in}}{\pgfqpoint{-0.014731in}{-0.014731in}}%
\pgfpathcurveto{\pgfqpoint{-0.010825in}{-0.018638in}}{\pgfqpoint{-0.005525in}{-0.020833in}}{\pgfqpoint{0.000000in}{-0.020833in}}%
\pgfpathlineto{\pgfqpoint{0.000000in}{-0.020833in}}%
\pgfpathclose%
\pgfusepath{stroke,fill}%
}%
\begin{pgfscope}%
\pgfsys@transformshift{2.233831in}{2.742642in}%
\pgfsys@useobject{currentmarker}{}%
\end{pgfscope}%
\end{pgfscope}%
\begin{pgfscope}%
\pgfpathrectangle{\pgfqpoint{0.726250in}{0.525000in}}{\pgfqpoint{3.320000in}{2.887500in}}%
\pgfusepath{clip}%
\pgfsetbuttcap%
\pgfsetroundjoin%
\definecolor{currentfill}{rgb}{0.000000,0.000000,0.000000}%
\pgfsetfillcolor{currentfill}%
\pgfsetlinewidth{1.003750pt}%
\definecolor{currentstroke}{rgb}{0.000000,0.000000,0.000000}%
\pgfsetstrokecolor{currentstroke}%
\pgfsetdash{}{0pt}%
\pgfsys@defobject{currentmarker}{\pgfqpoint{-0.020833in}{-0.020833in}}{\pgfqpoint{0.020833in}{0.020833in}}{%
\pgfpathmoveto{\pgfqpoint{0.000000in}{-0.020833in}}%
\pgfpathcurveto{\pgfqpoint{0.005525in}{-0.020833in}}{\pgfqpoint{0.010825in}{-0.018638in}}{\pgfqpoint{0.014731in}{-0.014731in}}%
\pgfpathcurveto{\pgfqpoint{0.018638in}{-0.010825in}}{\pgfqpoint{0.020833in}{-0.005525in}}{\pgfqpoint{0.020833in}{0.000000in}}%
\pgfpathcurveto{\pgfqpoint{0.020833in}{0.005525in}}{\pgfqpoint{0.018638in}{0.010825in}}{\pgfqpoint{0.014731in}{0.014731in}}%
\pgfpathcurveto{\pgfqpoint{0.010825in}{0.018638in}}{\pgfqpoint{0.005525in}{0.020833in}}{\pgfqpoint{0.000000in}{0.020833in}}%
\pgfpathcurveto{\pgfqpoint{-0.005525in}{0.020833in}}{\pgfqpoint{-0.010825in}{0.018638in}}{\pgfqpoint{-0.014731in}{0.014731in}}%
\pgfpathcurveto{\pgfqpoint{-0.018638in}{0.010825in}}{\pgfqpoint{-0.020833in}{0.005525in}}{\pgfqpoint{-0.020833in}{0.000000in}}%
\pgfpathcurveto{\pgfqpoint{-0.020833in}{-0.005525in}}{\pgfqpoint{-0.018638in}{-0.010825in}}{\pgfqpoint{-0.014731in}{-0.014731in}}%
\pgfpathcurveto{\pgfqpoint{-0.010825in}{-0.018638in}}{\pgfqpoint{-0.005525in}{-0.020833in}}{\pgfqpoint{0.000000in}{-0.020833in}}%
\pgfpathlineto{\pgfqpoint{0.000000in}{-0.020833in}}%
\pgfpathclose%
\pgfusepath{stroke,fill}%
}%
\begin{pgfscope}%
\pgfsys@transformshift{2.408366in}{2.828130in}%
\pgfsys@useobject{currentmarker}{}%
\end{pgfscope}%
\end{pgfscope}%
\begin{pgfscope}%
\pgfpathrectangle{\pgfqpoint{0.726250in}{0.525000in}}{\pgfqpoint{3.320000in}{2.887500in}}%
\pgfusepath{clip}%
\pgfsetbuttcap%
\pgfsetroundjoin%
\definecolor{currentfill}{rgb}{0.000000,0.000000,0.000000}%
\pgfsetfillcolor{currentfill}%
\pgfsetlinewidth{1.003750pt}%
\definecolor{currentstroke}{rgb}{0.000000,0.000000,0.000000}%
\pgfsetstrokecolor{currentstroke}%
\pgfsetdash{}{0pt}%
\pgfsys@defobject{currentmarker}{\pgfqpoint{-0.020833in}{-0.020833in}}{\pgfqpoint{0.020833in}{0.020833in}}{%
\pgfpathmoveto{\pgfqpoint{0.000000in}{-0.020833in}}%
\pgfpathcurveto{\pgfqpoint{0.005525in}{-0.020833in}}{\pgfqpoint{0.010825in}{-0.018638in}}{\pgfqpoint{0.014731in}{-0.014731in}}%
\pgfpathcurveto{\pgfqpoint{0.018638in}{-0.010825in}}{\pgfqpoint{0.020833in}{-0.005525in}}{\pgfqpoint{0.020833in}{0.000000in}}%
\pgfpathcurveto{\pgfqpoint{0.020833in}{0.005525in}}{\pgfqpoint{0.018638in}{0.010825in}}{\pgfqpoint{0.014731in}{0.014731in}}%
\pgfpathcurveto{\pgfqpoint{0.010825in}{0.018638in}}{\pgfqpoint{0.005525in}{0.020833in}}{\pgfqpoint{0.000000in}{0.020833in}}%
\pgfpathcurveto{\pgfqpoint{-0.005525in}{0.020833in}}{\pgfqpoint{-0.010825in}{0.018638in}}{\pgfqpoint{-0.014731in}{0.014731in}}%
\pgfpathcurveto{\pgfqpoint{-0.018638in}{0.010825in}}{\pgfqpoint{-0.020833in}{0.005525in}}{\pgfqpoint{-0.020833in}{0.000000in}}%
\pgfpathcurveto{\pgfqpoint{-0.020833in}{-0.005525in}}{\pgfqpoint{-0.018638in}{-0.010825in}}{\pgfqpoint{-0.014731in}{-0.014731in}}%
\pgfpathcurveto{\pgfqpoint{-0.010825in}{-0.018638in}}{\pgfqpoint{-0.005525in}{-0.020833in}}{\pgfqpoint{0.000000in}{-0.020833in}}%
\pgfpathlineto{\pgfqpoint{0.000000in}{-0.020833in}}%
\pgfpathclose%
\pgfusepath{stroke,fill}%
}%
\begin{pgfscope}%
\pgfsys@transformshift{3.168411in}{3.063215in}%
\pgfsys@useobject{currentmarker}{}%
\end{pgfscope}%
\end{pgfscope}%
\begin{pgfscope}%
\pgfsetrectcap%
\pgfsetmiterjoin%
\pgfsetlinewidth{1.003750pt}%
\definecolor{currentstroke}{rgb}{0.000000,0.000000,0.000000}%
\pgfsetstrokecolor{currentstroke}%
\pgfsetdash{}{0pt}%
\pgfpathmoveto{\pgfqpoint{0.726250in}{0.525000in}}%
\pgfpathlineto{\pgfqpoint{0.726250in}{3.412500in}}%
\pgfusepath{stroke}%
\end{pgfscope}%
\begin{pgfscope}%
\pgfsetrectcap%
\pgfsetmiterjoin%
\pgfsetlinewidth{1.003750pt}%
\definecolor{currentstroke}{rgb}{0.000000,0.000000,0.000000}%
\pgfsetstrokecolor{currentstroke}%
\pgfsetdash{}{0pt}%
\pgfpathmoveto{\pgfqpoint{4.046250in}{0.525000in}}%
\pgfpathlineto{\pgfqpoint{4.046250in}{3.412500in}}%
\pgfusepath{stroke}%
\end{pgfscope}%
\begin{pgfscope}%
\pgfsetrectcap%
\pgfsetmiterjoin%
\pgfsetlinewidth{1.003750pt}%
\definecolor{currentstroke}{rgb}{0.000000,0.000000,0.000000}%
\pgfsetstrokecolor{currentstroke}%
\pgfsetdash{}{0pt}%
\pgfpathmoveto{\pgfqpoint{0.726250in}{0.525000in}}%
\pgfpathlineto{\pgfqpoint{4.046250in}{0.525000in}}%
\pgfusepath{stroke}%
\end{pgfscope}%
\begin{pgfscope}%
\pgfsetrectcap%
\pgfsetmiterjoin%
\pgfsetlinewidth{1.003750pt}%
\definecolor{currentstroke}{rgb}{0.000000,0.000000,0.000000}%
\pgfsetstrokecolor{currentstroke}%
\pgfsetdash{}{0pt}%
\pgfpathmoveto{\pgfqpoint{0.726250in}{3.412500in}}%
\pgfpathlineto{\pgfqpoint{4.046250in}{3.412500in}}%
\pgfusepath{stroke}%
\end{pgfscope}%
\begin{pgfscope}%
\definecolor{textcolor}{rgb}{0.000000,0.000000,0.000000}%
\pgfsetstrokecolor{textcolor}%
\pgfsetfillcolor{textcolor}%
\pgftext[x=1.146779in,y=0.730672in,,]{\color{textcolor}\rmfamily\fontsize{9.000000}{10.800000}\selectfont \(\displaystyle {}_{1}^{2}\mathrm{H}\)}%
\end{pgfscope}%
\begin{pgfscope}%
\definecolor{textcolor}{rgb}{0.000000,0.000000,0.000000}%
\pgfsetstrokecolor{textcolor}%
\pgfsetfillcolor{textcolor}%
\pgftext[x=1.561211in,y=1.267861in,,]{\color{textcolor}\rmfamily\fontsize{9.000000}{10.800000}\selectfont \(\displaystyle {}_{2}^{3}\mathrm{He}\)}%
\end{pgfscope}%
\begin{pgfscope}%
\definecolor{textcolor}{rgb}{0.000000,0.000000,0.000000}%
\pgfsetstrokecolor{textcolor}%
\pgfsetfillcolor{textcolor}%
\pgftext[x=1.567308in,y=2.654218in,,]{\color{textcolor}\rmfamily\fontsize{9.000000}{10.800000}\selectfont \(\displaystyle {}_{2}^{4}\mathrm{He}\)}%
\end{pgfscope}%
\begin{pgfscope}%
\definecolor{textcolor}{rgb}{0.000000,0.000000,0.000000}%
\pgfsetstrokecolor{textcolor}%
\pgfsetfillcolor{textcolor}%
\pgftext[x=1.813302in,y=1.949211in,,]{\color{textcolor}\rmfamily\fontsize{9.000000}{10.800000}\selectfont \(\displaystyle {}_{3}^{6}\mathrm{Li}\)}%
\end{pgfscope}%
\begin{pgfscope}%
\definecolor{textcolor}{rgb}{0.000000,0.000000,0.000000}%
\pgfsetstrokecolor{textcolor}%
\pgfsetfillcolor{textcolor}%
\pgftext[x=2.075262in,y=2.143860in,,]{\color{textcolor}\rmfamily\fontsize{9.000000}{10.800000}\selectfont \(\displaystyle {}_{3}^{7}\mathrm{Li}\)}%
\end{pgfscope}%
\begin{pgfscope}%
\definecolor{textcolor}{rgb}{0.000000,0.000000,0.000000}%
\pgfsetstrokecolor{textcolor}%
\pgfsetfillcolor{textcolor}%
\pgftext[x=2.194676in,y=2.290033in,,]{\color{textcolor}\rmfamily\fontsize{9.000000}{10.800000}\selectfont \(\displaystyle {}_{4}^{9}\mathrm{Be}\)}%
\end{pgfscope}%
\begin{pgfscope}%
\definecolor{textcolor}{rgb}{0.000000,0.000000,0.000000}%
\pgfsetstrokecolor{textcolor}%
\pgfsetfillcolor{textcolor}%
\pgftext[x=2.349479in,y=2.525383in,,]{\color{textcolor}\rmfamily\fontsize{9.000000}{10.800000}\selectfont \(\displaystyle {}_{5}^{11}\mathrm{B}\)}%
\end{pgfscope}%
\begin{pgfscope}%
\definecolor{textcolor}{rgb}{0.000000,0.000000,0.000000}%
\pgfsetstrokecolor{textcolor}%
\pgfsetfillcolor{textcolor}%
\pgftext[x=2.098451in,y=2.829267in,,]{\color{textcolor}\rmfamily\fontsize{9.000000}{10.800000}\selectfont \(\displaystyle {}_{6}^{12}\mathrm{C}\)}%
\end{pgfscope}%
\begin{pgfscope}%
\definecolor{textcolor}{rgb}{0.000000,0.000000,0.000000}%
\pgfsetstrokecolor{textcolor}%
\pgfsetfillcolor{textcolor}%
\pgftext[x=2.408366in,y=2.914755in,,]{\color{textcolor}\rmfamily\fontsize{9.000000}{10.800000}\selectfont \(\displaystyle {}_{8}^{16}\mathrm{O}\)}%
\end{pgfscope}%
\begin{pgfscope}%
\definecolor{textcolor}{rgb}{0.000000,0.000000,0.000000}%
\pgfsetstrokecolor{textcolor}%
\pgfsetfillcolor{textcolor}%
\pgftext[x=3.168411in,y=3.207590in,,]{\color{textcolor}\rmfamily\fontsize{9.000000}{10.800000}\selectfont \(\displaystyle {}_{26}^{56}\mathrm{Fe}\)}%
\end{pgfscope}%
\end{pgfpicture}%
\makeatother%
\endgroup%

  \caption{Caption.}
  \label{fig:nucleon_binding_energy}
\end{figure}

The fusion of light nuclei into heavier ones, at least when the atomic number is
not too large, is energetically favored because the mass of the product of the fusion
is generally smaller than the sum of the masses of the reagents, and the mass
defect is available to create secondary particles, such as the positron and the
neutrino in the previous case. This basic notion is customarily expressed in
quantitative terms by the \emph{average binding energy per nucleon} as a function
of the mass number
\begin{align*}
  \ave{E_B} = \frac{(m_p Z + m_n(A - Z) - m_N)c^2}{A}.
\end{align*}
Generally speaking, the average binding energy per nucleon increases up to the iron,
which is why fusion is an efficient mechanism to generate heavy nuclei only up
to this element. The deuton binding energy is about 2.2~MeV (or 1.1~MeV per nucleon),
while for ${^4}He$ the binding energy is roughly 28.3~MeV, or 7.1~MeV per nucleon.
For most of the elements, the average binding energy per nucleon is comprised
between 8 and 9~MeV.

Now, in order for two nuclei to be able to undergo nuclear fusion---despite the
fact that the latter is energetically favored---they have to first overcome the
Coulomb potential barrier and get close enough to each other for the (short-range)
strong force to take over. This happens at a distance $r \leq r_0 \sim 1$~fm
of the order of the proton radius. We can model the problem in a very crude way
by assuming that the nucleus-nucleus interaction potential is given by the Coulomb
potential at large distances, and collapses into a well at $r_0 \sim A^\frac{1}{3}$~fm,
that is, we can write the potential energy as\sidenote{Note we are neglecting the
angular momentum, which is legitimate here, as we are interested in nuclear interactions
with a small impact parameter (otherwise the two nuclei would not get close enough
to each other), and the $s$-wave contribution is the only important term in the
partial-wave expansion.}
\begin{align*}
  U(r) =
  \begin{cases}
    \displaystyle\frac{Z_1 Z_2 e^2}{r} & r > r_0\\
    -U_0 & r \leq r_0.
  \end{cases}
  \quad\text{and}\quad
  U_\text{max} = U(r_0) = \frac{Z_1 Z_2 e^2}{r_0}
\end{align*}

\begin{marginfigure}
  %% Creator: Matplotlib, PGF backend
%%
%% To include the figure in your LaTeX document, write
%%   \input{<filename>.pgf}
%%
%% Make sure the required packages are loaded in your preamble
%%   \usepackage{pgf}
%%
%% Also ensure that all the required font packages are loaded; for instance,
%% the lmodern package is sometimes necessary when using math font.
%%   \usepackage{lmodern}
%%
%% Figures using additional raster images can only be included by \input if
%% they are in the same directory as the main LaTeX file. For loading figures
%% from other directories you can use the `import` package
%%   \usepackage{import}
%%
%% and then include the figures with
%%   \import{<path to file>}{<filename>.pgf}
%%
%% Matplotlib used the following preamble
%%   \usepackage{fontspec}
%%   \setmainfont{DejaVuSerif.ttf}[Path=\detokenize{/usr/share/matplotlib/mpl-data/fonts/ttf/}]
%%   \setsansfont{DejaVuSans.ttf}[Path=\detokenize{/usr/share/matplotlib/mpl-data/fonts/ttf/}]
%%   \setmonofont{DejaVuSansMono.ttf}[Path=\detokenize{/usr/share/matplotlib/mpl-data/fonts/ttf/}]
%%
\begingroup%
\makeatletter%
\begin{pgfpicture}%
\pgfpathrectangle{\pgfpointorigin}{\pgfqpoint{1.950000in}{2.250000in}}%
\pgfusepath{use as bounding box, clip}%
\begin{pgfscope}%
\pgfsetbuttcap%
\pgfsetmiterjoin%
\definecolor{currentfill}{rgb}{1.000000,1.000000,1.000000}%
\pgfsetfillcolor{currentfill}%
\pgfsetlinewidth{0.000000pt}%
\definecolor{currentstroke}{rgb}{1.000000,1.000000,1.000000}%
\pgfsetstrokecolor{currentstroke}%
\pgfsetdash{}{0pt}%
\pgfpathmoveto{\pgfqpoint{0.000000in}{0.000000in}}%
\pgfpathlineto{\pgfqpoint{1.950000in}{0.000000in}}%
\pgfpathlineto{\pgfqpoint{1.950000in}{2.250000in}}%
\pgfpathlineto{\pgfqpoint{0.000000in}{2.250000in}}%
\pgfpathlineto{\pgfqpoint{0.000000in}{0.000000in}}%
\pgfpathclose%
\pgfusepath{fill}%
\end{pgfscope}%
\begin{pgfscope}%
\pgfsetbuttcap%
\pgfsetmiterjoin%
\definecolor{currentfill}{rgb}{1.000000,1.000000,1.000000}%
\pgfsetfillcolor{currentfill}%
\pgfsetlinewidth{0.000000pt}%
\definecolor{currentstroke}{rgb}{0.000000,0.000000,0.000000}%
\pgfsetstrokecolor{currentstroke}%
\pgfsetstrokeopacity{0.000000}%
\pgfsetdash{}{0pt}%
\pgfpathmoveto{\pgfqpoint{0.726250in}{0.525000in}}%
\pgfpathlineto{\pgfqpoint{1.846250in}{0.525000in}}%
\pgfpathlineto{\pgfqpoint{1.846250in}{2.162500in}}%
\pgfpathlineto{\pgfqpoint{0.726250in}{2.162500in}}%
\pgfpathlineto{\pgfqpoint{0.726250in}{0.525000in}}%
\pgfpathclose%
\pgfusepath{fill}%
\end{pgfscope}%
\begin{pgfscope}%
\pgfpathrectangle{\pgfqpoint{0.726250in}{0.525000in}}{\pgfqpoint{1.120000in}{1.637500in}}%
\pgfusepath{clip}%
\pgfsetbuttcap%
\pgfsetroundjoin%
\pgfsetlinewidth{0.803000pt}%
\definecolor{currentstroke}{rgb}{0.752941,0.752941,0.752941}%
\pgfsetstrokecolor{currentstroke}%
\pgfsetdash{{2.960000pt}{1.280000pt}}{0.000000pt}%
\pgfpathmoveto{\pgfqpoint{0.950250in}{0.525000in}}%
\pgfpathlineto{\pgfqpoint{0.950250in}{2.162500in}}%
\pgfusepath{stroke}%
\end{pgfscope}%
\begin{pgfscope}%
\pgfsetbuttcap%
\pgfsetroundjoin%
\definecolor{currentfill}{rgb}{0.000000,0.000000,0.000000}%
\pgfsetfillcolor{currentfill}%
\pgfsetlinewidth{0.803000pt}%
\definecolor{currentstroke}{rgb}{0.000000,0.000000,0.000000}%
\pgfsetstrokecolor{currentstroke}%
\pgfsetdash{}{0pt}%
\pgfsys@defobject{currentmarker}{\pgfqpoint{0.000000in}{-0.048611in}}{\pgfqpoint{0.000000in}{0.000000in}}{%
\pgfpathmoveto{\pgfqpoint{0.000000in}{0.000000in}}%
\pgfpathlineto{\pgfqpoint{0.000000in}{-0.048611in}}%
\pgfusepath{stroke,fill}%
}%
\begin{pgfscope}%
\pgfsys@transformshift{0.950250in}{0.525000in}%
\pgfsys@useobject{currentmarker}{}%
\end{pgfscope}%
\end{pgfscope}%
\begin{pgfscope}%
\definecolor{textcolor}{rgb}{0.000000,0.000000,0.000000}%
\pgfsetstrokecolor{textcolor}%
\pgfsetfillcolor{textcolor}%
\pgftext[x=0.950250in,y=0.427778in,,top]{\color{textcolor}\rmfamily\fontsize{9.000000}{10.800000}\selectfont \(\displaystyle r_0\)}%
\end{pgfscope}%
\begin{pgfscope}%
\definecolor{textcolor}{rgb}{0.000000,0.000000,0.000000}%
\pgfsetstrokecolor{textcolor}%
\pgfsetfillcolor{textcolor}%
\pgftext[x=1.286250in,y=0.251251in,,top]{\color{textcolor}\rmfamily\fontsize{9.000000}{10.800000}\selectfont \(\displaystyle r\) [a.u]}%
\end{pgfscope}%
\begin{pgfscope}%
\pgfpathrectangle{\pgfqpoint{0.726250in}{0.525000in}}{\pgfqpoint{1.120000in}{1.637500in}}%
\pgfusepath{clip}%
\pgfsetbuttcap%
\pgfsetroundjoin%
\pgfsetlinewidth{0.803000pt}%
\definecolor{currentstroke}{rgb}{0.752941,0.752941,0.752941}%
\pgfsetstrokecolor{currentstroke}%
\pgfsetdash{{2.960000pt}{1.280000pt}}{0.000000pt}%
\pgfpathmoveto{\pgfqpoint{0.726250in}{0.599432in}}%
\pgfpathlineto{\pgfqpoint{1.846250in}{0.599432in}}%
\pgfusepath{stroke}%
\end{pgfscope}%
\begin{pgfscope}%
\pgfsetbuttcap%
\pgfsetroundjoin%
\definecolor{currentfill}{rgb}{0.000000,0.000000,0.000000}%
\pgfsetfillcolor{currentfill}%
\pgfsetlinewidth{0.803000pt}%
\definecolor{currentstroke}{rgb}{0.000000,0.000000,0.000000}%
\pgfsetstrokecolor{currentstroke}%
\pgfsetdash{}{0pt}%
\pgfsys@defobject{currentmarker}{\pgfqpoint{-0.048611in}{0.000000in}}{\pgfqpoint{-0.000000in}{0.000000in}}{%
\pgfpathmoveto{\pgfqpoint{-0.000000in}{0.000000in}}%
\pgfpathlineto{\pgfqpoint{-0.048611in}{0.000000in}}%
\pgfusepath{stroke,fill}%
}%
\begin{pgfscope}%
\pgfsys@transformshift{0.726250in}{0.599432in}%
\pgfsys@useobject{currentmarker}{}%
\end{pgfscope}%
\end{pgfscope}%
\begin{pgfscope}%
\definecolor{textcolor}{rgb}{0.000000,0.000000,0.000000}%
\pgfsetstrokecolor{textcolor}%
\pgfsetfillcolor{textcolor}%
\pgftext[x=0.383854in, y=0.551946in, left, base]{\color{textcolor}\rmfamily\fontsize{9.000000}{10.800000}\selectfont \(\displaystyle -U_0\)}%
\end{pgfscope}%
\begin{pgfscope}%
\pgfpathrectangle{\pgfqpoint{0.726250in}{0.525000in}}{\pgfqpoint{1.120000in}{1.637500in}}%
\pgfusepath{clip}%
\pgfsetbuttcap%
\pgfsetroundjoin%
\pgfsetlinewidth{0.803000pt}%
\definecolor{currentstroke}{rgb}{0.752941,0.752941,0.752941}%
\pgfsetstrokecolor{currentstroke}%
\pgfsetdash{{2.960000pt}{1.280000pt}}{0.000000pt}%
\pgfpathmoveto{\pgfqpoint{0.726250in}{1.095644in}}%
\pgfpathlineto{\pgfqpoint{1.846250in}{1.095644in}}%
\pgfusepath{stroke}%
\end{pgfscope}%
\begin{pgfscope}%
\pgfsetbuttcap%
\pgfsetroundjoin%
\definecolor{currentfill}{rgb}{0.000000,0.000000,0.000000}%
\pgfsetfillcolor{currentfill}%
\pgfsetlinewidth{0.803000pt}%
\definecolor{currentstroke}{rgb}{0.000000,0.000000,0.000000}%
\pgfsetstrokecolor{currentstroke}%
\pgfsetdash{}{0pt}%
\pgfsys@defobject{currentmarker}{\pgfqpoint{-0.048611in}{0.000000in}}{\pgfqpoint{-0.000000in}{0.000000in}}{%
\pgfpathmoveto{\pgfqpoint{-0.000000in}{0.000000in}}%
\pgfpathlineto{\pgfqpoint{-0.048611in}{0.000000in}}%
\pgfusepath{stroke,fill}%
}%
\begin{pgfscope}%
\pgfsys@transformshift{0.726250in}{1.095644in}%
\pgfsys@useobject{currentmarker}{}%
\end{pgfscope}%
\end{pgfscope}%
\begin{pgfscope}%
\definecolor{textcolor}{rgb}{0.000000,0.000000,0.000000}%
\pgfsetstrokecolor{textcolor}%
\pgfsetfillcolor{textcolor}%
\pgftext[x=0.549499in, y=1.048159in, left, base]{\color{textcolor}\rmfamily\fontsize{9.000000}{10.800000}\selectfont 0}%
\end{pgfscope}%
\begin{pgfscope}%
\pgfpathrectangle{\pgfqpoint{0.726250in}{0.525000in}}{\pgfqpoint{1.120000in}{1.637500in}}%
\pgfusepath{clip}%
\pgfsetbuttcap%
\pgfsetroundjoin%
\pgfsetlinewidth{0.803000pt}%
\definecolor{currentstroke}{rgb}{0.752941,0.752941,0.752941}%
\pgfsetstrokecolor{currentstroke}%
\pgfsetdash{{2.960000pt}{1.280000pt}}{0.000000pt}%
\pgfpathmoveto{\pgfqpoint{0.726250in}{2.088068in}}%
\pgfpathlineto{\pgfqpoint{1.846250in}{2.088068in}}%
\pgfusepath{stroke}%
\end{pgfscope}%
\begin{pgfscope}%
\pgfsetbuttcap%
\pgfsetroundjoin%
\definecolor{currentfill}{rgb}{0.000000,0.000000,0.000000}%
\pgfsetfillcolor{currentfill}%
\pgfsetlinewidth{0.803000pt}%
\definecolor{currentstroke}{rgb}{0.000000,0.000000,0.000000}%
\pgfsetstrokecolor{currentstroke}%
\pgfsetdash{}{0pt}%
\pgfsys@defobject{currentmarker}{\pgfqpoint{-0.048611in}{0.000000in}}{\pgfqpoint{-0.000000in}{0.000000in}}{%
\pgfpathmoveto{\pgfqpoint{-0.000000in}{0.000000in}}%
\pgfpathlineto{\pgfqpoint{-0.048611in}{0.000000in}}%
\pgfusepath{stroke,fill}%
}%
\begin{pgfscope}%
\pgfsys@transformshift{0.726250in}{2.088068in}%
\pgfsys@useobject{currentmarker}{}%
\end{pgfscope}%
\end{pgfscope}%
\begin{pgfscope}%
\definecolor{textcolor}{rgb}{0.000000,0.000000,0.000000}%
\pgfsetstrokecolor{textcolor}%
\pgfsetfillcolor{textcolor}%
\pgftext[x=0.359002in, y=2.040583in, left, base]{\color{textcolor}\rmfamily\fontsize{9.000000}{10.800000}\selectfont \(\displaystyle U_\mathrm{max}\)}%
\end{pgfscope}%
\begin{pgfscope}%
\definecolor{textcolor}{rgb}{0.000000,0.000000,0.000000}%
\pgfsetstrokecolor{textcolor}%
\pgfsetfillcolor{textcolor}%
\pgftext[x=0.303446in,y=1.343750in,,bottom,rotate=90.000000]{\color{textcolor}\rmfamily\fontsize{9.000000}{10.800000}\selectfont \(\displaystyle U(r)\) [a. u.]}%
\end{pgfscope}%
\begin{pgfscope}%
\pgfpathrectangle{\pgfqpoint{0.726250in}{0.525000in}}{\pgfqpoint{1.120000in}{1.637500in}}%
\pgfusepath{clip}%
\pgfsetrectcap%
\pgfsetroundjoin%
\pgfsetlinewidth{1.003750pt}%
\definecolor{currentstroke}{rgb}{0.000000,0.000000,0.000000}%
\pgfsetstrokecolor{currentstroke}%
\pgfsetdash{}{0pt}%
\pgfpathmoveto{\pgfqpoint{0.950250in}{2.088068in}}%
\pgfpathlineto{\pgfqpoint{0.959301in}{2.049527in}}%
\pgfpathlineto{\pgfqpoint{0.968351in}{2.013868in}}%
\pgfpathlineto{\pgfqpoint{0.977402in}{1.980779in}}%
\pgfpathlineto{\pgfqpoint{0.986452in}{1.949992in}}%
\pgfpathlineto{\pgfqpoint{0.995503in}{1.921274in}}%
\pgfpathlineto{\pgfqpoint{1.004553in}{1.894424in}}%
\pgfpathlineto{\pgfqpoint{1.013604in}{1.869266in}}%
\pgfpathlineto{\pgfqpoint{1.022654in}{1.845644in}}%
\pgfpathlineto{\pgfqpoint{1.031705in}{1.823422in}}%
\pgfpathlineto{\pgfqpoint{1.040755in}{1.802478in}}%
\pgfpathlineto{\pgfqpoint{1.049806in}{1.782707in}}%
\pgfpathlineto{\pgfqpoint{1.058856in}{1.764011in}}%
\pgfpathlineto{\pgfqpoint{1.067907in}{1.746306in}}%
\pgfpathlineto{\pgfqpoint{1.076957in}{1.729515in}}%
\pgfpathlineto{\pgfqpoint{1.086008in}{1.713568in}}%
\pgfpathlineto{\pgfqpoint{1.095058in}{1.698405in}}%
\pgfpathlineto{\pgfqpoint{1.104109in}{1.683967in}}%
\pgfpathlineto{\pgfqpoint{1.113159in}{1.670205in}}%
\pgfpathlineto{\pgfqpoint{1.122210in}{1.657073in}}%
\pgfpathlineto{\pgfqpoint{1.131260in}{1.644527in}}%
\pgfpathlineto{\pgfqpoint{1.140311in}{1.632529in}}%
\pgfpathlineto{\pgfqpoint{1.149361in}{1.621045in}}%
\pgfpathlineto{\pgfqpoint{1.158412in}{1.610042in}}%
\pgfpathlineto{\pgfqpoint{1.167462in}{1.599490in}}%
\pgfpathlineto{\pgfqpoint{1.176513in}{1.589363in}}%
\pgfpathlineto{\pgfqpoint{1.185563in}{1.579634in}}%
\pgfpathlineto{\pgfqpoint{1.194614in}{1.570282in}}%
\pgfpathlineto{\pgfqpoint{1.203664in}{1.561284in}}%
\pgfpathlineto{\pgfqpoint{1.212715in}{1.552621in}}%
\pgfpathlineto{\pgfqpoint{1.221765in}{1.544274in}}%
\pgfpathlineto{\pgfqpoint{1.230816in}{1.536227in}}%
\pgfpathlineto{\pgfqpoint{1.239866in}{1.528463in}}%
\pgfpathlineto{\pgfqpoint{1.248917in}{1.520969in}}%
\pgfpathlineto{\pgfqpoint{1.257967in}{1.513729in}}%
\pgfpathlineto{\pgfqpoint{1.267018in}{1.506732in}}%
\pgfpathlineto{\pgfqpoint{1.276068in}{1.499965in}}%
\pgfpathlineto{\pgfqpoint{1.285119in}{1.493417in}}%
\pgfpathlineto{\pgfqpoint{1.294169in}{1.487078in}}%
\pgfpathlineto{\pgfqpoint{1.303220in}{1.480938in}}%
\pgfpathlineto{\pgfqpoint{1.312270in}{1.474988in}}%
\pgfpathlineto{\pgfqpoint{1.321321in}{1.469218in}}%
\pgfpathlineto{\pgfqpoint{1.330371in}{1.463621in}}%
\pgfpathlineto{\pgfqpoint{1.339422in}{1.458190in}}%
\pgfpathlineto{\pgfqpoint{1.348472in}{1.452917in}}%
\pgfpathlineto{\pgfqpoint{1.357523in}{1.447794in}}%
\pgfpathlineto{\pgfqpoint{1.366573in}{1.442817in}}%
\pgfpathlineto{\pgfqpoint{1.375624in}{1.437978in}}%
\pgfpathlineto{\pgfqpoint{1.384674in}{1.433273in}}%
\pgfpathlineto{\pgfqpoint{1.393725in}{1.428695in}}%
\pgfpathlineto{\pgfqpoint{1.402775in}{1.424239in}}%
\pgfpathlineto{\pgfqpoint{1.411826in}{1.419901in}}%
\pgfpathlineto{\pgfqpoint{1.420876in}{1.415677in}}%
\pgfpathlineto{\pgfqpoint{1.429927in}{1.411560in}}%
\pgfpathlineto{\pgfqpoint{1.438977in}{1.407549in}}%
\pgfpathlineto{\pgfqpoint{1.448028in}{1.403638in}}%
\pgfpathlineto{\pgfqpoint{1.457078in}{1.399824in}}%
\pgfpathlineto{\pgfqpoint{1.466129in}{1.396103in}}%
\pgfpathlineto{\pgfqpoint{1.475179in}{1.392472in}}%
\pgfpathlineto{\pgfqpoint{1.484230in}{1.388928in}}%
\pgfpathlineto{\pgfqpoint{1.493280in}{1.385467in}}%
\pgfpathlineto{\pgfqpoint{1.502331in}{1.382087in}}%
\pgfpathlineto{\pgfqpoint{1.511381in}{1.378785in}}%
\pgfpathlineto{\pgfqpoint{1.520432in}{1.375558in}}%
\pgfpathlineto{\pgfqpoint{1.529482in}{1.372405in}}%
\pgfpathlineto{\pgfqpoint{1.538533in}{1.369321in}}%
\pgfpathlineto{\pgfqpoint{1.547583in}{1.366305in}}%
\pgfpathlineto{\pgfqpoint{1.556634in}{1.363355in}}%
\pgfpathlineto{\pgfqpoint{1.565684in}{1.360469in}}%
\pgfpathlineto{\pgfqpoint{1.574735in}{1.357644in}}%
\pgfpathlineto{\pgfqpoint{1.583785in}{1.354879in}}%
\pgfpathlineto{\pgfqpoint{1.592836in}{1.352171in}}%
\pgfpathlineto{\pgfqpoint{1.601886in}{1.349520in}}%
\pgfpathlineto{\pgfqpoint{1.610937in}{1.346923in}}%
\pgfpathlineto{\pgfqpoint{1.619987in}{1.344378in}}%
\pgfpathlineto{\pgfqpoint{1.629038in}{1.341885in}}%
\pgfpathlineto{\pgfqpoint{1.638088in}{1.339440in}}%
\pgfpathlineto{\pgfqpoint{1.647139in}{1.337044in}}%
\pgfpathlineto{\pgfqpoint{1.656189in}{1.334695in}}%
\pgfpathlineto{\pgfqpoint{1.665240in}{1.332391in}}%
\pgfpathlineto{\pgfqpoint{1.674290in}{1.330131in}}%
\pgfpathlineto{\pgfqpoint{1.683341in}{1.327913in}}%
\pgfpathlineto{\pgfqpoint{1.692391in}{1.325738in}}%
\pgfpathlineto{\pgfqpoint{1.701442in}{1.323602in}}%
\pgfpathlineto{\pgfqpoint{1.710492in}{1.321506in}}%
\pgfpathlineto{\pgfqpoint{1.719543in}{1.319448in}}%
\pgfpathlineto{\pgfqpoint{1.728593in}{1.317427in}}%
\pgfpathlineto{\pgfqpoint{1.737644in}{1.315443in}}%
\pgfpathlineto{\pgfqpoint{1.746694in}{1.313493in}}%
\pgfpathlineto{\pgfqpoint{1.755745in}{1.311578in}}%
\pgfpathlineto{\pgfqpoint{1.764795in}{1.309696in}}%
\pgfpathlineto{\pgfqpoint{1.773846in}{1.307847in}}%
\pgfpathlineto{\pgfqpoint{1.782896in}{1.306029in}}%
\pgfpathlineto{\pgfqpoint{1.791947in}{1.304243in}}%
\pgfpathlineto{\pgfqpoint{1.800997in}{1.302486in}}%
\pgfpathlineto{\pgfqpoint{1.810048in}{1.300759in}}%
\pgfpathlineto{\pgfqpoint{1.819098in}{1.299060in}}%
\pgfpathlineto{\pgfqpoint{1.828149in}{1.297389in}}%
\pgfpathlineto{\pgfqpoint{1.837199in}{1.295746in}}%
\pgfpathlineto{\pgfqpoint{1.846250in}{1.294129in}}%
\pgfusepath{stroke}%
\end{pgfscope}%
\begin{pgfscope}%
\pgfpathrectangle{\pgfqpoint{0.726250in}{0.525000in}}{\pgfqpoint{1.120000in}{1.637500in}}%
\pgfusepath{clip}%
\pgfsetbuttcap%
\pgfsetroundjoin%
\pgfsetlinewidth{1.003750pt}%
\definecolor{currentstroke}{rgb}{0.000000,0.000000,0.000000}%
\pgfsetstrokecolor{currentstroke}%
\pgfsetdash{}{0pt}%
\pgfpathmoveto{\pgfqpoint{0.726250in}{0.599432in}}%
\pgfpathlineto{\pgfqpoint{0.950250in}{0.599432in}}%
\pgfusepath{stroke}%
\end{pgfscope}%
\begin{pgfscope}%
\pgfpathrectangle{\pgfqpoint{0.726250in}{0.525000in}}{\pgfqpoint{1.120000in}{1.637500in}}%
\pgfusepath{clip}%
\pgfsetbuttcap%
\pgfsetroundjoin%
\pgfsetlinewidth{1.003750pt}%
\definecolor{currentstroke}{rgb}{0.000000,0.000000,0.000000}%
\pgfsetstrokecolor{currentstroke}%
\pgfsetdash{}{0pt}%
\pgfpathmoveto{\pgfqpoint{0.950250in}{0.599432in}}%
\pgfpathlineto{\pgfqpoint{0.950250in}{2.088068in}}%
\pgfusepath{stroke}%
\end{pgfscope}%
\begin{pgfscope}%
\pgfpathrectangle{\pgfqpoint{0.726250in}{0.525000in}}{\pgfqpoint{1.120000in}{1.637500in}}%
\pgfusepath{clip}%
\pgfsetbuttcap%
\pgfsetroundjoin%
\pgfsetlinewidth{1.003750pt}%
\definecolor{currentstroke}{rgb}{0.000000,0.000000,0.000000}%
\pgfsetstrokecolor{currentstroke}%
\pgfsetdash{{3.700000pt}{1.600000pt}}{0.000000pt}%
\pgfpathmoveto{\pgfqpoint{0.726250in}{1.492614in}}%
\pgfpathlineto{\pgfqpoint{1.846250in}{1.492614in}}%
\pgfusepath{stroke}%
\end{pgfscope}%
\begin{pgfscope}%
\pgfsetrectcap%
\pgfsetmiterjoin%
\pgfsetlinewidth{1.003750pt}%
\definecolor{currentstroke}{rgb}{0.000000,0.000000,0.000000}%
\pgfsetstrokecolor{currentstroke}%
\pgfsetdash{}{0pt}%
\pgfpathmoveto{\pgfqpoint{0.726250in}{0.525000in}}%
\pgfpathlineto{\pgfqpoint{0.726250in}{2.162500in}}%
\pgfusepath{stroke}%
\end{pgfscope}%
\begin{pgfscope}%
\pgfsetrectcap%
\pgfsetmiterjoin%
\pgfsetlinewidth{1.003750pt}%
\definecolor{currentstroke}{rgb}{0.000000,0.000000,0.000000}%
\pgfsetstrokecolor{currentstroke}%
\pgfsetdash{}{0pt}%
\pgfpathmoveto{\pgfqpoint{1.846250in}{0.525000in}}%
\pgfpathlineto{\pgfqpoint{1.846250in}{2.162500in}}%
\pgfusepath{stroke}%
\end{pgfscope}%
\begin{pgfscope}%
\pgfsetrectcap%
\pgfsetmiterjoin%
\pgfsetlinewidth{1.003750pt}%
\definecolor{currentstroke}{rgb}{0.000000,0.000000,0.000000}%
\pgfsetstrokecolor{currentstroke}%
\pgfsetdash{}{0pt}%
\pgfpathmoveto{\pgfqpoint{0.726250in}{0.525000in}}%
\pgfpathlineto{\pgfqpoint{1.846250in}{0.525000in}}%
\pgfusepath{stroke}%
\end{pgfscope}%
\begin{pgfscope}%
\pgfsetrectcap%
\pgfsetmiterjoin%
\pgfsetlinewidth{1.003750pt}%
\definecolor{currentstroke}{rgb}{0.000000,0.000000,0.000000}%
\pgfsetstrokecolor{currentstroke}%
\pgfsetdash{}{0pt}%
\pgfpathmoveto{\pgfqpoint{0.726250in}{2.162500in}}%
\pgfpathlineto{\pgfqpoint{1.846250in}{2.162500in}}%
\pgfusepath{stroke}%
\end{pgfscope}%
\begin{pgfscope}%
\definecolor{textcolor}{rgb}{0.000000,0.000000,0.000000}%
\pgfsetstrokecolor{textcolor}%
\pgfsetfillcolor{textcolor}%
\pgftext[x=1.398250in,y=1.512462in,left,base]{\color{textcolor}\rmfamily\fontsize{9.000000}{10.800000}\selectfont \(\displaystyle E\)}%
\end{pgfscope}%
\end{pgfpicture}%
\makeatother%
\endgroup%

  \caption{Schematic representation of the central potential describing the process
  of nuclear fusion process fusion as a well and a Coulomb barrier.}
  \label{fig:pp_fusion}
\end{marginfigure}

In a classical description, a nucleus is simply forbidden from entering the potential
well (and undergo the fusion process) unless its kinetic energy is larger than $U_\text{max}$.
If we plug in the numbers, this turns out to be about $1.44$~MeV for proton-proton
fusion, and larger for heavier nuclei. The question is now: what are the chances
for a nucleus (e.g., a proton) in the core of the Sun to have a kinetic energy in
excess this value?

If we model the core of the Sun as a non-relativistic perfect gas (or, more
precisely, as a plasma of protons and electrons, since the atoms are completely
ionized), the probability density function for the proton velocity is given by the
Maxwell-Boltzmann distribution
\begin{align}
  p_v(v) = \qty(\frac{m}{2\pi kT})^\frac{3}{2} 4\pi v^2 e^{-\frac{mv^2}{2kT}}
\end{align}
(with $kT \approx 1.5$~keV), which can be easily recasted in energy space by a simple
change of variables
\begin{align}\label{eq:maxwell_boltzmann_dist_energy}
  p_E(E) = \frac{2}{\sqrt{\pi}} \qty(\frac{1}{kT})^\frac{3}{2} \sqrt{E} e^{-\frac{E}{kT}}.
\end{align}

We are now facing a distinct problem: we have a potential barrier to cross for the fusion
to take place that is of the order of $U_\text{max} \approx 1.5$~MeV (or more, if we
consider heavier nuclei), but the protons at the center of the Sun only have $kT \approx 1.5$~keV
of kinetic energy available on average. Formally the probability for any given proton
to have enough energy to cross the barrier is calculated by integrating the Maxwell-Boltzmann
distribution~\eqref{eq:maxwell_boltzmann_dist_energy} from $x_0 = \nicefrac{U_\text{max}}{kT}$
to infinity
\begin{align*}
  P_\text{cross} = \frac{2}{\sqrt{\pi}} \int_{x_0}^\infty \sqrt{x} e^{-x} \diff{x} =
  \text{erfc}\qty(\sqrt{x_0}) + 2 \sqrt{\frac{x_0}{\pi}} e^{-x_0} \approx
  2 \sqrt{\frac{x_0}{\pi}} e^{-x_0}
\end{align*}
(the last approximation holds for large $x_0$), and for $x_0 = 1000$ the crossing
it is of the order of $10^{-436}$ which means that, even considering that the number of
protons in the Sun $N_p \approx N_A M_\odot~\text{[g]} \approx 1.2 \times 10^{57}$,
the process is classically forbidden.

Here is where quantum mechanics comes to rescue, as even if the crossing is classically
forbidden, there is potentially a non-zero probability of quantum tunneling. The
latter can be calculated in semi-classical (or WKB) approximation as
\begin{align*}
  P_\text{cross} = \exp\qty{-2\int_{r_0}^{r_1} \sqrt{\frac{2m_r}{\hbar^2} (U(r) - E)} \diff{r}},
\end{align*}
where $m_r$ is the reduced mass of the two particle (i.e., $\nicefrac{m_p}{2}$ for
a proton-proton fusion) and $r_1$ is defined by the condition\sidenote{It is important
to note that figure~\ref{fig:pp_fusion} is not quite to scale: since the kinetic
energy $E$ of the nucleus is typically much smaller than the height of the Coulomb
barrier $U_\text{max}$, we have $r_0 \ll r_1$ or, equivalently, $\nicefrac{r_0}{r_1} \ll 1$.}
$U(r_1) = E$. We can then rewrite our probability as
\begin{align*}
  P_\text{cross} = \exp\qty{-2 \sqrt{\frac{2m_r Z_1 Z_2 e^2}{\hbar^2}}
  \int_{r_0}^{r_1} \sqrt{\frac{1}{r} - \frac{1}{r_1}} \diff{r}},
\end{align*}
which, by virtue of the change of variable $x = \nicefrac{r}{r_1}$, becomes
\begin{align*}
  \int_{r_0}^{r_1} \sqrt{\frac{1}{r} - \frac{1}{r_1}} \diff{r} =
  \frac{e}{\sqrt{E}} \int_\frac{r_0}{r_1}^{1} \sqrt{\frac{1}{x} - 1} \diff{x}
  \approx \frac{e}{\sqrt{E}} \int_0^1 \sqrt{\frac{1}{x} - 1} \diff{x} =
  \frac{\pi}{2} \frac{e}{\sqrt{E}}.
\end{align*}
(in the penultimate passage we have used the fact that $\nicefrac{r_0}{r_1}$ is
small to change the lower limit if the integration to zero). The crossing probability
can be estimated as
\begin{align}
  P_\text{cross} = \exp\qty{-\pi e^2 Z_1 Z_2 \sqrt{\frac{2m_r}{\hbar^2 E}}}
  \quad\text{or}\quad
  P_\text{cross} = e^{-\sqrt{\frac{E_G}{E}}},
\end{align}
where we have introduced the \emph{Gamow energy}
\begin{align}
  E_G = \frac{2m_r \pi^2 e^4 Z_1^2 Z_2^2}{\hbar^2},
\end{align}
which turns out to be of the order of $493$~keV for a proton-proton fusion. The corresponding
crossing probability, assuming an average proton energy of 1.5~keV, is now of the
order of $10^{-8}$---this is still small for a single proton, but when we fold in
the number of protons available in the Sun, this is more than enough for the process
to actually take place.

Although we will not go into all the gory details of a complete calculation of the
neutrino flux from nuclear fusion, we note the thermonuclear reaction rate is
customarily written as
\begin{align}
  N_A \ave{\sigma v} = N_A \qty(\frac{8}{\pi m_r})^\frac{1}{2} \qty(\frac{1}{kT})^{\frac{3}{2}}
  \int_0^\infty S(E)
  \overbrace{\exp\qty{-\qty(\sqrt{\frac{E_G}{E}} + \frac{E}{kT})}}^\text{Gamow peak} \diff{E},
\end{align}
where $S(E)$, also known as the $S$-\emph{factor} or the \emph{astrophysical factor}
is motivated by a common parametrization of the nuclear cross section\sidenote{More
precisely, the nuclear cross section is written as
\begin{align*}
  \sigma(E) = \frac{S(E)}{E} e^{-\sqrt{\frac{E_G}{E}}}
\end{align*}
and the energy at the denominator cancels with the product of the $\sqrt{E}$ in the
Maxwell-Boltzmann distribution times the relative velocity, which also scales as
$\sqrt{E}$. Away from resonances, the $S$-factor is only slowly dependent on the energy.}.
We note explicitely that one of the terms that go into the relevant integral is
the product of the exponential in the Maxwell-Boltzmann distribution and the tunneling
probability, which is generally referred to as the \emph{Gamow peak} and is shown
for reasonable values of the parameters in figure~\ref{fig:gamow_peak}.

\begin{marginfigure}
  %% Creator: Matplotlib, PGF backend
%%
%% To include the figure in your LaTeX document, write
%%   \input{<filename>.pgf}
%%
%% Make sure the required packages are loaded in your preamble
%%   \usepackage{pgf}
%%
%% Also ensure that all the required font packages are loaded; for instance,
%% the lmodern package is sometimes necessary when using math font.
%%   \usepackage{lmodern}
%%
%% Figures using additional raster images can only be included by \input if
%% they are in the same directory as the main LaTeX file. For loading figures
%% from other directories you can use the `import` package
%%   \usepackage{import}
%%
%% and then include the figures with
%%   \import{<path to file>}{<filename>.pgf}
%%
%% Matplotlib used the following preamble
%%   \usepackage{fontspec}
%%   \setmainfont{DejaVuSerif.ttf}[Path=\detokenize{/usr/share/matplotlib/mpl-data/fonts/ttf/}]
%%   \setsansfont{DejaVuSans.ttf}[Path=\detokenize{/usr/share/matplotlib/mpl-data/fonts/ttf/}]
%%   \setmonofont{DejaVuSansMono.ttf}[Path=\detokenize{/usr/share/matplotlib/mpl-data/fonts/ttf/}]
%%
\begingroup%
\makeatletter%
\begin{pgfpicture}%
\pgfpathrectangle{\pgfpointorigin}{\pgfqpoint{1.950000in}{2.250000in}}%
\pgfusepath{use as bounding box, clip}%
\begin{pgfscope}%
\pgfsetbuttcap%
\pgfsetmiterjoin%
\definecolor{currentfill}{rgb}{1.000000,1.000000,1.000000}%
\pgfsetfillcolor{currentfill}%
\pgfsetlinewidth{0.000000pt}%
\definecolor{currentstroke}{rgb}{1.000000,1.000000,1.000000}%
\pgfsetstrokecolor{currentstroke}%
\pgfsetdash{}{0pt}%
\pgfpathmoveto{\pgfqpoint{0.000000in}{0.000000in}}%
\pgfpathlineto{\pgfqpoint{1.950000in}{0.000000in}}%
\pgfpathlineto{\pgfqpoint{1.950000in}{2.250000in}}%
\pgfpathlineto{\pgfqpoint{0.000000in}{2.250000in}}%
\pgfpathlineto{\pgfqpoint{0.000000in}{0.000000in}}%
\pgfpathclose%
\pgfusepath{fill}%
\end{pgfscope}%
\begin{pgfscope}%
\pgfsetbuttcap%
\pgfsetmiterjoin%
\definecolor{currentfill}{rgb}{1.000000,1.000000,1.000000}%
\pgfsetfillcolor{currentfill}%
\pgfsetlinewidth{0.000000pt}%
\definecolor{currentstroke}{rgb}{0.000000,0.000000,0.000000}%
\pgfsetstrokecolor{currentstroke}%
\pgfsetstrokeopacity{0.000000}%
\pgfsetdash{}{0pt}%
\pgfpathmoveto{\pgfqpoint{0.726250in}{0.525000in}}%
\pgfpathlineto{\pgfqpoint{1.846250in}{0.525000in}}%
\pgfpathlineto{\pgfqpoint{1.846250in}{2.162500in}}%
\pgfpathlineto{\pgfqpoint{0.726250in}{2.162500in}}%
\pgfpathlineto{\pgfqpoint{0.726250in}{0.525000in}}%
\pgfpathclose%
\pgfusepath{fill}%
\end{pgfscope}%
\begin{pgfscope}%
\pgfpathrectangle{\pgfqpoint{0.726250in}{0.525000in}}{\pgfqpoint{1.120000in}{1.637500in}}%
\pgfusepath{clip}%
\pgfsetbuttcap%
\pgfsetroundjoin%
\pgfsetlinewidth{0.803000pt}%
\definecolor{currentstroke}{rgb}{0.752941,0.752941,0.752941}%
\pgfsetstrokecolor{currentstroke}%
\pgfsetdash{{2.960000pt}{1.280000pt}}{0.000000pt}%
\pgfpathmoveto{\pgfqpoint{0.726250in}{0.525000in}}%
\pgfpathlineto{\pgfqpoint{0.726250in}{2.162500in}}%
\pgfusepath{stroke}%
\end{pgfscope}%
\begin{pgfscope}%
\pgfsetbuttcap%
\pgfsetroundjoin%
\definecolor{currentfill}{rgb}{0.000000,0.000000,0.000000}%
\pgfsetfillcolor{currentfill}%
\pgfsetlinewidth{0.803000pt}%
\definecolor{currentstroke}{rgb}{0.000000,0.000000,0.000000}%
\pgfsetstrokecolor{currentstroke}%
\pgfsetdash{}{0pt}%
\pgfsys@defobject{currentmarker}{\pgfqpoint{0.000000in}{-0.048611in}}{\pgfqpoint{0.000000in}{0.000000in}}{%
\pgfpathmoveto{\pgfqpoint{0.000000in}{0.000000in}}%
\pgfpathlineto{\pgfqpoint{0.000000in}{-0.048611in}}%
\pgfusepath{stroke,fill}%
}%
\begin{pgfscope}%
\pgfsys@transformshift{0.726250in}{0.525000in}%
\pgfsys@useobject{currentmarker}{}%
\end{pgfscope}%
\end{pgfscope}%
\begin{pgfscope}%
\definecolor{textcolor}{rgb}{0.000000,0.000000,0.000000}%
\pgfsetstrokecolor{textcolor}%
\pgfsetfillcolor{textcolor}%
\pgftext[x=0.726250in,y=0.427778in,,top]{\color{textcolor}\rmfamily\fontsize{9.000000}{10.800000}\selectfont \(\displaystyle {10^{-1}}\)}%
\end{pgfscope}%
\begin{pgfscope}%
\pgfpathrectangle{\pgfqpoint{0.726250in}{0.525000in}}{\pgfqpoint{1.120000in}{1.637500in}}%
\pgfusepath{clip}%
\pgfsetbuttcap%
\pgfsetroundjoin%
\pgfsetlinewidth{0.803000pt}%
\definecolor{currentstroke}{rgb}{0.752941,0.752941,0.752941}%
\pgfsetstrokecolor{currentstroke}%
\pgfsetdash{{2.960000pt}{1.280000pt}}{0.000000pt}%
\pgfpathmoveto{\pgfqpoint{1.006250in}{0.525000in}}%
\pgfpathlineto{\pgfqpoint{1.006250in}{2.162500in}}%
\pgfusepath{stroke}%
\end{pgfscope}%
\begin{pgfscope}%
\pgfsetbuttcap%
\pgfsetroundjoin%
\definecolor{currentfill}{rgb}{0.000000,0.000000,0.000000}%
\pgfsetfillcolor{currentfill}%
\pgfsetlinewidth{0.803000pt}%
\definecolor{currentstroke}{rgb}{0.000000,0.000000,0.000000}%
\pgfsetstrokecolor{currentstroke}%
\pgfsetdash{}{0pt}%
\pgfsys@defobject{currentmarker}{\pgfqpoint{0.000000in}{-0.048611in}}{\pgfqpoint{0.000000in}{0.000000in}}{%
\pgfpathmoveto{\pgfqpoint{0.000000in}{0.000000in}}%
\pgfpathlineto{\pgfqpoint{0.000000in}{-0.048611in}}%
\pgfusepath{stroke,fill}%
}%
\begin{pgfscope}%
\pgfsys@transformshift{1.006250in}{0.525000in}%
\pgfsys@useobject{currentmarker}{}%
\end{pgfscope}%
\end{pgfscope}%
\begin{pgfscope}%
\definecolor{textcolor}{rgb}{0.000000,0.000000,0.000000}%
\pgfsetstrokecolor{textcolor}%
\pgfsetfillcolor{textcolor}%
\pgftext[x=1.006250in,y=0.427778in,,top]{\color{textcolor}\rmfamily\fontsize{9.000000}{10.800000}\selectfont \(\displaystyle {10^{0}}\)}%
\end{pgfscope}%
\begin{pgfscope}%
\pgfpathrectangle{\pgfqpoint{0.726250in}{0.525000in}}{\pgfqpoint{1.120000in}{1.637500in}}%
\pgfusepath{clip}%
\pgfsetbuttcap%
\pgfsetroundjoin%
\pgfsetlinewidth{0.803000pt}%
\definecolor{currentstroke}{rgb}{0.752941,0.752941,0.752941}%
\pgfsetstrokecolor{currentstroke}%
\pgfsetdash{{2.960000pt}{1.280000pt}}{0.000000pt}%
\pgfpathmoveto{\pgfqpoint{1.286250in}{0.525000in}}%
\pgfpathlineto{\pgfqpoint{1.286250in}{2.162500in}}%
\pgfusepath{stroke}%
\end{pgfscope}%
\begin{pgfscope}%
\pgfsetbuttcap%
\pgfsetroundjoin%
\definecolor{currentfill}{rgb}{0.000000,0.000000,0.000000}%
\pgfsetfillcolor{currentfill}%
\pgfsetlinewidth{0.803000pt}%
\definecolor{currentstroke}{rgb}{0.000000,0.000000,0.000000}%
\pgfsetstrokecolor{currentstroke}%
\pgfsetdash{}{0pt}%
\pgfsys@defobject{currentmarker}{\pgfqpoint{0.000000in}{-0.048611in}}{\pgfqpoint{0.000000in}{0.000000in}}{%
\pgfpathmoveto{\pgfqpoint{0.000000in}{0.000000in}}%
\pgfpathlineto{\pgfqpoint{0.000000in}{-0.048611in}}%
\pgfusepath{stroke,fill}%
}%
\begin{pgfscope}%
\pgfsys@transformshift{1.286250in}{0.525000in}%
\pgfsys@useobject{currentmarker}{}%
\end{pgfscope}%
\end{pgfscope}%
\begin{pgfscope}%
\definecolor{textcolor}{rgb}{0.000000,0.000000,0.000000}%
\pgfsetstrokecolor{textcolor}%
\pgfsetfillcolor{textcolor}%
\pgftext[x=1.286250in,y=0.427778in,,top]{\color{textcolor}\rmfamily\fontsize{9.000000}{10.800000}\selectfont \(\displaystyle {10^{1}}\)}%
\end{pgfscope}%
\begin{pgfscope}%
\pgfpathrectangle{\pgfqpoint{0.726250in}{0.525000in}}{\pgfqpoint{1.120000in}{1.637500in}}%
\pgfusepath{clip}%
\pgfsetbuttcap%
\pgfsetroundjoin%
\pgfsetlinewidth{0.803000pt}%
\definecolor{currentstroke}{rgb}{0.752941,0.752941,0.752941}%
\pgfsetstrokecolor{currentstroke}%
\pgfsetdash{{2.960000pt}{1.280000pt}}{0.000000pt}%
\pgfpathmoveto{\pgfqpoint{1.566250in}{0.525000in}}%
\pgfpathlineto{\pgfqpoint{1.566250in}{2.162500in}}%
\pgfusepath{stroke}%
\end{pgfscope}%
\begin{pgfscope}%
\pgfsetbuttcap%
\pgfsetroundjoin%
\definecolor{currentfill}{rgb}{0.000000,0.000000,0.000000}%
\pgfsetfillcolor{currentfill}%
\pgfsetlinewidth{0.803000pt}%
\definecolor{currentstroke}{rgb}{0.000000,0.000000,0.000000}%
\pgfsetstrokecolor{currentstroke}%
\pgfsetdash{}{0pt}%
\pgfsys@defobject{currentmarker}{\pgfqpoint{0.000000in}{-0.048611in}}{\pgfqpoint{0.000000in}{0.000000in}}{%
\pgfpathmoveto{\pgfqpoint{0.000000in}{0.000000in}}%
\pgfpathlineto{\pgfqpoint{0.000000in}{-0.048611in}}%
\pgfusepath{stroke,fill}%
}%
\begin{pgfscope}%
\pgfsys@transformshift{1.566250in}{0.525000in}%
\pgfsys@useobject{currentmarker}{}%
\end{pgfscope}%
\end{pgfscope}%
\begin{pgfscope}%
\definecolor{textcolor}{rgb}{0.000000,0.000000,0.000000}%
\pgfsetstrokecolor{textcolor}%
\pgfsetfillcolor{textcolor}%
\pgftext[x=1.566250in,y=0.427778in,,top]{\color{textcolor}\rmfamily\fontsize{9.000000}{10.800000}\selectfont \(\displaystyle {10^{2}}\)}%
\end{pgfscope}%
\begin{pgfscope}%
\pgfpathrectangle{\pgfqpoint{0.726250in}{0.525000in}}{\pgfqpoint{1.120000in}{1.637500in}}%
\pgfusepath{clip}%
\pgfsetbuttcap%
\pgfsetroundjoin%
\pgfsetlinewidth{0.803000pt}%
\definecolor{currentstroke}{rgb}{0.752941,0.752941,0.752941}%
\pgfsetstrokecolor{currentstroke}%
\pgfsetdash{{2.960000pt}{1.280000pt}}{0.000000pt}%
\pgfpathmoveto{\pgfqpoint{1.846250in}{0.525000in}}%
\pgfpathlineto{\pgfqpoint{1.846250in}{2.162500in}}%
\pgfusepath{stroke}%
\end{pgfscope}%
\begin{pgfscope}%
\pgfsetbuttcap%
\pgfsetroundjoin%
\definecolor{currentfill}{rgb}{0.000000,0.000000,0.000000}%
\pgfsetfillcolor{currentfill}%
\pgfsetlinewidth{0.803000pt}%
\definecolor{currentstroke}{rgb}{0.000000,0.000000,0.000000}%
\pgfsetstrokecolor{currentstroke}%
\pgfsetdash{}{0pt}%
\pgfsys@defobject{currentmarker}{\pgfqpoint{0.000000in}{-0.048611in}}{\pgfqpoint{0.000000in}{0.000000in}}{%
\pgfpathmoveto{\pgfqpoint{0.000000in}{0.000000in}}%
\pgfpathlineto{\pgfqpoint{0.000000in}{-0.048611in}}%
\pgfusepath{stroke,fill}%
}%
\begin{pgfscope}%
\pgfsys@transformshift{1.846250in}{0.525000in}%
\pgfsys@useobject{currentmarker}{}%
\end{pgfscope}%
\end{pgfscope}%
\begin{pgfscope}%
\definecolor{textcolor}{rgb}{0.000000,0.000000,0.000000}%
\pgfsetstrokecolor{textcolor}%
\pgfsetfillcolor{textcolor}%
\pgftext[x=1.846250in,y=0.427778in,,top]{\color{textcolor}\rmfamily\fontsize{9.000000}{10.800000}\selectfont \(\displaystyle {10^{3}}\)}%
\end{pgfscope}%
\begin{pgfscope}%
\pgfpathrectangle{\pgfqpoint{0.726250in}{0.525000in}}{\pgfqpoint{1.120000in}{1.637500in}}%
\pgfusepath{clip}%
\pgfsetbuttcap%
\pgfsetroundjoin%
\pgfsetlinewidth{0.803000pt}%
\definecolor{currentstroke}{rgb}{0.752941,0.752941,0.752941}%
\pgfsetstrokecolor{currentstroke}%
\pgfsetdash{{2.960000pt}{1.280000pt}}{0.000000pt}%
\pgfpathmoveto{\pgfqpoint{0.810538in}{0.525000in}}%
\pgfpathlineto{\pgfqpoint{0.810538in}{2.162500in}}%
\pgfusepath{stroke}%
\end{pgfscope}%
\begin{pgfscope}%
\pgfsetbuttcap%
\pgfsetroundjoin%
\definecolor{currentfill}{rgb}{0.000000,0.000000,0.000000}%
\pgfsetfillcolor{currentfill}%
\pgfsetlinewidth{0.602250pt}%
\definecolor{currentstroke}{rgb}{0.000000,0.000000,0.000000}%
\pgfsetstrokecolor{currentstroke}%
\pgfsetdash{}{0pt}%
\pgfsys@defobject{currentmarker}{\pgfqpoint{0.000000in}{-0.027778in}}{\pgfqpoint{0.000000in}{0.000000in}}{%
\pgfpathmoveto{\pgfqpoint{0.000000in}{0.000000in}}%
\pgfpathlineto{\pgfqpoint{0.000000in}{-0.027778in}}%
\pgfusepath{stroke,fill}%
}%
\begin{pgfscope}%
\pgfsys@transformshift{0.810538in}{0.525000in}%
\pgfsys@useobject{currentmarker}{}%
\end{pgfscope}%
\end{pgfscope}%
\begin{pgfscope}%
\pgfpathrectangle{\pgfqpoint{0.726250in}{0.525000in}}{\pgfqpoint{1.120000in}{1.637500in}}%
\pgfusepath{clip}%
\pgfsetbuttcap%
\pgfsetroundjoin%
\pgfsetlinewidth{0.803000pt}%
\definecolor{currentstroke}{rgb}{0.752941,0.752941,0.752941}%
\pgfsetstrokecolor{currentstroke}%
\pgfsetdash{{2.960000pt}{1.280000pt}}{0.000000pt}%
\pgfpathmoveto{\pgfqpoint{0.859844in}{0.525000in}}%
\pgfpathlineto{\pgfqpoint{0.859844in}{2.162500in}}%
\pgfusepath{stroke}%
\end{pgfscope}%
\begin{pgfscope}%
\pgfsetbuttcap%
\pgfsetroundjoin%
\definecolor{currentfill}{rgb}{0.000000,0.000000,0.000000}%
\pgfsetfillcolor{currentfill}%
\pgfsetlinewidth{0.602250pt}%
\definecolor{currentstroke}{rgb}{0.000000,0.000000,0.000000}%
\pgfsetstrokecolor{currentstroke}%
\pgfsetdash{}{0pt}%
\pgfsys@defobject{currentmarker}{\pgfqpoint{0.000000in}{-0.027778in}}{\pgfqpoint{0.000000in}{0.000000in}}{%
\pgfpathmoveto{\pgfqpoint{0.000000in}{0.000000in}}%
\pgfpathlineto{\pgfqpoint{0.000000in}{-0.027778in}}%
\pgfusepath{stroke,fill}%
}%
\begin{pgfscope}%
\pgfsys@transformshift{0.859844in}{0.525000in}%
\pgfsys@useobject{currentmarker}{}%
\end{pgfscope}%
\end{pgfscope}%
\begin{pgfscope}%
\pgfpathrectangle{\pgfqpoint{0.726250in}{0.525000in}}{\pgfqpoint{1.120000in}{1.637500in}}%
\pgfusepath{clip}%
\pgfsetbuttcap%
\pgfsetroundjoin%
\pgfsetlinewidth{0.803000pt}%
\definecolor{currentstroke}{rgb}{0.752941,0.752941,0.752941}%
\pgfsetstrokecolor{currentstroke}%
\pgfsetdash{{2.960000pt}{1.280000pt}}{0.000000pt}%
\pgfpathmoveto{\pgfqpoint{0.894827in}{0.525000in}}%
\pgfpathlineto{\pgfqpoint{0.894827in}{2.162500in}}%
\pgfusepath{stroke}%
\end{pgfscope}%
\begin{pgfscope}%
\pgfsetbuttcap%
\pgfsetroundjoin%
\definecolor{currentfill}{rgb}{0.000000,0.000000,0.000000}%
\pgfsetfillcolor{currentfill}%
\pgfsetlinewidth{0.602250pt}%
\definecolor{currentstroke}{rgb}{0.000000,0.000000,0.000000}%
\pgfsetstrokecolor{currentstroke}%
\pgfsetdash{}{0pt}%
\pgfsys@defobject{currentmarker}{\pgfqpoint{0.000000in}{-0.027778in}}{\pgfqpoint{0.000000in}{0.000000in}}{%
\pgfpathmoveto{\pgfqpoint{0.000000in}{0.000000in}}%
\pgfpathlineto{\pgfqpoint{0.000000in}{-0.027778in}}%
\pgfusepath{stroke,fill}%
}%
\begin{pgfscope}%
\pgfsys@transformshift{0.894827in}{0.525000in}%
\pgfsys@useobject{currentmarker}{}%
\end{pgfscope}%
\end{pgfscope}%
\begin{pgfscope}%
\pgfpathrectangle{\pgfqpoint{0.726250in}{0.525000in}}{\pgfqpoint{1.120000in}{1.637500in}}%
\pgfusepath{clip}%
\pgfsetbuttcap%
\pgfsetroundjoin%
\pgfsetlinewidth{0.803000pt}%
\definecolor{currentstroke}{rgb}{0.752941,0.752941,0.752941}%
\pgfsetstrokecolor{currentstroke}%
\pgfsetdash{{2.960000pt}{1.280000pt}}{0.000000pt}%
\pgfpathmoveto{\pgfqpoint{0.921962in}{0.525000in}}%
\pgfpathlineto{\pgfqpoint{0.921962in}{2.162500in}}%
\pgfusepath{stroke}%
\end{pgfscope}%
\begin{pgfscope}%
\pgfsetbuttcap%
\pgfsetroundjoin%
\definecolor{currentfill}{rgb}{0.000000,0.000000,0.000000}%
\pgfsetfillcolor{currentfill}%
\pgfsetlinewidth{0.602250pt}%
\definecolor{currentstroke}{rgb}{0.000000,0.000000,0.000000}%
\pgfsetstrokecolor{currentstroke}%
\pgfsetdash{}{0pt}%
\pgfsys@defobject{currentmarker}{\pgfqpoint{0.000000in}{-0.027778in}}{\pgfqpoint{0.000000in}{0.000000in}}{%
\pgfpathmoveto{\pgfqpoint{0.000000in}{0.000000in}}%
\pgfpathlineto{\pgfqpoint{0.000000in}{-0.027778in}}%
\pgfusepath{stroke,fill}%
}%
\begin{pgfscope}%
\pgfsys@transformshift{0.921962in}{0.525000in}%
\pgfsys@useobject{currentmarker}{}%
\end{pgfscope}%
\end{pgfscope}%
\begin{pgfscope}%
\pgfpathrectangle{\pgfqpoint{0.726250in}{0.525000in}}{\pgfqpoint{1.120000in}{1.637500in}}%
\pgfusepath{clip}%
\pgfsetbuttcap%
\pgfsetroundjoin%
\pgfsetlinewidth{0.803000pt}%
\definecolor{currentstroke}{rgb}{0.752941,0.752941,0.752941}%
\pgfsetstrokecolor{currentstroke}%
\pgfsetdash{{2.960000pt}{1.280000pt}}{0.000000pt}%
\pgfpathmoveto{\pgfqpoint{0.944132in}{0.525000in}}%
\pgfpathlineto{\pgfqpoint{0.944132in}{2.162500in}}%
\pgfusepath{stroke}%
\end{pgfscope}%
\begin{pgfscope}%
\pgfsetbuttcap%
\pgfsetroundjoin%
\definecolor{currentfill}{rgb}{0.000000,0.000000,0.000000}%
\pgfsetfillcolor{currentfill}%
\pgfsetlinewidth{0.602250pt}%
\definecolor{currentstroke}{rgb}{0.000000,0.000000,0.000000}%
\pgfsetstrokecolor{currentstroke}%
\pgfsetdash{}{0pt}%
\pgfsys@defobject{currentmarker}{\pgfqpoint{0.000000in}{-0.027778in}}{\pgfqpoint{0.000000in}{0.000000in}}{%
\pgfpathmoveto{\pgfqpoint{0.000000in}{0.000000in}}%
\pgfpathlineto{\pgfqpoint{0.000000in}{-0.027778in}}%
\pgfusepath{stroke,fill}%
}%
\begin{pgfscope}%
\pgfsys@transformshift{0.944132in}{0.525000in}%
\pgfsys@useobject{currentmarker}{}%
\end{pgfscope}%
\end{pgfscope}%
\begin{pgfscope}%
\pgfpathrectangle{\pgfqpoint{0.726250in}{0.525000in}}{\pgfqpoint{1.120000in}{1.637500in}}%
\pgfusepath{clip}%
\pgfsetbuttcap%
\pgfsetroundjoin%
\pgfsetlinewidth{0.803000pt}%
\definecolor{currentstroke}{rgb}{0.752941,0.752941,0.752941}%
\pgfsetstrokecolor{currentstroke}%
\pgfsetdash{{2.960000pt}{1.280000pt}}{0.000000pt}%
\pgfpathmoveto{\pgfqpoint{0.962877in}{0.525000in}}%
\pgfpathlineto{\pgfqpoint{0.962877in}{2.162500in}}%
\pgfusepath{stroke}%
\end{pgfscope}%
\begin{pgfscope}%
\pgfsetbuttcap%
\pgfsetroundjoin%
\definecolor{currentfill}{rgb}{0.000000,0.000000,0.000000}%
\pgfsetfillcolor{currentfill}%
\pgfsetlinewidth{0.602250pt}%
\definecolor{currentstroke}{rgb}{0.000000,0.000000,0.000000}%
\pgfsetstrokecolor{currentstroke}%
\pgfsetdash{}{0pt}%
\pgfsys@defobject{currentmarker}{\pgfqpoint{0.000000in}{-0.027778in}}{\pgfqpoint{0.000000in}{0.000000in}}{%
\pgfpathmoveto{\pgfqpoint{0.000000in}{0.000000in}}%
\pgfpathlineto{\pgfqpoint{0.000000in}{-0.027778in}}%
\pgfusepath{stroke,fill}%
}%
\begin{pgfscope}%
\pgfsys@transformshift{0.962877in}{0.525000in}%
\pgfsys@useobject{currentmarker}{}%
\end{pgfscope}%
\end{pgfscope}%
\begin{pgfscope}%
\pgfpathrectangle{\pgfqpoint{0.726250in}{0.525000in}}{\pgfqpoint{1.120000in}{1.637500in}}%
\pgfusepath{clip}%
\pgfsetbuttcap%
\pgfsetroundjoin%
\pgfsetlinewidth{0.803000pt}%
\definecolor{currentstroke}{rgb}{0.752941,0.752941,0.752941}%
\pgfsetstrokecolor{currentstroke}%
\pgfsetdash{{2.960000pt}{1.280000pt}}{0.000000pt}%
\pgfpathmoveto{\pgfqpoint{0.979115in}{0.525000in}}%
\pgfpathlineto{\pgfqpoint{0.979115in}{2.162500in}}%
\pgfusepath{stroke}%
\end{pgfscope}%
\begin{pgfscope}%
\pgfsetbuttcap%
\pgfsetroundjoin%
\definecolor{currentfill}{rgb}{0.000000,0.000000,0.000000}%
\pgfsetfillcolor{currentfill}%
\pgfsetlinewidth{0.602250pt}%
\definecolor{currentstroke}{rgb}{0.000000,0.000000,0.000000}%
\pgfsetstrokecolor{currentstroke}%
\pgfsetdash{}{0pt}%
\pgfsys@defobject{currentmarker}{\pgfqpoint{0.000000in}{-0.027778in}}{\pgfqpoint{0.000000in}{0.000000in}}{%
\pgfpathmoveto{\pgfqpoint{0.000000in}{0.000000in}}%
\pgfpathlineto{\pgfqpoint{0.000000in}{-0.027778in}}%
\pgfusepath{stroke,fill}%
}%
\begin{pgfscope}%
\pgfsys@transformshift{0.979115in}{0.525000in}%
\pgfsys@useobject{currentmarker}{}%
\end{pgfscope}%
\end{pgfscope}%
\begin{pgfscope}%
\pgfpathrectangle{\pgfqpoint{0.726250in}{0.525000in}}{\pgfqpoint{1.120000in}{1.637500in}}%
\pgfusepath{clip}%
\pgfsetbuttcap%
\pgfsetroundjoin%
\pgfsetlinewidth{0.803000pt}%
\definecolor{currentstroke}{rgb}{0.752941,0.752941,0.752941}%
\pgfsetstrokecolor{currentstroke}%
\pgfsetdash{{2.960000pt}{1.280000pt}}{0.000000pt}%
\pgfpathmoveto{\pgfqpoint{0.993438in}{0.525000in}}%
\pgfpathlineto{\pgfqpoint{0.993438in}{2.162500in}}%
\pgfusepath{stroke}%
\end{pgfscope}%
\begin{pgfscope}%
\pgfsetbuttcap%
\pgfsetroundjoin%
\definecolor{currentfill}{rgb}{0.000000,0.000000,0.000000}%
\pgfsetfillcolor{currentfill}%
\pgfsetlinewidth{0.602250pt}%
\definecolor{currentstroke}{rgb}{0.000000,0.000000,0.000000}%
\pgfsetstrokecolor{currentstroke}%
\pgfsetdash{}{0pt}%
\pgfsys@defobject{currentmarker}{\pgfqpoint{0.000000in}{-0.027778in}}{\pgfqpoint{0.000000in}{0.000000in}}{%
\pgfpathmoveto{\pgfqpoint{0.000000in}{0.000000in}}%
\pgfpathlineto{\pgfqpoint{0.000000in}{-0.027778in}}%
\pgfusepath{stroke,fill}%
}%
\begin{pgfscope}%
\pgfsys@transformshift{0.993438in}{0.525000in}%
\pgfsys@useobject{currentmarker}{}%
\end{pgfscope}%
\end{pgfscope}%
\begin{pgfscope}%
\pgfpathrectangle{\pgfqpoint{0.726250in}{0.525000in}}{\pgfqpoint{1.120000in}{1.637500in}}%
\pgfusepath{clip}%
\pgfsetbuttcap%
\pgfsetroundjoin%
\pgfsetlinewidth{0.803000pt}%
\definecolor{currentstroke}{rgb}{0.752941,0.752941,0.752941}%
\pgfsetstrokecolor{currentstroke}%
\pgfsetdash{{2.960000pt}{1.280000pt}}{0.000000pt}%
\pgfpathmoveto{\pgfqpoint{1.090538in}{0.525000in}}%
\pgfpathlineto{\pgfqpoint{1.090538in}{2.162500in}}%
\pgfusepath{stroke}%
\end{pgfscope}%
\begin{pgfscope}%
\pgfsetbuttcap%
\pgfsetroundjoin%
\definecolor{currentfill}{rgb}{0.000000,0.000000,0.000000}%
\pgfsetfillcolor{currentfill}%
\pgfsetlinewidth{0.602250pt}%
\definecolor{currentstroke}{rgb}{0.000000,0.000000,0.000000}%
\pgfsetstrokecolor{currentstroke}%
\pgfsetdash{}{0pt}%
\pgfsys@defobject{currentmarker}{\pgfqpoint{0.000000in}{-0.027778in}}{\pgfqpoint{0.000000in}{0.000000in}}{%
\pgfpathmoveto{\pgfqpoint{0.000000in}{0.000000in}}%
\pgfpathlineto{\pgfqpoint{0.000000in}{-0.027778in}}%
\pgfusepath{stroke,fill}%
}%
\begin{pgfscope}%
\pgfsys@transformshift{1.090538in}{0.525000in}%
\pgfsys@useobject{currentmarker}{}%
\end{pgfscope}%
\end{pgfscope}%
\begin{pgfscope}%
\pgfpathrectangle{\pgfqpoint{0.726250in}{0.525000in}}{\pgfqpoint{1.120000in}{1.637500in}}%
\pgfusepath{clip}%
\pgfsetbuttcap%
\pgfsetroundjoin%
\pgfsetlinewidth{0.803000pt}%
\definecolor{currentstroke}{rgb}{0.752941,0.752941,0.752941}%
\pgfsetstrokecolor{currentstroke}%
\pgfsetdash{{2.960000pt}{1.280000pt}}{0.000000pt}%
\pgfpathmoveto{\pgfqpoint{1.139844in}{0.525000in}}%
\pgfpathlineto{\pgfqpoint{1.139844in}{2.162500in}}%
\pgfusepath{stroke}%
\end{pgfscope}%
\begin{pgfscope}%
\pgfsetbuttcap%
\pgfsetroundjoin%
\definecolor{currentfill}{rgb}{0.000000,0.000000,0.000000}%
\pgfsetfillcolor{currentfill}%
\pgfsetlinewidth{0.602250pt}%
\definecolor{currentstroke}{rgb}{0.000000,0.000000,0.000000}%
\pgfsetstrokecolor{currentstroke}%
\pgfsetdash{}{0pt}%
\pgfsys@defobject{currentmarker}{\pgfqpoint{0.000000in}{-0.027778in}}{\pgfqpoint{0.000000in}{0.000000in}}{%
\pgfpathmoveto{\pgfqpoint{0.000000in}{0.000000in}}%
\pgfpathlineto{\pgfqpoint{0.000000in}{-0.027778in}}%
\pgfusepath{stroke,fill}%
}%
\begin{pgfscope}%
\pgfsys@transformshift{1.139844in}{0.525000in}%
\pgfsys@useobject{currentmarker}{}%
\end{pgfscope}%
\end{pgfscope}%
\begin{pgfscope}%
\pgfpathrectangle{\pgfqpoint{0.726250in}{0.525000in}}{\pgfqpoint{1.120000in}{1.637500in}}%
\pgfusepath{clip}%
\pgfsetbuttcap%
\pgfsetroundjoin%
\pgfsetlinewidth{0.803000pt}%
\definecolor{currentstroke}{rgb}{0.752941,0.752941,0.752941}%
\pgfsetstrokecolor{currentstroke}%
\pgfsetdash{{2.960000pt}{1.280000pt}}{0.000000pt}%
\pgfpathmoveto{\pgfqpoint{1.174827in}{0.525000in}}%
\pgfpathlineto{\pgfqpoint{1.174827in}{2.162500in}}%
\pgfusepath{stroke}%
\end{pgfscope}%
\begin{pgfscope}%
\pgfsetbuttcap%
\pgfsetroundjoin%
\definecolor{currentfill}{rgb}{0.000000,0.000000,0.000000}%
\pgfsetfillcolor{currentfill}%
\pgfsetlinewidth{0.602250pt}%
\definecolor{currentstroke}{rgb}{0.000000,0.000000,0.000000}%
\pgfsetstrokecolor{currentstroke}%
\pgfsetdash{}{0pt}%
\pgfsys@defobject{currentmarker}{\pgfqpoint{0.000000in}{-0.027778in}}{\pgfqpoint{0.000000in}{0.000000in}}{%
\pgfpathmoveto{\pgfqpoint{0.000000in}{0.000000in}}%
\pgfpathlineto{\pgfqpoint{0.000000in}{-0.027778in}}%
\pgfusepath{stroke,fill}%
}%
\begin{pgfscope}%
\pgfsys@transformshift{1.174827in}{0.525000in}%
\pgfsys@useobject{currentmarker}{}%
\end{pgfscope}%
\end{pgfscope}%
\begin{pgfscope}%
\pgfpathrectangle{\pgfqpoint{0.726250in}{0.525000in}}{\pgfqpoint{1.120000in}{1.637500in}}%
\pgfusepath{clip}%
\pgfsetbuttcap%
\pgfsetroundjoin%
\pgfsetlinewidth{0.803000pt}%
\definecolor{currentstroke}{rgb}{0.752941,0.752941,0.752941}%
\pgfsetstrokecolor{currentstroke}%
\pgfsetdash{{2.960000pt}{1.280000pt}}{0.000000pt}%
\pgfpathmoveto{\pgfqpoint{1.201962in}{0.525000in}}%
\pgfpathlineto{\pgfqpoint{1.201962in}{2.162500in}}%
\pgfusepath{stroke}%
\end{pgfscope}%
\begin{pgfscope}%
\pgfsetbuttcap%
\pgfsetroundjoin%
\definecolor{currentfill}{rgb}{0.000000,0.000000,0.000000}%
\pgfsetfillcolor{currentfill}%
\pgfsetlinewidth{0.602250pt}%
\definecolor{currentstroke}{rgb}{0.000000,0.000000,0.000000}%
\pgfsetstrokecolor{currentstroke}%
\pgfsetdash{}{0pt}%
\pgfsys@defobject{currentmarker}{\pgfqpoint{0.000000in}{-0.027778in}}{\pgfqpoint{0.000000in}{0.000000in}}{%
\pgfpathmoveto{\pgfqpoint{0.000000in}{0.000000in}}%
\pgfpathlineto{\pgfqpoint{0.000000in}{-0.027778in}}%
\pgfusepath{stroke,fill}%
}%
\begin{pgfscope}%
\pgfsys@transformshift{1.201962in}{0.525000in}%
\pgfsys@useobject{currentmarker}{}%
\end{pgfscope}%
\end{pgfscope}%
\begin{pgfscope}%
\pgfpathrectangle{\pgfqpoint{0.726250in}{0.525000in}}{\pgfqpoint{1.120000in}{1.637500in}}%
\pgfusepath{clip}%
\pgfsetbuttcap%
\pgfsetroundjoin%
\pgfsetlinewidth{0.803000pt}%
\definecolor{currentstroke}{rgb}{0.752941,0.752941,0.752941}%
\pgfsetstrokecolor{currentstroke}%
\pgfsetdash{{2.960000pt}{1.280000pt}}{0.000000pt}%
\pgfpathmoveto{\pgfqpoint{1.224132in}{0.525000in}}%
\pgfpathlineto{\pgfqpoint{1.224132in}{2.162500in}}%
\pgfusepath{stroke}%
\end{pgfscope}%
\begin{pgfscope}%
\pgfsetbuttcap%
\pgfsetroundjoin%
\definecolor{currentfill}{rgb}{0.000000,0.000000,0.000000}%
\pgfsetfillcolor{currentfill}%
\pgfsetlinewidth{0.602250pt}%
\definecolor{currentstroke}{rgb}{0.000000,0.000000,0.000000}%
\pgfsetstrokecolor{currentstroke}%
\pgfsetdash{}{0pt}%
\pgfsys@defobject{currentmarker}{\pgfqpoint{0.000000in}{-0.027778in}}{\pgfqpoint{0.000000in}{0.000000in}}{%
\pgfpathmoveto{\pgfqpoint{0.000000in}{0.000000in}}%
\pgfpathlineto{\pgfqpoint{0.000000in}{-0.027778in}}%
\pgfusepath{stroke,fill}%
}%
\begin{pgfscope}%
\pgfsys@transformshift{1.224132in}{0.525000in}%
\pgfsys@useobject{currentmarker}{}%
\end{pgfscope}%
\end{pgfscope}%
\begin{pgfscope}%
\pgfpathrectangle{\pgfqpoint{0.726250in}{0.525000in}}{\pgfqpoint{1.120000in}{1.637500in}}%
\pgfusepath{clip}%
\pgfsetbuttcap%
\pgfsetroundjoin%
\pgfsetlinewidth{0.803000pt}%
\definecolor{currentstroke}{rgb}{0.752941,0.752941,0.752941}%
\pgfsetstrokecolor{currentstroke}%
\pgfsetdash{{2.960000pt}{1.280000pt}}{0.000000pt}%
\pgfpathmoveto{\pgfqpoint{1.242877in}{0.525000in}}%
\pgfpathlineto{\pgfqpoint{1.242877in}{2.162500in}}%
\pgfusepath{stroke}%
\end{pgfscope}%
\begin{pgfscope}%
\pgfsetbuttcap%
\pgfsetroundjoin%
\definecolor{currentfill}{rgb}{0.000000,0.000000,0.000000}%
\pgfsetfillcolor{currentfill}%
\pgfsetlinewidth{0.602250pt}%
\definecolor{currentstroke}{rgb}{0.000000,0.000000,0.000000}%
\pgfsetstrokecolor{currentstroke}%
\pgfsetdash{}{0pt}%
\pgfsys@defobject{currentmarker}{\pgfqpoint{0.000000in}{-0.027778in}}{\pgfqpoint{0.000000in}{0.000000in}}{%
\pgfpathmoveto{\pgfqpoint{0.000000in}{0.000000in}}%
\pgfpathlineto{\pgfqpoint{0.000000in}{-0.027778in}}%
\pgfusepath{stroke,fill}%
}%
\begin{pgfscope}%
\pgfsys@transformshift{1.242877in}{0.525000in}%
\pgfsys@useobject{currentmarker}{}%
\end{pgfscope}%
\end{pgfscope}%
\begin{pgfscope}%
\pgfpathrectangle{\pgfqpoint{0.726250in}{0.525000in}}{\pgfqpoint{1.120000in}{1.637500in}}%
\pgfusepath{clip}%
\pgfsetbuttcap%
\pgfsetroundjoin%
\pgfsetlinewidth{0.803000pt}%
\definecolor{currentstroke}{rgb}{0.752941,0.752941,0.752941}%
\pgfsetstrokecolor{currentstroke}%
\pgfsetdash{{2.960000pt}{1.280000pt}}{0.000000pt}%
\pgfpathmoveto{\pgfqpoint{1.259115in}{0.525000in}}%
\pgfpathlineto{\pgfqpoint{1.259115in}{2.162500in}}%
\pgfusepath{stroke}%
\end{pgfscope}%
\begin{pgfscope}%
\pgfsetbuttcap%
\pgfsetroundjoin%
\definecolor{currentfill}{rgb}{0.000000,0.000000,0.000000}%
\pgfsetfillcolor{currentfill}%
\pgfsetlinewidth{0.602250pt}%
\definecolor{currentstroke}{rgb}{0.000000,0.000000,0.000000}%
\pgfsetstrokecolor{currentstroke}%
\pgfsetdash{}{0pt}%
\pgfsys@defobject{currentmarker}{\pgfqpoint{0.000000in}{-0.027778in}}{\pgfqpoint{0.000000in}{0.000000in}}{%
\pgfpathmoveto{\pgfqpoint{0.000000in}{0.000000in}}%
\pgfpathlineto{\pgfqpoint{0.000000in}{-0.027778in}}%
\pgfusepath{stroke,fill}%
}%
\begin{pgfscope}%
\pgfsys@transformshift{1.259115in}{0.525000in}%
\pgfsys@useobject{currentmarker}{}%
\end{pgfscope}%
\end{pgfscope}%
\begin{pgfscope}%
\pgfpathrectangle{\pgfqpoint{0.726250in}{0.525000in}}{\pgfqpoint{1.120000in}{1.637500in}}%
\pgfusepath{clip}%
\pgfsetbuttcap%
\pgfsetroundjoin%
\pgfsetlinewidth{0.803000pt}%
\definecolor{currentstroke}{rgb}{0.752941,0.752941,0.752941}%
\pgfsetstrokecolor{currentstroke}%
\pgfsetdash{{2.960000pt}{1.280000pt}}{0.000000pt}%
\pgfpathmoveto{\pgfqpoint{1.273438in}{0.525000in}}%
\pgfpathlineto{\pgfqpoint{1.273438in}{2.162500in}}%
\pgfusepath{stroke}%
\end{pgfscope}%
\begin{pgfscope}%
\pgfsetbuttcap%
\pgfsetroundjoin%
\definecolor{currentfill}{rgb}{0.000000,0.000000,0.000000}%
\pgfsetfillcolor{currentfill}%
\pgfsetlinewidth{0.602250pt}%
\definecolor{currentstroke}{rgb}{0.000000,0.000000,0.000000}%
\pgfsetstrokecolor{currentstroke}%
\pgfsetdash{}{0pt}%
\pgfsys@defobject{currentmarker}{\pgfqpoint{0.000000in}{-0.027778in}}{\pgfqpoint{0.000000in}{0.000000in}}{%
\pgfpathmoveto{\pgfqpoint{0.000000in}{0.000000in}}%
\pgfpathlineto{\pgfqpoint{0.000000in}{-0.027778in}}%
\pgfusepath{stroke,fill}%
}%
\begin{pgfscope}%
\pgfsys@transformshift{1.273438in}{0.525000in}%
\pgfsys@useobject{currentmarker}{}%
\end{pgfscope}%
\end{pgfscope}%
\begin{pgfscope}%
\pgfpathrectangle{\pgfqpoint{0.726250in}{0.525000in}}{\pgfqpoint{1.120000in}{1.637500in}}%
\pgfusepath{clip}%
\pgfsetbuttcap%
\pgfsetroundjoin%
\pgfsetlinewidth{0.803000pt}%
\definecolor{currentstroke}{rgb}{0.752941,0.752941,0.752941}%
\pgfsetstrokecolor{currentstroke}%
\pgfsetdash{{2.960000pt}{1.280000pt}}{0.000000pt}%
\pgfpathmoveto{\pgfqpoint{1.370538in}{0.525000in}}%
\pgfpathlineto{\pgfqpoint{1.370538in}{2.162500in}}%
\pgfusepath{stroke}%
\end{pgfscope}%
\begin{pgfscope}%
\pgfsetbuttcap%
\pgfsetroundjoin%
\definecolor{currentfill}{rgb}{0.000000,0.000000,0.000000}%
\pgfsetfillcolor{currentfill}%
\pgfsetlinewidth{0.602250pt}%
\definecolor{currentstroke}{rgb}{0.000000,0.000000,0.000000}%
\pgfsetstrokecolor{currentstroke}%
\pgfsetdash{}{0pt}%
\pgfsys@defobject{currentmarker}{\pgfqpoint{0.000000in}{-0.027778in}}{\pgfqpoint{0.000000in}{0.000000in}}{%
\pgfpathmoveto{\pgfqpoint{0.000000in}{0.000000in}}%
\pgfpathlineto{\pgfqpoint{0.000000in}{-0.027778in}}%
\pgfusepath{stroke,fill}%
}%
\begin{pgfscope}%
\pgfsys@transformshift{1.370538in}{0.525000in}%
\pgfsys@useobject{currentmarker}{}%
\end{pgfscope}%
\end{pgfscope}%
\begin{pgfscope}%
\pgfpathrectangle{\pgfqpoint{0.726250in}{0.525000in}}{\pgfqpoint{1.120000in}{1.637500in}}%
\pgfusepath{clip}%
\pgfsetbuttcap%
\pgfsetroundjoin%
\pgfsetlinewidth{0.803000pt}%
\definecolor{currentstroke}{rgb}{0.752941,0.752941,0.752941}%
\pgfsetstrokecolor{currentstroke}%
\pgfsetdash{{2.960000pt}{1.280000pt}}{0.000000pt}%
\pgfpathmoveto{\pgfqpoint{1.419844in}{0.525000in}}%
\pgfpathlineto{\pgfqpoint{1.419844in}{2.162500in}}%
\pgfusepath{stroke}%
\end{pgfscope}%
\begin{pgfscope}%
\pgfsetbuttcap%
\pgfsetroundjoin%
\definecolor{currentfill}{rgb}{0.000000,0.000000,0.000000}%
\pgfsetfillcolor{currentfill}%
\pgfsetlinewidth{0.602250pt}%
\definecolor{currentstroke}{rgb}{0.000000,0.000000,0.000000}%
\pgfsetstrokecolor{currentstroke}%
\pgfsetdash{}{0pt}%
\pgfsys@defobject{currentmarker}{\pgfqpoint{0.000000in}{-0.027778in}}{\pgfqpoint{0.000000in}{0.000000in}}{%
\pgfpathmoveto{\pgfqpoint{0.000000in}{0.000000in}}%
\pgfpathlineto{\pgfqpoint{0.000000in}{-0.027778in}}%
\pgfusepath{stroke,fill}%
}%
\begin{pgfscope}%
\pgfsys@transformshift{1.419844in}{0.525000in}%
\pgfsys@useobject{currentmarker}{}%
\end{pgfscope}%
\end{pgfscope}%
\begin{pgfscope}%
\pgfpathrectangle{\pgfqpoint{0.726250in}{0.525000in}}{\pgfqpoint{1.120000in}{1.637500in}}%
\pgfusepath{clip}%
\pgfsetbuttcap%
\pgfsetroundjoin%
\pgfsetlinewidth{0.803000pt}%
\definecolor{currentstroke}{rgb}{0.752941,0.752941,0.752941}%
\pgfsetstrokecolor{currentstroke}%
\pgfsetdash{{2.960000pt}{1.280000pt}}{0.000000pt}%
\pgfpathmoveto{\pgfqpoint{1.454827in}{0.525000in}}%
\pgfpathlineto{\pgfqpoint{1.454827in}{2.162500in}}%
\pgfusepath{stroke}%
\end{pgfscope}%
\begin{pgfscope}%
\pgfsetbuttcap%
\pgfsetroundjoin%
\definecolor{currentfill}{rgb}{0.000000,0.000000,0.000000}%
\pgfsetfillcolor{currentfill}%
\pgfsetlinewidth{0.602250pt}%
\definecolor{currentstroke}{rgb}{0.000000,0.000000,0.000000}%
\pgfsetstrokecolor{currentstroke}%
\pgfsetdash{}{0pt}%
\pgfsys@defobject{currentmarker}{\pgfqpoint{0.000000in}{-0.027778in}}{\pgfqpoint{0.000000in}{0.000000in}}{%
\pgfpathmoveto{\pgfqpoint{0.000000in}{0.000000in}}%
\pgfpathlineto{\pgfqpoint{0.000000in}{-0.027778in}}%
\pgfusepath{stroke,fill}%
}%
\begin{pgfscope}%
\pgfsys@transformshift{1.454827in}{0.525000in}%
\pgfsys@useobject{currentmarker}{}%
\end{pgfscope}%
\end{pgfscope}%
\begin{pgfscope}%
\pgfpathrectangle{\pgfqpoint{0.726250in}{0.525000in}}{\pgfqpoint{1.120000in}{1.637500in}}%
\pgfusepath{clip}%
\pgfsetbuttcap%
\pgfsetroundjoin%
\pgfsetlinewidth{0.803000pt}%
\definecolor{currentstroke}{rgb}{0.752941,0.752941,0.752941}%
\pgfsetstrokecolor{currentstroke}%
\pgfsetdash{{2.960000pt}{1.280000pt}}{0.000000pt}%
\pgfpathmoveto{\pgfqpoint{1.481962in}{0.525000in}}%
\pgfpathlineto{\pgfqpoint{1.481962in}{2.162500in}}%
\pgfusepath{stroke}%
\end{pgfscope}%
\begin{pgfscope}%
\pgfsetbuttcap%
\pgfsetroundjoin%
\definecolor{currentfill}{rgb}{0.000000,0.000000,0.000000}%
\pgfsetfillcolor{currentfill}%
\pgfsetlinewidth{0.602250pt}%
\definecolor{currentstroke}{rgb}{0.000000,0.000000,0.000000}%
\pgfsetstrokecolor{currentstroke}%
\pgfsetdash{}{0pt}%
\pgfsys@defobject{currentmarker}{\pgfqpoint{0.000000in}{-0.027778in}}{\pgfqpoint{0.000000in}{0.000000in}}{%
\pgfpathmoveto{\pgfqpoint{0.000000in}{0.000000in}}%
\pgfpathlineto{\pgfqpoint{0.000000in}{-0.027778in}}%
\pgfusepath{stroke,fill}%
}%
\begin{pgfscope}%
\pgfsys@transformshift{1.481962in}{0.525000in}%
\pgfsys@useobject{currentmarker}{}%
\end{pgfscope}%
\end{pgfscope}%
\begin{pgfscope}%
\pgfpathrectangle{\pgfqpoint{0.726250in}{0.525000in}}{\pgfqpoint{1.120000in}{1.637500in}}%
\pgfusepath{clip}%
\pgfsetbuttcap%
\pgfsetroundjoin%
\pgfsetlinewidth{0.803000pt}%
\definecolor{currentstroke}{rgb}{0.752941,0.752941,0.752941}%
\pgfsetstrokecolor{currentstroke}%
\pgfsetdash{{2.960000pt}{1.280000pt}}{0.000000pt}%
\pgfpathmoveto{\pgfqpoint{1.504132in}{0.525000in}}%
\pgfpathlineto{\pgfqpoint{1.504132in}{2.162500in}}%
\pgfusepath{stroke}%
\end{pgfscope}%
\begin{pgfscope}%
\pgfsetbuttcap%
\pgfsetroundjoin%
\definecolor{currentfill}{rgb}{0.000000,0.000000,0.000000}%
\pgfsetfillcolor{currentfill}%
\pgfsetlinewidth{0.602250pt}%
\definecolor{currentstroke}{rgb}{0.000000,0.000000,0.000000}%
\pgfsetstrokecolor{currentstroke}%
\pgfsetdash{}{0pt}%
\pgfsys@defobject{currentmarker}{\pgfqpoint{0.000000in}{-0.027778in}}{\pgfqpoint{0.000000in}{0.000000in}}{%
\pgfpathmoveto{\pgfqpoint{0.000000in}{0.000000in}}%
\pgfpathlineto{\pgfqpoint{0.000000in}{-0.027778in}}%
\pgfusepath{stroke,fill}%
}%
\begin{pgfscope}%
\pgfsys@transformshift{1.504132in}{0.525000in}%
\pgfsys@useobject{currentmarker}{}%
\end{pgfscope}%
\end{pgfscope}%
\begin{pgfscope}%
\pgfpathrectangle{\pgfqpoint{0.726250in}{0.525000in}}{\pgfqpoint{1.120000in}{1.637500in}}%
\pgfusepath{clip}%
\pgfsetbuttcap%
\pgfsetroundjoin%
\pgfsetlinewidth{0.803000pt}%
\definecolor{currentstroke}{rgb}{0.752941,0.752941,0.752941}%
\pgfsetstrokecolor{currentstroke}%
\pgfsetdash{{2.960000pt}{1.280000pt}}{0.000000pt}%
\pgfpathmoveto{\pgfqpoint{1.522877in}{0.525000in}}%
\pgfpathlineto{\pgfqpoint{1.522877in}{2.162500in}}%
\pgfusepath{stroke}%
\end{pgfscope}%
\begin{pgfscope}%
\pgfsetbuttcap%
\pgfsetroundjoin%
\definecolor{currentfill}{rgb}{0.000000,0.000000,0.000000}%
\pgfsetfillcolor{currentfill}%
\pgfsetlinewidth{0.602250pt}%
\definecolor{currentstroke}{rgb}{0.000000,0.000000,0.000000}%
\pgfsetstrokecolor{currentstroke}%
\pgfsetdash{}{0pt}%
\pgfsys@defobject{currentmarker}{\pgfqpoint{0.000000in}{-0.027778in}}{\pgfqpoint{0.000000in}{0.000000in}}{%
\pgfpathmoveto{\pgfqpoint{0.000000in}{0.000000in}}%
\pgfpathlineto{\pgfqpoint{0.000000in}{-0.027778in}}%
\pgfusepath{stroke,fill}%
}%
\begin{pgfscope}%
\pgfsys@transformshift{1.522877in}{0.525000in}%
\pgfsys@useobject{currentmarker}{}%
\end{pgfscope}%
\end{pgfscope}%
\begin{pgfscope}%
\pgfpathrectangle{\pgfqpoint{0.726250in}{0.525000in}}{\pgfqpoint{1.120000in}{1.637500in}}%
\pgfusepath{clip}%
\pgfsetbuttcap%
\pgfsetroundjoin%
\pgfsetlinewidth{0.803000pt}%
\definecolor{currentstroke}{rgb}{0.752941,0.752941,0.752941}%
\pgfsetstrokecolor{currentstroke}%
\pgfsetdash{{2.960000pt}{1.280000pt}}{0.000000pt}%
\pgfpathmoveto{\pgfqpoint{1.539115in}{0.525000in}}%
\pgfpathlineto{\pgfqpoint{1.539115in}{2.162500in}}%
\pgfusepath{stroke}%
\end{pgfscope}%
\begin{pgfscope}%
\pgfsetbuttcap%
\pgfsetroundjoin%
\definecolor{currentfill}{rgb}{0.000000,0.000000,0.000000}%
\pgfsetfillcolor{currentfill}%
\pgfsetlinewidth{0.602250pt}%
\definecolor{currentstroke}{rgb}{0.000000,0.000000,0.000000}%
\pgfsetstrokecolor{currentstroke}%
\pgfsetdash{}{0pt}%
\pgfsys@defobject{currentmarker}{\pgfqpoint{0.000000in}{-0.027778in}}{\pgfqpoint{0.000000in}{0.000000in}}{%
\pgfpathmoveto{\pgfqpoint{0.000000in}{0.000000in}}%
\pgfpathlineto{\pgfqpoint{0.000000in}{-0.027778in}}%
\pgfusepath{stroke,fill}%
}%
\begin{pgfscope}%
\pgfsys@transformshift{1.539115in}{0.525000in}%
\pgfsys@useobject{currentmarker}{}%
\end{pgfscope}%
\end{pgfscope}%
\begin{pgfscope}%
\pgfpathrectangle{\pgfqpoint{0.726250in}{0.525000in}}{\pgfqpoint{1.120000in}{1.637500in}}%
\pgfusepath{clip}%
\pgfsetbuttcap%
\pgfsetroundjoin%
\pgfsetlinewidth{0.803000pt}%
\definecolor{currentstroke}{rgb}{0.752941,0.752941,0.752941}%
\pgfsetstrokecolor{currentstroke}%
\pgfsetdash{{2.960000pt}{1.280000pt}}{0.000000pt}%
\pgfpathmoveto{\pgfqpoint{1.553438in}{0.525000in}}%
\pgfpathlineto{\pgfqpoint{1.553438in}{2.162500in}}%
\pgfusepath{stroke}%
\end{pgfscope}%
\begin{pgfscope}%
\pgfsetbuttcap%
\pgfsetroundjoin%
\definecolor{currentfill}{rgb}{0.000000,0.000000,0.000000}%
\pgfsetfillcolor{currentfill}%
\pgfsetlinewidth{0.602250pt}%
\definecolor{currentstroke}{rgb}{0.000000,0.000000,0.000000}%
\pgfsetstrokecolor{currentstroke}%
\pgfsetdash{}{0pt}%
\pgfsys@defobject{currentmarker}{\pgfqpoint{0.000000in}{-0.027778in}}{\pgfqpoint{0.000000in}{0.000000in}}{%
\pgfpathmoveto{\pgfqpoint{0.000000in}{0.000000in}}%
\pgfpathlineto{\pgfqpoint{0.000000in}{-0.027778in}}%
\pgfusepath{stroke,fill}%
}%
\begin{pgfscope}%
\pgfsys@transformshift{1.553438in}{0.525000in}%
\pgfsys@useobject{currentmarker}{}%
\end{pgfscope}%
\end{pgfscope}%
\begin{pgfscope}%
\pgfpathrectangle{\pgfqpoint{0.726250in}{0.525000in}}{\pgfqpoint{1.120000in}{1.637500in}}%
\pgfusepath{clip}%
\pgfsetbuttcap%
\pgfsetroundjoin%
\pgfsetlinewidth{0.803000pt}%
\definecolor{currentstroke}{rgb}{0.752941,0.752941,0.752941}%
\pgfsetstrokecolor{currentstroke}%
\pgfsetdash{{2.960000pt}{1.280000pt}}{0.000000pt}%
\pgfpathmoveto{\pgfqpoint{1.650538in}{0.525000in}}%
\pgfpathlineto{\pgfqpoint{1.650538in}{2.162500in}}%
\pgfusepath{stroke}%
\end{pgfscope}%
\begin{pgfscope}%
\pgfsetbuttcap%
\pgfsetroundjoin%
\definecolor{currentfill}{rgb}{0.000000,0.000000,0.000000}%
\pgfsetfillcolor{currentfill}%
\pgfsetlinewidth{0.602250pt}%
\definecolor{currentstroke}{rgb}{0.000000,0.000000,0.000000}%
\pgfsetstrokecolor{currentstroke}%
\pgfsetdash{}{0pt}%
\pgfsys@defobject{currentmarker}{\pgfqpoint{0.000000in}{-0.027778in}}{\pgfqpoint{0.000000in}{0.000000in}}{%
\pgfpathmoveto{\pgfqpoint{0.000000in}{0.000000in}}%
\pgfpathlineto{\pgfqpoint{0.000000in}{-0.027778in}}%
\pgfusepath{stroke,fill}%
}%
\begin{pgfscope}%
\pgfsys@transformshift{1.650538in}{0.525000in}%
\pgfsys@useobject{currentmarker}{}%
\end{pgfscope}%
\end{pgfscope}%
\begin{pgfscope}%
\pgfpathrectangle{\pgfqpoint{0.726250in}{0.525000in}}{\pgfqpoint{1.120000in}{1.637500in}}%
\pgfusepath{clip}%
\pgfsetbuttcap%
\pgfsetroundjoin%
\pgfsetlinewidth{0.803000pt}%
\definecolor{currentstroke}{rgb}{0.752941,0.752941,0.752941}%
\pgfsetstrokecolor{currentstroke}%
\pgfsetdash{{2.960000pt}{1.280000pt}}{0.000000pt}%
\pgfpathmoveto{\pgfqpoint{1.699844in}{0.525000in}}%
\pgfpathlineto{\pgfqpoint{1.699844in}{2.162500in}}%
\pgfusepath{stroke}%
\end{pgfscope}%
\begin{pgfscope}%
\pgfsetbuttcap%
\pgfsetroundjoin%
\definecolor{currentfill}{rgb}{0.000000,0.000000,0.000000}%
\pgfsetfillcolor{currentfill}%
\pgfsetlinewidth{0.602250pt}%
\definecolor{currentstroke}{rgb}{0.000000,0.000000,0.000000}%
\pgfsetstrokecolor{currentstroke}%
\pgfsetdash{}{0pt}%
\pgfsys@defobject{currentmarker}{\pgfqpoint{0.000000in}{-0.027778in}}{\pgfqpoint{0.000000in}{0.000000in}}{%
\pgfpathmoveto{\pgfqpoint{0.000000in}{0.000000in}}%
\pgfpathlineto{\pgfqpoint{0.000000in}{-0.027778in}}%
\pgfusepath{stroke,fill}%
}%
\begin{pgfscope}%
\pgfsys@transformshift{1.699844in}{0.525000in}%
\pgfsys@useobject{currentmarker}{}%
\end{pgfscope}%
\end{pgfscope}%
\begin{pgfscope}%
\pgfpathrectangle{\pgfqpoint{0.726250in}{0.525000in}}{\pgfqpoint{1.120000in}{1.637500in}}%
\pgfusepath{clip}%
\pgfsetbuttcap%
\pgfsetroundjoin%
\pgfsetlinewidth{0.803000pt}%
\definecolor{currentstroke}{rgb}{0.752941,0.752941,0.752941}%
\pgfsetstrokecolor{currentstroke}%
\pgfsetdash{{2.960000pt}{1.280000pt}}{0.000000pt}%
\pgfpathmoveto{\pgfqpoint{1.734827in}{0.525000in}}%
\pgfpathlineto{\pgfqpoint{1.734827in}{2.162500in}}%
\pgfusepath{stroke}%
\end{pgfscope}%
\begin{pgfscope}%
\pgfsetbuttcap%
\pgfsetroundjoin%
\definecolor{currentfill}{rgb}{0.000000,0.000000,0.000000}%
\pgfsetfillcolor{currentfill}%
\pgfsetlinewidth{0.602250pt}%
\definecolor{currentstroke}{rgb}{0.000000,0.000000,0.000000}%
\pgfsetstrokecolor{currentstroke}%
\pgfsetdash{}{0pt}%
\pgfsys@defobject{currentmarker}{\pgfqpoint{0.000000in}{-0.027778in}}{\pgfqpoint{0.000000in}{0.000000in}}{%
\pgfpathmoveto{\pgfqpoint{0.000000in}{0.000000in}}%
\pgfpathlineto{\pgfqpoint{0.000000in}{-0.027778in}}%
\pgfusepath{stroke,fill}%
}%
\begin{pgfscope}%
\pgfsys@transformshift{1.734827in}{0.525000in}%
\pgfsys@useobject{currentmarker}{}%
\end{pgfscope}%
\end{pgfscope}%
\begin{pgfscope}%
\pgfpathrectangle{\pgfqpoint{0.726250in}{0.525000in}}{\pgfqpoint{1.120000in}{1.637500in}}%
\pgfusepath{clip}%
\pgfsetbuttcap%
\pgfsetroundjoin%
\pgfsetlinewidth{0.803000pt}%
\definecolor{currentstroke}{rgb}{0.752941,0.752941,0.752941}%
\pgfsetstrokecolor{currentstroke}%
\pgfsetdash{{2.960000pt}{1.280000pt}}{0.000000pt}%
\pgfpathmoveto{\pgfqpoint{1.761962in}{0.525000in}}%
\pgfpathlineto{\pgfqpoint{1.761962in}{2.162500in}}%
\pgfusepath{stroke}%
\end{pgfscope}%
\begin{pgfscope}%
\pgfsetbuttcap%
\pgfsetroundjoin%
\definecolor{currentfill}{rgb}{0.000000,0.000000,0.000000}%
\pgfsetfillcolor{currentfill}%
\pgfsetlinewidth{0.602250pt}%
\definecolor{currentstroke}{rgb}{0.000000,0.000000,0.000000}%
\pgfsetstrokecolor{currentstroke}%
\pgfsetdash{}{0pt}%
\pgfsys@defobject{currentmarker}{\pgfqpoint{0.000000in}{-0.027778in}}{\pgfqpoint{0.000000in}{0.000000in}}{%
\pgfpathmoveto{\pgfqpoint{0.000000in}{0.000000in}}%
\pgfpathlineto{\pgfqpoint{0.000000in}{-0.027778in}}%
\pgfusepath{stroke,fill}%
}%
\begin{pgfscope}%
\pgfsys@transformshift{1.761962in}{0.525000in}%
\pgfsys@useobject{currentmarker}{}%
\end{pgfscope}%
\end{pgfscope}%
\begin{pgfscope}%
\pgfpathrectangle{\pgfqpoint{0.726250in}{0.525000in}}{\pgfqpoint{1.120000in}{1.637500in}}%
\pgfusepath{clip}%
\pgfsetbuttcap%
\pgfsetroundjoin%
\pgfsetlinewidth{0.803000pt}%
\definecolor{currentstroke}{rgb}{0.752941,0.752941,0.752941}%
\pgfsetstrokecolor{currentstroke}%
\pgfsetdash{{2.960000pt}{1.280000pt}}{0.000000pt}%
\pgfpathmoveto{\pgfqpoint{1.784132in}{0.525000in}}%
\pgfpathlineto{\pgfqpoint{1.784132in}{2.162500in}}%
\pgfusepath{stroke}%
\end{pgfscope}%
\begin{pgfscope}%
\pgfsetbuttcap%
\pgfsetroundjoin%
\definecolor{currentfill}{rgb}{0.000000,0.000000,0.000000}%
\pgfsetfillcolor{currentfill}%
\pgfsetlinewidth{0.602250pt}%
\definecolor{currentstroke}{rgb}{0.000000,0.000000,0.000000}%
\pgfsetstrokecolor{currentstroke}%
\pgfsetdash{}{0pt}%
\pgfsys@defobject{currentmarker}{\pgfqpoint{0.000000in}{-0.027778in}}{\pgfqpoint{0.000000in}{0.000000in}}{%
\pgfpathmoveto{\pgfqpoint{0.000000in}{0.000000in}}%
\pgfpathlineto{\pgfqpoint{0.000000in}{-0.027778in}}%
\pgfusepath{stroke,fill}%
}%
\begin{pgfscope}%
\pgfsys@transformshift{1.784132in}{0.525000in}%
\pgfsys@useobject{currentmarker}{}%
\end{pgfscope}%
\end{pgfscope}%
\begin{pgfscope}%
\pgfpathrectangle{\pgfqpoint{0.726250in}{0.525000in}}{\pgfqpoint{1.120000in}{1.637500in}}%
\pgfusepath{clip}%
\pgfsetbuttcap%
\pgfsetroundjoin%
\pgfsetlinewidth{0.803000pt}%
\definecolor{currentstroke}{rgb}{0.752941,0.752941,0.752941}%
\pgfsetstrokecolor{currentstroke}%
\pgfsetdash{{2.960000pt}{1.280000pt}}{0.000000pt}%
\pgfpathmoveto{\pgfqpoint{1.802877in}{0.525000in}}%
\pgfpathlineto{\pgfqpoint{1.802877in}{2.162500in}}%
\pgfusepath{stroke}%
\end{pgfscope}%
\begin{pgfscope}%
\pgfsetbuttcap%
\pgfsetroundjoin%
\definecolor{currentfill}{rgb}{0.000000,0.000000,0.000000}%
\pgfsetfillcolor{currentfill}%
\pgfsetlinewidth{0.602250pt}%
\definecolor{currentstroke}{rgb}{0.000000,0.000000,0.000000}%
\pgfsetstrokecolor{currentstroke}%
\pgfsetdash{}{0pt}%
\pgfsys@defobject{currentmarker}{\pgfqpoint{0.000000in}{-0.027778in}}{\pgfqpoint{0.000000in}{0.000000in}}{%
\pgfpathmoveto{\pgfqpoint{0.000000in}{0.000000in}}%
\pgfpathlineto{\pgfqpoint{0.000000in}{-0.027778in}}%
\pgfusepath{stroke,fill}%
}%
\begin{pgfscope}%
\pgfsys@transformshift{1.802877in}{0.525000in}%
\pgfsys@useobject{currentmarker}{}%
\end{pgfscope}%
\end{pgfscope}%
\begin{pgfscope}%
\pgfpathrectangle{\pgfqpoint{0.726250in}{0.525000in}}{\pgfqpoint{1.120000in}{1.637500in}}%
\pgfusepath{clip}%
\pgfsetbuttcap%
\pgfsetroundjoin%
\pgfsetlinewidth{0.803000pt}%
\definecolor{currentstroke}{rgb}{0.752941,0.752941,0.752941}%
\pgfsetstrokecolor{currentstroke}%
\pgfsetdash{{2.960000pt}{1.280000pt}}{0.000000pt}%
\pgfpathmoveto{\pgfqpoint{1.819115in}{0.525000in}}%
\pgfpathlineto{\pgfqpoint{1.819115in}{2.162500in}}%
\pgfusepath{stroke}%
\end{pgfscope}%
\begin{pgfscope}%
\pgfsetbuttcap%
\pgfsetroundjoin%
\definecolor{currentfill}{rgb}{0.000000,0.000000,0.000000}%
\pgfsetfillcolor{currentfill}%
\pgfsetlinewidth{0.602250pt}%
\definecolor{currentstroke}{rgb}{0.000000,0.000000,0.000000}%
\pgfsetstrokecolor{currentstroke}%
\pgfsetdash{}{0pt}%
\pgfsys@defobject{currentmarker}{\pgfqpoint{0.000000in}{-0.027778in}}{\pgfqpoint{0.000000in}{0.000000in}}{%
\pgfpathmoveto{\pgfqpoint{0.000000in}{0.000000in}}%
\pgfpathlineto{\pgfqpoint{0.000000in}{-0.027778in}}%
\pgfusepath{stroke,fill}%
}%
\begin{pgfscope}%
\pgfsys@transformshift{1.819115in}{0.525000in}%
\pgfsys@useobject{currentmarker}{}%
\end{pgfscope}%
\end{pgfscope}%
\begin{pgfscope}%
\pgfpathrectangle{\pgfqpoint{0.726250in}{0.525000in}}{\pgfqpoint{1.120000in}{1.637500in}}%
\pgfusepath{clip}%
\pgfsetbuttcap%
\pgfsetroundjoin%
\pgfsetlinewidth{0.803000pt}%
\definecolor{currentstroke}{rgb}{0.752941,0.752941,0.752941}%
\pgfsetstrokecolor{currentstroke}%
\pgfsetdash{{2.960000pt}{1.280000pt}}{0.000000pt}%
\pgfpathmoveto{\pgfqpoint{1.833438in}{0.525000in}}%
\pgfpathlineto{\pgfqpoint{1.833438in}{2.162500in}}%
\pgfusepath{stroke}%
\end{pgfscope}%
\begin{pgfscope}%
\pgfsetbuttcap%
\pgfsetroundjoin%
\definecolor{currentfill}{rgb}{0.000000,0.000000,0.000000}%
\pgfsetfillcolor{currentfill}%
\pgfsetlinewidth{0.602250pt}%
\definecolor{currentstroke}{rgb}{0.000000,0.000000,0.000000}%
\pgfsetstrokecolor{currentstroke}%
\pgfsetdash{}{0pt}%
\pgfsys@defobject{currentmarker}{\pgfqpoint{0.000000in}{-0.027778in}}{\pgfqpoint{0.000000in}{0.000000in}}{%
\pgfpathmoveto{\pgfqpoint{0.000000in}{0.000000in}}%
\pgfpathlineto{\pgfqpoint{0.000000in}{-0.027778in}}%
\pgfusepath{stroke,fill}%
}%
\begin{pgfscope}%
\pgfsys@transformshift{1.833438in}{0.525000in}%
\pgfsys@useobject{currentmarker}{}%
\end{pgfscope}%
\end{pgfscope}%
\begin{pgfscope}%
\definecolor{textcolor}{rgb}{0.000000,0.000000,0.000000}%
\pgfsetstrokecolor{textcolor}%
\pgfsetfillcolor{textcolor}%
\pgftext[x=1.286250in,y=0.251251in,,top]{\color{textcolor}\rmfamily\fontsize{9.000000}{10.800000}\selectfont Energy [keV]}%
\end{pgfscope}%
\begin{pgfscope}%
\pgfpathrectangle{\pgfqpoint{0.726250in}{0.525000in}}{\pgfqpoint{1.120000in}{1.637500in}}%
\pgfusepath{clip}%
\pgfsetbuttcap%
\pgfsetroundjoin%
\pgfsetlinewidth{0.803000pt}%
\definecolor{currentstroke}{rgb}{0.752941,0.752941,0.752941}%
\pgfsetstrokecolor{currentstroke}%
\pgfsetdash{{2.960000pt}{1.280000pt}}{0.000000pt}%
\pgfpathmoveto{\pgfqpoint{0.726250in}{0.525000in}}%
\pgfpathlineto{\pgfqpoint{1.846250in}{0.525000in}}%
\pgfusepath{stroke}%
\end{pgfscope}%
\begin{pgfscope}%
\pgfsetbuttcap%
\pgfsetroundjoin%
\definecolor{currentfill}{rgb}{0.000000,0.000000,0.000000}%
\pgfsetfillcolor{currentfill}%
\pgfsetlinewidth{0.803000pt}%
\definecolor{currentstroke}{rgb}{0.000000,0.000000,0.000000}%
\pgfsetstrokecolor{currentstroke}%
\pgfsetdash{}{0pt}%
\pgfsys@defobject{currentmarker}{\pgfqpoint{-0.048611in}{0.000000in}}{\pgfqpoint{-0.000000in}{0.000000in}}{%
\pgfpathmoveto{\pgfqpoint{-0.000000in}{0.000000in}}%
\pgfpathlineto{\pgfqpoint{-0.048611in}{0.000000in}}%
\pgfusepath{stroke,fill}%
}%
\begin{pgfscope}%
\pgfsys@transformshift{0.726250in}{0.525000in}%
\pgfsys@useobject{currentmarker}{}%
\end{pgfscope}%
\end{pgfscope}%
\begin{pgfscope}%
\definecolor{textcolor}{rgb}{0.000000,0.000000,0.000000}%
\pgfsetstrokecolor{textcolor}%
\pgfsetfillcolor{textcolor}%
\pgftext[x=0.362441in, y=0.477515in, left, base]{\color{textcolor}\rmfamily\fontsize{9.000000}{10.800000}\selectfont \(\displaystyle {10^{-8}}\)}%
\end{pgfscope}%
\begin{pgfscope}%
\pgfpathrectangle{\pgfqpoint{0.726250in}{0.525000in}}{\pgfqpoint{1.120000in}{1.637500in}}%
\pgfusepath{clip}%
\pgfsetbuttcap%
\pgfsetroundjoin%
\pgfsetlinewidth{0.803000pt}%
\definecolor{currentstroke}{rgb}{0.752941,0.752941,0.752941}%
\pgfsetstrokecolor{currentstroke}%
\pgfsetdash{{2.960000pt}{1.280000pt}}{0.000000pt}%
\pgfpathmoveto{\pgfqpoint{0.726250in}{0.822727in}}%
\pgfpathlineto{\pgfqpoint{1.846250in}{0.822727in}}%
\pgfusepath{stroke}%
\end{pgfscope}%
\begin{pgfscope}%
\pgfsetbuttcap%
\pgfsetroundjoin%
\definecolor{currentfill}{rgb}{0.000000,0.000000,0.000000}%
\pgfsetfillcolor{currentfill}%
\pgfsetlinewidth{0.803000pt}%
\definecolor{currentstroke}{rgb}{0.000000,0.000000,0.000000}%
\pgfsetstrokecolor{currentstroke}%
\pgfsetdash{}{0pt}%
\pgfsys@defobject{currentmarker}{\pgfqpoint{-0.048611in}{0.000000in}}{\pgfqpoint{-0.000000in}{0.000000in}}{%
\pgfpathmoveto{\pgfqpoint{-0.000000in}{0.000000in}}%
\pgfpathlineto{\pgfqpoint{-0.048611in}{0.000000in}}%
\pgfusepath{stroke,fill}%
}%
\begin{pgfscope}%
\pgfsys@transformshift{0.726250in}{0.822727in}%
\pgfsys@useobject{currentmarker}{}%
\end{pgfscope}%
\end{pgfscope}%
\begin{pgfscope}%
\definecolor{textcolor}{rgb}{0.000000,0.000000,0.000000}%
\pgfsetstrokecolor{textcolor}%
\pgfsetfillcolor{textcolor}%
\pgftext[x=0.362441in, y=0.775242in, left, base]{\color{textcolor}\rmfamily\fontsize{9.000000}{10.800000}\selectfont \(\displaystyle {10^{-6}}\)}%
\end{pgfscope}%
\begin{pgfscope}%
\pgfpathrectangle{\pgfqpoint{0.726250in}{0.525000in}}{\pgfqpoint{1.120000in}{1.637500in}}%
\pgfusepath{clip}%
\pgfsetbuttcap%
\pgfsetroundjoin%
\pgfsetlinewidth{0.803000pt}%
\definecolor{currentstroke}{rgb}{0.752941,0.752941,0.752941}%
\pgfsetstrokecolor{currentstroke}%
\pgfsetdash{{2.960000pt}{1.280000pt}}{0.000000pt}%
\pgfpathmoveto{\pgfqpoint{0.726250in}{1.120455in}}%
\pgfpathlineto{\pgfqpoint{1.846250in}{1.120455in}}%
\pgfusepath{stroke}%
\end{pgfscope}%
\begin{pgfscope}%
\pgfsetbuttcap%
\pgfsetroundjoin%
\definecolor{currentfill}{rgb}{0.000000,0.000000,0.000000}%
\pgfsetfillcolor{currentfill}%
\pgfsetlinewidth{0.803000pt}%
\definecolor{currentstroke}{rgb}{0.000000,0.000000,0.000000}%
\pgfsetstrokecolor{currentstroke}%
\pgfsetdash{}{0pt}%
\pgfsys@defobject{currentmarker}{\pgfqpoint{-0.048611in}{0.000000in}}{\pgfqpoint{-0.000000in}{0.000000in}}{%
\pgfpathmoveto{\pgfqpoint{-0.000000in}{0.000000in}}%
\pgfpathlineto{\pgfqpoint{-0.048611in}{0.000000in}}%
\pgfusepath{stroke,fill}%
}%
\begin{pgfscope}%
\pgfsys@transformshift{0.726250in}{1.120455in}%
\pgfsys@useobject{currentmarker}{}%
\end{pgfscope}%
\end{pgfscope}%
\begin{pgfscope}%
\definecolor{textcolor}{rgb}{0.000000,0.000000,0.000000}%
\pgfsetstrokecolor{textcolor}%
\pgfsetfillcolor{textcolor}%
\pgftext[x=0.362441in, y=1.072969in, left, base]{\color{textcolor}\rmfamily\fontsize{9.000000}{10.800000}\selectfont \(\displaystyle {10^{-4}}\)}%
\end{pgfscope}%
\begin{pgfscope}%
\pgfpathrectangle{\pgfqpoint{0.726250in}{0.525000in}}{\pgfqpoint{1.120000in}{1.637500in}}%
\pgfusepath{clip}%
\pgfsetbuttcap%
\pgfsetroundjoin%
\pgfsetlinewidth{0.803000pt}%
\definecolor{currentstroke}{rgb}{0.752941,0.752941,0.752941}%
\pgfsetstrokecolor{currentstroke}%
\pgfsetdash{{2.960000pt}{1.280000pt}}{0.000000pt}%
\pgfpathmoveto{\pgfqpoint{0.726250in}{1.418182in}}%
\pgfpathlineto{\pgfqpoint{1.846250in}{1.418182in}}%
\pgfusepath{stroke}%
\end{pgfscope}%
\begin{pgfscope}%
\pgfsetbuttcap%
\pgfsetroundjoin%
\definecolor{currentfill}{rgb}{0.000000,0.000000,0.000000}%
\pgfsetfillcolor{currentfill}%
\pgfsetlinewidth{0.803000pt}%
\definecolor{currentstroke}{rgb}{0.000000,0.000000,0.000000}%
\pgfsetstrokecolor{currentstroke}%
\pgfsetdash{}{0pt}%
\pgfsys@defobject{currentmarker}{\pgfqpoint{-0.048611in}{0.000000in}}{\pgfqpoint{-0.000000in}{0.000000in}}{%
\pgfpathmoveto{\pgfqpoint{-0.000000in}{0.000000in}}%
\pgfpathlineto{\pgfqpoint{-0.048611in}{0.000000in}}%
\pgfusepath{stroke,fill}%
}%
\begin{pgfscope}%
\pgfsys@transformshift{0.726250in}{1.418182in}%
\pgfsys@useobject{currentmarker}{}%
\end{pgfscope}%
\end{pgfscope}%
\begin{pgfscope}%
\definecolor{textcolor}{rgb}{0.000000,0.000000,0.000000}%
\pgfsetstrokecolor{textcolor}%
\pgfsetfillcolor{textcolor}%
\pgftext[x=0.362441in, y=1.370696in, left, base]{\color{textcolor}\rmfamily\fontsize{9.000000}{10.800000}\selectfont \(\displaystyle {10^{-2}}\)}%
\end{pgfscope}%
\begin{pgfscope}%
\pgfpathrectangle{\pgfqpoint{0.726250in}{0.525000in}}{\pgfqpoint{1.120000in}{1.637500in}}%
\pgfusepath{clip}%
\pgfsetbuttcap%
\pgfsetroundjoin%
\pgfsetlinewidth{0.803000pt}%
\definecolor{currentstroke}{rgb}{0.752941,0.752941,0.752941}%
\pgfsetstrokecolor{currentstroke}%
\pgfsetdash{{2.960000pt}{1.280000pt}}{0.000000pt}%
\pgfpathmoveto{\pgfqpoint{0.726250in}{1.715909in}}%
\pgfpathlineto{\pgfqpoint{1.846250in}{1.715909in}}%
\pgfusepath{stroke}%
\end{pgfscope}%
\begin{pgfscope}%
\pgfsetbuttcap%
\pgfsetroundjoin%
\definecolor{currentfill}{rgb}{0.000000,0.000000,0.000000}%
\pgfsetfillcolor{currentfill}%
\pgfsetlinewidth{0.803000pt}%
\definecolor{currentstroke}{rgb}{0.000000,0.000000,0.000000}%
\pgfsetstrokecolor{currentstroke}%
\pgfsetdash{}{0pt}%
\pgfsys@defobject{currentmarker}{\pgfqpoint{-0.048611in}{0.000000in}}{\pgfqpoint{-0.000000in}{0.000000in}}{%
\pgfpathmoveto{\pgfqpoint{-0.000000in}{0.000000in}}%
\pgfpathlineto{\pgfqpoint{-0.048611in}{0.000000in}}%
\pgfusepath{stroke,fill}%
}%
\begin{pgfscope}%
\pgfsys@transformshift{0.726250in}{1.715909in}%
\pgfsys@useobject{currentmarker}{}%
\end{pgfscope}%
\end{pgfscope}%
\begin{pgfscope}%
\definecolor{textcolor}{rgb}{0.000000,0.000000,0.000000}%
\pgfsetstrokecolor{textcolor}%
\pgfsetfillcolor{textcolor}%
\pgftext[x=0.442687in, y=1.668424in, left, base]{\color{textcolor}\rmfamily\fontsize{9.000000}{10.800000}\selectfont \(\displaystyle {10^{0}}\)}%
\end{pgfscope}%
\begin{pgfscope}%
\pgfpathrectangle{\pgfqpoint{0.726250in}{0.525000in}}{\pgfqpoint{1.120000in}{1.637500in}}%
\pgfusepath{clip}%
\pgfsetbuttcap%
\pgfsetroundjoin%
\pgfsetlinewidth{0.803000pt}%
\definecolor{currentstroke}{rgb}{0.752941,0.752941,0.752941}%
\pgfsetstrokecolor{currentstroke}%
\pgfsetdash{{2.960000pt}{1.280000pt}}{0.000000pt}%
\pgfpathmoveto{\pgfqpoint{0.726250in}{2.013636in}}%
\pgfpathlineto{\pgfqpoint{1.846250in}{2.013636in}}%
\pgfusepath{stroke}%
\end{pgfscope}%
\begin{pgfscope}%
\pgfsetbuttcap%
\pgfsetroundjoin%
\definecolor{currentfill}{rgb}{0.000000,0.000000,0.000000}%
\pgfsetfillcolor{currentfill}%
\pgfsetlinewidth{0.803000pt}%
\definecolor{currentstroke}{rgb}{0.000000,0.000000,0.000000}%
\pgfsetstrokecolor{currentstroke}%
\pgfsetdash{}{0pt}%
\pgfsys@defobject{currentmarker}{\pgfqpoint{-0.048611in}{0.000000in}}{\pgfqpoint{-0.000000in}{0.000000in}}{%
\pgfpathmoveto{\pgfqpoint{-0.000000in}{0.000000in}}%
\pgfpathlineto{\pgfqpoint{-0.048611in}{0.000000in}}%
\pgfusepath{stroke,fill}%
}%
\begin{pgfscope}%
\pgfsys@transformshift{0.726250in}{2.013636in}%
\pgfsys@useobject{currentmarker}{}%
\end{pgfscope}%
\end{pgfscope}%
\begin{pgfscope}%
\definecolor{textcolor}{rgb}{0.000000,0.000000,0.000000}%
\pgfsetstrokecolor{textcolor}%
\pgfsetfillcolor{textcolor}%
\pgftext[x=0.442687in, y=1.966151in, left, base]{\color{textcolor}\rmfamily\fontsize{9.000000}{10.800000}\selectfont \(\displaystyle {10^{2}}\)}%
\end{pgfscope}%
\begin{pgfscope}%
\pgfpathrectangle{\pgfqpoint{0.726250in}{0.525000in}}{\pgfqpoint{1.120000in}{1.637500in}}%
\pgfusepath{clip}%
\pgfsetbuttcap%
\pgfsetroundjoin%
\pgfsetlinewidth{1.003750pt}%
\definecolor{currentstroke}{rgb}{0.000000,0.000000,0.000000}%
\pgfsetstrokecolor{currentstroke}%
\pgfsetdash{{3.700000pt}{1.600000pt}}{0.000000pt}%
\pgfpathmoveto{\pgfqpoint{1.073583in}{0.515000in}}%
\pgfpathlineto{\pgfqpoint{1.076957in}{0.531816in}}%
\pgfpathlineto{\pgfqpoint{1.088270in}{0.585635in}}%
\pgfpathlineto{\pgfqpoint{1.099583in}{0.637007in}}%
\pgfpathlineto{\pgfqpoint{1.110896in}{0.686045in}}%
\pgfpathlineto{\pgfqpoint{1.122210in}{0.732854in}}%
\pgfpathlineto{\pgfqpoint{1.133523in}{0.777535in}}%
\pgfpathlineto{\pgfqpoint{1.144836in}{0.820186in}}%
\pgfpathlineto{\pgfqpoint{1.156149in}{0.860898in}}%
\pgfpathlineto{\pgfqpoint{1.167462in}{0.899759in}}%
\pgfpathlineto{\pgfqpoint{1.178775in}{0.936855in}}%
\pgfpathlineto{\pgfqpoint{1.190088in}{0.972264in}}%
\pgfpathlineto{\pgfqpoint{1.201402in}{1.006064in}}%
\pgfpathlineto{\pgfqpoint{1.212715in}{1.038327in}}%
\pgfpathlineto{\pgfqpoint{1.224028in}{1.069124in}}%
\pgfpathlineto{\pgfqpoint{1.235341in}{1.098522in}}%
\pgfpathlineto{\pgfqpoint{1.246654in}{1.126583in}}%
\pgfpathlineto{\pgfqpoint{1.257967in}{1.153369in}}%
\pgfpathlineto{\pgfqpoint{1.269280in}{1.178937in}}%
\pgfpathlineto{\pgfqpoint{1.280593in}{1.203343in}}%
\pgfpathlineto{\pgfqpoint{1.291907in}{1.226640in}}%
\pgfpathlineto{\pgfqpoint{1.303220in}{1.248878in}}%
\pgfpathlineto{\pgfqpoint{1.314533in}{1.270106in}}%
\pgfpathlineto{\pgfqpoint{1.325846in}{1.290368in}}%
\pgfpathlineto{\pgfqpoint{1.337159in}{1.309710in}}%
\pgfpathlineto{\pgfqpoint{1.348472in}{1.328172in}}%
\pgfpathlineto{\pgfqpoint{1.359785in}{1.345795in}}%
\pgfpathlineto{\pgfqpoint{1.371098in}{1.362617in}}%
\pgfpathlineto{\pgfqpoint{1.382412in}{1.378675in}}%
\pgfpathlineto{\pgfqpoint{1.393725in}{1.394003in}}%
\pgfpathlineto{\pgfqpoint{1.405038in}{1.408634in}}%
\pgfpathlineto{\pgfqpoint{1.416351in}{1.422600in}}%
\pgfpathlineto{\pgfqpoint{1.427664in}{1.435931in}}%
\pgfpathlineto{\pgfqpoint{1.438977in}{1.448657in}}%
\pgfpathlineto{\pgfqpoint{1.450290in}{1.460804in}}%
\pgfpathlineto{\pgfqpoint{1.461604in}{1.472399in}}%
\pgfpathlineto{\pgfqpoint{1.472917in}{1.483467in}}%
\pgfpathlineto{\pgfqpoint{1.484230in}{1.494032in}}%
\pgfpathlineto{\pgfqpoint{1.495543in}{1.504116in}}%
\pgfpathlineto{\pgfqpoint{1.506856in}{1.513743in}}%
\pgfpathlineto{\pgfqpoint{1.518169in}{1.522931in}}%
\pgfpathlineto{\pgfqpoint{1.529482in}{1.531703in}}%
\pgfpathlineto{\pgfqpoint{1.540795in}{1.540075in}}%
\pgfpathlineto{\pgfqpoint{1.552109in}{1.548067in}}%
\pgfpathlineto{\pgfqpoint{1.563422in}{1.555696in}}%
\pgfpathlineto{\pgfqpoint{1.574735in}{1.562978in}}%
\pgfpathlineto{\pgfqpoint{1.586048in}{1.569929in}}%
\pgfpathlineto{\pgfqpoint{1.597361in}{1.576564in}}%
\pgfpathlineto{\pgfqpoint{1.608674in}{1.582897in}}%
\pgfpathlineto{\pgfqpoint{1.619987in}{1.588943in}}%
\pgfpathlineto{\pgfqpoint{1.631301in}{1.594713in}}%
\pgfpathlineto{\pgfqpoint{1.642614in}{1.600222in}}%
\pgfpathlineto{\pgfqpoint{1.653927in}{1.605480in}}%
\pgfpathlineto{\pgfqpoint{1.665240in}{1.610499in}}%
\pgfpathlineto{\pgfqpoint{1.676553in}{1.615290in}}%
\pgfpathlineto{\pgfqpoint{1.687866in}{1.619864in}}%
\pgfpathlineto{\pgfqpoint{1.699179in}{1.624229in}}%
\pgfpathlineto{\pgfqpoint{1.710492in}{1.628396in}}%
\pgfpathlineto{\pgfqpoint{1.721806in}{1.632374in}}%
\pgfpathlineto{\pgfqpoint{1.733119in}{1.636170in}}%
\pgfpathlineto{\pgfqpoint{1.744432in}{1.639795in}}%
\pgfpathlineto{\pgfqpoint{1.755745in}{1.643254in}}%
\pgfpathlineto{\pgfqpoint{1.767058in}{1.646556in}}%
\pgfpathlineto{\pgfqpoint{1.778371in}{1.649709in}}%
\pgfpathlineto{\pgfqpoint{1.789684in}{1.652718in}}%
\pgfpathlineto{\pgfqpoint{1.800997in}{1.655590in}}%
\pgfpathlineto{\pgfqpoint{1.812311in}{1.658331in}}%
\pgfpathlineto{\pgfqpoint{1.823624in}{1.660948in}}%
\pgfpathlineto{\pgfqpoint{1.834937in}{1.663446in}}%
\pgfpathlineto{\pgfqpoint{1.846250in}{1.665831in}}%
\pgfusepath{stroke}%
\end{pgfscope}%
\begin{pgfscope}%
\pgfpathrectangle{\pgfqpoint{0.726250in}{0.525000in}}{\pgfqpoint{1.120000in}{1.637500in}}%
\pgfusepath{clip}%
\pgfsetbuttcap%
\pgfsetroundjoin%
\pgfsetlinewidth{1.003750pt}%
\definecolor{currentstroke}{rgb}{0.000000,0.000000,0.000000}%
\pgfsetstrokecolor{currentstroke}%
\pgfsetdash{{3.700000pt}{1.600000pt}}{0.000000pt}%
\pgfpathmoveto{\pgfqpoint{0.726250in}{1.711599in}}%
\pgfpathlineto{\pgfqpoint{0.737563in}{1.711179in}}%
\pgfpathlineto{\pgfqpoint{0.748876in}{1.710718in}}%
\pgfpathlineto{\pgfqpoint{0.760189in}{1.710211in}}%
\pgfpathlineto{\pgfqpoint{0.771503in}{1.709656in}}%
\pgfpathlineto{\pgfqpoint{0.782816in}{1.709046in}}%
\pgfpathlineto{\pgfqpoint{0.794129in}{1.708377in}}%
\pgfpathlineto{\pgfqpoint{0.805442in}{1.707643in}}%
\pgfpathlineto{\pgfqpoint{0.816755in}{1.706837in}}%
\pgfpathlineto{\pgfqpoint{0.828068in}{1.705952in}}%
\pgfpathlineto{\pgfqpoint{0.839381in}{1.704982in}}%
\pgfpathlineto{\pgfqpoint{0.850694in}{1.703916in}}%
\pgfpathlineto{\pgfqpoint{0.862008in}{1.702747in}}%
\pgfpathlineto{\pgfqpoint{0.873321in}{1.701464in}}%
\pgfpathlineto{\pgfqpoint{0.884634in}{1.700055in}}%
\pgfpathlineto{\pgfqpoint{0.895947in}{1.698509in}}%
\pgfpathlineto{\pgfqpoint{0.907260in}{1.696813in}}%
\pgfpathlineto{\pgfqpoint{0.918573in}{1.694951in}}%
\pgfpathlineto{\pgfqpoint{0.929886in}{1.692908in}}%
\pgfpathlineto{\pgfqpoint{0.941199in}{1.690665in}}%
\pgfpathlineto{\pgfqpoint{0.952513in}{1.688204in}}%
\pgfpathlineto{\pgfqpoint{0.963826in}{1.685503in}}%
\pgfpathlineto{\pgfqpoint{0.975139in}{1.682538in}}%
\pgfpathlineto{\pgfqpoint{0.986452in}{1.679284in}}%
\pgfpathlineto{\pgfqpoint{0.997765in}{1.675713in}}%
\pgfpathlineto{\pgfqpoint{1.009078in}{1.671794in}}%
\pgfpathlineto{\pgfqpoint{1.020391in}{1.667493in}}%
\pgfpathlineto{\pgfqpoint{1.031705in}{1.662773in}}%
\pgfpathlineto{\pgfqpoint{1.043018in}{1.657592in}}%
\pgfpathlineto{\pgfqpoint{1.054331in}{1.651906in}}%
\pgfpathlineto{\pgfqpoint{1.065644in}{1.645666in}}%
\pgfpathlineto{\pgfqpoint{1.076957in}{1.638818in}}%
\pgfpathlineto{\pgfqpoint{1.088270in}{1.631301in}}%
\pgfpathlineto{\pgfqpoint{1.099583in}{1.623052in}}%
\pgfpathlineto{\pgfqpoint{1.110896in}{1.613999in}}%
\pgfpathlineto{\pgfqpoint{1.122210in}{1.604062in}}%
\pgfpathlineto{\pgfqpoint{1.133523in}{1.593157in}}%
\pgfpathlineto{\pgfqpoint{1.144836in}{1.581189in}}%
\pgfpathlineto{\pgfqpoint{1.156149in}{1.568054in}}%
\pgfpathlineto{\pgfqpoint{1.167462in}{1.553639in}}%
\pgfpathlineto{\pgfqpoint{1.178775in}{1.537818in}}%
\pgfpathlineto{\pgfqpoint{1.190088in}{1.520454in}}%
\pgfpathlineto{\pgfqpoint{1.201402in}{1.501397in}}%
\pgfpathlineto{\pgfqpoint{1.212715in}{1.480483in}}%
\pgfpathlineto{\pgfqpoint{1.224028in}{1.457529in}}%
\pgfpathlineto{\pgfqpoint{1.235341in}{1.432337in}}%
\pgfpathlineto{\pgfqpoint{1.246654in}{1.404689in}}%
\pgfpathlineto{\pgfqpoint{1.257967in}{1.374345in}}%
\pgfpathlineto{\pgfqpoint{1.269280in}{1.341043in}}%
\pgfpathlineto{\pgfqpoint{1.280593in}{1.304495in}}%
\pgfpathlineto{\pgfqpoint{1.291907in}{1.264382in}}%
\pgfpathlineto{\pgfqpoint{1.303220in}{1.220359in}}%
\pgfpathlineto{\pgfqpoint{1.314533in}{1.172043in}}%
\pgfpathlineto{\pgfqpoint{1.325846in}{1.119017in}}%
\pgfpathlineto{\pgfqpoint{1.337159in}{1.060821in}}%
\pgfpathlineto{\pgfqpoint{1.348472in}{0.996950in}}%
\pgfpathlineto{\pgfqpoint{1.359785in}{0.926853in}}%
\pgfpathlineto{\pgfqpoint{1.371098in}{0.849921in}}%
\pgfpathlineto{\pgfqpoint{1.382412in}{0.765488in}}%
\pgfpathlineto{\pgfqpoint{1.393725in}{0.672823in}}%
\pgfpathlineto{\pgfqpoint{1.405038in}{0.571124in}}%
\pgfpathlineto{\pgfqpoint{1.410726in}{0.515000in}}%
\pgfusepath{stroke}%
\end{pgfscope}%
\begin{pgfscope}%
\pgfpathrectangle{\pgfqpoint{0.726250in}{0.525000in}}{\pgfqpoint{1.120000in}{1.637500in}}%
\pgfusepath{clip}%
\pgfsetrectcap%
\pgfsetroundjoin%
\pgfsetlinewidth{1.003750pt}%
\definecolor{currentstroke}{rgb}{0.000000,0.000000,0.000000}%
\pgfsetstrokecolor{currentstroke}%
\pgfsetdash{}{0pt}%
\pgfpathmoveto{\pgfqpoint{0.907941in}{0.515000in}}%
\pgfpathlineto{\pgfqpoint{0.918573in}{0.614875in}}%
\pgfpathlineto{\pgfqpoint{0.929886in}{0.716052in}}%
\pgfpathlineto{\pgfqpoint{0.941199in}{0.812337in}}%
\pgfpathlineto{\pgfqpoint{0.952513in}{0.903926in}}%
\pgfpathlineto{\pgfqpoint{0.963826in}{0.991000in}}%
\pgfpathlineto{\pgfqpoint{0.975139in}{1.073731in}}%
\pgfpathlineto{\pgfqpoint{0.986452in}{1.152277in}}%
\pgfpathlineto{\pgfqpoint{0.997765in}{1.226788in}}%
\pgfpathlineto{\pgfqpoint{1.009078in}{1.297402in}}%
\pgfpathlineto{\pgfqpoint{1.020391in}{1.364246in}}%
\pgfpathlineto{\pgfqpoint{1.031705in}{1.427437in}}%
\pgfpathlineto{\pgfqpoint{1.043018in}{1.487082in}}%
\pgfpathlineto{\pgfqpoint{1.054331in}{1.543274in}}%
\pgfpathlineto{\pgfqpoint{1.065644in}{1.596100in}}%
\pgfpathlineto{\pgfqpoint{1.076957in}{1.645633in}}%
\pgfpathlineto{\pgfqpoint{1.088270in}{1.691936in}}%
\pgfpathlineto{\pgfqpoint{1.099583in}{1.735059in}}%
\pgfpathlineto{\pgfqpoint{1.110896in}{1.775044in}}%
\pgfpathlineto{\pgfqpoint{1.122210in}{1.811916in}}%
\pgfpathlineto{\pgfqpoint{1.133523in}{1.845693in}}%
\pgfpathlineto{\pgfqpoint{1.144836in}{1.876375in}}%
\pgfpathlineto{\pgfqpoint{1.156149in}{1.903952in}}%
\pgfpathlineto{\pgfqpoint{1.167462in}{1.928398in}}%
\pgfpathlineto{\pgfqpoint{1.178775in}{1.949672in}}%
\pgfpathlineto{\pgfqpoint{1.190088in}{1.967718in}}%
\pgfpathlineto{\pgfqpoint{1.201402in}{1.982461in}}%
\pgfpathlineto{\pgfqpoint{1.212715in}{1.993810in}}%
\pgfpathlineto{\pgfqpoint{1.224028in}{2.001653in}}%
\pgfpathlineto{\pgfqpoint{1.235341in}{2.005859in}}%
\pgfpathlineto{\pgfqpoint{1.246654in}{2.006272in}}%
\pgfpathlineto{\pgfqpoint{1.257967in}{2.002714in}}%
\pgfpathlineto{\pgfqpoint{1.269280in}{1.994981in}}%
\pgfpathlineto{\pgfqpoint{1.280593in}{1.982838in}}%
\pgfpathlineto{\pgfqpoint{1.291907in}{1.966022in}}%
\pgfpathlineto{\pgfqpoint{1.303220in}{1.944237in}}%
\pgfpathlineto{\pgfqpoint{1.314533in}{1.917149in}}%
\pgfpathlineto{\pgfqpoint{1.325846in}{1.884385in}}%
\pgfpathlineto{\pgfqpoint{1.337159in}{1.845530in}}%
\pgfpathlineto{\pgfqpoint{1.348472in}{1.800122in}}%
\pgfpathlineto{\pgfqpoint{1.359785in}{1.747648in}}%
\pgfpathlineto{\pgfqpoint{1.371098in}{1.687538in}}%
\pgfpathlineto{\pgfqpoint{1.382412in}{1.619163in}}%
\pgfpathlineto{\pgfqpoint{1.393725in}{1.541826in}}%
\pgfpathlineto{\pgfqpoint{1.405038in}{1.454758in}}%
\pgfpathlineto{\pgfqpoint{1.416351in}{1.357109in}}%
\pgfpathlineto{\pgfqpoint{1.427664in}{1.247942in}}%
\pgfpathlineto{\pgfqpoint{1.438977in}{1.126227in}}%
\pgfpathlineto{\pgfqpoint{1.450290in}{0.990825in}}%
\pgfpathlineto{\pgfqpoint{1.461604in}{0.840486in}}%
\pgfpathlineto{\pgfqpoint{1.472917in}{0.673831in}}%
\pgfpathlineto{\pgfqpoint{1.482657in}{0.515000in}}%
\pgfusepath{stroke}%
\end{pgfscope}%
\begin{pgfscope}%
\pgfsetrectcap%
\pgfsetmiterjoin%
\pgfsetlinewidth{1.003750pt}%
\definecolor{currentstroke}{rgb}{0.000000,0.000000,0.000000}%
\pgfsetstrokecolor{currentstroke}%
\pgfsetdash{}{0pt}%
\pgfpathmoveto{\pgfqpoint{0.726250in}{0.525000in}}%
\pgfpathlineto{\pgfqpoint{0.726250in}{2.162500in}}%
\pgfusepath{stroke}%
\end{pgfscope}%
\begin{pgfscope}%
\pgfsetrectcap%
\pgfsetmiterjoin%
\pgfsetlinewidth{1.003750pt}%
\definecolor{currentstroke}{rgb}{0.000000,0.000000,0.000000}%
\pgfsetstrokecolor{currentstroke}%
\pgfsetdash{}{0pt}%
\pgfpathmoveto{\pgfqpoint{1.846250in}{0.525000in}}%
\pgfpathlineto{\pgfqpoint{1.846250in}{2.162500in}}%
\pgfusepath{stroke}%
\end{pgfscope}%
\begin{pgfscope}%
\pgfsetrectcap%
\pgfsetmiterjoin%
\pgfsetlinewidth{1.003750pt}%
\definecolor{currentstroke}{rgb}{0.000000,0.000000,0.000000}%
\pgfsetstrokecolor{currentstroke}%
\pgfsetdash{}{0pt}%
\pgfpathmoveto{\pgfqpoint{0.726250in}{0.525000in}}%
\pgfpathlineto{\pgfqpoint{1.846250in}{0.525000in}}%
\pgfusepath{stroke}%
\end{pgfscope}%
\begin{pgfscope}%
\pgfsetrectcap%
\pgfsetmiterjoin%
\pgfsetlinewidth{1.003750pt}%
\definecolor{currentstroke}{rgb}{0.000000,0.000000,0.000000}%
\pgfsetstrokecolor{currentstroke}%
\pgfsetdash{}{0pt}%
\pgfpathmoveto{\pgfqpoint{0.726250in}{2.162500in}}%
\pgfpathlineto{\pgfqpoint{1.846250in}{2.162500in}}%
\pgfusepath{stroke}%
\end{pgfscope}%
\end{pgfpicture}%
\makeatother%
\endgroup%

  \caption{Illustration of the Gamow peak for $E_G = 600$~keV and $kT = 1.5$~keV.}
  \label{fig:gamow_peak}
\end{marginfigure}

Trying to gauge the temperature dependence of the neutrino flux is an interesting
exercise for several reasons that will be clear in a second. If we let
\begin{align*}
  -f_G(E) = \sqrt{\frac{E_G}{E}} + \frac{E}{kT},
\end{align*}
it is easy to show that the function has a minimum (and therefore the Gamow peak
has a maximum) for
\begin{align}
  E_0 = E_G^\frac{1}{3} \qty(\frac{kT}{2})^\frac{2}{3}
  \quad\text{with}\quad
  -f(E_0) = \frac{3E0}{kT} = 3 \qty(\frac{E_G}{4kT})^\frac{1}{3} \coloneqq \tau.
\end{align}
We note that $E_0 \approx 6.5$~keV, and therefore $\tau \approx 13$, for a two-proton reaction.
Under the assumption that the astrophysical factor does not vary significantly within
the energy range subtended by the Gamow peak, we can take $S(E)$ out of the integral
\begin{align*}
  N_A \ave{\sigma v} = N_A \qty(\frac{8}{\pi m_r})^\frac{1}{2} \qty(\frac{1}{kT})^{\frac{3}{2}}
  S(E_0) \int_0^\infty e^{f_G(E)} \diff{E}.
\end{align*}
This is still too difficult to solve analytically, but we can expand $f_G(E)$ in a
Taylor series around $E_0$ up to the second order
\begin{align*}
  -f_G(E) \approx \tau + \frac{\tau}{4E_0^2} (E - E_0)^2
  \quad\text{or}\quad
  e^{f_G(E)} = e^{-\tau} \exp\qty{-\frac{\tau}{4E_0^2} (E - E_0)^2},
\end{align*}
from which it appears evident that all we have done is to approximate the Gamow
peak with a Gaussian with mean $\mu = E_0$ and standard deviation
\begin{align*}
  \sigma = \frac{2E_0}{\sqrt{\tau}} =
  \frac{2}{3} kT \sqrt{\tau} = \frac{2}{\sqrt{3}} (E_0 kT)^\frac{1}{2}.
\end{align*}

\begin{marginfigure}
  %% Creator: Matplotlib, PGF backend
%%
%% To include the figure in your LaTeX document, write
%%   \input{<filename>.pgf}
%%
%% Make sure the required packages are loaded in your preamble
%%   \usepackage{pgf}
%%
%% Also ensure that all the required font packages are loaded; for instance,
%% the lmodern package is sometimes necessary when using math font.
%%   \usepackage{lmodern}
%%
%% Figures using additional raster images can only be included by \input if
%% they are in the same directory as the main LaTeX file. For loading figures
%% from other directories you can use the `import` package
%%   \usepackage{import}
%%
%% and then include the figures with
%%   \import{<path to file>}{<filename>.pgf}
%%
%% Matplotlib used the following preamble
%%   \usepackage{fontspec}
%%   \setmainfont{DejaVuSerif.ttf}[Path=\detokenize{/usr/share/matplotlib/mpl-data/fonts/ttf/}]
%%   \setsansfont{DejaVuSans.ttf}[Path=\detokenize{/usr/share/matplotlib/mpl-data/fonts/ttf/}]
%%   \setmonofont{DejaVuSansMono.ttf}[Path=\detokenize{/usr/share/matplotlib/mpl-data/fonts/ttf/}]
%%
\begingroup%
\makeatletter%
\begin{pgfpicture}%
\pgfpathrectangle{\pgfpointorigin}{\pgfqpoint{1.950000in}{2.250000in}}%
\pgfusepath{use as bounding box, clip}%
\begin{pgfscope}%
\pgfsetbuttcap%
\pgfsetmiterjoin%
\definecolor{currentfill}{rgb}{1.000000,1.000000,1.000000}%
\pgfsetfillcolor{currentfill}%
\pgfsetlinewidth{0.000000pt}%
\definecolor{currentstroke}{rgb}{1.000000,1.000000,1.000000}%
\pgfsetstrokecolor{currentstroke}%
\pgfsetdash{}{0pt}%
\pgfpathmoveto{\pgfqpoint{0.000000in}{0.000000in}}%
\pgfpathlineto{\pgfqpoint{1.950000in}{0.000000in}}%
\pgfpathlineto{\pgfqpoint{1.950000in}{2.250000in}}%
\pgfpathlineto{\pgfqpoint{0.000000in}{2.250000in}}%
\pgfpathlineto{\pgfqpoint{0.000000in}{0.000000in}}%
\pgfpathclose%
\pgfusepath{fill}%
\end{pgfscope}%
\begin{pgfscope}%
\pgfsetbuttcap%
\pgfsetmiterjoin%
\definecolor{currentfill}{rgb}{1.000000,1.000000,1.000000}%
\pgfsetfillcolor{currentfill}%
\pgfsetlinewidth{0.000000pt}%
\definecolor{currentstroke}{rgb}{0.000000,0.000000,0.000000}%
\pgfsetstrokecolor{currentstroke}%
\pgfsetstrokeopacity{0.000000}%
\pgfsetdash{}{0pt}%
\pgfpathmoveto{\pgfqpoint{0.726250in}{0.525000in}}%
\pgfpathlineto{\pgfqpoint{1.846250in}{0.525000in}}%
\pgfpathlineto{\pgfqpoint{1.846250in}{2.162500in}}%
\pgfpathlineto{\pgfqpoint{0.726250in}{2.162500in}}%
\pgfpathlineto{\pgfqpoint{0.726250in}{0.525000in}}%
\pgfpathclose%
\pgfusepath{fill}%
\end{pgfscope}%
\begin{pgfscope}%
\pgfpathrectangle{\pgfqpoint{0.726250in}{0.525000in}}{\pgfqpoint{1.120000in}{1.637500in}}%
\pgfusepath{clip}%
\pgfsetbuttcap%
\pgfsetroundjoin%
\pgfsetlinewidth{0.803000pt}%
\definecolor{currentstroke}{rgb}{0.752941,0.752941,0.752941}%
\pgfsetstrokecolor{currentstroke}%
\pgfsetdash{{2.960000pt}{1.280000pt}}{0.000000pt}%
\pgfpathmoveto{\pgfqpoint{0.726250in}{0.525000in}}%
\pgfpathlineto{\pgfqpoint{0.726250in}{2.162500in}}%
\pgfusepath{stroke}%
\end{pgfscope}%
\begin{pgfscope}%
\pgfsetbuttcap%
\pgfsetroundjoin%
\definecolor{currentfill}{rgb}{0.000000,0.000000,0.000000}%
\pgfsetfillcolor{currentfill}%
\pgfsetlinewidth{0.803000pt}%
\definecolor{currentstroke}{rgb}{0.000000,0.000000,0.000000}%
\pgfsetstrokecolor{currentstroke}%
\pgfsetdash{}{0pt}%
\pgfsys@defobject{currentmarker}{\pgfqpoint{0.000000in}{-0.048611in}}{\pgfqpoint{0.000000in}{0.000000in}}{%
\pgfpathmoveto{\pgfqpoint{0.000000in}{0.000000in}}%
\pgfpathlineto{\pgfqpoint{0.000000in}{-0.048611in}}%
\pgfusepath{stroke,fill}%
}%
\begin{pgfscope}%
\pgfsys@transformshift{0.726250in}{0.525000in}%
\pgfsys@useobject{currentmarker}{}%
\end{pgfscope}%
\end{pgfscope}%
\begin{pgfscope}%
\definecolor{textcolor}{rgb}{0.000000,0.000000,0.000000}%
\pgfsetstrokecolor{textcolor}%
\pgfsetfillcolor{textcolor}%
\pgftext[x=0.726250in,y=0.427778in,,top]{\color{textcolor}\rmfamily\fontsize{9.000000}{10.800000}\selectfont 0}%
\end{pgfscope}%
\begin{pgfscope}%
\pgfpathrectangle{\pgfqpoint{0.726250in}{0.525000in}}{\pgfqpoint{1.120000in}{1.637500in}}%
\pgfusepath{clip}%
\pgfsetbuttcap%
\pgfsetroundjoin%
\pgfsetlinewidth{0.803000pt}%
\definecolor{currentstroke}{rgb}{0.752941,0.752941,0.752941}%
\pgfsetstrokecolor{currentstroke}%
\pgfsetdash{{2.960000pt}{1.280000pt}}{0.000000pt}%
\pgfpathmoveto{\pgfqpoint{1.472917in}{0.525000in}}%
\pgfpathlineto{\pgfqpoint{1.472917in}{2.162500in}}%
\pgfusepath{stroke}%
\end{pgfscope}%
\begin{pgfscope}%
\pgfsetbuttcap%
\pgfsetroundjoin%
\definecolor{currentfill}{rgb}{0.000000,0.000000,0.000000}%
\pgfsetfillcolor{currentfill}%
\pgfsetlinewidth{0.803000pt}%
\definecolor{currentstroke}{rgb}{0.000000,0.000000,0.000000}%
\pgfsetstrokecolor{currentstroke}%
\pgfsetdash{}{0pt}%
\pgfsys@defobject{currentmarker}{\pgfqpoint{0.000000in}{-0.048611in}}{\pgfqpoint{0.000000in}{0.000000in}}{%
\pgfpathmoveto{\pgfqpoint{0.000000in}{0.000000in}}%
\pgfpathlineto{\pgfqpoint{0.000000in}{-0.048611in}}%
\pgfusepath{stroke,fill}%
}%
\begin{pgfscope}%
\pgfsys@transformshift{1.472917in}{0.525000in}%
\pgfsys@useobject{currentmarker}{}%
\end{pgfscope}%
\end{pgfscope}%
\begin{pgfscope}%
\definecolor{textcolor}{rgb}{0.000000,0.000000,0.000000}%
\pgfsetstrokecolor{textcolor}%
\pgfsetfillcolor{textcolor}%
\pgftext[x=1.472917in,y=0.427778in,,top]{\color{textcolor}\rmfamily\fontsize{9.000000}{10.800000}\selectfont 20}%
\end{pgfscope}%
\begin{pgfscope}%
\definecolor{textcolor}{rgb}{0.000000,0.000000,0.000000}%
\pgfsetstrokecolor{textcolor}%
\pgfsetfillcolor{textcolor}%
\pgftext[x=1.286250in,y=0.251251in,,top]{\color{textcolor}\rmfamily\fontsize{9.000000}{10.800000}\selectfont Energy [keV]}%
\end{pgfscope}%
\begin{pgfscope}%
\pgfpathrectangle{\pgfqpoint{0.726250in}{0.525000in}}{\pgfqpoint{1.120000in}{1.637500in}}%
\pgfusepath{clip}%
\pgfsetbuttcap%
\pgfsetroundjoin%
\pgfsetlinewidth{0.803000pt}%
\definecolor{currentstroke}{rgb}{0.752941,0.752941,0.752941}%
\pgfsetstrokecolor{currentstroke}%
\pgfsetdash{{2.960000pt}{1.280000pt}}{0.000000pt}%
\pgfpathmoveto{\pgfqpoint{0.726250in}{0.599432in}}%
\pgfpathlineto{\pgfqpoint{1.846250in}{0.599432in}}%
\pgfusepath{stroke}%
\end{pgfscope}%
\begin{pgfscope}%
\pgfsetbuttcap%
\pgfsetroundjoin%
\definecolor{currentfill}{rgb}{0.000000,0.000000,0.000000}%
\pgfsetfillcolor{currentfill}%
\pgfsetlinewidth{0.803000pt}%
\definecolor{currentstroke}{rgb}{0.000000,0.000000,0.000000}%
\pgfsetstrokecolor{currentstroke}%
\pgfsetdash{}{0pt}%
\pgfsys@defobject{currentmarker}{\pgfqpoint{-0.048611in}{0.000000in}}{\pgfqpoint{-0.000000in}{0.000000in}}{%
\pgfpathmoveto{\pgfqpoint{-0.000000in}{0.000000in}}%
\pgfpathlineto{\pgfqpoint{-0.048611in}{0.000000in}}%
\pgfusepath{stroke,fill}%
}%
\begin{pgfscope}%
\pgfsys@transformshift{0.726250in}{0.599432in}%
\pgfsys@useobject{currentmarker}{}%
\end{pgfscope}%
\end{pgfscope}%
\begin{pgfscope}%
\definecolor{textcolor}{rgb}{0.000000,0.000000,0.000000}%
\pgfsetstrokecolor{textcolor}%
\pgfsetfillcolor{textcolor}%
\pgftext[x=0.549499in, y=0.551946in, left, base]{\color{textcolor}\rmfamily\fontsize{9.000000}{10.800000}\selectfont 0}%
\end{pgfscope}%
\begin{pgfscope}%
\pgfpathrectangle{\pgfqpoint{0.726250in}{0.525000in}}{\pgfqpoint{1.120000in}{1.637500in}}%
\pgfusepath{clip}%
\pgfsetbuttcap%
\pgfsetroundjoin%
\pgfsetlinewidth{0.803000pt}%
\definecolor{currentstroke}{rgb}{0.752941,0.752941,0.752941}%
\pgfsetstrokecolor{currentstroke}%
\pgfsetdash{{2.960000pt}{1.280000pt}}{0.000000pt}%
\pgfpathmoveto{\pgfqpoint{0.726250in}{0.931549in}}%
\pgfpathlineto{\pgfqpoint{1.846250in}{0.931549in}}%
\pgfusepath{stroke}%
\end{pgfscope}%
\begin{pgfscope}%
\pgfsetbuttcap%
\pgfsetroundjoin%
\definecolor{currentfill}{rgb}{0.000000,0.000000,0.000000}%
\pgfsetfillcolor{currentfill}%
\pgfsetlinewidth{0.803000pt}%
\definecolor{currentstroke}{rgb}{0.000000,0.000000,0.000000}%
\pgfsetstrokecolor{currentstroke}%
\pgfsetdash{}{0pt}%
\pgfsys@defobject{currentmarker}{\pgfqpoint{-0.048611in}{0.000000in}}{\pgfqpoint{-0.000000in}{0.000000in}}{%
\pgfpathmoveto{\pgfqpoint{-0.000000in}{0.000000in}}%
\pgfpathlineto{\pgfqpoint{-0.048611in}{0.000000in}}%
\pgfusepath{stroke,fill}%
}%
\begin{pgfscope}%
\pgfsys@transformshift{0.726250in}{0.931549in}%
\pgfsys@useobject{currentmarker}{}%
\end{pgfscope}%
\end{pgfscope}%
\begin{pgfscope}%
\definecolor{textcolor}{rgb}{0.000000,0.000000,0.000000}%
\pgfsetstrokecolor{textcolor}%
\pgfsetfillcolor{textcolor}%
\pgftext[x=0.469970in, y=0.884064in, left, base]{\color{textcolor}\rmfamily\fontsize{9.000000}{10.800000}\selectfont 20}%
\end{pgfscope}%
\begin{pgfscope}%
\pgfpathrectangle{\pgfqpoint{0.726250in}{0.525000in}}{\pgfqpoint{1.120000in}{1.637500in}}%
\pgfusepath{clip}%
\pgfsetbuttcap%
\pgfsetroundjoin%
\pgfsetlinewidth{0.803000pt}%
\definecolor{currentstroke}{rgb}{0.752941,0.752941,0.752941}%
\pgfsetstrokecolor{currentstroke}%
\pgfsetdash{{2.960000pt}{1.280000pt}}{0.000000pt}%
\pgfpathmoveto{\pgfqpoint{0.726250in}{1.263667in}}%
\pgfpathlineto{\pgfqpoint{1.846250in}{1.263667in}}%
\pgfusepath{stroke}%
\end{pgfscope}%
\begin{pgfscope}%
\pgfsetbuttcap%
\pgfsetroundjoin%
\definecolor{currentfill}{rgb}{0.000000,0.000000,0.000000}%
\pgfsetfillcolor{currentfill}%
\pgfsetlinewidth{0.803000pt}%
\definecolor{currentstroke}{rgb}{0.000000,0.000000,0.000000}%
\pgfsetstrokecolor{currentstroke}%
\pgfsetdash{}{0pt}%
\pgfsys@defobject{currentmarker}{\pgfqpoint{-0.048611in}{0.000000in}}{\pgfqpoint{-0.000000in}{0.000000in}}{%
\pgfpathmoveto{\pgfqpoint{-0.000000in}{0.000000in}}%
\pgfpathlineto{\pgfqpoint{-0.048611in}{0.000000in}}%
\pgfusepath{stroke,fill}%
}%
\begin{pgfscope}%
\pgfsys@transformshift{0.726250in}{1.263667in}%
\pgfsys@useobject{currentmarker}{}%
\end{pgfscope}%
\end{pgfscope}%
\begin{pgfscope}%
\definecolor{textcolor}{rgb}{0.000000,0.000000,0.000000}%
\pgfsetstrokecolor{textcolor}%
\pgfsetfillcolor{textcolor}%
\pgftext[x=0.469970in, y=1.216182in, left, base]{\color{textcolor}\rmfamily\fontsize{9.000000}{10.800000}\selectfont 40}%
\end{pgfscope}%
\begin{pgfscope}%
\pgfpathrectangle{\pgfqpoint{0.726250in}{0.525000in}}{\pgfqpoint{1.120000in}{1.637500in}}%
\pgfusepath{clip}%
\pgfsetbuttcap%
\pgfsetroundjoin%
\pgfsetlinewidth{0.803000pt}%
\definecolor{currentstroke}{rgb}{0.752941,0.752941,0.752941}%
\pgfsetstrokecolor{currentstroke}%
\pgfsetdash{{2.960000pt}{1.280000pt}}{0.000000pt}%
\pgfpathmoveto{\pgfqpoint{0.726250in}{1.595785in}}%
\pgfpathlineto{\pgfqpoint{1.846250in}{1.595785in}}%
\pgfusepath{stroke}%
\end{pgfscope}%
\begin{pgfscope}%
\pgfsetbuttcap%
\pgfsetroundjoin%
\definecolor{currentfill}{rgb}{0.000000,0.000000,0.000000}%
\pgfsetfillcolor{currentfill}%
\pgfsetlinewidth{0.803000pt}%
\definecolor{currentstroke}{rgb}{0.000000,0.000000,0.000000}%
\pgfsetstrokecolor{currentstroke}%
\pgfsetdash{}{0pt}%
\pgfsys@defobject{currentmarker}{\pgfqpoint{-0.048611in}{0.000000in}}{\pgfqpoint{-0.000000in}{0.000000in}}{%
\pgfpathmoveto{\pgfqpoint{-0.000000in}{0.000000in}}%
\pgfpathlineto{\pgfqpoint{-0.048611in}{0.000000in}}%
\pgfusepath{stroke,fill}%
}%
\begin{pgfscope}%
\pgfsys@transformshift{0.726250in}{1.595785in}%
\pgfsys@useobject{currentmarker}{}%
\end{pgfscope}%
\end{pgfscope}%
\begin{pgfscope}%
\definecolor{textcolor}{rgb}{0.000000,0.000000,0.000000}%
\pgfsetstrokecolor{textcolor}%
\pgfsetfillcolor{textcolor}%
\pgftext[x=0.469970in, y=1.548299in, left, base]{\color{textcolor}\rmfamily\fontsize{9.000000}{10.800000}\selectfont 60}%
\end{pgfscope}%
\begin{pgfscope}%
\pgfpathrectangle{\pgfqpoint{0.726250in}{0.525000in}}{\pgfqpoint{1.120000in}{1.637500in}}%
\pgfusepath{clip}%
\pgfsetbuttcap%
\pgfsetroundjoin%
\pgfsetlinewidth{0.803000pt}%
\definecolor{currentstroke}{rgb}{0.752941,0.752941,0.752941}%
\pgfsetstrokecolor{currentstroke}%
\pgfsetdash{{2.960000pt}{1.280000pt}}{0.000000pt}%
\pgfpathmoveto{\pgfqpoint{0.726250in}{1.927902in}}%
\pgfpathlineto{\pgfqpoint{1.846250in}{1.927902in}}%
\pgfusepath{stroke}%
\end{pgfscope}%
\begin{pgfscope}%
\pgfsetbuttcap%
\pgfsetroundjoin%
\definecolor{currentfill}{rgb}{0.000000,0.000000,0.000000}%
\pgfsetfillcolor{currentfill}%
\pgfsetlinewidth{0.803000pt}%
\definecolor{currentstroke}{rgb}{0.000000,0.000000,0.000000}%
\pgfsetstrokecolor{currentstroke}%
\pgfsetdash{}{0pt}%
\pgfsys@defobject{currentmarker}{\pgfqpoint{-0.048611in}{0.000000in}}{\pgfqpoint{-0.000000in}{0.000000in}}{%
\pgfpathmoveto{\pgfqpoint{-0.000000in}{0.000000in}}%
\pgfpathlineto{\pgfqpoint{-0.048611in}{0.000000in}}%
\pgfusepath{stroke,fill}%
}%
\begin{pgfscope}%
\pgfsys@transformshift{0.726250in}{1.927902in}%
\pgfsys@useobject{currentmarker}{}%
\end{pgfscope}%
\end{pgfscope}%
\begin{pgfscope}%
\definecolor{textcolor}{rgb}{0.000000,0.000000,0.000000}%
\pgfsetstrokecolor{textcolor}%
\pgfsetfillcolor{textcolor}%
\pgftext[x=0.469970in, y=1.880417in, left, base]{\color{textcolor}\rmfamily\fontsize{9.000000}{10.800000}\selectfont 80}%
\end{pgfscope}%
\begin{pgfscope}%
\pgfpathrectangle{\pgfqpoint{0.726250in}{0.525000in}}{\pgfqpoint{1.120000in}{1.637500in}}%
\pgfusepath{clip}%
\pgfsetrectcap%
\pgfsetroundjoin%
\pgfsetlinewidth{1.003750pt}%
\definecolor{currentstroke}{rgb}{0.000000,0.000000,0.000000}%
\pgfsetstrokecolor{currentstroke}%
\pgfsetdash{}{0pt}%
\pgfpathmoveto{\pgfqpoint{0.726250in}{0.599432in}}%
\pgfpathlineto{\pgfqpoint{0.776903in}{0.599927in}}%
\pgfpathlineto{\pgfqpoint{0.788160in}{0.602446in}}%
\pgfpathlineto{\pgfqpoint{0.793788in}{0.605559in}}%
\pgfpathlineto{\pgfqpoint{0.799416in}{0.610753in}}%
\pgfpathlineto{\pgfqpoint{0.805044in}{0.618781in}}%
\pgfpathlineto{\pgfqpoint{0.810672in}{0.630427in}}%
\pgfpathlineto{\pgfqpoint{0.816300in}{0.646452in}}%
\pgfpathlineto{\pgfqpoint{0.821928in}{0.667533in}}%
\pgfpathlineto{\pgfqpoint{0.827557in}{0.694215in}}%
\pgfpathlineto{\pgfqpoint{0.838813in}{0.765708in}}%
\pgfpathlineto{\pgfqpoint{0.850069in}{0.861640in}}%
\pgfpathlineto{\pgfqpoint{0.861325in}{0.979651in}}%
\pgfpathlineto{\pgfqpoint{0.878210in}{1.186843in}}%
\pgfpathlineto{\pgfqpoint{0.917607in}{1.689483in}}%
\pgfpathlineto{\pgfqpoint{0.928863in}{1.808744in}}%
\pgfpathlineto{\pgfqpoint{0.940119in}{1.908722in}}%
\pgfpathlineto{\pgfqpoint{0.951376in}{1.987084in}}%
\pgfpathlineto{\pgfqpoint{0.962632in}{2.042836in}}%
\pgfpathlineto{\pgfqpoint{0.968260in}{2.062243in}}%
\pgfpathlineto{\pgfqpoint{0.973888in}{2.076135in}}%
\pgfpathlineto{\pgfqpoint{0.979516in}{2.084675in}}%
\pgfpathlineto{\pgfqpoint{0.985144in}{2.088068in}}%
\pgfpathlineto{\pgfqpoint{0.990773in}{2.086556in}}%
\pgfpathlineto{\pgfqpoint{0.996401in}{2.080410in}}%
\pgfpathlineto{\pgfqpoint{1.002029in}{2.069923in}}%
\pgfpathlineto{\pgfqpoint{1.007657in}{2.055403in}}%
\pgfpathlineto{\pgfqpoint{1.018913in}{2.015558in}}%
\pgfpathlineto{\pgfqpoint{1.030170in}{1.963485in}}%
\pgfpathlineto{\pgfqpoint{1.041426in}{1.901761in}}%
\pgfpathlineto{\pgfqpoint{1.058310in}{1.796380in}}%
\pgfpathlineto{\pgfqpoint{1.092079in}{1.565047in}}%
\pgfpathlineto{\pgfqpoint{1.120220in}{1.376008in}}%
\pgfpathlineto{\pgfqpoint{1.142732in}{1.238179in}}%
\pgfpathlineto{\pgfqpoint{1.165245in}{1.116419in}}%
\pgfpathlineto{\pgfqpoint{1.182129in}{1.036479in}}%
\pgfpathlineto{\pgfqpoint{1.199014in}{0.966242in}}%
\pgfpathlineto{\pgfqpoint{1.215898in}{0.905270in}}%
\pgfpathlineto{\pgfqpoint{1.232783in}{0.852898in}}%
\pgfpathlineto{\pgfqpoint{1.249667in}{0.808333in}}%
\pgfpathlineto{\pgfqpoint{1.266552in}{0.770726in}}%
\pgfpathlineto{\pgfqpoint{1.283436in}{0.739228in}}%
\pgfpathlineto{\pgfqpoint{1.300320in}{0.713024in}}%
\pgfpathlineto{\pgfqpoint{1.317205in}{0.691358in}}%
\pgfpathlineto{\pgfqpoint{1.334089in}{0.673545in}}%
\pgfpathlineto{\pgfqpoint{1.350974in}{0.658974in}}%
\pgfpathlineto{\pgfqpoint{1.367858in}{0.647111in}}%
\pgfpathlineto{\pgfqpoint{1.384742in}{0.637494in}}%
\pgfpathlineto{\pgfqpoint{1.407255in}{0.627494in}}%
\pgfpathlineto{\pgfqpoint{1.429768in}{0.620024in}}%
\pgfpathlineto{\pgfqpoint{1.457908in}{0.613333in}}%
\pgfpathlineto{\pgfqpoint{1.491677in}{0.608037in}}%
\pgfpathlineto{\pgfqpoint{1.531074in}{0.604301in}}%
\pgfpathlineto{\pgfqpoint{1.587356in}{0.601556in}}%
\pgfpathlineto{\pgfqpoint{1.671778in}{0.600026in}}%
\pgfpathlineto{\pgfqpoint{1.846250in}{0.599471in}}%
\pgfpathlineto{\pgfqpoint{1.846250in}{0.599471in}}%
\pgfusepath{stroke}%
\end{pgfscope}%
\begin{pgfscope}%
\pgfpathrectangle{\pgfqpoint{0.726250in}{0.525000in}}{\pgfqpoint{1.120000in}{1.637500in}}%
\pgfusepath{clip}%
\pgfsetbuttcap%
\pgfsetroundjoin%
\pgfsetlinewidth{1.003750pt}%
\definecolor{currentstroke}{rgb}{0.000000,0.000000,0.000000}%
\pgfsetstrokecolor{currentstroke}%
\pgfsetdash{{3.700000pt}{1.600000pt}}{0.000000pt}%
\pgfpathmoveto{\pgfqpoint{0.726250in}{0.859682in}}%
\pgfpathlineto{\pgfqpoint{0.743134in}{0.923329in}}%
\pgfpathlineto{\pgfqpoint{0.760019in}{0.996664in}}%
\pgfpathlineto{\pgfqpoint{0.776903in}{1.079497in}}%
\pgfpathlineto{\pgfqpoint{0.799416in}{1.203449in}}%
\pgfpathlineto{\pgfqpoint{0.827557in}{1.375293in}}%
\pgfpathlineto{\pgfqpoint{0.889466in}{1.765203in}}%
\pgfpathlineto{\pgfqpoint{0.906351in}{1.858217in}}%
\pgfpathlineto{\pgfqpoint{0.923235in}{1.938828in}}%
\pgfpathlineto{\pgfqpoint{0.934491in}{1.984069in}}%
\pgfpathlineto{\pgfqpoint{0.945747in}{2.021522in}}%
\pgfpathlineto{\pgfqpoint{0.957004in}{2.050482in}}%
\pgfpathlineto{\pgfqpoint{0.968260in}{2.070396in}}%
\pgfpathlineto{\pgfqpoint{0.973888in}{2.076833in}}%
\pgfpathlineto{\pgfqpoint{0.979516in}{2.080878in}}%
\pgfpathlineto{\pgfqpoint{0.985144in}{2.082511in}}%
\pgfpathlineto{\pgfqpoint{0.990773in}{2.081724in}}%
\pgfpathlineto{\pgfqpoint{0.996401in}{2.078521in}}%
\pgfpathlineto{\pgfqpoint{1.002029in}{2.072917in}}%
\pgfpathlineto{\pgfqpoint{1.007657in}{2.064941in}}%
\pgfpathlineto{\pgfqpoint{1.018913in}{2.042035in}}%
\pgfpathlineto{\pgfqpoint{1.030170in}{2.010245in}}%
\pgfpathlineto{\pgfqpoint{1.041426in}{1.970175in}}%
\pgfpathlineto{\pgfqpoint{1.052682in}{1.922576in}}%
\pgfpathlineto{\pgfqpoint{1.069567in}{1.838989in}}%
\pgfpathlineto{\pgfqpoint{1.086451in}{1.743746in}}%
\pgfpathlineto{\pgfqpoint{1.114592in}{1.568819in}}%
\pgfpathlineto{\pgfqpoint{1.165245in}{1.248286in}}%
\pgfpathlineto{\pgfqpoint{1.187758in}{1.119698in}}%
\pgfpathlineto{\pgfqpoint{1.210270in}{1.005838in}}%
\pgfpathlineto{\pgfqpoint{1.227155in}{0.931378in}}%
\pgfpathlineto{\pgfqpoint{1.244039in}{0.866606in}}%
\pgfpathlineto{\pgfqpoint{1.260923in}{0.811337in}}%
\pgfpathlineto{\pgfqpoint{1.277808in}{0.765049in}}%
\pgfpathlineto{\pgfqpoint{1.294692in}{0.726985in}}%
\pgfpathlineto{\pgfqpoint{1.311577in}{0.696236in}}%
\pgfpathlineto{\pgfqpoint{1.328461in}{0.671829in}}%
\pgfpathlineto{\pgfqpoint{1.345345in}{0.652785in}}%
\pgfpathlineto{\pgfqpoint{1.362230in}{0.638178in}}%
\pgfpathlineto{\pgfqpoint{1.379114in}{0.627159in}}%
\pgfpathlineto{\pgfqpoint{1.395999in}{0.618985in}}%
\pgfpathlineto{\pgfqpoint{1.412883in}{0.613019in}}%
\pgfpathlineto{\pgfqpoint{1.435396in}{0.607606in}}%
\pgfpathlineto{\pgfqpoint{1.463536in}{0.603606in}}%
\pgfpathlineto{\pgfqpoint{1.497305in}{0.601198in}}%
\pgfpathlineto{\pgfqpoint{1.553587in}{0.599801in}}%
\pgfpathlineto{\pgfqpoint{1.699918in}{0.599435in}}%
\pgfpathlineto{\pgfqpoint{1.846250in}{0.599432in}}%
\pgfpathlineto{\pgfqpoint{1.846250in}{0.599432in}}%
\pgfusepath{stroke}%
\end{pgfscope}%
\begin{pgfscope}%
\pgfsetrectcap%
\pgfsetmiterjoin%
\pgfsetlinewidth{1.003750pt}%
\definecolor{currentstroke}{rgb}{0.000000,0.000000,0.000000}%
\pgfsetstrokecolor{currentstroke}%
\pgfsetdash{}{0pt}%
\pgfpathmoveto{\pgfqpoint{0.726250in}{0.525000in}}%
\pgfpathlineto{\pgfqpoint{0.726250in}{2.162500in}}%
\pgfusepath{stroke}%
\end{pgfscope}%
\begin{pgfscope}%
\pgfsetrectcap%
\pgfsetmiterjoin%
\pgfsetlinewidth{1.003750pt}%
\definecolor{currentstroke}{rgb}{0.000000,0.000000,0.000000}%
\pgfsetstrokecolor{currentstroke}%
\pgfsetdash{}{0pt}%
\pgfpathmoveto{\pgfqpoint{1.846250in}{0.525000in}}%
\pgfpathlineto{\pgfqpoint{1.846250in}{2.162500in}}%
\pgfusepath{stroke}%
\end{pgfscope}%
\begin{pgfscope}%
\pgfsetrectcap%
\pgfsetmiterjoin%
\pgfsetlinewidth{1.003750pt}%
\definecolor{currentstroke}{rgb}{0.000000,0.000000,0.000000}%
\pgfsetstrokecolor{currentstroke}%
\pgfsetdash{}{0pt}%
\pgfpathmoveto{\pgfqpoint{0.726250in}{0.525000in}}%
\pgfpathlineto{\pgfqpoint{1.846250in}{0.525000in}}%
\pgfusepath{stroke}%
\end{pgfscope}%
\begin{pgfscope}%
\pgfsetrectcap%
\pgfsetmiterjoin%
\pgfsetlinewidth{1.003750pt}%
\definecolor{currentstroke}{rgb}{0.000000,0.000000,0.000000}%
\pgfsetstrokecolor{currentstroke}%
\pgfsetdash{}{0pt}%
\pgfpathmoveto{\pgfqpoint{0.726250in}{2.162500in}}%
\pgfpathlineto{\pgfqpoint{1.846250in}{2.162500in}}%
\pgfusepath{stroke}%
\end{pgfscope}%
\end{pgfpicture}%
\makeatother%
\endgroup%

  \caption{Illustration of the Gaussian approximation of the Gamow peak for
  $E_G = 600$~keV and $kT = 1.5$~keV.}
  \label{fig:gamow_peak_gaussian}
\end{marginfigure}

The standard deviation of our approximate Gaussian is germane to the harmonic mean
of $kT$ and $E_0$, and always smaller than the latter---in the two-proton case it
turns out to be $3.7$~keV, so that the left intergration extreme is about
$2\sigma$ away from the mean, and we are not very far off if we perform the integration
over the entire real axis, which now can be done analytically and amounts to
\begin{align*}
  \int_0^\infty e^{f_G(E)} \diff{E} = \sqrt{\pi} e^{-\tau} \frac{2E_0}{\sqrt{\tau}}.
\end{align*}
This all said, we have an expression for our reaction rate in the form\sidenote{The
relevant scalings here, are $\tau \propto T^{-\frac{1}{3}}$, and
$E_0 \propto T^\frac{2}{3}$.}
\begin{align}
  R \propto \tau^2 e^{-\tau}
\end{align}
and if, for any given value of $\tau$, we want to calculate the index of the power
law that best fits the temperature dependence of the neutrino rate, all we have
to do is calculate the logarithmic derivative
\begin{align}
  \dv{\log R}{\log T} = 2 \dv{\log \tau}{\log T} - \dv{\tau}{\log T} =
  (2 - \tau) \dv{\log \tau}{\log T} = \frac{\tau - 2}{3}.
\end{align}
For the two-proton case, where $\tau \approx 14$, the neutrino production rate scales
roughly as the fourth power of the temperature; for heavier nuclei the scaling is
even more violent, and we shall get back to this in the next section.\todo{We should
clarify the connection between this and the scaling of the different neutrino
spectra with temperature.}


\subsection{Neutrino production in the Sun}

Neutrinos are produced in the Sun via thermonuclear reactions primarily via the
so-called $pp$~cycle, with a subdominant contribution from the fusion of heavier
elements---the CNO~cycle, that is only important for stars heavier than the Sun
and we shall not describe in any detail here. The $pp$~cycle consist in the fusion
of protons into He~nuclei to the chain of reactions listed in table~\ref{tab:pp_cycle},
that we shall briefly describe in the remaining of this section.

\bgroup
\def\arraystretch{1.25}
\begin{table}[!htbp]
  \begin{tabular}{lllll}
    \hline
    No. & Reaction & Q [MeV] & $\nu_e$ energy [MeV] & $\tau$~[yr]\\
    \hline
    \hline
    1 & $\ce{^1H} (p,~e^+ \nu_e) \ce{^2H}$ & 1.442 & $< 0.423~\ave{0.265}$ & $10^{10}$\\
    2 & $\ce{^1H} (pe^-,~\nu_e) \ce{^2H}$ & 1.442 & 1.442 & $10^{12}$\\
    3 & $\ce{^2H} (p,~\gamma) \ce{^3He}$ & 5.494 & - & $10^{-8}$\\
    4 & $\ce{^3He} (p,~e^+ \nu_e) \ce{^4He}$ & 19.795 & $< 19.795~\ave{9.625}$ & $10^{12}$\\
    5 & $\ce{^3He} (\ce{^3He},~2p) \ce{^4He}$ & 12.860 & - & $10^{5}$\\
    6 & $\ce{^3He} (\ce{^4He},~\gamma) \ce{^7Be}$ & 1.586 & - & $10^{6}$\\
    7 & $\ce{^7Be} (e^-,~\nu_e) \ce{^7Li}$ & 0.384,~0.862 & 0.384,~0.862 & $10^{-1}$\\
    8 & $\ce{^7Li} (p,~\alpha) \ce{^4He}$ & 17.347 & - & $10^{-5}$\\
    9 & $\ce{^7Be} (p,~\gamma) \ce{^8B}$ & 0.137 & - & $10^{2}$\\
    10 & $\ce{^8B} (e^+ \nu_e) \ce{^8Be}^* (\alpha) \ce{^4He}$ & 17.980 & $< 17.980~\ave{6.710}$ & $10^{-8}$\\
    \hline
  \end{tabular}
  \caption{Basic nuclear reactions of the $pp$~cycle in the Sun, with the associated
  $Q$, the energy going into neutrinos (when relevant) and the approximate lifetime
  (adapted from~\cite{1988RvMP...60..297B}). Note that, due to space constraints,
  we use the compact notation, where
  \begin{align*}
    A (b, c) D
  \end{align*}
  is equivalent to
  \begin{align*}
    A + b \rightarrow c + D.
  \end{align*}}
  \label{tab:pp_cycle}
\end{table}
\egroup

As we said, the simplest nuclear fusion process is the fusion of two protons into a
deuterium nucleus (i.e., a nucleus comprise of a proton and a neutron)
\begin{align*}
  p + p \rightarrow \ce{^2H} + e^+ + \nu_e + 0.423~\text{MeV},
\end{align*}
producing a netrino with a continuous spectrum ($pp$) extending to 0.423~MeV and
and average energy of 0.265~MeV. Since the positron annihilates immediately with
an electron of the environment into two gamma rays, the very same reaction is
sometimes written as
\begin{align*}
  p + e^- + p \rightarrow \ce{^2H} + \nu_e + 2\gamma + 1.442~\text{MeV},
\end{align*}
where the $Q$ value of the reaction now explicitely includes the energy going into
the two photons in the final state, whereas the maximum neutrino energy is still
$0.423$~MeV\sidenote{Interestingly enough, a deuton can be also formed by electron
capture in the so called $pep$ reaction
\begin{align*}
  p + e^- + p \rightarrow \ce{^2H} + \nu_e + 1.442~\text{MeV}
\end{align*}
which proceeds with no photons in the final state, producing monochromatic neutrinos
at $1.44$~MeV. Being a real three-body reaction, it is much less frequent in the Sun.}.
We note that the annihilation gamma rays are not always explicitely indicated in
the reaction because, as opposed to the neutrinos (which can traverse the Sun largely
undisturbed), they are quickly degraded in energy and contribute to the thermonuclear
heat being generated.

Being mediated by the weak nuclear interaction, the deuton production is comparatively
slow. Once created, though, the deuton does not last very long, producing a $\ce{^3He}$
isotope via a reaction meadiated by the strong nuclear force (and therefore much faster)
\begin{align*}
  \ce{^2H} + p \rightarrow \ce{^3He} + \gamma + 5.494~\text{MeV}.
\end{align*}

\begin{figure}[!htbp]
  %% Creator: Matplotlib, PGF backend
%%
%% To include the figure in your LaTeX document, write
%%   \input{<filename>.pgf}
%%
%% Make sure the required packages are loaded in your preamble
%%   \usepackage{pgf}
%%
%% Also ensure that all the required font packages are loaded; for instance,
%% the lmodern package is sometimes necessary when using math font.
%%   \usepackage{lmodern}
%%
%% Figures using additional raster images can only be included by \input if
%% they are in the same directory as the main LaTeX file. For loading figures
%% from other directories you can use the `import` package
%%   \usepackage{import}
%%
%% and then include the figures with
%%   \import{<path to file>}{<filename>.pgf}
%%
%% Matplotlib used the following preamble
%%   \usepackage{fontspec}
%%   \setmainfont{DejaVuSerif.ttf}[Path=\detokenize{/usr/share/matplotlib/mpl-data/fonts/ttf/}]
%%   \setsansfont{DejaVuSans.ttf}[Path=\detokenize{/usr/share/matplotlib/mpl-data/fonts/ttf/}]
%%   \setmonofont{DejaVuSansMono.ttf}[Path=\detokenize{/usr/share/matplotlib/mpl-data/fonts/ttf/}]
%%
\begingroup%
\makeatletter%
\begin{pgfpicture}%
\pgfpathrectangle{\pgfpointorigin}{\pgfqpoint{4.150000in}{3.500000in}}%
\pgfusepath{use as bounding box, clip}%
\begin{pgfscope}%
\pgfsetbuttcap%
\pgfsetmiterjoin%
\definecolor{currentfill}{rgb}{1.000000,1.000000,1.000000}%
\pgfsetfillcolor{currentfill}%
\pgfsetlinewidth{0.000000pt}%
\definecolor{currentstroke}{rgb}{1.000000,1.000000,1.000000}%
\pgfsetstrokecolor{currentstroke}%
\pgfsetdash{}{0pt}%
\pgfpathmoveto{\pgfqpoint{0.000000in}{0.000000in}}%
\pgfpathlineto{\pgfqpoint{4.150000in}{0.000000in}}%
\pgfpathlineto{\pgfqpoint{4.150000in}{3.500000in}}%
\pgfpathlineto{\pgfqpoint{0.000000in}{3.500000in}}%
\pgfpathlineto{\pgfqpoint{0.000000in}{0.000000in}}%
\pgfpathclose%
\pgfusepath{fill}%
\end{pgfscope}%
\begin{pgfscope}%
\pgfsetbuttcap%
\pgfsetmiterjoin%
\definecolor{currentfill}{rgb}{1.000000,1.000000,1.000000}%
\pgfsetfillcolor{currentfill}%
\pgfsetlinewidth{0.000000pt}%
\definecolor{currentstroke}{rgb}{0.000000,0.000000,0.000000}%
\pgfsetstrokecolor{currentstroke}%
\pgfsetstrokeopacity{0.000000}%
\pgfsetdash{}{0pt}%
\pgfpathmoveto{\pgfqpoint{0.726250in}{0.525000in}}%
\pgfpathlineto{\pgfqpoint{4.046250in}{0.525000in}}%
\pgfpathlineto{\pgfqpoint{4.046250in}{3.412500in}}%
\pgfpathlineto{\pgfqpoint{0.726250in}{3.412500in}}%
\pgfpathlineto{\pgfqpoint{0.726250in}{0.525000in}}%
\pgfpathclose%
\pgfusepath{fill}%
\end{pgfscope}%
\begin{pgfscope}%
\pgfpathrectangle{\pgfqpoint{0.726250in}{0.525000in}}{\pgfqpoint{3.320000in}{2.887500in}}%
\pgfusepath{clip}%
\pgfsetbuttcap%
\pgfsetroundjoin%
\pgfsetlinewidth{0.803000pt}%
\definecolor{currentstroke}{rgb}{0.752941,0.752941,0.752941}%
\pgfsetstrokecolor{currentstroke}%
\pgfsetdash{{2.960000pt}{1.280000pt}}{0.000000pt}%
\pgfpathmoveto{\pgfqpoint{0.726250in}{0.525000in}}%
\pgfpathlineto{\pgfqpoint{0.726250in}{3.412500in}}%
\pgfusepath{stroke}%
\end{pgfscope}%
\begin{pgfscope}%
\pgfsetbuttcap%
\pgfsetroundjoin%
\definecolor{currentfill}{rgb}{0.000000,0.000000,0.000000}%
\pgfsetfillcolor{currentfill}%
\pgfsetlinewidth{0.803000pt}%
\definecolor{currentstroke}{rgb}{0.000000,0.000000,0.000000}%
\pgfsetstrokecolor{currentstroke}%
\pgfsetdash{}{0pt}%
\pgfsys@defobject{currentmarker}{\pgfqpoint{0.000000in}{-0.048611in}}{\pgfqpoint{0.000000in}{0.000000in}}{%
\pgfpathmoveto{\pgfqpoint{0.000000in}{0.000000in}}%
\pgfpathlineto{\pgfqpoint{0.000000in}{-0.048611in}}%
\pgfusepath{stroke,fill}%
}%
\begin{pgfscope}%
\pgfsys@transformshift{0.726250in}{0.525000in}%
\pgfsys@useobject{currentmarker}{}%
\end{pgfscope}%
\end{pgfscope}%
\begin{pgfscope}%
\definecolor{textcolor}{rgb}{0.000000,0.000000,0.000000}%
\pgfsetstrokecolor{textcolor}%
\pgfsetfillcolor{textcolor}%
\pgftext[x=0.726250in,y=0.427778in,,top]{\color{textcolor}\rmfamily\fontsize{9.000000}{10.800000}\selectfont \(\displaystyle {10^{-1}}\)}%
\end{pgfscope}%
\begin{pgfscope}%
\pgfpathrectangle{\pgfqpoint{0.726250in}{0.525000in}}{\pgfqpoint{3.320000in}{2.887500in}}%
\pgfusepath{clip}%
\pgfsetbuttcap%
\pgfsetroundjoin%
\pgfsetlinewidth{0.803000pt}%
\definecolor{currentstroke}{rgb}{0.752941,0.752941,0.752941}%
\pgfsetstrokecolor{currentstroke}%
\pgfsetdash{{2.960000pt}{1.280000pt}}{0.000000pt}%
\pgfpathmoveto{\pgfqpoint{2.066515in}{0.525000in}}%
\pgfpathlineto{\pgfqpoint{2.066515in}{3.412500in}}%
\pgfusepath{stroke}%
\end{pgfscope}%
\begin{pgfscope}%
\pgfsetbuttcap%
\pgfsetroundjoin%
\definecolor{currentfill}{rgb}{0.000000,0.000000,0.000000}%
\pgfsetfillcolor{currentfill}%
\pgfsetlinewidth{0.803000pt}%
\definecolor{currentstroke}{rgb}{0.000000,0.000000,0.000000}%
\pgfsetstrokecolor{currentstroke}%
\pgfsetdash{}{0pt}%
\pgfsys@defobject{currentmarker}{\pgfqpoint{0.000000in}{-0.048611in}}{\pgfqpoint{0.000000in}{0.000000in}}{%
\pgfpathmoveto{\pgfqpoint{0.000000in}{0.000000in}}%
\pgfpathlineto{\pgfqpoint{0.000000in}{-0.048611in}}%
\pgfusepath{stroke,fill}%
}%
\begin{pgfscope}%
\pgfsys@transformshift{2.066515in}{0.525000in}%
\pgfsys@useobject{currentmarker}{}%
\end{pgfscope}%
\end{pgfscope}%
\begin{pgfscope}%
\definecolor{textcolor}{rgb}{0.000000,0.000000,0.000000}%
\pgfsetstrokecolor{textcolor}%
\pgfsetfillcolor{textcolor}%
\pgftext[x=2.066515in,y=0.427778in,,top]{\color{textcolor}\rmfamily\fontsize{9.000000}{10.800000}\selectfont \(\displaystyle {10^{0}}\)}%
\end{pgfscope}%
\begin{pgfscope}%
\pgfpathrectangle{\pgfqpoint{0.726250in}{0.525000in}}{\pgfqpoint{3.320000in}{2.887500in}}%
\pgfusepath{clip}%
\pgfsetbuttcap%
\pgfsetroundjoin%
\pgfsetlinewidth{0.803000pt}%
\definecolor{currentstroke}{rgb}{0.752941,0.752941,0.752941}%
\pgfsetstrokecolor{currentstroke}%
\pgfsetdash{{2.960000pt}{1.280000pt}}{0.000000pt}%
\pgfpathmoveto{\pgfqpoint{3.406781in}{0.525000in}}%
\pgfpathlineto{\pgfqpoint{3.406781in}{3.412500in}}%
\pgfusepath{stroke}%
\end{pgfscope}%
\begin{pgfscope}%
\pgfsetbuttcap%
\pgfsetroundjoin%
\definecolor{currentfill}{rgb}{0.000000,0.000000,0.000000}%
\pgfsetfillcolor{currentfill}%
\pgfsetlinewidth{0.803000pt}%
\definecolor{currentstroke}{rgb}{0.000000,0.000000,0.000000}%
\pgfsetstrokecolor{currentstroke}%
\pgfsetdash{}{0pt}%
\pgfsys@defobject{currentmarker}{\pgfqpoint{0.000000in}{-0.048611in}}{\pgfqpoint{0.000000in}{0.000000in}}{%
\pgfpathmoveto{\pgfqpoint{0.000000in}{0.000000in}}%
\pgfpathlineto{\pgfqpoint{0.000000in}{-0.048611in}}%
\pgfusepath{stroke,fill}%
}%
\begin{pgfscope}%
\pgfsys@transformshift{3.406781in}{0.525000in}%
\pgfsys@useobject{currentmarker}{}%
\end{pgfscope}%
\end{pgfscope}%
\begin{pgfscope}%
\definecolor{textcolor}{rgb}{0.000000,0.000000,0.000000}%
\pgfsetstrokecolor{textcolor}%
\pgfsetfillcolor{textcolor}%
\pgftext[x=3.406781in,y=0.427778in,,top]{\color{textcolor}\rmfamily\fontsize{9.000000}{10.800000}\selectfont \(\displaystyle {10^{1}}\)}%
\end{pgfscope}%
\begin{pgfscope}%
\pgfpathrectangle{\pgfqpoint{0.726250in}{0.525000in}}{\pgfqpoint{3.320000in}{2.887500in}}%
\pgfusepath{clip}%
\pgfsetbuttcap%
\pgfsetroundjoin%
\pgfsetlinewidth{0.803000pt}%
\definecolor{currentstroke}{rgb}{0.752941,0.752941,0.752941}%
\pgfsetstrokecolor{currentstroke}%
\pgfsetdash{{2.960000pt}{1.280000pt}}{0.000000pt}%
\pgfpathmoveto{\pgfqpoint{1.129710in}{0.525000in}}%
\pgfpathlineto{\pgfqpoint{1.129710in}{3.412500in}}%
\pgfusepath{stroke}%
\end{pgfscope}%
\begin{pgfscope}%
\pgfsetbuttcap%
\pgfsetroundjoin%
\definecolor{currentfill}{rgb}{0.000000,0.000000,0.000000}%
\pgfsetfillcolor{currentfill}%
\pgfsetlinewidth{0.602250pt}%
\definecolor{currentstroke}{rgb}{0.000000,0.000000,0.000000}%
\pgfsetstrokecolor{currentstroke}%
\pgfsetdash{}{0pt}%
\pgfsys@defobject{currentmarker}{\pgfqpoint{0.000000in}{-0.027778in}}{\pgfqpoint{0.000000in}{0.000000in}}{%
\pgfpathmoveto{\pgfqpoint{0.000000in}{0.000000in}}%
\pgfpathlineto{\pgfqpoint{0.000000in}{-0.027778in}}%
\pgfusepath{stroke,fill}%
}%
\begin{pgfscope}%
\pgfsys@transformshift{1.129710in}{0.525000in}%
\pgfsys@useobject{currentmarker}{}%
\end{pgfscope}%
\end{pgfscope}%
\begin{pgfscope}%
\pgfpathrectangle{\pgfqpoint{0.726250in}{0.525000in}}{\pgfqpoint{3.320000in}{2.887500in}}%
\pgfusepath{clip}%
\pgfsetbuttcap%
\pgfsetroundjoin%
\pgfsetlinewidth{0.803000pt}%
\definecolor{currentstroke}{rgb}{0.752941,0.752941,0.752941}%
\pgfsetstrokecolor{currentstroke}%
\pgfsetdash{{2.960000pt}{1.280000pt}}{0.000000pt}%
\pgfpathmoveto{\pgfqpoint{1.365719in}{0.525000in}}%
\pgfpathlineto{\pgfqpoint{1.365719in}{3.412500in}}%
\pgfusepath{stroke}%
\end{pgfscope}%
\begin{pgfscope}%
\pgfsetbuttcap%
\pgfsetroundjoin%
\definecolor{currentfill}{rgb}{0.000000,0.000000,0.000000}%
\pgfsetfillcolor{currentfill}%
\pgfsetlinewidth{0.602250pt}%
\definecolor{currentstroke}{rgb}{0.000000,0.000000,0.000000}%
\pgfsetstrokecolor{currentstroke}%
\pgfsetdash{}{0pt}%
\pgfsys@defobject{currentmarker}{\pgfqpoint{0.000000in}{-0.027778in}}{\pgfqpoint{0.000000in}{0.000000in}}{%
\pgfpathmoveto{\pgfqpoint{0.000000in}{0.000000in}}%
\pgfpathlineto{\pgfqpoint{0.000000in}{-0.027778in}}%
\pgfusepath{stroke,fill}%
}%
\begin{pgfscope}%
\pgfsys@transformshift{1.365719in}{0.525000in}%
\pgfsys@useobject{currentmarker}{}%
\end{pgfscope}%
\end{pgfscope}%
\begin{pgfscope}%
\pgfpathrectangle{\pgfqpoint{0.726250in}{0.525000in}}{\pgfqpoint{3.320000in}{2.887500in}}%
\pgfusepath{clip}%
\pgfsetbuttcap%
\pgfsetroundjoin%
\pgfsetlinewidth{0.803000pt}%
\definecolor{currentstroke}{rgb}{0.752941,0.752941,0.752941}%
\pgfsetstrokecolor{currentstroke}%
\pgfsetdash{{2.960000pt}{1.280000pt}}{0.000000pt}%
\pgfpathmoveto{\pgfqpoint{1.533170in}{0.525000in}}%
\pgfpathlineto{\pgfqpoint{1.533170in}{3.412500in}}%
\pgfusepath{stroke}%
\end{pgfscope}%
\begin{pgfscope}%
\pgfsetbuttcap%
\pgfsetroundjoin%
\definecolor{currentfill}{rgb}{0.000000,0.000000,0.000000}%
\pgfsetfillcolor{currentfill}%
\pgfsetlinewidth{0.602250pt}%
\definecolor{currentstroke}{rgb}{0.000000,0.000000,0.000000}%
\pgfsetstrokecolor{currentstroke}%
\pgfsetdash{}{0pt}%
\pgfsys@defobject{currentmarker}{\pgfqpoint{0.000000in}{-0.027778in}}{\pgfqpoint{0.000000in}{0.000000in}}{%
\pgfpathmoveto{\pgfqpoint{0.000000in}{0.000000in}}%
\pgfpathlineto{\pgfqpoint{0.000000in}{-0.027778in}}%
\pgfusepath{stroke,fill}%
}%
\begin{pgfscope}%
\pgfsys@transformshift{1.533170in}{0.525000in}%
\pgfsys@useobject{currentmarker}{}%
\end{pgfscope}%
\end{pgfscope}%
\begin{pgfscope}%
\pgfpathrectangle{\pgfqpoint{0.726250in}{0.525000in}}{\pgfqpoint{3.320000in}{2.887500in}}%
\pgfusepath{clip}%
\pgfsetbuttcap%
\pgfsetroundjoin%
\pgfsetlinewidth{0.803000pt}%
\definecolor{currentstroke}{rgb}{0.752941,0.752941,0.752941}%
\pgfsetstrokecolor{currentstroke}%
\pgfsetdash{{2.960000pt}{1.280000pt}}{0.000000pt}%
\pgfpathmoveto{\pgfqpoint{1.663055in}{0.525000in}}%
\pgfpathlineto{\pgfqpoint{1.663055in}{3.412500in}}%
\pgfusepath{stroke}%
\end{pgfscope}%
\begin{pgfscope}%
\pgfsetbuttcap%
\pgfsetroundjoin%
\definecolor{currentfill}{rgb}{0.000000,0.000000,0.000000}%
\pgfsetfillcolor{currentfill}%
\pgfsetlinewidth{0.602250pt}%
\definecolor{currentstroke}{rgb}{0.000000,0.000000,0.000000}%
\pgfsetstrokecolor{currentstroke}%
\pgfsetdash{}{0pt}%
\pgfsys@defobject{currentmarker}{\pgfqpoint{0.000000in}{-0.027778in}}{\pgfqpoint{0.000000in}{0.000000in}}{%
\pgfpathmoveto{\pgfqpoint{0.000000in}{0.000000in}}%
\pgfpathlineto{\pgfqpoint{0.000000in}{-0.027778in}}%
\pgfusepath{stroke,fill}%
}%
\begin{pgfscope}%
\pgfsys@transformshift{1.663055in}{0.525000in}%
\pgfsys@useobject{currentmarker}{}%
\end{pgfscope}%
\end{pgfscope}%
\begin{pgfscope}%
\pgfpathrectangle{\pgfqpoint{0.726250in}{0.525000in}}{\pgfqpoint{3.320000in}{2.887500in}}%
\pgfusepath{clip}%
\pgfsetbuttcap%
\pgfsetroundjoin%
\pgfsetlinewidth{0.803000pt}%
\definecolor{currentstroke}{rgb}{0.752941,0.752941,0.752941}%
\pgfsetstrokecolor{currentstroke}%
\pgfsetdash{{2.960000pt}{1.280000pt}}{0.000000pt}%
\pgfpathmoveto{\pgfqpoint{1.769179in}{0.525000in}}%
\pgfpathlineto{\pgfqpoint{1.769179in}{3.412500in}}%
\pgfusepath{stroke}%
\end{pgfscope}%
\begin{pgfscope}%
\pgfsetbuttcap%
\pgfsetroundjoin%
\definecolor{currentfill}{rgb}{0.000000,0.000000,0.000000}%
\pgfsetfillcolor{currentfill}%
\pgfsetlinewidth{0.602250pt}%
\definecolor{currentstroke}{rgb}{0.000000,0.000000,0.000000}%
\pgfsetstrokecolor{currentstroke}%
\pgfsetdash{}{0pt}%
\pgfsys@defobject{currentmarker}{\pgfqpoint{0.000000in}{-0.027778in}}{\pgfqpoint{0.000000in}{0.000000in}}{%
\pgfpathmoveto{\pgfqpoint{0.000000in}{0.000000in}}%
\pgfpathlineto{\pgfqpoint{0.000000in}{-0.027778in}}%
\pgfusepath{stroke,fill}%
}%
\begin{pgfscope}%
\pgfsys@transformshift{1.769179in}{0.525000in}%
\pgfsys@useobject{currentmarker}{}%
\end{pgfscope}%
\end{pgfscope}%
\begin{pgfscope}%
\pgfpathrectangle{\pgfqpoint{0.726250in}{0.525000in}}{\pgfqpoint{3.320000in}{2.887500in}}%
\pgfusepath{clip}%
\pgfsetbuttcap%
\pgfsetroundjoin%
\pgfsetlinewidth{0.803000pt}%
\definecolor{currentstroke}{rgb}{0.752941,0.752941,0.752941}%
\pgfsetstrokecolor{currentstroke}%
\pgfsetdash{{2.960000pt}{1.280000pt}}{0.000000pt}%
\pgfpathmoveto{\pgfqpoint{1.858906in}{0.525000in}}%
\pgfpathlineto{\pgfqpoint{1.858906in}{3.412500in}}%
\pgfusepath{stroke}%
\end{pgfscope}%
\begin{pgfscope}%
\pgfsetbuttcap%
\pgfsetroundjoin%
\definecolor{currentfill}{rgb}{0.000000,0.000000,0.000000}%
\pgfsetfillcolor{currentfill}%
\pgfsetlinewidth{0.602250pt}%
\definecolor{currentstroke}{rgb}{0.000000,0.000000,0.000000}%
\pgfsetstrokecolor{currentstroke}%
\pgfsetdash{}{0pt}%
\pgfsys@defobject{currentmarker}{\pgfqpoint{0.000000in}{-0.027778in}}{\pgfqpoint{0.000000in}{0.000000in}}{%
\pgfpathmoveto{\pgfqpoint{0.000000in}{0.000000in}}%
\pgfpathlineto{\pgfqpoint{0.000000in}{-0.027778in}}%
\pgfusepath{stroke,fill}%
}%
\begin{pgfscope}%
\pgfsys@transformshift{1.858906in}{0.525000in}%
\pgfsys@useobject{currentmarker}{}%
\end{pgfscope}%
\end{pgfscope}%
\begin{pgfscope}%
\pgfpathrectangle{\pgfqpoint{0.726250in}{0.525000in}}{\pgfqpoint{3.320000in}{2.887500in}}%
\pgfusepath{clip}%
\pgfsetbuttcap%
\pgfsetroundjoin%
\pgfsetlinewidth{0.803000pt}%
\definecolor{currentstroke}{rgb}{0.752941,0.752941,0.752941}%
\pgfsetstrokecolor{currentstroke}%
\pgfsetdash{{2.960000pt}{1.280000pt}}{0.000000pt}%
\pgfpathmoveto{\pgfqpoint{1.936630in}{0.525000in}}%
\pgfpathlineto{\pgfqpoint{1.936630in}{3.412500in}}%
\pgfusepath{stroke}%
\end{pgfscope}%
\begin{pgfscope}%
\pgfsetbuttcap%
\pgfsetroundjoin%
\definecolor{currentfill}{rgb}{0.000000,0.000000,0.000000}%
\pgfsetfillcolor{currentfill}%
\pgfsetlinewidth{0.602250pt}%
\definecolor{currentstroke}{rgb}{0.000000,0.000000,0.000000}%
\pgfsetstrokecolor{currentstroke}%
\pgfsetdash{}{0pt}%
\pgfsys@defobject{currentmarker}{\pgfqpoint{0.000000in}{-0.027778in}}{\pgfqpoint{0.000000in}{0.000000in}}{%
\pgfpathmoveto{\pgfqpoint{0.000000in}{0.000000in}}%
\pgfpathlineto{\pgfqpoint{0.000000in}{-0.027778in}}%
\pgfusepath{stroke,fill}%
}%
\begin{pgfscope}%
\pgfsys@transformshift{1.936630in}{0.525000in}%
\pgfsys@useobject{currentmarker}{}%
\end{pgfscope}%
\end{pgfscope}%
\begin{pgfscope}%
\pgfpathrectangle{\pgfqpoint{0.726250in}{0.525000in}}{\pgfqpoint{3.320000in}{2.887500in}}%
\pgfusepath{clip}%
\pgfsetbuttcap%
\pgfsetroundjoin%
\pgfsetlinewidth{0.803000pt}%
\definecolor{currentstroke}{rgb}{0.752941,0.752941,0.752941}%
\pgfsetstrokecolor{currentstroke}%
\pgfsetdash{{2.960000pt}{1.280000pt}}{0.000000pt}%
\pgfpathmoveto{\pgfqpoint{2.005188in}{0.525000in}}%
\pgfpathlineto{\pgfqpoint{2.005188in}{3.412500in}}%
\pgfusepath{stroke}%
\end{pgfscope}%
\begin{pgfscope}%
\pgfsetbuttcap%
\pgfsetroundjoin%
\definecolor{currentfill}{rgb}{0.000000,0.000000,0.000000}%
\pgfsetfillcolor{currentfill}%
\pgfsetlinewidth{0.602250pt}%
\definecolor{currentstroke}{rgb}{0.000000,0.000000,0.000000}%
\pgfsetstrokecolor{currentstroke}%
\pgfsetdash{}{0pt}%
\pgfsys@defobject{currentmarker}{\pgfqpoint{0.000000in}{-0.027778in}}{\pgfqpoint{0.000000in}{0.000000in}}{%
\pgfpathmoveto{\pgfqpoint{0.000000in}{0.000000in}}%
\pgfpathlineto{\pgfqpoint{0.000000in}{-0.027778in}}%
\pgfusepath{stroke,fill}%
}%
\begin{pgfscope}%
\pgfsys@transformshift{2.005188in}{0.525000in}%
\pgfsys@useobject{currentmarker}{}%
\end{pgfscope}%
\end{pgfscope}%
\begin{pgfscope}%
\pgfpathrectangle{\pgfqpoint{0.726250in}{0.525000in}}{\pgfqpoint{3.320000in}{2.887500in}}%
\pgfusepath{clip}%
\pgfsetbuttcap%
\pgfsetroundjoin%
\pgfsetlinewidth{0.803000pt}%
\definecolor{currentstroke}{rgb}{0.752941,0.752941,0.752941}%
\pgfsetstrokecolor{currentstroke}%
\pgfsetdash{{2.960000pt}{1.280000pt}}{0.000000pt}%
\pgfpathmoveto{\pgfqpoint{2.469976in}{0.525000in}}%
\pgfpathlineto{\pgfqpoint{2.469976in}{3.412500in}}%
\pgfusepath{stroke}%
\end{pgfscope}%
\begin{pgfscope}%
\pgfsetbuttcap%
\pgfsetroundjoin%
\definecolor{currentfill}{rgb}{0.000000,0.000000,0.000000}%
\pgfsetfillcolor{currentfill}%
\pgfsetlinewidth{0.602250pt}%
\definecolor{currentstroke}{rgb}{0.000000,0.000000,0.000000}%
\pgfsetstrokecolor{currentstroke}%
\pgfsetdash{}{0pt}%
\pgfsys@defobject{currentmarker}{\pgfqpoint{0.000000in}{-0.027778in}}{\pgfqpoint{0.000000in}{0.000000in}}{%
\pgfpathmoveto{\pgfqpoint{0.000000in}{0.000000in}}%
\pgfpathlineto{\pgfqpoint{0.000000in}{-0.027778in}}%
\pgfusepath{stroke,fill}%
}%
\begin{pgfscope}%
\pgfsys@transformshift{2.469976in}{0.525000in}%
\pgfsys@useobject{currentmarker}{}%
\end{pgfscope}%
\end{pgfscope}%
\begin{pgfscope}%
\pgfpathrectangle{\pgfqpoint{0.726250in}{0.525000in}}{\pgfqpoint{3.320000in}{2.887500in}}%
\pgfusepath{clip}%
\pgfsetbuttcap%
\pgfsetroundjoin%
\pgfsetlinewidth{0.803000pt}%
\definecolor{currentstroke}{rgb}{0.752941,0.752941,0.752941}%
\pgfsetstrokecolor{currentstroke}%
\pgfsetdash{{2.960000pt}{1.280000pt}}{0.000000pt}%
\pgfpathmoveto{\pgfqpoint{2.705985in}{0.525000in}}%
\pgfpathlineto{\pgfqpoint{2.705985in}{3.412500in}}%
\pgfusepath{stroke}%
\end{pgfscope}%
\begin{pgfscope}%
\pgfsetbuttcap%
\pgfsetroundjoin%
\definecolor{currentfill}{rgb}{0.000000,0.000000,0.000000}%
\pgfsetfillcolor{currentfill}%
\pgfsetlinewidth{0.602250pt}%
\definecolor{currentstroke}{rgb}{0.000000,0.000000,0.000000}%
\pgfsetstrokecolor{currentstroke}%
\pgfsetdash{}{0pt}%
\pgfsys@defobject{currentmarker}{\pgfqpoint{0.000000in}{-0.027778in}}{\pgfqpoint{0.000000in}{0.000000in}}{%
\pgfpathmoveto{\pgfqpoint{0.000000in}{0.000000in}}%
\pgfpathlineto{\pgfqpoint{0.000000in}{-0.027778in}}%
\pgfusepath{stroke,fill}%
}%
\begin{pgfscope}%
\pgfsys@transformshift{2.705985in}{0.525000in}%
\pgfsys@useobject{currentmarker}{}%
\end{pgfscope}%
\end{pgfscope}%
\begin{pgfscope}%
\pgfpathrectangle{\pgfqpoint{0.726250in}{0.525000in}}{\pgfqpoint{3.320000in}{2.887500in}}%
\pgfusepath{clip}%
\pgfsetbuttcap%
\pgfsetroundjoin%
\pgfsetlinewidth{0.803000pt}%
\definecolor{currentstroke}{rgb}{0.752941,0.752941,0.752941}%
\pgfsetstrokecolor{currentstroke}%
\pgfsetdash{{2.960000pt}{1.280000pt}}{0.000000pt}%
\pgfpathmoveto{\pgfqpoint{2.873436in}{0.525000in}}%
\pgfpathlineto{\pgfqpoint{2.873436in}{3.412500in}}%
\pgfusepath{stroke}%
\end{pgfscope}%
\begin{pgfscope}%
\pgfsetbuttcap%
\pgfsetroundjoin%
\definecolor{currentfill}{rgb}{0.000000,0.000000,0.000000}%
\pgfsetfillcolor{currentfill}%
\pgfsetlinewidth{0.602250pt}%
\definecolor{currentstroke}{rgb}{0.000000,0.000000,0.000000}%
\pgfsetstrokecolor{currentstroke}%
\pgfsetdash{}{0pt}%
\pgfsys@defobject{currentmarker}{\pgfqpoint{0.000000in}{-0.027778in}}{\pgfqpoint{0.000000in}{0.000000in}}{%
\pgfpathmoveto{\pgfqpoint{0.000000in}{0.000000in}}%
\pgfpathlineto{\pgfqpoint{0.000000in}{-0.027778in}}%
\pgfusepath{stroke,fill}%
}%
\begin{pgfscope}%
\pgfsys@transformshift{2.873436in}{0.525000in}%
\pgfsys@useobject{currentmarker}{}%
\end{pgfscope}%
\end{pgfscope}%
\begin{pgfscope}%
\pgfpathrectangle{\pgfqpoint{0.726250in}{0.525000in}}{\pgfqpoint{3.320000in}{2.887500in}}%
\pgfusepath{clip}%
\pgfsetbuttcap%
\pgfsetroundjoin%
\pgfsetlinewidth{0.803000pt}%
\definecolor{currentstroke}{rgb}{0.752941,0.752941,0.752941}%
\pgfsetstrokecolor{currentstroke}%
\pgfsetdash{{2.960000pt}{1.280000pt}}{0.000000pt}%
\pgfpathmoveto{\pgfqpoint{3.003321in}{0.525000in}}%
\pgfpathlineto{\pgfqpoint{3.003321in}{3.412500in}}%
\pgfusepath{stroke}%
\end{pgfscope}%
\begin{pgfscope}%
\pgfsetbuttcap%
\pgfsetroundjoin%
\definecolor{currentfill}{rgb}{0.000000,0.000000,0.000000}%
\pgfsetfillcolor{currentfill}%
\pgfsetlinewidth{0.602250pt}%
\definecolor{currentstroke}{rgb}{0.000000,0.000000,0.000000}%
\pgfsetstrokecolor{currentstroke}%
\pgfsetdash{}{0pt}%
\pgfsys@defobject{currentmarker}{\pgfqpoint{0.000000in}{-0.027778in}}{\pgfqpoint{0.000000in}{0.000000in}}{%
\pgfpathmoveto{\pgfqpoint{0.000000in}{0.000000in}}%
\pgfpathlineto{\pgfqpoint{0.000000in}{-0.027778in}}%
\pgfusepath{stroke,fill}%
}%
\begin{pgfscope}%
\pgfsys@transformshift{3.003321in}{0.525000in}%
\pgfsys@useobject{currentmarker}{}%
\end{pgfscope}%
\end{pgfscope}%
\begin{pgfscope}%
\pgfpathrectangle{\pgfqpoint{0.726250in}{0.525000in}}{\pgfqpoint{3.320000in}{2.887500in}}%
\pgfusepath{clip}%
\pgfsetbuttcap%
\pgfsetroundjoin%
\pgfsetlinewidth{0.803000pt}%
\definecolor{currentstroke}{rgb}{0.752941,0.752941,0.752941}%
\pgfsetstrokecolor{currentstroke}%
\pgfsetdash{{2.960000pt}{1.280000pt}}{0.000000pt}%
\pgfpathmoveto{\pgfqpoint{3.109445in}{0.525000in}}%
\pgfpathlineto{\pgfqpoint{3.109445in}{3.412500in}}%
\pgfusepath{stroke}%
\end{pgfscope}%
\begin{pgfscope}%
\pgfsetbuttcap%
\pgfsetroundjoin%
\definecolor{currentfill}{rgb}{0.000000,0.000000,0.000000}%
\pgfsetfillcolor{currentfill}%
\pgfsetlinewidth{0.602250pt}%
\definecolor{currentstroke}{rgb}{0.000000,0.000000,0.000000}%
\pgfsetstrokecolor{currentstroke}%
\pgfsetdash{}{0pt}%
\pgfsys@defobject{currentmarker}{\pgfqpoint{0.000000in}{-0.027778in}}{\pgfqpoint{0.000000in}{0.000000in}}{%
\pgfpathmoveto{\pgfqpoint{0.000000in}{0.000000in}}%
\pgfpathlineto{\pgfqpoint{0.000000in}{-0.027778in}}%
\pgfusepath{stroke,fill}%
}%
\begin{pgfscope}%
\pgfsys@transformshift{3.109445in}{0.525000in}%
\pgfsys@useobject{currentmarker}{}%
\end{pgfscope}%
\end{pgfscope}%
\begin{pgfscope}%
\pgfpathrectangle{\pgfqpoint{0.726250in}{0.525000in}}{\pgfqpoint{3.320000in}{2.887500in}}%
\pgfusepath{clip}%
\pgfsetbuttcap%
\pgfsetroundjoin%
\pgfsetlinewidth{0.803000pt}%
\definecolor{currentstroke}{rgb}{0.752941,0.752941,0.752941}%
\pgfsetstrokecolor{currentstroke}%
\pgfsetdash{{2.960000pt}{1.280000pt}}{0.000000pt}%
\pgfpathmoveto{\pgfqpoint{3.199171in}{0.525000in}}%
\pgfpathlineto{\pgfqpoint{3.199171in}{3.412500in}}%
\pgfusepath{stroke}%
\end{pgfscope}%
\begin{pgfscope}%
\pgfsetbuttcap%
\pgfsetroundjoin%
\definecolor{currentfill}{rgb}{0.000000,0.000000,0.000000}%
\pgfsetfillcolor{currentfill}%
\pgfsetlinewidth{0.602250pt}%
\definecolor{currentstroke}{rgb}{0.000000,0.000000,0.000000}%
\pgfsetstrokecolor{currentstroke}%
\pgfsetdash{}{0pt}%
\pgfsys@defobject{currentmarker}{\pgfqpoint{0.000000in}{-0.027778in}}{\pgfqpoint{0.000000in}{0.000000in}}{%
\pgfpathmoveto{\pgfqpoint{0.000000in}{0.000000in}}%
\pgfpathlineto{\pgfqpoint{0.000000in}{-0.027778in}}%
\pgfusepath{stroke,fill}%
}%
\begin{pgfscope}%
\pgfsys@transformshift{3.199171in}{0.525000in}%
\pgfsys@useobject{currentmarker}{}%
\end{pgfscope}%
\end{pgfscope}%
\begin{pgfscope}%
\pgfpathrectangle{\pgfqpoint{0.726250in}{0.525000in}}{\pgfqpoint{3.320000in}{2.887500in}}%
\pgfusepath{clip}%
\pgfsetbuttcap%
\pgfsetroundjoin%
\pgfsetlinewidth{0.803000pt}%
\definecolor{currentstroke}{rgb}{0.752941,0.752941,0.752941}%
\pgfsetstrokecolor{currentstroke}%
\pgfsetdash{{2.960000pt}{1.280000pt}}{0.000000pt}%
\pgfpathmoveto{\pgfqpoint{3.276896in}{0.525000in}}%
\pgfpathlineto{\pgfqpoint{3.276896in}{3.412500in}}%
\pgfusepath{stroke}%
\end{pgfscope}%
\begin{pgfscope}%
\pgfsetbuttcap%
\pgfsetroundjoin%
\definecolor{currentfill}{rgb}{0.000000,0.000000,0.000000}%
\pgfsetfillcolor{currentfill}%
\pgfsetlinewidth{0.602250pt}%
\definecolor{currentstroke}{rgb}{0.000000,0.000000,0.000000}%
\pgfsetstrokecolor{currentstroke}%
\pgfsetdash{}{0pt}%
\pgfsys@defobject{currentmarker}{\pgfqpoint{0.000000in}{-0.027778in}}{\pgfqpoint{0.000000in}{0.000000in}}{%
\pgfpathmoveto{\pgfqpoint{0.000000in}{0.000000in}}%
\pgfpathlineto{\pgfqpoint{0.000000in}{-0.027778in}}%
\pgfusepath{stroke,fill}%
}%
\begin{pgfscope}%
\pgfsys@transformshift{3.276896in}{0.525000in}%
\pgfsys@useobject{currentmarker}{}%
\end{pgfscope}%
\end{pgfscope}%
\begin{pgfscope}%
\pgfpathrectangle{\pgfqpoint{0.726250in}{0.525000in}}{\pgfqpoint{3.320000in}{2.887500in}}%
\pgfusepath{clip}%
\pgfsetbuttcap%
\pgfsetroundjoin%
\pgfsetlinewidth{0.803000pt}%
\definecolor{currentstroke}{rgb}{0.752941,0.752941,0.752941}%
\pgfsetstrokecolor{currentstroke}%
\pgfsetdash{{2.960000pt}{1.280000pt}}{0.000000pt}%
\pgfpathmoveto{\pgfqpoint{3.345454in}{0.525000in}}%
\pgfpathlineto{\pgfqpoint{3.345454in}{3.412500in}}%
\pgfusepath{stroke}%
\end{pgfscope}%
\begin{pgfscope}%
\pgfsetbuttcap%
\pgfsetroundjoin%
\definecolor{currentfill}{rgb}{0.000000,0.000000,0.000000}%
\pgfsetfillcolor{currentfill}%
\pgfsetlinewidth{0.602250pt}%
\definecolor{currentstroke}{rgb}{0.000000,0.000000,0.000000}%
\pgfsetstrokecolor{currentstroke}%
\pgfsetdash{}{0pt}%
\pgfsys@defobject{currentmarker}{\pgfqpoint{0.000000in}{-0.027778in}}{\pgfqpoint{0.000000in}{0.000000in}}{%
\pgfpathmoveto{\pgfqpoint{0.000000in}{0.000000in}}%
\pgfpathlineto{\pgfqpoint{0.000000in}{-0.027778in}}%
\pgfusepath{stroke,fill}%
}%
\begin{pgfscope}%
\pgfsys@transformshift{3.345454in}{0.525000in}%
\pgfsys@useobject{currentmarker}{}%
\end{pgfscope}%
\end{pgfscope}%
\begin{pgfscope}%
\pgfpathrectangle{\pgfqpoint{0.726250in}{0.525000in}}{\pgfqpoint{3.320000in}{2.887500in}}%
\pgfusepath{clip}%
\pgfsetbuttcap%
\pgfsetroundjoin%
\pgfsetlinewidth{0.803000pt}%
\definecolor{currentstroke}{rgb}{0.752941,0.752941,0.752941}%
\pgfsetstrokecolor{currentstroke}%
\pgfsetdash{{2.960000pt}{1.280000pt}}{0.000000pt}%
\pgfpathmoveto{\pgfqpoint{3.810241in}{0.525000in}}%
\pgfpathlineto{\pgfqpoint{3.810241in}{3.412500in}}%
\pgfusepath{stroke}%
\end{pgfscope}%
\begin{pgfscope}%
\pgfsetbuttcap%
\pgfsetroundjoin%
\definecolor{currentfill}{rgb}{0.000000,0.000000,0.000000}%
\pgfsetfillcolor{currentfill}%
\pgfsetlinewidth{0.602250pt}%
\definecolor{currentstroke}{rgb}{0.000000,0.000000,0.000000}%
\pgfsetstrokecolor{currentstroke}%
\pgfsetdash{}{0pt}%
\pgfsys@defobject{currentmarker}{\pgfqpoint{0.000000in}{-0.027778in}}{\pgfqpoint{0.000000in}{0.000000in}}{%
\pgfpathmoveto{\pgfqpoint{0.000000in}{0.000000in}}%
\pgfpathlineto{\pgfqpoint{0.000000in}{-0.027778in}}%
\pgfusepath{stroke,fill}%
}%
\begin{pgfscope}%
\pgfsys@transformshift{3.810241in}{0.525000in}%
\pgfsys@useobject{currentmarker}{}%
\end{pgfscope}%
\end{pgfscope}%
\begin{pgfscope}%
\pgfpathrectangle{\pgfqpoint{0.726250in}{0.525000in}}{\pgfqpoint{3.320000in}{2.887500in}}%
\pgfusepath{clip}%
\pgfsetbuttcap%
\pgfsetroundjoin%
\pgfsetlinewidth{0.803000pt}%
\definecolor{currentstroke}{rgb}{0.752941,0.752941,0.752941}%
\pgfsetstrokecolor{currentstroke}%
\pgfsetdash{{2.960000pt}{1.280000pt}}{0.000000pt}%
\pgfpathmoveto{\pgfqpoint{4.046250in}{0.525000in}}%
\pgfpathlineto{\pgfqpoint{4.046250in}{3.412500in}}%
\pgfusepath{stroke}%
\end{pgfscope}%
\begin{pgfscope}%
\pgfsetbuttcap%
\pgfsetroundjoin%
\definecolor{currentfill}{rgb}{0.000000,0.000000,0.000000}%
\pgfsetfillcolor{currentfill}%
\pgfsetlinewidth{0.602250pt}%
\definecolor{currentstroke}{rgb}{0.000000,0.000000,0.000000}%
\pgfsetstrokecolor{currentstroke}%
\pgfsetdash{}{0pt}%
\pgfsys@defobject{currentmarker}{\pgfqpoint{0.000000in}{-0.027778in}}{\pgfqpoint{0.000000in}{0.000000in}}{%
\pgfpathmoveto{\pgfqpoint{0.000000in}{0.000000in}}%
\pgfpathlineto{\pgfqpoint{0.000000in}{-0.027778in}}%
\pgfusepath{stroke,fill}%
}%
\begin{pgfscope}%
\pgfsys@transformshift{4.046250in}{0.525000in}%
\pgfsys@useobject{currentmarker}{}%
\end{pgfscope}%
\end{pgfscope}%
\begin{pgfscope}%
\definecolor{textcolor}{rgb}{0.000000,0.000000,0.000000}%
\pgfsetstrokecolor{textcolor}%
\pgfsetfillcolor{textcolor}%
\pgftext[x=2.386250in,y=0.251251in,,top]{\color{textcolor}\rmfamily\fontsize{9.000000}{10.800000}\selectfont Energy [MeV]}%
\end{pgfscope}%
\begin{pgfscope}%
\pgfpathrectangle{\pgfqpoint{0.726250in}{0.525000in}}{\pgfqpoint{3.320000in}{2.887500in}}%
\pgfusepath{clip}%
\pgfsetbuttcap%
\pgfsetroundjoin%
\pgfsetlinewidth{0.803000pt}%
\definecolor{currentstroke}{rgb}{0.752941,0.752941,0.752941}%
\pgfsetstrokecolor{currentstroke}%
\pgfsetdash{{2.960000pt}{1.280000pt}}{0.000000pt}%
\pgfpathmoveto{\pgfqpoint{0.726250in}{0.525000in}}%
\pgfpathlineto{\pgfqpoint{4.046250in}{0.525000in}}%
\pgfusepath{stroke}%
\end{pgfscope}%
\begin{pgfscope}%
\pgfsetbuttcap%
\pgfsetroundjoin%
\definecolor{currentfill}{rgb}{0.000000,0.000000,0.000000}%
\pgfsetfillcolor{currentfill}%
\pgfsetlinewidth{0.803000pt}%
\definecolor{currentstroke}{rgb}{0.000000,0.000000,0.000000}%
\pgfsetstrokecolor{currentstroke}%
\pgfsetdash{}{0pt}%
\pgfsys@defobject{currentmarker}{\pgfqpoint{-0.048611in}{0.000000in}}{\pgfqpoint{-0.000000in}{0.000000in}}{%
\pgfpathmoveto{\pgfqpoint{-0.000000in}{0.000000in}}%
\pgfpathlineto{\pgfqpoint{-0.048611in}{0.000000in}}%
\pgfusepath{stroke,fill}%
}%
\begin{pgfscope}%
\pgfsys@transformshift{0.726250in}{0.525000in}%
\pgfsys@useobject{currentmarker}{}%
\end{pgfscope}%
\end{pgfscope}%
\begin{pgfscope}%
\definecolor{textcolor}{rgb}{0.000000,0.000000,0.000000}%
\pgfsetstrokecolor{textcolor}%
\pgfsetfillcolor{textcolor}%
\pgftext[x=0.442687in, y=0.477515in, left, base]{\color{textcolor}\rmfamily\fontsize{9.000000}{10.800000}\selectfont \(\displaystyle {10^{0}}\)}%
\end{pgfscope}%
\begin{pgfscope}%
\pgfpathrectangle{\pgfqpoint{0.726250in}{0.525000in}}{\pgfqpoint{3.320000in}{2.887500in}}%
\pgfusepath{clip}%
\pgfsetbuttcap%
\pgfsetroundjoin%
\pgfsetlinewidth{0.803000pt}%
\definecolor{currentstroke}{rgb}{0.752941,0.752941,0.752941}%
\pgfsetstrokecolor{currentstroke}%
\pgfsetdash{{2.960000pt}{1.280000pt}}{0.000000pt}%
\pgfpathmoveto{\pgfqpoint{0.726250in}{1.006250in}}%
\pgfpathlineto{\pgfqpoint{4.046250in}{1.006250in}}%
\pgfusepath{stroke}%
\end{pgfscope}%
\begin{pgfscope}%
\pgfsetbuttcap%
\pgfsetroundjoin%
\definecolor{currentfill}{rgb}{0.000000,0.000000,0.000000}%
\pgfsetfillcolor{currentfill}%
\pgfsetlinewidth{0.803000pt}%
\definecolor{currentstroke}{rgb}{0.000000,0.000000,0.000000}%
\pgfsetstrokecolor{currentstroke}%
\pgfsetdash{}{0pt}%
\pgfsys@defobject{currentmarker}{\pgfqpoint{-0.048611in}{0.000000in}}{\pgfqpoint{-0.000000in}{0.000000in}}{%
\pgfpathmoveto{\pgfqpoint{-0.000000in}{0.000000in}}%
\pgfpathlineto{\pgfqpoint{-0.048611in}{0.000000in}}%
\pgfusepath{stroke,fill}%
}%
\begin{pgfscope}%
\pgfsys@transformshift{0.726250in}{1.006250in}%
\pgfsys@useobject{currentmarker}{}%
\end{pgfscope}%
\end{pgfscope}%
\begin{pgfscope}%
\definecolor{textcolor}{rgb}{0.000000,0.000000,0.000000}%
\pgfsetstrokecolor{textcolor}%
\pgfsetfillcolor{textcolor}%
\pgftext[x=0.442687in, y=0.958765in, left, base]{\color{textcolor}\rmfamily\fontsize{9.000000}{10.800000}\selectfont \(\displaystyle {10^{2}}\)}%
\end{pgfscope}%
\begin{pgfscope}%
\pgfpathrectangle{\pgfqpoint{0.726250in}{0.525000in}}{\pgfqpoint{3.320000in}{2.887500in}}%
\pgfusepath{clip}%
\pgfsetbuttcap%
\pgfsetroundjoin%
\pgfsetlinewidth{0.803000pt}%
\definecolor{currentstroke}{rgb}{0.752941,0.752941,0.752941}%
\pgfsetstrokecolor{currentstroke}%
\pgfsetdash{{2.960000pt}{1.280000pt}}{0.000000pt}%
\pgfpathmoveto{\pgfqpoint{0.726250in}{1.487500in}}%
\pgfpathlineto{\pgfqpoint{4.046250in}{1.487500in}}%
\pgfusepath{stroke}%
\end{pgfscope}%
\begin{pgfscope}%
\pgfsetbuttcap%
\pgfsetroundjoin%
\definecolor{currentfill}{rgb}{0.000000,0.000000,0.000000}%
\pgfsetfillcolor{currentfill}%
\pgfsetlinewidth{0.803000pt}%
\definecolor{currentstroke}{rgb}{0.000000,0.000000,0.000000}%
\pgfsetstrokecolor{currentstroke}%
\pgfsetdash{}{0pt}%
\pgfsys@defobject{currentmarker}{\pgfqpoint{-0.048611in}{0.000000in}}{\pgfqpoint{-0.000000in}{0.000000in}}{%
\pgfpathmoveto{\pgfqpoint{-0.000000in}{0.000000in}}%
\pgfpathlineto{\pgfqpoint{-0.048611in}{0.000000in}}%
\pgfusepath{stroke,fill}%
}%
\begin{pgfscope}%
\pgfsys@transformshift{0.726250in}{1.487500in}%
\pgfsys@useobject{currentmarker}{}%
\end{pgfscope}%
\end{pgfscope}%
\begin{pgfscope}%
\definecolor{textcolor}{rgb}{0.000000,0.000000,0.000000}%
\pgfsetstrokecolor{textcolor}%
\pgfsetfillcolor{textcolor}%
\pgftext[x=0.442687in, y=1.440015in, left, base]{\color{textcolor}\rmfamily\fontsize{9.000000}{10.800000}\selectfont \(\displaystyle {10^{4}}\)}%
\end{pgfscope}%
\begin{pgfscope}%
\pgfpathrectangle{\pgfqpoint{0.726250in}{0.525000in}}{\pgfqpoint{3.320000in}{2.887500in}}%
\pgfusepath{clip}%
\pgfsetbuttcap%
\pgfsetroundjoin%
\pgfsetlinewidth{0.803000pt}%
\definecolor{currentstroke}{rgb}{0.752941,0.752941,0.752941}%
\pgfsetstrokecolor{currentstroke}%
\pgfsetdash{{2.960000pt}{1.280000pt}}{0.000000pt}%
\pgfpathmoveto{\pgfqpoint{0.726250in}{1.968750in}}%
\pgfpathlineto{\pgfqpoint{4.046250in}{1.968750in}}%
\pgfusepath{stroke}%
\end{pgfscope}%
\begin{pgfscope}%
\pgfsetbuttcap%
\pgfsetroundjoin%
\definecolor{currentfill}{rgb}{0.000000,0.000000,0.000000}%
\pgfsetfillcolor{currentfill}%
\pgfsetlinewidth{0.803000pt}%
\definecolor{currentstroke}{rgb}{0.000000,0.000000,0.000000}%
\pgfsetstrokecolor{currentstroke}%
\pgfsetdash{}{0pt}%
\pgfsys@defobject{currentmarker}{\pgfqpoint{-0.048611in}{0.000000in}}{\pgfqpoint{-0.000000in}{0.000000in}}{%
\pgfpathmoveto{\pgfqpoint{-0.000000in}{0.000000in}}%
\pgfpathlineto{\pgfqpoint{-0.048611in}{0.000000in}}%
\pgfusepath{stroke,fill}%
}%
\begin{pgfscope}%
\pgfsys@transformshift{0.726250in}{1.968750in}%
\pgfsys@useobject{currentmarker}{}%
\end{pgfscope}%
\end{pgfscope}%
\begin{pgfscope}%
\definecolor{textcolor}{rgb}{0.000000,0.000000,0.000000}%
\pgfsetstrokecolor{textcolor}%
\pgfsetfillcolor{textcolor}%
\pgftext[x=0.442687in, y=1.921265in, left, base]{\color{textcolor}\rmfamily\fontsize{9.000000}{10.800000}\selectfont \(\displaystyle {10^{6}}\)}%
\end{pgfscope}%
\begin{pgfscope}%
\pgfpathrectangle{\pgfqpoint{0.726250in}{0.525000in}}{\pgfqpoint{3.320000in}{2.887500in}}%
\pgfusepath{clip}%
\pgfsetbuttcap%
\pgfsetroundjoin%
\pgfsetlinewidth{0.803000pt}%
\definecolor{currentstroke}{rgb}{0.752941,0.752941,0.752941}%
\pgfsetstrokecolor{currentstroke}%
\pgfsetdash{{2.960000pt}{1.280000pt}}{0.000000pt}%
\pgfpathmoveto{\pgfqpoint{0.726250in}{2.450000in}}%
\pgfpathlineto{\pgfqpoint{4.046250in}{2.450000in}}%
\pgfusepath{stroke}%
\end{pgfscope}%
\begin{pgfscope}%
\pgfsetbuttcap%
\pgfsetroundjoin%
\definecolor{currentfill}{rgb}{0.000000,0.000000,0.000000}%
\pgfsetfillcolor{currentfill}%
\pgfsetlinewidth{0.803000pt}%
\definecolor{currentstroke}{rgb}{0.000000,0.000000,0.000000}%
\pgfsetstrokecolor{currentstroke}%
\pgfsetdash{}{0pt}%
\pgfsys@defobject{currentmarker}{\pgfqpoint{-0.048611in}{0.000000in}}{\pgfqpoint{-0.000000in}{0.000000in}}{%
\pgfpathmoveto{\pgfqpoint{-0.000000in}{0.000000in}}%
\pgfpathlineto{\pgfqpoint{-0.048611in}{0.000000in}}%
\pgfusepath{stroke,fill}%
}%
\begin{pgfscope}%
\pgfsys@transformshift{0.726250in}{2.450000in}%
\pgfsys@useobject{currentmarker}{}%
\end{pgfscope}%
\end{pgfscope}%
\begin{pgfscope}%
\definecolor{textcolor}{rgb}{0.000000,0.000000,0.000000}%
\pgfsetstrokecolor{textcolor}%
\pgfsetfillcolor{textcolor}%
\pgftext[x=0.442687in, y=2.402515in, left, base]{\color{textcolor}\rmfamily\fontsize{9.000000}{10.800000}\selectfont \(\displaystyle {10^{8}}\)}%
\end{pgfscope}%
\begin{pgfscope}%
\pgfpathrectangle{\pgfqpoint{0.726250in}{0.525000in}}{\pgfqpoint{3.320000in}{2.887500in}}%
\pgfusepath{clip}%
\pgfsetbuttcap%
\pgfsetroundjoin%
\pgfsetlinewidth{0.803000pt}%
\definecolor{currentstroke}{rgb}{0.752941,0.752941,0.752941}%
\pgfsetstrokecolor{currentstroke}%
\pgfsetdash{{2.960000pt}{1.280000pt}}{0.000000pt}%
\pgfpathmoveto{\pgfqpoint{0.726250in}{2.931250in}}%
\pgfpathlineto{\pgfqpoint{4.046250in}{2.931250in}}%
\pgfusepath{stroke}%
\end{pgfscope}%
\begin{pgfscope}%
\pgfsetbuttcap%
\pgfsetroundjoin%
\definecolor{currentfill}{rgb}{0.000000,0.000000,0.000000}%
\pgfsetfillcolor{currentfill}%
\pgfsetlinewidth{0.803000pt}%
\definecolor{currentstroke}{rgb}{0.000000,0.000000,0.000000}%
\pgfsetstrokecolor{currentstroke}%
\pgfsetdash{}{0pt}%
\pgfsys@defobject{currentmarker}{\pgfqpoint{-0.048611in}{0.000000in}}{\pgfqpoint{-0.000000in}{0.000000in}}{%
\pgfpathmoveto{\pgfqpoint{-0.000000in}{0.000000in}}%
\pgfpathlineto{\pgfqpoint{-0.048611in}{0.000000in}}%
\pgfusepath{stroke,fill}%
}%
\begin{pgfscope}%
\pgfsys@transformshift{0.726250in}{2.931250in}%
\pgfsys@useobject{currentmarker}{}%
\end{pgfscope}%
\end{pgfscope}%
\begin{pgfscope}%
\definecolor{textcolor}{rgb}{0.000000,0.000000,0.000000}%
\pgfsetstrokecolor{textcolor}%
\pgfsetfillcolor{textcolor}%
\pgftext[x=0.391762in, y=2.883765in, left, base]{\color{textcolor}\rmfamily\fontsize{9.000000}{10.800000}\selectfont \(\displaystyle {10^{10}}\)}%
\end{pgfscope}%
\begin{pgfscope}%
\pgfpathrectangle{\pgfqpoint{0.726250in}{0.525000in}}{\pgfqpoint{3.320000in}{2.887500in}}%
\pgfusepath{clip}%
\pgfsetbuttcap%
\pgfsetroundjoin%
\pgfsetlinewidth{0.803000pt}%
\definecolor{currentstroke}{rgb}{0.752941,0.752941,0.752941}%
\pgfsetstrokecolor{currentstroke}%
\pgfsetdash{{2.960000pt}{1.280000pt}}{0.000000pt}%
\pgfpathmoveto{\pgfqpoint{0.726250in}{3.412500in}}%
\pgfpathlineto{\pgfqpoint{4.046250in}{3.412500in}}%
\pgfusepath{stroke}%
\end{pgfscope}%
\begin{pgfscope}%
\pgfsetbuttcap%
\pgfsetroundjoin%
\definecolor{currentfill}{rgb}{0.000000,0.000000,0.000000}%
\pgfsetfillcolor{currentfill}%
\pgfsetlinewidth{0.803000pt}%
\definecolor{currentstroke}{rgb}{0.000000,0.000000,0.000000}%
\pgfsetstrokecolor{currentstroke}%
\pgfsetdash{}{0pt}%
\pgfsys@defobject{currentmarker}{\pgfqpoint{-0.048611in}{0.000000in}}{\pgfqpoint{-0.000000in}{0.000000in}}{%
\pgfpathmoveto{\pgfqpoint{-0.000000in}{0.000000in}}%
\pgfpathlineto{\pgfqpoint{-0.048611in}{0.000000in}}%
\pgfusepath{stroke,fill}%
}%
\begin{pgfscope}%
\pgfsys@transformshift{0.726250in}{3.412500in}%
\pgfsys@useobject{currentmarker}{}%
\end{pgfscope}%
\end{pgfscope}%
\begin{pgfscope}%
\definecolor{textcolor}{rgb}{0.000000,0.000000,0.000000}%
\pgfsetstrokecolor{textcolor}%
\pgfsetfillcolor{textcolor}%
\pgftext[x=0.391762in, y=3.365015in, left, base]{\color{textcolor}\rmfamily\fontsize{9.000000}{10.800000}\selectfont \(\displaystyle {10^{12}}\)}%
\end{pgfscope}%
\begin{pgfscope}%
\definecolor{textcolor}{rgb}{0.000000,0.000000,0.000000}%
\pgfsetstrokecolor{textcolor}%
\pgfsetfillcolor{textcolor}%
\pgftext[x=0.336206in,y=1.968750in,,bottom,rotate=90.000000]{\color{textcolor}\rmfamily\fontsize{9.000000}{10.800000}\selectfont Flux [cm\(\displaystyle ^{-2}\) s\(\displaystyle ^{-1}\) (100 keV)\(\displaystyle ^{-1}\) or cm\(\displaystyle ^{-2}\) s\(\displaystyle ^{-1}\)]}%
\end{pgfscope}%
\begin{pgfscope}%
\pgfpathrectangle{\pgfqpoint{0.726250in}{0.525000in}}{\pgfqpoint{3.320000in}{2.887500in}}%
\pgfusepath{clip}%
\pgfsetrectcap%
\pgfsetroundjoin%
\pgfsetlinewidth{1.003750pt}%
\definecolor{currentstroke}{rgb}{0.000000,0.000000,0.000000}%
\pgfsetstrokecolor{currentstroke}%
\pgfsetdash{}{0pt}%
\pgfpathmoveto{\pgfqpoint{0.716250in}{2.879453in}}%
\pgfpathlineto{\pgfqpoint{0.845668in}{2.917721in}}%
\pgfpathlineto{\pgfqpoint{0.962658in}{2.949997in}}%
\pgfpathlineto{\pgfqpoint{1.060024in}{2.974543in}}%
\pgfpathlineto{\pgfqpoint{1.143414in}{2.993244in}}%
\pgfpathlineto{\pgfqpoint{1.211043in}{3.006239in}}%
\pgfpathlineto{\pgfqpoint{1.271626in}{3.015627in}}%
\pgfpathlineto{\pgfqpoint{1.322114in}{3.021196in}}%
\pgfpathlineto{\pgfqpoint{1.364497in}{3.023663in}}%
\pgfpathlineto{\pgfqpoint{1.400170in}{3.023536in}}%
\pgfpathlineto{\pgfqpoint{1.430142in}{3.021209in}}%
\pgfpathlineto{\pgfqpoint{1.455158in}{3.017018in}}%
\pgfpathlineto{\pgfqpoint{1.475777in}{3.011303in}}%
\pgfpathlineto{\pgfqpoint{1.492419in}{3.004482in}}%
\pgfpathlineto{\pgfqpoint{1.505398in}{2.997143in}}%
\pgfpathlineto{\pgfqpoint{1.518093in}{2.987352in}}%
\pgfpathlineto{\pgfqpoint{1.527437in}{2.977625in}}%
\pgfpathlineto{\pgfqpoint{1.536632in}{2.964642in}}%
\pgfpathlineto{\pgfqpoint{1.542683in}{2.953193in}}%
\pgfpathlineto{\pgfqpoint{1.548672in}{2.937533in}}%
\pgfpathlineto{\pgfqpoint{1.554599in}{2.915901in}}%
\pgfpathlineto{\pgfqpoint{1.557541in}{2.899665in}}%
\pgfpathlineto{\pgfqpoint{1.560467in}{2.872417in}}%
\pgfpathlineto{\pgfqpoint{1.563379in}{2.815693in}}%
\pgfpathlineto{\pgfqpoint{1.563406in}{0.515000in}}%
\pgfpathlineto{\pgfqpoint{1.563406in}{0.515000in}}%
\pgfusepath{stroke}%
\end{pgfscope}%
\begin{pgfscope}%
\pgfpathrectangle{\pgfqpoint{0.726250in}{0.525000in}}{\pgfqpoint{3.320000in}{2.887500in}}%
\pgfusepath{clip}%
\pgfsetrectcap%
\pgfsetroundjoin%
\pgfsetlinewidth{1.003750pt}%
\definecolor{currentstroke}{rgb}{0.000000,0.000000,0.000000}%
\pgfsetstrokecolor{currentstroke}%
\pgfsetdash{}{0pt}%
\pgfpathmoveto{\pgfqpoint{2.279571in}{0.515000in}}%
\pgfpathlineto{\pgfqpoint{2.279571in}{2.488685in}}%
\pgfusepath{stroke}%
\end{pgfscope}%
\begin{pgfscope}%
\pgfpathrectangle{\pgfqpoint{0.726250in}{0.525000in}}{\pgfqpoint{3.320000in}{2.887500in}}%
\pgfusepath{clip}%
\pgfsetrectcap%
\pgfsetroundjoin%
\pgfsetlinewidth{1.003750pt}%
\definecolor{currentstroke}{rgb}{0.000000,0.000000,0.000000}%
\pgfsetstrokecolor{currentstroke}%
\pgfsetdash{}{0pt}%
\pgfpathmoveto{\pgfqpoint{1.703884in}{0.515000in}}%
\pgfpathlineto{\pgfqpoint{1.983755in}{0.611794in}}%
\pgfpathlineto{\pgfqpoint{2.236413in}{0.697104in}}%
\pgfpathlineto{\pgfqpoint{2.441568in}{0.764071in}}%
\pgfpathlineto{\pgfqpoint{2.614760in}{0.818163in}}%
\pgfpathlineto{\pgfqpoint{2.748062in}{0.857614in}}%
\pgfpathlineto{\pgfqpoint{2.870413in}{0.891489in}}%
\pgfpathlineto{\pgfqpoint{2.971459in}{0.917212in}}%
\pgfpathlineto{\pgfqpoint{3.057531in}{0.936994in}}%
\pgfpathlineto{\pgfqpoint{3.132500in}{0.952160in}}%
\pgfpathlineto{\pgfqpoint{3.198905in}{0.963554in}}%
\pgfpathlineto{\pgfqpoint{3.258505in}{0.971711in}}%
\pgfpathlineto{\pgfqpoint{3.312564in}{0.976983in}}%
\pgfpathlineto{\pgfqpoint{3.362027in}{0.979576in}}%
\pgfpathlineto{\pgfqpoint{3.407614in}{0.979589in}}%
\pgfpathlineto{\pgfqpoint{3.444769in}{0.977484in}}%
\pgfpathlineto{\pgfqpoint{3.479694in}{0.973344in}}%
\pgfpathlineto{\pgfqpoint{3.512642in}{0.967042in}}%
\pgfpathlineto{\pgfqpoint{3.539470in}{0.959767in}}%
\pgfpathlineto{\pgfqpoint{3.565116in}{0.950579in}}%
\pgfpathlineto{\pgfqpoint{3.589680in}{0.939205in}}%
\pgfpathlineto{\pgfqpoint{3.609386in}{0.927831in}}%
\pgfpathlineto{\pgfqpoint{3.628447in}{0.914427in}}%
\pgfpathlineto{\pgfqpoint{3.646904in}{0.898586in}}%
\pgfpathlineto{\pgfqpoint{3.664793in}{0.879834in}}%
\pgfpathlineto{\pgfqpoint{3.678719in}{0.862196in}}%
\pgfpathlineto{\pgfqpoint{3.692320in}{0.841697in}}%
\pgfpathlineto{\pgfqpoint{3.705609in}{0.817515in}}%
\pgfpathlineto{\pgfqpoint{3.718603in}{0.788325in}}%
\pgfpathlineto{\pgfqpoint{3.728161in}{0.762148in}}%
\pgfpathlineto{\pgfqpoint{3.737564in}{0.730780in}}%
\pgfpathlineto{\pgfqpoint{3.746819in}{0.691724in}}%
\pgfpathlineto{\pgfqpoint{3.755928in}{0.640581in}}%
\pgfpathlineto{\pgfqpoint{3.761922in}{0.594419in}}%
\pgfpathlineto{\pgfqpoint{3.767856in}{0.526664in}}%
\pgfpathlineto{\pgfqpoint{3.768464in}{0.515000in}}%
\pgfpathlineto{\pgfqpoint{3.768464in}{0.515000in}}%
\pgfusepath{stroke}%
\end{pgfscope}%
\begin{pgfscope}%
\pgfpathrectangle{\pgfqpoint{0.726250in}{0.525000in}}{\pgfqpoint{3.320000in}{2.887500in}}%
\pgfusepath{clip}%
\pgfsetrectcap%
\pgfsetroundjoin%
\pgfsetlinewidth{1.003750pt}%
\definecolor{currentstroke}{rgb}{0.000000,0.000000,0.000000}%
\pgfsetstrokecolor{currentstroke}%
\pgfsetdash{}{0pt}%
\pgfpathmoveto{\pgfqpoint{1.509864in}{0.515000in}}%
\pgfpathlineto{\pgfqpoint{1.509864in}{2.615993in}}%
\pgfusepath{stroke}%
\end{pgfscope}%
\begin{pgfscope}%
\pgfpathrectangle{\pgfqpoint{0.726250in}{0.525000in}}{\pgfqpoint{3.320000in}{2.887500in}}%
\pgfusepath{clip}%
\pgfsetrectcap%
\pgfsetroundjoin%
\pgfsetlinewidth{1.003750pt}%
\definecolor{currentstroke}{rgb}{0.000000,0.000000,0.000000}%
\pgfsetstrokecolor{currentstroke}%
\pgfsetdash{}{0pt}%
\pgfpathmoveto{\pgfqpoint{1.979605in}{0.515000in}}%
\pgfpathlineto{\pgfqpoint{1.979605in}{2.843306in}}%
\pgfusepath{stroke}%
\end{pgfscope}%
\begin{pgfscope}%
\pgfpathrectangle{\pgfqpoint{0.726250in}{0.525000in}}{\pgfqpoint{3.320000in}{2.887500in}}%
\pgfusepath{clip}%
\pgfsetrectcap%
\pgfsetroundjoin%
\pgfsetlinewidth{1.003750pt}%
\definecolor{currentstroke}{rgb}{0.000000,0.000000,0.000000}%
\pgfsetstrokecolor{currentstroke}%
\pgfsetdash{}{0pt}%
\pgfpathmoveto{\pgfqpoint{0.716250in}{1.006638in}}%
\pgfpathlineto{\pgfqpoint{1.022767in}{1.111109in}}%
\pgfpathlineto{\pgfqpoint{1.258776in}{1.188273in}}%
\pgfpathlineto{\pgfqpoint{1.426227in}{1.239162in}}%
\pgfpathlineto{\pgfqpoint{1.959572in}{1.412319in}}%
\pgfpathlineto{\pgfqpoint{2.195581in}{1.485642in}}%
\pgfpathlineto{\pgfqpoint{2.363032in}{1.535141in}}%
\pgfpathlineto{\pgfqpoint{2.492917in}{1.571365in}}%
\pgfpathlineto{\pgfqpoint{2.618127in}{1.603910in}}%
\pgfpathlineto{\pgfqpoint{2.721113in}{1.628394in}}%
\pgfpathlineto{\pgfqpoint{2.808588in}{1.647082in}}%
\pgfpathlineto{\pgfqpoint{2.884618in}{1.661333in}}%
\pgfpathlineto{\pgfqpoint{2.962343in}{1.673501in}}%
\pgfpathlineto{\pgfqpoint{3.030900in}{1.681691in}}%
\pgfpathlineto{\pgfqpoint{3.083852in}{1.685968in}}%
\pgfpathlineto{\pgfqpoint{3.132386in}{1.687948in}}%
\pgfpathlineto{\pgfqpoint{3.177183in}{1.687767in}}%
\pgfpathlineto{\pgfqpoint{3.218777in}{1.685494in}}%
\pgfpathlineto{\pgfqpoint{3.257596in}{1.681132in}}%
\pgfpathlineto{\pgfqpoint{3.293987in}{1.674627in}}%
\pgfpathlineto{\pgfqpoint{3.322667in}{1.667486in}}%
\pgfpathlineto{\pgfqpoint{3.349999in}{1.658672in}}%
\pgfpathlineto{\pgfqpoint{3.376105in}{1.648033in}}%
\pgfpathlineto{\pgfqpoint{3.401090in}{1.635366in}}%
\pgfpathlineto{\pgfqpoint{3.425047in}{1.620412in}}%
\pgfpathlineto{\pgfqpoint{3.443527in}{1.606558in}}%
\pgfpathlineto{\pgfqpoint{3.461438in}{1.590762in}}%
\pgfpathlineto{\pgfqpoint{3.478815in}{1.572731in}}%
\pgfpathlineto{\pgfqpoint{3.495688in}{1.552031in}}%
\pgfpathlineto{\pgfqpoint{3.512085in}{1.528155in}}%
\pgfpathlineto{\pgfqpoint{3.528033in}{1.500314in}}%
\pgfpathlineto{\pgfqpoint{3.543556in}{1.467530in}}%
\pgfpathlineto{\pgfqpoint{3.554932in}{1.438763in}}%
\pgfpathlineto{\pgfqpoint{3.566091in}{1.405592in}}%
\pgfpathlineto{\pgfqpoint{3.577039in}{1.366615in}}%
\pgfpathlineto{\pgfqpoint{3.587785in}{1.320970in}}%
\pgfpathlineto{\pgfqpoint{3.598337in}{1.266515in}}%
\pgfpathlineto{\pgfqpoint{3.612114in}{1.180842in}}%
\pgfpathlineto{\pgfqpoint{3.625573in}{1.079760in}}%
\pgfpathlineto{\pgfqpoint{3.641970in}{0.940576in}}%
\pgfpathlineto{\pgfqpoint{3.654764in}{0.810193in}}%
\pgfpathlineto{\pgfqpoint{3.664177in}{0.698977in}}%
\pgfpathlineto{\pgfqpoint{3.673441in}{0.564547in}}%
\pgfpathlineto{\pgfqpoint{3.676174in}{0.515000in}}%
\pgfpathlineto{\pgfqpoint{3.676174in}{0.515000in}}%
\pgfusepath{stroke}%
\end{pgfscope}%
\begin{pgfscope}%
\pgfpathrectangle{\pgfqpoint{0.726250in}{0.525000in}}{\pgfqpoint{3.320000in}{2.887500in}}%
\pgfusepath{clip}%
\pgfsetbuttcap%
\pgfsetroundjoin%
\pgfsetlinewidth{1.003750pt}%
\definecolor{currentstroke}{rgb}{0.000000,0.000000,0.000000}%
\pgfsetstrokecolor{currentstroke}%
\pgfsetdash{{3.700000pt}{1.600000pt}}{0.000000pt}%
\pgfpathmoveto{\pgfqpoint{0.716250in}{2.048511in}}%
\pgfpathlineto{\pgfqpoint{0.938013in}{2.121760in}}%
\pgfpathlineto{\pgfqpoint{1.123375in}{2.180473in}}%
\pgfpathlineto{\pgfqpoint{1.263748in}{2.222705in}}%
\pgfpathlineto{\pgfqpoint{1.388063in}{2.257834in}}%
\pgfpathlineto{\pgfqpoint{1.490444in}{2.284589in}}%
\pgfpathlineto{\pgfqpoint{1.585510in}{2.307067in}}%
\pgfpathlineto{\pgfqpoint{1.667208in}{2.323975in}}%
\pgfpathlineto{\pgfqpoint{1.738838in}{2.336358in}}%
\pgfpathlineto{\pgfqpoint{1.797093in}{2.344234in}}%
\pgfpathlineto{\pgfqpoint{1.850045in}{2.349175in}}%
\pgfpathlineto{\pgfqpoint{1.893904in}{2.351191in}}%
\pgfpathlineto{\pgfqpoint{1.934688in}{2.350884in}}%
\pgfpathlineto{\pgfqpoint{1.968687in}{2.348533in}}%
\pgfpathlineto{\pgfqpoint{2.000810in}{2.344017in}}%
\pgfpathlineto{\pgfqpoint{2.027532in}{2.337984in}}%
\pgfpathlineto{\pgfqpoint{2.049500in}{2.330947in}}%
\pgfpathlineto{\pgfqpoint{2.070668in}{2.321761in}}%
\pgfpathlineto{\pgfqpoint{2.087739in}{2.311905in}}%
\pgfpathlineto{\pgfqpoint{2.101044in}{2.302149in}}%
\pgfpathlineto{\pgfqpoint{2.114052in}{2.290062in}}%
\pgfpathlineto{\pgfqpoint{2.126775in}{2.274573in}}%
\pgfpathlineto{\pgfqpoint{2.136138in}{2.259505in}}%
\pgfpathlineto{\pgfqpoint{2.145353in}{2.239347in}}%
\pgfpathlineto{\pgfqpoint{2.151417in}{2.221230in}}%
\pgfpathlineto{\pgfqpoint{2.157417in}{2.195976in}}%
\pgfpathlineto{\pgfqpoint{2.163357in}{2.154853in}}%
\pgfpathlineto{\pgfqpoint{2.166304in}{2.119035in}}%
\pgfpathlineto{\pgfqpoint{2.169236in}{2.048478in}}%
\pgfpathlineto{\pgfqpoint{2.169255in}{0.515000in}}%
\pgfpathlineto{\pgfqpoint{2.169255in}{0.515000in}}%
\pgfusepath{stroke}%
\end{pgfscope}%
\begin{pgfscope}%
\pgfpathrectangle{\pgfqpoint{0.726250in}{0.525000in}}{\pgfqpoint{3.320000in}{2.887500in}}%
\pgfusepath{clip}%
\pgfsetbuttcap%
\pgfsetroundjoin%
\pgfsetlinewidth{1.003750pt}%
\definecolor{currentstroke}{rgb}{0.000000,0.000000,0.000000}%
\pgfsetstrokecolor{currentstroke}%
\pgfsetdash{{3.700000pt}{1.600000pt}}{0.000000pt}%
\pgfpathmoveto{\pgfqpoint{0.716250in}{1.938440in}}%
\pgfpathlineto{\pgfqpoint{0.932165in}{2.011485in}}%
\pgfpathlineto{\pgfqpoint{1.139077in}{2.079413in}}%
\pgfpathlineto{\pgfqpoint{1.310255in}{2.133344in}}%
\pgfpathlineto{\pgfqpoint{1.456954in}{2.177198in}}%
\pgfpathlineto{\pgfqpoint{1.574024in}{2.209999in}}%
\pgfpathlineto{\pgfqpoint{1.681327in}{2.237737in}}%
\pgfpathlineto{\pgfqpoint{1.771895in}{2.258843in}}%
\pgfpathlineto{\pgfqpoint{1.850249in}{2.274856in}}%
\pgfpathlineto{\pgfqpoint{1.919297in}{2.286719in}}%
\pgfpathlineto{\pgfqpoint{1.981015in}{2.295019in}}%
\pgfpathlineto{\pgfqpoint{2.031467in}{2.299733in}}%
\pgfpathlineto{\pgfqpoint{2.077893in}{2.301968in}}%
\pgfpathlineto{\pgfqpoint{2.120888in}{2.301743in}}%
\pgfpathlineto{\pgfqpoint{2.156609in}{2.299407in}}%
\pgfpathlineto{\pgfqpoint{2.190264in}{2.294893in}}%
\pgfpathlineto{\pgfqpoint{2.218197in}{2.288954in}}%
\pgfpathlineto{\pgfqpoint{2.244849in}{2.280848in}}%
\pgfpathlineto{\pgfqpoint{2.266762in}{2.271794in}}%
\pgfpathlineto{\pgfqpoint{2.284413in}{2.262447in}}%
\pgfpathlineto{\pgfqpoint{2.301544in}{2.250942in}}%
\pgfpathlineto{\pgfqpoint{2.314895in}{2.239779in}}%
\pgfpathlineto{\pgfqpoint{2.327946in}{2.226137in}}%
\pgfpathlineto{\pgfqpoint{2.340712in}{2.209030in}}%
\pgfpathlineto{\pgfqpoint{2.350105in}{2.192677in}}%
\pgfpathlineto{\pgfqpoint{2.359349in}{2.171280in}}%
\pgfpathlineto{\pgfqpoint{2.365432in}{2.152274in}}%
\pgfpathlineto{\pgfqpoint{2.371451in}{2.126290in}}%
\pgfpathlineto{\pgfqpoint{2.377409in}{2.084734in}}%
\pgfpathlineto{\pgfqpoint{2.380365in}{2.048880in}}%
\pgfpathlineto{\pgfqpoint{2.383306in}{1.979437in}}%
\pgfpathlineto{\pgfqpoint{2.383324in}{0.515000in}}%
\pgfpathlineto{\pgfqpoint{2.383324in}{0.515000in}}%
\pgfusepath{stroke}%
\end{pgfscope}%
\begin{pgfscope}%
\pgfpathrectangle{\pgfqpoint{0.726250in}{0.525000in}}{\pgfqpoint{3.320000in}{2.887500in}}%
\pgfusepath{clip}%
\pgfsetbuttcap%
\pgfsetroundjoin%
\pgfsetlinewidth{1.003750pt}%
\definecolor{currentstroke}{rgb}{0.000000,0.000000,0.000000}%
\pgfsetstrokecolor{currentstroke}%
\pgfsetdash{{3.700000pt}{1.600000pt}}{0.000000pt}%
\pgfpathmoveto{\pgfqpoint{0.716250in}{1.534579in}}%
\pgfpathlineto{\pgfqpoint{0.934847in}{1.608526in}}%
\pgfpathlineto{\pgfqpoint{1.141759in}{1.676449in}}%
\pgfpathlineto{\pgfqpoint{1.312937in}{1.730376in}}%
\pgfpathlineto{\pgfqpoint{1.459637in}{1.774213in}}%
\pgfpathlineto{\pgfqpoint{1.576706in}{1.807000in}}%
\pgfpathlineto{\pgfqpoint{1.684010in}{1.834719in}}%
\pgfpathlineto{\pgfqpoint{1.774577in}{1.855803in}}%
\pgfpathlineto{\pgfqpoint{1.852932in}{1.871791in}}%
\pgfpathlineto{\pgfqpoint{1.921979in}{1.883626in}}%
\pgfpathlineto{\pgfqpoint{1.983698in}{1.891891in}}%
\pgfpathlineto{\pgfqpoint{2.034150in}{1.896565in}}%
\pgfpathlineto{\pgfqpoint{2.080575in}{1.898753in}}%
\pgfpathlineto{\pgfqpoint{2.123570in}{1.898473in}}%
\pgfpathlineto{\pgfqpoint{2.159291in}{1.896073in}}%
\pgfpathlineto{\pgfqpoint{2.192947in}{1.891485in}}%
\pgfpathlineto{\pgfqpoint{2.220879in}{1.885453in}}%
\pgfpathlineto{\pgfqpoint{2.243798in}{1.878553in}}%
\pgfpathlineto{\pgfqpoint{2.265849in}{1.869743in}}%
\pgfpathlineto{\pgfqpoint{2.283607in}{1.860624in}}%
\pgfpathlineto{\pgfqpoint{2.300840in}{1.849433in}}%
\pgfpathlineto{\pgfqpoint{2.314268in}{1.838565in}}%
\pgfpathlineto{\pgfqpoint{2.327393in}{1.825393in}}%
\pgfpathlineto{\pgfqpoint{2.340229in}{1.808954in}}%
\pgfpathlineto{\pgfqpoint{2.349673in}{1.793430in}}%
\pgfpathlineto{\pgfqpoint{2.358967in}{1.773504in}}%
\pgfpathlineto{\pgfqpoint{2.365081in}{1.756266in}}%
\pgfpathlineto{\pgfqpoint{2.371132in}{1.733513in}}%
\pgfpathlineto{\pgfqpoint{2.377120in}{1.700037in}}%
\pgfpathlineto{\pgfqpoint{2.380091in}{1.674924in}}%
\pgfpathlineto{\pgfqpoint{2.383048in}{1.636867in}}%
\pgfpathlineto{\pgfqpoint{2.385989in}{1.561768in}}%
\pgfpathlineto{\pgfqpoint{2.386001in}{0.515000in}}%
\pgfpathlineto{\pgfqpoint{2.386001in}{0.515000in}}%
\pgfusepath{stroke}%
\end{pgfscope}%
\begin{pgfscope}%
\pgfpathrectangle{\pgfqpoint{0.726250in}{0.525000in}}{\pgfqpoint{3.320000in}{2.887500in}}%
\pgfusepath{clip}%
\pgfsetbuttcap%
\pgfsetroundjoin%
\pgfsetlinewidth{1.003750pt}%
\definecolor{currentstroke}{rgb}{0.000000,0.000000,0.000000}%
\pgfsetstrokecolor{currentstroke}%
\pgfsetdash{{3.700000pt}{1.600000pt}}{0.000000pt}%
\pgfpathmoveto{\pgfqpoint{2.530196in}{0.515000in}}%
\pgfpathlineto{\pgfqpoint{2.530196in}{1.818299in}}%
\pgfusepath{stroke}%
\end{pgfscope}%
\begin{pgfscope}%
\pgfpathrectangle{\pgfqpoint{0.726250in}{0.525000in}}{\pgfqpoint{3.320000in}{2.887500in}}%
\pgfusepath{clip}%
\pgfsetbuttcap%
\pgfsetroundjoin%
\pgfsetlinewidth{1.003750pt}%
\definecolor{currentstroke}{rgb}{0.000000,0.000000,0.000000}%
\pgfsetstrokecolor{currentstroke}%
\pgfsetdash{{3.700000pt}{1.600000pt}}{0.000000pt}%
\pgfpathmoveto{\pgfqpoint{2.655549in}{0.515000in}}%
\pgfpathlineto{\pgfqpoint{2.655549in}{1.707450in}}%
\pgfusepath{stroke}%
\end{pgfscope}%
\begin{pgfscope}%
\pgfsetrectcap%
\pgfsetmiterjoin%
\pgfsetlinewidth{1.003750pt}%
\definecolor{currentstroke}{rgb}{0.000000,0.000000,0.000000}%
\pgfsetstrokecolor{currentstroke}%
\pgfsetdash{}{0pt}%
\pgfpathmoveto{\pgfqpoint{0.726250in}{0.525000in}}%
\pgfpathlineto{\pgfqpoint{0.726250in}{3.412500in}}%
\pgfusepath{stroke}%
\end{pgfscope}%
\begin{pgfscope}%
\pgfsetrectcap%
\pgfsetmiterjoin%
\pgfsetlinewidth{1.003750pt}%
\definecolor{currentstroke}{rgb}{0.000000,0.000000,0.000000}%
\pgfsetstrokecolor{currentstroke}%
\pgfsetdash{}{0pt}%
\pgfpathmoveto{\pgfqpoint{4.046250in}{0.525000in}}%
\pgfpathlineto{\pgfqpoint{4.046250in}{3.412500in}}%
\pgfusepath{stroke}%
\end{pgfscope}%
\begin{pgfscope}%
\pgfsetrectcap%
\pgfsetmiterjoin%
\pgfsetlinewidth{1.003750pt}%
\definecolor{currentstroke}{rgb}{0.000000,0.000000,0.000000}%
\pgfsetstrokecolor{currentstroke}%
\pgfsetdash{}{0pt}%
\pgfpathmoveto{\pgfqpoint{0.726250in}{0.525000in}}%
\pgfpathlineto{\pgfqpoint{4.046250in}{0.525000in}}%
\pgfusepath{stroke}%
\end{pgfscope}%
\begin{pgfscope}%
\pgfsetrectcap%
\pgfsetmiterjoin%
\pgfsetlinewidth{1.003750pt}%
\definecolor{currentstroke}{rgb}{0.000000,0.000000,0.000000}%
\pgfsetstrokecolor{currentstroke}%
\pgfsetdash{}{0pt}%
\pgfpathmoveto{\pgfqpoint{0.726250in}{3.412500in}}%
\pgfpathlineto{\pgfqpoint{4.046250in}{3.412500in}}%
\pgfusepath{stroke}%
\end{pgfscope}%
\begin{pgfscope}%
\definecolor{textcolor}{rgb}{0.000000,0.000000,0.000000}%
\pgfsetstrokecolor{textcolor}%
\pgfsetfillcolor{textcolor}%
\pgftext[x=1.365719in,y=3.099440in,left,base]{\color{textcolor}\rmfamily\fontsize{9.000000}{10.800000}\selectfont \textit{pp} (91.6\%)}%
\end{pgfscope}%
\begin{pgfscope}%
\definecolor{textcolor}{rgb}{0.000000,0.000000,0.000000}%
\pgfsetstrokecolor{textcolor}%
\pgfsetfillcolor{textcolor}%
\pgftext[x=2.279571in,y=2.561120in,left,base]{\color{textcolor}\rmfamily\fontsize{9.000000}{10.800000}\selectfont \textit{pep} (0.2\%)}%
\end{pgfscope}%
\begin{pgfscope}%
\definecolor{textcolor}{rgb}{0.000000,0.000000,0.000000}%
\pgfsetstrokecolor{textcolor}%
\pgfsetfillcolor{textcolor}%
\pgftext[x=3.406781in,y=1.048622in,right,base]{\color{textcolor}\rmfamily\fontsize{9.000000}{10.800000}\selectfont \textit{hep}}%
\end{pgfscope}%
\begin{pgfscope}%
\definecolor{textcolor}{rgb}{0.000000,0.000000,0.000000}%
\pgfsetstrokecolor{textcolor}%
\pgfsetfillcolor{textcolor}%
\pgftext[x=1.509864in,y=2.658365in,right,base]{\color{textcolor}\rmfamily\fontsize{9.000000}{10.800000}\selectfont \(\displaystyle {}^7Be\) (0.8\%)}%
\end{pgfscope}%
\begin{pgfscope}%
\definecolor{textcolor}{rgb}{0.000000,0.000000,0.000000}%
\pgfsetstrokecolor{textcolor}%
\pgfsetfillcolor{textcolor}%
\pgftext[x=1.979605in,y=2.885678in,left,base]{\color{textcolor}\rmfamily\fontsize{9.000000}{10.800000}\selectfont \(\displaystyle {}^7Be\) (6.6\%)}%
\end{pgfscope}%
\begin{pgfscope}%
\definecolor{textcolor}{rgb}{0.000000,0.000000,0.000000}%
\pgfsetstrokecolor{textcolor}%
\pgfsetfillcolor{textcolor}%
\pgftext[x=3.199171in,y=1.728125in,left,base]{\color{textcolor}\rmfamily\fontsize{9.000000}{10.800000}\selectfont \(\displaystyle {}^8B\) (0.008\%)}%
\end{pgfscope}%
\end{pgfpicture}%
\makeatother%
\endgroup%

  \caption{Summary of all the neutrino spectral components from the $pp$ cycle. For
  completeness the neutrino spectra from the CNO cycle are also shown with dashed
  lines. Note that continuous spectra are shown in units of cm$^{-2}$~s$^{-1}$~(100~keV)$^{-1}$,
  while lines from electron-capture reactions are shown in units of cm$^{-2}$~s$^{-1}$,
  so any visual comparison between the correspon normalization should be done
  \emph{cum grano salis}.}
  \label{fig:sun_neutrino_spectra}
\end{figure}

At this point the chain of fusion reactions can proceed via different avenues, all
ending up with a $\ce{^2He}$, whose relative importance is determined by the temperature
at which the process takes place. In the Sun, the most probable reaction is the so
called $pp$-I branch, or
\begin{align*}
  \ce{^3He} + \ce{^3He} \rightarrow \ce{^4H} + p + p + 12.860~\text{MeV}.
\end{align*}
The entire branch $p$-$p$-I is sometimes written more compactly as a single
reaction with the proper sum of the $Q$ values
\begin{align*}
  4p + 2e^- \rightarrow \ce{^4H} + 2\nu_e + 26.73~\text{MeV}.
\end{align*}
It is worth noting that, out of the $26.73$~MeV available, at most $0.84$~MeV can
go into neutrinos, and $0.53$~MeV (or 2\%) on average, which is consistent with the
assumption we have made at the beginning of the section when estimating the total
neutrino flux from the Sun electromagnetic luminosity.

The other two branches of the $pp$ chain, customarily referred as $pp$-II and $pp$-II,
proceed via the synthesis of heavier nuclei, with two distinct reactions producing
electron neutrinos: the electronic capture by $\ce{^7Be}$
\begin{align*}
  \ce{^7Be} + e^- \rightarrow \ce{^7Li} + \nu_e + 0.384~\text{or}~0.862~\text{MeV},
\end{align*}
producing monochromatic\sidenote{The neutrino spectrum is a sharp line because this
process has two bodies in the final states, as opposed to channels with three bodies
in the final state, producing continuous spectra.} neutrinos at two distinct energies,
corresponding to the ground state and an excited (metastable) state of the daughter
$\ce{^7Li}$ nucleus, and the decay of $\ce{^8B}$
\begin{align*}
  \ce{^8B} \rightarrow \ce{^8Be} + e^+ + \nu_e + 17.980~\text{MeV},
\end{align*}
producing a continuous spectrum extending to almost 20~MeV.

There is one last reaction producing neutrino that we have left out (number~4 in
table~\ref{tab:pp_cycle}), which goes under the name of $hep$, or $pp$-IV
\begin{align*}
  \ce{^3He} + p \rightarrow \ce{^4He} + e^+ + \nu_e + 19.795~\text{MeV}
\end{align*}
producing again a continuous spectrum extending to almost 20~MeV.

\begin{figure}[!htbp]
  %% Creator: Matplotlib, PGF backend
%%
%% To include the figure in your LaTeX document, write
%%   \input{<filename>.pgf}
%%
%% Make sure the required packages are loaded in your preamble
%%   \usepackage{pgf}
%%
%% Also ensure that all the required font packages are loaded; for instance,
%% the lmodern package is sometimes necessary when using math font.
%%   \usepackage{lmodern}
%%
%% Figures using additional raster images can only be included by \input if
%% they are in the same directory as the main LaTeX file. For loading figures
%% from other directories you can use the `import` package
%%   \usepackage{import}
%%
%% and then include the figures with
%%   \import{<path to file>}{<filename>.pgf}
%%
%% Matplotlib used the following preamble
%%   \usepackage{fontspec}
%%   \setmainfont{DejaVuSerif.ttf}[Path=\detokenize{/usr/share/matplotlib/mpl-data/fonts/ttf/}]
%%   \setsansfont{DejaVuSans.ttf}[Path=\detokenize{/usr/share/matplotlib/mpl-data/fonts/ttf/}]
%%   \setmonofont{DejaVuSansMono.ttf}[Path=\detokenize{/usr/share/matplotlib/mpl-data/fonts/ttf/}]
%%
\begingroup%
\makeatletter%
\begin{pgfpicture}%
\pgfpathrectangle{\pgfpointorigin}{\pgfqpoint{4.150000in}{3.500000in}}%
\pgfusepath{use as bounding box, clip}%
\begin{pgfscope}%
\pgfsetbuttcap%
\pgfsetmiterjoin%
\definecolor{currentfill}{rgb}{1.000000,1.000000,1.000000}%
\pgfsetfillcolor{currentfill}%
\pgfsetlinewidth{0.000000pt}%
\definecolor{currentstroke}{rgb}{1.000000,1.000000,1.000000}%
\pgfsetstrokecolor{currentstroke}%
\pgfsetdash{}{0pt}%
\pgfpathmoveto{\pgfqpoint{0.000000in}{0.000000in}}%
\pgfpathlineto{\pgfqpoint{4.150000in}{0.000000in}}%
\pgfpathlineto{\pgfqpoint{4.150000in}{3.500000in}}%
\pgfpathlineto{\pgfqpoint{0.000000in}{3.500000in}}%
\pgfpathlineto{\pgfqpoint{0.000000in}{0.000000in}}%
\pgfpathclose%
\pgfusepath{fill}%
\end{pgfscope}%
\begin{pgfscope}%
\pgfsetbuttcap%
\pgfsetmiterjoin%
\definecolor{currentfill}{rgb}{1.000000,1.000000,1.000000}%
\pgfsetfillcolor{currentfill}%
\pgfsetlinewidth{0.000000pt}%
\definecolor{currentstroke}{rgb}{0.000000,0.000000,0.000000}%
\pgfsetstrokecolor{currentstroke}%
\pgfsetstrokeopacity{0.000000}%
\pgfsetdash{}{0pt}%
\pgfpathmoveto{\pgfqpoint{0.726250in}{0.525000in}}%
\pgfpathlineto{\pgfqpoint{4.046250in}{0.525000in}}%
\pgfpathlineto{\pgfqpoint{4.046250in}{3.412500in}}%
\pgfpathlineto{\pgfqpoint{0.726250in}{3.412500in}}%
\pgfpathlineto{\pgfqpoint{0.726250in}{0.525000in}}%
\pgfpathclose%
\pgfusepath{fill}%
\end{pgfscope}%
\begin{pgfscope}%
\pgfpathrectangle{\pgfqpoint{0.726250in}{0.525000in}}{\pgfqpoint{3.320000in}{2.887500in}}%
\pgfusepath{clip}%
\pgfsetbuttcap%
\pgfsetroundjoin%
\pgfsetlinewidth{0.803000pt}%
\definecolor{currentstroke}{rgb}{0.752941,0.752941,0.752941}%
\pgfsetstrokecolor{currentstroke}%
\pgfsetdash{{2.960000pt}{1.280000pt}}{0.000000pt}%
\pgfpathmoveto{\pgfqpoint{1.325830in}{0.525000in}}%
\pgfpathlineto{\pgfqpoint{1.325830in}{3.412500in}}%
\pgfusepath{stroke}%
\end{pgfscope}%
\begin{pgfscope}%
\pgfsetbuttcap%
\pgfsetroundjoin%
\definecolor{currentfill}{rgb}{0.000000,0.000000,0.000000}%
\pgfsetfillcolor{currentfill}%
\pgfsetlinewidth{0.803000pt}%
\definecolor{currentstroke}{rgb}{0.000000,0.000000,0.000000}%
\pgfsetstrokecolor{currentstroke}%
\pgfsetdash{}{0pt}%
\pgfsys@defobject{currentmarker}{\pgfqpoint{0.000000in}{-0.048611in}}{\pgfqpoint{0.000000in}{0.000000in}}{%
\pgfpathmoveto{\pgfqpoint{0.000000in}{0.000000in}}%
\pgfpathlineto{\pgfqpoint{0.000000in}{-0.048611in}}%
\pgfusepath{stroke,fill}%
}%
\begin{pgfscope}%
\pgfsys@transformshift{1.325830in}{0.525000in}%
\pgfsys@useobject{currentmarker}{}%
\end{pgfscope}%
\end{pgfscope}%
\begin{pgfscope}%
\definecolor{textcolor}{rgb}{0.000000,0.000000,0.000000}%
\pgfsetstrokecolor{textcolor}%
\pgfsetfillcolor{textcolor}%
\pgftext[x=1.325830in,y=0.427778in,,top]{\color{textcolor}\rmfamily\fontsize{9.000000}{10.800000}\selectfont 0.05}%
\end{pgfscope}%
\begin{pgfscope}%
\pgfpathrectangle{\pgfqpoint{0.726250in}{0.525000in}}{\pgfqpoint{3.320000in}{2.887500in}}%
\pgfusepath{clip}%
\pgfsetbuttcap%
\pgfsetroundjoin%
\pgfsetlinewidth{0.803000pt}%
\definecolor{currentstroke}{rgb}{0.752941,0.752941,0.752941}%
\pgfsetstrokecolor{currentstroke}%
\pgfsetdash{{2.960000pt}{1.280000pt}}{0.000000pt}%
\pgfpathmoveto{\pgfqpoint{1.930368in}{0.525000in}}%
\pgfpathlineto{\pgfqpoint{1.930368in}{3.412500in}}%
\pgfusepath{stroke}%
\end{pgfscope}%
\begin{pgfscope}%
\pgfsetbuttcap%
\pgfsetroundjoin%
\definecolor{currentfill}{rgb}{0.000000,0.000000,0.000000}%
\pgfsetfillcolor{currentfill}%
\pgfsetlinewidth{0.803000pt}%
\definecolor{currentstroke}{rgb}{0.000000,0.000000,0.000000}%
\pgfsetstrokecolor{currentstroke}%
\pgfsetdash{}{0pt}%
\pgfsys@defobject{currentmarker}{\pgfqpoint{0.000000in}{-0.048611in}}{\pgfqpoint{0.000000in}{0.000000in}}{%
\pgfpathmoveto{\pgfqpoint{0.000000in}{0.000000in}}%
\pgfpathlineto{\pgfqpoint{0.000000in}{-0.048611in}}%
\pgfusepath{stroke,fill}%
}%
\begin{pgfscope}%
\pgfsys@transformshift{1.930368in}{0.525000in}%
\pgfsys@useobject{currentmarker}{}%
\end{pgfscope}%
\end{pgfscope}%
\begin{pgfscope}%
\definecolor{textcolor}{rgb}{0.000000,0.000000,0.000000}%
\pgfsetstrokecolor{textcolor}%
\pgfsetfillcolor{textcolor}%
\pgftext[x=1.930368in,y=0.427778in,,top]{\color{textcolor}\rmfamily\fontsize{9.000000}{10.800000}\selectfont 0.10}%
\end{pgfscope}%
\begin{pgfscope}%
\pgfpathrectangle{\pgfqpoint{0.726250in}{0.525000in}}{\pgfqpoint{3.320000in}{2.887500in}}%
\pgfusepath{clip}%
\pgfsetbuttcap%
\pgfsetroundjoin%
\pgfsetlinewidth{0.803000pt}%
\definecolor{currentstroke}{rgb}{0.752941,0.752941,0.752941}%
\pgfsetstrokecolor{currentstroke}%
\pgfsetdash{{2.960000pt}{1.280000pt}}{0.000000pt}%
\pgfpathmoveto{\pgfqpoint{2.534906in}{0.525000in}}%
\pgfpathlineto{\pgfqpoint{2.534906in}{3.412500in}}%
\pgfusepath{stroke}%
\end{pgfscope}%
\begin{pgfscope}%
\pgfsetbuttcap%
\pgfsetroundjoin%
\definecolor{currentfill}{rgb}{0.000000,0.000000,0.000000}%
\pgfsetfillcolor{currentfill}%
\pgfsetlinewidth{0.803000pt}%
\definecolor{currentstroke}{rgb}{0.000000,0.000000,0.000000}%
\pgfsetstrokecolor{currentstroke}%
\pgfsetdash{}{0pt}%
\pgfsys@defobject{currentmarker}{\pgfqpoint{0.000000in}{-0.048611in}}{\pgfqpoint{0.000000in}{0.000000in}}{%
\pgfpathmoveto{\pgfqpoint{0.000000in}{0.000000in}}%
\pgfpathlineto{\pgfqpoint{0.000000in}{-0.048611in}}%
\pgfusepath{stroke,fill}%
}%
\begin{pgfscope}%
\pgfsys@transformshift{2.534906in}{0.525000in}%
\pgfsys@useobject{currentmarker}{}%
\end{pgfscope}%
\end{pgfscope}%
\begin{pgfscope}%
\definecolor{textcolor}{rgb}{0.000000,0.000000,0.000000}%
\pgfsetstrokecolor{textcolor}%
\pgfsetfillcolor{textcolor}%
\pgftext[x=2.534906in,y=0.427778in,,top]{\color{textcolor}\rmfamily\fontsize{9.000000}{10.800000}\selectfont 0.15}%
\end{pgfscope}%
\begin{pgfscope}%
\pgfpathrectangle{\pgfqpoint{0.726250in}{0.525000in}}{\pgfqpoint{3.320000in}{2.887500in}}%
\pgfusepath{clip}%
\pgfsetbuttcap%
\pgfsetroundjoin%
\pgfsetlinewidth{0.803000pt}%
\definecolor{currentstroke}{rgb}{0.752941,0.752941,0.752941}%
\pgfsetstrokecolor{currentstroke}%
\pgfsetdash{{2.960000pt}{1.280000pt}}{0.000000pt}%
\pgfpathmoveto{\pgfqpoint{3.139443in}{0.525000in}}%
\pgfpathlineto{\pgfqpoint{3.139443in}{3.412500in}}%
\pgfusepath{stroke}%
\end{pgfscope}%
\begin{pgfscope}%
\pgfsetbuttcap%
\pgfsetroundjoin%
\definecolor{currentfill}{rgb}{0.000000,0.000000,0.000000}%
\pgfsetfillcolor{currentfill}%
\pgfsetlinewidth{0.803000pt}%
\definecolor{currentstroke}{rgb}{0.000000,0.000000,0.000000}%
\pgfsetstrokecolor{currentstroke}%
\pgfsetdash{}{0pt}%
\pgfsys@defobject{currentmarker}{\pgfqpoint{0.000000in}{-0.048611in}}{\pgfqpoint{0.000000in}{0.000000in}}{%
\pgfpathmoveto{\pgfqpoint{0.000000in}{0.000000in}}%
\pgfpathlineto{\pgfqpoint{0.000000in}{-0.048611in}}%
\pgfusepath{stroke,fill}%
}%
\begin{pgfscope}%
\pgfsys@transformshift{3.139443in}{0.525000in}%
\pgfsys@useobject{currentmarker}{}%
\end{pgfscope}%
\end{pgfscope}%
\begin{pgfscope}%
\definecolor{textcolor}{rgb}{0.000000,0.000000,0.000000}%
\pgfsetstrokecolor{textcolor}%
\pgfsetfillcolor{textcolor}%
\pgftext[x=3.139443in,y=0.427778in,,top]{\color{textcolor}\rmfamily\fontsize{9.000000}{10.800000}\selectfont 0.20}%
\end{pgfscope}%
\begin{pgfscope}%
\pgfpathrectangle{\pgfqpoint{0.726250in}{0.525000in}}{\pgfqpoint{3.320000in}{2.887500in}}%
\pgfusepath{clip}%
\pgfsetbuttcap%
\pgfsetroundjoin%
\pgfsetlinewidth{0.803000pt}%
\definecolor{currentstroke}{rgb}{0.752941,0.752941,0.752941}%
\pgfsetstrokecolor{currentstroke}%
\pgfsetdash{{2.960000pt}{1.280000pt}}{0.000000pt}%
\pgfpathmoveto{\pgfqpoint{3.743981in}{0.525000in}}%
\pgfpathlineto{\pgfqpoint{3.743981in}{3.412500in}}%
\pgfusepath{stroke}%
\end{pgfscope}%
\begin{pgfscope}%
\pgfsetbuttcap%
\pgfsetroundjoin%
\definecolor{currentfill}{rgb}{0.000000,0.000000,0.000000}%
\pgfsetfillcolor{currentfill}%
\pgfsetlinewidth{0.803000pt}%
\definecolor{currentstroke}{rgb}{0.000000,0.000000,0.000000}%
\pgfsetstrokecolor{currentstroke}%
\pgfsetdash{}{0pt}%
\pgfsys@defobject{currentmarker}{\pgfqpoint{0.000000in}{-0.048611in}}{\pgfqpoint{0.000000in}{0.000000in}}{%
\pgfpathmoveto{\pgfqpoint{0.000000in}{0.000000in}}%
\pgfpathlineto{\pgfqpoint{0.000000in}{-0.048611in}}%
\pgfusepath{stroke,fill}%
}%
\begin{pgfscope}%
\pgfsys@transformshift{3.743981in}{0.525000in}%
\pgfsys@useobject{currentmarker}{}%
\end{pgfscope}%
\end{pgfscope}%
\begin{pgfscope}%
\definecolor{textcolor}{rgb}{0.000000,0.000000,0.000000}%
\pgfsetstrokecolor{textcolor}%
\pgfsetfillcolor{textcolor}%
\pgftext[x=3.743981in,y=0.427778in,,top]{\color{textcolor}\rmfamily\fontsize{9.000000}{10.800000}\selectfont 0.25}%
\end{pgfscope}%
\begin{pgfscope}%
\definecolor{textcolor}{rgb}{0.000000,0.000000,0.000000}%
\pgfsetstrokecolor{textcolor}%
\pgfsetfillcolor{textcolor}%
\pgftext[x=2.386250in,y=0.251251in,,top]{\color{textcolor}\rmfamily\fontsize{9.000000}{10.800000}\selectfont \(\displaystyle r\) [\(\displaystyle R_\odot\)]}%
\end{pgfscope}%
\begin{pgfscope}%
\pgfpathrectangle{\pgfqpoint{0.726250in}{0.525000in}}{\pgfqpoint{3.320000in}{2.887500in}}%
\pgfusepath{clip}%
\pgfsetbuttcap%
\pgfsetroundjoin%
\pgfsetlinewidth{0.803000pt}%
\definecolor{currentstroke}{rgb}{0.752941,0.752941,0.752941}%
\pgfsetstrokecolor{currentstroke}%
\pgfsetdash{{2.960000pt}{1.280000pt}}{0.000000pt}%
\pgfpathmoveto{\pgfqpoint{0.726250in}{0.525000in}}%
\pgfpathlineto{\pgfqpoint{4.046250in}{0.525000in}}%
\pgfusepath{stroke}%
\end{pgfscope}%
\begin{pgfscope}%
\pgfsetbuttcap%
\pgfsetroundjoin%
\definecolor{currentfill}{rgb}{0.000000,0.000000,0.000000}%
\pgfsetfillcolor{currentfill}%
\pgfsetlinewidth{0.803000pt}%
\definecolor{currentstroke}{rgb}{0.000000,0.000000,0.000000}%
\pgfsetstrokecolor{currentstroke}%
\pgfsetdash{}{0pt}%
\pgfsys@defobject{currentmarker}{\pgfqpoint{-0.048611in}{0.000000in}}{\pgfqpoint{-0.000000in}{0.000000in}}{%
\pgfpathmoveto{\pgfqpoint{-0.000000in}{0.000000in}}%
\pgfpathlineto{\pgfqpoint{-0.048611in}{0.000000in}}%
\pgfusepath{stroke,fill}%
}%
\begin{pgfscope}%
\pgfsys@transformshift{0.726250in}{0.525000in}%
\pgfsys@useobject{currentmarker}{}%
\end{pgfscope}%
\end{pgfscope}%
\begin{pgfscope}%
\definecolor{textcolor}{rgb}{0.000000,0.000000,0.000000}%
\pgfsetstrokecolor{textcolor}%
\pgfsetfillcolor{textcolor}%
\pgftext[x=0.271179in, y=0.477515in, left, base]{\color{textcolor}\rmfamily\fontsize{9.000000}{10.800000}\selectfont 0.000}%
\end{pgfscope}%
\begin{pgfscope}%
\pgfpathrectangle{\pgfqpoint{0.726250in}{0.525000in}}{\pgfqpoint{3.320000in}{2.887500in}}%
\pgfusepath{clip}%
\pgfsetbuttcap%
\pgfsetroundjoin%
\pgfsetlinewidth{0.803000pt}%
\definecolor{currentstroke}{rgb}{0.752941,0.752941,0.752941}%
\pgfsetstrokecolor{currentstroke}%
\pgfsetdash{{2.960000pt}{1.280000pt}}{0.000000pt}%
\pgfpathmoveto{\pgfqpoint{0.726250in}{0.890477in}}%
\pgfpathlineto{\pgfqpoint{4.046250in}{0.890477in}}%
\pgfusepath{stroke}%
\end{pgfscope}%
\begin{pgfscope}%
\pgfsetbuttcap%
\pgfsetroundjoin%
\definecolor{currentfill}{rgb}{0.000000,0.000000,0.000000}%
\pgfsetfillcolor{currentfill}%
\pgfsetlinewidth{0.803000pt}%
\definecolor{currentstroke}{rgb}{0.000000,0.000000,0.000000}%
\pgfsetstrokecolor{currentstroke}%
\pgfsetdash{}{0pt}%
\pgfsys@defobject{currentmarker}{\pgfqpoint{-0.048611in}{0.000000in}}{\pgfqpoint{-0.000000in}{0.000000in}}{%
\pgfpathmoveto{\pgfqpoint{-0.000000in}{0.000000in}}%
\pgfpathlineto{\pgfqpoint{-0.048611in}{0.000000in}}%
\pgfusepath{stroke,fill}%
}%
\begin{pgfscope}%
\pgfsys@transformshift{0.726250in}{0.890477in}%
\pgfsys@useobject{currentmarker}{}%
\end{pgfscope}%
\end{pgfscope}%
\begin{pgfscope}%
\definecolor{textcolor}{rgb}{0.000000,0.000000,0.000000}%
\pgfsetstrokecolor{textcolor}%
\pgfsetfillcolor{textcolor}%
\pgftext[x=0.271179in, y=0.842992in, left, base]{\color{textcolor}\rmfamily\fontsize{9.000000}{10.800000}\selectfont 0.001}%
\end{pgfscope}%
\begin{pgfscope}%
\pgfpathrectangle{\pgfqpoint{0.726250in}{0.525000in}}{\pgfqpoint{3.320000in}{2.887500in}}%
\pgfusepath{clip}%
\pgfsetbuttcap%
\pgfsetroundjoin%
\pgfsetlinewidth{0.803000pt}%
\definecolor{currentstroke}{rgb}{0.752941,0.752941,0.752941}%
\pgfsetstrokecolor{currentstroke}%
\pgfsetdash{{2.960000pt}{1.280000pt}}{0.000000pt}%
\pgfpathmoveto{\pgfqpoint{0.726250in}{1.255954in}}%
\pgfpathlineto{\pgfqpoint{4.046250in}{1.255954in}}%
\pgfusepath{stroke}%
\end{pgfscope}%
\begin{pgfscope}%
\pgfsetbuttcap%
\pgfsetroundjoin%
\definecolor{currentfill}{rgb}{0.000000,0.000000,0.000000}%
\pgfsetfillcolor{currentfill}%
\pgfsetlinewidth{0.803000pt}%
\definecolor{currentstroke}{rgb}{0.000000,0.000000,0.000000}%
\pgfsetstrokecolor{currentstroke}%
\pgfsetdash{}{0pt}%
\pgfsys@defobject{currentmarker}{\pgfqpoint{-0.048611in}{0.000000in}}{\pgfqpoint{-0.000000in}{0.000000in}}{%
\pgfpathmoveto{\pgfqpoint{-0.000000in}{0.000000in}}%
\pgfpathlineto{\pgfqpoint{-0.048611in}{0.000000in}}%
\pgfusepath{stroke,fill}%
}%
\begin{pgfscope}%
\pgfsys@transformshift{0.726250in}{1.255954in}%
\pgfsys@useobject{currentmarker}{}%
\end{pgfscope}%
\end{pgfscope}%
\begin{pgfscope}%
\definecolor{textcolor}{rgb}{0.000000,0.000000,0.000000}%
\pgfsetstrokecolor{textcolor}%
\pgfsetfillcolor{textcolor}%
\pgftext[x=0.271179in, y=1.208469in, left, base]{\color{textcolor}\rmfamily\fontsize{9.000000}{10.800000}\selectfont 0.002}%
\end{pgfscope}%
\begin{pgfscope}%
\pgfpathrectangle{\pgfqpoint{0.726250in}{0.525000in}}{\pgfqpoint{3.320000in}{2.887500in}}%
\pgfusepath{clip}%
\pgfsetbuttcap%
\pgfsetroundjoin%
\pgfsetlinewidth{0.803000pt}%
\definecolor{currentstroke}{rgb}{0.752941,0.752941,0.752941}%
\pgfsetstrokecolor{currentstroke}%
\pgfsetdash{{2.960000pt}{1.280000pt}}{0.000000pt}%
\pgfpathmoveto{\pgfqpoint{0.726250in}{1.621431in}}%
\pgfpathlineto{\pgfqpoint{4.046250in}{1.621431in}}%
\pgfusepath{stroke}%
\end{pgfscope}%
\begin{pgfscope}%
\pgfsetbuttcap%
\pgfsetroundjoin%
\definecolor{currentfill}{rgb}{0.000000,0.000000,0.000000}%
\pgfsetfillcolor{currentfill}%
\pgfsetlinewidth{0.803000pt}%
\definecolor{currentstroke}{rgb}{0.000000,0.000000,0.000000}%
\pgfsetstrokecolor{currentstroke}%
\pgfsetdash{}{0pt}%
\pgfsys@defobject{currentmarker}{\pgfqpoint{-0.048611in}{0.000000in}}{\pgfqpoint{-0.000000in}{0.000000in}}{%
\pgfpathmoveto{\pgfqpoint{-0.000000in}{0.000000in}}%
\pgfpathlineto{\pgfqpoint{-0.048611in}{0.000000in}}%
\pgfusepath{stroke,fill}%
}%
\begin{pgfscope}%
\pgfsys@transformshift{0.726250in}{1.621431in}%
\pgfsys@useobject{currentmarker}{}%
\end{pgfscope}%
\end{pgfscope}%
\begin{pgfscope}%
\definecolor{textcolor}{rgb}{0.000000,0.000000,0.000000}%
\pgfsetstrokecolor{textcolor}%
\pgfsetfillcolor{textcolor}%
\pgftext[x=0.271179in, y=1.573946in, left, base]{\color{textcolor}\rmfamily\fontsize{9.000000}{10.800000}\selectfont 0.003}%
\end{pgfscope}%
\begin{pgfscope}%
\pgfpathrectangle{\pgfqpoint{0.726250in}{0.525000in}}{\pgfqpoint{3.320000in}{2.887500in}}%
\pgfusepath{clip}%
\pgfsetbuttcap%
\pgfsetroundjoin%
\pgfsetlinewidth{0.803000pt}%
\definecolor{currentstroke}{rgb}{0.752941,0.752941,0.752941}%
\pgfsetstrokecolor{currentstroke}%
\pgfsetdash{{2.960000pt}{1.280000pt}}{0.000000pt}%
\pgfpathmoveto{\pgfqpoint{0.726250in}{1.986908in}}%
\pgfpathlineto{\pgfqpoint{4.046250in}{1.986908in}}%
\pgfusepath{stroke}%
\end{pgfscope}%
\begin{pgfscope}%
\pgfsetbuttcap%
\pgfsetroundjoin%
\definecolor{currentfill}{rgb}{0.000000,0.000000,0.000000}%
\pgfsetfillcolor{currentfill}%
\pgfsetlinewidth{0.803000pt}%
\definecolor{currentstroke}{rgb}{0.000000,0.000000,0.000000}%
\pgfsetstrokecolor{currentstroke}%
\pgfsetdash{}{0pt}%
\pgfsys@defobject{currentmarker}{\pgfqpoint{-0.048611in}{0.000000in}}{\pgfqpoint{-0.000000in}{0.000000in}}{%
\pgfpathmoveto{\pgfqpoint{-0.000000in}{0.000000in}}%
\pgfpathlineto{\pgfqpoint{-0.048611in}{0.000000in}}%
\pgfusepath{stroke,fill}%
}%
\begin{pgfscope}%
\pgfsys@transformshift{0.726250in}{1.986908in}%
\pgfsys@useobject{currentmarker}{}%
\end{pgfscope}%
\end{pgfscope}%
\begin{pgfscope}%
\definecolor{textcolor}{rgb}{0.000000,0.000000,0.000000}%
\pgfsetstrokecolor{textcolor}%
\pgfsetfillcolor{textcolor}%
\pgftext[x=0.271179in, y=1.939423in, left, base]{\color{textcolor}\rmfamily\fontsize{9.000000}{10.800000}\selectfont 0.004}%
\end{pgfscope}%
\begin{pgfscope}%
\pgfpathrectangle{\pgfqpoint{0.726250in}{0.525000in}}{\pgfqpoint{3.320000in}{2.887500in}}%
\pgfusepath{clip}%
\pgfsetbuttcap%
\pgfsetroundjoin%
\pgfsetlinewidth{0.803000pt}%
\definecolor{currentstroke}{rgb}{0.752941,0.752941,0.752941}%
\pgfsetstrokecolor{currentstroke}%
\pgfsetdash{{2.960000pt}{1.280000pt}}{0.000000pt}%
\pgfpathmoveto{\pgfqpoint{0.726250in}{2.352385in}}%
\pgfpathlineto{\pgfqpoint{4.046250in}{2.352385in}}%
\pgfusepath{stroke}%
\end{pgfscope}%
\begin{pgfscope}%
\pgfsetbuttcap%
\pgfsetroundjoin%
\definecolor{currentfill}{rgb}{0.000000,0.000000,0.000000}%
\pgfsetfillcolor{currentfill}%
\pgfsetlinewidth{0.803000pt}%
\definecolor{currentstroke}{rgb}{0.000000,0.000000,0.000000}%
\pgfsetstrokecolor{currentstroke}%
\pgfsetdash{}{0pt}%
\pgfsys@defobject{currentmarker}{\pgfqpoint{-0.048611in}{0.000000in}}{\pgfqpoint{-0.000000in}{0.000000in}}{%
\pgfpathmoveto{\pgfqpoint{-0.000000in}{0.000000in}}%
\pgfpathlineto{\pgfqpoint{-0.048611in}{0.000000in}}%
\pgfusepath{stroke,fill}%
}%
\begin{pgfscope}%
\pgfsys@transformshift{0.726250in}{2.352385in}%
\pgfsys@useobject{currentmarker}{}%
\end{pgfscope}%
\end{pgfscope}%
\begin{pgfscope}%
\definecolor{textcolor}{rgb}{0.000000,0.000000,0.000000}%
\pgfsetstrokecolor{textcolor}%
\pgfsetfillcolor{textcolor}%
\pgftext[x=0.271179in, y=2.304900in, left, base]{\color{textcolor}\rmfamily\fontsize{9.000000}{10.800000}\selectfont 0.005}%
\end{pgfscope}%
\begin{pgfscope}%
\pgfpathrectangle{\pgfqpoint{0.726250in}{0.525000in}}{\pgfqpoint{3.320000in}{2.887500in}}%
\pgfusepath{clip}%
\pgfsetbuttcap%
\pgfsetroundjoin%
\pgfsetlinewidth{0.803000pt}%
\definecolor{currentstroke}{rgb}{0.752941,0.752941,0.752941}%
\pgfsetstrokecolor{currentstroke}%
\pgfsetdash{{2.960000pt}{1.280000pt}}{0.000000pt}%
\pgfpathmoveto{\pgfqpoint{0.726250in}{2.717862in}}%
\pgfpathlineto{\pgfqpoint{4.046250in}{2.717862in}}%
\pgfusepath{stroke}%
\end{pgfscope}%
\begin{pgfscope}%
\pgfsetbuttcap%
\pgfsetroundjoin%
\definecolor{currentfill}{rgb}{0.000000,0.000000,0.000000}%
\pgfsetfillcolor{currentfill}%
\pgfsetlinewidth{0.803000pt}%
\definecolor{currentstroke}{rgb}{0.000000,0.000000,0.000000}%
\pgfsetstrokecolor{currentstroke}%
\pgfsetdash{}{0pt}%
\pgfsys@defobject{currentmarker}{\pgfqpoint{-0.048611in}{0.000000in}}{\pgfqpoint{-0.000000in}{0.000000in}}{%
\pgfpathmoveto{\pgfqpoint{-0.000000in}{0.000000in}}%
\pgfpathlineto{\pgfqpoint{-0.048611in}{0.000000in}}%
\pgfusepath{stroke,fill}%
}%
\begin{pgfscope}%
\pgfsys@transformshift{0.726250in}{2.717862in}%
\pgfsys@useobject{currentmarker}{}%
\end{pgfscope}%
\end{pgfscope}%
\begin{pgfscope}%
\definecolor{textcolor}{rgb}{0.000000,0.000000,0.000000}%
\pgfsetstrokecolor{textcolor}%
\pgfsetfillcolor{textcolor}%
\pgftext[x=0.271179in, y=2.670377in, left, base]{\color{textcolor}\rmfamily\fontsize{9.000000}{10.800000}\selectfont 0.006}%
\end{pgfscope}%
\begin{pgfscope}%
\pgfpathrectangle{\pgfqpoint{0.726250in}{0.525000in}}{\pgfqpoint{3.320000in}{2.887500in}}%
\pgfusepath{clip}%
\pgfsetbuttcap%
\pgfsetroundjoin%
\pgfsetlinewidth{0.803000pt}%
\definecolor{currentstroke}{rgb}{0.752941,0.752941,0.752941}%
\pgfsetstrokecolor{currentstroke}%
\pgfsetdash{{2.960000pt}{1.280000pt}}{0.000000pt}%
\pgfpathmoveto{\pgfqpoint{0.726250in}{3.083339in}}%
\pgfpathlineto{\pgfqpoint{4.046250in}{3.083339in}}%
\pgfusepath{stroke}%
\end{pgfscope}%
\begin{pgfscope}%
\pgfsetbuttcap%
\pgfsetroundjoin%
\definecolor{currentfill}{rgb}{0.000000,0.000000,0.000000}%
\pgfsetfillcolor{currentfill}%
\pgfsetlinewidth{0.803000pt}%
\definecolor{currentstroke}{rgb}{0.000000,0.000000,0.000000}%
\pgfsetstrokecolor{currentstroke}%
\pgfsetdash{}{0pt}%
\pgfsys@defobject{currentmarker}{\pgfqpoint{-0.048611in}{0.000000in}}{\pgfqpoint{-0.000000in}{0.000000in}}{%
\pgfpathmoveto{\pgfqpoint{-0.000000in}{0.000000in}}%
\pgfpathlineto{\pgfqpoint{-0.048611in}{0.000000in}}%
\pgfusepath{stroke,fill}%
}%
\begin{pgfscope}%
\pgfsys@transformshift{0.726250in}{3.083339in}%
\pgfsys@useobject{currentmarker}{}%
\end{pgfscope}%
\end{pgfscope}%
\begin{pgfscope}%
\definecolor{textcolor}{rgb}{0.000000,0.000000,0.000000}%
\pgfsetstrokecolor{textcolor}%
\pgfsetfillcolor{textcolor}%
\pgftext[x=0.271179in, y=3.035854in, left, base]{\color{textcolor}\rmfamily\fontsize{9.000000}{10.800000}\selectfont 0.007}%
\end{pgfscope}%
\begin{pgfscope}%
\definecolor{textcolor}{rgb}{0.000000,0.000000,0.000000}%
\pgfsetstrokecolor{textcolor}%
\pgfsetfillcolor{textcolor}%
\pgftext[x=0.215623in,y=1.968750in,,bottom,rotate=90.000000]{\color{textcolor}\rmfamily\fontsize{9.000000}{10.800000}\selectfont Neutrino production [a. u.]}%
\end{pgfscope}%
\begin{pgfscope}%
\pgfpathrectangle{\pgfqpoint{0.726250in}{0.525000in}}{\pgfqpoint{3.320000in}{2.887500in}}%
\pgfusepath{clip}%
\pgfsetrectcap%
\pgfsetroundjoin%
\pgfsetlinewidth{1.003750pt}%
\definecolor{currentstroke}{rgb}{0.000000,0.000000,0.000000}%
\pgfsetstrokecolor{currentstroke}%
\pgfsetdash{}{0pt}%
\pgfpathmoveto{\pgfqpoint{0.726250in}{0.525142in}}%
\pgfpathlineto{\pgfqpoint{0.756579in}{0.528035in}}%
\pgfpathlineto{\pgfqpoint{0.786908in}{0.534841in}}%
\pgfpathlineto{\pgfqpoint{0.817237in}{0.545505in}}%
\pgfpathlineto{\pgfqpoint{0.847566in}{0.559965in}}%
\pgfpathlineto{\pgfqpoint{0.877895in}{0.578236in}}%
\pgfpathlineto{\pgfqpoint{0.908224in}{0.600032in}}%
\pgfpathlineto{\pgfqpoint{0.938553in}{0.625328in}}%
\pgfpathlineto{\pgfqpoint{0.968882in}{0.653973in}}%
\pgfpathlineto{\pgfqpoint{0.999211in}{0.685813in}}%
\pgfpathlineto{\pgfqpoint{1.029540in}{0.720580in}}%
\pgfpathlineto{\pgfqpoint{1.059869in}{0.758150in}}%
\pgfpathlineto{\pgfqpoint{1.090197in}{0.798180in}}%
\pgfpathlineto{\pgfqpoint{1.120526in}{0.840483in}}%
\pgfpathlineto{\pgfqpoint{1.181184in}{0.930761in}}%
\pgfpathlineto{\pgfqpoint{1.241842in}{1.026516in}}%
\pgfpathlineto{\pgfqpoint{1.423816in}{1.317743in}}%
\pgfpathlineto{\pgfqpoint{1.484474in}{1.407446in}}%
\pgfpathlineto{\pgfqpoint{1.514803in}{1.449542in}}%
\pgfpathlineto{\pgfqpoint{1.545132in}{1.489342in}}%
\pgfpathlineto{\pgfqpoint{1.575461in}{1.526435in}}%
\pgfpathlineto{\pgfqpoint{1.605790in}{1.560934in}}%
\pgfpathlineto{\pgfqpoint{1.636119in}{1.592826in}}%
\pgfpathlineto{\pgfqpoint{1.666448in}{1.621393in}}%
\pgfpathlineto{\pgfqpoint{1.696777in}{1.646787in}}%
\pgfpathlineto{\pgfqpoint{1.727106in}{1.669219in}}%
\pgfpathlineto{\pgfqpoint{1.757435in}{1.688188in}}%
\pgfpathlineto{\pgfqpoint{1.787763in}{1.703801in}}%
\pgfpathlineto{\pgfqpoint{1.818092in}{1.716028in}}%
\pgfpathlineto{\pgfqpoint{1.848421in}{1.725052in}}%
\pgfpathlineto{\pgfqpoint{1.878750in}{1.730535in}}%
\pgfpathlineto{\pgfqpoint{1.909079in}{1.732885in}}%
\pgfpathlineto{\pgfqpoint{1.939408in}{1.732233in}}%
\pgfpathlineto{\pgfqpoint{1.969737in}{1.728571in}}%
\pgfpathlineto{\pgfqpoint{2.000066in}{1.721847in}}%
\pgfpathlineto{\pgfqpoint{2.030395in}{1.712507in}}%
\pgfpathlineto{\pgfqpoint{2.060724in}{1.700438in}}%
\pgfpathlineto{\pgfqpoint{2.091053in}{1.685735in}}%
\pgfpathlineto{\pgfqpoint{2.121382in}{1.669214in}}%
\pgfpathlineto{\pgfqpoint{2.151711in}{1.650159in}}%
\pgfpathlineto{\pgfqpoint{2.182040in}{1.629431in}}%
\pgfpathlineto{\pgfqpoint{2.212369in}{1.606783in}}%
\pgfpathlineto{\pgfqpoint{2.273027in}{1.557218in}}%
\pgfpathlineto{\pgfqpoint{2.333685in}{1.502704in}}%
\pgfpathlineto{\pgfqpoint{2.394343in}{1.444527in}}%
\pgfpathlineto{\pgfqpoint{2.485329in}{1.353906in}}%
\pgfpathlineto{\pgfqpoint{2.636974in}{1.202387in}}%
\pgfpathlineto{\pgfqpoint{2.697632in}{1.143728in}}%
\pgfpathlineto{\pgfqpoint{2.758290in}{1.087459in}}%
\pgfpathlineto{\pgfqpoint{2.818948in}{1.034198in}}%
\pgfpathlineto{\pgfqpoint{2.879606in}{0.983627in}}%
\pgfpathlineto{\pgfqpoint{2.940264in}{0.936547in}}%
\pgfpathlineto{\pgfqpoint{3.000922in}{0.892837in}}%
\pgfpathlineto{\pgfqpoint{3.061580in}{0.852272in}}%
\pgfpathlineto{\pgfqpoint{3.122238in}{0.815309in}}%
\pgfpathlineto{\pgfqpoint{3.182895in}{0.781630in}}%
\pgfpathlineto{\pgfqpoint{3.243553in}{0.751117in}}%
\pgfpathlineto{\pgfqpoint{3.304211in}{0.723551in}}%
\pgfpathlineto{\pgfqpoint{3.364869in}{0.698934in}}%
\pgfpathlineto{\pgfqpoint{3.425527in}{0.676906in}}%
\pgfpathlineto{\pgfqpoint{3.486185in}{0.657347in}}%
\pgfpathlineto{\pgfqpoint{3.546843in}{0.640032in}}%
\pgfpathlineto{\pgfqpoint{3.607501in}{0.624694in}}%
\pgfpathlineto{\pgfqpoint{3.668159in}{0.611269in}}%
\pgfpathlineto{\pgfqpoint{3.759146in}{0.594172in}}%
\pgfpathlineto{\pgfqpoint{3.850133in}{0.580226in}}%
\pgfpathlineto{\pgfqpoint{3.941119in}{0.568935in}}%
\pgfpathlineto{\pgfqpoint{4.032106in}{0.559831in}}%
\pgfpathlineto{\pgfqpoint{4.056250in}{0.557753in}}%
\pgfpathlineto{\pgfqpoint{4.056250in}{0.557753in}}%
\pgfusepath{stroke}%
\end{pgfscope}%
\begin{pgfscope}%
\pgfpathrectangle{\pgfqpoint{0.726250in}{0.525000in}}{\pgfqpoint{3.320000in}{2.887500in}}%
\pgfusepath{clip}%
\pgfsetrectcap%
\pgfsetroundjoin%
\pgfsetlinewidth{1.003750pt}%
\definecolor{currentstroke}{rgb}{0.000000,0.000000,0.000000}%
\pgfsetstrokecolor{currentstroke}%
\pgfsetdash{}{0pt}%
\pgfpathmoveto{\pgfqpoint{0.726250in}{0.526733in}}%
\pgfpathlineto{\pgfqpoint{0.756579in}{0.561926in}}%
\pgfpathlineto{\pgfqpoint{0.786908in}{0.643637in}}%
\pgfpathlineto{\pgfqpoint{0.817237in}{0.768608in}}%
\pgfpathlineto{\pgfqpoint{0.847566in}{0.932149in}}%
\pgfpathlineto{\pgfqpoint{0.877895in}{1.129012in}}%
\pgfpathlineto{\pgfqpoint{0.908224in}{1.350701in}}%
\pgfpathlineto{\pgfqpoint{0.968882in}{1.836918in}}%
\pgfpathlineto{\pgfqpoint{1.029540in}{2.324791in}}%
\pgfpathlineto{\pgfqpoint{1.059869in}{2.549109in}}%
\pgfpathlineto{\pgfqpoint{1.090197in}{2.751302in}}%
\pgfpathlineto{\pgfqpoint{1.120526in}{2.925165in}}%
\pgfpathlineto{\pgfqpoint{1.150855in}{3.066528in}}%
\pgfpathlineto{\pgfqpoint{1.181184in}{3.172884in}}%
\pgfpathlineto{\pgfqpoint{1.211513in}{3.242530in}}%
\pgfpathlineto{\pgfqpoint{1.241842in}{3.275000in}}%
\pgfpathlineto{\pgfqpoint{1.272171in}{3.271670in}}%
\pgfpathlineto{\pgfqpoint{1.302500in}{3.234747in}}%
\pgfpathlineto{\pgfqpoint{1.332829in}{3.167501in}}%
\pgfpathlineto{\pgfqpoint{1.363158in}{3.073933in}}%
\pgfpathlineto{\pgfqpoint{1.393487in}{2.957901in}}%
\pgfpathlineto{\pgfqpoint{1.423816in}{2.824153in}}%
\pgfpathlineto{\pgfqpoint{1.454145in}{2.677264in}}%
\pgfpathlineto{\pgfqpoint{1.514803in}{2.359914in}}%
\pgfpathlineto{\pgfqpoint{1.575461in}{2.038189in}}%
\pgfpathlineto{\pgfqpoint{1.605790in}{1.883134in}}%
\pgfpathlineto{\pgfqpoint{1.636119in}{1.734878in}}%
\pgfpathlineto{\pgfqpoint{1.666448in}{1.595139in}}%
\pgfpathlineto{\pgfqpoint{1.696777in}{1.465043in}}%
\pgfpathlineto{\pgfqpoint{1.727106in}{1.345292in}}%
\pgfpathlineto{\pgfqpoint{1.757435in}{1.236538in}}%
\pgfpathlineto{\pgfqpoint{1.787763in}{1.137696in}}%
\pgfpathlineto{\pgfqpoint{1.818092in}{1.049928in}}%
\pgfpathlineto{\pgfqpoint{1.848421in}{0.972310in}}%
\pgfpathlineto{\pgfqpoint{1.878750in}{0.903962in}}%
\pgfpathlineto{\pgfqpoint{1.909079in}{0.844416in}}%
\pgfpathlineto{\pgfqpoint{1.939408in}{0.792880in}}%
\pgfpathlineto{\pgfqpoint{1.969737in}{0.748550in}}%
\pgfpathlineto{\pgfqpoint{2.000066in}{0.710687in}}%
\pgfpathlineto{\pgfqpoint{2.030395in}{0.678554in}}%
\pgfpathlineto{\pgfqpoint{2.060724in}{0.651413in}}%
\pgfpathlineto{\pgfqpoint{2.091053in}{0.628550in}}%
\pgfpathlineto{\pgfqpoint{2.121382in}{0.609543in}}%
\pgfpathlineto{\pgfqpoint{2.151711in}{0.593778in}}%
\pgfpathlineto{\pgfqpoint{2.182040in}{0.580730in}}%
\pgfpathlineto{\pgfqpoint{2.212369in}{0.569983in}}%
\pgfpathlineto{\pgfqpoint{2.242698in}{0.561202in}}%
\pgfpathlineto{\pgfqpoint{2.273027in}{0.554030in}}%
\pgfpathlineto{\pgfqpoint{2.333685in}{0.543490in}}%
\pgfpathlineto{\pgfqpoint{2.394343in}{0.536623in}}%
\pgfpathlineto{\pgfqpoint{2.455001in}{0.532234in}}%
\pgfpathlineto{\pgfqpoint{2.545987in}{0.528484in}}%
\pgfpathlineto{\pgfqpoint{2.667303in}{0.526277in}}%
\pgfpathlineto{\pgfqpoint{2.879606in}{0.525207in}}%
\pgfpathlineto{\pgfqpoint{3.698488in}{0.525000in}}%
\pgfpathlineto{\pgfqpoint{4.056250in}{0.525000in}}%
\pgfpathlineto{\pgfqpoint{4.056250in}{0.525000in}}%
\pgfusepath{stroke}%
\end{pgfscope}%
\begin{pgfscope}%
\pgfpathrectangle{\pgfqpoint{0.726250in}{0.525000in}}{\pgfqpoint{3.320000in}{2.887500in}}%
\pgfusepath{clip}%
\pgfsetrectcap%
\pgfsetroundjoin%
\pgfsetlinewidth{1.003750pt}%
\definecolor{currentstroke}{rgb}{0.000000,0.000000,0.000000}%
\pgfsetstrokecolor{currentstroke}%
\pgfsetdash{}{0pt}%
\pgfpathmoveto{\pgfqpoint{0.726250in}{0.525698in}}%
\pgfpathlineto{\pgfqpoint{0.756579in}{0.539908in}}%
\pgfpathlineto{\pgfqpoint{0.786908in}{0.573152in}}%
\pgfpathlineto{\pgfqpoint{0.817237in}{0.624575in}}%
\pgfpathlineto{\pgfqpoint{0.847566in}{0.693198in}}%
\pgfpathlineto{\pgfqpoint{0.877895in}{0.777916in}}%
\pgfpathlineto{\pgfqpoint{0.908224in}{0.876251in}}%
\pgfpathlineto{\pgfqpoint{0.938553in}{0.986692in}}%
\pgfpathlineto{\pgfqpoint{0.968882in}{1.106588in}}%
\pgfpathlineto{\pgfqpoint{1.029540in}{1.365323in}}%
\pgfpathlineto{\pgfqpoint{1.120526in}{1.762484in}}%
\pgfpathlineto{\pgfqpoint{1.150855in}{1.887369in}}%
\pgfpathlineto{\pgfqpoint{1.181184in}{2.004875in}}%
\pgfpathlineto{\pgfqpoint{1.211513in}{2.112355in}}%
\pgfpathlineto{\pgfqpoint{1.241842in}{2.208872in}}%
\pgfpathlineto{\pgfqpoint{1.272171in}{2.292451in}}%
\pgfpathlineto{\pgfqpoint{1.302500in}{2.362799in}}%
\pgfpathlineto{\pgfqpoint{1.332829in}{2.418189in}}%
\pgfpathlineto{\pgfqpoint{1.363158in}{2.459853in}}%
\pgfpathlineto{\pgfqpoint{1.393487in}{2.486469in}}%
\pgfpathlineto{\pgfqpoint{1.423816in}{2.499079in}}%
\pgfpathlineto{\pgfqpoint{1.454145in}{2.497617in}}%
\pgfpathlineto{\pgfqpoint{1.484474in}{2.483131in}}%
\pgfpathlineto{\pgfqpoint{1.514803in}{2.456494in}}%
\pgfpathlineto{\pgfqpoint{1.545132in}{2.418898in}}%
\pgfpathlineto{\pgfqpoint{1.575461in}{2.371484in}}%
\pgfpathlineto{\pgfqpoint{1.605790in}{2.315672in}}%
\pgfpathlineto{\pgfqpoint{1.636119in}{2.252433in}}%
\pgfpathlineto{\pgfqpoint{1.666448in}{2.183389in}}%
\pgfpathlineto{\pgfqpoint{1.696777in}{2.109343in}}%
\pgfpathlineto{\pgfqpoint{1.757435in}{1.952455in}}%
\pgfpathlineto{\pgfqpoint{1.878750in}{1.629235in}}%
\pgfpathlineto{\pgfqpoint{1.939408in}{1.476272in}}%
\pgfpathlineto{\pgfqpoint{1.969737in}{1.403545in}}%
\pgfpathlineto{\pgfqpoint{2.000066in}{1.334242in}}%
\pgfpathlineto{\pgfqpoint{2.030395in}{1.268019in}}%
\pgfpathlineto{\pgfqpoint{2.060724in}{1.205485in}}%
\pgfpathlineto{\pgfqpoint{2.091053in}{1.146544in}}%
\pgfpathlineto{\pgfqpoint{2.121382in}{1.091362in}}%
\pgfpathlineto{\pgfqpoint{2.151711in}{1.039757in}}%
\pgfpathlineto{\pgfqpoint{2.182040in}{0.991803in}}%
\pgfpathlineto{\pgfqpoint{2.212369in}{0.947376in}}%
\pgfpathlineto{\pgfqpoint{2.242698in}{0.906369in}}%
\pgfpathlineto{\pgfqpoint{2.273027in}{0.868846in}}%
\pgfpathlineto{\pgfqpoint{2.303356in}{0.834253in}}%
\pgfpathlineto{\pgfqpoint{2.333685in}{0.802663in}}%
\pgfpathlineto{\pgfqpoint{2.364014in}{0.773853in}}%
\pgfpathlineto{\pgfqpoint{2.394343in}{0.747569in}}%
\pgfpathlineto{\pgfqpoint{2.424672in}{0.723785in}}%
\pgfpathlineto{\pgfqpoint{2.455001in}{0.702287in}}%
\pgfpathlineto{\pgfqpoint{2.485329in}{0.682889in}}%
\pgfpathlineto{\pgfqpoint{2.515658in}{0.665389in}}%
\pgfpathlineto{\pgfqpoint{2.545987in}{0.649678in}}%
\pgfpathlineto{\pgfqpoint{2.576316in}{0.635582in}}%
\pgfpathlineto{\pgfqpoint{2.606645in}{0.622938in}}%
\pgfpathlineto{\pgfqpoint{2.636974in}{0.611679in}}%
\pgfpathlineto{\pgfqpoint{2.697632in}{0.592599in}}%
\pgfpathlineto{\pgfqpoint{2.758290in}{0.577533in}}%
\pgfpathlineto{\pgfqpoint{2.818948in}{0.565695in}}%
\pgfpathlineto{\pgfqpoint{2.879606in}{0.556410in}}%
\pgfpathlineto{\pgfqpoint{2.940264in}{0.549166in}}%
\pgfpathlineto{\pgfqpoint{3.031251in}{0.541213in}}%
\pgfpathlineto{\pgfqpoint{3.122238in}{0.535818in}}%
\pgfpathlineto{\pgfqpoint{3.243553in}{0.531257in}}%
\pgfpathlineto{\pgfqpoint{3.395198in}{0.528124in}}%
\pgfpathlineto{\pgfqpoint{3.607501in}{0.526169in}}%
\pgfpathlineto{\pgfqpoint{4.032106in}{0.525132in}}%
\pgfpathlineto{\pgfqpoint{4.056250in}{0.525113in}}%
\pgfpathlineto{\pgfqpoint{4.056250in}{0.525113in}}%
\pgfusepath{stroke}%
\end{pgfscope}%
\begin{pgfscope}%
\pgfpathrectangle{\pgfqpoint{0.726250in}{0.525000in}}{\pgfqpoint{3.320000in}{2.887500in}}%
\pgfusepath{clip}%
\pgfsetrectcap%
\pgfsetroundjoin%
\pgfsetlinewidth{1.003750pt}%
\definecolor{currentstroke}{rgb}{0.000000,0.000000,0.000000}%
\pgfsetstrokecolor{currentstroke}%
\pgfsetdash{}{0pt}%
\pgfpathmoveto{\pgfqpoint{0.726250in}{0.525222in}}%
\pgfpathlineto{\pgfqpoint{0.756579in}{0.529743in}}%
\pgfpathlineto{\pgfqpoint{0.786908in}{0.540360in}}%
\pgfpathlineto{\pgfqpoint{0.817237in}{0.556956in}}%
\pgfpathlineto{\pgfqpoint{0.847566in}{0.579379in}}%
\pgfpathlineto{\pgfqpoint{0.877895in}{0.607567in}}%
\pgfpathlineto{\pgfqpoint{0.908224in}{0.640963in}}%
\pgfpathlineto{\pgfqpoint{0.938553in}{0.679479in}}%
\pgfpathlineto{\pgfqpoint{0.968882in}{0.722717in}}%
\pgfpathlineto{\pgfqpoint{0.999211in}{0.770213in}}%
\pgfpathlineto{\pgfqpoint{1.029540in}{0.821655in}}%
\pgfpathlineto{\pgfqpoint{1.059869in}{0.876467in}}%
\pgfpathlineto{\pgfqpoint{1.090197in}{0.934115in}}%
\pgfpathlineto{\pgfqpoint{1.150855in}{1.056036in}}%
\pgfpathlineto{\pgfqpoint{1.302500in}{1.371801in}}%
\pgfpathlineto{\pgfqpoint{1.363158in}{1.490508in}}%
\pgfpathlineto{\pgfqpoint{1.393487in}{1.545974in}}%
\pgfpathlineto{\pgfqpoint{1.423816in}{1.598672in}}%
\pgfpathlineto{\pgfqpoint{1.454145in}{1.647589in}}%
\pgfpathlineto{\pgfqpoint{1.484474in}{1.692884in}}%
\pgfpathlineto{\pgfqpoint{1.514803in}{1.734189in}}%
\pgfpathlineto{\pgfqpoint{1.545132in}{1.770922in}}%
\pgfpathlineto{\pgfqpoint{1.575461in}{1.803080in}}%
\pgfpathlineto{\pgfqpoint{1.605790in}{1.830297in}}%
\pgfpathlineto{\pgfqpoint{1.636119in}{1.852964in}}%
\pgfpathlineto{\pgfqpoint{1.666448in}{1.871172in}}%
\pgfpathlineto{\pgfqpoint{1.696777in}{1.883933in}}%
\pgfpathlineto{\pgfqpoint{1.727106in}{1.892226in}}%
\pgfpathlineto{\pgfqpoint{1.757435in}{1.895920in}}%
\pgfpathlineto{\pgfqpoint{1.787763in}{1.895058in}}%
\pgfpathlineto{\pgfqpoint{1.818092in}{1.890061in}}%
\pgfpathlineto{\pgfqpoint{1.848421in}{1.880778in}}%
\pgfpathlineto{\pgfqpoint{1.878750in}{1.867559in}}%
\pgfpathlineto{\pgfqpoint{1.909079in}{1.850756in}}%
\pgfpathlineto{\pgfqpoint{1.939408in}{1.830411in}}%
\pgfpathlineto{\pgfqpoint{1.969737in}{1.806963in}}%
\pgfpathlineto{\pgfqpoint{2.000066in}{1.780885in}}%
\pgfpathlineto{\pgfqpoint{2.030395in}{1.752167in}}%
\pgfpathlineto{\pgfqpoint{2.060724in}{1.721177in}}%
\pgfpathlineto{\pgfqpoint{2.091053in}{1.688137in}}%
\pgfpathlineto{\pgfqpoint{2.151711in}{1.617215in}}%
\pgfpathlineto{\pgfqpoint{2.212369in}{1.541937in}}%
\pgfpathlineto{\pgfqpoint{2.364014in}{1.346565in}}%
\pgfpathlineto{\pgfqpoint{2.424672in}{1.269488in}}%
\pgfpathlineto{\pgfqpoint{2.485329in}{1.194924in}}%
\pgfpathlineto{\pgfqpoint{2.545987in}{1.124124in}}%
\pgfpathlineto{\pgfqpoint{2.606645in}{1.057275in}}%
\pgfpathlineto{\pgfqpoint{2.667303in}{0.995336in}}%
\pgfpathlineto{\pgfqpoint{2.697632in}{0.965719in}}%
\pgfpathlineto{\pgfqpoint{2.758290in}{0.910953in}}%
\pgfpathlineto{\pgfqpoint{2.818948in}{0.861345in}}%
\pgfpathlineto{\pgfqpoint{2.879606in}{0.816575in}}%
\pgfpathlineto{\pgfqpoint{2.940264in}{0.776455in}}%
\pgfpathlineto{\pgfqpoint{3.000922in}{0.740843in}}%
\pgfpathlineto{\pgfqpoint{3.061580in}{0.709486in}}%
\pgfpathlineto{\pgfqpoint{3.122238in}{0.682038in}}%
\pgfpathlineto{\pgfqpoint{3.182895in}{0.658134in}}%
\pgfpathlineto{\pgfqpoint{3.243553in}{0.637418in}}%
\pgfpathlineto{\pgfqpoint{3.304211in}{0.619559in}}%
\pgfpathlineto{\pgfqpoint{3.364869in}{0.604270in}}%
\pgfpathlineto{\pgfqpoint{3.425527in}{0.591225in}}%
\pgfpathlineto{\pgfqpoint{3.486185in}{0.580174in}}%
\pgfpathlineto{\pgfqpoint{3.546843in}{0.570834in}}%
\pgfpathlineto{\pgfqpoint{3.637830in}{0.559501in}}%
\pgfpathlineto{\pgfqpoint{3.728817in}{0.550831in}}%
\pgfpathlineto{\pgfqpoint{3.819804in}{0.544238in}}%
\pgfpathlineto{\pgfqpoint{3.941119in}{0.537883in}}%
\pgfpathlineto{\pgfqpoint{4.056250in}{0.533761in}}%
\pgfpathlineto{\pgfqpoint{4.056250in}{0.533761in}}%
\pgfusepath{stroke}%
\end{pgfscope}%
\begin{pgfscope}%
\pgfsetrectcap%
\pgfsetmiterjoin%
\pgfsetlinewidth{1.003750pt}%
\definecolor{currentstroke}{rgb}{0.000000,0.000000,0.000000}%
\pgfsetstrokecolor{currentstroke}%
\pgfsetdash{}{0pt}%
\pgfpathmoveto{\pgfqpoint{0.726250in}{0.525000in}}%
\pgfpathlineto{\pgfqpoint{0.726250in}{3.412500in}}%
\pgfusepath{stroke}%
\end{pgfscope}%
\begin{pgfscope}%
\pgfsetrectcap%
\pgfsetmiterjoin%
\pgfsetlinewidth{1.003750pt}%
\definecolor{currentstroke}{rgb}{0.000000,0.000000,0.000000}%
\pgfsetstrokecolor{currentstroke}%
\pgfsetdash{}{0pt}%
\pgfpathmoveto{\pgfqpoint{4.046250in}{0.525000in}}%
\pgfpathlineto{\pgfqpoint{4.046250in}{3.412500in}}%
\pgfusepath{stroke}%
\end{pgfscope}%
\begin{pgfscope}%
\pgfsetrectcap%
\pgfsetmiterjoin%
\pgfsetlinewidth{1.003750pt}%
\definecolor{currentstroke}{rgb}{0.000000,0.000000,0.000000}%
\pgfsetstrokecolor{currentstroke}%
\pgfsetdash{}{0pt}%
\pgfpathmoveto{\pgfqpoint{0.726250in}{0.525000in}}%
\pgfpathlineto{\pgfqpoint{4.046250in}{0.525000in}}%
\pgfusepath{stroke}%
\end{pgfscope}%
\begin{pgfscope}%
\pgfsetrectcap%
\pgfsetmiterjoin%
\pgfsetlinewidth{1.003750pt}%
\definecolor{currentstroke}{rgb}{0.000000,0.000000,0.000000}%
\pgfsetstrokecolor{currentstroke}%
\pgfsetdash{}{0pt}%
\pgfpathmoveto{\pgfqpoint{0.726250in}{3.412500in}}%
\pgfpathlineto{\pgfqpoint{4.046250in}{3.412500in}}%
\pgfusepath{stroke}%
\end{pgfscope}%
\begin{pgfscope}%
\definecolor{textcolor}{rgb}{0.000000,0.000000,0.000000}%
\pgfsetstrokecolor{textcolor}%
\pgfsetfillcolor{textcolor}%
\pgftext[x=2.293091in,y=1.621431in,left,base]{\color{textcolor}\rmfamily\fontsize{9.000000}{10.800000}\selectfont \textit{pp}}%
\end{pgfscope}%
\begin{pgfscope}%
\definecolor{textcolor}{rgb}{0.000000,0.000000,0.000000}%
\pgfsetstrokecolor{textcolor}%
\pgfsetfillcolor{textcolor}%
\pgftext[x=1.386284in,y=3.083339in,left,base]{\color{textcolor}\rmfamily\fontsize{9.000000}{10.800000}\selectfont \(\displaystyle ^8B\)}%
\end{pgfscope}%
\begin{pgfscope}%
\definecolor{textcolor}{rgb}{0.000000,0.000000,0.000000}%
\pgfsetstrokecolor{textcolor}%
\pgfsetfillcolor{textcolor}%
\pgftext[x=1.628099in,y=2.352385in,left,base]{\color{textcolor}\rmfamily\fontsize{9.000000}{10.800000}\selectfont \(\displaystyle ^7Be\)}%
\end{pgfscope}%
\begin{pgfscope}%
\definecolor{textcolor}{rgb}{0.000000,0.000000,0.000000}%
\pgfsetstrokecolor{textcolor}%
\pgfsetfillcolor{textcolor}%
\pgftext[x=1.930368in,y=1.913813in,left,base]{\color{textcolor}\rmfamily\fontsize{9.000000}{10.800000}\selectfont \textit{pep}}%
\end{pgfscope}%
\end{pgfpicture}%
\makeatother%
\endgroup%

  \caption{Radial profiles of the neutrino production for the most important
  components of the pp cycle. (All the curves are intended as probability density
  functions, i.e., they are normalized to unity). In general terms, higher energy
  neutrinos are produced closer to the center of the Sun, where the temperature
  is higher.}
  \label{fig:sun_neutrino_location}
\end{figure}



\subsection{Solar neutrino experiments}

The fact that the Sun was an ideal source to study neutrinos was widely accepted in
the 1960s, although accompanyied by a distinct skeptikism that the knowledge of the
underlying physical processes and the environmental conditions in the Sun would
ever be known to the precision necessary for a meaningful comparison with the theory.

The history of the measurement of Solar neutrino is a long and difficult one, and
begins with the Brookhaven Solar Neutrino Experiment, performed in the Homestake
gold mine (South Dakota) in the late 1960s, under the lead of Raymond Davis and
John Bahcall. The experimental setup was based on 615~ton of perchloroethylene
($\ce{C_2Cl_4}$), placed $\sim 1500$~m underground to mitigate the large cosmic-ray
background, and the basic reaction exploited was
\begin{align*}
  \nu_e + \ce{^37Cl} \rightarrow \ce{^37Ar} + e^-,
\end{align*}
with a kinematic threshold of 814~keV on the neutrino energy. This is a particular
case of the radiochemical detection method\sidenote{The fact that the inverse beta
decay could be used to detect neutrinos via the radiactive decay of the dauther nucleus
was originally proposed by B.~Pontecorvo in 1946~\cite{1946PontecorvoInvBeta}.
Interestingly enough, the original paper lists all the necessary characteristic of
the active medium to be used: it has to be cheap, due to the large volumes involved,
the lifetime of the radiactive nuclide produced in the process must be long enough,
the separation of the radiactive atoms from the bulk of the medium must be simplem
and the threshold for the reaction must be as low as possible (clorine was identified
as a prime candidate).}, that relies on nuclear neutrino capture
(or \emph{inverse $\beta$ decay})
\begin{align*}
  \nu_e + (A,~Z) \xrightarrow[\text{decay}]{\text{capture}} (A,~Z+1) + e^-,
\end{align*}
where the product nuclide then experiments $\beta$ decay back to the original one
with some characteristic half-life $T_\frac{1}{2}$ (about 35~days for $\ce{^37Ar}$).
Radiochemical experiments are peculiar in that they are run in \emph{batch} mode
(as opposed to \emph{real time} mode): the target is exposed to neutrinos for periods
of time of the order of $T_\frac{1}{2}$, and then the products of neutrino captures
are separated and removed through chemical methods (e.g., bubbling He in the case
of Homestake), and need to be converted in suitable gaseous form for the subsequent
$\beta$ decays to be measured by proportional counters in low-radioactivity environments.
The challenges connected with the choice of the target nucleus---beyond the fact
for it to be amenable to inverse $\beta$ decay---are availability in large quantities
(and with high enough purity) as well as the need for practical threshold and
half-life values.

In the case of Homestake, with about $10^{31}$ target Cl atoms, the expected rate
was of the order of a few tens of neutrino interactions per months---that is, a
few tens of radioactive $\ce{^37Ar}$ atoms that needed to be removed every month
to measure their $\beta$ decay, which gives an idea of the extraordinary difficulty
of the experiment. The first result of the experiment~\cite{1968PhRvL..20.1205D}
was an upper limit that was a factor $\sim 2.5$ below the prediction of the standard
solar model described in the previous section, while additional data~\cite{1998ApJ...496..505C}
provided a measured flux of $2.56 \pm 0.16~(\text{stat}) \pm 0.16~(\text{sys})$~SNU\sidenote{The solar neutrino unit (SNU) is conventionally
defined as the neutrino flux producing $10^{-36}$ captures per target atom per second
in a given experimental setup.}, to be compared with a prediction of $7.8$~SNU (with
a relatively large uncertainty, of the order of 20\%), that is, a deficit of a factor
of $\sim 3$. This constitutes the beginning of the
so called \emph{solar neutrino problem}, although in the early days, given the relatively
large uncertainities on both the measurements and the theoretical predictions, the
concern of the actors involved was quite limited.

The next chapter in the saga coincides with the gallium experiments: Gallex, running
between 1991 and 1997 in Italy, and SAGE (1989--2011) in Russia. These are radiochemical
experiments expoiting the reaction
\begin{align*}
  \nu_e + \ce{^71Ga} \rightarrow \ce{^71Ge} + e^-,
\end{align*}
which provides an exceptionally low kinematic threshold of 233.2~keV. Most notably,
such a low threshold allows access to the $pp$ neutrinos, constituting the vast
majority of the solar neutrino flux, and guarantees much higher rates of neutrino
interactions. The gallium experiments qualitatively confirmed the deficit measured
by Homestake, altough the ratio between the measured rate (77.5~SNU) and the
expected one (128~SNU) was of the order of $\sim 0.6$, almost twice the value measured
with clorine\cite{1999PhLB..447..127H}.\todo{Expand and put some reference.}

The next step is the advent of the water \cherenkov\ experiments, championed by
Kamiokande and Super-Kamiokande, an deep-underground tank containing 50~kton of
pure water equipped with $11,200$ photomultiplier tubes, operated in Japan between
1996 and 2007. The detection principle is interesting: as the nuclear neutrino
capture is suppressed in oxygen, the basic interaction mechanism for neutrinos is
the elastic scattering on the atomic electrons. This is qualitatively different from
the radiochemical experiments\sidenote{Historically, another notable difference is
that water \cherenkov\ experiments were the first to detect neutrino in real time.}
in that, while the latter are only sensitive to $\nu_e$, \emph{water \cherenkov\
experiments are sensity to all three generations of neutrinos} via the reactions:
\begin{align*}
  \nu_{e, \mu, \tau} + e^- \rightarrow  \nu_{e, \mu, \tau} + e^-
\end{align*}
(provided, of course, that the neutrino flavor is conserved at the vertex).
This is not to say that the three flavors of neutrinos all interact \emph{in the same way}:
there is a fundamental asymmetry due to the fact that ordinary matters includes
electrons but not $\mu$ and $\tau$. As a consequence, while the elastic scattering
for $\nu_e$ proceeds both with through neutral current (NC) and charged current (CC)
interactions (i.e., with the exchange of a $Z^0$ or a $W$) for $\nu_\mu$ and
$\nu_\tau$ only NC interactions are allowed---at least up to the kinematic threshold
where inelastic interactions with the production of $\mu$ (or $\tau$) leptons.
\todo{Add the relevant Feynman diagrams.}
It follows that the elastic scattering cross section on electrons is about six times
larger for electrons than for muons and tauons
\begin{align*}
  \frac{\sigma_e(\nu_e)}{\sigma_e(\nu_\mu)} \approx
  \frac{\sigma_e(\nu_e)}{\sigma_e(\nu_\tau)} \approx 6
\end{align*}
and, as we shall see in a second, this will be crucial when interpreting the results.

From an experimental point of view, although the kinetic energy threshold for an
electron to emit \cherenkov\ light in water is only $\sim 260$~keV, one needs enough
light to be able to see a signal over the electronic noise, and the actual detection
threshold for Super-Kamiokande is significantly higher---of the order of 5~MeV. It
follows that we are basically only sensitive to the $\ce{^8B}$ neutrinos, as in the
case of the clorine experiments. The range of a few-MeV electron in water is measured
in~cm (that is, much smaller than the size of the tank), which means that the \cherenkov\
light manifests itself in the form of rings, as opposed to full circles. From the
photomultiplier tubes, in addition to the physical location, we have access to the
pulse-height and time information, which allows to recover the arrival time, energy
and direction of the electrons. Since the neutrino energy $E \gg m_e c^2$, the electron
is typically scattered in the forward direction, wich means that not only we are
measuring neutrino in real time, but we can also project them back in the sky, and
confirm directly their solar origin\sidenote{This is not only worth mentioning as the
starting point of the neutrino astronomy, but it is also relevant for the background
rejection.}. In addition, the topological characteristics of the signal pattern in
the photomultiplier tubes allows for some level of discrimination between electrons
(which radiate, and tend to produce more irregular patterns) and background muons,
producing more regular circles.\todo{We might want to add a plot of the SK solar
peak?} The Super-Kamiokande measurement of the flux of $\ce{^8B}$ neutrinos, amounting
to some $0.45$ of the SSM prediction, constitutes a turning point in the solar neutrino
saga.

\begin{table}[!htbp]
  \begin{tabular}{llll}
    \hline
    Experiment & Threshold & Measured flux & Flux deficit\\
    \hline
    \hline
    Homestake & $814$~keV & $\Phi(\nu_e)$ & $0.33 \pm 0.02$~\cite{1998ApJ...496..505C}\\
    Gallex & $233.2$~keV & $\Phi(\nu_e)$ & $0.60 \pm 0.06$~\cite{1999PhLB..447..127H}\\
    SAGE & $233.2$~keV & $\Phi(\nu_e)$ & 0.$52 \pm 0.05$~\cite{1999PhRvL..83.4686A}\\
    SK & $\sim 5$~MeV & $\Phi(\nu_e)$ + \nicefrac{1}{6} $\Phi(\nu_\mu,\nu_\tau)$ &
    $0.451 \pm 0.002$~\cite{2001PhRvL..86.5656F} \\
    \hline
  \end{tabular}
  \caption{Partial compilation of the relevant solar neutrino measurements available
  at the end of the 1990s. (Note that the error bars are provided to allow for a
  meaningful comparison between different experiments, but are largely inhomogeneous,
  as the neutrino deficits are not necessarily referred to the same solar model,
  and the systematic uncertainties on the latter are not properly included.)}
  \label{tab:solar_nu_results}
\end{table}

All in all it is fair to say that, at the end of the 1990s the general solar neutrino
deficit, compared with the prediction of the standard solar model, was well established
from an experimental standpoint. There was some evidence that the effect was
energy-dependent, with the Gallium experiments (sensitive to the $pp$ component)
measuring a significant higher flux than the other, sensitive to relatively high
energy. We shall also note that, is we assume a $\nu_{\mu,\tau}$ flux comparable
of that of $\nu_e$, the SK results is not incompatible with Homestake. In the next
section we shall summarize the two pieces of information that allow to compose the
puzzle: neutrino oscillation in vacuum and the MSW effect.

%In our regime, where the typical energies are much smaller than the masses of the
%vector bosons, the $Z^0$ and $W$ propagators simplify and the weak force can be
%treated as a point interaction (a la Fermi) via the introduction of an effective
%coupling constant
%\begin{align*}
%  G_F = \frac{g_w^2}{M_W^2}
%\end{align*}

% https://www.slideserve.com/jalia/solar-neutrino-physics
% Slide 3, status of the field:
% Cl: 2.56 +- 0.23 SNU, Apj 496 (1998) 505
% Ga: 66.1 +- 3.1 SNU, PRC80 0150807 (2009)
% SK 1496 days, > 5 MeV
% 8B flux: 2.35 +- 0.02 +- 0.08 e6 cm^{-2} s^{-1}
% See plot at slide 6
% Solar experiments all consistent with SSM + 2-flavor MSW oscillations


\subsection{Two-flavor neutrino oscillation in vacuum}

The possibility that, \emph{if the neutrino mass is non zero}, neutrinos could mix
across different generations and oscillate from one flavor to another, dates back to a
seminal paper by B. Pontecorvo in 1957---the idea stemming from the analogy with
mixing in the systems of other neutral particles, such as the Kaons. The fact that
neutrino oscillations might resolve the solar neutrino problem was considered early
on in the course of the events, but the implications were so far reaching that some
sort of conclusive evidence was necessary for people to accept that this radically
new paradigm was not just a mere speculation. In this section we briefly review the
mechanism for neutrino oscillation in vacuum with two flavors\sidenote{The formalism
for the description of neutrino oscillation is much easier to develop in the context
of two-flavour mixing rather than three-flavour mixing. In addition, nature is such
that some of the keyobservations (solar and atmospheric neutrinos) can in fact be
interpreted, to a good approximation, in terms of two flavour mixing---therefore the
relevance of this discussion is not merely academic.}.

The basic idea behind neutrino oscillation is that neutrinos are created and absorbed
via the weak nuclear force as \emph{flavor} eigenstates, but then propagate, like all
other particles, as \emph{mass} eigenstate---and in a quantum field theory there is
nothing that guarantees that the two components of the lagrangian (that for the mass and that
for the interaction) can be made simultaneously diagonal. More specifically, if
$\nu_e$ and $\nu_\mu$ are the flavor eigenstates\sidenote{We are calling the two
flavor eigenstates $\nu_e$ and $\nu_\mu$ in order not to multiply the notation, but
there is absolutely nothing special to prefer these two generations to the fellow
$\nu_\tau$. The main point is that at each vertex of a weak Feynman diagram, each
neutrino has a well defined flavor (and not a defined mass).} and $\nu_1$ and $\nu_2$
are the mass eigenstate, we shall assume that a unitary tranformation exists connecting
the two representations
\begin{align*}
  \begin{pmatrix}\nu_e\\ \nu_\mu\end{pmatrix} = U
  \begin{pmatrix}\nu_1\\ \nu_2 \end{pmatrix}
  \quad\text{and}\quad
  \begin{pmatrix}\nu_1\\ \nu_2 \end{pmatrix} = U^{-1}
  \begin{pmatrix}\nu_e\\ \nu_\mu\end{pmatrix}.
\end{align*}
In two dimensions this unitary transformation is a rotation, and can be parametrized
by means of a single \emph{mixing angle}\sidenote{As we shall see, in three dimensions
the situation is more complicated, and we do have three angles and a phase, just like
in the CKM matrix.}
\begin{align*}
  U =
  \begin{pmatrix}
    \cos\theta & \sin\theta \\
    -\sin\theta & \cos\theta \\
  \end{pmatrix}
  \quad\text{and}\quad
  U^{-1} =
  \begin{pmatrix}
    \cos\theta & -\sin\theta \\
    \sin\theta & \cos\theta \\
  \end{pmatrix}.
\end{align*}
We note that for $\theta = 0^\circ$ (or $\theta = 90^\circ$, for what matters) the
flavor and mass eigenstates coincide, i.e., there is no mixing, while the latter
is maximal for $\theta = 45^\circ$, when each flavor eigenstate is an equal mixture
of the two mass eigestates and vice-versa.

If we start with a flavour eigenstate (say $\nu_e$) with a given momentum $p$ at $t=0$
\begin{align*}
  \ket{\nu(0)} = \ket{\nu_e} = \cos\theta \ket{\nu_1} + \sin\theta \ket{\nu_2},
\end{align*}
the two mass eigenstates will evolve in time each with its energy-dependent phase
\begin{align*}
  \ket{\nu(t)} =
    \exp\qty{-i\frac{E_1t}{\hbar}} \cos\theta \ket{\nu_1} +
    \exp\qty{-i\frac{E_2t}{\hbar}} \sin\theta \ket{\nu_2}.
\end{align*}
We can calculate the probability $P(\nu_e \rightarrow \nu_\mu)$ to detect our
neutrino ($\nu_e$ at $t = 0$) as $\nu_\mu$ at a generic time $t$:
\begin{align*}
  \abs{\braket{\nu_\mu}{\nu(t)}}^2 & = \sin^2\theta \cos^2\theta \;
    \abs{\,\exp\qty{-i\frac{E_2t}{\hbar}} - \exp\qty{-i\frac{E_1t}{\hbar}}}^2 = \\
  & = 2 \sin^2\theta \cos^2\theta \qty[1 - \cos\qty(\frac{(E_2 - E_1)t}{\hbar})] = \\
  & = \sin^2(2\theta)\sin^2 \qty(\frac{(E_2 - E_1)t}{2 \hbar}).
\end{align*}
Since neutrinos are ultrarelativistic in all the situations that we might come
across, we can approximate the dispersion relation as\sidenote{In the last passage
we have used the ultrarelativistic limit $pc \approx E$. Although it might be
confusing to switch back and forth two times between energy and momentum, once we
have isolated the relevant terms of the expansion the two are interchangeable, as
this is the notation customarily used in the literature.}
\begin{align*}
  E_i = \sqrt{m_i^2 c^4 + p^2 c^2} =
  pc \qty(1 + \frac{m_i^2 c^2}{p})^\frac{1}{2} \approx
  pc + \frac{m_i^2 c^4}{2pc} \approx E + \frac{m_i^2 c^4}{2E},
\end{align*}
and can plug this into our probability to get the final formula for oscillation
in vacuum
\begin{align}\label{eq:nu_two_flavor_osc_prob}
  P(\nu_e \rightarrow \nu_\mu) =
  %\sin^2 2\theta \sin^2 \qty(\frac{\Delta m^2 c^3 t}{4 \hbar p}) =
  \sin^2(2\theta)\sin^2 \qty(\frac{\Delta m^2 L}{4 \hbar c E}),
\end{align}
where we have defined\sidenote{Note that we include the $c^4$ here, so this
is really an energy (as opposed to a mass) squared.}
$\Delta m^2 = (m_2^2 - m_1^2) c^4$ and $L = ct$ is the distance traveled by the
neutrino.

Equation~\eqref{eq:nu_two_flavor_osc_prob} depends on two fundamental parameters
that we have no control on and need to be gauged through measurements. The
\emph{mixing angle} $\theta$ determines the specific mix of flavor eigenstates in
the mass eigenestates: if $\theta = 0$, then the two are the same and oscillations
do not occurr; for $\theta = 45^\circ$ the oscillation is maximal and there exist
a point (or infinite points, really) on the line of flight of the neutrino beam
where all the initial $\nu_e$ have oscillated to $\nu_\mu$. The other important
parameter is the difference of the two squared masses $\Delta m^2$: \emph{for oscillations
to occur at least of of the neutrinos must have non vanishing mass}\sidenote{The
implications of this very basic fact can hardly be overstated, as the observation
of neutrino oscillations per se is a direct proof that neutrinos are massive.}, and
the two masses must be different---otherwise we are never going to generate a
non-trivial phase difference during the propagation. It is also worth noting that,
while $\Delta m^2$ can be positive or negative, since~\eqref{eq:nu_two_flavor_osc_prob}
is even, the oscillation probability is rigorously the same if the change the sign
of $\Delta m^2$: \emph{interferometric experiments are only sensitive to the absolute value
of the difference of square masses but cannot resolve the mass hyerarchy}---we shall
come back to this in a second.

\begin{marginfigure}
  %% Creator: Matplotlib, PGF backend
%%
%% To include the figure in your LaTeX document, write
%%   \input{<filename>.pgf}
%%
%% Make sure the required packages are loaded in your preamble
%%   \usepackage{pgf}
%%
%% Also ensure that all the required font packages are loaded; for instance,
%% the lmodern package is sometimes necessary when using math font.
%%   \usepackage{lmodern}
%%
%% Figures using additional raster images can only be included by \input if
%% they are in the same directory as the main LaTeX file. For loading figures
%% from other directories you can use the `import` package
%%   \usepackage{import}
%%
%% and then include the figures with
%%   \import{<path to file>}{<filename>.pgf}
%%
%% Matplotlib used the following preamble
%%   \usepackage{fontspec}
%%   \setmainfont{DejaVuSerif.ttf}[Path=\detokenize{/usr/share/matplotlib/mpl-data/fonts/ttf/}]
%%   \setsansfont{DejaVuSans.ttf}[Path=\detokenize{/usr/share/matplotlib/mpl-data/fonts/ttf/}]
%%   \setmonofont{DejaVuSansMono.ttf}[Path=\detokenize{/usr/share/matplotlib/mpl-data/fonts/ttf/}]
%%
\begingroup%
\makeatletter%
\begin{pgfpicture}%
\pgfpathrectangle{\pgfpointorigin}{\pgfqpoint{1.950000in}{2.500000in}}%
\pgfusepath{use as bounding box, clip}%
\begin{pgfscope}%
\pgfsetbuttcap%
\pgfsetmiterjoin%
\definecolor{currentfill}{rgb}{1.000000,1.000000,1.000000}%
\pgfsetfillcolor{currentfill}%
\pgfsetlinewidth{0.000000pt}%
\definecolor{currentstroke}{rgb}{1.000000,1.000000,1.000000}%
\pgfsetstrokecolor{currentstroke}%
\pgfsetdash{}{0pt}%
\pgfpathmoveto{\pgfqpoint{0.000000in}{0.000000in}}%
\pgfpathlineto{\pgfqpoint{1.950000in}{0.000000in}}%
\pgfpathlineto{\pgfqpoint{1.950000in}{2.500000in}}%
\pgfpathlineto{\pgfqpoint{0.000000in}{2.500000in}}%
\pgfpathlineto{\pgfqpoint{0.000000in}{0.000000in}}%
\pgfpathclose%
\pgfusepath{fill}%
\end{pgfscope}%
\begin{pgfscope}%
\pgfsetbuttcap%
\pgfsetmiterjoin%
\definecolor{currentfill}{rgb}{1.000000,1.000000,1.000000}%
\pgfsetfillcolor{currentfill}%
\pgfsetlinewidth{0.000000pt}%
\definecolor{currentstroke}{rgb}{0.000000,0.000000,0.000000}%
\pgfsetstrokecolor{currentstroke}%
\pgfsetstrokeopacity{0.000000}%
\pgfsetdash{}{0pt}%
\pgfpathmoveto{\pgfqpoint{0.726250in}{0.525000in}}%
\pgfpathlineto{\pgfqpoint{1.846250in}{0.525000in}}%
\pgfpathlineto{\pgfqpoint{1.846250in}{2.412500in}}%
\pgfpathlineto{\pgfqpoint{0.726250in}{2.412500in}}%
\pgfpathlineto{\pgfqpoint{0.726250in}{0.525000in}}%
\pgfpathclose%
\pgfusepath{fill}%
\end{pgfscope}%
\begin{pgfscope}%
\pgfpathrectangle{\pgfqpoint{0.726250in}{0.525000in}}{\pgfqpoint{1.120000in}{1.887500in}}%
\pgfusepath{clip}%
\pgfsetbuttcap%
\pgfsetroundjoin%
\pgfsetlinewidth{0.803000pt}%
\definecolor{currentstroke}{rgb}{0.752941,0.752941,0.752941}%
\pgfsetstrokecolor{currentstroke}%
\pgfsetdash{{2.960000pt}{1.280000pt}}{0.000000pt}%
\pgfpathmoveto{\pgfqpoint{0.904253in}{0.525000in}}%
\pgfpathlineto{\pgfqpoint{0.904253in}{2.412500in}}%
\pgfusepath{stroke}%
\end{pgfscope}%
\begin{pgfscope}%
\pgfsetbuttcap%
\pgfsetroundjoin%
\definecolor{currentfill}{rgb}{0.000000,0.000000,0.000000}%
\pgfsetfillcolor{currentfill}%
\pgfsetlinewidth{0.803000pt}%
\definecolor{currentstroke}{rgb}{0.000000,0.000000,0.000000}%
\pgfsetstrokecolor{currentstroke}%
\pgfsetdash{}{0pt}%
\pgfsys@defobject{currentmarker}{\pgfqpoint{0.000000in}{-0.048611in}}{\pgfqpoint{0.000000in}{0.000000in}}{%
\pgfpathmoveto{\pgfqpoint{0.000000in}{0.000000in}}%
\pgfpathlineto{\pgfqpoint{0.000000in}{-0.048611in}}%
\pgfusepath{stroke,fill}%
}%
\begin{pgfscope}%
\pgfsys@transformshift{0.904253in}{0.525000in}%
\pgfsys@useobject{currentmarker}{}%
\end{pgfscope}%
\end{pgfscope}%
\begin{pgfscope}%
\definecolor{textcolor}{rgb}{0.000000,0.000000,0.000000}%
\pgfsetstrokecolor{textcolor}%
\pgfsetfillcolor{textcolor}%
\pgftext[x=0.904253in,y=0.427778in,,top]{\color{textcolor}\rmfamily\fontsize{9.000000}{10.800000}\selectfont \(\displaystyle {10^{0}}\)}%
\end{pgfscope}%
\begin{pgfscope}%
\pgfpathrectangle{\pgfqpoint{0.726250in}{0.525000in}}{\pgfqpoint{1.120000in}{1.887500in}}%
\pgfusepath{clip}%
\pgfsetbuttcap%
\pgfsetroundjoin%
\pgfsetlinewidth{0.803000pt}%
\definecolor{currentstroke}{rgb}{0.752941,0.752941,0.752941}%
\pgfsetstrokecolor{currentstroke}%
\pgfsetdash{{2.960000pt}{1.280000pt}}{0.000000pt}%
\pgfpathmoveto{\pgfqpoint{1.158918in}{0.525000in}}%
\pgfpathlineto{\pgfqpoint{1.158918in}{2.412500in}}%
\pgfusepath{stroke}%
\end{pgfscope}%
\begin{pgfscope}%
\pgfsetbuttcap%
\pgfsetroundjoin%
\definecolor{currentfill}{rgb}{0.000000,0.000000,0.000000}%
\pgfsetfillcolor{currentfill}%
\pgfsetlinewidth{0.803000pt}%
\definecolor{currentstroke}{rgb}{0.000000,0.000000,0.000000}%
\pgfsetstrokecolor{currentstroke}%
\pgfsetdash{}{0pt}%
\pgfsys@defobject{currentmarker}{\pgfqpoint{0.000000in}{-0.048611in}}{\pgfqpoint{0.000000in}{0.000000in}}{%
\pgfpathmoveto{\pgfqpoint{0.000000in}{0.000000in}}%
\pgfpathlineto{\pgfqpoint{0.000000in}{-0.048611in}}%
\pgfusepath{stroke,fill}%
}%
\begin{pgfscope}%
\pgfsys@transformshift{1.158918in}{0.525000in}%
\pgfsys@useobject{currentmarker}{}%
\end{pgfscope}%
\end{pgfscope}%
\begin{pgfscope}%
\definecolor{textcolor}{rgb}{0.000000,0.000000,0.000000}%
\pgfsetstrokecolor{textcolor}%
\pgfsetfillcolor{textcolor}%
\pgftext[x=1.158918in,y=0.427778in,,top]{\color{textcolor}\rmfamily\fontsize{9.000000}{10.800000}\selectfont \(\displaystyle {10^{1}}\)}%
\end{pgfscope}%
\begin{pgfscope}%
\pgfpathrectangle{\pgfqpoint{0.726250in}{0.525000in}}{\pgfqpoint{1.120000in}{1.887500in}}%
\pgfusepath{clip}%
\pgfsetbuttcap%
\pgfsetroundjoin%
\pgfsetlinewidth{0.803000pt}%
\definecolor{currentstroke}{rgb}{0.752941,0.752941,0.752941}%
\pgfsetstrokecolor{currentstroke}%
\pgfsetdash{{2.960000pt}{1.280000pt}}{0.000000pt}%
\pgfpathmoveto{\pgfqpoint{1.413582in}{0.525000in}}%
\pgfpathlineto{\pgfqpoint{1.413582in}{2.412500in}}%
\pgfusepath{stroke}%
\end{pgfscope}%
\begin{pgfscope}%
\pgfsetbuttcap%
\pgfsetroundjoin%
\definecolor{currentfill}{rgb}{0.000000,0.000000,0.000000}%
\pgfsetfillcolor{currentfill}%
\pgfsetlinewidth{0.803000pt}%
\definecolor{currentstroke}{rgb}{0.000000,0.000000,0.000000}%
\pgfsetstrokecolor{currentstroke}%
\pgfsetdash{}{0pt}%
\pgfsys@defobject{currentmarker}{\pgfqpoint{0.000000in}{-0.048611in}}{\pgfqpoint{0.000000in}{0.000000in}}{%
\pgfpathmoveto{\pgfqpoint{0.000000in}{0.000000in}}%
\pgfpathlineto{\pgfqpoint{0.000000in}{-0.048611in}}%
\pgfusepath{stroke,fill}%
}%
\begin{pgfscope}%
\pgfsys@transformshift{1.413582in}{0.525000in}%
\pgfsys@useobject{currentmarker}{}%
\end{pgfscope}%
\end{pgfscope}%
\begin{pgfscope}%
\definecolor{textcolor}{rgb}{0.000000,0.000000,0.000000}%
\pgfsetstrokecolor{textcolor}%
\pgfsetfillcolor{textcolor}%
\pgftext[x=1.413582in,y=0.427778in,,top]{\color{textcolor}\rmfamily\fontsize{9.000000}{10.800000}\selectfont \(\displaystyle {10^{2}}\)}%
\end{pgfscope}%
\begin{pgfscope}%
\pgfpathrectangle{\pgfqpoint{0.726250in}{0.525000in}}{\pgfqpoint{1.120000in}{1.887500in}}%
\pgfusepath{clip}%
\pgfsetbuttcap%
\pgfsetroundjoin%
\pgfsetlinewidth{0.803000pt}%
\definecolor{currentstroke}{rgb}{0.752941,0.752941,0.752941}%
\pgfsetstrokecolor{currentstroke}%
\pgfsetdash{{2.960000pt}{1.280000pt}}{0.000000pt}%
\pgfpathmoveto{\pgfqpoint{1.668247in}{0.525000in}}%
\pgfpathlineto{\pgfqpoint{1.668247in}{2.412500in}}%
\pgfusepath{stroke}%
\end{pgfscope}%
\begin{pgfscope}%
\pgfsetbuttcap%
\pgfsetroundjoin%
\definecolor{currentfill}{rgb}{0.000000,0.000000,0.000000}%
\pgfsetfillcolor{currentfill}%
\pgfsetlinewidth{0.803000pt}%
\definecolor{currentstroke}{rgb}{0.000000,0.000000,0.000000}%
\pgfsetstrokecolor{currentstroke}%
\pgfsetdash{}{0pt}%
\pgfsys@defobject{currentmarker}{\pgfqpoint{0.000000in}{-0.048611in}}{\pgfqpoint{0.000000in}{0.000000in}}{%
\pgfpathmoveto{\pgfqpoint{0.000000in}{0.000000in}}%
\pgfpathlineto{\pgfqpoint{0.000000in}{-0.048611in}}%
\pgfusepath{stroke,fill}%
}%
\begin{pgfscope}%
\pgfsys@transformshift{1.668247in}{0.525000in}%
\pgfsys@useobject{currentmarker}{}%
\end{pgfscope}%
\end{pgfscope}%
\begin{pgfscope}%
\definecolor{textcolor}{rgb}{0.000000,0.000000,0.000000}%
\pgfsetstrokecolor{textcolor}%
\pgfsetfillcolor{textcolor}%
\pgftext[x=1.668247in,y=0.427778in,,top]{\color{textcolor}\rmfamily\fontsize{9.000000}{10.800000}\selectfont \(\displaystyle {10^{3}}\)}%
\end{pgfscope}%
\begin{pgfscope}%
\definecolor{textcolor}{rgb}{0.000000,0.000000,0.000000}%
\pgfsetstrokecolor{textcolor}%
\pgfsetfillcolor{textcolor}%
\pgftext[x=1.286250in,y=0.251251in,,top]{\color{textcolor}\rmfamily\fontsize{9.000000}{10.800000}\selectfont Energy [MeV]}%
\end{pgfscope}%
\begin{pgfscope}%
\pgfpathrectangle{\pgfqpoint{0.726250in}{0.525000in}}{\pgfqpoint{1.120000in}{1.887500in}}%
\pgfusepath{clip}%
\pgfsetbuttcap%
\pgfsetroundjoin%
\pgfsetlinewidth{0.803000pt}%
\definecolor{currentstroke}{rgb}{0.752941,0.752941,0.752941}%
\pgfsetstrokecolor{currentstroke}%
\pgfsetdash{{2.960000pt}{1.280000pt}}{0.000000pt}%
\pgfpathmoveto{\pgfqpoint{0.726250in}{0.525000in}}%
\pgfpathlineto{\pgfqpoint{1.846250in}{0.525000in}}%
\pgfusepath{stroke}%
\end{pgfscope}%
\begin{pgfscope}%
\pgfsetbuttcap%
\pgfsetroundjoin%
\definecolor{currentfill}{rgb}{0.000000,0.000000,0.000000}%
\pgfsetfillcolor{currentfill}%
\pgfsetlinewidth{0.803000pt}%
\definecolor{currentstroke}{rgb}{0.000000,0.000000,0.000000}%
\pgfsetstrokecolor{currentstroke}%
\pgfsetdash{}{0pt}%
\pgfsys@defobject{currentmarker}{\pgfqpoint{-0.048611in}{0.000000in}}{\pgfqpoint{-0.000000in}{0.000000in}}{%
\pgfpathmoveto{\pgfqpoint{-0.000000in}{0.000000in}}%
\pgfpathlineto{\pgfqpoint{-0.048611in}{0.000000in}}%
\pgfusepath{stroke,fill}%
}%
\begin{pgfscope}%
\pgfsys@transformshift{0.726250in}{0.525000in}%
\pgfsys@useobject{currentmarker}{}%
\end{pgfscope}%
\end{pgfscope}%
\begin{pgfscope}%
\definecolor{textcolor}{rgb}{0.000000,0.000000,0.000000}%
\pgfsetstrokecolor{textcolor}%
\pgfsetfillcolor{textcolor}%
\pgftext[x=0.442687in, y=0.477515in, left, base]{\color{textcolor}\rmfamily\fontsize{9.000000}{10.800000}\selectfont \(\displaystyle {10^{1}}\)}%
\end{pgfscope}%
\begin{pgfscope}%
\pgfpathrectangle{\pgfqpoint{0.726250in}{0.525000in}}{\pgfqpoint{1.120000in}{1.887500in}}%
\pgfusepath{clip}%
\pgfsetbuttcap%
\pgfsetroundjoin%
\pgfsetlinewidth{0.803000pt}%
\definecolor{currentstroke}{rgb}{0.752941,0.752941,0.752941}%
\pgfsetstrokecolor{currentstroke}%
\pgfsetdash{{2.960000pt}{1.280000pt}}{0.000000pt}%
\pgfpathmoveto{\pgfqpoint{0.726250in}{0.839583in}}%
\pgfpathlineto{\pgfqpoint{1.846250in}{0.839583in}}%
\pgfusepath{stroke}%
\end{pgfscope}%
\begin{pgfscope}%
\pgfsetbuttcap%
\pgfsetroundjoin%
\definecolor{currentfill}{rgb}{0.000000,0.000000,0.000000}%
\pgfsetfillcolor{currentfill}%
\pgfsetlinewidth{0.803000pt}%
\definecolor{currentstroke}{rgb}{0.000000,0.000000,0.000000}%
\pgfsetstrokecolor{currentstroke}%
\pgfsetdash{}{0pt}%
\pgfsys@defobject{currentmarker}{\pgfqpoint{-0.048611in}{0.000000in}}{\pgfqpoint{-0.000000in}{0.000000in}}{%
\pgfpathmoveto{\pgfqpoint{-0.000000in}{0.000000in}}%
\pgfpathlineto{\pgfqpoint{-0.048611in}{0.000000in}}%
\pgfusepath{stroke,fill}%
}%
\begin{pgfscope}%
\pgfsys@transformshift{0.726250in}{0.839583in}%
\pgfsys@useobject{currentmarker}{}%
\end{pgfscope}%
\end{pgfscope}%
\begin{pgfscope}%
\definecolor{textcolor}{rgb}{0.000000,0.000000,0.000000}%
\pgfsetstrokecolor{textcolor}%
\pgfsetfillcolor{textcolor}%
\pgftext[x=0.442687in, y=0.792098in, left, base]{\color{textcolor}\rmfamily\fontsize{9.000000}{10.800000}\selectfont \(\displaystyle {10^{2}}\)}%
\end{pgfscope}%
\begin{pgfscope}%
\pgfpathrectangle{\pgfqpoint{0.726250in}{0.525000in}}{\pgfqpoint{1.120000in}{1.887500in}}%
\pgfusepath{clip}%
\pgfsetbuttcap%
\pgfsetroundjoin%
\pgfsetlinewidth{0.803000pt}%
\definecolor{currentstroke}{rgb}{0.752941,0.752941,0.752941}%
\pgfsetstrokecolor{currentstroke}%
\pgfsetdash{{2.960000pt}{1.280000pt}}{0.000000pt}%
\pgfpathmoveto{\pgfqpoint{0.726250in}{1.154167in}}%
\pgfpathlineto{\pgfqpoint{1.846250in}{1.154167in}}%
\pgfusepath{stroke}%
\end{pgfscope}%
\begin{pgfscope}%
\pgfsetbuttcap%
\pgfsetroundjoin%
\definecolor{currentfill}{rgb}{0.000000,0.000000,0.000000}%
\pgfsetfillcolor{currentfill}%
\pgfsetlinewidth{0.803000pt}%
\definecolor{currentstroke}{rgb}{0.000000,0.000000,0.000000}%
\pgfsetstrokecolor{currentstroke}%
\pgfsetdash{}{0pt}%
\pgfsys@defobject{currentmarker}{\pgfqpoint{-0.048611in}{0.000000in}}{\pgfqpoint{-0.000000in}{0.000000in}}{%
\pgfpathmoveto{\pgfqpoint{-0.000000in}{0.000000in}}%
\pgfpathlineto{\pgfqpoint{-0.048611in}{0.000000in}}%
\pgfusepath{stroke,fill}%
}%
\begin{pgfscope}%
\pgfsys@transformshift{0.726250in}{1.154167in}%
\pgfsys@useobject{currentmarker}{}%
\end{pgfscope}%
\end{pgfscope}%
\begin{pgfscope}%
\definecolor{textcolor}{rgb}{0.000000,0.000000,0.000000}%
\pgfsetstrokecolor{textcolor}%
\pgfsetfillcolor{textcolor}%
\pgftext[x=0.442687in, y=1.106681in, left, base]{\color{textcolor}\rmfamily\fontsize{9.000000}{10.800000}\selectfont \(\displaystyle {10^{3}}\)}%
\end{pgfscope}%
\begin{pgfscope}%
\pgfpathrectangle{\pgfqpoint{0.726250in}{0.525000in}}{\pgfqpoint{1.120000in}{1.887500in}}%
\pgfusepath{clip}%
\pgfsetbuttcap%
\pgfsetroundjoin%
\pgfsetlinewidth{0.803000pt}%
\definecolor{currentstroke}{rgb}{0.752941,0.752941,0.752941}%
\pgfsetstrokecolor{currentstroke}%
\pgfsetdash{{2.960000pt}{1.280000pt}}{0.000000pt}%
\pgfpathmoveto{\pgfqpoint{0.726250in}{1.468750in}}%
\pgfpathlineto{\pgfqpoint{1.846250in}{1.468750in}}%
\pgfusepath{stroke}%
\end{pgfscope}%
\begin{pgfscope}%
\pgfsetbuttcap%
\pgfsetroundjoin%
\definecolor{currentfill}{rgb}{0.000000,0.000000,0.000000}%
\pgfsetfillcolor{currentfill}%
\pgfsetlinewidth{0.803000pt}%
\definecolor{currentstroke}{rgb}{0.000000,0.000000,0.000000}%
\pgfsetstrokecolor{currentstroke}%
\pgfsetdash{}{0pt}%
\pgfsys@defobject{currentmarker}{\pgfqpoint{-0.048611in}{0.000000in}}{\pgfqpoint{-0.000000in}{0.000000in}}{%
\pgfpathmoveto{\pgfqpoint{-0.000000in}{0.000000in}}%
\pgfpathlineto{\pgfqpoint{-0.048611in}{0.000000in}}%
\pgfusepath{stroke,fill}%
}%
\begin{pgfscope}%
\pgfsys@transformshift{0.726250in}{1.468750in}%
\pgfsys@useobject{currentmarker}{}%
\end{pgfscope}%
\end{pgfscope}%
\begin{pgfscope}%
\definecolor{textcolor}{rgb}{0.000000,0.000000,0.000000}%
\pgfsetstrokecolor{textcolor}%
\pgfsetfillcolor{textcolor}%
\pgftext[x=0.442687in, y=1.421265in, left, base]{\color{textcolor}\rmfamily\fontsize{9.000000}{10.800000}\selectfont \(\displaystyle {10^{4}}\)}%
\end{pgfscope}%
\begin{pgfscope}%
\pgfpathrectangle{\pgfqpoint{0.726250in}{0.525000in}}{\pgfqpoint{1.120000in}{1.887500in}}%
\pgfusepath{clip}%
\pgfsetbuttcap%
\pgfsetroundjoin%
\pgfsetlinewidth{0.803000pt}%
\definecolor{currentstroke}{rgb}{0.752941,0.752941,0.752941}%
\pgfsetstrokecolor{currentstroke}%
\pgfsetdash{{2.960000pt}{1.280000pt}}{0.000000pt}%
\pgfpathmoveto{\pgfqpoint{0.726250in}{1.783333in}}%
\pgfpathlineto{\pgfqpoint{1.846250in}{1.783333in}}%
\pgfusepath{stroke}%
\end{pgfscope}%
\begin{pgfscope}%
\pgfsetbuttcap%
\pgfsetroundjoin%
\definecolor{currentfill}{rgb}{0.000000,0.000000,0.000000}%
\pgfsetfillcolor{currentfill}%
\pgfsetlinewidth{0.803000pt}%
\definecolor{currentstroke}{rgb}{0.000000,0.000000,0.000000}%
\pgfsetstrokecolor{currentstroke}%
\pgfsetdash{}{0pt}%
\pgfsys@defobject{currentmarker}{\pgfqpoint{-0.048611in}{0.000000in}}{\pgfqpoint{-0.000000in}{0.000000in}}{%
\pgfpathmoveto{\pgfqpoint{-0.000000in}{0.000000in}}%
\pgfpathlineto{\pgfqpoint{-0.048611in}{0.000000in}}%
\pgfusepath{stroke,fill}%
}%
\begin{pgfscope}%
\pgfsys@transformshift{0.726250in}{1.783333in}%
\pgfsys@useobject{currentmarker}{}%
\end{pgfscope}%
\end{pgfscope}%
\begin{pgfscope}%
\definecolor{textcolor}{rgb}{0.000000,0.000000,0.000000}%
\pgfsetstrokecolor{textcolor}%
\pgfsetfillcolor{textcolor}%
\pgftext[x=0.442687in, y=1.735848in, left, base]{\color{textcolor}\rmfamily\fontsize{9.000000}{10.800000}\selectfont \(\displaystyle {10^{5}}\)}%
\end{pgfscope}%
\begin{pgfscope}%
\pgfpathrectangle{\pgfqpoint{0.726250in}{0.525000in}}{\pgfqpoint{1.120000in}{1.887500in}}%
\pgfusepath{clip}%
\pgfsetbuttcap%
\pgfsetroundjoin%
\pgfsetlinewidth{0.803000pt}%
\definecolor{currentstroke}{rgb}{0.752941,0.752941,0.752941}%
\pgfsetstrokecolor{currentstroke}%
\pgfsetdash{{2.960000pt}{1.280000pt}}{0.000000pt}%
\pgfpathmoveto{\pgfqpoint{0.726250in}{2.097917in}}%
\pgfpathlineto{\pgfqpoint{1.846250in}{2.097917in}}%
\pgfusepath{stroke}%
\end{pgfscope}%
\begin{pgfscope}%
\pgfsetbuttcap%
\pgfsetroundjoin%
\definecolor{currentfill}{rgb}{0.000000,0.000000,0.000000}%
\pgfsetfillcolor{currentfill}%
\pgfsetlinewidth{0.803000pt}%
\definecolor{currentstroke}{rgb}{0.000000,0.000000,0.000000}%
\pgfsetstrokecolor{currentstroke}%
\pgfsetdash{}{0pt}%
\pgfsys@defobject{currentmarker}{\pgfqpoint{-0.048611in}{0.000000in}}{\pgfqpoint{-0.000000in}{0.000000in}}{%
\pgfpathmoveto{\pgfqpoint{-0.000000in}{0.000000in}}%
\pgfpathlineto{\pgfqpoint{-0.048611in}{0.000000in}}%
\pgfusepath{stroke,fill}%
}%
\begin{pgfscope}%
\pgfsys@transformshift{0.726250in}{2.097917in}%
\pgfsys@useobject{currentmarker}{}%
\end{pgfscope}%
\end{pgfscope}%
\begin{pgfscope}%
\definecolor{textcolor}{rgb}{0.000000,0.000000,0.000000}%
\pgfsetstrokecolor{textcolor}%
\pgfsetfillcolor{textcolor}%
\pgftext[x=0.442687in, y=2.050431in, left, base]{\color{textcolor}\rmfamily\fontsize{9.000000}{10.800000}\selectfont \(\displaystyle {10^{6}}\)}%
\end{pgfscope}%
\begin{pgfscope}%
\pgfpathrectangle{\pgfqpoint{0.726250in}{0.525000in}}{\pgfqpoint{1.120000in}{1.887500in}}%
\pgfusepath{clip}%
\pgfsetbuttcap%
\pgfsetroundjoin%
\pgfsetlinewidth{0.803000pt}%
\definecolor{currentstroke}{rgb}{0.752941,0.752941,0.752941}%
\pgfsetstrokecolor{currentstroke}%
\pgfsetdash{{2.960000pt}{1.280000pt}}{0.000000pt}%
\pgfpathmoveto{\pgfqpoint{0.726250in}{2.412500in}}%
\pgfpathlineto{\pgfqpoint{1.846250in}{2.412500in}}%
\pgfusepath{stroke}%
\end{pgfscope}%
\begin{pgfscope}%
\pgfsetbuttcap%
\pgfsetroundjoin%
\definecolor{currentfill}{rgb}{0.000000,0.000000,0.000000}%
\pgfsetfillcolor{currentfill}%
\pgfsetlinewidth{0.803000pt}%
\definecolor{currentstroke}{rgb}{0.000000,0.000000,0.000000}%
\pgfsetstrokecolor{currentstroke}%
\pgfsetdash{}{0pt}%
\pgfsys@defobject{currentmarker}{\pgfqpoint{-0.048611in}{0.000000in}}{\pgfqpoint{-0.000000in}{0.000000in}}{%
\pgfpathmoveto{\pgfqpoint{-0.000000in}{0.000000in}}%
\pgfpathlineto{\pgfqpoint{-0.048611in}{0.000000in}}%
\pgfusepath{stroke,fill}%
}%
\begin{pgfscope}%
\pgfsys@transformshift{0.726250in}{2.412500in}%
\pgfsys@useobject{currentmarker}{}%
\end{pgfscope}%
\end{pgfscope}%
\begin{pgfscope}%
\definecolor{textcolor}{rgb}{0.000000,0.000000,0.000000}%
\pgfsetstrokecolor{textcolor}%
\pgfsetfillcolor{textcolor}%
\pgftext[x=0.442687in, y=2.365015in, left, base]{\color{textcolor}\rmfamily\fontsize{9.000000}{10.800000}\selectfont \(\displaystyle {10^{7}}\)}%
\end{pgfscope}%
\begin{pgfscope}%
\definecolor{textcolor}{rgb}{0.000000,0.000000,0.000000}%
\pgfsetstrokecolor{textcolor}%
\pgfsetfillcolor{textcolor}%
\pgftext[x=0.387131in,y=1.468750in,,bottom,rotate=90.000000]{\color{textcolor}\rmfamily\fontsize{9.000000}{10.800000}\selectfont \(\displaystyle L_\mathrm{osc}\) [m]}%
\end{pgfscope}%
\begin{pgfscope}%
\pgfpathrectangle{\pgfqpoint{0.726250in}{0.525000in}}{\pgfqpoint{1.120000in}{1.887500in}}%
\pgfusepath{clip}%
\pgfsetbuttcap%
\pgfsetroundjoin%
\pgfsetlinewidth{1.003750pt}%
\definecolor{currentstroke}{rgb}{0.000000,0.000000,0.000000}%
\pgfsetstrokecolor{currentstroke}%
\pgfsetdash{{3.700000pt}{1.600000pt}}{0.000000pt}%
\pgfpathmoveto{\pgfqpoint{0.726250in}{1.101099in}}%
\pgfpathlineto{\pgfqpoint{0.749107in}{1.129334in}}%
\pgfpathlineto{\pgfqpoint{0.771964in}{1.157569in}}%
\pgfpathlineto{\pgfqpoint{0.794821in}{1.185804in}}%
\pgfpathlineto{\pgfqpoint{0.817679in}{1.214039in}}%
\pgfpathlineto{\pgfqpoint{0.840536in}{1.242274in}}%
\pgfpathlineto{\pgfqpoint{0.863393in}{1.270510in}}%
\pgfpathlineto{\pgfqpoint{0.886250in}{1.298745in}}%
\pgfpathlineto{\pgfqpoint{0.909107in}{1.326980in}}%
\pgfpathlineto{\pgfqpoint{0.931964in}{1.355215in}}%
\pgfpathlineto{\pgfqpoint{0.954821in}{1.383450in}}%
\pgfpathlineto{\pgfqpoint{0.977679in}{1.411685in}}%
\pgfpathlineto{\pgfqpoint{1.000536in}{1.439920in}}%
\pgfpathlineto{\pgfqpoint{1.023393in}{1.468155in}}%
\pgfpathlineto{\pgfqpoint{1.046250in}{1.496390in}}%
\pgfpathlineto{\pgfqpoint{1.069107in}{1.524625in}}%
\pgfpathlineto{\pgfqpoint{1.091964in}{1.552860in}}%
\pgfpathlineto{\pgfqpoint{1.114821in}{1.581095in}}%
\pgfpathlineto{\pgfqpoint{1.137679in}{1.609330in}}%
\pgfpathlineto{\pgfqpoint{1.160536in}{1.637566in}}%
\pgfpathlineto{\pgfqpoint{1.183393in}{1.665801in}}%
\pgfpathlineto{\pgfqpoint{1.206250in}{1.694036in}}%
\pgfpathlineto{\pgfqpoint{1.229107in}{1.722271in}}%
\pgfpathlineto{\pgfqpoint{1.251964in}{1.750506in}}%
\pgfpathlineto{\pgfqpoint{1.274821in}{1.778741in}}%
\pgfpathlineto{\pgfqpoint{1.297679in}{1.806976in}}%
\pgfpathlineto{\pgfqpoint{1.320536in}{1.835211in}}%
\pgfpathlineto{\pgfqpoint{1.343393in}{1.863446in}}%
\pgfpathlineto{\pgfqpoint{1.366250in}{1.891681in}}%
\pgfpathlineto{\pgfqpoint{1.389107in}{1.919916in}}%
\pgfpathlineto{\pgfqpoint{1.411964in}{1.948151in}}%
\pgfpathlineto{\pgfqpoint{1.434821in}{1.976386in}}%
\pgfpathlineto{\pgfqpoint{1.457679in}{2.004621in}}%
\pgfpathlineto{\pgfqpoint{1.480536in}{2.032857in}}%
\pgfpathlineto{\pgfqpoint{1.503393in}{2.061092in}}%
\pgfpathlineto{\pgfqpoint{1.526250in}{2.089327in}}%
\pgfpathlineto{\pgfqpoint{1.549107in}{2.117562in}}%
\pgfpathlineto{\pgfqpoint{1.571964in}{2.145797in}}%
\pgfpathlineto{\pgfqpoint{1.594821in}{2.174032in}}%
\pgfpathlineto{\pgfqpoint{1.617679in}{2.202267in}}%
\pgfpathlineto{\pgfqpoint{1.640536in}{2.230502in}}%
\pgfpathlineto{\pgfqpoint{1.663393in}{2.258737in}}%
\pgfpathlineto{\pgfqpoint{1.686250in}{2.286972in}}%
\pgfpathlineto{\pgfqpoint{1.709107in}{2.315207in}}%
\pgfpathlineto{\pgfqpoint{1.731964in}{2.343442in}}%
\pgfpathlineto{\pgfqpoint{1.754821in}{2.371677in}}%
\pgfpathlineto{\pgfqpoint{1.777679in}{2.399912in}}%
\pgfpathlineto{\pgfqpoint{1.795964in}{2.422500in}}%
\pgfusepath{stroke}%
\end{pgfscope}%
\begin{pgfscope}%
\pgfpathrectangle{\pgfqpoint{0.726250in}{0.525000in}}{\pgfqpoint{1.120000in}{1.887500in}}%
\pgfusepath{clip}%
\pgfsetbuttcap%
\pgfsetroundjoin%
\pgfsetlinewidth{1.003750pt}%
\definecolor{currentstroke}{rgb}{0.000000,0.000000,0.000000}%
\pgfsetstrokecolor{currentstroke}%
\pgfsetdash{{3.700000pt}{1.600000pt}}{0.000000pt}%
\pgfpathmoveto{\pgfqpoint{0.726250in}{0.619995in}}%
\pgfpathlineto{\pgfqpoint{0.749107in}{0.648231in}}%
\pgfpathlineto{\pgfqpoint{0.771964in}{0.676466in}}%
\pgfpathlineto{\pgfqpoint{0.794821in}{0.704701in}}%
\pgfpathlineto{\pgfqpoint{0.817679in}{0.732936in}}%
\pgfpathlineto{\pgfqpoint{0.840536in}{0.761171in}}%
\pgfpathlineto{\pgfqpoint{0.863393in}{0.789406in}}%
\pgfpathlineto{\pgfqpoint{0.886250in}{0.817641in}}%
\pgfpathlineto{\pgfqpoint{0.909107in}{0.845876in}}%
\pgfpathlineto{\pgfqpoint{0.931964in}{0.874111in}}%
\pgfpathlineto{\pgfqpoint{0.954821in}{0.902346in}}%
\pgfpathlineto{\pgfqpoint{0.977679in}{0.930581in}}%
\pgfpathlineto{\pgfqpoint{1.000536in}{0.958816in}}%
\pgfpathlineto{\pgfqpoint{1.023393in}{0.987051in}}%
\pgfpathlineto{\pgfqpoint{1.046250in}{1.015286in}}%
\pgfpathlineto{\pgfqpoint{1.069107in}{1.043522in}}%
\pgfpathlineto{\pgfqpoint{1.091964in}{1.071757in}}%
\pgfpathlineto{\pgfqpoint{1.114821in}{1.099992in}}%
\pgfpathlineto{\pgfqpoint{1.137679in}{1.128227in}}%
\pgfpathlineto{\pgfqpoint{1.160536in}{1.156462in}}%
\pgfpathlineto{\pgfqpoint{1.183393in}{1.184697in}}%
\pgfpathlineto{\pgfqpoint{1.206250in}{1.212932in}}%
\pgfpathlineto{\pgfqpoint{1.229107in}{1.241167in}}%
\pgfpathlineto{\pgfqpoint{1.251964in}{1.269402in}}%
\pgfpathlineto{\pgfqpoint{1.274821in}{1.297637in}}%
\pgfpathlineto{\pgfqpoint{1.297679in}{1.325872in}}%
\pgfpathlineto{\pgfqpoint{1.320536in}{1.354107in}}%
\pgfpathlineto{\pgfqpoint{1.343393in}{1.382342in}}%
\pgfpathlineto{\pgfqpoint{1.366250in}{1.410578in}}%
\pgfpathlineto{\pgfqpoint{1.389107in}{1.438813in}}%
\pgfpathlineto{\pgfqpoint{1.411964in}{1.467048in}}%
\pgfpathlineto{\pgfqpoint{1.434821in}{1.495283in}}%
\pgfpathlineto{\pgfqpoint{1.457679in}{1.523518in}}%
\pgfpathlineto{\pgfqpoint{1.480536in}{1.551753in}}%
\pgfpathlineto{\pgfqpoint{1.503393in}{1.579988in}}%
\pgfpathlineto{\pgfqpoint{1.526250in}{1.608223in}}%
\pgfpathlineto{\pgfqpoint{1.549107in}{1.636458in}}%
\pgfpathlineto{\pgfqpoint{1.571964in}{1.664693in}}%
\pgfpathlineto{\pgfqpoint{1.594821in}{1.692928in}}%
\pgfpathlineto{\pgfqpoint{1.617679in}{1.721163in}}%
\pgfpathlineto{\pgfqpoint{1.640536in}{1.749398in}}%
\pgfpathlineto{\pgfqpoint{1.663393in}{1.777633in}}%
\pgfpathlineto{\pgfqpoint{1.686250in}{1.805869in}}%
\pgfpathlineto{\pgfqpoint{1.709107in}{1.834104in}}%
\pgfpathlineto{\pgfqpoint{1.731964in}{1.862339in}}%
\pgfpathlineto{\pgfqpoint{1.754821in}{1.890574in}}%
\pgfpathlineto{\pgfqpoint{1.777679in}{1.918809in}}%
\pgfpathlineto{\pgfqpoint{1.800536in}{1.947044in}}%
\pgfpathlineto{\pgfqpoint{1.823393in}{1.975279in}}%
\pgfpathlineto{\pgfqpoint{1.846250in}{2.003514in}}%
\pgfusepath{stroke}%
\end{pgfscope}%
\begin{pgfscope}%
\pgfsetrectcap%
\pgfsetmiterjoin%
\pgfsetlinewidth{1.003750pt}%
\definecolor{currentstroke}{rgb}{0.000000,0.000000,0.000000}%
\pgfsetstrokecolor{currentstroke}%
\pgfsetdash{}{0pt}%
\pgfpathmoveto{\pgfqpoint{0.726250in}{0.525000in}}%
\pgfpathlineto{\pgfqpoint{0.726250in}{2.412500in}}%
\pgfusepath{stroke}%
\end{pgfscope}%
\begin{pgfscope}%
\pgfsetrectcap%
\pgfsetmiterjoin%
\pgfsetlinewidth{1.003750pt}%
\definecolor{currentstroke}{rgb}{0.000000,0.000000,0.000000}%
\pgfsetstrokecolor{currentstroke}%
\pgfsetdash{}{0pt}%
\pgfpathmoveto{\pgfqpoint{1.846250in}{0.525000in}}%
\pgfpathlineto{\pgfqpoint{1.846250in}{2.412500in}}%
\pgfusepath{stroke}%
\end{pgfscope}%
\begin{pgfscope}%
\pgfsetrectcap%
\pgfsetmiterjoin%
\pgfsetlinewidth{1.003750pt}%
\definecolor{currentstroke}{rgb}{0.000000,0.000000,0.000000}%
\pgfsetstrokecolor{currentstroke}%
\pgfsetdash{}{0pt}%
\pgfpathmoveto{\pgfqpoint{0.726250in}{0.525000in}}%
\pgfpathlineto{\pgfqpoint{1.846250in}{0.525000in}}%
\pgfusepath{stroke}%
\end{pgfscope}%
\begin{pgfscope}%
\pgfsetrectcap%
\pgfsetmiterjoin%
\pgfsetlinewidth{1.003750pt}%
\definecolor{currentstroke}{rgb}{0.000000,0.000000,0.000000}%
\pgfsetstrokecolor{currentstroke}%
\pgfsetdash{}{0pt}%
\pgfpathmoveto{\pgfqpoint{0.726250in}{2.412500in}}%
\pgfpathlineto{\pgfqpoint{1.846250in}{2.412500in}}%
\pgfusepath{stroke}%
\end{pgfscope}%
\begin{pgfscope}%
\definecolor{textcolor}{rgb}{0.000000,0.000000,0.000000}%
\pgfsetstrokecolor{textcolor}%
\pgfsetfillcolor{textcolor}%
\pgftext[x=1.413582in,y=1.950150in,right,bottom]{\color{textcolor}\rmfamily\fontsize{7.497000}{8.996400}\selectfont \(\displaystyle \Delta m^2_\odot\)}%
\end{pgfscope}%
\begin{pgfscope}%
\definecolor{textcolor}{rgb}{0.000000,0.000000,0.000000}%
\pgfsetstrokecolor{textcolor}%
\pgfsetfillcolor{textcolor}%
\pgftext[x=1.158918in,y=1.154463in,left,top]{\color{textcolor}\rmfamily\fontsize{7.497000}{8.996400}\selectfont \(\displaystyle \Delta m^2_\mathrm{atm}\)}%
\end{pgfscope}%
\end{pgfpicture}%
\makeatother%
\endgroup%

  \caption{Neutrino oscillation length as a function of energy for the two relevant
  values of $\Delta m^2$, see section~\ref{sec:}}
  \label{fig:nu_oscillation_length}
\end{marginfigure}

Once the mixing angle $\theta$ and $\Delta m^2$ are fixed, the relevant quantity
is the phase of the second term in~\eqref{eq:nu_two_flavor_osc_prob}, which
can be conveniently rewritten as
\begin{align}
  \varphi(L, E) = \frac{\Delta m^2 L}{4 \hbar c E} =
  1.27\, \frac{\Delta m^2~[\text{eV}^2]~L~[\text{m}]}{E~[\text{MeV}]}.
\end{align}
A concept that is often found in literature and is helpful to grasp the meaning of
neutrino oscillation is that of \emph{oscillation length}
\begin{align*}
  L_\text{osc}(E) = \frac{1}{\pi} \frac{4\hbar c E}{\Delta m^2},
\end{align*}
using which the formula for the oscillation probability can can be recasted in the
equivalent form
\begin{align*}
  P(\nu_e \rightarrow \nu_\mu) =\sin^2(2\theta)\sin^2 \qty(\pi \frac{L}{L_\text{osc}}).
\end{align*}
If, for a given energy $E$ and propagation length $L$, $\varphi \ll 1$, then there
is no oscillation. The regime in which $\varphi \approx 1$ is where it is possible
to study the minimum/maximum in the oscillation behaviour.

\begin{figure}[!htbp]
  %% Creator: Matplotlib, PGF backend
%%
%% To include the figure in your LaTeX document, write
%%   \input{<filename>.pgf}
%%
%% Make sure the required packages are loaded in your preamble
%%   \usepackage{pgf}
%%
%% Also ensure that all the required font packages are loaded; for instance,
%% the lmodern package is sometimes necessary when using math font.
%%   \usepackage{lmodern}
%%
%% Figures using additional raster images can only be included by \input if
%% they are in the same directory as the main LaTeX file. For loading figures
%% from other directories you can use the `import` package
%%   \usepackage{import}
%%
%% and then include the figures with
%%   \import{<path to file>}{<filename>.pgf}
%%
%% Matplotlib used the following preamble
%%   \usepackage{fontspec}
%%   \setmainfont{DejaVuSerif.ttf}[Path=\detokenize{/usr/share/matplotlib/mpl-data/fonts/ttf/}]
%%   \setsansfont{DejaVuSans.ttf}[Path=\detokenize{/usr/share/matplotlib/mpl-data/fonts/ttf/}]
%%   \setmonofont{DejaVuSansMono.ttf}[Path=\detokenize{/usr/share/matplotlib/mpl-data/fonts/ttf/}]
%%
\begingroup%
\makeatletter%
\begin{pgfpicture}%
\pgfpathrectangle{\pgfpointorigin}{\pgfqpoint{4.150000in}{2.500000in}}%
\pgfusepath{use as bounding box, clip}%
\begin{pgfscope}%
\pgfsetbuttcap%
\pgfsetmiterjoin%
\definecolor{currentfill}{rgb}{1.000000,1.000000,1.000000}%
\pgfsetfillcolor{currentfill}%
\pgfsetlinewidth{0.000000pt}%
\definecolor{currentstroke}{rgb}{1.000000,1.000000,1.000000}%
\pgfsetstrokecolor{currentstroke}%
\pgfsetdash{}{0pt}%
\pgfpathmoveto{\pgfqpoint{0.000000in}{0.000000in}}%
\pgfpathlineto{\pgfqpoint{4.150000in}{0.000000in}}%
\pgfpathlineto{\pgfqpoint{4.150000in}{2.500000in}}%
\pgfpathlineto{\pgfqpoint{0.000000in}{2.500000in}}%
\pgfpathlineto{\pgfqpoint{0.000000in}{0.000000in}}%
\pgfpathclose%
\pgfusepath{fill}%
\end{pgfscope}%
\begin{pgfscope}%
\pgfsetbuttcap%
\pgfsetmiterjoin%
\definecolor{currentfill}{rgb}{1.000000,1.000000,1.000000}%
\pgfsetfillcolor{currentfill}%
\pgfsetlinewidth{0.000000pt}%
\definecolor{currentstroke}{rgb}{0.000000,0.000000,0.000000}%
\pgfsetstrokecolor{currentstroke}%
\pgfsetstrokeopacity{0.000000}%
\pgfsetdash{}{0pt}%
\pgfpathmoveto{\pgfqpoint{0.726250in}{0.525000in}}%
\pgfpathlineto{\pgfqpoint{4.046250in}{0.525000in}}%
\pgfpathlineto{\pgfqpoint{4.046250in}{2.412500in}}%
\pgfpathlineto{\pgfqpoint{0.726250in}{2.412500in}}%
\pgfpathlineto{\pgfqpoint{0.726250in}{0.525000in}}%
\pgfpathclose%
\pgfusepath{fill}%
\end{pgfscope}%
\begin{pgfscope}%
\pgfsetbuttcap%
\pgfsetroundjoin%
\definecolor{currentfill}{rgb}{0.000000,0.000000,0.000000}%
\pgfsetfillcolor{currentfill}%
\pgfsetlinewidth{0.803000pt}%
\definecolor{currentstroke}{rgb}{0.000000,0.000000,0.000000}%
\pgfsetstrokecolor{currentstroke}%
\pgfsetdash{}{0pt}%
\pgfsys@defobject{currentmarker}{\pgfqpoint{0.000000in}{-0.048611in}}{\pgfqpoint{0.000000in}{0.000000in}}{%
\pgfpathmoveto{\pgfqpoint{0.000000in}{0.000000in}}%
\pgfpathlineto{\pgfqpoint{0.000000in}{-0.048611in}}%
\pgfusepath{stroke,fill}%
}%
\begin{pgfscope}%
\pgfsys@transformshift{0.726250in}{0.525000in}%
\pgfsys@useobject{currentmarker}{}%
\end{pgfscope}%
\end{pgfscope}%
\begin{pgfscope}%
\definecolor{textcolor}{rgb}{0.000000,0.000000,0.000000}%
\pgfsetstrokecolor{textcolor}%
\pgfsetfillcolor{textcolor}%
\pgftext[x=0.726250in,y=0.427778in,,top]{\color{textcolor}\rmfamily\fontsize{9.000000}{10.800000}\selectfont 0}%
\end{pgfscope}%
\begin{pgfscope}%
\pgfsetbuttcap%
\pgfsetroundjoin%
\definecolor{currentfill}{rgb}{0.000000,0.000000,0.000000}%
\pgfsetfillcolor{currentfill}%
\pgfsetlinewidth{0.803000pt}%
\definecolor{currentstroke}{rgb}{0.000000,0.000000,0.000000}%
\pgfsetstrokecolor{currentstroke}%
\pgfsetdash{}{0pt}%
\pgfsys@defobject{currentmarker}{\pgfqpoint{0.000000in}{-0.048611in}}{\pgfqpoint{0.000000in}{0.000000in}}{%
\pgfpathmoveto{\pgfqpoint{0.000000in}{0.000000in}}%
\pgfpathlineto{\pgfqpoint{0.000000in}{-0.048611in}}%
\pgfusepath{stroke,fill}%
}%
\begin{pgfscope}%
\pgfsys@transformshift{1.141250in}{0.525000in}%
\pgfsys@useobject{currentmarker}{}%
\end{pgfscope}%
\end{pgfscope}%
\begin{pgfscope}%
\definecolor{textcolor}{rgb}{0.000000,0.000000,0.000000}%
\pgfsetstrokecolor{textcolor}%
\pgfsetfillcolor{textcolor}%
\pgftext[x=1.141250in,y=0.427778in,,top]{\color{textcolor}\rmfamily\fontsize{9.000000}{10.800000}\selectfont 25}%
\end{pgfscope}%
\begin{pgfscope}%
\pgfsetbuttcap%
\pgfsetroundjoin%
\definecolor{currentfill}{rgb}{0.000000,0.000000,0.000000}%
\pgfsetfillcolor{currentfill}%
\pgfsetlinewidth{0.803000pt}%
\definecolor{currentstroke}{rgb}{0.000000,0.000000,0.000000}%
\pgfsetstrokecolor{currentstroke}%
\pgfsetdash{}{0pt}%
\pgfsys@defobject{currentmarker}{\pgfqpoint{0.000000in}{-0.048611in}}{\pgfqpoint{0.000000in}{0.000000in}}{%
\pgfpathmoveto{\pgfqpoint{0.000000in}{0.000000in}}%
\pgfpathlineto{\pgfqpoint{0.000000in}{-0.048611in}}%
\pgfusepath{stroke,fill}%
}%
\begin{pgfscope}%
\pgfsys@transformshift{1.556250in}{0.525000in}%
\pgfsys@useobject{currentmarker}{}%
\end{pgfscope}%
\end{pgfscope}%
\begin{pgfscope}%
\definecolor{textcolor}{rgb}{0.000000,0.000000,0.000000}%
\pgfsetstrokecolor{textcolor}%
\pgfsetfillcolor{textcolor}%
\pgftext[x=1.556250in,y=0.427778in,,top]{\color{textcolor}\rmfamily\fontsize{9.000000}{10.800000}\selectfont 50}%
\end{pgfscope}%
\begin{pgfscope}%
\pgfsetbuttcap%
\pgfsetroundjoin%
\definecolor{currentfill}{rgb}{0.000000,0.000000,0.000000}%
\pgfsetfillcolor{currentfill}%
\pgfsetlinewidth{0.803000pt}%
\definecolor{currentstroke}{rgb}{0.000000,0.000000,0.000000}%
\pgfsetstrokecolor{currentstroke}%
\pgfsetdash{}{0pt}%
\pgfsys@defobject{currentmarker}{\pgfqpoint{0.000000in}{-0.048611in}}{\pgfqpoint{0.000000in}{0.000000in}}{%
\pgfpathmoveto{\pgfqpoint{0.000000in}{0.000000in}}%
\pgfpathlineto{\pgfqpoint{0.000000in}{-0.048611in}}%
\pgfusepath{stroke,fill}%
}%
\begin{pgfscope}%
\pgfsys@transformshift{1.971250in}{0.525000in}%
\pgfsys@useobject{currentmarker}{}%
\end{pgfscope}%
\end{pgfscope}%
\begin{pgfscope}%
\definecolor{textcolor}{rgb}{0.000000,0.000000,0.000000}%
\pgfsetstrokecolor{textcolor}%
\pgfsetfillcolor{textcolor}%
\pgftext[x=1.971250in,y=0.427778in,,top]{\color{textcolor}\rmfamily\fontsize{9.000000}{10.800000}\selectfont 75}%
\end{pgfscope}%
\begin{pgfscope}%
\pgfsetbuttcap%
\pgfsetroundjoin%
\definecolor{currentfill}{rgb}{0.000000,0.000000,0.000000}%
\pgfsetfillcolor{currentfill}%
\pgfsetlinewidth{0.803000pt}%
\definecolor{currentstroke}{rgb}{0.000000,0.000000,0.000000}%
\pgfsetstrokecolor{currentstroke}%
\pgfsetdash{}{0pt}%
\pgfsys@defobject{currentmarker}{\pgfqpoint{0.000000in}{-0.048611in}}{\pgfqpoint{0.000000in}{0.000000in}}{%
\pgfpathmoveto{\pgfqpoint{0.000000in}{0.000000in}}%
\pgfpathlineto{\pgfqpoint{0.000000in}{-0.048611in}}%
\pgfusepath{stroke,fill}%
}%
\begin{pgfscope}%
\pgfsys@transformshift{2.386250in}{0.525000in}%
\pgfsys@useobject{currentmarker}{}%
\end{pgfscope}%
\end{pgfscope}%
\begin{pgfscope}%
\definecolor{textcolor}{rgb}{0.000000,0.000000,0.000000}%
\pgfsetstrokecolor{textcolor}%
\pgfsetfillcolor{textcolor}%
\pgftext[x=2.386250in,y=0.427778in,,top]{\color{textcolor}\rmfamily\fontsize{9.000000}{10.800000}\selectfont 100}%
\end{pgfscope}%
\begin{pgfscope}%
\pgfsetbuttcap%
\pgfsetroundjoin%
\definecolor{currentfill}{rgb}{0.000000,0.000000,0.000000}%
\pgfsetfillcolor{currentfill}%
\pgfsetlinewidth{0.803000pt}%
\definecolor{currentstroke}{rgb}{0.000000,0.000000,0.000000}%
\pgfsetstrokecolor{currentstroke}%
\pgfsetdash{}{0pt}%
\pgfsys@defobject{currentmarker}{\pgfqpoint{0.000000in}{-0.048611in}}{\pgfqpoint{0.000000in}{0.000000in}}{%
\pgfpathmoveto{\pgfqpoint{0.000000in}{0.000000in}}%
\pgfpathlineto{\pgfqpoint{0.000000in}{-0.048611in}}%
\pgfusepath{stroke,fill}%
}%
\begin{pgfscope}%
\pgfsys@transformshift{2.801250in}{0.525000in}%
\pgfsys@useobject{currentmarker}{}%
\end{pgfscope}%
\end{pgfscope}%
\begin{pgfscope}%
\definecolor{textcolor}{rgb}{0.000000,0.000000,0.000000}%
\pgfsetstrokecolor{textcolor}%
\pgfsetfillcolor{textcolor}%
\pgftext[x=2.801250in,y=0.427778in,,top]{\color{textcolor}\rmfamily\fontsize{9.000000}{10.800000}\selectfont 125}%
\end{pgfscope}%
\begin{pgfscope}%
\pgfsetbuttcap%
\pgfsetroundjoin%
\definecolor{currentfill}{rgb}{0.000000,0.000000,0.000000}%
\pgfsetfillcolor{currentfill}%
\pgfsetlinewidth{0.803000pt}%
\definecolor{currentstroke}{rgb}{0.000000,0.000000,0.000000}%
\pgfsetstrokecolor{currentstroke}%
\pgfsetdash{}{0pt}%
\pgfsys@defobject{currentmarker}{\pgfqpoint{0.000000in}{-0.048611in}}{\pgfqpoint{0.000000in}{0.000000in}}{%
\pgfpathmoveto{\pgfqpoint{0.000000in}{0.000000in}}%
\pgfpathlineto{\pgfqpoint{0.000000in}{-0.048611in}}%
\pgfusepath{stroke,fill}%
}%
\begin{pgfscope}%
\pgfsys@transformshift{3.216250in}{0.525000in}%
\pgfsys@useobject{currentmarker}{}%
\end{pgfscope}%
\end{pgfscope}%
\begin{pgfscope}%
\definecolor{textcolor}{rgb}{0.000000,0.000000,0.000000}%
\pgfsetstrokecolor{textcolor}%
\pgfsetfillcolor{textcolor}%
\pgftext[x=3.216250in,y=0.427778in,,top]{\color{textcolor}\rmfamily\fontsize{9.000000}{10.800000}\selectfont 150}%
\end{pgfscope}%
\begin{pgfscope}%
\pgfsetbuttcap%
\pgfsetroundjoin%
\definecolor{currentfill}{rgb}{0.000000,0.000000,0.000000}%
\pgfsetfillcolor{currentfill}%
\pgfsetlinewidth{0.803000pt}%
\definecolor{currentstroke}{rgb}{0.000000,0.000000,0.000000}%
\pgfsetstrokecolor{currentstroke}%
\pgfsetdash{}{0pt}%
\pgfsys@defobject{currentmarker}{\pgfqpoint{0.000000in}{-0.048611in}}{\pgfqpoint{0.000000in}{0.000000in}}{%
\pgfpathmoveto{\pgfqpoint{0.000000in}{0.000000in}}%
\pgfpathlineto{\pgfqpoint{0.000000in}{-0.048611in}}%
\pgfusepath{stroke,fill}%
}%
\begin{pgfscope}%
\pgfsys@transformshift{3.631250in}{0.525000in}%
\pgfsys@useobject{currentmarker}{}%
\end{pgfscope}%
\end{pgfscope}%
\begin{pgfscope}%
\definecolor{textcolor}{rgb}{0.000000,0.000000,0.000000}%
\pgfsetstrokecolor{textcolor}%
\pgfsetfillcolor{textcolor}%
\pgftext[x=3.631250in,y=0.427778in,,top]{\color{textcolor}\rmfamily\fontsize{9.000000}{10.800000}\selectfont 175}%
\end{pgfscope}%
\begin{pgfscope}%
\pgfsetbuttcap%
\pgfsetroundjoin%
\definecolor{currentfill}{rgb}{0.000000,0.000000,0.000000}%
\pgfsetfillcolor{currentfill}%
\pgfsetlinewidth{0.803000pt}%
\definecolor{currentstroke}{rgb}{0.000000,0.000000,0.000000}%
\pgfsetstrokecolor{currentstroke}%
\pgfsetdash{}{0pt}%
\pgfsys@defobject{currentmarker}{\pgfqpoint{0.000000in}{-0.048611in}}{\pgfqpoint{0.000000in}{0.000000in}}{%
\pgfpathmoveto{\pgfqpoint{0.000000in}{0.000000in}}%
\pgfpathlineto{\pgfqpoint{0.000000in}{-0.048611in}}%
\pgfusepath{stroke,fill}%
}%
\begin{pgfscope}%
\pgfsys@transformshift{4.046250in}{0.525000in}%
\pgfsys@useobject{currentmarker}{}%
\end{pgfscope}%
\end{pgfscope}%
\begin{pgfscope}%
\definecolor{textcolor}{rgb}{0.000000,0.000000,0.000000}%
\pgfsetstrokecolor{textcolor}%
\pgfsetfillcolor{textcolor}%
\pgftext[x=4.046250in,y=0.427778in,,top]{\color{textcolor}\rmfamily\fontsize{9.000000}{10.800000}\selectfont 200}%
\end{pgfscope}%
\begin{pgfscope}%
\definecolor{textcolor}{rgb}{0.000000,0.000000,0.000000}%
\pgfsetstrokecolor{textcolor}%
\pgfsetfillcolor{textcolor}%
\pgftext[x=2.386250in,y=0.251251in,,top]{\color{textcolor}\rmfamily\fontsize{9.000000}{10.800000}\selectfont \(\displaystyle L\) [km]}%
\end{pgfscope}%
\begin{pgfscope}%
\pgfsetbuttcap%
\pgfsetroundjoin%
\definecolor{currentfill}{rgb}{0.000000,0.000000,0.000000}%
\pgfsetfillcolor{currentfill}%
\pgfsetlinewidth{0.803000pt}%
\definecolor{currentstroke}{rgb}{0.000000,0.000000,0.000000}%
\pgfsetstrokecolor{currentstroke}%
\pgfsetdash{}{0pt}%
\pgfsys@defobject{currentmarker}{\pgfqpoint{-0.048611in}{0.000000in}}{\pgfqpoint{-0.000000in}{0.000000in}}{%
\pgfpathmoveto{\pgfqpoint{-0.000000in}{0.000000in}}%
\pgfpathlineto{\pgfqpoint{-0.048611in}{0.000000in}}%
\pgfusepath{stroke,fill}%
}%
\begin{pgfscope}%
\pgfsys@transformshift{0.726250in}{0.601519in}%
\pgfsys@useobject{currentmarker}{}%
\end{pgfscope}%
\end{pgfscope}%
\begin{pgfscope}%
\definecolor{textcolor}{rgb}{0.000000,0.000000,0.000000}%
\pgfsetstrokecolor{textcolor}%
\pgfsetfillcolor{textcolor}%
\pgftext[x=0.430236in, y=0.554034in, left, base]{\color{textcolor}\rmfamily\fontsize{9.000000}{10.800000}\selectfont 0.0}%
\end{pgfscope}%
\begin{pgfscope}%
\pgfsetbuttcap%
\pgfsetroundjoin%
\definecolor{currentfill}{rgb}{0.000000,0.000000,0.000000}%
\pgfsetfillcolor{currentfill}%
\pgfsetlinewidth{0.803000pt}%
\definecolor{currentstroke}{rgb}{0.000000,0.000000,0.000000}%
\pgfsetstrokecolor{currentstroke}%
\pgfsetdash{}{0pt}%
\pgfsys@defobject{currentmarker}{\pgfqpoint{-0.048611in}{0.000000in}}{\pgfqpoint{-0.000000in}{0.000000in}}{%
\pgfpathmoveto{\pgfqpoint{-0.000000in}{0.000000in}}%
\pgfpathlineto{\pgfqpoint{-0.048611in}{0.000000in}}%
\pgfusepath{stroke,fill}%
}%
\begin{pgfscope}%
\pgfsys@transformshift{0.726250in}{0.963715in}%
\pgfsys@useobject{currentmarker}{}%
\end{pgfscope}%
\end{pgfscope}%
\begin{pgfscope}%
\definecolor{textcolor}{rgb}{0.000000,0.000000,0.000000}%
\pgfsetstrokecolor{textcolor}%
\pgfsetfillcolor{textcolor}%
\pgftext[x=0.430236in, y=0.916230in, left, base]{\color{textcolor}\rmfamily\fontsize{9.000000}{10.800000}\selectfont 0.2}%
\end{pgfscope}%
\begin{pgfscope}%
\pgfsetbuttcap%
\pgfsetroundjoin%
\definecolor{currentfill}{rgb}{0.000000,0.000000,0.000000}%
\pgfsetfillcolor{currentfill}%
\pgfsetlinewidth{0.803000pt}%
\definecolor{currentstroke}{rgb}{0.000000,0.000000,0.000000}%
\pgfsetstrokecolor{currentstroke}%
\pgfsetdash{}{0pt}%
\pgfsys@defobject{currentmarker}{\pgfqpoint{-0.048611in}{0.000000in}}{\pgfqpoint{-0.000000in}{0.000000in}}{%
\pgfpathmoveto{\pgfqpoint{-0.000000in}{0.000000in}}%
\pgfpathlineto{\pgfqpoint{-0.048611in}{0.000000in}}%
\pgfusepath{stroke,fill}%
}%
\begin{pgfscope}%
\pgfsys@transformshift{0.726250in}{1.325912in}%
\pgfsys@useobject{currentmarker}{}%
\end{pgfscope}%
\end{pgfscope}%
\begin{pgfscope}%
\definecolor{textcolor}{rgb}{0.000000,0.000000,0.000000}%
\pgfsetstrokecolor{textcolor}%
\pgfsetfillcolor{textcolor}%
\pgftext[x=0.430236in, y=1.278426in, left, base]{\color{textcolor}\rmfamily\fontsize{9.000000}{10.800000}\selectfont 0.4}%
\end{pgfscope}%
\begin{pgfscope}%
\pgfsetbuttcap%
\pgfsetroundjoin%
\definecolor{currentfill}{rgb}{0.000000,0.000000,0.000000}%
\pgfsetfillcolor{currentfill}%
\pgfsetlinewidth{0.803000pt}%
\definecolor{currentstroke}{rgb}{0.000000,0.000000,0.000000}%
\pgfsetstrokecolor{currentstroke}%
\pgfsetdash{}{0pt}%
\pgfsys@defobject{currentmarker}{\pgfqpoint{-0.048611in}{0.000000in}}{\pgfqpoint{-0.000000in}{0.000000in}}{%
\pgfpathmoveto{\pgfqpoint{-0.000000in}{0.000000in}}%
\pgfpathlineto{\pgfqpoint{-0.048611in}{0.000000in}}%
\pgfusepath{stroke,fill}%
}%
\begin{pgfscope}%
\pgfsys@transformshift{0.726250in}{1.688108in}%
\pgfsys@useobject{currentmarker}{}%
\end{pgfscope}%
\end{pgfscope}%
\begin{pgfscope}%
\definecolor{textcolor}{rgb}{0.000000,0.000000,0.000000}%
\pgfsetstrokecolor{textcolor}%
\pgfsetfillcolor{textcolor}%
\pgftext[x=0.430236in, y=1.640622in, left, base]{\color{textcolor}\rmfamily\fontsize{9.000000}{10.800000}\selectfont 0.6}%
\end{pgfscope}%
\begin{pgfscope}%
\pgfsetbuttcap%
\pgfsetroundjoin%
\definecolor{currentfill}{rgb}{0.000000,0.000000,0.000000}%
\pgfsetfillcolor{currentfill}%
\pgfsetlinewidth{0.803000pt}%
\definecolor{currentstroke}{rgb}{0.000000,0.000000,0.000000}%
\pgfsetstrokecolor{currentstroke}%
\pgfsetdash{}{0pt}%
\pgfsys@defobject{currentmarker}{\pgfqpoint{-0.048611in}{0.000000in}}{\pgfqpoint{-0.000000in}{0.000000in}}{%
\pgfpathmoveto{\pgfqpoint{-0.000000in}{0.000000in}}%
\pgfpathlineto{\pgfqpoint{-0.048611in}{0.000000in}}%
\pgfusepath{stroke,fill}%
}%
\begin{pgfscope}%
\pgfsys@transformshift{0.726250in}{2.050304in}%
\pgfsys@useobject{currentmarker}{}%
\end{pgfscope}%
\end{pgfscope}%
\begin{pgfscope}%
\definecolor{textcolor}{rgb}{0.000000,0.000000,0.000000}%
\pgfsetstrokecolor{textcolor}%
\pgfsetfillcolor{textcolor}%
\pgftext[x=0.430236in, y=2.002818in, left, base]{\color{textcolor}\rmfamily\fontsize{9.000000}{10.800000}\selectfont 0.8}%
\end{pgfscope}%
\begin{pgfscope}%
\pgfsetbuttcap%
\pgfsetroundjoin%
\definecolor{currentfill}{rgb}{0.000000,0.000000,0.000000}%
\pgfsetfillcolor{currentfill}%
\pgfsetlinewidth{0.803000pt}%
\definecolor{currentstroke}{rgb}{0.000000,0.000000,0.000000}%
\pgfsetstrokecolor{currentstroke}%
\pgfsetdash{}{0pt}%
\pgfsys@defobject{currentmarker}{\pgfqpoint{-0.048611in}{0.000000in}}{\pgfqpoint{-0.000000in}{0.000000in}}{%
\pgfpathmoveto{\pgfqpoint{-0.000000in}{0.000000in}}%
\pgfpathlineto{\pgfqpoint{-0.048611in}{0.000000in}}%
\pgfusepath{stroke,fill}%
}%
\begin{pgfscope}%
\pgfsys@transformshift{0.726250in}{2.412500in}%
\pgfsys@useobject{currentmarker}{}%
\end{pgfscope}%
\end{pgfscope}%
\begin{pgfscope}%
\definecolor{textcolor}{rgb}{0.000000,0.000000,0.000000}%
\pgfsetstrokecolor{textcolor}%
\pgfsetfillcolor{textcolor}%
\pgftext[x=0.430236in, y=2.365015in, left, base]{\color{textcolor}\rmfamily\fontsize{9.000000}{10.800000}\selectfont 1.0}%
\end{pgfscope}%
\begin{pgfscope}%
\definecolor{textcolor}{rgb}{0.000000,0.000000,0.000000}%
\pgfsetstrokecolor{textcolor}%
\pgfsetfillcolor{textcolor}%
\pgftext[x=0.374681in,y=1.468750in,,bottom,rotate=90.000000]{\color{textcolor}\rmfamily\fontsize{9.000000}{10.800000}\selectfont Oscillation probability}%
\end{pgfscope}%
\begin{pgfscope}%
\pgfpathrectangle{\pgfqpoint{0.726250in}{0.525000in}}{\pgfqpoint{3.320000in}{1.887500in}}%
\pgfusepath{clip}%
\pgfsetbuttcap%
\pgfsetroundjoin%
\pgfsetlinewidth{1.003750pt}%
\definecolor{currentstroke}{rgb}{0.501961,0.501961,0.501961}%
\pgfsetstrokecolor{currentstroke}%
\pgfsetdash{{3.700000pt}{1.600000pt}}{0.000000pt}%
\pgfpathmoveto{\pgfqpoint{0.726250in}{0.601519in}}%
\pgfpathlineto{\pgfqpoint{0.739583in}{0.610247in}}%
\pgfpathlineto{\pgfqpoint{0.752917in}{0.636229in}}%
\pgfpathlineto{\pgfqpoint{0.766250in}{0.678875in}}%
\pgfpathlineto{\pgfqpoint{0.779583in}{0.737211in}}%
\pgfpathlineto{\pgfqpoint{0.792917in}{0.809906in}}%
\pgfpathlineto{\pgfqpoint{0.806250in}{0.895303in}}%
\pgfpathlineto{\pgfqpoint{0.819583in}{0.991452in}}%
\pgfpathlineto{\pgfqpoint{0.846250in}{1.207043in}}%
\pgfpathlineto{\pgfqpoint{0.899583in}{1.660809in}}%
\pgfpathlineto{\pgfqpoint{0.912917in}{1.763843in}}%
\pgfpathlineto{\pgfqpoint{0.926250in}{1.857817in}}%
\pgfpathlineto{\pgfqpoint{0.939583in}{1.940590in}}%
\pgfpathlineto{\pgfqpoint{0.952917in}{2.010272in}}%
\pgfpathlineto{\pgfqpoint{0.966250in}{2.065274in}}%
\pgfpathlineto{\pgfqpoint{0.979583in}{2.104342in}}%
\pgfpathlineto{\pgfqpoint{0.992917in}{2.126583in}}%
\pgfpathlineto{\pgfqpoint{1.006250in}{2.131492in}}%
\pgfpathlineto{\pgfqpoint{1.019583in}{2.118956in}}%
\pgfpathlineto{\pgfqpoint{1.032917in}{2.089260in}}%
\pgfpathlineto{\pgfqpoint{1.046250in}{2.043083in}}%
\pgfpathlineto{\pgfqpoint{1.059583in}{1.981477in}}%
\pgfpathlineto{\pgfqpoint{1.072917in}{1.905848in}}%
\pgfpathlineto{\pgfqpoint{1.086250in}{1.817922in}}%
\pgfpathlineto{\pgfqpoint{1.112917in}{1.613432in}}%
\pgfpathlineto{\pgfqpoint{1.192917in}{0.948159in}}%
\pgfpathlineto{\pgfqpoint{1.206250in}{0.856461in}}%
\pgfpathlineto{\pgfqpoint{1.219583in}{0.776403in}}%
\pgfpathlineto{\pgfqpoint{1.232917in}{0.709811in}}%
\pgfpathlineto{\pgfqpoint{1.246250in}{0.658203in}}%
\pgfpathlineto{\pgfqpoint{1.259583in}{0.622756in}}%
\pgfpathlineto{\pgfqpoint{1.272917in}{0.604280in}}%
\pgfpathlineto{\pgfqpoint{1.286250in}{0.603195in}}%
\pgfpathlineto{\pgfqpoint{1.299583in}{0.619527in}}%
\pgfpathlineto{\pgfqpoint{1.312917in}{0.652903in}}%
\pgfpathlineto{\pgfqpoint{1.326250in}{0.702561in}}%
\pgfpathlineto{\pgfqpoint{1.339583in}{0.767369in}}%
\pgfpathlineto{\pgfqpoint{1.352917in}{0.845848in}}%
\pgfpathlineto{\pgfqpoint{1.366250in}{0.936209in}}%
\pgfpathlineto{\pgfqpoint{1.392917in}{1.144106in}}%
\pgfpathlineto{\pgfqpoint{1.459583in}{1.706898in}}%
\pgfpathlineto{\pgfqpoint{1.472917in}{1.806242in}}%
\pgfpathlineto{\pgfqpoint{1.486250in}{1.895561in}}%
\pgfpathlineto{\pgfqpoint{1.499583in}{1.972817in}}%
\pgfpathlineto{\pgfqpoint{1.512917in}{2.036247in}}%
\pgfpathlineto{\pgfqpoint{1.526250in}{2.084405in}}%
\pgfpathlineto{\pgfqpoint{1.539583in}{2.116191in}}%
\pgfpathlineto{\pgfqpoint{1.552917in}{2.130882in}}%
\pgfpathlineto{\pgfqpoint{1.566250in}{2.128141in}}%
\pgfpathlineto{\pgfqpoint{1.579583in}{2.108032in}}%
\pgfpathlineto{\pgfqpoint{1.592917in}{2.071013in}}%
\pgfpathlineto{\pgfqpoint{1.606250in}{2.017928in}}%
\pgfpathlineto{\pgfqpoint{1.619583in}{1.949989in}}%
\pgfpathlineto{\pgfqpoint{1.632917in}{1.868745in}}%
\pgfpathlineto{\pgfqpoint{1.646250in}{1.776049in}}%
\pgfpathlineto{\pgfqpoint{1.672917in}{1.564974in}}%
\pgfpathlineto{\pgfqpoint{1.739583in}{1.004035in}}%
\pgfpathlineto{\pgfqpoint{1.752917in}{0.906699in}}%
\pgfpathlineto{\pgfqpoint{1.766250in}{0.819856in}}%
\pgfpathlineto{\pgfqpoint{1.779583in}{0.745487in}}%
\pgfpathlineto{\pgfqpoint{1.792917in}{0.685289in}}%
\pgfpathlineto{\pgfqpoint{1.806250in}{0.640635in}}%
\pgfpathlineto{\pgfqpoint{1.819583in}{0.612543in}}%
\pgfpathlineto{\pgfqpoint{1.832917in}{0.601654in}}%
\pgfpathlineto{\pgfqpoint{1.846250in}{0.608217in}}%
\pgfpathlineto{\pgfqpoint{1.859583in}{0.632081in}}%
\pgfpathlineto{\pgfqpoint{1.872917in}{0.672704in}}%
\pgfpathlineto{\pgfqpoint{1.886250in}{0.729157in}}%
\pgfpathlineto{\pgfqpoint{1.899583in}{0.800153in}}%
\pgfpathlineto{\pgfqpoint{1.912917in}{0.884072in}}%
\pgfpathlineto{\pgfqpoint{1.926250in}{0.979002in}}%
\pgfpathlineto{\pgfqpoint{1.952917in}{1.193025in}}%
\pgfpathlineto{\pgfqpoint{2.006250in}{1.647498in}}%
\pgfpathlineto{\pgfqpoint{2.019583in}{1.751496in}}%
\pgfpathlineto{\pgfqpoint{2.032917in}{1.846717in}}%
\pgfpathlineto{\pgfqpoint{2.046250in}{1.930989in}}%
\pgfpathlineto{\pgfqpoint{2.059583in}{2.002389in}}%
\pgfpathlineto{\pgfqpoint{2.072917in}{2.059289in}}%
\pgfpathlineto{\pgfqpoint{2.086250in}{2.100391in}}%
\pgfpathlineto{\pgfqpoint{2.099583in}{2.124758in}}%
\pgfpathlineto{\pgfqpoint{2.112917in}{2.131833in}}%
\pgfpathlineto{\pgfqpoint{2.126250in}{2.121455in}}%
\pgfpathlineto{\pgfqpoint{2.139583in}{2.093861in}}%
\pgfpathlineto{\pgfqpoint{2.152917in}{2.049680in}}%
\pgfpathlineto{\pgfqpoint{2.166250in}{1.989921in}}%
\pgfpathlineto{\pgfqpoint{2.179583in}{1.915946in}}%
\pgfpathlineto{\pgfqpoint{2.192917in}{1.829442in}}%
\pgfpathlineto{\pgfqpoint{2.206250in}{1.732383in}}%
\pgfpathlineto{\pgfqpoint{2.232917in}{1.515647in}}%
\pgfpathlineto{\pgfqpoint{2.286250in}{1.062528in}}%
\pgfpathlineto{\pgfqpoint{2.299583in}{0.960256in}}%
\pgfpathlineto{\pgfqpoint{2.312917in}{0.867254in}}%
\pgfpathlineto{\pgfqpoint{2.326250in}{0.785646in}}%
\pgfpathlineto{\pgfqpoint{2.339583in}{0.717292in}}%
\pgfpathlineto{\pgfqpoint{2.352917in}{0.663753in}}%
\pgfpathlineto{\pgfqpoint{2.366250in}{0.626248in}}%
\pgfpathlineto{\pgfqpoint{2.379583in}{0.605633in}}%
\pgfpathlineto{\pgfqpoint{2.392917in}{0.602380in}}%
\pgfpathlineto{\pgfqpoint{2.406250in}{0.616561in}}%
\pgfpathlineto{\pgfqpoint{2.419583in}{0.647854in}}%
\pgfpathlineto{\pgfqpoint{2.432917in}{0.695545in}}%
\pgfpathlineto{\pgfqpoint{2.446250in}{0.758545in}}%
\pgfpathlineto{\pgfqpoint{2.459583in}{0.835419in}}%
\pgfpathlineto{\pgfqpoint{2.472917in}{0.924411in}}%
\pgfpathlineto{\pgfqpoint{2.499583in}{1.130404in}}%
\pgfpathlineto{\pgfqpoint{2.566250in}{1.693973in}}%
\pgfpathlineto{\pgfqpoint{2.579583in}{1.794408in}}%
\pgfpathlineto{\pgfqpoint{2.592917in}{1.885088in}}%
\pgfpathlineto{\pgfqpoint{2.606250in}{1.963943in}}%
\pgfpathlineto{\pgfqpoint{2.619583in}{2.029176in}}%
\pgfpathlineto{\pgfqpoint{2.632917in}{2.079297in}}%
\pgfpathlineto{\pgfqpoint{2.646250in}{2.113163in}}%
\pgfpathlineto{\pgfqpoint{2.659583in}{2.130003in}}%
\pgfpathlineto{\pgfqpoint{2.672917in}{2.129431in}}%
\pgfpathlineto{\pgfqpoint{2.686250in}{2.111461in}}%
\pgfpathlineto{\pgfqpoint{2.699583in}{2.076504in}}%
\pgfpathlineto{\pgfqpoint{2.712917in}{2.025355in}}%
\pgfpathlineto{\pgfqpoint{2.726250in}{1.959183in}}%
\pgfpathlineto{\pgfqpoint{2.739583in}{1.879496in}}%
\pgfpathlineto{\pgfqpoint{2.752917in}{1.788112in}}%
\pgfpathlineto{\pgfqpoint{2.779583in}{1.578811in}}%
\pgfpathlineto{\pgfqpoint{2.846250in}{1.016745in}}%
\pgfpathlineto{\pgfqpoint{2.859583in}{0.918257in}}%
\pgfpathlineto{\pgfqpoint{2.872917in}{0.829998in}}%
\pgfpathlineto{\pgfqpoint{2.886250in}{0.753982in}}%
\pgfpathlineto{\pgfqpoint{2.899583in}{0.691943in}}%
\pgfpathlineto{\pgfqpoint{2.912917in}{0.645296in}}%
\pgfpathlineto{\pgfqpoint{2.926250in}{0.615105in}}%
\pgfpathlineto{\pgfqpoint{2.939583in}{0.602058in}}%
\pgfpathlineto{\pgfqpoint{2.952917in}{0.606454in}}%
\pgfpathlineto{\pgfqpoint{2.966250in}{0.628192in}}%
\pgfpathlineto{\pgfqpoint{2.979583in}{0.666777in}}%
\pgfpathlineto{\pgfqpoint{2.992917in}{0.721327in}}%
\pgfpathlineto{\pgfqpoint{3.006250in}{0.790599in}}%
\pgfpathlineto{\pgfqpoint{3.019583in}{0.873012in}}%
\pgfpathlineto{\pgfqpoint{3.032917in}{0.966688in}}%
\pgfpathlineto{\pgfqpoint{3.059583in}{1.179068in}}%
\pgfpathlineto{\pgfqpoint{3.126250in}{1.739014in}}%
\pgfpathlineto{\pgfqpoint{3.139583in}{1.835447in}}%
\pgfpathlineto{\pgfqpoint{3.152917in}{1.921188in}}%
\pgfpathlineto{\pgfqpoint{3.166250in}{1.994282in}}%
\pgfpathlineto{\pgfqpoint{3.179583in}{2.053060in}}%
\pgfpathlineto{\pgfqpoint{3.192917in}{2.096182in}}%
\pgfpathlineto{\pgfqpoint{3.206250in}{2.122665in}}%
\pgfpathlineto{\pgfqpoint{3.219583in}{2.131904in}}%
\pgfpathlineto{\pgfqpoint{3.232917in}{2.123688in}}%
\pgfpathlineto{\pgfqpoint{3.246250in}{2.098205in}}%
\pgfpathlineto{\pgfqpoint{3.259583in}{2.056037in}}%
\pgfpathlineto{\pgfqpoint{3.272917in}{1.998145in}}%
\pgfpathlineto{\pgfqpoint{3.286250in}{1.925849in}}%
\pgfpathlineto{\pgfqpoint{3.299583in}{1.840799in}}%
\pgfpathlineto{\pgfqpoint{3.312917in}{1.744935in}}%
\pgfpathlineto{\pgfqpoint{3.339583in}{1.529708in}}%
\pgfpathlineto{\pgfqpoint{3.392917in}{1.075761in}}%
\pgfpathlineto{\pgfqpoint{3.406250in}{0.972496in}}%
\pgfpathlineto{\pgfqpoint{3.419583in}{0.878223in}}%
\pgfpathlineto{\pgfqpoint{3.432917in}{0.795093in}}%
\pgfpathlineto{\pgfqpoint{3.446250in}{0.725003in}}%
\pgfpathlineto{\pgfqpoint{3.459583in}{0.669550in}}%
\pgfpathlineto{\pgfqpoint{3.472917in}{0.630000in}}%
\pgfpathlineto{\pgfqpoint{3.486250in}{0.607255in}}%
\pgfpathlineto{\pgfqpoint{3.499583in}{0.601833in}}%
\pgfpathlineto{\pgfqpoint{3.512917in}{0.613860in}}%
\pgfpathlineto{\pgfqpoint{3.526250in}{0.643059in}}%
\pgfpathlineto{\pgfqpoint{3.539583in}{0.688765in}}%
\pgfpathlineto{\pgfqpoint{3.552917in}{0.749936in}}%
\pgfpathlineto{\pgfqpoint{3.566250in}{0.825176in}}%
\pgfpathlineto{\pgfqpoint{3.579583in}{0.912769in}}%
\pgfpathlineto{\pgfqpoint{3.606250in}{1.116785in}}%
\pgfpathlineto{\pgfqpoint{3.686250in}{1.782424in}}%
\pgfpathlineto{\pgfqpoint{3.699583in}{1.874432in}}%
\pgfpathlineto{\pgfqpoint{3.712917in}{1.954859in}}%
\pgfpathlineto{\pgfqpoint{3.726250in}{2.021871in}}%
\pgfpathlineto{\pgfqpoint{3.739583in}{2.073937in}}%
\pgfpathlineto{\pgfqpoint{3.752917in}{2.109872in}}%
\pgfpathlineto{\pgfqpoint{3.766250in}{2.128854in}}%
\pgfpathlineto{\pgfqpoint{3.779583in}{2.130452in}}%
\pgfpathlineto{\pgfqpoint{3.792917in}{2.114628in}}%
\pgfpathlineto{\pgfqpoint{3.806250in}{2.081744in}}%
\pgfpathlineto{\pgfqpoint{3.819583in}{2.032550in}}%
\pgfpathlineto{\pgfqpoint{3.832917in}{1.968168in}}%
\pgfpathlineto{\pgfqpoint{3.846250in}{1.890066in}}%
\pgfpathlineto{\pgfqpoint{3.859583in}{1.800026in}}%
\pgfpathlineto{\pgfqpoint{3.886250in}{1.592573in}}%
\pgfpathlineto{\pgfqpoint{3.952917in}{1.029579in}}%
\pgfpathlineto{\pgfqpoint{3.966250in}{0.929973in}}%
\pgfpathlineto{\pgfqpoint{3.979583in}{0.840330in}}%
\pgfpathlineto{\pgfqpoint{3.992917in}{0.762693in}}%
\pgfpathlineto{\pgfqpoint{4.006250in}{0.698835in}}%
\pgfpathlineto{\pgfqpoint{4.019583in}{0.650211in}}%
\pgfpathlineto{\pgfqpoint{4.032917in}{0.617931in}}%
\pgfpathlineto{\pgfqpoint{4.046250in}{0.602732in}}%
\pgfpathlineto{\pgfqpoint{4.046250in}{0.602732in}}%
\pgfusepath{stroke}%
\end{pgfscope}%
\begin{pgfscope}%
\pgfpathrectangle{\pgfqpoint{0.726250in}{0.525000in}}{\pgfqpoint{3.320000in}{1.887500in}}%
\pgfusepath{clip}%
\pgfsetrectcap%
\pgfsetroundjoin%
\pgfsetlinewidth{1.003750pt}%
\definecolor{currentstroke}{rgb}{0.000000,0.000000,0.000000}%
\pgfsetstrokecolor{currentstroke}%
\pgfsetdash{}{0pt}%
\pgfpathmoveto{\pgfqpoint{0.726250in}{0.601519in}}%
\pgfpathlineto{\pgfqpoint{0.739583in}{0.610521in}}%
\pgfpathlineto{\pgfqpoint{0.752917in}{0.637306in}}%
\pgfpathlineto{\pgfqpoint{0.766250in}{0.681214in}}%
\pgfpathlineto{\pgfqpoint{0.779583in}{0.741170in}}%
\pgfpathlineto{\pgfqpoint{0.792917in}{0.815702in}}%
\pgfpathlineto{\pgfqpoint{0.806250in}{0.902986in}}%
\pgfpathlineto{\pgfqpoint{0.819583in}{1.000889in}}%
\pgfpathlineto{\pgfqpoint{0.846250in}{1.218824in}}%
\pgfpathlineto{\pgfqpoint{0.886250in}{1.560895in}}%
\pgfpathlineto{\pgfqpoint{0.912917in}{1.767434in}}%
\pgfpathlineto{\pgfqpoint{0.926250in}{1.856750in}}%
\pgfpathlineto{\pgfqpoint{0.939583in}{1.933939in}}%
\pgfpathlineto{\pgfqpoint{0.952917in}{1.997260in}}%
\pgfpathlineto{\pgfqpoint{0.966250in}{2.045326in}}%
\pgfpathlineto{\pgfqpoint{0.979583in}{2.077137in}}%
\pgfpathlineto{\pgfqpoint{0.992917in}{2.092095in}}%
\pgfpathlineto{\pgfqpoint{1.006250in}{2.090020in}}%
\pgfpathlineto{\pgfqpoint{1.019583in}{2.071146in}}%
\pgfpathlineto{\pgfqpoint{1.032917in}{2.036113in}}%
\pgfpathlineto{\pgfqpoint{1.046250in}{1.985943in}}%
\pgfpathlineto{\pgfqpoint{1.059583in}{1.922012in}}%
\pgfpathlineto{\pgfqpoint{1.072917in}{1.846011in}}%
\pgfpathlineto{\pgfqpoint{1.086250in}{1.759896in}}%
\pgfpathlineto{\pgfqpoint{1.112917in}{1.566182in}}%
\pgfpathlineto{\pgfqpoint{1.166250in}{1.160297in}}%
\pgfpathlineto{\pgfqpoint{1.179583in}{1.068918in}}%
\pgfpathlineto{\pgfqpoint{1.192917in}{0.985893in}}%
\pgfpathlineto{\pgfqpoint{1.206250in}{0.913018in}}%
\pgfpathlineto{\pgfqpoint{1.219583in}{0.851812in}}%
\pgfpathlineto{\pgfqpoint{1.232917in}{0.803484in}}%
\pgfpathlineto{\pgfqpoint{1.246250in}{0.768917in}}%
\pgfpathlineto{\pgfqpoint{1.259583in}{0.748645in}}%
\pgfpathlineto{\pgfqpoint{1.272917in}{0.742854in}}%
\pgfpathlineto{\pgfqpoint{1.286250in}{0.751383in}}%
\pgfpathlineto{\pgfqpoint{1.299583in}{0.773735in}}%
\pgfpathlineto{\pgfqpoint{1.312917in}{0.809099in}}%
\pgfpathlineto{\pgfqpoint{1.326250in}{0.856372in}}%
\pgfpathlineto{\pgfqpoint{1.339583in}{0.914201in}}%
\pgfpathlineto{\pgfqpoint{1.352917in}{0.981011in}}%
\pgfpathlineto{\pgfqpoint{1.379583in}{1.134472in}}%
\pgfpathlineto{\pgfqpoint{1.446250in}{1.543626in}}%
\pgfpathlineto{\pgfqpoint{1.459583in}{1.614810in}}%
\pgfpathlineto{\pgfqpoint{1.472917in}{1.678653in}}%
\pgfpathlineto{\pgfqpoint{1.486250in}{1.733898in}}%
\pgfpathlineto{\pgfqpoint{1.499583in}{1.779525in}}%
\pgfpathlineto{\pgfqpoint{1.512917in}{1.814762in}}%
\pgfpathlineto{\pgfqpoint{1.526250in}{1.839101in}}%
\pgfpathlineto{\pgfqpoint{1.539583in}{1.852299in}}%
\pgfpathlineto{\pgfqpoint{1.552917in}{1.854375in}}%
\pgfpathlineto{\pgfqpoint{1.566250in}{1.845605in}}%
\pgfpathlineto{\pgfqpoint{1.579583in}{1.826501in}}%
\pgfpathlineto{\pgfqpoint{1.592917in}{1.797799in}}%
\pgfpathlineto{\pgfqpoint{1.606250in}{1.760429in}}%
\pgfpathlineto{\pgfqpoint{1.619583in}{1.715490in}}%
\pgfpathlineto{\pgfqpoint{1.632917in}{1.664218in}}%
\pgfpathlineto{\pgfqpoint{1.659583in}{1.548102in}}%
\pgfpathlineto{\pgfqpoint{1.712917in}{1.301549in}}%
\pgfpathlineto{\pgfqpoint{1.739583in}{1.192829in}}%
\pgfpathlineto{\pgfqpoint{1.752917in}{1.146152in}}%
\pgfpathlineto{\pgfqpoint{1.766250in}{1.105770in}}%
\pgfpathlineto{\pgfqpoint{1.779583in}{1.072354in}}%
\pgfpathlineto{\pgfqpoint{1.792917in}{1.046397in}}%
\pgfpathlineto{\pgfqpoint{1.806250in}{1.028205in}}%
\pgfpathlineto{\pgfqpoint{1.819583in}{1.017903in}}%
\pgfpathlineto{\pgfqpoint{1.832917in}{1.015437in}}%
\pgfpathlineto{\pgfqpoint{1.846250in}{1.020581in}}%
\pgfpathlineto{\pgfqpoint{1.859583in}{1.032949in}}%
\pgfpathlineto{\pgfqpoint{1.872917in}{1.052011in}}%
\pgfpathlineto{\pgfqpoint{1.886250in}{1.077109in}}%
\pgfpathlineto{\pgfqpoint{1.899583in}{1.107478in}}%
\pgfpathlineto{\pgfqpoint{1.912917in}{1.142267in}}%
\pgfpathlineto{\pgfqpoint{1.939583in}{1.221407in}}%
\pgfpathlineto{\pgfqpoint{2.006250in}{1.430379in}}%
\pgfpathlineto{\pgfqpoint{2.019583in}{1.466929in}}%
\pgfpathlineto{\pgfqpoint{2.032917in}{1.499985in}}%
\pgfpathlineto{\pgfqpoint{2.046250in}{1.528983in}}%
\pgfpathlineto{\pgfqpoint{2.059583in}{1.553471in}}%
\pgfpathlineto{\pgfqpoint{2.072917in}{1.573113in}}%
\pgfpathlineto{\pgfqpoint{2.086250in}{1.587690in}}%
\pgfpathlineto{\pgfqpoint{2.099583in}{1.597100in}}%
\pgfpathlineto{\pgfqpoint{2.112917in}{1.601359in}}%
\pgfpathlineto{\pgfqpoint{2.126250in}{1.600587in}}%
\pgfpathlineto{\pgfqpoint{2.139583in}{1.595010in}}%
\pgfpathlineto{\pgfqpoint{2.152917in}{1.584944in}}%
\pgfpathlineto{\pgfqpoint{2.166250in}{1.570788in}}%
\pgfpathlineto{\pgfqpoint{2.179583in}{1.553010in}}%
\pgfpathlineto{\pgfqpoint{2.192917in}{1.532132in}}%
\pgfpathlineto{\pgfqpoint{2.219583in}{1.483373in}}%
\pgfpathlineto{\pgfqpoint{2.299583in}{1.324015in}}%
\pgfpathlineto{\pgfqpoint{2.312917in}{1.301331in}}%
\pgfpathlineto{\pgfqpoint{2.326250in}{1.280947in}}%
\pgfpathlineto{\pgfqpoint{2.339583in}{1.263177in}}%
\pgfpathlineto{\pgfqpoint{2.352917in}{1.248267in}}%
\pgfpathlineto{\pgfqpoint{2.366250in}{1.236392in}}%
\pgfpathlineto{\pgfqpoint{2.379583in}{1.227656in}}%
\pgfpathlineto{\pgfqpoint{2.392917in}{1.222095in}}%
\pgfpathlineto{\pgfqpoint{2.406250in}{1.219676in}}%
\pgfpathlineto{\pgfqpoint{2.419583in}{1.220305in}}%
\pgfpathlineto{\pgfqpoint{2.432917in}{1.223829in}}%
\pgfpathlineto{\pgfqpoint{2.446250in}{1.230044in}}%
\pgfpathlineto{\pgfqpoint{2.459583in}{1.238699in}}%
\pgfpathlineto{\pgfqpoint{2.472917in}{1.249507in}}%
\pgfpathlineto{\pgfqpoint{2.499583in}{1.276290in}}%
\pgfpathlineto{\pgfqpoint{2.526250in}{1.307636in}}%
\pgfpathlineto{\pgfqpoint{2.579583in}{1.372742in}}%
\pgfpathlineto{\pgfqpoint{2.606250in}{1.401410in}}%
\pgfpathlineto{\pgfqpoint{2.632917in}{1.424812in}}%
\pgfpathlineto{\pgfqpoint{2.646250in}{1.434110in}}%
\pgfpathlineto{\pgfqpoint{2.659583in}{1.441656in}}%
\pgfpathlineto{\pgfqpoint{2.672917in}{1.447387in}}%
\pgfpathlineto{\pgfqpoint{2.686250in}{1.451281in}}%
\pgfpathlineto{\pgfqpoint{2.699583in}{1.453351in}}%
\pgfpathlineto{\pgfqpoint{2.712917in}{1.453648in}}%
\pgfpathlineto{\pgfqpoint{2.726250in}{1.452253in}}%
\pgfpathlineto{\pgfqpoint{2.739583in}{1.449278in}}%
\pgfpathlineto{\pgfqpoint{2.752917in}{1.444860in}}%
\pgfpathlineto{\pgfqpoint{2.779583in}{1.432341in}}%
\pgfpathlineto{\pgfqpoint{2.806250in}{1.416140in}}%
\pgfpathlineto{\pgfqpoint{2.859583in}{1.378998in}}%
\pgfpathlineto{\pgfqpoint{2.899583in}{1.352876in}}%
\pgfpathlineto{\pgfqpoint{2.926250in}{1.338584in}}%
\pgfpathlineto{\pgfqpoint{2.952917in}{1.327754in}}%
\pgfpathlineto{\pgfqpoint{2.979583in}{1.320833in}}%
\pgfpathlineto{\pgfqpoint{3.006250in}{1.317931in}}%
\pgfpathlineto{\pgfqpoint{3.032917in}{1.318851in}}%
\pgfpathlineto{\pgfqpoint{3.059583in}{1.323134in}}%
\pgfpathlineto{\pgfqpoint{3.086250in}{1.330121in}}%
\pgfpathlineto{\pgfqpoint{3.126250in}{1.343937in}}%
\pgfpathlineto{\pgfqpoint{3.206250in}{1.373400in}}%
\pgfpathlineto{\pgfqpoint{3.232917in}{1.381181in}}%
\pgfpathlineto{\pgfqpoint{3.259583in}{1.387136in}}%
\pgfpathlineto{\pgfqpoint{3.286250in}{1.391035in}}%
\pgfpathlineto{\pgfqpoint{3.312917in}{1.392822in}}%
\pgfpathlineto{\pgfqpoint{3.339583in}{1.392595in}}%
\pgfpathlineto{\pgfqpoint{3.366250in}{1.390586in}}%
\pgfpathlineto{\pgfqpoint{3.406250in}{1.384979in}}%
\pgfpathlineto{\pgfqpoint{3.472917in}{1.372163in}}%
\pgfpathlineto{\pgfqpoint{3.526250in}{1.362425in}}%
\pgfpathlineto{\pgfqpoint{3.566250in}{1.357054in}}%
\pgfpathlineto{\pgfqpoint{3.606250in}{1.353957in}}%
\pgfpathlineto{\pgfqpoint{3.646250in}{1.353229in}}%
\pgfpathlineto{\pgfqpoint{3.686250in}{1.354593in}}%
\pgfpathlineto{\pgfqpoint{3.739583in}{1.358713in}}%
\pgfpathlineto{\pgfqpoint{3.872917in}{1.370693in}}%
\pgfpathlineto{\pgfqpoint{3.926250in}{1.373017in}}%
\pgfpathlineto{\pgfqpoint{3.979583in}{1.373315in}}%
\pgfpathlineto{\pgfqpoint{4.046250in}{1.371426in}}%
\pgfpathlineto{\pgfqpoint{4.046250in}{1.371426in}}%
\pgfusepath{stroke}%
\end{pgfscope}%
\begin{pgfscope}%
\pgfpathrectangle{\pgfqpoint{0.726250in}{0.525000in}}{\pgfqpoint{3.320000in}{1.887500in}}%
\pgfusepath{clip}%
\pgfsetbuttcap%
\pgfsetroundjoin%
\pgfsetlinewidth{1.003750pt}%
\definecolor{currentstroke}{rgb}{0.000000,0.000000,0.000000}%
\pgfsetstrokecolor{currentstroke}%
\pgfsetdash{{3.700000pt}{1.600000pt}}{0.000000pt}%
\pgfpathmoveto{\pgfqpoint{0.726250in}{1.366715in}}%
\pgfpathlineto{\pgfqpoint{4.046250in}{1.366715in}}%
\pgfusepath{stroke}%
\end{pgfscope}%
\begin{pgfscope}%
\pgfsetrectcap%
\pgfsetmiterjoin%
\pgfsetlinewidth{1.003750pt}%
\definecolor{currentstroke}{rgb}{0.000000,0.000000,0.000000}%
\pgfsetstrokecolor{currentstroke}%
\pgfsetdash{}{0pt}%
\pgfpathmoveto{\pgfqpoint{0.726250in}{0.525000in}}%
\pgfpathlineto{\pgfqpoint{0.726250in}{2.412500in}}%
\pgfusepath{stroke}%
\end{pgfscope}%
\begin{pgfscope}%
\pgfsetrectcap%
\pgfsetmiterjoin%
\pgfsetlinewidth{1.003750pt}%
\definecolor{currentstroke}{rgb}{0.000000,0.000000,0.000000}%
\pgfsetstrokecolor{currentstroke}%
\pgfsetdash{}{0pt}%
\pgfpathmoveto{\pgfqpoint{4.046250in}{0.525000in}}%
\pgfpathlineto{\pgfqpoint{4.046250in}{2.412500in}}%
\pgfusepath{stroke}%
\end{pgfscope}%
\begin{pgfscope}%
\pgfsetrectcap%
\pgfsetmiterjoin%
\pgfsetlinewidth{1.003750pt}%
\definecolor{currentstroke}{rgb}{0.000000,0.000000,0.000000}%
\pgfsetstrokecolor{currentstroke}%
\pgfsetdash{}{0pt}%
\pgfpathmoveto{\pgfqpoint{0.726250in}{0.525000in}}%
\pgfpathlineto{\pgfqpoint{4.046250in}{0.525000in}}%
\pgfusepath{stroke}%
\end{pgfscope}%
\begin{pgfscope}%
\pgfsetrectcap%
\pgfsetmiterjoin%
\pgfsetlinewidth{1.003750pt}%
\definecolor{currentstroke}{rgb}{0.000000,0.000000,0.000000}%
\pgfsetstrokecolor{currentstroke}%
\pgfsetdash{}{0pt}%
\pgfpathmoveto{\pgfqpoint{0.726250in}{2.412500in}}%
\pgfpathlineto{\pgfqpoint{4.046250in}{2.412500in}}%
\pgfusepath{stroke}%
\end{pgfscope}%
\begin{pgfscope}%
\definecolor{textcolor}{rgb}{0.501961,0.501961,0.501961}%
\pgfsetstrokecolor{textcolor}%
\pgfsetfillcolor{textcolor}%
\pgftext[x=2.054250in,y=2.186127in,left,base]{\color{textcolor}\rmfamily\fontsize{9.000000}{10.800000}\selectfont Monochromatic 1 MeV beam}%
\end{pgfscope}%
\begin{pgfscope}%
\definecolor{textcolor}{rgb}{0.000000,0.000000,0.000000}%
\pgfsetstrokecolor{textcolor}%
\pgfsetfillcolor{textcolor}%
\pgftext[x=3.216250in,y=1.416461in,,base]{\color{textcolor}\rmfamily\fontsize{9.000000}{10.800000}\selectfont \(\displaystyle \sigma_E / E = 10\%\)}%
\end{pgfscope}%
\end{pgfpicture}%
\makeatother%
\endgroup%

  \caption{Oscillation probability $P(\nu_e \rightarrow \nu_\mu)$ as a function of
  the distance $L$ traveled for a monochromatic 1~MeV beam, and for a gaussian beam
  with the same average energy but with a standard deviation of 100~keV. The plot is
  made assuming $\theta = 33.41^\circ$ and $\Delta m^2 = 7.41 \times 10^{-5}$~eV$^2$
  that, as we shall see in a moment, are appropriate for solar neutrino oscillations.}
  \label{fig:sun_neutrino_oscillation}
\end{figure}

An interesting situation arises when $\varphi$ is large. While for a monochromatic
neutrino beam and fixed $L$ the oscillating behavior proceed as before (and the
neutrino undergoes many oscillation cycles), in the typical case where we are dealing
with a mixture of different energies or/and propagation lengths, what we see is
effectively an average of a (quickly oscillating) trigonometric function
\begin{align*}
  \ave{\sin^2\varphi(L, E)} = \frac{1}{2}.
\end{align*}
This is illustrated in figure~\ref{fig:sun_neutrino_oscillation} with values for the
oscillating parameters appropriate for solar neutrinos: while a monochromatic 1~MeV
beam oscillates coherently according to the basic formula, in the (largely academic)
case of a gaussian beam with an energy dispersion of $10\%$, everything averages
out in a few cycles. The oscillation probability at large distance is constant and
only driven by the mixing angle
\begin{align*}
  P(\nu_e \rightarrow \nu_\mu) = \frac{1}{2} \sin^2 2\theta,
\end{align*}
and the corresponding probability of survival for the $\nu_e$ is
\begin{align*}
  P(\nu_e \rightarrow \nu_e) = 1 - P(\nu_e \rightarrow \nu_\mu) = 1 - \frac{1}{2} \sin^2 2\theta.
\end{align*}

We are now starting to see on the horizon a viable solution to the solar neutrino
problem, in the form of the oscillation of some of the electron neutrinos produced
in the Sun to other flavors. If the mixing is maximal ($\theta \sim 45^\circ$) and
the $\Delta m^2$ is such that, combined with the Sun-Earth distance results into an
accumulated phase $\varphi \gg 1$, then it is not unconceivable that a faction of
$\nu_e \lesssim 50\%$ might have oscillated, on average, before reaching the Earth.
This would be in line with the Gallium experiments, but it cannnot possibly be the
end of the story, as the discrepancy between the different experiments are still
troubling, and incorporating any king of energy dependence into this seems hopeless
at this point. The last piece of the puzzle, as we shall see in a moment, comes from
the effect of the matter that neutrino have to traverse to the emerge over the Sun
surface.


\subsection{Neutrino oscillation in the presence of matter}

When neutrino propagate in matter there is an additional important piece of the
puzzle entering the game due to the fact that, as we have already seen, while all the
three neutrino species interact with the fermions the same way through NC forward
scattering (mediated by a $Z^0$), the electron neutrino can also be forward scattered by
ordinary electrons through CC interaction (mediated by a $W$) which implies a non
trivial phase difference with respect to the other two flavour eigenstates. This
additional process (which is customarily called MSW effect) can in fact be interpreted
in terms of an effective refractive index (or, equivalently, and effective mass)
which is different for $\nu_e$ with respect to $\nu_\mu$ and $\nu_\tau$.

Let us rewrite the time evolution of the mass eigenstates for the two-flavor case
\emph{in vacuum}\sidenote{Since in this section we shall deal with several different
versions of the Hamiltonian, we shall explicitely indicate with two subscript letters
(in this order) if the latter is in vacuum or in matter (v, m) and if the explicit
formula is in the mass of flavor basis (m, f).} as
\begin{align*}
  i\hbar \pdv{t} \begin{pmatrix}\nu_1\\ \nu_2 \end{pmatrix} =
  H_\text{v,m} \begin{pmatrix}\nu_1\\ \nu_2 \end{pmatrix}
\end{align*}
where the Hamiltonian, expanded in series in the ultrarelativistic limit reads
\begin{align*}
  H_\text{v,m} =
  \begin{pmatrix}
  E_1 & 0\\
  0 & E_2\\
  \end{pmatrix} \approx
  \begin{pmatrix}
  E + \frac{m_1^2 c^4}{2E} & 0\\
  0 & E + \frac{m_2^2 c^4}{2E}\\
  \end{pmatrix}
  %= E + \frac{c^4}{2E}
  %\begin{pmatrix}
  %m_1^2 & 0\\
  %0 & m_2^2\\
  %\end{pmatrix}
\end{align*}
Since any term proportional to the $2 \times 2$ identity is only causing a common
phase that is irrelevant for the oscillation, it is useful to decompose the hamiltonian
isolating the traceless component, which is the only one that really matters to us
\begin{align*}
  H_\text{v,m} & =
  \frac{1}{2}\text{Tr}(H_\text{v,m}) I  + \qty{H_\text{v,m} -
  \frac{1}{2}\text{Tr}(H_\text{v,m}) I} =\\
  & = \qty{E + \frac{(m_1^2 + m_2^2)c^4}{4E}} I +
  \frac{\Delta m^2}{4E}
  \begin{pmatrix}
    -1 & 0\\
    0 & +1\\
  \end{pmatrix}
\end{align*}
If we now switch to the base of the flavor eigenstates, the Hamiltonian becomes
\begin{align*}
  H_\text{v,f} = U H_\text{v,m} U^\dag = \frac{\Delta m^2}{4E}
  \begin{pmatrix}
   -\cos2\theta & \sin2\theta\\
   \sin2\theta & \cos2\theta\\
  \end{pmatrix}
\end{align*}
where we have omitted all the terms proportional to the diagonal.

All of this in vacuum. When neutrinos propagate through matter, $\nu_e$ CC interactions
give rise to an extra term\sidenote{Technically there is an extra minus sign that
applies to electron antineutrinos that we omit because it is irrelevant in this
context.} in the Hamiltonian, which only affects the $\nu_e$-$\nu_e$\sidenote{And,
for completeness, all neutrino flavors interact with matter via neutral currents,
but since this contribution is proportional to the identity, it only produces a
global phase, which is irrelevant for our argument.} matrix element
\begin{align*}
  H_\text{m,f} = H_\text{v,f} +
  \begin{pmatrix}
  \sqrt{2} G_F n_e & 0\\
  0 & 0\\
\end{pmatrix}
\end{align*}
where $n_e$ is the electron density in matter, and $G_F$ is the Fermi constant.
Also for this extra term we can subtract the trace, and write the Hamiltonian in
matter in the equivalent form
\begin{align}\label{eq:msw_hamiltonian_1}
  H_\text{m,f} = \frac{\Delta m^2}{4E}
  \begin{pmatrix}
   -\cos 2\theta + \zeta & \sin 2\theta\\
   \sin 2\theta & \cos 2\theta - \zeta\\
  \end{pmatrix}
  \quad\text{where}\quad
  \zeta = \frac{2\sqrt{2}G_F n_e E}{\Delta m^2}.
\end{align}
It is important to realize that $\zeta$ can be positive or negative, depending on
the sign of $\Delta m^2$. We shall come back to this in a second, but the implications
are far reaching as, unlike purely interferometric effects, \emph{matter effect are sensitive
to the sign of the difference of mass squared}.

If we now transform this back into the basis of the (original) mass eigenstates,
$H_\text{v}$ goes back into its diagonal form, but the additional term proportional
to $\zeta$ causes non diagonal terms
\begin{align*}
  H_\text{m,m} = U^\dag H_\text{m,f} U =
  \frac{\Delta m^2}{4E}
  \begin{pmatrix}
    -1  + \zeta \cos2\theta & \zeta \sin2\theta\\
    \zeta \sin2\theta & 1 - \zeta \cos2\theta\\
  \end{pmatrix},
\end{align*}
that is, the mass eigenstates in vacuum are no longer eigenstates of the new Hamiltonian
in presence of matter. At this point we \emph{could} diagonalize the new Hamiltonian
and then transform it back into the flavor basis, but since the mixing matrix with
two generation is bound to be a simple rotation, the most general form of the final
result will be
\begin{align}\label{eq:msw_hamiltonian_2}
  H_\text{m,f} = \frac{\Delta \tilde{m}^2}{4E}
  \begin{pmatrix}
   -\cos 2\tilde{\theta} & \sin 2\tilde{\theta}\\
   \sin 2\tilde{\theta} & \cos 2\tilde{\theta}\\
  \end{pmatrix},
\end{align}
and we all we have left to do is to calculate the new, effective $\Delta \tilde{m}^2$
and $\tilde{\theta}$ by just comparing~\eqref{eq:msw_hamiltonian_2} and~\eqref{eq:msw_hamiltonian_1}.
We have two independent equations
\begin{align*}
  \begin{cases}
    \Delta \tilde{m}^2 \cos 2\tilde{\theta} & = \Delta m^2 (\cos 2\theta - \zeta)\\
    \Delta \tilde{m}^2 \sin 2\tilde{\theta} & = \Delta m^2 \sin 2\theta,
  \end{cases}
\end{align*}
that are readily solved as
\begin{align}\label{eq:matter_deltam_theta}
  \Delta \tilde{m}^2 = C(\zeta) \Delta m^2
  \quad\text{and}\quad
  \sin 2\tilde{\theta} = \frac{\sin 2\theta}{C(\zeta)},
\end{align}
where
\begin{align}
  C(\zeta) = \sqrt{(\cos 2\theta - \zeta)^2 + \sin^2 2\theta}.
\end{align}

Let us take a step back and examine the results we have obtained. If the electron
density is constant, neutrino oscillations in matter have the same basic phenomenology
we have derived in vacuum, except for the (important) fact that the effective mass
difference and mixing angle are modified according to~\eqref{eq:matter_deltam_theta}.
Since $\lim_{\zeta \rightarrow 0} C(\zeta) = 1$, we recover the initial setup in vacuum for
$\zeta \ll 1$, or, equivalently, when the electron density is sufficiently
small\sidenote{The effect is energy-dependent, and more pronounced for larger energies.
If we plug in $\Delta m^2 = 7.41 \times 10^{-5}$~eV$^2$ and $E = 1$~MeV, we get that
the critical electron number density is $n_e \approx 2 \times 10^{26}$~cm$^{-3}$.}
\begin{align*}
  n_e \ll \frac{\Delta m^2}{2\sqrt{2}G_F E}.
\end{align*}
It is interesting to note that if $\sin 2\theta = 0$, then $\sin 2\tilde{\theta} = 0$
no matter what $C(\zeta)$ amounts to: for oscillations to happen in matter, the mixing
angle in vacuum must be non-vanishing. On the other hand, when $C(\zeta)$ is very large,
$\sin 2\tilde{\theta}$ is small no matter what the mixing angle is in vacuum, that is,
when the electron number density is very large, oscillations are suprressed
altogether.

Something even more interesting happens when $\zeta = \cos 2\theta$\sidenote{Note that,
assuming that $\theta$ lies in the first quadrant, this can only take place for
$\zeta > 0$, i.e., $m_2^2 > m_1^2$. Again, matter effect are sensitive to the sign
of $\Delta m^2$.}, in that $C = \sin 2\theta$, and therefore $\sin 2\tilde{\theta} = 1$,
no matter what the original value of $\sin 2\theta$ is. In other words, \emph{the
mixing becomes maximal even if the oscillation probability in vacuum is small, and
arbitrarily so}. This resonant condition happens for
\begin{align}\label{eq:msw_resonance_energy}
  E_\text{res} = \frac{\Delta m^2 \cos 2\theta}{2\sqrt{2}G_F n_e}.
\end{align}

Up to this point we have treated $n_e$ as a number, i.e., we have operated under
the implicit assumption that the electron number density of the medium traversed by
the neutrino was constant---which is hardly the case for the Sun. If, once again, we
jump for a second all the way to the end of the story and we plug into~\eqref{eq:msw_resonance_energy}
the values of $\Delta m^2$ and $\theta$ that are appropriate for Solar neutrinos,
we find that the resonance energy at the center of the Sun, where the electron density
is $\sim 6 \times 10^{25}$~cm$^{-3}$, is of the order of 1~MeV. This means that every
neutrino created at the center of the Sun with an energy greater than 1~MeV, as it
propagates toward less dense material, will at some point come across the resonance
condition.

\begin{marginfigure}
  %% Creator: Matplotlib, PGF backend
%%
%% To include the figure in your LaTeX document, write
%%   \input{<filename>.pgf}
%%
%% Make sure the required packages are loaded in your preamble
%%   \usepackage{pgf}
%%
%% Also ensure that all the required font packages are loaded; for instance,
%% the lmodern package is sometimes necessary when using math font.
%%   \usepackage{lmodern}
%%
%% Figures using additional raster images can only be included by \input if
%% they are in the same directory as the main LaTeX file. For loading figures
%% from other directories you can use the `import` package
%%   \usepackage{import}
%%
%% and then include the figures with
%%   \import{<path to file>}{<filename>.pgf}
%%
%% Matplotlib used the following preamble
%%   \usepackage{fontspec}
%%   \setmainfont{DejaVuSerif.ttf}[Path=\detokenize{/usr/share/matplotlib/mpl-data/fonts/ttf/}]
%%   \setsansfont{DejaVuSans.ttf}[Path=\detokenize{/usr/share/matplotlib/mpl-data/fonts/ttf/}]
%%   \setmonofont{DejaVuSansMono.ttf}[Path=\detokenize{/usr/share/matplotlib/mpl-data/fonts/ttf/}]
%%
\begingroup%
\makeatletter%
\begin{pgfpicture}%
\pgfpathrectangle{\pgfpointorigin}{\pgfqpoint{1.950000in}{2.500000in}}%
\pgfusepath{use as bounding box, clip}%
\begin{pgfscope}%
\pgfsetbuttcap%
\pgfsetmiterjoin%
\definecolor{currentfill}{rgb}{1.000000,1.000000,1.000000}%
\pgfsetfillcolor{currentfill}%
\pgfsetlinewidth{0.000000pt}%
\definecolor{currentstroke}{rgb}{1.000000,1.000000,1.000000}%
\pgfsetstrokecolor{currentstroke}%
\pgfsetdash{}{0pt}%
\pgfpathmoveto{\pgfqpoint{0.000000in}{0.000000in}}%
\pgfpathlineto{\pgfqpoint{1.950000in}{0.000000in}}%
\pgfpathlineto{\pgfqpoint{1.950000in}{2.500000in}}%
\pgfpathlineto{\pgfqpoint{0.000000in}{2.500000in}}%
\pgfpathlineto{\pgfqpoint{0.000000in}{0.000000in}}%
\pgfpathclose%
\pgfusepath{fill}%
\end{pgfscope}%
\begin{pgfscope}%
\pgfsetbuttcap%
\pgfsetmiterjoin%
\definecolor{currentfill}{rgb}{1.000000,1.000000,1.000000}%
\pgfsetfillcolor{currentfill}%
\pgfsetlinewidth{0.000000pt}%
\definecolor{currentstroke}{rgb}{0.000000,0.000000,0.000000}%
\pgfsetstrokecolor{currentstroke}%
\pgfsetstrokeopacity{0.000000}%
\pgfsetdash{}{0pt}%
\pgfpathmoveto{\pgfqpoint{0.726250in}{0.525000in}}%
\pgfpathlineto{\pgfqpoint{1.846250in}{0.525000in}}%
\pgfpathlineto{\pgfqpoint{1.846250in}{2.412500in}}%
\pgfpathlineto{\pgfqpoint{0.726250in}{2.412500in}}%
\pgfpathlineto{\pgfqpoint{0.726250in}{0.525000in}}%
\pgfpathclose%
\pgfusepath{fill}%
\end{pgfscope}%
\begin{pgfscope}%
\pgfpathrectangle{\pgfqpoint{0.726250in}{0.525000in}}{\pgfqpoint{1.120000in}{1.887500in}}%
\pgfusepath{clip}%
\pgfsetbuttcap%
\pgfsetroundjoin%
\pgfsetlinewidth{0.803000pt}%
\definecolor{currentstroke}{rgb}{0.752941,0.752941,0.752941}%
\pgfsetstrokecolor{currentstroke}%
\pgfsetdash{{2.960000pt}{1.280000pt}}{0.000000pt}%
\pgfpathmoveto{\pgfqpoint{0.726250in}{0.525000in}}%
\pgfpathlineto{\pgfqpoint{0.726250in}{2.412500in}}%
\pgfusepath{stroke}%
\end{pgfscope}%
\begin{pgfscope}%
\pgfsetbuttcap%
\pgfsetroundjoin%
\definecolor{currentfill}{rgb}{0.000000,0.000000,0.000000}%
\pgfsetfillcolor{currentfill}%
\pgfsetlinewidth{0.803000pt}%
\definecolor{currentstroke}{rgb}{0.000000,0.000000,0.000000}%
\pgfsetstrokecolor{currentstroke}%
\pgfsetdash{}{0pt}%
\pgfsys@defobject{currentmarker}{\pgfqpoint{0.000000in}{-0.048611in}}{\pgfqpoint{0.000000in}{0.000000in}}{%
\pgfpathmoveto{\pgfqpoint{0.000000in}{0.000000in}}%
\pgfpathlineto{\pgfqpoint{0.000000in}{-0.048611in}}%
\pgfusepath{stroke,fill}%
}%
\begin{pgfscope}%
\pgfsys@transformshift{0.726250in}{0.525000in}%
\pgfsys@useobject{currentmarker}{}%
\end{pgfscope}%
\end{pgfscope}%
\begin{pgfscope}%
\definecolor{textcolor}{rgb}{0.000000,0.000000,0.000000}%
\pgfsetstrokecolor{textcolor}%
\pgfsetfillcolor{textcolor}%
\pgftext[x=0.726250in,y=0.427778in,,top]{\color{textcolor}\rmfamily\fontsize{9.000000}{10.800000}\selectfont 0.0}%
\end{pgfscope}%
\begin{pgfscope}%
\pgfpathrectangle{\pgfqpoint{0.726250in}{0.525000in}}{\pgfqpoint{1.120000in}{1.887500in}}%
\pgfusepath{clip}%
\pgfsetbuttcap%
\pgfsetroundjoin%
\pgfsetlinewidth{0.803000pt}%
\definecolor{currentstroke}{rgb}{0.752941,0.752941,0.752941}%
\pgfsetstrokecolor{currentstroke}%
\pgfsetdash{{2.960000pt}{1.280000pt}}{0.000000pt}%
\pgfpathmoveto{\pgfqpoint{1.133523in}{0.525000in}}%
\pgfpathlineto{\pgfqpoint{1.133523in}{2.412500in}}%
\pgfusepath{stroke}%
\end{pgfscope}%
\begin{pgfscope}%
\pgfsetbuttcap%
\pgfsetroundjoin%
\definecolor{currentfill}{rgb}{0.000000,0.000000,0.000000}%
\pgfsetfillcolor{currentfill}%
\pgfsetlinewidth{0.803000pt}%
\definecolor{currentstroke}{rgb}{0.000000,0.000000,0.000000}%
\pgfsetstrokecolor{currentstroke}%
\pgfsetdash{}{0pt}%
\pgfsys@defobject{currentmarker}{\pgfqpoint{0.000000in}{-0.048611in}}{\pgfqpoint{0.000000in}{0.000000in}}{%
\pgfpathmoveto{\pgfqpoint{0.000000in}{0.000000in}}%
\pgfpathlineto{\pgfqpoint{0.000000in}{-0.048611in}}%
\pgfusepath{stroke,fill}%
}%
\begin{pgfscope}%
\pgfsys@transformshift{1.133523in}{0.525000in}%
\pgfsys@useobject{currentmarker}{}%
\end{pgfscope}%
\end{pgfscope}%
\begin{pgfscope}%
\definecolor{textcolor}{rgb}{0.000000,0.000000,0.000000}%
\pgfsetstrokecolor{textcolor}%
\pgfsetfillcolor{textcolor}%
\pgftext[x=1.133523in,y=0.427778in,,top]{\color{textcolor}\rmfamily\fontsize{9.000000}{10.800000}\selectfont 0.1}%
\end{pgfscope}%
\begin{pgfscope}%
\pgfpathrectangle{\pgfqpoint{0.726250in}{0.525000in}}{\pgfqpoint{1.120000in}{1.887500in}}%
\pgfusepath{clip}%
\pgfsetbuttcap%
\pgfsetroundjoin%
\pgfsetlinewidth{0.803000pt}%
\definecolor{currentstroke}{rgb}{0.752941,0.752941,0.752941}%
\pgfsetstrokecolor{currentstroke}%
\pgfsetdash{{2.960000pt}{1.280000pt}}{0.000000pt}%
\pgfpathmoveto{\pgfqpoint{1.540795in}{0.525000in}}%
\pgfpathlineto{\pgfqpoint{1.540795in}{2.412500in}}%
\pgfusepath{stroke}%
\end{pgfscope}%
\begin{pgfscope}%
\pgfsetbuttcap%
\pgfsetroundjoin%
\definecolor{currentfill}{rgb}{0.000000,0.000000,0.000000}%
\pgfsetfillcolor{currentfill}%
\pgfsetlinewidth{0.803000pt}%
\definecolor{currentstroke}{rgb}{0.000000,0.000000,0.000000}%
\pgfsetstrokecolor{currentstroke}%
\pgfsetdash{}{0pt}%
\pgfsys@defobject{currentmarker}{\pgfqpoint{0.000000in}{-0.048611in}}{\pgfqpoint{0.000000in}{0.000000in}}{%
\pgfpathmoveto{\pgfqpoint{0.000000in}{0.000000in}}%
\pgfpathlineto{\pgfqpoint{0.000000in}{-0.048611in}}%
\pgfusepath{stroke,fill}%
}%
\begin{pgfscope}%
\pgfsys@transformshift{1.540795in}{0.525000in}%
\pgfsys@useobject{currentmarker}{}%
\end{pgfscope}%
\end{pgfscope}%
\begin{pgfscope}%
\definecolor{textcolor}{rgb}{0.000000,0.000000,0.000000}%
\pgfsetstrokecolor{textcolor}%
\pgfsetfillcolor{textcolor}%
\pgftext[x=1.540795in,y=0.427778in,,top]{\color{textcolor}\rmfamily\fontsize{9.000000}{10.800000}\selectfont 0.2}%
\end{pgfscope}%
\begin{pgfscope}%
\definecolor{textcolor}{rgb}{0.000000,0.000000,0.000000}%
\pgfsetstrokecolor{textcolor}%
\pgfsetfillcolor{textcolor}%
\pgftext[x=1.286250in,y=0.251251in,,top]{\color{textcolor}\rmfamily\fontsize{9.000000}{10.800000}\selectfont \(\displaystyle r\) [\(\displaystyle R_\odot\)]}%
\end{pgfscope}%
\begin{pgfscope}%
\pgfpathrectangle{\pgfqpoint{0.726250in}{0.525000in}}{\pgfqpoint{1.120000in}{1.887500in}}%
\pgfusepath{clip}%
\pgfsetbuttcap%
\pgfsetroundjoin%
\pgfsetlinewidth{0.803000pt}%
\definecolor{currentstroke}{rgb}{0.752941,0.752941,0.752941}%
\pgfsetstrokecolor{currentstroke}%
\pgfsetdash{{2.960000pt}{1.280000pt}}{0.000000pt}%
\pgfpathmoveto{\pgfqpoint{0.726250in}{0.525000in}}%
\pgfpathlineto{\pgfqpoint{1.846250in}{0.525000in}}%
\pgfusepath{stroke}%
\end{pgfscope}%
\begin{pgfscope}%
\pgfsetbuttcap%
\pgfsetroundjoin%
\definecolor{currentfill}{rgb}{0.000000,0.000000,0.000000}%
\pgfsetfillcolor{currentfill}%
\pgfsetlinewidth{0.803000pt}%
\definecolor{currentstroke}{rgb}{0.000000,0.000000,0.000000}%
\pgfsetstrokecolor{currentstroke}%
\pgfsetdash{}{0pt}%
\pgfsys@defobject{currentmarker}{\pgfqpoint{-0.048611in}{0.000000in}}{\pgfqpoint{-0.000000in}{0.000000in}}{%
\pgfpathmoveto{\pgfqpoint{-0.000000in}{0.000000in}}%
\pgfpathlineto{\pgfqpoint{-0.048611in}{0.000000in}}%
\pgfusepath{stroke,fill}%
}%
\begin{pgfscope}%
\pgfsys@transformshift{0.726250in}{0.525000in}%
\pgfsys@useobject{currentmarker}{}%
\end{pgfscope}%
\end{pgfscope}%
\begin{pgfscope}%
\definecolor{textcolor}{rgb}{0.000000,0.000000,0.000000}%
\pgfsetstrokecolor{textcolor}%
\pgfsetfillcolor{textcolor}%
\pgftext[x=0.549499in, y=0.477515in, left, base]{\color{textcolor}\rmfamily\fontsize{9.000000}{10.800000}\selectfont 0}%
\end{pgfscope}%
\begin{pgfscope}%
\pgfpathrectangle{\pgfqpoint{0.726250in}{0.525000in}}{\pgfqpoint{1.120000in}{1.887500in}}%
\pgfusepath{clip}%
\pgfsetbuttcap%
\pgfsetroundjoin%
\pgfsetlinewidth{0.803000pt}%
\definecolor{currentstroke}{rgb}{0.752941,0.752941,0.752941}%
\pgfsetstrokecolor{currentstroke}%
\pgfsetdash{{2.960000pt}{1.280000pt}}{0.000000pt}%
\pgfpathmoveto{\pgfqpoint{0.726250in}{0.897640in}}%
\pgfpathlineto{\pgfqpoint{1.846250in}{0.897640in}}%
\pgfusepath{stroke}%
\end{pgfscope}%
\begin{pgfscope}%
\pgfsetbuttcap%
\pgfsetroundjoin%
\definecolor{currentfill}{rgb}{0.000000,0.000000,0.000000}%
\pgfsetfillcolor{currentfill}%
\pgfsetlinewidth{0.803000pt}%
\definecolor{currentstroke}{rgb}{0.000000,0.000000,0.000000}%
\pgfsetstrokecolor{currentstroke}%
\pgfsetdash{}{0pt}%
\pgfsys@defobject{currentmarker}{\pgfqpoint{-0.048611in}{0.000000in}}{\pgfqpoint{-0.000000in}{0.000000in}}{%
\pgfpathmoveto{\pgfqpoint{-0.000000in}{0.000000in}}%
\pgfpathlineto{\pgfqpoint{-0.048611in}{0.000000in}}%
\pgfusepath{stroke,fill}%
}%
\begin{pgfscope}%
\pgfsys@transformshift{0.726250in}{0.897640in}%
\pgfsys@useobject{currentmarker}{}%
\end{pgfscope}%
\end{pgfscope}%
\begin{pgfscope}%
\definecolor{textcolor}{rgb}{0.000000,0.000000,0.000000}%
\pgfsetstrokecolor{textcolor}%
\pgfsetfillcolor{textcolor}%
\pgftext[x=0.549499in, y=0.850154in, left, base]{\color{textcolor}\rmfamily\fontsize{9.000000}{10.800000}\selectfont 2}%
\end{pgfscope}%
\begin{pgfscope}%
\pgfpathrectangle{\pgfqpoint{0.726250in}{0.525000in}}{\pgfqpoint{1.120000in}{1.887500in}}%
\pgfusepath{clip}%
\pgfsetbuttcap%
\pgfsetroundjoin%
\pgfsetlinewidth{0.803000pt}%
\definecolor{currentstroke}{rgb}{0.752941,0.752941,0.752941}%
\pgfsetstrokecolor{currentstroke}%
\pgfsetdash{{2.960000pt}{1.280000pt}}{0.000000pt}%
\pgfpathmoveto{\pgfqpoint{0.726250in}{1.270279in}}%
\pgfpathlineto{\pgfqpoint{1.846250in}{1.270279in}}%
\pgfusepath{stroke}%
\end{pgfscope}%
\begin{pgfscope}%
\pgfsetbuttcap%
\pgfsetroundjoin%
\definecolor{currentfill}{rgb}{0.000000,0.000000,0.000000}%
\pgfsetfillcolor{currentfill}%
\pgfsetlinewidth{0.803000pt}%
\definecolor{currentstroke}{rgb}{0.000000,0.000000,0.000000}%
\pgfsetstrokecolor{currentstroke}%
\pgfsetdash{}{0pt}%
\pgfsys@defobject{currentmarker}{\pgfqpoint{-0.048611in}{0.000000in}}{\pgfqpoint{-0.000000in}{0.000000in}}{%
\pgfpathmoveto{\pgfqpoint{-0.000000in}{0.000000in}}%
\pgfpathlineto{\pgfqpoint{-0.048611in}{0.000000in}}%
\pgfusepath{stroke,fill}%
}%
\begin{pgfscope}%
\pgfsys@transformshift{0.726250in}{1.270279in}%
\pgfsys@useobject{currentmarker}{}%
\end{pgfscope}%
\end{pgfscope}%
\begin{pgfscope}%
\definecolor{textcolor}{rgb}{0.000000,0.000000,0.000000}%
\pgfsetstrokecolor{textcolor}%
\pgfsetfillcolor{textcolor}%
\pgftext[x=0.549499in, y=1.222794in, left, base]{\color{textcolor}\rmfamily\fontsize{9.000000}{10.800000}\selectfont 4}%
\end{pgfscope}%
\begin{pgfscope}%
\pgfpathrectangle{\pgfqpoint{0.726250in}{0.525000in}}{\pgfqpoint{1.120000in}{1.887500in}}%
\pgfusepath{clip}%
\pgfsetbuttcap%
\pgfsetroundjoin%
\pgfsetlinewidth{0.803000pt}%
\definecolor{currentstroke}{rgb}{0.752941,0.752941,0.752941}%
\pgfsetstrokecolor{currentstroke}%
\pgfsetdash{{2.960000pt}{1.280000pt}}{0.000000pt}%
\pgfpathmoveto{\pgfqpoint{0.726250in}{1.642919in}}%
\pgfpathlineto{\pgfqpoint{1.846250in}{1.642919in}}%
\pgfusepath{stroke}%
\end{pgfscope}%
\begin{pgfscope}%
\pgfsetbuttcap%
\pgfsetroundjoin%
\definecolor{currentfill}{rgb}{0.000000,0.000000,0.000000}%
\pgfsetfillcolor{currentfill}%
\pgfsetlinewidth{0.803000pt}%
\definecolor{currentstroke}{rgb}{0.000000,0.000000,0.000000}%
\pgfsetstrokecolor{currentstroke}%
\pgfsetdash{}{0pt}%
\pgfsys@defobject{currentmarker}{\pgfqpoint{-0.048611in}{0.000000in}}{\pgfqpoint{-0.000000in}{0.000000in}}{%
\pgfpathmoveto{\pgfqpoint{-0.000000in}{0.000000in}}%
\pgfpathlineto{\pgfqpoint{-0.048611in}{0.000000in}}%
\pgfusepath{stroke,fill}%
}%
\begin{pgfscope}%
\pgfsys@transformshift{0.726250in}{1.642919in}%
\pgfsys@useobject{currentmarker}{}%
\end{pgfscope}%
\end{pgfscope}%
\begin{pgfscope}%
\definecolor{textcolor}{rgb}{0.000000,0.000000,0.000000}%
\pgfsetstrokecolor{textcolor}%
\pgfsetfillcolor{textcolor}%
\pgftext[x=0.549499in, y=1.595433in, left, base]{\color{textcolor}\rmfamily\fontsize{9.000000}{10.800000}\selectfont 6}%
\end{pgfscope}%
\begin{pgfscope}%
\pgfpathrectangle{\pgfqpoint{0.726250in}{0.525000in}}{\pgfqpoint{1.120000in}{1.887500in}}%
\pgfusepath{clip}%
\pgfsetbuttcap%
\pgfsetroundjoin%
\pgfsetlinewidth{0.803000pt}%
\definecolor{currentstroke}{rgb}{0.752941,0.752941,0.752941}%
\pgfsetstrokecolor{currentstroke}%
\pgfsetdash{{2.960000pt}{1.280000pt}}{0.000000pt}%
\pgfpathmoveto{\pgfqpoint{0.726250in}{2.015558in}}%
\pgfpathlineto{\pgfqpoint{1.846250in}{2.015558in}}%
\pgfusepath{stroke}%
\end{pgfscope}%
\begin{pgfscope}%
\pgfsetbuttcap%
\pgfsetroundjoin%
\definecolor{currentfill}{rgb}{0.000000,0.000000,0.000000}%
\pgfsetfillcolor{currentfill}%
\pgfsetlinewidth{0.803000pt}%
\definecolor{currentstroke}{rgb}{0.000000,0.000000,0.000000}%
\pgfsetstrokecolor{currentstroke}%
\pgfsetdash{}{0pt}%
\pgfsys@defobject{currentmarker}{\pgfqpoint{-0.048611in}{0.000000in}}{\pgfqpoint{-0.000000in}{0.000000in}}{%
\pgfpathmoveto{\pgfqpoint{-0.000000in}{0.000000in}}%
\pgfpathlineto{\pgfqpoint{-0.048611in}{0.000000in}}%
\pgfusepath{stroke,fill}%
}%
\begin{pgfscope}%
\pgfsys@transformshift{0.726250in}{2.015558in}%
\pgfsys@useobject{currentmarker}{}%
\end{pgfscope}%
\end{pgfscope}%
\begin{pgfscope}%
\definecolor{textcolor}{rgb}{0.000000,0.000000,0.000000}%
\pgfsetstrokecolor{textcolor}%
\pgfsetfillcolor{textcolor}%
\pgftext[x=0.549499in, y=1.968073in, left, base]{\color{textcolor}\rmfamily\fontsize{9.000000}{10.800000}\selectfont 8}%
\end{pgfscope}%
\begin{pgfscope}%
\pgfpathrectangle{\pgfqpoint{0.726250in}{0.525000in}}{\pgfqpoint{1.120000in}{1.887500in}}%
\pgfusepath{clip}%
\pgfsetbuttcap%
\pgfsetroundjoin%
\pgfsetlinewidth{0.803000pt}%
\definecolor{currentstroke}{rgb}{0.752941,0.752941,0.752941}%
\pgfsetstrokecolor{currentstroke}%
\pgfsetdash{{2.960000pt}{1.280000pt}}{0.000000pt}%
\pgfpathmoveto{\pgfqpoint{0.726250in}{2.388198in}}%
\pgfpathlineto{\pgfqpoint{1.846250in}{2.388198in}}%
\pgfusepath{stroke}%
\end{pgfscope}%
\begin{pgfscope}%
\pgfsetbuttcap%
\pgfsetroundjoin%
\definecolor{currentfill}{rgb}{0.000000,0.000000,0.000000}%
\pgfsetfillcolor{currentfill}%
\pgfsetlinewidth{0.803000pt}%
\definecolor{currentstroke}{rgb}{0.000000,0.000000,0.000000}%
\pgfsetstrokecolor{currentstroke}%
\pgfsetdash{}{0pt}%
\pgfsys@defobject{currentmarker}{\pgfqpoint{-0.048611in}{0.000000in}}{\pgfqpoint{-0.000000in}{0.000000in}}{%
\pgfpathmoveto{\pgfqpoint{-0.000000in}{0.000000in}}%
\pgfpathlineto{\pgfqpoint{-0.048611in}{0.000000in}}%
\pgfusepath{stroke,fill}%
}%
\begin{pgfscope}%
\pgfsys@transformshift{0.726250in}{2.388198in}%
\pgfsys@useobject{currentmarker}{}%
\end{pgfscope}%
\end{pgfscope}%
\begin{pgfscope}%
\definecolor{textcolor}{rgb}{0.000000,0.000000,0.000000}%
\pgfsetstrokecolor{textcolor}%
\pgfsetfillcolor{textcolor}%
\pgftext[x=0.469970in, y=2.340712in, left, base]{\color{textcolor}\rmfamily\fontsize{9.000000}{10.800000}\selectfont 10}%
\end{pgfscope}%
\begin{pgfscope}%
\definecolor{textcolor}{rgb}{0.000000,0.000000,0.000000}%
\pgfsetstrokecolor{textcolor}%
\pgfsetfillcolor{textcolor}%
\pgftext[x=0.414415in,y=1.468750in,,bottom,rotate=90.000000]{\color{textcolor}\rmfamily\fontsize{9.000000}{10.800000}\selectfont \(\displaystyle E_c\) [MeV]}%
\end{pgfscope}%
\begin{pgfscope}%
\pgfpathrectangle{\pgfqpoint{0.726250in}{0.525000in}}{\pgfqpoint{1.120000in}{1.887500in}}%
\pgfusepath{clip}%
\pgfsetrectcap%
\pgfsetroundjoin%
\pgfsetlinewidth{1.003750pt}%
\definecolor{currentstroke}{rgb}{0.000000,0.000000,0.000000}%
\pgfsetstrokecolor{currentstroke}%
\pgfsetdash{}{0pt}%
\pgfpathmoveto{\pgfqpoint{0.726250in}{0.768042in}}%
\pgfpathlineto{\pgfqpoint{0.737563in}{0.768042in}}%
\pgfpathlineto{\pgfqpoint{0.748876in}{0.768057in}}%
\pgfpathlineto{\pgfqpoint{0.760189in}{0.768613in}}%
\pgfpathlineto{\pgfqpoint{0.771503in}{0.769205in}}%
\pgfpathlineto{\pgfqpoint{0.782816in}{0.770299in}}%
\pgfpathlineto{\pgfqpoint{0.794129in}{0.771478in}}%
\pgfpathlineto{\pgfqpoint{0.805442in}{0.772492in}}%
\pgfpathlineto{\pgfqpoint{0.816755in}{0.774341in}}%
\pgfpathlineto{\pgfqpoint{0.828068in}{0.776004in}}%
\pgfpathlineto{\pgfqpoint{0.839381in}{0.777641in}}%
\pgfpathlineto{\pgfqpoint{0.850694in}{0.779901in}}%
\pgfpathlineto{\pgfqpoint{0.862008in}{0.781889in}}%
\pgfpathlineto{\pgfqpoint{0.873321in}{0.784308in}}%
\pgfpathlineto{\pgfqpoint{0.884634in}{0.787203in}}%
\pgfpathlineto{\pgfqpoint{0.895947in}{0.789634in}}%
\pgfpathlineto{\pgfqpoint{0.907260in}{0.792720in}}%
\pgfpathlineto{\pgfqpoint{0.918573in}{0.795948in}}%
\pgfpathlineto{\pgfqpoint{0.929886in}{0.798956in}}%
\pgfpathlineto{\pgfqpoint{0.941199in}{0.802593in}}%
\pgfpathlineto{\pgfqpoint{0.952513in}{0.806356in}}%
\pgfpathlineto{\pgfqpoint{0.963826in}{0.810568in}}%
\pgfpathlineto{\pgfqpoint{0.975139in}{0.814588in}}%
\pgfpathlineto{\pgfqpoint{0.986452in}{0.818468in}}%
\pgfpathlineto{\pgfqpoint{0.997765in}{0.823096in}}%
\pgfpathlineto{\pgfqpoint{1.009078in}{0.827782in}}%
\pgfpathlineto{\pgfqpoint{1.020391in}{0.832543in}}%
\pgfpathlineto{\pgfqpoint{1.031705in}{0.837378in}}%
\pgfpathlineto{\pgfqpoint{1.043018in}{0.843021in}}%
\pgfpathlineto{\pgfqpoint{1.054331in}{0.848042in}}%
\pgfpathlineto{\pgfqpoint{1.065644in}{0.853856in}}%
\pgfpathlineto{\pgfqpoint{1.076957in}{0.859786in}}%
\pgfpathlineto{\pgfqpoint{1.088270in}{0.865879in}}%
\pgfpathlineto{\pgfqpoint{1.099583in}{0.872077in}}%
\pgfpathlineto{\pgfqpoint{1.110896in}{0.878268in}}%
\pgfpathlineto{\pgfqpoint{1.122210in}{0.884687in}}%
\pgfpathlineto{\pgfqpoint{1.133523in}{0.891168in}}%
\pgfpathlineto{\pgfqpoint{1.144836in}{0.898678in}}%
\pgfpathlineto{\pgfqpoint{1.156149in}{0.905298in}}%
\pgfpathlineto{\pgfqpoint{1.167462in}{0.913147in}}%
\pgfpathlineto{\pgfqpoint{1.178775in}{0.921132in}}%
\pgfpathlineto{\pgfqpoint{1.190088in}{0.929117in}}%
\pgfpathlineto{\pgfqpoint{1.201402in}{0.937150in}}%
\pgfpathlineto{\pgfqpoint{1.212715in}{0.945303in}}%
\pgfpathlineto{\pgfqpoint{1.224028in}{0.954545in}}%
\pgfpathlineto{\pgfqpoint{1.235341in}{0.963209in}}%
\pgfpathlineto{\pgfqpoint{1.246654in}{0.972047in}}%
\pgfpathlineto{\pgfqpoint{1.257967in}{0.981865in}}%
\pgfpathlineto{\pgfqpoint{1.269280in}{0.991559in}}%
\pgfpathlineto{\pgfqpoint{1.280593in}{1.002357in}}%
\pgfpathlineto{\pgfqpoint{1.291907in}{1.012021in}}%
\pgfpathlineto{\pgfqpoint{1.303220in}{1.023267in}}%
\pgfpathlineto{\pgfqpoint{1.314533in}{1.034649in}}%
\pgfpathlineto{\pgfqpoint{1.325846in}{1.046218in}}%
\pgfpathlineto{\pgfqpoint{1.337159in}{1.057943in}}%
\pgfpathlineto{\pgfqpoint{1.348472in}{1.069874in}}%
\pgfpathlineto{\pgfqpoint{1.359785in}{1.082467in}}%
\pgfpathlineto{\pgfqpoint{1.371098in}{1.095548in}}%
\pgfpathlineto{\pgfqpoint{1.382412in}{1.109544in}}%
\pgfpathlineto{\pgfqpoint{1.393725in}{1.122785in}}%
\pgfpathlineto{\pgfqpoint{1.405038in}{1.137445in}}%
\pgfpathlineto{\pgfqpoint{1.416351in}{1.152448in}}%
\pgfpathlineto{\pgfqpoint{1.427664in}{1.167997in}}%
\pgfpathlineto{\pgfqpoint{1.438977in}{1.183690in}}%
\pgfpathlineto{\pgfqpoint{1.450290in}{1.199766in}}%
\pgfpathlineto{\pgfqpoint{1.461604in}{1.217130in}}%
\pgfpathlineto{\pgfqpoint{1.472917in}{1.234688in}}%
\pgfpathlineto{\pgfqpoint{1.484230in}{1.252525in}}%
\pgfpathlineto{\pgfqpoint{1.495543in}{1.272373in}}%
\pgfpathlineto{\pgfqpoint{1.506856in}{1.290468in}}%
\pgfpathlineto{\pgfqpoint{1.518169in}{1.311502in}}%
\pgfpathlineto{\pgfqpoint{1.529482in}{1.332112in}}%
\pgfpathlineto{\pgfqpoint{1.540795in}{1.354241in}}%
\pgfpathlineto{\pgfqpoint{1.552109in}{1.376761in}}%
\pgfpathlineto{\pgfqpoint{1.563422in}{1.398401in}}%
\pgfpathlineto{\pgfqpoint{1.574735in}{1.423243in}}%
\pgfpathlineto{\pgfqpoint{1.586048in}{1.447391in}}%
\pgfpathlineto{\pgfqpoint{1.597361in}{1.472249in}}%
\pgfpathlineto{\pgfqpoint{1.608674in}{1.500299in}}%
\pgfpathlineto{\pgfqpoint{1.619987in}{1.526419in}}%
\pgfpathlineto{\pgfqpoint{1.631301in}{1.555251in}}%
\pgfpathlineto{\pgfqpoint{1.642614in}{1.583476in}}%
\pgfpathlineto{\pgfqpoint{1.653927in}{1.614439in}}%
\pgfpathlineto{\pgfqpoint{1.665240in}{1.646722in}}%
\pgfpathlineto{\pgfqpoint{1.676553in}{1.676396in}}%
\pgfpathlineto{\pgfqpoint{1.687866in}{1.710300in}}%
\pgfpathlineto{\pgfqpoint{1.699179in}{1.745283in}}%
\pgfpathlineto{\pgfqpoint{1.710492in}{1.780913in}}%
\pgfpathlineto{\pgfqpoint{1.721806in}{1.817632in}}%
\pgfpathlineto{\pgfqpoint{1.733119in}{1.858341in}}%
\pgfpathlineto{\pgfqpoint{1.744432in}{1.898033in}}%
\pgfpathlineto{\pgfqpoint{1.755745in}{1.938290in}}%
\pgfpathlineto{\pgfqpoint{1.767058in}{1.979811in}}%
\pgfpathlineto{\pgfqpoint{1.778371in}{2.025765in}}%
\pgfpathlineto{\pgfqpoint{1.789684in}{2.069831in}}%
\pgfpathlineto{\pgfqpoint{1.800997in}{2.119165in}}%
\pgfpathlineto{\pgfqpoint{1.812311in}{2.165705in}}%
\pgfpathlineto{\pgfqpoint{1.823624in}{2.217728in}}%
\pgfpathlineto{\pgfqpoint{1.834937in}{2.267338in}}%
\pgfpathlineto{\pgfqpoint{1.846250in}{2.322549in}}%
\pgfusepath{stroke}%
\end{pgfscope}%
\begin{pgfscope}%
\pgfpathrectangle{\pgfqpoint{0.726250in}{0.525000in}}{\pgfqpoint{1.120000in}{1.887500in}}%
\pgfusepath{clip}%
\pgfsetbuttcap%
\pgfsetroundjoin%
\pgfsetlinewidth{1.003750pt}%
\definecolor{currentstroke}{rgb}{0.501961,0.501961,0.501961}%
\pgfsetstrokecolor{currentstroke}%
\pgfsetdash{{3.700000pt}{1.600000pt}}{0.000000pt}%
\pgfpathmoveto{\pgfqpoint{0.726250in}{0.523536in}}%
\pgfpathlineto{\pgfqpoint{0.737563in}{0.542175in}}%
\pgfpathlineto{\pgfqpoint{0.748876in}{0.588235in}}%
\pgfpathlineto{\pgfqpoint{0.760189in}{0.661192in}}%
\pgfpathlineto{\pgfqpoint{0.771503in}{0.757618in}}%
\pgfpathlineto{\pgfqpoint{0.782816in}{0.873826in}}%
\pgfpathlineto{\pgfqpoint{0.794129in}{1.003540in}}%
\pgfpathlineto{\pgfqpoint{0.805442in}{1.141579in}}%
\pgfpathlineto{\pgfqpoint{0.816755in}{1.282016in}}%
\pgfpathlineto{\pgfqpoint{0.828068in}{1.418962in}}%
\pgfpathlineto{\pgfqpoint{0.839381in}{1.547010in}}%
\pgfpathlineto{\pgfqpoint{0.850694in}{1.661653in}}%
\pgfpathlineto{\pgfqpoint{0.862008in}{1.758509in}}%
\pgfpathlineto{\pgfqpoint{0.873321in}{1.834864in}}%
\pgfpathlineto{\pgfqpoint{0.884634in}{1.889177in}}%
\pgfpathlineto{\pgfqpoint{0.895947in}{1.919988in}}%
\pgfpathlineto{\pgfqpoint{0.907260in}{1.928193in}}%
\pgfpathlineto{\pgfqpoint{0.918573in}{1.914518in}}%
\pgfpathlineto{\pgfqpoint{0.929886in}{1.881408in}}%
\pgfpathlineto{\pgfqpoint{0.941199in}{1.831006in}}%
\pgfpathlineto{\pgfqpoint{0.952513in}{1.766313in}}%
\pgfpathlineto{\pgfqpoint{0.963826in}{1.690473in}}%
\pgfpathlineto{\pgfqpoint{0.975139in}{1.606600in}}%
\pgfpathlineto{\pgfqpoint{0.986452in}{1.517373in}}%
\pgfpathlineto{\pgfqpoint{0.997765in}{1.426314in}}%
\pgfpathlineto{\pgfqpoint{1.009078in}{1.335237in}}%
\pgfpathlineto{\pgfqpoint{1.020391in}{1.246307in}}%
\pgfpathlineto{\pgfqpoint{1.031705in}{1.161282in}}%
\pgfpathlineto{\pgfqpoint{1.043018in}{1.081480in}}%
\pgfpathlineto{\pgfqpoint{1.054331in}{1.007391in}}%
\pgfpathlineto{\pgfqpoint{1.065644in}{0.939783in}}%
\pgfpathlineto{\pgfqpoint{1.076957in}{0.879073in}}%
\pgfpathlineto{\pgfqpoint{1.088270in}{0.824617in}}%
\pgfpathlineto{\pgfqpoint{1.099583in}{0.776954in}}%
\pgfpathlineto{\pgfqpoint{1.110896in}{0.735422in}}%
\pgfpathlineto{\pgfqpoint{1.122210in}{0.699634in}}%
\pgfpathlineto{\pgfqpoint{1.133523in}{0.668988in}}%
\pgfpathlineto{\pgfqpoint{1.144836in}{0.643002in}}%
\pgfpathlineto{\pgfqpoint{1.156149in}{0.621150in}}%
\pgfpathlineto{\pgfqpoint{1.167462in}{0.602921in}}%
\pgfpathlineto{\pgfqpoint{1.178775in}{0.587791in}}%
\pgfpathlineto{\pgfqpoint{1.190088in}{0.575324in}}%
\pgfpathlineto{\pgfqpoint{1.201402in}{0.565141in}}%
\pgfpathlineto{\pgfqpoint{1.212715in}{0.556883in}}%
\pgfpathlineto{\pgfqpoint{1.224028in}{0.550216in}}%
\pgfpathlineto{\pgfqpoint{1.235341in}{0.544843in}}%
\pgfpathlineto{\pgfqpoint{1.246654in}{0.540555in}}%
\pgfpathlineto{\pgfqpoint{1.257967in}{0.537145in}}%
\pgfpathlineto{\pgfqpoint{1.269280in}{0.534447in}}%
\pgfpathlineto{\pgfqpoint{1.280593in}{0.532321in}}%
\pgfpathlineto{\pgfqpoint{1.291907in}{0.530647in}}%
\pgfpathlineto{\pgfqpoint{1.303220in}{0.529345in}}%
\pgfpathlineto{\pgfqpoint{1.314533in}{0.528334in}}%
\pgfpathlineto{\pgfqpoint{1.325846in}{0.527550in}}%
\pgfpathlineto{\pgfqpoint{1.337159in}{0.526943in}}%
\pgfpathlineto{\pgfqpoint{1.348472in}{0.526478in}}%
\pgfpathlineto{\pgfqpoint{1.359785in}{0.526121in}}%
\pgfpathlineto{\pgfqpoint{1.371098in}{0.525849in}}%
\pgfpathlineto{\pgfqpoint{1.382412in}{0.525641in}}%
\pgfpathlineto{\pgfqpoint{1.393725in}{0.525482in}}%
\pgfpathlineto{\pgfqpoint{1.405038in}{0.525363in}}%
\pgfpathlineto{\pgfqpoint{1.416351in}{0.525272in}}%
\pgfpathlineto{\pgfqpoint{1.427664in}{0.525204in}}%
\pgfpathlineto{\pgfqpoint{1.438977in}{0.525153in}}%
\pgfpathlineto{\pgfqpoint{1.450290in}{0.525114in}}%
\pgfpathlineto{\pgfqpoint{1.461604in}{0.525085in}}%
\pgfpathlineto{\pgfqpoint{1.472917in}{0.525063in}}%
\pgfpathlineto{\pgfqpoint{1.484230in}{0.525047in}}%
\pgfpathlineto{\pgfqpoint{1.495543in}{0.525035in}}%
\pgfpathlineto{\pgfqpoint{1.506856in}{0.525026in}}%
\pgfpathlineto{\pgfqpoint{1.518169in}{0.525019in}}%
\pgfpathlineto{\pgfqpoint{1.529482in}{0.525014in}}%
\pgfpathlineto{\pgfqpoint{1.540795in}{0.525010in}}%
\pgfpathlineto{\pgfqpoint{1.552109in}{0.525008in}}%
\pgfpathlineto{\pgfqpoint{1.563422in}{0.525006in}}%
\pgfpathlineto{\pgfqpoint{1.574735in}{0.525004in}}%
\pgfpathlineto{\pgfqpoint{1.586048in}{0.525003in}}%
\pgfpathlineto{\pgfqpoint{1.597361in}{0.525002in}}%
\pgfpathlineto{\pgfqpoint{1.608674in}{0.525002in}}%
\pgfpathlineto{\pgfqpoint{1.619987in}{0.525001in}}%
\pgfpathlineto{\pgfqpoint{1.631301in}{0.525001in}}%
\pgfpathlineto{\pgfqpoint{1.642614in}{0.525001in}}%
\pgfpathlineto{\pgfqpoint{1.653927in}{0.525000in}}%
\pgfpathlineto{\pgfqpoint{1.665240in}{0.525000in}}%
\pgfpathlineto{\pgfqpoint{1.676553in}{0.525000in}}%
\pgfpathlineto{\pgfqpoint{1.687866in}{0.525000in}}%
\pgfpathlineto{\pgfqpoint{1.699179in}{0.525000in}}%
\pgfpathlineto{\pgfqpoint{1.710492in}{0.525000in}}%
\pgfpathlineto{\pgfqpoint{1.721806in}{0.525000in}}%
\pgfpathlineto{\pgfqpoint{1.733119in}{0.525000in}}%
\pgfpathlineto{\pgfqpoint{1.744432in}{0.525000in}}%
\pgfpathlineto{\pgfqpoint{1.755745in}{0.525000in}}%
\pgfpathlineto{\pgfqpoint{1.767058in}{0.525000in}}%
\pgfpathlineto{\pgfqpoint{1.778371in}{0.525000in}}%
\pgfpathlineto{\pgfqpoint{1.789684in}{0.525000in}}%
\pgfpathlineto{\pgfqpoint{1.800997in}{0.525000in}}%
\pgfpathlineto{\pgfqpoint{1.812311in}{0.525000in}}%
\pgfpathlineto{\pgfqpoint{1.823624in}{0.525000in}}%
\pgfpathlineto{\pgfqpoint{1.834937in}{0.525000in}}%
\pgfpathlineto{\pgfqpoint{1.846250in}{0.525000in}}%
\pgfusepath{stroke}%
\end{pgfscope}%
\begin{pgfscope}%
\pgfsetrectcap%
\pgfsetmiterjoin%
\pgfsetlinewidth{1.003750pt}%
\definecolor{currentstroke}{rgb}{0.000000,0.000000,0.000000}%
\pgfsetstrokecolor{currentstroke}%
\pgfsetdash{}{0pt}%
\pgfpathmoveto{\pgfqpoint{0.726250in}{0.525000in}}%
\pgfpathlineto{\pgfqpoint{0.726250in}{2.412500in}}%
\pgfusepath{stroke}%
\end{pgfscope}%
\begin{pgfscope}%
\pgfsetrectcap%
\pgfsetmiterjoin%
\pgfsetlinewidth{1.003750pt}%
\definecolor{currentstroke}{rgb}{0.000000,0.000000,0.000000}%
\pgfsetstrokecolor{currentstroke}%
\pgfsetdash{}{0pt}%
\pgfpathmoveto{\pgfqpoint{1.846250in}{0.525000in}}%
\pgfpathlineto{\pgfqpoint{1.846250in}{2.412500in}}%
\pgfusepath{stroke}%
\end{pgfscope}%
\begin{pgfscope}%
\pgfsetrectcap%
\pgfsetmiterjoin%
\pgfsetlinewidth{1.003750pt}%
\definecolor{currentstroke}{rgb}{0.000000,0.000000,0.000000}%
\pgfsetstrokecolor{currentstroke}%
\pgfsetdash{}{0pt}%
\pgfpathmoveto{\pgfqpoint{0.726250in}{0.525000in}}%
\pgfpathlineto{\pgfqpoint{1.846250in}{0.525000in}}%
\pgfusepath{stroke}%
\end{pgfscope}%
\begin{pgfscope}%
\pgfsetrectcap%
\pgfsetmiterjoin%
\pgfsetlinewidth{1.003750pt}%
\definecolor{currentstroke}{rgb}{0.000000,0.000000,0.000000}%
\pgfsetstrokecolor{currentstroke}%
\pgfsetdash{}{0pt}%
\pgfpathmoveto{\pgfqpoint{0.726250in}{2.412500in}}%
\pgfpathlineto{\pgfqpoint{1.846250in}{2.412500in}}%
\pgfusepath{stroke}%
\end{pgfscope}%
\end{pgfpicture}%
\makeatother%
\endgroup%

  \caption{Plot of the MSW resonance energy in the Sun, as a function of the distance
  from the center. Overlaied is the pdf for the Boron neutrino production.}
  \label{fig:msw_resonance_energy}
\end{marginfigure}

MSW conversion in presence of varying electron density is customarily illustrated in
textbooks in the (somewhat fictional) two-generation, small-mixing-angle setup. This
makes the discussion easier and more intuitive, as the mass and flavor eigenstates
coincide, and, since the matter term due to CC interaction is only affecting the
first mass eigenstate, the effective mass of the second does not depend
on $n_e$. When $\theta = 0$, we have $C(\zeta) = (1 - \zeta)$, and the two mass
eigevalues read
\begin{align*}
  \begin{cases}
  \tilde{m}_1^2 & = m_1^2 + \zeta \Delta_m^2 \quad (\text{for } \nu_e)\\
  \tilde{m}_2^2 & = m_2^2 \quad (\text{for } \nu_\mu).
\end{cases}
\end{align*}
An electron neutrino created above the resonance, where the electron number density
is large (e.g. at the center of the Sun), propagating toward regions of comparatively
lower density sees it effective mass gradually decreasing, until it reaches the
resonance condition, where it is converted to a muon neutrino. At this point the
electron density is no longer high enough to support matter oscillations, and the
neutrino remains in its $\nu_\mu$ state.

\begin{marginfigure}
  %% Creator: Matplotlib, PGF backend
%%
%% To include the figure in your LaTeX document, write
%%   \input{<filename>.pgf}
%%
%% Make sure the required packages are loaded in your preamble
%%   \usepackage{pgf}
%%
%% Also ensure that all the required font packages are loaded; for instance,
%% the lmodern package is sometimes necessary when using math font.
%%   \usepackage{lmodern}
%%
%% Figures using additional raster images can only be included by \input if
%% they are in the same directory as the main LaTeX file. For loading figures
%% from other directories you can use the `import` package
%%   \usepackage{import}
%%
%% and then include the figures with
%%   \import{<path to file>}{<filename>.pgf}
%%
%% Matplotlib used the following preamble
%%   \usepackage{fontspec}
%%   \setmainfont{DejaVuSerif.ttf}[Path=\detokenize{/usr/share/matplotlib/mpl-data/fonts/ttf/}]
%%   \setsansfont{DejaVuSans.ttf}[Path=\detokenize{/usr/share/matplotlib/mpl-data/fonts/ttf/}]
%%   \setmonofont{DejaVuSansMono.ttf}[Path=\detokenize{/usr/share/matplotlib/mpl-data/fonts/ttf/}]
%%
\begingroup%
\makeatletter%
\begin{pgfpicture}%
\pgfpathrectangle{\pgfpointorigin}{\pgfqpoint{1.950000in}{2.500000in}}%
\pgfusepath{use as bounding box, clip}%
\begin{pgfscope}%
\pgfsetbuttcap%
\pgfsetmiterjoin%
\definecolor{currentfill}{rgb}{1.000000,1.000000,1.000000}%
\pgfsetfillcolor{currentfill}%
\pgfsetlinewidth{0.000000pt}%
\definecolor{currentstroke}{rgb}{1.000000,1.000000,1.000000}%
\pgfsetstrokecolor{currentstroke}%
\pgfsetdash{}{0pt}%
\pgfpathmoveto{\pgfqpoint{0.000000in}{0.000000in}}%
\pgfpathlineto{\pgfqpoint{1.950000in}{0.000000in}}%
\pgfpathlineto{\pgfqpoint{1.950000in}{2.500000in}}%
\pgfpathlineto{\pgfqpoint{0.000000in}{2.500000in}}%
\pgfpathlineto{\pgfqpoint{0.000000in}{0.000000in}}%
\pgfpathclose%
\pgfusepath{fill}%
\end{pgfscope}%
\begin{pgfscope}%
\pgfsetbuttcap%
\pgfsetmiterjoin%
\definecolor{currentfill}{rgb}{1.000000,1.000000,1.000000}%
\pgfsetfillcolor{currentfill}%
\pgfsetlinewidth{0.000000pt}%
\definecolor{currentstroke}{rgb}{0.000000,0.000000,0.000000}%
\pgfsetstrokecolor{currentstroke}%
\pgfsetstrokeopacity{0.000000}%
\pgfsetdash{}{0pt}%
\pgfpathmoveto{\pgfqpoint{0.390000in}{0.525000in}}%
\pgfpathlineto{\pgfqpoint{1.846250in}{0.525000in}}%
\pgfpathlineto{\pgfqpoint{1.846250in}{2.412500in}}%
\pgfpathlineto{\pgfqpoint{0.390000in}{2.412500in}}%
\pgfpathlineto{\pgfqpoint{0.390000in}{0.525000in}}%
\pgfpathclose%
\pgfusepath{fill}%
\end{pgfscope}%
\begin{pgfscope}%
\pgfpathrectangle{\pgfqpoint{0.390000in}{0.525000in}}{\pgfqpoint{1.456250in}{1.887500in}}%
\pgfusepath{clip}%
\pgfsetbuttcap%
\pgfsetroundjoin%
\pgfsetlinewidth{0.803000pt}%
\definecolor{currentstroke}{rgb}{0.752941,0.752941,0.752941}%
\pgfsetstrokecolor{currentstroke}%
\pgfsetdash{{2.960000pt}{1.280000pt}}{0.000000pt}%
\pgfpathmoveto{\pgfqpoint{0.390000in}{0.525000in}}%
\pgfpathlineto{\pgfqpoint{0.390000in}{2.412500in}}%
\pgfusepath{stroke}%
\end{pgfscope}%
\begin{pgfscope}%
\pgfsetbuttcap%
\pgfsetroundjoin%
\definecolor{currentfill}{rgb}{0.000000,0.000000,0.000000}%
\pgfsetfillcolor{currentfill}%
\pgfsetlinewidth{0.803000pt}%
\definecolor{currentstroke}{rgb}{0.000000,0.000000,0.000000}%
\pgfsetstrokecolor{currentstroke}%
\pgfsetdash{}{0pt}%
\pgfsys@defobject{currentmarker}{\pgfqpoint{0.000000in}{-0.048611in}}{\pgfqpoint{0.000000in}{0.000000in}}{%
\pgfpathmoveto{\pgfqpoint{0.000000in}{0.000000in}}%
\pgfpathlineto{\pgfqpoint{0.000000in}{-0.048611in}}%
\pgfusepath{stroke,fill}%
}%
\begin{pgfscope}%
\pgfsys@transformshift{0.390000in}{0.525000in}%
\pgfsys@useobject{currentmarker}{}%
\end{pgfscope}%
\end{pgfscope}%
\begin{pgfscope}%
\definecolor{textcolor}{rgb}{0.000000,0.000000,0.000000}%
\pgfsetstrokecolor{textcolor}%
\pgfsetfillcolor{textcolor}%
\pgftext[x=0.390000in,y=0.427778in,,top]{\color{textcolor}\rmfamily\fontsize{9.000000}{10.800000}\selectfont 0}%
\end{pgfscope}%
\begin{pgfscope}%
\pgfpathrectangle{\pgfqpoint{0.390000in}{0.525000in}}{\pgfqpoint{1.456250in}{1.887500in}}%
\pgfusepath{clip}%
\pgfsetbuttcap%
\pgfsetroundjoin%
\pgfsetlinewidth{0.803000pt}%
\definecolor{currentstroke}{rgb}{0.752941,0.752941,0.752941}%
\pgfsetstrokecolor{currentstroke}%
\pgfsetdash{{2.960000pt}{1.280000pt}}{0.000000pt}%
\pgfpathmoveto{\pgfqpoint{0.972500in}{0.525000in}}%
\pgfpathlineto{\pgfqpoint{0.972500in}{2.412500in}}%
\pgfusepath{stroke}%
\end{pgfscope}%
\begin{pgfscope}%
\pgfsetbuttcap%
\pgfsetroundjoin%
\definecolor{currentfill}{rgb}{0.000000,0.000000,0.000000}%
\pgfsetfillcolor{currentfill}%
\pgfsetlinewidth{0.803000pt}%
\definecolor{currentstroke}{rgb}{0.000000,0.000000,0.000000}%
\pgfsetstrokecolor{currentstroke}%
\pgfsetdash{}{0pt}%
\pgfsys@defobject{currentmarker}{\pgfqpoint{0.000000in}{-0.048611in}}{\pgfqpoint{0.000000in}{0.000000in}}{%
\pgfpathmoveto{\pgfqpoint{0.000000in}{0.000000in}}%
\pgfpathlineto{\pgfqpoint{0.000000in}{-0.048611in}}%
\pgfusepath{stroke,fill}%
}%
\begin{pgfscope}%
\pgfsys@transformshift{0.972500in}{0.525000in}%
\pgfsys@useobject{currentmarker}{}%
\end{pgfscope}%
\end{pgfscope}%
\begin{pgfscope}%
\definecolor{textcolor}{rgb}{0.000000,0.000000,0.000000}%
\pgfsetstrokecolor{textcolor}%
\pgfsetfillcolor{textcolor}%
\pgftext[x=0.972500in,y=0.427778in,,top]{\color{textcolor}\rmfamily\fontsize{9.000000}{10.800000}\selectfont 1}%
\end{pgfscope}%
\begin{pgfscope}%
\pgfpathrectangle{\pgfqpoint{0.390000in}{0.525000in}}{\pgfqpoint{1.456250in}{1.887500in}}%
\pgfusepath{clip}%
\pgfsetbuttcap%
\pgfsetroundjoin%
\pgfsetlinewidth{0.803000pt}%
\definecolor{currentstroke}{rgb}{0.752941,0.752941,0.752941}%
\pgfsetstrokecolor{currentstroke}%
\pgfsetdash{{2.960000pt}{1.280000pt}}{0.000000pt}%
\pgfpathmoveto{\pgfqpoint{1.555000in}{0.525000in}}%
\pgfpathlineto{\pgfqpoint{1.555000in}{2.412500in}}%
\pgfusepath{stroke}%
\end{pgfscope}%
\begin{pgfscope}%
\pgfsetbuttcap%
\pgfsetroundjoin%
\definecolor{currentfill}{rgb}{0.000000,0.000000,0.000000}%
\pgfsetfillcolor{currentfill}%
\pgfsetlinewidth{0.803000pt}%
\definecolor{currentstroke}{rgb}{0.000000,0.000000,0.000000}%
\pgfsetstrokecolor{currentstroke}%
\pgfsetdash{}{0pt}%
\pgfsys@defobject{currentmarker}{\pgfqpoint{0.000000in}{-0.048611in}}{\pgfqpoint{0.000000in}{0.000000in}}{%
\pgfpathmoveto{\pgfqpoint{0.000000in}{0.000000in}}%
\pgfpathlineto{\pgfqpoint{0.000000in}{-0.048611in}}%
\pgfusepath{stroke,fill}%
}%
\begin{pgfscope}%
\pgfsys@transformshift{1.555000in}{0.525000in}%
\pgfsys@useobject{currentmarker}{}%
\end{pgfscope}%
\end{pgfscope}%
\begin{pgfscope}%
\definecolor{textcolor}{rgb}{0.000000,0.000000,0.000000}%
\pgfsetstrokecolor{textcolor}%
\pgfsetfillcolor{textcolor}%
\pgftext[x=1.555000in,y=0.427778in,,top]{\color{textcolor}\rmfamily\fontsize{9.000000}{10.800000}\selectfont 2}%
\end{pgfscope}%
\begin{pgfscope}%
\definecolor{textcolor}{rgb}{0.000000,0.000000,0.000000}%
\pgfsetstrokecolor{textcolor}%
\pgfsetfillcolor{textcolor}%
\pgftext[x=1.118125in,y=0.251251in,,top]{\color{textcolor}\rmfamily\fontsize{9.000000}{10.800000}\selectfont \(\displaystyle \zeta\)}%
\end{pgfscope}%
\begin{pgfscope}%
\definecolor{textcolor}{rgb}{0.000000,0.000000,0.000000}%
\pgfsetstrokecolor{textcolor}%
\pgfsetfillcolor{textcolor}%
\pgftext[x=0.285833in,y=1.468750in,,bottom,rotate=90.000000]{\color{textcolor}\rmfamily\fontsize{9.000000}{10.800000}\selectfont \(\displaystyle m^2\)}%
\end{pgfscope}%
\begin{pgfscope}%
\pgfpathrectangle{\pgfqpoint{0.390000in}{0.525000in}}{\pgfqpoint{1.456250in}{1.887500in}}%
\pgfusepath{clip}%
\pgfsetbuttcap%
\pgfsetroundjoin%
\pgfsetlinewidth{1.003750pt}%
\definecolor{currentstroke}{rgb}{0.000000,0.000000,0.000000}%
\pgfsetstrokecolor{currentstroke}%
\pgfsetdash{}{0pt}%
\pgfpathmoveto{\pgfqpoint{0.390000in}{1.297159in}}%
\pgfpathlineto{\pgfqpoint{0.972500in}{1.297159in}}%
\pgfusepath{stroke}%
\end{pgfscope}%
\begin{pgfscope}%
\pgfpathrectangle{\pgfqpoint{0.390000in}{0.525000in}}{\pgfqpoint{1.456250in}{1.887500in}}%
\pgfusepath{clip}%
\pgfsetbuttcap%
\pgfsetroundjoin%
\pgfsetlinewidth{1.003750pt}%
\definecolor{currentstroke}{rgb}{0.000000,0.000000,0.000000}%
\pgfsetstrokecolor{currentstroke}%
\pgfsetdash{{3.700000pt}{1.600000pt}}{0.000000pt}%
\pgfpathmoveto{\pgfqpoint{0.972500in}{1.297159in}}%
\pgfpathlineto{\pgfqpoint{1.846250in}{1.297159in}}%
\pgfusepath{stroke}%
\end{pgfscope}%
\begin{pgfscope}%
\pgfpathrectangle{\pgfqpoint{0.390000in}{0.525000in}}{\pgfqpoint{1.456250in}{1.887500in}}%
\pgfusepath{clip}%
\pgfsetbuttcap%
\pgfsetroundjoin%
\pgfsetlinewidth{1.003750pt}%
\definecolor{currentstroke}{rgb}{0.000000,0.000000,0.000000}%
\pgfsetstrokecolor{currentstroke}%
\pgfsetdash{{3.700000pt}{1.600000pt}}{0.000000pt}%
\pgfpathmoveto{\pgfqpoint{0.390000in}{0.610795in}}%
\pgfpathlineto{\pgfqpoint{0.404710in}{0.628128in}}%
\pgfpathlineto{\pgfqpoint{0.419419in}{0.645460in}}%
\pgfpathlineto{\pgfqpoint{0.434129in}{0.662793in}}%
\pgfpathlineto{\pgfqpoint{0.448838in}{0.680125in}}%
\pgfpathlineto{\pgfqpoint{0.463548in}{0.697458in}}%
\pgfpathlineto{\pgfqpoint{0.478258in}{0.714790in}}%
\pgfpathlineto{\pgfqpoint{0.492967in}{0.732122in}}%
\pgfpathlineto{\pgfqpoint{0.507677in}{0.749455in}}%
\pgfpathlineto{\pgfqpoint{0.522386in}{0.766787in}}%
\pgfpathlineto{\pgfqpoint{0.537096in}{0.784120in}}%
\pgfpathlineto{\pgfqpoint{0.551806in}{0.801452in}}%
\pgfpathlineto{\pgfqpoint{0.566515in}{0.818784in}}%
\pgfpathlineto{\pgfqpoint{0.581225in}{0.836117in}}%
\pgfpathlineto{\pgfqpoint{0.595934in}{0.853449in}}%
\pgfpathlineto{\pgfqpoint{0.610644in}{0.870782in}}%
\pgfpathlineto{\pgfqpoint{0.625354in}{0.888114in}}%
\pgfpathlineto{\pgfqpoint{0.640063in}{0.905447in}}%
\pgfpathlineto{\pgfqpoint{0.654773in}{0.922779in}}%
\pgfpathlineto{\pgfqpoint{0.669482in}{0.940111in}}%
\pgfpathlineto{\pgfqpoint{0.684192in}{0.957444in}}%
\pgfpathlineto{\pgfqpoint{0.698902in}{0.974776in}}%
\pgfpathlineto{\pgfqpoint{0.713611in}{0.992109in}}%
\pgfpathlineto{\pgfqpoint{0.728321in}{1.009441in}}%
\pgfpathlineto{\pgfqpoint{0.743030in}{1.026773in}}%
\pgfpathlineto{\pgfqpoint{0.757740in}{1.044106in}}%
\pgfpathlineto{\pgfqpoint{0.772449in}{1.061438in}}%
\pgfpathlineto{\pgfqpoint{0.787159in}{1.078771in}}%
\pgfpathlineto{\pgfqpoint{0.801869in}{1.096103in}}%
\pgfpathlineto{\pgfqpoint{0.816578in}{1.113435in}}%
\pgfpathlineto{\pgfqpoint{0.831288in}{1.130768in}}%
\pgfpathlineto{\pgfqpoint{0.845997in}{1.148100in}}%
\pgfpathlineto{\pgfqpoint{0.860707in}{1.165433in}}%
\pgfpathlineto{\pgfqpoint{0.875417in}{1.182765in}}%
\pgfpathlineto{\pgfqpoint{0.890126in}{1.200098in}}%
\pgfpathlineto{\pgfqpoint{0.904836in}{1.217430in}}%
\pgfpathlineto{\pgfqpoint{0.919545in}{1.234762in}}%
\pgfpathlineto{\pgfqpoint{0.934255in}{1.252095in}}%
\pgfpathlineto{\pgfqpoint{0.948965in}{1.269427in}}%
\pgfpathlineto{\pgfqpoint{0.963674in}{1.286760in}}%
\pgfusepath{stroke}%
\end{pgfscope}%
\begin{pgfscope}%
\pgfpathrectangle{\pgfqpoint{0.390000in}{0.525000in}}{\pgfqpoint{1.456250in}{1.887500in}}%
\pgfusepath{clip}%
\pgfsetrectcap%
\pgfsetroundjoin%
\pgfsetlinewidth{1.003750pt}%
\definecolor{currentstroke}{rgb}{0.000000,0.000000,0.000000}%
\pgfsetstrokecolor{currentstroke}%
\pgfsetdash{}{0pt}%
\pgfpathmoveto{\pgfqpoint{0.978384in}{1.304092in}}%
\pgfpathlineto{\pgfqpoint{0.993093in}{1.321424in}}%
\pgfpathlineto{\pgfqpoint{1.007803in}{1.338757in}}%
\pgfpathlineto{\pgfqpoint{1.022513in}{1.356089in}}%
\pgfpathlineto{\pgfqpoint{1.037222in}{1.373422in}}%
\pgfpathlineto{\pgfqpoint{1.051932in}{1.390754in}}%
\pgfpathlineto{\pgfqpoint{1.066641in}{1.408087in}}%
\pgfpathlineto{\pgfqpoint{1.081351in}{1.425419in}}%
\pgfpathlineto{\pgfqpoint{1.096061in}{1.442751in}}%
\pgfpathlineto{\pgfqpoint{1.110770in}{1.460084in}}%
\pgfpathlineto{\pgfqpoint{1.125480in}{1.477416in}}%
\pgfpathlineto{\pgfqpoint{1.140189in}{1.494749in}}%
\pgfpathlineto{\pgfqpoint{1.154899in}{1.512081in}}%
\pgfpathlineto{\pgfqpoint{1.169609in}{1.529413in}}%
\pgfpathlineto{\pgfqpoint{1.184318in}{1.546746in}}%
\pgfpathlineto{\pgfqpoint{1.199028in}{1.564078in}}%
\pgfpathlineto{\pgfqpoint{1.213737in}{1.581411in}}%
\pgfpathlineto{\pgfqpoint{1.228447in}{1.598743in}}%
\pgfpathlineto{\pgfqpoint{1.243157in}{1.616076in}}%
\pgfpathlineto{\pgfqpoint{1.257866in}{1.633408in}}%
\pgfpathlineto{\pgfqpoint{1.272576in}{1.650740in}}%
\pgfpathlineto{\pgfqpoint{1.287285in}{1.668073in}}%
\pgfpathlineto{\pgfqpoint{1.301995in}{1.685405in}}%
\pgfpathlineto{\pgfqpoint{1.316705in}{1.702738in}}%
\pgfpathlineto{\pgfqpoint{1.331414in}{1.720070in}}%
\pgfpathlineto{\pgfqpoint{1.346124in}{1.737402in}}%
\pgfpathlineto{\pgfqpoint{1.360833in}{1.754735in}}%
\pgfpathlineto{\pgfqpoint{1.375543in}{1.772067in}}%
\pgfpathlineto{\pgfqpoint{1.390253in}{1.789400in}}%
\pgfpathlineto{\pgfqpoint{1.404962in}{1.806732in}}%
\pgfpathlineto{\pgfqpoint{1.419672in}{1.824065in}}%
\pgfpathlineto{\pgfqpoint{1.434381in}{1.841397in}}%
\pgfpathlineto{\pgfqpoint{1.449091in}{1.858729in}}%
\pgfpathlineto{\pgfqpoint{1.463801in}{1.876062in}}%
\pgfpathlineto{\pgfqpoint{1.478510in}{1.893394in}}%
\pgfpathlineto{\pgfqpoint{1.493220in}{1.910727in}}%
\pgfpathlineto{\pgfqpoint{1.507929in}{1.928059in}}%
\pgfpathlineto{\pgfqpoint{1.522639in}{1.945391in}}%
\pgfpathlineto{\pgfqpoint{1.537348in}{1.962724in}}%
\pgfpathlineto{\pgfqpoint{1.552058in}{1.980056in}}%
\pgfpathlineto{\pgfqpoint{1.566768in}{1.997389in}}%
\pgfpathlineto{\pgfqpoint{1.581477in}{2.014721in}}%
\pgfpathlineto{\pgfqpoint{1.596187in}{2.032053in}}%
\pgfpathlineto{\pgfqpoint{1.610896in}{2.049386in}}%
\pgfpathlineto{\pgfqpoint{1.625606in}{2.066718in}}%
\pgfpathlineto{\pgfqpoint{1.640316in}{2.084051in}}%
\pgfpathlineto{\pgfqpoint{1.655025in}{2.101383in}}%
\pgfpathlineto{\pgfqpoint{1.669735in}{2.118716in}}%
\pgfpathlineto{\pgfqpoint{1.684444in}{2.136048in}}%
\pgfpathlineto{\pgfqpoint{1.699154in}{2.153380in}}%
\pgfpathlineto{\pgfqpoint{1.713864in}{2.170713in}}%
\pgfpathlineto{\pgfqpoint{1.728573in}{2.188045in}}%
\pgfpathlineto{\pgfqpoint{1.743283in}{2.205378in}}%
\pgfpathlineto{\pgfqpoint{1.757992in}{2.222710in}}%
\pgfpathlineto{\pgfqpoint{1.772702in}{2.240042in}}%
\pgfpathlineto{\pgfqpoint{1.787412in}{2.257375in}}%
\pgfpathlineto{\pgfqpoint{1.802121in}{2.274707in}}%
\pgfpathlineto{\pgfqpoint{1.816831in}{2.292040in}}%
\pgfpathlineto{\pgfqpoint{1.831540in}{2.309372in}}%
\pgfpathlineto{\pgfqpoint{1.846250in}{2.326705in}}%
\pgfusepath{stroke}%
\end{pgfscope}%
\begin{pgfscope}%
\pgfpathrectangle{\pgfqpoint{0.390000in}{0.525000in}}{\pgfqpoint{1.456250in}{1.887500in}}%
\pgfusepath{clip}%
\pgfsetbuttcap%
\pgfsetroundjoin%
\definecolor{currentfill}{rgb}{0.000000,0.000000,0.000000}%
\pgfsetfillcolor{currentfill}%
\pgfsetlinewidth{1.003750pt}%
\definecolor{currentstroke}{rgb}{0.000000,0.000000,0.000000}%
\pgfsetstrokecolor{currentstroke}%
\pgfsetdash{}{0pt}%
\pgfsys@defobject{currentmarker}{\pgfqpoint{-0.027778in}{-0.027778in}}{\pgfqpoint{0.027778in}{0.027778in}}{%
\pgfpathmoveto{\pgfqpoint{0.000000in}{-0.027778in}}%
\pgfpathcurveto{\pgfqpoint{0.007367in}{-0.027778in}}{\pgfqpoint{0.014433in}{-0.024851in}}{\pgfqpoint{0.019642in}{-0.019642in}}%
\pgfpathcurveto{\pgfqpoint{0.024851in}{-0.014433in}}{\pgfqpoint{0.027778in}{-0.007367in}}{\pgfqpoint{0.027778in}{0.000000in}}%
\pgfpathcurveto{\pgfqpoint{0.027778in}{0.007367in}}{\pgfqpoint{0.024851in}{0.014433in}}{\pgfqpoint{0.019642in}{0.019642in}}%
\pgfpathcurveto{\pgfqpoint{0.014433in}{0.024851in}}{\pgfqpoint{0.007367in}{0.027778in}}{\pgfqpoint{0.000000in}{0.027778in}}%
\pgfpathcurveto{\pgfqpoint{-0.007367in}{0.027778in}}{\pgfqpoint{-0.014433in}{0.024851in}}{\pgfqpoint{-0.019642in}{0.019642in}}%
\pgfpathcurveto{\pgfqpoint{-0.024851in}{0.014433in}}{\pgfqpoint{-0.027778in}{0.007367in}}{\pgfqpoint{-0.027778in}{0.000000in}}%
\pgfpathcurveto{\pgfqpoint{-0.027778in}{-0.007367in}}{\pgfqpoint{-0.024851in}{-0.014433in}}{\pgfqpoint{-0.019642in}{-0.019642in}}%
\pgfpathcurveto{\pgfqpoint{-0.014433in}{-0.024851in}}{\pgfqpoint{-0.007367in}{-0.027778in}}{\pgfqpoint{0.000000in}{-0.027778in}}%
\pgfpathlineto{\pgfqpoint{0.000000in}{-0.027778in}}%
\pgfpathclose%
\pgfusepath{stroke,fill}%
}%
\begin{pgfscope}%
\pgfsys@transformshift{0.972500in}{1.297159in}%
\pgfsys@useobject{currentmarker}{}%
\end{pgfscope}%
\end{pgfscope}%
\begin{pgfscope}%
\pgfsetrectcap%
\pgfsetmiterjoin%
\pgfsetlinewidth{1.003750pt}%
\definecolor{currentstroke}{rgb}{0.000000,0.000000,0.000000}%
\pgfsetstrokecolor{currentstroke}%
\pgfsetdash{}{0pt}%
\pgfpathmoveto{\pgfqpoint{0.390000in}{0.525000in}}%
\pgfpathlineto{\pgfqpoint{0.390000in}{2.412500in}}%
\pgfusepath{stroke}%
\end{pgfscope}%
\begin{pgfscope}%
\pgfsetrectcap%
\pgfsetmiterjoin%
\pgfsetlinewidth{1.003750pt}%
\definecolor{currentstroke}{rgb}{0.000000,0.000000,0.000000}%
\pgfsetstrokecolor{currentstroke}%
\pgfsetdash{}{0pt}%
\pgfpathmoveto{\pgfqpoint{1.846250in}{0.525000in}}%
\pgfpathlineto{\pgfqpoint{1.846250in}{2.412500in}}%
\pgfusepath{stroke}%
\end{pgfscope}%
\begin{pgfscope}%
\pgfsetrectcap%
\pgfsetmiterjoin%
\pgfsetlinewidth{1.003750pt}%
\definecolor{currentstroke}{rgb}{0.000000,0.000000,0.000000}%
\pgfsetstrokecolor{currentstroke}%
\pgfsetdash{}{0pt}%
\pgfpathmoveto{\pgfqpoint{0.390000in}{0.525000in}}%
\pgfpathlineto{\pgfqpoint{1.846250in}{0.525000in}}%
\pgfusepath{stroke}%
\end{pgfscope}%
\begin{pgfscope}%
\pgfsetrectcap%
\pgfsetmiterjoin%
\pgfsetlinewidth{1.003750pt}%
\definecolor{currentstroke}{rgb}{0.000000,0.000000,0.000000}%
\pgfsetstrokecolor{currentstroke}%
\pgfsetdash{}{0pt}%
\pgfpathmoveto{\pgfqpoint{0.390000in}{2.412500in}}%
\pgfpathlineto{\pgfqpoint{1.846250in}{2.412500in}}%
\pgfusepath{stroke}%
\end{pgfscope}%
\begin{pgfscope}%
\definecolor{textcolor}{rgb}{0.000000,0.000000,0.000000}%
\pgfsetstrokecolor{textcolor}%
\pgfsetfillcolor{textcolor}%
\pgftext[x=1.555000in,y=1.297159in,left,bottom]{\color{textcolor}\rmfamily\fontsize{9.000000}{10.800000}\selectfont \(\displaystyle \nu_\mu\)}%
\end{pgfscope}%
\begin{pgfscope}%
\definecolor{textcolor}{rgb}{0.000000,0.000000,0.000000}%
\pgfsetstrokecolor{textcolor}%
\pgfsetfillcolor{textcolor}%
\pgftext[x=1.555000in,y=2.189432in,left,bottom]{\color{textcolor}\rmfamily\fontsize{9.000000}{10.800000}\selectfont \(\displaystyle \nu_e\)}%
\end{pgfscope}%
\begin{pgfscope}%
\definecolor{textcolor}{rgb}{0.000000,0.000000,0.000000}%
\pgfsetstrokecolor{textcolor}%
\pgfsetfillcolor{textcolor}%
\pgftext[x=0.448250in,y=2.258068in,left,base]{\color{textcolor}\rmfamily\fontsize{9.000000}{10.800000}\selectfont \(\displaystyle \rightarrow\) increasing \(\displaystyle \varrho\)}%
\end{pgfscope}%
\begin{pgfscope}%
\definecolor{textcolor}{rgb}{0.000000,0.000000,0.000000}%
\pgfsetstrokecolor{textcolor}%
\pgfsetfillcolor{textcolor}%
\pgftext[x=0.972500in,y=1.262841in,left,top]{\color{textcolor}\rmfamily\fontsize{6.246000}{7.495200}\selectfont Resonant conversion}%
\end{pgfscope}%
\end{pgfpicture}%
\makeatother%
\endgroup%

  \caption{Sketch of the MSW resonant conversion in the illustrative case of no
  vacuum mixing (i.e., when mass and flavor eigenstates coincide). Note that
  this phenomenon only takes place if $m_1 < m_2$, otherwise the two lines do not
  cross.}
  \label{fig:msw_conversion}
\end{marginfigure}

Of course the situation in real life is more complicated, as $\nu_e$ is a mix of
$\nu_1$ and $\nu_2$, and, in the presence of a varying electron density $n_e$, the
Shroendinger equation for the evolution cannot be diagonalized along the entire path
via a simple rotation. The basic phenomenology we have described here, though,
remains qualitatively valid in the \emph{adiabatic} limit (that is, when the change
of $n_e$ is small), in which case the mass eigenstates follow fixed trajectories in the
phase space, with no level crossing, and adiabatic conversion can still occur.
In this case the evolution equation needs to be solved numerically, and we shall see
and example of such solution at the end of the next section.


\subsection{The solution of the solar neutrino problem}

\todo{Start with a short recap at of the situation at the end of the 1990s.}
One of the fundamental limitations of the early solar neautrino experiments was the
lack of sensitivity to the neutrino flavors associated to the heavier leptons. Radiochemical
experiments exploited CC interactions, and were only sensitive to $\nu_e$; water
\cherenkov\ experiments exploiting the neutrino elastic scattering used both
CC and NC currents, but had limited sensitivity to $\nu_\mu$ and $\nu_\tau$.
It appeared clear that a possible key breakthrough to the solution of the solar
neutrino problem would be an experiment sensitive to \emph{all} neutrino types. The
elastic scattering off protons, for instance
\begin{align*}
  \nu_{e, \mu, \tau} + p \rightarrow  \nu_{e, \mu, \tau} + p
  \quad\text{(NC)}
\end{align*}
would be a sensible candidate, but one where the enormous background effectively
renders the approach hopeless: all we have a few-MeV proton in the final state,
which is higly non relativistic and therefore not collinear with the incoming neutrino,
and therefore would not allow to rely on the solar peak as in SK.

Deuterium is peculiar, in this respect, in that its nucleus is only loosely bound
(the binding energy is $\sim 2.2$~MeV) and, in addition to the charged-current reaction
\begin{align*}
  \nu_e + d \rightarrow e^- + p + p \quad\text{(CC)},
\end{align*}
with a threshold of 1.4~MeV, and the ordinary elastic neutrino scattering off the
atomic electrons
\begin{align*}
  \nu_{e, \mu, \tau} + e^- \rightarrow  \nu_{e, \mu, \tau} + e^-
  \quad\text{(ES)}
\end{align*}
is amenable to the inelastic scattering reaction, mediated by neutral current
\begin{align*}
  \nu_{e, \mu, \tau} + d \rightarrow  \nu_{e, \mu, \tau} + p + n
  \quad\text{(NC)}
\end{align*}
with a threshold of 2.2~MeV. In this case we have a neutron in the final state, that
can be detected directly or via neutron capture by either the deuton itself (followed
by the emission of a characteristic $\gamma$ ray of 6.25~MeV) or by some some additive
(e.g., NaCl, with the emission of a $\gamma$ of 8.6~MeV).

The Sudbury Neutrino Observatory (SNO), located 2100~m underground in Ontario, Canada,
was designed and operated between 1999 and 2006 with the precise intent of settling once
and for all the solar neutrino problem. The detector target was similar in spirit,
and technology, to SK, but the vessel was filled with heavy water---a form of water
whose hydrogen atoms are in fact both deuterium\sidenote{Procuring 1~kton of heavy
water is not a trivial undertake, and this one one of the difficulties connected
with the deployment of the experiment.}. For the CC and ES channel is the electron
in the final state being detected through its \cherenkov\ radiation, just like in
SK. For the (chiefly interesting) NC channel, the gamma ray in the final state can
Compton scatter off an atomic electron, with the latter having enough energy to trigger
the detector\sidenote{We note, though, that since the $\gamma$ is emitted isotropically,
the Compton electron bears no memory of the incoming direction of the neutrino, and
the same holds for the electron in CC channel. The only channel with directional
capabilities is the elastic scattering pioneered by SK.}

The SNO results~\cite{2004PhRvL..92r1301A} constitute the first contemporaneous
measurements of the solar $\nu_e$ flux (through the CC) and the \emph{total} neutrino
flux (through the NC) above an effective threshold of $\sim 5.5$~MeV, yielding
\begin{align*}
  \frac{\Phi_\text{CC}}{\Phi_\text{NC}} = 0.306 \pm 0.026~\text{(stat)} \pm 0.024~\text{(syst)}
\end{align*}
which is in agreement with the findings by the Homestake experiment. Most importantly,
a detailed analysis of all the available solar neutrino data showed that the SSM
could be reconciled with all the available information, providing that neutrino
oscillate and that the MSW mechanism is at play, the global best fits for the relevant
parameters being
\begin{align}
  \Delta m^2_\odot \approx 7.1 \times 10^{-5}~\text{eV}^2
  \quad\text{and}\quad
  \theta_\odot \approx 32.5^\circ
\end{align}

\todo{Include a combined plot of the electron survival probability, with a compilation
of different measurements. The one from Borexino Nature (2018) is a good starting
point.}

More recent measurements include those by Borexino, but it is fair to say that, by
the beginning of the years 2000s, the solar neutrino problem was solved and the
very basic facts that neutrino have mass and oscillate accepted. In the next section
we shall briefly present some more direct and complementary evidence obtained using
neutrinos produced by nuclear reactors.


\subsection{Verification with reactor neutrinos}

Reactor experiments typically consists of tanks of liquid scintillator read out by
photomultipliers and are designed to detect $\overline{\nu_e}$ (with a few MeV energy
on average) through the reaction
\begin{align*}
  \overline{\nu_e} + p \rightarrow e^+ + n
\end{align*}
whose signature is a prompt signal from the electrons and a delayed signal from
the gammas emitted when the neutron is captured after its termalization.

At a distance of $\sim 1$~km from the reactor, given the value of $\Delta m^2$ from
the solar neutrino experiments, the typical oscillation is $\sim 10^{-2}$, meaning
that near experiments are not sensitive to the mixing parameters between $\nu_1$
and $\nu_2$.  Much further from the reactor (say hundreds of km) the oscillation phase
at the MeV energy becomes $O(1)$ and experiments are sensitive to the $\nu_1\nu_2$
oscillation parameters. In fact Kamland observations~\cite{2005PhRvL..94h1801A} provide
further support to the LMA solution and, combined with the solar neutrino experiments,
allow to improve the accuracy of the measurement of $\theta_{12}$ and $\Delta m^2_{12}$.
\todo{Consider adding a plot from Kamland.}


\section{Atmospheric neutrinos}

We have already covered the reactions taking place in the atmosphere when hit from
primary cosmic rays, the dominant part of the decay chain involving neutrinos being
\begin{align*}
  \pi^+ \rightarrow \mu^+ + \bar{\nu}_\mu
  \quad\text{and}\quad
  \mu^+ \rightarrow e^+ + \nu_e + \bar{\nu}_\mu,
\end{align*}
along with the charge coniugates. At moderate energies, where the pions and the
muons have a small $\gamma$ factor and therefore tend to decay relatively quickly,
we have two muon neutrinos (or anti-neutrinos) for each electron neutrino (or anti-neutrino)
and one expects\sidenote{This is actually more accurately verified by detailed
Monte Carlo simulations than one would expect from a back-of-the envelope calculation.
In real life the key metrics is the double ration $R_\text{data} / R_\text{sim}$
where a number of systematic effects cancel out.}
\begin{align*}
  R = \frac{\nu_\mu + \bar{\nu}_\mu}{\nu_e + \bar{\nu}_e} \approx 2.
\end{align*}

\begin{marginfigure}
  %% Creator: Matplotlib, PGF backend
%%
%% To include the figure in your LaTeX document, write
%%   \input{<filename>.pgf}
%%
%% Make sure the required packages are loaded in your preamble
%%   \usepackage{pgf}
%%
%% Also ensure that all the required font packages are loaded; for instance,
%% the lmodern package is sometimes necessary when using math font.
%%   \usepackage{lmodern}
%%
%% Figures using additional raster images can only be included by \input if
%% they are in the same directory as the main LaTeX file. For loading figures
%% from other directories you can use the `import` package
%%   \usepackage{import}
%%
%% and then include the figures with
%%   \import{<path to file>}{<filename>.pgf}
%%
%% Matplotlib used the following preamble
%%   \usepackage{fontspec}
%%   \setmainfont{DejaVuSerif.ttf}[Path=\detokenize{/usr/share/matplotlib/mpl-data/fonts/ttf/}]
%%   \setsansfont{DejaVuSans.ttf}[Path=\detokenize{/usr/share/matplotlib/mpl-data/fonts/ttf/}]
%%   \setmonofont{DejaVuSansMono.ttf}[Path=\detokenize{/usr/share/matplotlib/mpl-data/fonts/ttf/}]
%%
\begingroup%
\makeatletter%
\begin{pgfpicture}%
\pgfpathrectangle{\pgfpointorigin}{\pgfqpoint{1.950000in}{2.200000in}}%
\pgfusepath{use as bounding box, clip}%
\begin{pgfscope}%
\pgfsetbuttcap%
\pgfsetmiterjoin%
\definecolor{currentfill}{rgb}{1.000000,1.000000,1.000000}%
\pgfsetfillcolor{currentfill}%
\pgfsetlinewidth{0.000000pt}%
\definecolor{currentstroke}{rgb}{1.000000,1.000000,1.000000}%
\pgfsetstrokecolor{currentstroke}%
\pgfsetdash{}{0pt}%
\pgfpathmoveto{\pgfqpoint{0.000000in}{0.000000in}}%
\pgfpathlineto{\pgfqpoint{1.950000in}{0.000000in}}%
\pgfpathlineto{\pgfqpoint{1.950000in}{2.200000in}}%
\pgfpathlineto{\pgfqpoint{0.000000in}{2.200000in}}%
\pgfpathlineto{\pgfqpoint{0.000000in}{0.000000in}}%
\pgfpathclose%
\pgfusepath{fill}%
\end{pgfscope}%
\begin{pgfscope}%
\pgfpathrectangle{\pgfqpoint{0.097500in}{0.233750in}}{\pgfqpoint{1.755000in}{1.755000in}}%
\pgfusepath{clip}%
\pgfsetbuttcap%
\pgfsetmiterjoin%
\pgfsetlinewidth{1.003750pt}%
\definecolor{currentstroke}{rgb}{0.000000,0.000000,0.000000}%
\pgfsetstrokecolor{currentstroke}%
\pgfsetdash{}{0pt}%
\pgfpathmoveto{\pgfqpoint{0.975000in}{0.359107in}}%
\pgfpathcurveto{\pgfqpoint{1.174471in}{0.359107in}}{\pgfqpoint{1.365798in}{0.438358in}}{\pgfqpoint{1.506845in}{0.579405in}}%
\pgfpathcurveto{\pgfqpoint{1.647892in}{0.720452in}}{\pgfqpoint{1.727143in}{0.911779in}}{\pgfqpoint{1.727143in}{1.111250in}}%
\pgfpathcurveto{\pgfqpoint{1.727143in}{1.310721in}}{\pgfqpoint{1.647892in}{1.502048in}}{\pgfqpoint{1.506845in}{1.643095in}}%
\pgfpathcurveto{\pgfqpoint{1.365798in}{1.784142in}}{\pgfqpoint{1.174471in}{1.863393in}}{\pgfqpoint{0.975000in}{1.863393in}}%
\pgfpathcurveto{\pgfqpoint{0.775529in}{1.863393in}}{\pgfqpoint{0.584202in}{1.784142in}}{\pgfqpoint{0.443155in}{1.643095in}}%
\pgfpathcurveto{\pgfqpoint{0.302108in}{1.502048in}}{\pgfqpoint{0.222857in}{1.310721in}}{\pgfqpoint{0.222857in}{1.111250in}}%
\pgfpathcurveto{\pgfqpoint{0.222857in}{0.911779in}}{\pgfqpoint{0.302108in}{0.720452in}}{\pgfqpoint{0.443155in}{0.579405in}}%
\pgfpathcurveto{\pgfqpoint{0.584202in}{0.438358in}}{\pgfqpoint{0.775529in}{0.359107in}}{\pgfqpoint{0.975000in}{0.359107in}}%
\pgfpathlineto{\pgfqpoint{0.975000in}{0.359107in}}%
\pgfpathclose%
\pgfusepath{stroke}%
\end{pgfscope}%
\begin{pgfscope}%
\pgfpathrectangle{\pgfqpoint{0.097500in}{0.233750in}}{\pgfqpoint{1.755000in}{1.755000in}}%
\pgfusepath{clip}%
\pgfsetbuttcap%
\pgfsetmiterjoin%
\pgfsetlinewidth{1.003750pt}%
\definecolor{currentstroke}{rgb}{0.000000,0.000000,0.000000}%
\pgfsetstrokecolor{currentstroke}%
\pgfsetdash{{3.700000pt}{1.600000pt}}{0.000000pt}%
\pgfpathmoveto{\pgfqpoint{0.975000in}{0.275536in}}%
\pgfpathcurveto{\pgfqpoint{1.196634in}{0.275536in}}{\pgfqpoint{1.409220in}{0.363592in}}{\pgfqpoint{1.565939in}{0.520311in}}%
\pgfpathcurveto{\pgfqpoint{1.722658in}{0.677030in}}{\pgfqpoint{1.810714in}{0.889616in}}{\pgfqpoint{1.810714in}{1.111250in}}%
\pgfpathcurveto{\pgfqpoint{1.810714in}{1.332884in}}{\pgfqpoint{1.722658in}{1.545470in}}{\pgfqpoint{1.565939in}{1.702189in}}%
\pgfpathcurveto{\pgfqpoint{1.409220in}{1.858908in}}{\pgfqpoint{1.196634in}{1.946964in}}{\pgfqpoint{0.975000in}{1.946964in}}%
\pgfpathcurveto{\pgfqpoint{0.753366in}{1.946964in}}{\pgfqpoint{0.540780in}{1.858908in}}{\pgfqpoint{0.384061in}{1.702189in}}%
\pgfpathcurveto{\pgfqpoint{0.227342in}{1.545470in}}{\pgfqpoint{0.139286in}{1.332884in}}{\pgfqpoint{0.139286in}{1.111250in}}%
\pgfpathcurveto{\pgfqpoint{0.139286in}{0.889616in}}{\pgfqpoint{0.227342in}{0.677030in}}{\pgfqpoint{0.384061in}{0.520311in}}%
\pgfpathcurveto{\pgfqpoint{0.540780in}{0.363592in}}{\pgfqpoint{0.753366in}{0.275536in}}{\pgfqpoint{0.975000in}{0.275536in}}%
\pgfpathlineto{\pgfqpoint{0.975000in}{0.275536in}}%
\pgfpathclose%
\pgfusepath{stroke}%
\end{pgfscope}%
\end{pgfpicture}%
\makeatother%
\endgroup%

  \caption{Sketch of an underground neutrino detector (not to scale). The Dashed
  line represents the average altitude where neutrinos are created within the
  Earth atmosphere.}
  \label{fig:atmospheric_nu_sketch}
\end{marginfigure}

Now, there are two notable differences between the solar and atmospheric neutrinos,
beside the fact that in the first case we start with 100\% $\nu_e$, while in the latter
we start with a mixture of $\nu_e$ and $\nu_\mu$: the energy is now significantly
larger (we area dealing with a continuous spectrum peaking at the GeV scale), and
the distances traveled by the neutrino from the creation to the detection site are
much shorter. Atmospheric neutrinos are created in the high atmosphere and, since
a typical detector has nearly full ($4\pi$) coverage, the actual path length depends
on the arrival direction with respect to the local zenith angle and can vary from
$\sim 10$~km (for neutrinos arriving from above) to more than $10,000$~km (for neutrinos
arriving from below and traversing the Earth), in any case much less than the A.~U.
that separates the Earth from the Sun. We should also note the the overall dynamics
in terms of path length, here, is about three orders of magnitude, which is also
much larger than the equivalent for Solar neutrinos.

\begin{marginfigure}
  %% Creator: Matplotlib, PGF backend
%%
%% To include the figure in your LaTeX document, write
%%   \input{<filename>.pgf}
%%
%% Make sure the required packages are loaded in your preamble
%%   \usepackage{pgf}
%%
%% Also ensure that all the required font packages are loaded; for instance,
%% the lmodern package is sometimes necessary when using math font.
%%   \usepackage{lmodern}
%%
%% Figures using additional raster images can only be included by \input if
%% they are in the same directory as the main LaTeX file. For loading figures
%% from other directories you can use the `import` package
%%   \usepackage{import}
%%
%% and then include the figures with
%%   \import{<path to file>}{<filename>.pgf}
%%
%% Matplotlib used the following preamble
%%   \usepackage{fontspec}
%%   \setmainfont{DejaVuSerif.ttf}[Path=\detokenize{/usr/share/matplotlib/mpl-data/fonts/ttf/}]
%%   \setsansfont{DejaVuSans.ttf}[Path=\detokenize{/usr/share/matplotlib/mpl-data/fonts/ttf/}]
%%   \setmonofont{DejaVuSansMono.ttf}[Path=\detokenize{/usr/share/matplotlib/mpl-data/fonts/ttf/}]
%%
\begingroup%
\makeatletter%
\begin{pgfpicture}%
\pgfpathrectangle{\pgfpointorigin}{\pgfqpoint{1.950000in}{2.250000in}}%
\pgfusepath{use as bounding box, clip}%
\begin{pgfscope}%
\pgfsetbuttcap%
\pgfsetmiterjoin%
\definecolor{currentfill}{rgb}{1.000000,1.000000,1.000000}%
\pgfsetfillcolor{currentfill}%
\pgfsetlinewidth{0.000000pt}%
\definecolor{currentstroke}{rgb}{1.000000,1.000000,1.000000}%
\pgfsetstrokecolor{currentstroke}%
\pgfsetdash{}{0pt}%
\pgfpathmoveto{\pgfqpoint{0.000000in}{0.000000in}}%
\pgfpathlineto{\pgfqpoint{1.950000in}{0.000000in}}%
\pgfpathlineto{\pgfqpoint{1.950000in}{2.250000in}}%
\pgfpathlineto{\pgfqpoint{0.000000in}{2.250000in}}%
\pgfpathlineto{\pgfqpoint{0.000000in}{0.000000in}}%
\pgfpathclose%
\pgfusepath{fill}%
\end{pgfscope}%
\begin{pgfscope}%
\pgfsetbuttcap%
\pgfsetmiterjoin%
\definecolor{currentfill}{rgb}{1.000000,1.000000,1.000000}%
\pgfsetfillcolor{currentfill}%
\pgfsetlinewidth{0.000000pt}%
\definecolor{currentstroke}{rgb}{0.000000,0.000000,0.000000}%
\pgfsetstrokecolor{currentstroke}%
\pgfsetstrokeopacity{0.000000}%
\pgfsetdash{}{0pt}%
\pgfpathmoveto{\pgfqpoint{0.726250in}{0.525000in}}%
\pgfpathlineto{\pgfqpoint{1.846250in}{0.525000in}}%
\pgfpathlineto{\pgfqpoint{1.846250in}{2.162500in}}%
\pgfpathlineto{\pgfqpoint{0.726250in}{2.162500in}}%
\pgfpathlineto{\pgfqpoint{0.726250in}{0.525000in}}%
\pgfpathclose%
\pgfusepath{fill}%
\end{pgfscope}%
\begin{pgfscope}%
\pgfpathrectangle{\pgfqpoint{0.726250in}{0.525000in}}{\pgfqpoint{1.120000in}{1.637500in}}%
\pgfusepath{clip}%
\pgfsetbuttcap%
\pgfsetroundjoin%
\pgfsetlinewidth{0.803000pt}%
\definecolor{currentstroke}{rgb}{0.752941,0.752941,0.752941}%
\pgfsetstrokecolor{currentstroke}%
\pgfsetdash{{2.960000pt}{1.280000pt}}{0.000000pt}%
\pgfpathmoveto{\pgfqpoint{0.726250in}{0.525000in}}%
\pgfpathlineto{\pgfqpoint{0.726250in}{2.162500in}}%
\pgfusepath{stroke}%
\end{pgfscope}%
\begin{pgfscope}%
\pgfsetbuttcap%
\pgfsetroundjoin%
\definecolor{currentfill}{rgb}{0.000000,0.000000,0.000000}%
\pgfsetfillcolor{currentfill}%
\pgfsetlinewidth{0.803000pt}%
\definecolor{currentstroke}{rgb}{0.000000,0.000000,0.000000}%
\pgfsetstrokecolor{currentstroke}%
\pgfsetdash{}{0pt}%
\pgfsys@defobject{currentmarker}{\pgfqpoint{0.000000in}{-0.048611in}}{\pgfqpoint{0.000000in}{0.000000in}}{%
\pgfpathmoveto{\pgfqpoint{0.000000in}{0.000000in}}%
\pgfpathlineto{\pgfqpoint{0.000000in}{-0.048611in}}%
\pgfusepath{stroke,fill}%
}%
\begin{pgfscope}%
\pgfsys@transformshift{0.726250in}{0.525000in}%
\pgfsys@useobject{currentmarker}{}%
\end{pgfscope}%
\end{pgfscope}%
\begin{pgfscope}%
\definecolor{textcolor}{rgb}{0.000000,0.000000,0.000000}%
\pgfsetstrokecolor{textcolor}%
\pgfsetfillcolor{textcolor}%
\pgftext[x=0.726250in,y=0.427778in,,top]{\color{textcolor}\rmfamily\fontsize{9.000000}{10.800000}\selectfont \ensuremath{-}1}%
\end{pgfscope}%
\begin{pgfscope}%
\pgfpathrectangle{\pgfqpoint{0.726250in}{0.525000in}}{\pgfqpoint{1.120000in}{1.637500in}}%
\pgfusepath{clip}%
\pgfsetbuttcap%
\pgfsetroundjoin%
\pgfsetlinewidth{0.803000pt}%
\definecolor{currentstroke}{rgb}{0.752941,0.752941,0.752941}%
\pgfsetstrokecolor{currentstroke}%
\pgfsetdash{{2.960000pt}{1.280000pt}}{0.000000pt}%
\pgfpathmoveto{\pgfqpoint{1.286250in}{0.525000in}}%
\pgfpathlineto{\pgfqpoint{1.286250in}{2.162500in}}%
\pgfusepath{stroke}%
\end{pgfscope}%
\begin{pgfscope}%
\pgfsetbuttcap%
\pgfsetroundjoin%
\definecolor{currentfill}{rgb}{0.000000,0.000000,0.000000}%
\pgfsetfillcolor{currentfill}%
\pgfsetlinewidth{0.803000pt}%
\definecolor{currentstroke}{rgb}{0.000000,0.000000,0.000000}%
\pgfsetstrokecolor{currentstroke}%
\pgfsetdash{}{0pt}%
\pgfsys@defobject{currentmarker}{\pgfqpoint{0.000000in}{-0.048611in}}{\pgfqpoint{0.000000in}{0.000000in}}{%
\pgfpathmoveto{\pgfqpoint{0.000000in}{0.000000in}}%
\pgfpathlineto{\pgfqpoint{0.000000in}{-0.048611in}}%
\pgfusepath{stroke,fill}%
}%
\begin{pgfscope}%
\pgfsys@transformshift{1.286250in}{0.525000in}%
\pgfsys@useobject{currentmarker}{}%
\end{pgfscope}%
\end{pgfscope}%
\begin{pgfscope}%
\definecolor{textcolor}{rgb}{0.000000,0.000000,0.000000}%
\pgfsetstrokecolor{textcolor}%
\pgfsetfillcolor{textcolor}%
\pgftext[x=1.286250in,y=0.427778in,,top]{\color{textcolor}\rmfamily\fontsize{9.000000}{10.800000}\selectfont 0}%
\end{pgfscope}%
\begin{pgfscope}%
\pgfpathrectangle{\pgfqpoint{0.726250in}{0.525000in}}{\pgfqpoint{1.120000in}{1.637500in}}%
\pgfusepath{clip}%
\pgfsetbuttcap%
\pgfsetroundjoin%
\pgfsetlinewidth{0.803000pt}%
\definecolor{currentstroke}{rgb}{0.752941,0.752941,0.752941}%
\pgfsetstrokecolor{currentstroke}%
\pgfsetdash{{2.960000pt}{1.280000pt}}{0.000000pt}%
\pgfpathmoveto{\pgfqpoint{1.846250in}{0.525000in}}%
\pgfpathlineto{\pgfqpoint{1.846250in}{2.162500in}}%
\pgfusepath{stroke}%
\end{pgfscope}%
\begin{pgfscope}%
\pgfsetbuttcap%
\pgfsetroundjoin%
\definecolor{currentfill}{rgb}{0.000000,0.000000,0.000000}%
\pgfsetfillcolor{currentfill}%
\pgfsetlinewidth{0.803000pt}%
\definecolor{currentstroke}{rgb}{0.000000,0.000000,0.000000}%
\pgfsetstrokecolor{currentstroke}%
\pgfsetdash{}{0pt}%
\pgfsys@defobject{currentmarker}{\pgfqpoint{0.000000in}{-0.048611in}}{\pgfqpoint{0.000000in}{0.000000in}}{%
\pgfpathmoveto{\pgfqpoint{0.000000in}{0.000000in}}%
\pgfpathlineto{\pgfqpoint{0.000000in}{-0.048611in}}%
\pgfusepath{stroke,fill}%
}%
\begin{pgfscope}%
\pgfsys@transformshift{1.846250in}{0.525000in}%
\pgfsys@useobject{currentmarker}{}%
\end{pgfscope}%
\end{pgfscope}%
\begin{pgfscope}%
\definecolor{textcolor}{rgb}{0.000000,0.000000,0.000000}%
\pgfsetstrokecolor{textcolor}%
\pgfsetfillcolor{textcolor}%
\pgftext[x=1.846250in,y=0.427778in,,top]{\color{textcolor}\rmfamily\fontsize{9.000000}{10.800000}\selectfont 1}%
\end{pgfscope}%
\begin{pgfscope}%
\definecolor{textcolor}{rgb}{0.000000,0.000000,0.000000}%
\pgfsetstrokecolor{textcolor}%
\pgfsetfillcolor{textcolor}%
\pgftext[x=1.286250in,y=0.251251in,,top]{\color{textcolor}\rmfamily\fontsize{9.000000}{10.800000}\selectfont \(\displaystyle \cos\theta\)}%
\end{pgfscope}%
\begin{pgfscope}%
\pgfpathrectangle{\pgfqpoint{0.726250in}{0.525000in}}{\pgfqpoint{1.120000in}{1.637500in}}%
\pgfusepath{clip}%
\pgfsetbuttcap%
\pgfsetroundjoin%
\pgfsetlinewidth{0.803000pt}%
\definecolor{currentstroke}{rgb}{0.752941,0.752941,0.752941}%
\pgfsetstrokecolor{currentstroke}%
\pgfsetdash{{2.960000pt}{1.280000pt}}{0.000000pt}%
\pgfpathmoveto{\pgfqpoint{0.726250in}{0.993741in}}%
\pgfpathlineto{\pgfqpoint{1.846250in}{0.993741in}}%
\pgfusepath{stroke}%
\end{pgfscope}%
\begin{pgfscope}%
\pgfsetbuttcap%
\pgfsetroundjoin%
\definecolor{currentfill}{rgb}{0.000000,0.000000,0.000000}%
\pgfsetfillcolor{currentfill}%
\pgfsetlinewidth{0.803000pt}%
\definecolor{currentstroke}{rgb}{0.000000,0.000000,0.000000}%
\pgfsetstrokecolor{currentstroke}%
\pgfsetdash{}{0pt}%
\pgfsys@defobject{currentmarker}{\pgfqpoint{-0.048611in}{0.000000in}}{\pgfqpoint{-0.000000in}{0.000000in}}{%
\pgfpathmoveto{\pgfqpoint{-0.000000in}{0.000000in}}%
\pgfpathlineto{\pgfqpoint{-0.048611in}{0.000000in}}%
\pgfusepath{stroke,fill}%
}%
\begin{pgfscope}%
\pgfsys@transformshift{0.726250in}{0.993741in}%
\pgfsys@useobject{currentmarker}{}%
\end{pgfscope}%
\end{pgfscope}%
\begin{pgfscope}%
\definecolor{textcolor}{rgb}{0.000000,0.000000,0.000000}%
\pgfsetstrokecolor{textcolor}%
\pgfsetfillcolor{textcolor}%
\pgftext[x=0.442687in, y=0.946255in, left, base]{\color{textcolor}\rmfamily\fontsize{9.000000}{10.800000}\selectfont \(\displaystyle {10^{2}}\)}%
\end{pgfscope}%
\begin{pgfscope}%
\pgfpathrectangle{\pgfqpoint{0.726250in}{0.525000in}}{\pgfqpoint{1.120000in}{1.637500in}}%
\pgfusepath{clip}%
\pgfsetbuttcap%
\pgfsetroundjoin%
\pgfsetlinewidth{0.803000pt}%
\definecolor{currentstroke}{rgb}{0.752941,0.752941,0.752941}%
\pgfsetstrokecolor{currentstroke}%
\pgfsetdash{{2.960000pt}{1.280000pt}}{0.000000pt}%
\pgfpathmoveto{\pgfqpoint{0.726250in}{1.514650in}}%
\pgfpathlineto{\pgfqpoint{1.846250in}{1.514650in}}%
\pgfusepath{stroke}%
\end{pgfscope}%
\begin{pgfscope}%
\pgfsetbuttcap%
\pgfsetroundjoin%
\definecolor{currentfill}{rgb}{0.000000,0.000000,0.000000}%
\pgfsetfillcolor{currentfill}%
\pgfsetlinewidth{0.803000pt}%
\definecolor{currentstroke}{rgb}{0.000000,0.000000,0.000000}%
\pgfsetstrokecolor{currentstroke}%
\pgfsetdash{}{0pt}%
\pgfsys@defobject{currentmarker}{\pgfqpoint{-0.048611in}{0.000000in}}{\pgfqpoint{-0.000000in}{0.000000in}}{%
\pgfpathmoveto{\pgfqpoint{-0.000000in}{0.000000in}}%
\pgfpathlineto{\pgfqpoint{-0.048611in}{0.000000in}}%
\pgfusepath{stroke,fill}%
}%
\begin{pgfscope}%
\pgfsys@transformshift{0.726250in}{1.514650in}%
\pgfsys@useobject{currentmarker}{}%
\end{pgfscope}%
\end{pgfscope}%
\begin{pgfscope}%
\definecolor{textcolor}{rgb}{0.000000,0.000000,0.000000}%
\pgfsetstrokecolor{textcolor}%
\pgfsetfillcolor{textcolor}%
\pgftext[x=0.442687in, y=1.467165in, left, base]{\color{textcolor}\rmfamily\fontsize{9.000000}{10.800000}\selectfont \(\displaystyle {10^{3}}\)}%
\end{pgfscope}%
\begin{pgfscope}%
\pgfpathrectangle{\pgfqpoint{0.726250in}{0.525000in}}{\pgfqpoint{1.120000in}{1.637500in}}%
\pgfusepath{clip}%
\pgfsetbuttcap%
\pgfsetroundjoin%
\pgfsetlinewidth{0.803000pt}%
\definecolor{currentstroke}{rgb}{0.752941,0.752941,0.752941}%
\pgfsetstrokecolor{currentstroke}%
\pgfsetdash{{2.960000pt}{1.280000pt}}{0.000000pt}%
\pgfpathmoveto{\pgfqpoint{0.726250in}{2.035560in}}%
\pgfpathlineto{\pgfqpoint{1.846250in}{2.035560in}}%
\pgfusepath{stroke}%
\end{pgfscope}%
\begin{pgfscope}%
\pgfsetbuttcap%
\pgfsetroundjoin%
\definecolor{currentfill}{rgb}{0.000000,0.000000,0.000000}%
\pgfsetfillcolor{currentfill}%
\pgfsetlinewidth{0.803000pt}%
\definecolor{currentstroke}{rgb}{0.000000,0.000000,0.000000}%
\pgfsetstrokecolor{currentstroke}%
\pgfsetdash{}{0pt}%
\pgfsys@defobject{currentmarker}{\pgfqpoint{-0.048611in}{0.000000in}}{\pgfqpoint{-0.000000in}{0.000000in}}{%
\pgfpathmoveto{\pgfqpoint{-0.000000in}{0.000000in}}%
\pgfpathlineto{\pgfqpoint{-0.048611in}{0.000000in}}%
\pgfusepath{stroke,fill}%
}%
\begin{pgfscope}%
\pgfsys@transformshift{0.726250in}{2.035560in}%
\pgfsys@useobject{currentmarker}{}%
\end{pgfscope}%
\end{pgfscope}%
\begin{pgfscope}%
\definecolor{textcolor}{rgb}{0.000000,0.000000,0.000000}%
\pgfsetstrokecolor{textcolor}%
\pgfsetfillcolor{textcolor}%
\pgftext[x=0.442687in, y=1.988075in, left, base]{\color{textcolor}\rmfamily\fontsize{9.000000}{10.800000}\selectfont \(\displaystyle {10^{4}}\)}%
\end{pgfscope}%
\begin{pgfscope}%
\pgfpathrectangle{\pgfqpoint{0.726250in}{0.525000in}}{\pgfqpoint{1.120000in}{1.637500in}}%
\pgfusepath{clip}%
\pgfsetbuttcap%
\pgfsetroundjoin%
\pgfsetlinewidth{0.803000pt}%
\definecolor{currentstroke}{rgb}{0.752941,0.752941,0.752941}%
\pgfsetstrokecolor{currentstroke}%
\pgfsetdash{{2.960000pt}{1.280000pt}}{0.000000pt}%
\pgfpathmoveto{\pgfqpoint{0.726250in}{0.629640in}}%
\pgfpathlineto{\pgfqpoint{1.846250in}{0.629640in}}%
\pgfusepath{stroke}%
\end{pgfscope}%
\begin{pgfscope}%
\pgfsetbuttcap%
\pgfsetroundjoin%
\definecolor{currentfill}{rgb}{0.000000,0.000000,0.000000}%
\pgfsetfillcolor{currentfill}%
\pgfsetlinewidth{0.602250pt}%
\definecolor{currentstroke}{rgb}{0.000000,0.000000,0.000000}%
\pgfsetstrokecolor{currentstroke}%
\pgfsetdash{}{0pt}%
\pgfsys@defobject{currentmarker}{\pgfqpoint{-0.027778in}{0.000000in}}{\pgfqpoint{-0.000000in}{0.000000in}}{%
\pgfpathmoveto{\pgfqpoint{-0.000000in}{0.000000in}}%
\pgfpathlineto{\pgfqpoint{-0.027778in}{0.000000in}}%
\pgfusepath{stroke,fill}%
}%
\begin{pgfscope}%
\pgfsys@transformshift{0.726250in}{0.629640in}%
\pgfsys@useobject{currentmarker}{}%
\end{pgfscope}%
\end{pgfscope}%
\begin{pgfscope}%
\pgfpathrectangle{\pgfqpoint{0.726250in}{0.525000in}}{\pgfqpoint{1.120000in}{1.637500in}}%
\pgfusepath{clip}%
\pgfsetbuttcap%
\pgfsetroundjoin%
\pgfsetlinewidth{0.803000pt}%
\definecolor{currentstroke}{rgb}{0.752941,0.752941,0.752941}%
\pgfsetstrokecolor{currentstroke}%
\pgfsetdash{{2.960000pt}{1.280000pt}}{0.000000pt}%
\pgfpathmoveto{\pgfqpoint{0.726250in}{0.721368in}}%
\pgfpathlineto{\pgfqpoint{1.846250in}{0.721368in}}%
\pgfusepath{stroke}%
\end{pgfscope}%
\begin{pgfscope}%
\pgfsetbuttcap%
\pgfsetroundjoin%
\definecolor{currentfill}{rgb}{0.000000,0.000000,0.000000}%
\pgfsetfillcolor{currentfill}%
\pgfsetlinewidth{0.602250pt}%
\definecolor{currentstroke}{rgb}{0.000000,0.000000,0.000000}%
\pgfsetstrokecolor{currentstroke}%
\pgfsetdash{}{0pt}%
\pgfsys@defobject{currentmarker}{\pgfqpoint{-0.027778in}{0.000000in}}{\pgfqpoint{-0.000000in}{0.000000in}}{%
\pgfpathmoveto{\pgfqpoint{-0.000000in}{0.000000in}}%
\pgfpathlineto{\pgfqpoint{-0.027778in}{0.000000in}}%
\pgfusepath{stroke,fill}%
}%
\begin{pgfscope}%
\pgfsys@transformshift{0.726250in}{0.721368in}%
\pgfsys@useobject{currentmarker}{}%
\end{pgfscope}%
\end{pgfscope}%
\begin{pgfscope}%
\pgfpathrectangle{\pgfqpoint{0.726250in}{0.525000in}}{\pgfqpoint{1.120000in}{1.637500in}}%
\pgfusepath{clip}%
\pgfsetbuttcap%
\pgfsetroundjoin%
\pgfsetlinewidth{0.803000pt}%
\definecolor{currentstroke}{rgb}{0.752941,0.752941,0.752941}%
\pgfsetstrokecolor{currentstroke}%
\pgfsetdash{{2.960000pt}{1.280000pt}}{0.000000pt}%
\pgfpathmoveto{\pgfqpoint{0.726250in}{0.786450in}}%
\pgfpathlineto{\pgfqpoint{1.846250in}{0.786450in}}%
\pgfusepath{stroke}%
\end{pgfscope}%
\begin{pgfscope}%
\pgfsetbuttcap%
\pgfsetroundjoin%
\definecolor{currentfill}{rgb}{0.000000,0.000000,0.000000}%
\pgfsetfillcolor{currentfill}%
\pgfsetlinewidth{0.602250pt}%
\definecolor{currentstroke}{rgb}{0.000000,0.000000,0.000000}%
\pgfsetstrokecolor{currentstroke}%
\pgfsetdash{}{0pt}%
\pgfsys@defobject{currentmarker}{\pgfqpoint{-0.027778in}{0.000000in}}{\pgfqpoint{-0.000000in}{0.000000in}}{%
\pgfpathmoveto{\pgfqpoint{-0.000000in}{0.000000in}}%
\pgfpathlineto{\pgfqpoint{-0.027778in}{0.000000in}}%
\pgfusepath{stroke,fill}%
}%
\begin{pgfscope}%
\pgfsys@transformshift{0.726250in}{0.786450in}%
\pgfsys@useobject{currentmarker}{}%
\end{pgfscope}%
\end{pgfscope}%
\begin{pgfscope}%
\pgfpathrectangle{\pgfqpoint{0.726250in}{0.525000in}}{\pgfqpoint{1.120000in}{1.637500in}}%
\pgfusepath{clip}%
\pgfsetbuttcap%
\pgfsetroundjoin%
\pgfsetlinewidth{0.803000pt}%
\definecolor{currentstroke}{rgb}{0.752941,0.752941,0.752941}%
\pgfsetstrokecolor{currentstroke}%
\pgfsetdash{{2.960000pt}{1.280000pt}}{0.000000pt}%
\pgfpathmoveto{\pgfqpoint{0.726250in}{0.836931in}}%
\pgfpathlineto{\pgfqpoint{1.846250in}{0.836931in}}%
\pgfusepath{stroke}%
\end{pgfscope}%
\begin{pgfscope}%
\pgfsetbuttcap%
\pgfsetroundjoin%
\definecolor{currentfill}{rgb}{0.000000,0.000000,0.000000}%
\pgfsetfillcolor{currentfill}%
\pgfsetlinewidth{0.602250pt}%
\definecolor{currentstroke}{rgb}{0.000000,0.000000,0.000000}%
\pgfsetstrokecolor{currentstroke}%
\pgfsetdash{}{0pt}%
\pgfsys@defobject{currentmarker}{\pgfqpoint{-0.027778in}{0.000000in}}{\pgfqpoint{-0.000000in}{0.000000in}}{%
\pgfpathmoveto{\pgfqpoint{-0.000000in}{0.000000in}}%
\pgfpathlineto{\pgfqpoint{-0.027778in}{0.000000in}}%
\pgfusepath{stroke,fill}%
}%
\begin{pgfscope}%
\pgfsys@transformshift{0.726250in}{0.836931in}%
\pgfsys@useobject{currentmarker}{}%
\end{pgfscope}%
\end{pgfscope}%
\begin{pgfscope}%
\pgfpathrectangle{\pgfqpoint{0.726250in}{0.525000in}}{\pgfqpoint{1.120000in}{1.637500in}}%
\pgfusepath{clip}%
\pgfsetbuttcap%
\pgfsetroundjoin%
\pgfsetlinewidth{0.803000pt}%
\definecolor{currentstroke}{rgb}{0.752941,0.752941,0.752941}%
\pgfsetstrokecolor{currentstroke}%
\pgfsetdash{{2.960000pt}{1.280000pt}}{0.000000pt}%
\pgfpathmoveto{\pgfqpoint{0.726250in}{0.878177in}}%
\pgfpathlineto{\pgfqpoint{1.846250in}{0.878177in}}%
\pgfusepath{stroke}%
\end{pgfscope}%
\begin{pgfscope}%
\pgfsetbuttcap%
\pgfsetroundjoin%
\definecolor{currentfill}{rgb}{0.000000,0.000000,0.000000}%
\pgfsetfillcolor{currentfill}%
\pgfsetlinewidth{0.602250pt}%
\definecolor{currentstroke}{rgb}{0.000000,0.000000,0.000000}%
\pgfsetstrokecolor{currentstroke}%
\pgfsetdash{}{0pt}%
\pgfsys@defobject{currentmarker}{\pgfqpoint{-0.027778in}{0.000000in}}{\pgfqpoint{-0.000000in}{0.000000in}}{%
\pgfpathmoveto{\pgfqpoint{-0.000000in}{0.000000in}}%
\pgfpathlineto{\pgfqpoint{-0.027778in}{0.000000in}}%
\pgfusepath{stroke,fill}%
}%
\begin{pgfscope}%
\pgfsys@transformshift{0.726250in}{0.878177in}%
\pgfsys@useobject{currentmarker}{}%
\end{pgfscope}%
\end{pgfscope}%
\begin{pgfscope}%
\pgfpathrectangle{\pgfqpoint{0.726250in}{0.525000in}}{\pgfqpoint{1.120000in}{1.637500in}}%
\pgfusepath{clip}%
\pgfsetbuttcap%
\pgfsetroundjoin%
\pgfsetlinewidth{0.803000pt}%
\definecolor{currentstroke}{rgb}{0.752941,0.752941,0.752941}%
\pgfsetstrokecolor{currentstroke}%
\pgfsetdash{{2.960000pt}{1.280000pt}}{0.000000pt}%
\pgfpathmoveto{\pgfqpoint{0.726250in}{0.913051in}}%
\pgfpathlineto{\pgfqpoint{1.846250in}{0.913051in}}%
\pgfusepath{stroke}%
\end{pgfscope}%
\begin{pgfscope}%
\pgfsetbuttcap%
\pgfsetroundjoin%
\definecolor{currentfill}{rgb}{0.000000,0.000000,0.000000}%
\pgfsetfillcolor{currentfill}%
\pgfsetlinewidth{0.602250pt}%
\definecolor{currentstroke}{rgb}{0.000000,0.000000,0.000000}%
\pgfsetstrokecolor{currentstroke}%
\pgfsetdash{}{0pt}%
\pgfsys@defobject{currentmarker}{\pgfqpoint{-0.027778in}{0.000000in}}{\pgfqpoint{-0.000000in}{0.000000in}}{%
\pgfpathmoveto{\pgfqpoint{-0.000000in}{0.000000in}}%
\pgfpathlineto{\pgfqpoint{-0.027778in}{0.000000in}}%
\pgfusepath{stroke,fill}%
}%
\begin{pgfscope}%
\pgfsys@transformshift{0.726250in}{0.913051in}%
\pgfsys@useobject{currentmarker}{}%
\end{pgfscope}%
\end{pgfscope}%
\begin{pgfscope}%
\pgfpathrectangle{\pgfqpoint{0.726250in}{0.525000in}}{\pgfqpoint{1.120000in}{1.637500in}}%
\pgfusepath{clip}%
\pgfsetbuttcap%
\pgfsetroundjoin%
\pgfsetlinewidth{0.803000pt}%
\definecolor{currentstroke}{rgb}{0.752941,0.752941,0.752941}%
\pgfsetstrokecolor{currentstroke}%
\pgfsetdash{{2.960000pt}{1.280000pt}}{0.000000pt}%
\pgfpathmoveto{\pgfqpoint{0.726250in}{0.943259in}}%
\pgfpathlineto{\pgfqpoint{1.846250in}{0.943259in}}%
\pgfusepath{stroke}%
\end{pgfscope}%
\begin{pgfscope}%
\pgfsetbuttcap%
\pgfsetroundjoin%
\definecolor{currentfill}{rgb}{0.000000,0.000000,0.000000}%
\pgfsetfillcolor{currentfill}%
\pgfsetlinewidth{0.602250pt}%
\definecolor{currentstroke}{rgb}{0.000000,0.000000,0.000000}%
\pgfsetstrokecolor{currentstroke}%
\pgfsetdash{}{0pt}%
\pgfsys@defobject{currentmarker}{\pgfqpoint{-0.027778in}{0.000000in}}{\pgfqpoint{-0.000000in}{0.000000in}}{%
\pgfpathmoveto{\pgfqpoint{-0.000000in}{0.000000in}}%
\pgfpathlineto{\pgfqpoint{-0.027778in}{0.000000in}}%
\pgfusepath{stroke,fill}%
}%
\begin{pgfscope}%
\pgfsys@transformshift{0.726250in}{0.943259in}%
\pgfsys@useobject{currentmarker}{}%
\end{pgfscope}%
\end{pgfscope}%
\begin{pgfscope}%
\pgfpathrectangle{\pgfqpoint{0.726250in}{0.525000in}}{\pgfqpoint{1.120000in}{1.637500in}}%
\pgfusepath{clip}%
\pgfsetbuttcap%
\pgfsetroundjoin%
\pgfsetlinewidth{0.803000pt}%
\definecolor{currentstroke}{rgb}{0.752941,0.752941,0.752941}%
\pgfsetstrokecolor{currentstroke}%
\pgfsetdash{{2.960000pt}{1.280000pt}}{0.000000pt}%
\pgfpathmoveto{\pgfqpoint{0.726250in}{0.969905in}}%
\pgfpathlineto{\pgfqpoint{1.846250in}{0.969905in}}%
\pgfusepath{stroke}%
\end{pgfscope}%
\begin{pgfscope}%
\pgfsetbuttcap%
\pgfsetroundjoin%
\definecolor{currentfill}{rgb}{0.000000,0.000000,0.000000}%
\pgfsetfillcolor{currentfill}%
\pgfsetlinewidth{0.602250pt}%
\definecolor{currentstroke}{rgb}{0.000000,0.000000,0.000000}%
\pgfsetstrokecolor{currentstroke}%
\pgfsetdash{}{0pt}%
\pgfsys@defobject{currentmarker}{\pgfqpoint{-0.027778in}{0.000000in}}{\pgfqpoint{-0.000000in}{0.000000in}}{%
\pgfpathmoveto{\pgfqpoint{-0.000000in}{0.000000in}}%
\pgfpathlineto{\pgfqpoint{-0.027778in}{0.000000in}}%
\pgfusepath{stroke,fill}%
}%
\begin{pgfscope}%
\pgfsys@transformshift{0.726250in}{0.969905in}%
\pgfsys@useobject{currentmarker}{}%
\end{pgfscope}%
\end{pgfscope}%
\begin{pgfscope}%
\pgfpathrectangle{\pgfqpoint{0.726250in}{0.525000in}}{\pgfqpoint{1.120000in}{1.637500in}}%
\pgfusepath{clip}%
\pgfsetbuttcap%
\pgfsetroundjoin%
\pgfsetlinewidth{0.803000pt}%
\definecolor{currentstroke}{rgb}{0.752941,0.752941,0.752941}%
\pgfsetstrokecolor{currentstroke}%
\pgfsetdash{{2.960000pt}{1.280000pt}}{0.000000pt}%
\pgfpathmoveto{\pgfqpoint{0.726250in}{1.150550in}}%
\pgfpathlineto{\pgfqpoint{1.846250in}{1.150550in}}%
\pgfusepath{stroke}%
\end{pgfscope}%
\begin{pgfscope}%
\pgfsetbuttcap%
\pgfsetroundjoin%
\definecolor{currentfill}{rgb}{0.000000,0.000000,0.000000}%
\pgfsetfillcolor{currentfill}%
\pgfsetlinewidth{0.602250pt}%
\definecolor{currentstroke}{rgb}{0.000000,0.000000,0.000000}%
\pgfsetstrokecolor{currentstroke}%
\pgfsetdash{}{0pt}%
\pgfsys@defobject{currentmarker}{\pgfqpoint{-0.027778in}{0.000000in}}{\pgfqpoint{-0.000000in}{0.000000in}}{%
\pgfpathmoveto{\pgfqpoint{-0.000000in}{0.000000in}}%
\pgfpathlineto{\pgfqpoint{-0.027778in}{0.000000in}}%
\pgfusepath{stroke,fill}%
}%
\begin{pgfscope}%
\pgfsys@transformshift{0.726250in}{1.150550in}%
\pgfsys@useobject{currentmarker}{}%
\end{pgfscope}%
\end{pgfscope}%
\begin{pgfscope}%
\pgfpathrectangle{\pgfqpoint{0.726250in}{0.525000in}}{\pgfqpoint{1.120000in}{1.637500in}}%
\pgfusepath{clip}%
\pgfsetbuttcap%
\pgfsetroundjoin%
\pgfsetlinewidth{0.803000pt}%
\definecolor{currentstroke}{rgb}{0.752941,0.752941,0.752941}%
\pgfsetstrokecolor{currentstroke}%
\pgfsetdash{{2.960000pt}{1.280000pt}}{0.000000pt}%
\pgfpathmoveto{\pgfqpoint{0.726250in}{1.242278in}}%
\pgfpathlineto{\pgfqpoint{1.846250in}{1.242278in}}%
\pgfusepath{stroke}%
\end{pgfscope}%
\begin{pgfscope}%
\pgfsetbuttcap%
\pgfsetroundjoin%
\definecolor{currentfill}{rgb}{0.000000,0.000000,0.000000}%
\pgfsetfillcolor{currentfill}%
\pgfsetlinewidth{0.602250pt}%
\definecolor{currentstroke}{rgb}{0.000000,0.000000,0.000000}%
\pgfsetstrokecolor{currentstroke}%
\pgfsetdash{}{0pt}%
\pgfsys@defobject{currentmarker}{\pgfqpoint{-0.027778in}{0.000000in}}{\pgfqpoint{-0.000000in}{0.000000in}}{%
\pgfpathmoveto{\pgfqpoint{-0.000000in}{0.000000in}}%
\pgfpathlineto{\pgfqpoint{-0.027778in}{0.000000in}}%
\pgfusepath{stroke,fill}%
}%
\begin{pgfscope}%
\pgfsys@transformshift{0.726250in}{1.242278in}%
\pgfsys@useobject{currentmarker}{}%
\end{pgfscope}%
\end{pgfscope}%
\begin{pgfscope}%
\pgfpathrectangle{\pgfqpoint{0.726250in}{0.525000in}}{\pgfqpoint{1.120000in}{1.637500in}}%
\pgfusepath{clip}%
\pgfsetbuttcap%
\pgfsetroundjoin%
\pgfsetlinewidth{0.803000pt}%
\definecolor{currentstroke}{rgb}{0.752941,0.752941,0.752941}%
\pgfsetstrokecolor{currentstroke}%
\pgfsetdash{{2.960000pt}{1.280000pt}}{0.000000pt}%
\pgfpathmoveto{\pgfqpoint{0.726250in}{1.307359in}}%
\pgfpathlineto{\pgfqpoint{1.846250in}{1.307359in}}%
\pgfusepath{stroke}%
\end{pgfscope}%
\begin{pgfscope}%
\pgfsetbuttcap%
\pgfsetroundjoin%
\definecolor{currentfill}{rgb}{0.000000,0.000000,0.000000}%
\pgfsetfillcolor{currentfill}%
\pgfsetlinewidth{0.602250pt}%
\definecolor{currentstroke}{rgb}{0.000000,0.000000,0.000000}%
\pgfsetstrokecolor{currentstroke}%
\pgfsetdash{}{0pt}%
\pgfsys@defobject{currentmarker}{\pgfqpoint{-0.027778in}{0.000000in}}{\pgfqpoint{-0.000000in}{0.000000in}}{%
\pgfpathmoveto{\pgfqpoint{-0.000000in}{0.000000in}}%
\pgfpathlineto{\pgfqpoint{-0.027778in}{0.000000in}}%
\pgfusepath{stroke,fill}%
}%
\begin{pgfscope}%
\pgfsys@transformshift{0.726250in}{1.307359in}%
\pgfsys@useobject{currentmarker}{}%
\end{pgfscope}%
\end{pgfscope}%
\begin{pgfscope}%
\pgfpathrectangle{\pgfqpoint{0.726250in}{0.525000in}}{\pgfqpoint{1.120000in}{1.637500in}}%
\pgfusepath{clip}%
\pgfsetbuttcap%
\pgfsetroundjoin%
\pgfsetlinewidth{0.803000pt}%
\definecolor{currentstroke}{rgb}{0.752941,0.752941,0.752941}%
\pgfsetstrokecolor{currentstroke}%
\pgfsetdash{{2.960000pt}{1.280000pt}}{0.000000pt}%
\pgfpathmoveto{\pgfqpoint{0.726250in}{1.357841in}}%
\pgfpathlineto{\pgfqpoint{1.846250in}{1.357841in}}%
\pgfusepath{stroke}%
\end{pgfscope}%
\begin{pgfscope}%
\pgfsetbuttcap%
\pgfsetroundjoin%
\definecolor{currentfill}{rgb}{0.000000,0.000000,0.000000}%
\pgfsetfillcolor{currentfill}%
\pgfsetlinewidth{0.602250pt}%
\definecolor{currentstroke}{rgb}{0.000000,0.000000,0.000000}%
\pgfsetstrokecolor{currentstroke}%
\pgfsetdash{}{0pt}%
\pgfsys@defobject{currentmarker}{\pgfqpoint{-0.027778in}{0.000000in}}{\pgfqpoint{-0.000000in}{0.000000in}}{%
\pgfpathmoveto{\pgfqpoint{-0.000000in}{0.000000in}}%
\pgfpathlineto{\pgfqpoint{-0.027778in}{0.000000in}}%
\pgfusepath{stroke,fill}%
}%
\begin{pgfscope}%
\pgfsys@transformshift{0.726250in}{1.357841in}%
\pgfsys@useobject{currentmarker}{}%
\end{pgfscope}%
\end{pgfscope}%
\begin{pgfscope}%
\pgfpathrectangle{\pgfqpoint{0.726250in}{0.525000in}}{\pgfqpoint{1.120000in}{1.637500in}}%
\pgfusepath{clip}%
\pgfsetbuttcap%
\pgfsetroundjoin%
\pgfsetlinewidth{0.803000pt}%
\definecolor{currentstroke}{rgb}{0.752941,0.752941,0.752941}%
\pgfsetstrokecolor{currentstroke}%
\pgfsetdash{{2.960000pt}{1.280000pt}}{0.000000pt}%
\pgfpathmoveto{\pgfqpoint{0.726250in}{1.399087in}}%
\pgfpathlineto{\pgfqpoint{1.846250in}{1.399087in}}%
\pgfusepath{stroke}%
\end{pgfscope}%
\begin{pgfscope}%
\pgfsetbuttcap%
\pgfsetroundjoin%
\definecolor{currentfill}{rgb}{0.000000,0.000000,0.000000}%
\pgfsetfillcolor{currentfill}%
\pgfsetlinewidth{0.602250pt}%
\definecolor{currentstroke}{rgb}{0.000000,0.000000,0.000000}%
\pgfsetstrokecolor{currentstroke}%
\pgfsetdash{}{0pt}%
\pgfsys@defobject{currentmarker}{\pgfqpoint{-0.027778in}{0.000000in}}{\pgfqpoint{-0.000000in}{0.000000in}}{%
\pgfpathmoveto{\pgfqpoint{-0.000000in}{0.000000in}}%
\pgfpathlineto{\pgfqpoint{-0.027778in}{0.000000in}}%
\pgfusepath{stroke,fill}%
}%
\begin{pgfscope}%
\pgfsys@transformshift{0.726250in}{1.399087in}%
\pgfsys@useobject{currentmarker}{}%
\end{pgfscope}%
\end{pgfscope}%
\begin{pgfscope}%
\pgfpathrectangle{\pgfqpoint{0.726250in}{0.525000in}}{\pgfqpoint{1.120000in}{1.637500in}}%
\pgfusepath{clip}%
\pgfsetbuttcap%
\pgfsetroundjoin%
\pgfsetlinewidth{0.803000pt}%
\definecolor{currentstroke}{rgb}{0.752941,0.752941,0.752941}%
\pgfsetstrokecolor{currentstroke}%
\pgfsetdash{{2.960000pt}{1.280000pt}}{0.000000pt}%
\pgfpathmoveto{\pgfqpoint{0.726250in}{1.433960in}}%
\pgfpathlineto{\pgfqpoint{1.846250in}{1.433960in}}%
\pgfusepath{stroke}%
\end{pgfscope}%
\begin{pgfscope}%
\pgfsetbuttcap%
\pgfsetroundjoin%
\definecolor{currentfill}{rgb}{0.000000,0.000000,0.000000}%
\pgfsetfillcolor{currentfill}%
\pgfsetlinewidth{0.602250pt}%
\definecolor{currentstroke}{rgb}{0.000000,0.000000,0.000000}%
\pgfsetstrokecolor{currentstroke}%
\pgfsetdash{}{0pt}%
\pgfsys@defobject{currentmarker}{\pgfqpoint{-0.027778in}{0.000000in}}{\pgfqpoint{-0.000000in}{0.000000in}}{%
\pgfpathmoveto{\pgfqpoint{-0.000000in}{0.000000in}}%
\pgfpathlineto{\pgfqpoint{-0.027778in}{0.000000in}}%
\pgfusepath{stroke,fill}%
}%
\begin{pgfscope}%
\pgfsys@transformshift{0.726250in}{1.433960in}%
\pgfsys@useobject{currentmarker}{}%
\end{pgfscope}%
\end{pgfscope}%
\begin{pgfscope}%
\pgfpathrectangle{\pgfqpoint{0.726250in}{0.525000in}}{\pgfqpoint{1.120000in}{1.637500in}}%
\pgfusepath{clip}%
\pgfsetbuttcap%
\pgfsetroundjoin%
\pgfsetlinewidth{0.803000pt}%
\definecolor{currentstroke}{rgb}{0.752941,0.752941,0.752941}%
\pgfsetstrokecolor{currentstroke}%
\pgfsetdash{{2.960000pt}{1.280000pt}}{0.000000pt}%
\pgfpathmoveto{\pgfqpoint{0.726250in}{1.464169in}}%
\pgfpathlineto{\pgfqpoint{1.846250in}{1.464169in}}%
\pgfusepath{stroke}%
\end{pgfscope}%
\begin{pgfscope}%
\pgfsetbuttcap%
\pgfsetroundjoin%
\definecolor{currentfill}{rgb}{0.000000,0.000000,0.000000}%
\pgfsetfillcolor{currentfill}%
\pgfsetlinewidth{0.602250pt}%
\definecolor{currentstroke}{rgb}{0.000000,0.000000,0.000000}%
\pgfsetstrokecolor{currentstroke}%
\pgfsetdash{}{0pt}%
\pgfsys@defobject{currentmarker}{\pgfqpoint{-0.027778in}{0.000000in}}{\pgfqpoint{-0.000000in}{0.000000in}}{%
\pgfpathmoveto{\pgfqpoint{-0.000000in}{0.000000in}}%
\pgfpathlineto{\pgfqpoint{-0.027778in}{0.000000in}}%
\pgfusepath{stroke,fill}%
}%
\begin{pgfscope}%
\pgfsys@transformshift{0.726250in}{1.464169in}%
\pgfsys@useobject{currentmarker}{}%
\end{pgfscope}%
\end{pgfscope}%
\begin{pgfscope}%
\pgfpathrectangle{\pgfqpoint{0.726250in}{0.525000in}}{\pgfqpoint{1.120000in}{1.637500in}}%
\pgfusepath{clip}%
\pgfsetbuttcap%
\pgfsetroundjoin%
\pgfsetlinewidth{0.803000pt}%
\definecolor{currentstroke}{rgb}{0.752941,0.752941,0.752941}%
\pgfsetstrokecolor{currentstroke}%
\pgfsetdash{{2.960000pt}{1.280000pt}}{0.000000pt}%
\pgfpathmoveto{\pgfqpoint{0.726250in}{1.490815in}}%
\pgfpathlineto{\pgfqpoint{1.846250in}{1.490815in}}%
\pgfusepath{stroke}%
\end{pgfscope}%
\begin{pgfscope}%
\pgfsetbuttcap%
\pgfsetroundjoin%
\definecolor{currentfill}{rgb}{0.000000,0.000000,0.000000}%
\pgfsetfillcolor{currentfill}%
\pgfsetlinewidth{0.602250pt}%
\definecolor{currentstroke}{rgb}{0.000000,0.000000,0.000000}%
\pgfsetstrokecolor{currentstroke}%
\pgfsetdash{}{0pt}%
\pgfsys@defobject{currentmarker}{\pgfqpoint{-0.027778in}{0.000000in}}{\pgfqpoint{-0.000000in}{0.000000in}}{%
\pgfpathmoveto{\pgfqpoint{-0.000000in}{0.000000in}}%
\pgfpathlineto{\pgfqpoint{-0.027778in}{0.000000in}}%
\pgfusepath{stroke,fill}%
}%
\begin{pgfscope}%
\pgfsys@transformshift{0.726250in}{1.490815in}%
\pgfsys@useobject{currentmarker}{}%
\end{pgfscope}%
\end{pgfscope}%
\begin{pgfscope}%
\pgfpathrectangle{\pgfqpoint{0.726250in}{0.525000in}}{\pgfqpoint{1.120000in}{1.637500in}}%
\pgfusepath{clip}%
\pgfsetbuttcap%
\pgfsetroundjoin%
\pgfsetlinewidth{0.803000pt}%
\definecolor{currentstroke}{rgb}{0.752941,0.752941,0.752941}%
\pgfsetstrokecolor{currentstroke}%
\pgfsetdash{{2.960000pt}{1.280000pt}}{0.000000pt}%
\pgfpathmoveto{\pgfqpoint{0.726250in}{1.671460in}}%
\pgfpathlineto{\pgfqpoint{1.846250in}{1.671460in}}%
\pgfusepath{stroke}%
\end{pgfscope}%
\begin{pgfscope}%
\pgfsetbuttcap%
\pgfsetroundjoin%
\definecolor{currentfill}{rgb}{0.000000,0.000000,0.000000}%
\pgfsetfillcolor{currentfill}%
\pgfsetlinewidth{0.602250pt}%
\definecolor{currentstroke}{rgb}{0.000000,0.000000,0.000000}%
\pgfsetstrokecolor{currentstroke}%
\pgfsetdash{}{0pt}%
\pgfsys@defobject{currentmarker}{\pgfqpoint{-0.027778in}{0.000000in}}{\pgfqpoint{-0.000000in}{0.000000in}}{%
\pgfpathmoveto{\pgfqpoint{-0.000000in}{0.000000in}}%
\pgfpathlineto{\pgfqpoint{-0.027778in}{0.000000in}}%
\pgfusepath{stroke,fill}%
}%
\begin{pgfscope}%
\pgfsys@transformshift{0.726250in}{1.671460in}%
\pgfsys@useobject{currentmarker}{}%
\end{pgfscope}%
\end{pgfscope}%
\begin{pgfscope}%
\pgfpathrectangle{\pgfqpoint{0.726250in}{0.525000in}}{\pgfqpoint{1.120000in}{1.637500in}}%
\pgfusepath{clip}%
\pgfsetbuttcap%
\pgfsetroundjoin%
\pgfsetlinewidth{0.803000pt}%
\definecolor{currentstroke}{rgb}{0.752941,0.752941,0.752941}%
\pgfsetstrokecolor{currentstroke}%
\pgfsetdash{{2.960000pt}{1.280000pt}}{0.000000pt}%
\pgfpathmoveto{\pgfqpoint{0.726250in}{1.763187in}}%
\pgfpathlineto{\pgfqpoint{1.846250in}{1.763187in}}%
\pgfusepath{stroke}%
\end{pgfscope}%
\begin{pgfscope}%
\pgfsetbuttcap%
\pgfsetroundjoin%
\definecolor{currentfill}{rgb}{0.000000,0.000000,0.000000}%
\pgfsetfillcolor{currentfill}%
\pgfsetlinewidth{0.602250pt}%
\definecolor{currentstroke}{rgb}{0.000000,0.000000,0.000000}%
\pgfsetstrokecolor{currentstroke}%
\pgfsetdash{}{0pt}%
\pgfsys@defobject{currentmarker}{\pgfqpoint{-0.027778in}{0.000000in}}{\pgfqpoint{-0.000000in}{0.000000in}}{%
\pgfpathmoveto{\pgfqpoint{-0.000000in}{0.000000in}}%
\pgfpathlineto{\pgfqpoint{-0.027778in}{0.000000in}}%
\pgfusepath{stroke,fill}%
}%
\begin{pgfscope}%
\pgfsys@transformshift{0.726250in}{1.763187in}%
\pgfsys@useobject{currentmarker}{}%
\end{pgfscope}%
\end{pgfscope}%
\begin{pgfscope}%
\pgfpathrectangle{\pgfqpoint{0.726250in}{0.525000in}}{\pgfqpoint{1.120000in}{1.637500in}}%
\pgfusepath{clip}%
\pgfsetbuttcap%
\pgfsetroundjoin%
\pgfsetlinewidth{0.803000pt}%
\definecolor{currentstroke}{rgb}{0.752941,0.752941,0.752941}%
\pgfsetstrokecolor{currentstroke}%
\pgfsetdash{{2.960000pt}{1.280000pt}}{0.000000pt}%
\pgfpathmoveto{\pgfqpoint{0.726250in}{1.828269in}}%
\pgfpathlineto{\pgfqpoint{1.846250in}{1.828269in}}%
\pgfusepath{stroke}%
\end{pgfscope}%
\begin{pgfscope}%
\pgfsetbuttcap%
\pgfsetroundjoin%
\definecolor{currentfill}{rgb}{0.000000,0.000000,0.000000}%
\pgfsetfillcolor{currentfill}%
\pgfsetlinewidth{0.602250pt}%
\definecolor{currentstroke}{rgb}{0.000000,0.000000,0.000000}%
\pgfsetstrokecolor{currentstroke}%
\pgfsetdash{}{0pt}%
\pgfsys@defobject{currentmarker}{\pgfqpoint{-0.027778in}{0.000000in}}{\pgfqpoint{-0.000000in}{0.000000in}}{%
\pgfpathmoveto{\pgfqpoint{-0.000000in}{0.000000in}}%
\pgfpathlineto{\pgfqpoint{-0.027778in}{0.000000in}}%
\pgfusepath{stroke,fill}%
}%
\begin{pgfscope}%
\pgfsys@transformshift{0.726250in}{1.828269in}%
\pgfsys@useobject{currentmarker}{}%
\end{pgfscope}%
\end{pgfscope}%
\begin{pgfscope}%
\pgfpathrectangle{\pgfqpoint{0.726250in}{0.525000in}}{\pgfqpoint{1.120000in}{1.637500in}}%
\pgfusepath{clip}%
\pgfsetbuttcap%
\pgfsetroundjoin%
\pgfsetlinewidth{0.803000pt}%
\definecolor{currentstroke}{rgb}{0.752941,0.752941,0.752941}%
\pgfsetstrokecolor{currentstroke}%
\pgfsetdash{{2.960000pt}{1.280000pt}}{0.000000pt}%
\pgfpathmoveto{\pgfqpoint{0.726250in}{1.878750in}}%
\pgfpathlineto{\pgfqpoint{1.846250in}{1.878750in}}%
\pgfusepath{stroke}%
\end{pgfscope}%
\begin{pgfscope}%
\pgfsetbuttcap%
\pgfsetroundjoin%
\definecolor{currentfill}{rgb}{0.000000,0.000000,0.000000}%
\pgfsetfillcolor{currentfill}%
\pgfsetlinewidth{0.602250pt}%
\definecolor{currentstroke}{rgb}{0.000000,0.000000,0.000000}%
\pgfsetstrokecolor{currentstroke}%
\pgfsetdash{}{0pt}%
\pgfsys@defobject{currentmarker}{\pgfqpoint{-0.027778in}{0.000000in}}{\pgfqpoint{-0.000000in}{0.000000in}}{%
\pgfpathmoveto{\pgfqpoint{-0.000000in}{0.000000in}}%
\pgfpathlineto{\pgfqpoint{-0.027778in}{0.000000in}}%
\pgfusepath{stroke,fill}%
}%
\begin{pgfscope}%
\pgfsys@transformshift{0.726250in}{1.878750in}%
\pgfsys@useobject{currentmarker}{}%
\end{pgfscope}%
\end{pgfscope}%
\begin{pgfscope}%
\pgfpathrectangle{\pgfqpoint{0.726250in}{0.525000in}}{\pgfqpoint{1.120000in}{1.637500in}}%
\pgfusepath{clip}%
\pgfsetbuttcap%
\pgfsetroundjoin%
\pgfsetlinewidth{0.803000pt}%
\definecolor{currentstroke}{rgb}{0.752941,0.752941,0.752941}%
\pgfsetstrokecolor{currentstroke}%
\pgfsetdash{{2.960000pt}{1.280000pt}}{0.000000pt}%
\pgfpathmoveto{\pgfqpoint{0.726250in}{1.919997in}}%
\pgfpathlineto{\pgfqpoint{1.846250in}{1.919997in}}%
\pgfusepath{stroke}%
\end{pgfscope}%
\begin{pgfscope}%
\pgfsetbuttcap%
\pgfsetroundjoin%
\definecolor{currentfill}{rgb}{0.000000,0.000000,0.000000}%
\pgfsetfillcolor{currentfill}%
\pgfsetlinewidth{0.602250pt}%
\definecolor{currentstroke}{rgb}{0.000000,0.000000,0.000000}%
\pgfsetstrokecolor{currentstroke}%
\pgfsetdash{}{0pt}%
\pgfsys@defobject{currentmarker}{\pgfqpoint{-0.027778in}{0.000000in}}{\pgfqpoint{-0.000000in}{0.000000in}}{%
\pgfpathmoveto{\pgfqpoint{-0.000000in}{0.000000in}}%
\pgfpathlineto{\pgfqpoint{-0.027778in}{0.000000in}}%
\pgfusepath{stroke,fill}%
}%
\begin{pgfscope}%
\pgfsys@transformshift{0.726250in}{1.919997in}%
\pgfsys@useobject{currentmarker}{}%
\end{pgfscope}%
\end{pgfscope}%
\begin{pgfscope}%
\pgfpathrectangle{\pgfqpoint{0.726250in}{0.525000in}}{\pgfqpoint{1.120000in}{1.637500in}}%
\pgfusepath{clip}%
\pgfsetbuttcap%
\pgfsetroundjoin%
\pgfsetlinewidth{0.803000pt}%
\definecolor{currentstroke}{rgb}{0.752941,0.752941,0.752941}%
\pgfsetstrokecolor{currentstroke}%
\pgfsetdash{{2.960000pt}{1.280000pt}}{0.000000pt}%
\pgfpathmoveto{\pgfqpoint{0.726250in}{1.954870in}}%
\pgfpathlineto{\pgfqpoint{1.846250in}{1.954870in}}%
\pgfusepath{stroke}%
\end{pgfscope}%
\begin{pgfscope}%
\pgfsetbuttcap%
\pgfsetroundjoin%
\definecolor{currentfill}{rgb}{0.000000,0.000000,0.000000}%
\pgfsetfillcolor{currentfill}%
\pgfsetlinewidth{0.602250pt}%
\definecolor{currentstroke}{rgb}{0.000000,0.000000,0.000000}%
\pgfsetstrokecolor{currentstroke}%
\pgfsetdash{}{0pt}%
\pgfsys@defobject{currentmarker}{\pgfqpoint{-0.027778in}{0.000000in}}{\pgfqpoint{-0.000000in}{0.000000in}}{%
\pgfpathmoveto{\pgfqpoint{-0.000000in}{0.000000in}}%
\pgfpathlineto{\pgfqpoint{-0.027778in}{0.000000in}}%
\pgfusepath{stroke,fill}%
}%
\begin{pgfscope}%
\pgfsys@transformshift{0.726250in}{1.954870in}%
\pgfsys@useobject{currentmarker}{}%
\end{pgfscope}%
\end{pgfscope}%
\begin{pgfscope}%
\pgfpathrectangle{\pgfqpoint{0.726250in}{0.525000in}}{\pgfqpoint{1.120000in}{1.637500in}}%
\pgfusepath{clip}%
\pgfsetbuttcap%
\pgfsetroundjoin%
\pgfsetlinewidth{0.803000pt}%
\definecolor{currentstroke}{rgb}{0.752941,0.752941,0.752941}%
\pgfsetstrokecolor{currentstroke}%
\pgfsetdash{{2.960000pt}{1.280000pt}}{0.000000pt}%
\pgfpathmoveto{\pgfqpoint{0.726250in}{1.985079in}}%
\pgfpathlineto{\pgfqpoint{1.846250in}{1.985079in}}%
\pgfusepath{stroke}%
\end{pgfscope}%
\begin{pgfscope}%
\pgfsetbuttcap%
\pgfsetroundjoin%
\definecolor{currentfill}{rgb}{0.000000,0.000000,0.000000}%
\pgfsetfillcolor{currentfill}%
\pgfsetlinewidth{0.602250pt}%
\definecolor{currentstroke}{rgb}{0.000000,0.000000,0.000000}%
\pgfsetstrokecolor{currentstroke}%
\pgfsetdash{}{0pt}%
\pgfsys@defobject{currentmarker}{\pgfqpoint{-0.027778in}{0.000000in}}{\pgfqpoint{-0.000000in}{0.000000in}}{%
\pgfpathmoveto{\pgfqpoint{-0.000000in}{0.000000in}}%
\pgfpathlineto{\pgfqpoint{-0.027778in}{0.000000in}}%
\pgfusepath{stroke,fill}%
}%
\begin{pgfscope}%
\pgfsys@transformshift{0.726250in}{1.985079in}%
\pgfsys@useobject{currentmarker}{}%
\end{pgfscope}%
\end{pgfscope}%
\begin{pgfscope}%
\pgfpathrectangle{\pgfqpoint{0.726250in}{0.525000in}}{\pgfqpoint{1.120000in}{1.637500in}}%
\pgfusepath{clip}%
\pgfsetbuttcap%
\pgfsetroundjoin%
\pgfsetlinewidth{0.803000pt}%
\definecolor{currentstroke}{rgb}{0.752941,0.752941,0.752941}%
\pgfsetstrokecolor{currentstroke}%
\pgfsetdash{{2.960000pt}{1.280000pt}}{0.000000pt}%
\pgfpathmoveto{\pgfqpoint{0.726250in}{2.011724in}}%
\pgfpathlineto{\pgfqpoint{1.846250in}{2.011724in}}%
\pgfusepath{stroke}%
\end{pgfscope}%
\begin{pgfscope}%
\pgfsetbuttcap%
\pgfsetroundjoin%
\definecolor{currentfill}{rgb}{0.000000,0.000000,0.000000}%
\pgfsetfillcolor{currentfill}%
\pgfsetlinewidth{0.602250pt}%
\definecolor{currentstroke}{rgb}{0.000000,0.000000,0.000000}%
\pgfsetstrokecolor{currentstroke}%
\pgfsetdash{}{0pt}%
\pgfsys@defobject{currentmarker}{\pgfqpoint{-0.027778in}{0.000000in}}{\pgfqpoint{-0.000000in}{0.000000in}}{%
\pgfpathmoveto{\pgfqpoint{-0.000000in}{0.000000in}}%
\pgfpathlineto{\pgfqpoint{-0.027778in}{0.000000in}}%
\pgfusepath{stroke,fill}%
}%
\begin{pgfscope}%
\pgfsys@transformshift{0.726250in}{2.011724in}%
\pgfsys@useobject{currentmarker}{}%
\end{pgfscope}%
\end{pgfscope}%
\begin{pgfscope}%
\definecolor{textcolor}{rgb}{0.000000,0.000000,0.000000}%
\pgfsetstrokecolor{textcolor}%
\pgfsetfillcolor{textcolor}%
\pgftext[x=0.387131in,y=1.343750in,,bottom,rotate=90.000000]{\color{textcolor}\rmfamily\fontsize{9.000000}{10.800000}\selectfont \(\displaystyle \nu\) path length [km]}%
\end{pgfscope}%
\begin{pgfscope}%
\pgfpathrectangle{\pgfqpoint{0.726250in}{0.525000in}}{\pgfqpoint{1.120000in}{1.637500in}}%
\pgfusepath{clip}%
\pgfsetrectcap%
\pgfsetroundjoin%
\pgfsetlinewidth{1.003750pt}%
\definecolor{currentstroke}{rgb}{0.000000,0.000000,0.000000}%
\pgfsetstrokecolor{currentstroke}%
\pgfsetdash{}{0pt}%
\pgfpathmoveto{\pgfqpoint{1.846250in}{0.599432in}}%
\pgfpathlineto{\pgfqpoint{1.792425in}{0.622223in}}%
\pgfpathlineto{\pgfqpoint{1.743420in}{0.645172in}}%
\pgfpathlineto{\pgfqpoint{1.699042in}{0.668167in}}%
\pgfpathlineto{\pgfqpoint{1.654791in}{0.693673in}}%
\pgfpathlineto{\pgfqpoint{1.613260in}{0.720527in}}%
\pgfpathlineto{\pgfqpoint{1.576396in}{0.747340in}}%
\pgfpathlineto{\pgfqpoint{1.545591in}{0.772441in}}%
\pgfpathlineto{\pgfqpoint{1.513753in}{0.801642in}}%
\pgfpathlineto{\pgfqpoint{1.489273in}{0.826924in}}%
\pgfpathlineto{\pgfqpoint{1.464336in}{0.855878in}}%
\pgfpathlineto{\pgfqpoint{1.439001in}{0.889529in}}%
\pgfpathlineto{\pgfqpoint{1.421917in}{0.915296in}}%
\pgfpathlineto{\pgfqpoint{1.404697in}{0.944455in}}%
\pgfpathlineto{\pgfqpoint{1.387360in}{0.977897in}}%
\pgfpathlineto{\pgfqpoint{1.369922in}{1.016875in}}%
\pgfpathlineto{\pgfqpoint{1.352400in}{1.063192in}}%
\pgfpathlineto{\pgfqpoint{1.334813in}{1.119402in}}%
\pgfpathlineto{\pgfqpoint{1.317177in}{1.188705in}}%
\pgfpathlineto{\pgfqpoint{1.299510in}{1.273101in}}%
\pgfpathlineto{\pgfqpoint{1.264154in}{1.458427in}}%
\pgfpathlineto{\pgfqpoint{1.246500in}{1.535282in}}%
\pgfpathlineto{\pgfqpoint{1.228886in}{1.597631in}}%
\pgfpathlineto{\pgfqpoint{1.211330in}{1.648524in}}%
\pgfpathlineto{\pgfqpoint{1.193848in}{1.690904in}}%
\pgfpathlineto{\pgfqpoint{1.176458in}{1.726927in}}%
\pgfpathlineto{\pgfqpoint{1.159177in}{1.758097in}}%
\pgfpathlineto{\pgfqpoint{1.142023in}{1.785467in}}%
\pgfpathlineto{\pgfqpoint{1.116567in}{1.820989in}}%
\pgfpathlineto{\pgfqpoint{1.091492in}{1.851376in}}%
\pgfpathlineto{\pgfqpoint{1.066853in}{1.877789in}}%
\pgfpathlineto{\pgfqpoint{1.034776in}{1.908181in}}%
\pgfpathlineto{\pgfqpoint{1.003702in}{1.934221in}}%
\pgfpathlineto{\pgfqpoint{0.966457in}{1.961966in}}%
\pgfpathlineto{\pgfqpoint{0.931203in}{1.985444in}}%
\pgfpathlineto{\pgfqpoint{0.891836in}{2.009085in}}%
\pgfpathlineto{\pgfqpoint{0.844904in}{2.034393in}}%
\pgfpathlineto{\pgfqpoint{0.796201in}{2.057978in}}%
\pgfpathlineto{\pgfqpoint{0.741878in}{2.081683in}}%
\pgfpathlineto{\pgfqpoint{0.726250in}{2.088068in}}%
\pgfpathlineto{\pgfqpoint{0.726250in}{2.088068in}}%
\pgfusepath{stroke}%
\end{pgfscope}%
\begin{pgfscope}%
\pgfsetrectcap%
\pgfsetmiterjoin%
\pgfsetlinewidth{1.003750pt}%
\definecolor{currentstroke}{rgb}{0.000000,0.000000,0.000000}%
\pgfsetstrokecolor{currentstroke}%
\pgfsetdash{}{0pt}%
\pgfpathmoveto{\pgfqpoint{0.726250in}{0.525000in}}%
\pgfpathlineto{\pgfqpoint{0.726250in}{2.162500in}}%
\pgfusepath{stroke}%
\end{pgfscope}%
\begin{pgfscope}%
\pgfsetrectcap%
\pgfsetmiterjoin%
\pgfsetlinewidth{1.003750pt}%
\definecolor{currentstroke}{rgb}{0.000000,0.000000,0.000000}%
\pgfsetstrokecolor{currentstroke}%
\pgfsetdash{}{0pt}%
\pgfpathmoveto{\pgfqpoint{1.846250in}{0.525000in}}%
\pgfpathlineto{\pgfqpoint{1.846250in}{2.162500in}}%
\pgfusepath{stroke}%
\end{pgfscope}%
\begin{pgfscope}%
\pgfsetrectcap%
\pgfsetmiterjoin%
\pgfsetlinewidth{1.003750pt}%
\definecolor{currentstroke}{rgb}{0.000000,0.000000,0.000000}%
\pgfsetstrokecolor{currentstroke}%
\pgfsetdash{}{0pt}%
\pgfpathmoveto{\pgfqpoint{0.726250in}{0.525000in}}%
\pgfpathlineto{\pgfqpoint{1.846250in}{0.525000in}}%
\pgfusepath{stroke}%
\end{pgfscope}%
\begin{pgfscope}%
\pgfsetrectcap%
\pgfsetmiterjoin%
\pgfsetlinewidth{1.003750pt}%
\definecolor{currentstroke}{rgb}{0.000000,0.000000,0.000000}%
\pgfsetstrokecolor{currentstroke}%
\pgfsetdash{}{0pt}%
\pgfpathmoveto{\pgfqpoint{0.726250in}{2.162500in}}%
\pgfpathlineto{\pgfqpoint{1.846250in}{2.162500in}}%
\pgfusepath{stroke}%
\end{pgfscope}%
\end{pgfpicture}%
\makeatother%
\endgroup%

  \caption{Distance traverse by atmospheric neutrinos between the creation and
  the detection site as a function of the cosine director with respect to the local
  zenith. The plot assumes an average altitude $h = 15$~km for the
  neutrino creation and a detector $2.5$~km undergrund.}
  \label{fig:atmospheric_nu_sketch}
\end{marginfigure}

If we plug the $\Delta m^2$ found for solar neutrino into the oscillation formula,
we get a typical phase $\varphi \approx 0.1$ for $E = 1$~GeV and $L = 1,000$~km,
that is, electron neutrinos hardly have room to oscillate to different flavors.
The fact that the neutrino energies are so much higher, though, opens the possibility
of creating muons in the final state, and deep-underground water-\cherenkov~experiments
such as SK can measure the flux ratio between $\nu_\mu$ and $\nu_e$ via CC interactions
on nuclei
\begin{align*}
  \nu_{e,~\mu} + N \rightarrow e~,\mu + X
\end{align*}
identifying the lepton in the final state. This, in conjuction with the ability to
measure the neutrino energy and direction, opens the possibility of studying the
separate $\nu_e$ and $\nu_\mu$ fluxes, as well as their ratio, as a function of the
zenith angle---i.e., as a function of the distance $L$ traveled by the neutrino.

The main SK results are that, while $\nu_e$ measurements agree reasonably well with
expectations in absense of oscillations, we have a significant deficit of
$\nu_\mu$, amounting to some $5\sim 0\%$ and extending to all zenith angle at low
energy and to upward-going neutrinos above $\sim 1$~GeV. This basic findings can be
interpreted naturally as $\nu_\mu \rightarrow \nu_\tau$ oscillations with
\begin{align}
  \Delta m^2_\text{atm} \approx 3 \times 10^{-3}~\text{eV}^2
  \quad\text{and}\quad
  \theta_\text{atm} \approx 45~^\circ.
\end{align}
More specifically, if we plug the numbers into the formula for the oscillation phase,
assuming an energy of 1~GeV, we have
\begin{align*}
  \varphi \approx \begin{cases}
  0.04 & \quad\text{for } L = 10~\text{km (upward-going)}\\
  50   & \quad\text{for } L = 13,000~\text{km (downward-going)}
\end{cases}
\end{align*}
that is, moving across different bins of zenith angles we explore all the possible
regimes, from the situation where the oscillation phase is small $\varphi \ll 1$
and oscillations do not take place, to that where the phase is large $\varphi \gg 1$
and the oscillations terms averages out to the sine squared of the mixing angle,
that in this case appears to be maximal.


\subsection{Verification with accelerator neutrinos}


\section{Three-flavor neutrino oscillations}

Up to this point we have discussed all the measurements in terms of oscillations
for two generations of neutrino, ending up with two distinct sets of $\Delta m^2$
and mixing angles, while in fact we know that three neutrino flavor exists, related
to the mass eigenstates by a $3 \times 3$ unitary matrix\sidenote{
This means that $U U^\dag = I$ or, equivalently $U^\dag = U^{-1} = (U^*)^T$.
In other words
\begin{align*}
  \begin{pmatrix}\nu_1\\ \nu_2\\ \nu_2  \end{pmatrix} =
  \begin{pmatrix}
    U_{e1}^* & U_{\mu 1}^* & U_{\tau 1}^*\\
    U_{e2}^* & U_{\mu 2}^* & U_{\tau 2}^*\\
    U_{e3}^* & U_{\mu 3}^* & U_{\tau 3}^*\\
  \end{pmatrix}
  \begin{pmatrix}\nu_e\\ \nu_\mu\\ \nu_\tau  \end{pmatrix}.
\end{align*}
}
\begin{align*}
  \begin{pmatrix}\nu_e\\ \nu_\mu\\ \nu_\tau  \end{pmatrix} = U
  \begin{pmatrix}\nu_1\\ \nu_2\\ \nu_3 \end{pmatrix} =
  \begin{pmatrix}
    U_{e1} & U_{e2} & U_{e3}\\
    U_{\mu 1} & U_{\mu 2} & U_{\mu 3}\\
    U_{\tau 1} & U_{\tau 2} & U_{\tau 3}\\
  \end{pmatrix}
  \begin{pmatrix}\nu_1\\ \nu_2\\ \nu_3 \end{pmatrix},
\end{align*}
which is customarily called the Pontecorvo-Maki-Nakagawa-Sakata (PMNS) mixing matrix.
In this section we shall briefly review the formalism for three-flavor neutrino
mixing, and how that ties to all we have said so far.

A $3 \times 3$ unitary matrix can in general be parametrized with 6~numbers (three
angles and three phases). This is the case if the neutrinos are Majorana
particles\sidenote{We shall come back to this when discussion neutrinoless double
$\beta$ decay.} (that is, if neutrino and antineutrino corresponds to the same
physical state), while if neutrinos are ordinary Dirac spinors, two of the phases
can be reabsorbed into the physical states, and the mixing matrix, just like the
CMK matrix for quarks, can be written in terms of three angles ($\theta_{12}$,
$\theta_{13}$, and $\theta_{23}$) and a phase $\delta_\text{CP}$\sidenote{We shall not
insist on this, but if this phase is non trivial, it implier that the CP discrete
symmetry is not conserved in neutrino interaction, hence the name.}, the most popular
parametrization being
\begin{align}
  U =
  \overbrace{
  \begin{pmatrix}
    1 & 0 & 0\\
    0 & c_{23} & s_{23}\\
    0 & -s_{23} & c_23 \\
  \end{pmatrix}
  }^\text{atmospheric}
  \begin{pmatrix}
    c_{13} & 0 & s_{13} e^{-i\delta_\text{CP}}\\
    0 & 1 & 0\\
    -s_{13} e^{i\delta_\text{CP}} & 0 & c_13\\
  \end{pmatrix}
  \overbrace{
  \begin{pmatrix}
    c_{12} & s_{12} & 0\\
    -s_{12} & c_12 & 0\\
    0 & 0 & 1\\
  \end{pmatrix}
  }^\text{solar},
\end{align}
where we have introduced the shorthand notation $s_{ij} = \sin\theta_{ij}$ and
$c_{ij} = \cos\theta_{ij}$. It is instructive to work out the full matrix multiplication
and spell out the PMNS mixing matrix in its full form
\begin{align}
  U =
  \begin{pmatrix}
    c_{12}c_{13} & s_{12}c_{13} & s_{13} e^{-i\delta_\text{CP}}\\
    -s_{12}c_{23} -c_{12}s_{13}s_{23}e^{i\delta_\text{CP}} &
      c_{12}c_{23} - s_{12}s_{13}s_{23}e^{i\delta_\text{CP}} & c_{13}s_{23}\\
    s_{12}s_{23} - c_{12}s_{13}c_{23}e^{i\delta_\text{CP}} &
      -c_{12}s_{23} - s_{12}s_{13}c_{23}e^{i\delta_\text{CP}} & c_{13}c_{23}\\
  \end{pmatrix},
\end{align}
which gives an immediate idea of how the formulation is algebrically more complicated
in the three-generation case, compared to the two-generation one.

Several global analyses of all available neutrino data exist, a recent, representative
one~\cite{2020JHEP...09..178E} yielding\sidenote{We are deliberately glitching over
some subtle details, as the fit results are slightly different in normal and inverse
mass ordering, but the differences are hardly relevant in our case.}
\begin{align}
  \theta_{12} & = 33.41^{+0.75}_{-0.72}~^\circ \nonumber\\
  \theta_{23} & = 42.2^{+1.1}_{-0.9}~^\circ \nonumber\\
  \theta_{13} & = 8.58^{+0.11}_{-0.11}~^\circ \nonumber\\
  \delta_\text{CP} & = 232^{+36}_{-26}~^\circ \nonumber\\
  \Delta m^2_{21} & = 7.41^{+0.21}_{-0.20} \times 10^{-5}~\text{eV}^2 \nonumber\\
  \Delta m^2_{31} & = \pm 2.507^{+0.026}_{-0.027} \times 10^{-3}~\text{eV}^2
\end{align}
Note the CP-violating phase is compatible, within current experimental uncertainities,
with $\delta_\text{CP} = 180^\circ$, i.e. with no CP violation. And, since we are at it,
this particular configuration translates into full matrix whose modulus can be written as
\begin{align*}
  \abs{U} =
  \begin{pmatrix}
    0.825 & 0.544 & 0.149\\
    0.362 & 0.654 & 0.664\\
    0.433 & 0.525 & 0.733\\
  \end{pmatrix}
  \quad\text{and}\quad
  \abs{U}^2 =
  \begin{pmatrix}
    0.681 & 0.296 & 0.022\\
    0.131 & 0.427 & 0.441\\
    0.187 & 0.276 & 0.537\\
  \end{pmatrix}
\end{align*}
which makes it immediately obvious how different is the mixing in the neutrino
sector with respect to the quark sector\sidenote{We note as a historical curiosity that
the \emph{bi-tri-maximal mixing}, where $\tan^2{\theta_{12}} = \nicefrac{1}{2}$
(or $\theta_{12} = 35.26^\circ$), $\theta_{23} = 45^\circ$ and $\theta_{13} = 0$
has been compatible with all experimental data for some time, and a very popular
choice, leading to the surprisingly simple (and elegant) form of the mixing matrix
\begin{align*}
  \abs{U}^2 =
  \begin{pmatrix}
    \nicefrac{2}{3} & \nicefrac{1}{3} & 0\\
    \nicefrac{1}{6} & \nicefrac{1}{3} & \nicefrac{1}{2}\\
    \nicefrac{1}{6} & \nicefrac{1}{3} & \nicefrac{1}{2}\\
  \end{pmatrix}.
\end{align*}
This is now definitely ruled out by the most recent measurements, but it is still
found in recent literature.}.
The explicit form of the modulus of the PMNS matrix is useful, as it allows to grasp
immediately the flavor content of the neutrino mass eigenstate, i.e.,
\begin{align}
  \abs{\braket{\nu_e}{\nu_1}}^2 = U_{e1}U^*_{e1} = \abs{U_{e1}}^2,
\end{align}
and so on and so forth. By taking the squares of the numerical values of the first
(second, third) column of the matrix, we get the $\nu_e$,  $\nu_\mu$ and $\nu_\tau$
content of the first (second, third) mass eigenstate: $\nu_1$ is prevalently $\nu_e$,
with non negligible contributions from $\nu_\mu$ and $\nu_\tau$, $\nu_2$ has roughly
equal contents of the three flavors, and $\nu_3$ is largely dominated by $\nu_\mu$
and $\nu_\tau$, with a fairly small $\nu_e$ content.


\subsection{Neutrino masses and mass hyerarchy}

The neutrino mass hierarchy cannot be fully resolved with current data. We do have
relatively precise measurements of the absolute values of $\Delta m^2_{21}$ and
$\Delta m^2_{32}$, and for the former that fact that MSW effect is playing a fundamental
role for solar neutrino allows to break the degeneracy and in the direction $m_2 > m_1$.
The third $\Delta m^2$, i.e., $\Delta m^2_{31}$, in not independent from the other
two, as
\begin{align*}
  \Delta m^2_{31} = \Delta m^2_{21} + \Delta m^2_{32} \approx \Delta m^2_{32}.
\end{align*}
In other words, we know that $\nu_1$ and $\nu_2$ consitute a closely-spaced doublet,
a relatively small small difference, while $\nu_3$ is comparatively further away.

What we don't know for sure, yet, is whether  $m_3 > m_{1,2}$ or $m_3 < m_{1,2}$;
we refer to the first case as \emph{normal ordering} and to the second as
\emph{inverse ordering} and we do a theroretical prejudice to impute the largest
mass to $m_3$, wich has the largest $\tau$ content---but it is fair to remember that
in the old days another such prejudice existed for the PMNS matrix to be quasi-diagonal,
as in the case of the CKM matrix, and that did not work out well.
\todo{Consider adding the standard plot of neutrino masses.}

\begin{marginfigure}
  %% Creator: Matplotlib, PGF backend
%%
%% To include the figure in your LaTeX document, write
%%   \input{<filename>.pgf}
%%
%% Make sure the required packages are loaded in your preamble
%%   \usepackage{pgf}
%%
%% Also ensure that all the required font packages are loaded; for instance,
%% the lmodern package is sometimes necessary when using math font.
%%   \usepackage{lmodern}
%%
%% Figures using additional raster images can only be included by \input if
%% they are in the same directory as the main LaTeX file. For loading figures
%% from other directories you can use the `import` package
%%   \usepackage{import}
%%
%% and then include the figures with
%%   \import{<path to file>}{<filename>.pgf}
%%
%% Matplotlib used the following preamble
%%   \usepackage{fontspec}
%%   \setmainfont{DejaVuSerif.ttf}[Path=\detokenize{/usr/share/matplotlib/mpl-data/fonts/ttf/}]
%%   \setsansfont{DejaVuSans.ttf}[Path=\detokenize{/usr/share/matplotlib/mpl-data/fonts/ttf/}]
%%   \setmonofont{DejaVuSansMono.ttf}[Path=\detokenize{/usr/share/matplotlib/mpl-data/fonts/ttf/}]
%%
\begingroup%
\makeatletter%
\begin{pgfpicture}%
\pgfpathrectangle{\pgfpointorigin}{\pgfqpoint{1.950000in}{2.250000in}}%
\pgfusepath{use as bounding box, clip}%
\begin{pgfscope}%
\pgfsetbuttcap%
\pgfsetmiterjoin%
\definecolor{currentfill}{rgb}{1.000000,1.000000,1.000000}%
\pgfsetfillcolor{currentfill}%
\pgfsetlinewidth{0.000000pt}%
\definecolor{currentstroke}{rgb}{1.000000,1.000000,1.000000}%
\pgfsetstrokecolor{currentstroke}%
\pgfsetdash{}{0pt}%
\pgfpathmoveto{\pgfqpoint{0.000000in}{0.000000in}}%
\pgfpathlineto{\pgfqpoint{1.950000in}{0.000000in}}%
\pgfpathlineto{\pgfqpoint{1.950000in}{2.250000in}}%
\pgfpathlineto{\pgfqpoint{0.000000in}{2.250000in}}%
\pgfpathlineto{\pgfqpoint{0.000000in}{0.000000in}}%
\pgfpathclose%
\pgfusepath{fill}%
\end{pgfscope}%
\begin{pgfscope}%
\pgfsetbuttcap%
\pgfsetmiterjoin%
\definecolor{currentfill}{rgb}{1.000000,1.000000,1.000000}%
\pgfsetfillcolor{currentfill}%
\pgfsetlinewidth{0.000000pt}%
\definecolor{currentstroke}{rgb}{0.000000,0.000000,0.000000}%
\pgfsetstrokecolor{currentstroke}%
\pgfsetstrokeopacity{0.000000}%
\pgfsetdash{}{0pt}%
\pgfpathmoveto{\pgfqpoint{0.726250in}{0.525000in}}%
\pgfpathlineto{\pgfqpoint{1.846250in}{0.525000in}}%
\pgfpathlineto{\pgfqpoint{1.846250in}{2.162500in}}%
\pgfpathlineto{\pgfqpoint{0.726250in}{2.162500in}}%
\pgfpathlineto{\pgfqpoint{0.726250in}{0.525000in}}%
\pgfpathclose%
\pgfusepath{fill}%
\end{pgfscope}%
\begin{pgfscope}%
\pgfpathrectangle{\pgfqpoint{0.726250in}{0.525000in}}{\pgfqpoint{1.120000in}{1.637500in}}%
\pgfusepath{clip}%
\pgfsetbuttcap%
\pgfsetroundjoin%
\pgfsetlinewidth{0.803000pt}%
\definecolor{currentstroke}{rgb}{0.752941,0.752941,0.752941}%
\pgfsetstrokecolor{currentstroke}%
\pgfsetdash{{2.960000pt}{1.280000pt}}{0.000000pt}%
\pgfpathmoveto{\pgfqpoint{0.726250in}{0.525000in}}%
\pgfpathlineto{\pgfqpoint{0.726250in}{2.162500in}}%
\pgfusepath{stroke}%
\end{pgfscope}%
\begin{pgfscope}%
\pgfsetbuttcap%
\pgfsetroundjoin%
\definecolor{currentfill}{rgb}{0.000000,0.000000,0.000000}%
\pgfsetfillcolor{currentfill}%
\pgfsetlinewidth{0.803000pt}%
\definecolor{currentstroke}{rgb}{0.000000,0.000000,0.000000}%
\pgfsetstrokecolor{currentstroke}%
\pgfsetdash{}{0pt}%
\pgfsys@defobject{currentmarker}{\pgfqpoint{0.000000in}{-0.048611in}}{\pgfqpoint{0.000000in}{0.000000in}}{%
\pgfpathmoveto{\pgfqpoint{0.000000in}{0.000000in}}%
\pgfpathlineto{\pgfqpoint{0.000000in}{-0.048611in}}%
\pgfusepath{stroke,fill}%
}%
\begin{pgfscope}%
\pgfsys@transformshift{0.726250in}{0.525000in}%
\pgfsys@useobject{currentmarker}{}%
\end{pgfscope}%
\end{pgfscope}%
\begin{pgfscope}%
\definecolor{textcolor}{rgb}{0.000000,0.000000,0.000000}%
\pgfsetstrokecolor{textcolor}%
\pgfsetfillcolor{textcolor}%
\pgftext[x=0.726250in,y=0.427778in,,top]{\color{textcolor}\rmfamily\fontsize{9.000000}{10.800000}\selectfont \(\displaystyle {10^{-3}}\)}%
\end{pgfscope}%
\begin{pgfscope}%
\pgfpathrectangle{\pgfqpoint{0.726250in}{0.525000in}}{\pgfqpoint{1.120000in}{1.637500in}}%
\pgfusepath{clip}%
\pgfsetbuttcap%
\pgfsetroundjoin%
\pgfsetlinewidth{0.803000pt}%
\definecolor{currentstroke}{rgb}{0.752941,0.752941,0.752941}%
\pgfsetstrokecolor{currentstroke}%
\pgfsetdash{{2.960000pt}{1.280000pt}}{0.000000pt}%
\pgfpathmoveto{\pgfqpoint{1.178388in}{0.525000in}}%
\pgfpathlineto{\pgfqpoint{1.178388in}{2.162500in}}%
\pgfusepath{stroke}%
\end{pgfscope}%
\begin{pgfscope}%
\pgfsetbuttcap%
\pgfsetroundjoin%
\definecolor{currentfill}{rgb}{0.000000,0.000000,0.000000}%
\pgfsetfillcolor{currentfill}%
\pgfsetlinewidth{0.803000pt}%
\definecolor{currentstroke}{rgb}{0.000000,0.000000,0.000000}%
\pgfsetstrokecolor{currentstroke}%
\pgfsetdash{}{0pt}%
\pgfsys@defobject{currentmarker}{\pgfqpoint{0.000000in}{-0.048611in}}{\pgfqpoint{0.000000in}{0.000000in}}{%
\pgfpathmoveto{\pgfqpoint{0.000000in}{0.000000in}}%
\pgfpathlineto{\pgfqpoint{0.000000in}{-0.048611in}}%
\pgfusepath{stroke,fill}%
}%
\begin{pgfscope}%
\pgfsys@transformshift{1.178388in}{0.525000in}%
\pgfsys@useobject{currentmarker}{}%
\end{pgfscope}%
\end{pgfscope}%
\begin{pgfscope}%
\definecolor{textcolor}{rgb}{0.000000,0.000000,0.000000}%
\pgfsetstrokecolor{textcolor}%
\pgfsetfillcolor{textcolor}%
\pgftext[x=1.178388in,y=0.427778in,,top]{\color{textcolor}\rmfamily\fontsize{9.000000}{10.800000}\selectfont \(\displaystyle {10^{-2}}\)}%
\end{pgfscope}%
\begin{pgfscope}%
\pgfpathrectangle{\pgfqpoint{0.726250in}{0.525000in}}{\pgfqpoint{1.120000in}{1.637500in}}%
\pgfusepath{clip}%
\pgfsetbuttcap%
\pgfsetroundjoin%
\pgfsetlinewidth{0.803000pt}%
\definecolor{currentstroke}{rgb}{0.752941,0.752941,0.752941}%
\pgfsetstrokecolor{currentstroke}%
\pgfsetdash{{2.960000pt}{1.280000pt}}{0.000000pt}%
\pgfpathmoveto{\pgfqpoint{1.630525in}{0.525000in}}%
\pgfpathlineto{\pgfqpoint{1.630525in}{2.162500in}}%
\pgfusepath{stroke}%
\end{pgfscope}%
\begin{pgfscope}%
\pgfsetbuttcap%
\pgfsetroundjoin%
\definecolor{currentfill}{rgb}{0.000000,0.000000,0.000000}%
\pgfsetfillcolor{currentfill}%
\pgfsetlinewidth{0.803000pt}%
\definecolor{currentstroke}{rgb}{0.000000,0.000000,0.000000}%
\pgfsetstrokecolor{currentstroke}%
\pgfsetdash{}{0pt}%
\pgfsys@defobject{currentmarker}{\pgfqpoint{0.000000in}{-0.048611in}}{\pgfqpoint{0.000000in}{0.000000in}}{%
\pgfpathmoveto{\pgfqpoint{0.000000in}{0.000000in}}%
\pgfpathlineto{\pgfqpoint{0.000000in}{-0.048611in}}%
\pgfusepath{stroke,fill}%
}%
\begin{pgfscope}%
\pgfsys@transformshift{1.630525in}{0.525000in}%
\pgfsys@useobject{currentmarker}{}%
\end{pgfscope}%
\end{pgfscope}%
\begin{pgfscope}%
\definecolor{textcolor}{rgb}{0.000000,0.000000,0.000000}%
\pgfsetstrokecolor{textcolor}%
\pgfsetfillcolor{textcolor}%
\pgftext[x=1.630525in,y=0.427778in,,top]{\color{textcolor}\rmfamily\fontsize{9.000000}{10.800000}\selectfont \(\displaystyle {10^{-1}}\)}%
\end{pgfscope}%
\begin{pgfscope}%
\pgfpathrectangle{\pgfqpoint{0.726250in}{0.525000in}}{\pgfqpoint{1.120000in}{1.637500in}}%
\pgfusepath{clip}%
\pgfsetbuttcap%
\pgfsetroundjoin%
\pgfsetlinewidth{0.803000pt}%
\definecolor{currentstroke}{rgb}{0.752941,0.752941,0.752941}%
\pgfsetstrokecolor{currentstroke}%
\pgfsetdash{{2.960000pt}{1.280000pt}}{0.000000pt}%
\pgfpathmoveto{\pgfqpoint{0.862357in}{0.525000in}}%
\pgfpathlineto{\pgfqpoint{0.862357in}{2.162500in}}%
\pgfusepath{stroke}%
\end{pgfscope}%
\begin{pgfscope}%
\pgfsetbuttcap%
\pgfsetroundjoin%
\definecolor{currentfill}{rgb}{0.000000,0.000000,0.000000}%
\pgfsetfillcolor{currentfill}%
\pgfsetlinewidth{0.602250pt}%
\definecolor{currentstroke}{rgb}{0.000000,0.000000,0.000000}%
\pgfsetstrokecolor{currentstroke}%
\pgfsetdash{}{0pt}%
\pgfsys@defobject{currentmarker}{\pgfqpoint{0.000000in}{-0.027778in}}{\pgfqpoint{0.000000in}{0.000000in}}{%
\pgfpathmoveto{\pgfqpoint{0.000000in}{0.000000in}}%
\pgfpathlineto{\pgfqpoint{0.000000in}{-0.027778in}}%
\pgfusepath{stroke,fill}%
}%
\begin{pgfscope}%
\pgfsys@transformshift{0.862357in}{0.525000in}%
\pgfsys@useobject{currentmarker}{}%
\end{pgfscope}%
\end{pgfscope}%
\begin{pgfscope}%
\pgfpathrectangle{\pgfqpoint{0.726250in}{0.525000in}}{\pgfqpoint{1.120000in}{1.637500in}}%
\pgfusepath{clip}%
\pgfsetbuttcap%
\pgfsetroundjoin%
\pgfsetlinewidth{0.803000pt}%
\definecolor{currentstroke}{rgb}{0.752941,0.752941,0.752941}%
\pgfsetstrokecolor{currentstroke}%
\pgfsetdash{{2.960000pt}{1.280000pt}}{0.000000pt}%
\pgfpathmoveto{\pgfqpoint{0.941975in}{0.525000in}}%
\pgfpathlineto{\pgfqpoint{0.941975in}{2.162500in}}%
\pgfusepath{stroke}%
\end{pgfscope}%
\begin{pgfscope}%
\pgfsetbuttcap%
\pgfsetroundjoin%
\definecolor{currentfill}{rgb}{0.000000,0.000000,0.000000}%
\pgfsetfillcolor{currentfill}%
\pgfsetlinewidth{0.602250pt}%
\definecolor{currentstroke}{rgb}{0.000000,0.000000,0.000000}%
\pgfsetstrokecolor{currentstroke}%
\pgfsetdash{}{0pt}%
\pgfsys@defobject{currentmarker}{\pgfqpoint{0.000000in}{-0.027778in}}{\pgfqpoint{0.000000in}{0.000000in}}{%
\pgfpathmoveto{\pgfqpoint{0.000000in}{0.000000in}}%
\pgfpathlineto{\pgfqpoint{0.000000in}{-0.027778in}}%
\pgfusepath{stroke,fill}%
}%
\begin{pgfscope}%
\pgfsys@transformshift{0.941975in}{0.525000in}%
\pgfsys@useobject{currentmarker}{}%
\end{pgfscope}%
\end{pgfscope}%
\begin{pgfscope}%
\pgfpathrectangle{\pgfqpoint{0.726250in}{0.525000in}}{\pgfqpoint{1.120000in}{1.637500in}}%
\pgfusepath{clip}%
\pgfsetbuttcap%
\pgfsetroundjoin%
\pgfsetlinewidth{0.803000pt}%
\definecolor{currentstroke}{rgb}{0.752941,0.752941,0.752941}%
\pgfsetstrokecolor{currentstroke}%
\pgfsetdash{{2.960000pt}{1.280000pt}}{0.000000pt}%
\pgfpathmoveto{\pgfqpoint{0.998464in}{0.525000in}}%
\pgfpathlineto{\pgfqpoint{0.998464in}{2.162500in}}%
\pgfusepath{stroke}%
\end{pgfscope}%
\begin{pgfscope}%
\pgfsetbuttcap%
\pgfsetroundjoin%
\definecolor{currentfill}{rgb}{0.000000,0.000000,0.000000}%
\pgfsetfillcolor{currentfill}%
\pgfsetlinewidth{0.602250pt}%
\definecolor{currentstroke}{rgb}{0.000000,0.000000,0.000000}%
\pgfsetstrokecolor{currentstroke}%
\pgfsetdash{}{0pt}%
\pgfsys@defobject{currentmarker}{\pgfqpoint{0.000000in}{-0.027778in}}{\pgfqpoint{0.000000in}{0.000000in}}{%
\pgfpathmoveto{\pgfqpoint{0.000000in}{0.000000in}}%
\pgfpathlineto{\pgfqpoint{0.000000in}{-0.027778in}}%
\pgfusepath{stroke,fill}%
}%
\begin{pgfscope}%
\pgfsys@transformshift{0.998464in}{0.525000in}%
\pgfsys@useobject{currentmarker}{}%
\end{pgfscope}%
\end{pgfscope}%
\begin{pgfscope}%
\pgfpathrectangle{\pgfqpoint{0.726250in}{0.525000in}}{\pgfqpoint{1.120000in}{1.637500in}}%
\pgfusepath{clip}%
\pgfsetbuttcap%
\pgfsetroundjoin%
\pgfsetlinewidth{0.803000pt}%
\definecolor{currentstroke}{rgb}{0.752941,0.752941,0.752941}%
\pgfsetstrokecolor{currentstroke}%
\pgfsetdash{{2.960000pt}{1.280000pt}}{0.000000pt}%
\pgfpathmoveto{\pgfqpoint{1.042281in}{0.525000in}}%
\pgfpathlineto{\pgfqpoint{1.042281in}{2.162500in}}%
\pgfusepath{stroke}%
\end{pgfscope}%
\begin{pgfscope}%
\pgfsetbuttcap%
\pgfsetroundjoin%
\definecolor{currentfill}{rgb}{0.000000,0.000000,0.000000}%
\pgfsetfillcolor{currentfill}%
\pgfsetlinewidth{0.602250pt}%
\definecolor{currentstroke}{rgb}{0.000000,0.000000,0.000000}%
\pgfsetstrokecolor{currentstroke}%
\pgfsetdash{}{0pt}%
\pgfsys@defobject{currentmarker}{\pgfqpoint{0.000000in}{-0.027778in}}{\pgfqpoint{0.000000in}{0.000000in}}{%
\pgfpathmoveto{\pgfqpoint{0.000000in}{0.000000in}}%
\pgfpathlineto{\pgfqpoint{0.000000in}{-0.027778in}}%
\pgfusepath{stroke,fill}%
}%
\begin{pgfscope}%
\pgfsys@transformshift{1.042281in}{0.525000in}%
\pgfsys@useobject{currentmarker}{}%
\end{pgfscope}%
\end{pgfscope}%
\begin{pgfscope}%
\pgfpathrectangle{\pgfqpoint{0.726250in}{0.525000in}}{\pgfqpoint{1.120000in}{1.637500in}}%
\pgfusepath{clip}%
\pgfsetbuttcap%
\pgfsetroundjoin%
\pgfsetlinewidth{0.803000pt}%
\definecolor{currentstroke}{rgb}{0.752941,0.752941,0.752941}%
\pgfsetstrokecolor{currentstroke}%
\pgfsetdash{{2.960000pt}{1.280000pt}}{0.000000pt}%
\pgfpathmoveto{\pgfqpoint{1.078082in}{0.525000in}}%
\pgfpathlineto{\pgfqpoint{1.078082in}{2.162500in}}%
\pgfusepath{stroke}%
\end{pgfscope}%
\begin{pgfscope}%
\pgfsetbuttcap%
\pgfsetroundjoin%
\definecolor{currentfill}{rgb}{0.000000,0.000000,0.000000}%
\pgfsetfillcolor{currentfill}%
\pgfsetlinewidth{0.602250pt}%
\definecolor{currentstroke}{rgb}{0.000000,0.000000,0.000000}%
\pgfsetstrokecolor{currentstroke}%
\pgfsetdash{}{0pt}%
\pgfsys@defobject{currentmarker}{\pgfqpoint{0.000000in}{-0.027778in}}{\pgfqpoint{0.000000in}{0.000000in}}{%
\pgfpathmoveto{\pgfqpoint{0.000000in}{0.000000in}}%
\pgfpathlineto{\pgfqpoint{0.000000in}{-0.027778in}}%
\pgfusepath{stroke,fill}%
}%
\begin{pgfscope}%
\pgfsys@transformshift{1.078082in}{0.525000in}%
\pgfsys@useobject{currentmarker}{}%
\end{pgfscope}%
\end{pgfscope}%
\begin{pgfscope}%
\pgfpathrectangle{\pgfqpoint{0.726250in}{0.525000in}}{\pgfqpoint{1.120000in}{1.637500in}}%
\pgfusepath{clip}%
\pgfsetbuttcap%
\pgfsetroundjoin%
\pgfsetlinewidth{0.803000pt}%
\definecolor{currentstroke}{rgb}{0.752941,0.752941,0.752941}%
\pgfsetstrokecolor{currentstroke}%
\pgfsetdash{{2.960000pt}{1.280000pt}}{0.000000pt}%
\pgfpathmoveto{\pgfqpoint{1.108351in}{0.525000in}}%
\pgfpathlineto{\pgfqpoint{1.108351in}{2.162500in}}%
\pgfusepath{stroke}%
\end{pgfscope}%
\begin{pgfscope}%
\pgfsetbuttcap%
\pgfsetroundjoin%
\definecolor{currentfill}{rgb}{0.000000,0.000000,0.000000}%
\pgfsetfillcolor{currentfill}%
\pgfsetlinewidth{0.602250pt}%
\definecolor{currentstroke}{rgb}{0.000000,0.000000,0.000000}%
\pgfsetstrokecolor{currentstroke}%
\pgfsetdash{}{0pt}%
\pgfsys@defobject{currentmarker}{\pgfqpoint{0.000000in}{-0.027778in}}{\pgfqpoint{0.000000in}{0.000000in}}{%
\pgfpathmoveto{\pgfqpoint{0.000000in}{0.000000in}}%
\pgfpathlineto{\pgfqpoint{0.000000in}{-0.027778in}}%
\pgfusepath{stroke,fill}%
}%
\begin{pgfscope}%
\pgfsys@transformshift{1.108351in}{0.525000in}%
\pgfsys@useobject{currentmarker}{}%
\end{pgfscope}%
\end{pgfscope}%
\begin{pgfscope}%
\pgfpathrectangle{\pgfqpoint{0.726250in}{0.525000in}}{\pgfqpoint{1.120000in}{1.637500in}}%
\pgfusepath{clip}%
\pgfsetbuttcap%
\pgfsetroundjoin%
\pgfsetlinewidth{0.803000pt}%
\definecolor{currentstroke}{rgb}{0.752941,0.752941,0.752941}%
\pgfsetstrokecolor{currentstroke}%
\pgfsetdash{{2.960000pt}{1.280000pt}}{0.000000pt}%
\pgfpathmoveto{\pgfqpoint{1.134571in}{0.525000in}}%
\pgfpathlineto{\pgfqpoint{1.134571in}{2.162500in}}%
\pgfusepath{stroke}%
\end{pgfscope}%
\begin{pgfscope}%
\pgfsetbuttcap%
\pgfsetroundjoin%
\definecolor{currentfill}{rgb}{0.000000,0.000000,0.000000}%
\pgfsetfillcolor{currentfill}%
\pgfsetlinewidth{0.602250pt}%
\definecolor{currentstroke}{rgb}{0.000000,0.000000,0.000000}%
\pgfsetstrokecolor{currentstroke}%
\pgfsetdash{}{0pt}%
\pgfsys@defobject{currentmarker}{\pgfqpoint{0.000000in}{-0.027778in}}{\pgfqpoint{0.000000in}{0.000000in}}{%
\pgfpathmoveto{\pgfqpoint{0.000000in}{0.000000in}}%
\pgfpathlineto{\pgfqpoint{0.000000in}{-0.027778in}}%
\pgfusepath{stroke,fill}%
}%
\begin{pgfscope}%
\pgfsys@transformshift{1.134571in}{0.525000in}%
\pgfsys@useobject{currentmarker}{}%
\end{pgfscope}%
\end{pgfscope}%
\begin{pgfscope}%
\pgfpathrectangle{\pgfqpoint{0.726250in}{0.525000in}}{\pgfqpoint{1.120000in}{1.637500in}}%
\pgfusepath{clip}%
\pgfsetbuttcap%
\pgfsetroundjoin%
\pgfsetlinewidth{0.803000pt}%
\definecolor{currentstroke}{rgb}{0.752941,0.752941,0.752941}%
\pgfsetstrokecolor{currentstroke}%
\pgfsetdash{{2.960000pt}{1.280000pt}}{0.000000pt}%
\pgfpathmoveto{\pgfqpoint{1.157699in}{0.525000in}}%
\pgfpathlineto{\pgfqpoint{1.157699in}{2.162500in}}%
\pgfusepath{stroke}%
\end{pgfscope}%
\begin{pgfscope}%
\pgfsetbuttcap%
\pgfsetroundjoin%
\definecolor{currentfill}{rgb}{0.000000,0.000000,0.000000}%
\pgfsetfillcolor{currentfill}%
\pgfsetlinewidth{0.602250pt}%
\definecolor{currentstroke}{rgb}{0.000000,0.000000,0.000000}%
\pgfsetstrokecolor{currentstroke}%
\pgfsetdash{}{0pt}%
\pgfsys@defobject{currentmarker}{\pgfqpoint{0.000000in}{-0.027778in}}{\pgfqpoint{0.000000in}{0.000000in}}{%
\pgfpathmoveto{\pgfqpoint{0.000000in}{0.000000in}}%
\pgfpathlineto{\pgfqpoint{0.000000in}{-0.027778in}}%
\pgfusepath{stroke,fill}%
}%
\begin{pgfscope}%
\pgfsys@transformshift{1.157699in}{0.525000in}%
\pgfsys@useobject{currentmarker}{}%
\end{pgfscope}%
\end{pgfscope}%
\begin{pgfscope}%
\pgfpathrectangle{\pgfqpoint{0.726250in}{0.525000in}}{\pgfqpoint{1.120000in}{1.637500in}}%
\pgfusepath{clip}%
\pgfsetbuttcap%
\pgfsetroundjoin%
\pgfsetlinewidth{0.803000pt}%
\definecolor{currentstroke}{rgb}{0.752941,0.752941,0.752941}%
\pgfsetstrokecolor{currentstroke}%
\pgfsetdash{{2.960000pt}{1.280000pt}}{0.000000pt}%
\pgfpathmoveto{\pgfqpoint{1.314495in}{0.525000in}}%
\pgfpathlineto{\pgfqpoint{1.314495in}{2.162500in}}%
\pgfusepath{stroke}%
\end{pgfscope}%
\begin{pgfscope}%
\pgfsetbuttcap%
\pgfsetroundjoin%
\definecolor{currentfill}{rgb}{0.000000,0.000000,0.000000}%
\pgfsetfillcolor{currentfill}%
\pgfsetlinewidth{0.602250pt}%
\definecolor{currentstroke}{rgb}{0.000000,0.000000,0.000000}%
\pgfsetstrokecolor{currentstroke}%
\pgfsetdash{}{0pt}%
\pgfsys@defobject{currentmarker}{\pgfqpoint{0.000000in}{-0.027778in}}{\pgfqpoint{0.000000in}{0.000000in}}{%
\pgfpathmoveto{\pgfqpoint{0.000000in}{0.000000in}}%
\pgfpathlineto{\pgfqpoint{0.000000in}{-0.027778in}}%
\pgfusepath{stroke,fill}%
}%
\begin{pgfscope}%
\pgfsys@transformshift{1.314495in}{0.525000in}%
\pgfsys@useobject{currentmarker}{}%
\end{pgfscope}%
\end{pgfscope}%
\begin{pgfscope}%
\pgfpathrectangle{\pgfqpoint{0.726250in}{0.525000in}}{\pgfqpoint{1.120000in}{1.637500in}}%
\pgfusepath{clip}%
\pgfsetbuttcap%
\pgfsetroundjoin%
\pgfsetlinewidth{0.803000pt}%
\definecolor{currentstroke}{rgb}{0.752941,0.752941,0.752941}%
\pgfsetstrokecolor{currentstroke}%
\pgfsetdash{{2.960000pt}{1.280000pt}}{0.000000pt}%
\pgfpathmoveto{\pgfqpoint{1.394112in}{0.525000in}}%
\pgfpathlineto{\pgfqpoint{1.394112in}{2.162500in}}%
\pgfusepath{stroke}%
\end{pgfscope}%
\begin{pgfscope}%
\pgfsetbuttcap%
\pgfsetroundjoin%
\definecolor{currentfill}{rgb}{0.000000,0.000000,0.000000}%
\pgfsetfillcolor{currentfill}%
\pgfsetlinewidth{0.602250pt}%
\definecolor{currentstroke}{rgb}{0.000000,0.000000,0.000000}%
\pgfsetstrokecolor{currentstroke}%
\pgfsetdash{}{0pt}%
\pgfsys@defobject{currentmarker}{\pgfqpoint{0.000000in}{-0.027778in}}{\pgfqpoint{0.000000in}{0.000000in}}{%
\pgfpathmoveto{\pgfqpoint{0.000000in}{0.000000in}}%
\pgfpathlineto{\pgfqpoint{0.000000in}{-0.027778in}}%
\pgfusepath{stroke,fill}%
}%
\begin{pgfscope}%
\pgfsys@transformshift{1.394112in}{0.525000in}%
\pgfsys@useobject{currentmarker}{}%
\end{pgfscope}%
\end{pgfscope}%
\begin{pgfscope}%
\pgfpathrectangle{\pgfqpoint{0.726250in}{0.525000in}}{\pgfqpoint{1.120000in}{1.637500in}}%
\pgfusepath{clip}%
\pgfsetbuttcap%
\pgfsetroundjoin%
\pgfsetlinewidth{0.803000pt}%
\definecolor{currentstroke}{rgb}{0.752941,0.752941,0.752941}%
\pgfsetstrokecolor{currentstroke}%
\pgfsetdash{{2.960000pt}{1.280000pt}}{0.000000pt}%
\pgfpathmoveto{\pgfqpoint{1.450602in}{0.525000in}}%
\pgfpathlineto{\pgfqpoint{1.450602in}{2.162500in}}%
\pgfusepath{stroke}%
\end{pgfscope}%
\begin{pgfscope}%
\pgfsetbuttcap%
\pgfsetroundjoin%
\definecolor{currentfill}{rgb}{0.000000,0.000000,0.000000}%
\pgfsetfillcolor{currentfill}%
\pgfsetlinewidth{0.602250pt}%
\definecolor{currentstroke}{rgb}{0.000000,0.000000,0.000000}%
\pgfsetstrokecolor{currentstroke}%
\pgfsetdash{}{0pt}%
\pgfsys@defobject{currentmarker}{\pgfqpoint{0.000000in}{-0.027778in}}{\pgfqpoint{0.000000in}{0.000000in}}{%
\pgfpathmoveto{\pgfqpoint{0.000000in}{0.000000in}}%
\pgfpathlineto{\pgfqpoint{0.000000in}{-0.027778in}}%
\pgfusepath{stroke,fill}%
}%
\begin{pgfscope}%
\pgfsys@transformshift{1.450602in}{0.525000in}%
\pgfsys@useobject{currentmarker}{}%
\end{pgfscope}%
\end{pgfscope}%
\begin{pgfscope}%
\pgfpathrectangle{\pgfqpoint{0.726250in}{0.525000in}}{\pgfqpoint{1.120000in}{1.637500in}}%
\pgfusepath{clip}%
\pgfsetbuttcap%
\pgfsetroundjoin%
\pgfsetlinewidth{0.803000pt}%
\definecolor{currentstroke}{rgb}{0.752941,0.752941,0.752941}%
\pgfsetstrokecolor{currentstroke}%
\pgfsetdash{{2.960000pt}{1.280000pt}}{0.000000pt}%
\pgfpathmoveto{\pgfqpoint{1.494418in}{0.525000in}}%
\pgfpathlineto{\pgfqpoint{1.494418in}{2.162500in}}%
\pgfusepath{stroke}%
\end{pgfscope}%
\begin{pgfscope}%
\pgfsetbuttcap%
\pgfsetroundjoin%
\definecolor{currentfill}{rgb}{0.000000,0.000000,0.000000}%
\pgfsetfillcolor{currentfill}%
\pgfsetlinewidth{0.602250pt}%
\definecolor{currentstroke}{rgb}{0.000000,0.000000,0.000000}%
\pgfsetstrokecolor{currentstroke}%
\pgfsetdash{}{0pt}%
\pgfsys@defobject{currentmarker}{\pgfqpoint{0.000000in}{-0.027778in}}{\pgfqpoint{0.000000in}{0.000000in}}{%
\pgfpathmoveto{\pgfqpoint{0.000000in}{0.000000in}}%
\pgfpathlineto{\pgfqpoint{0.000000in}{-0.027778in}}%
\pgfusepath{stroke,fill}%
}%
\begin{pgfscope}%
\pgfsys@transformshift{1.494418in}{0.525000in}%
\pgfsys@useobject{currentmarker}{}%
\end{pgfscope}%
\end{pgfscope}%
\begin{pgfscope}%
\pgfpathrectangle{\pgfqpoint{0.726250in}{0.525000in}}{\pgfqpoint{1.120000in}{1.637500in}}%
\pgfusepath{clip}%
\pgfsetbuttcap%
\pgfsetroundjoin%
\pgfsetlinewidth{0.803000pt}%
\definecolor{currentstroke}{rgb}{0.752941,0.752941,0.752941}%
\pgfsetstrokecolor{currentstroke}%
\pgfsetdash{{2.960000pt}{1.280000pt}}{0.000000pt}%
\pgfpathmoveto{\pgfqpoint{1.530219in}{0.525000in}}%
\pgfpathlineto{\pgfqpoint{1.530219in}{2.162500in}}%
\pgfusepath{stroke}%
\end{pgfscope}%
\begin{pgfscope}%
\pgfsetbuttcap%
\pgfsetroundjoin%
\definecolor{currentfill}{rgb}{0.000000,0.000000,0.000000}%
\pgfsetfillcolor{currentfill}%
\pgfsetlinewidth{0.602250pt}%
\definecolor{currentstroke}{rgb}{0.000000,0.000000,0.000000}%
\pgfsetstrokecolor{currentstroke}%
\pgfsetdash{}{0pt}%
\pgfsys@defobject{currentmarker}{\pgfqpoint{0.000000in}{-0.027778in}}{\pgfqpoint{0.000000in}{0.000000in}}{%
\pgfpathmoveto{\pgfqpoint{0.000000in}{0.000000in}}%
\pgfpathlineto{\pgfqpoint{0.000000in}{-0.027778in}}%
\pgfusepath{stroke,fill}%
}%
\begin{pgfscope}%
\pgfsys@transformshift{1.530219in}{0.525000in}%
\pgfsys@useobject{currentmarker}{}%
\end{pgfscope}%
\end{pgfscope}%
\begin{pgfscope}%
\pgfpathrectangle{\pgfqpoint{0.726250in}{0.525000in}}{\pgfqpoint{1.120000in}{1.637500in}}%
\pgfusepath{clip}%
\pgfsetbuttcap%
\pgfsetroundjoin%
\pgfsetlinewidth{0.803000pt}%
\definecolor{currentstroke}{rgb}{0.752941,0.752941,0.752941}%
\pgfsetstrokecolor{currentstroke}%
\pgfsetdash{{2.960000pt}{1.280000pt}}{0.000000pt}%
\pgfpathmoveto{\pgfqpoint{1.560488in}{0.525000in}}%
\pgfpathlineto{\pgfqpoint{1.560488in}{2.162500in}}%
\pgfusepath{stroke}%
\end{pgfscope}%
\begin{pgfscope}%
\pgfsetbuttcap%
\pgfsetroundjoin%
\definecolor{currentfill}{rgb}{0.000000,0.000000,0.000000}%
\pgfsetfillcolor{currentfill}%
\pgfsetlinewidth{0.602250pt}%
\definecolor{currentstroke}{rgb}{0.000000,0.000000,0.000000}%
\pgfsetstrokecolor{currentstroke}%
\pgfsetdash{}{0pt}%
\pgfsys@defobject{currentmarker}{\pgfqpoint{0.000000in}{-0.027778in}}{\pgfqpoint{0.000000in}{0.000000in}}{%
\pgfpathmoveto{\pgfqpoint{0.000000in}{0.000000in}}%
\pgfpathlineto{\pgfqpoint{0.000000in}{-0.027778in}}%
\pgfusepath{stroke,fill}%
}%
\begin{pgfscope}%
\pgfsys@transformshift{1.560488in}{0.525000in}%
\pgfsys@useobject{currentmarker}{}%
\end{pgfscope}%
\end{pgfscope}%
\begin{pgfscope}%
\pgfpathrectangle{\pgfqpoint{0.726250in}{0.525000in}}{\pgfqpoint{1.120000in}{1.637500in}}%
\pgfusepath{clip}%
\pgfsetbuttcap%
\pgfsetroundjoin%
\pgfsetlinewidth{0.803000pt}%
\definecolor{currentstroke}{rgb}{0.752941,0.752941,0.752941}%
\pgfsetstrokecolor{currentstroke}%
\pgfsetdash{{2.960000pt}{1.280000pt}}{0.000000pt}%
\pgfpathmoveto{\pgfqpoint{1.586709in}{0.525000in}}%
\pgfpathlineto{\pgfqpoint{1.586709in}{2.162500in}}%
\pgfusepath{stroke}%
\end{pgfscope}%
\begin{pgfscope}%
\pgfsetbuttcap%
\pgfsetroundjoin%
\definecolor{currentfill}{rgb}{0.000000,0.000000,0.000000}%
\pgfsetfillcolor{currentfill}%
\pgfsetlinewidth{0.602250pt}%
\definecolor{currentstroke}{rgb}{0.000000,0.000000,0.000000}%
\pgfsetstrokecolor{currentstroke}%
\pgfsetdash{}{0pt}%
\pgfsys@defobject{currentmarker}{\pgfqpoint{0.000000in}{-0.027778in}}{\pgfqpoint{0.000000in}{0.000000in}}{%
\pgfpathmoveto{\pgfqpoint{0.000000in}{0.000000in}}%
\pgfpathlineto{\pgfqpoint{0.000000in}{-0.027778in}}%
\pgfusepath{stroke,fill}%
}%
\begin{pgfscope}%
\pgfsys@transformshift{1.586709in}{0.525000in}%
\pgfsys@useobject{currentmarker}{}%
\end{pgfscope}%
\end{pgfscope}%
\begin{pgfscope}%
\pgfpathrectangle{\pgfqpoint{0.726250in}{0.525000in}}{\pgfqpoint{1.120000in}{1.637500in}}%
\pgfusepath{clip}%
\pgfsetbuttcap%
\pgfsetroundjoin%
\pgfsetlinewidth{0.803000pt}%
\definecolor{currentstroke}{rgb}{0.752941,0.752941,0.752941}%
\pgfsetstrokecolor{currentstroke}%
\pgfsetdash{{2.960000pt}{1.280000pt}}{0.000000pt}%
\pgfpathmoveto{\pgfqpoint{1.609837in}{0.525000in}}%
\pgfpathlineto{\pgfqpoint{1.609837in}{2.162500in}}%
\pgfusepath{stroke}%
\end{pgfscope}%
\begin{pgfscope}%
\pgfsetbuttcap%
\pgfsetroundjoin%
\definecolor{currentfill}{rgb}{0.000000,0.000000,0.000000}%
\pgfsetfillcolor{currentfill}%
\pgfsetlinewidth{0.602250pt}%
\definecolor{currentstroke}{rgb}{0.000000,0.000000,0.000000}%
\pgfsetstrokecolor{currentstroke}%
\pgfsetdash{}{0pt}%
\pgfsys@defobject{currentmarker}{\pgfqpoint{0.000000in}{-0.027778in}}{\pgfqpoint{0.000000in}{0.000000in}}{%
\pgfpathmoveto{\pgfqpoint{0.000000in}{0.000000in}}%
\pgfpathlineto{\pgfqpoint{0.000000in}{-0.027778in}}%
\pgfusepath{stroke,fill}%
}%
\begin{pgfscope}%
\pgfsys@transformshift{1.609837in}{0.525000in}%
\pgfsys@useobject{currentmarker}{}%
\end{pgfscope}%
\end{pgfscope}%
\begin{pgfscope}%
\pgfpathrectangle{\pgfqpoint{0.726250in}{0.525000in}}{\pgfqpoint{1.120000in}{1.637500in}}%
\pgfusepath{clip}%
\pgfsetbuttcap%
\pgfsetroundjoin%
\pgfsetlinewidth{0.803000pt}%
\definecolor{currentstroke}{rgb}{0.752941,0.752941,0.752941}%
\pgfsetstrokecolor{currentstroke}%
\pgfsetdash{{2.960000pt}{1.280000pt}}{0.000000pt}%
\pgfpathmoveto{\pgfqpoint{1.766632in}{0.525000in}}%
\pgfpathlineto{\pgfqpoint{1.766632in}{2.162500in}}%
\pgfusepath{stroke}%
\end{pgfscope}%
\begin{pgfscope}%
\pgfsetbuttcap%
\pgfsetroundjoin%
\definecolor{currentfill}{rgb}{0.000000,0.000000,0.000000}%
\pgfsetfillcolor{currentfill}%
\pgfsetlinewidth{0.602250pt}%
\definecolor{currentstroke}{rgb}{0.000000,0.000000,0.000000}%
\pgfsetstrokecolor{currentstroke}%
\pgfsetdash{}{0pt}%
\pgfsys@defobject{currentmarker}{\pgfqpoint{0.000000in}{-0.027778in}}{\pgfqpoint{0.000000in}{0.000000in}}{%
\pgfpathmoveto{\pgfqpoint{0.000000in}{0.000000in}}%
\pgfpathlineto{\pgfqpoint{0.000000in}{-0.027778in}}%
\pgfusepath{stroke,fill}%
}%
\begin{pgfscope}%
\pgfsys@transformshift{1.766632in}{0.525000in}%
\pgfsys@useobject{currentmarker}{}%
\end{pgfscope}%
\end{pgfscope}%
\begin{pgfscope}%
\pgfpathrectangle{\pgfqpoint{0.726250in}{0.525000in}}{\pgfqpoint{1.120000in}{1.637500in}}%
\pgfusepath{clip}%
\pgfsetbuttcap%
\pgfsetroundjoin%
\pgfsetlinewidth{0.803000pt}%
\definecolor{currentstroke}{rgb}{0.752941,0.752941,0.752941}%
\pgfsetstrokecolor{currentstroke}%
\pgfsetdash{{2.960000pt}{1.280000pt}}{0.000000pt}%
\pgfpathmoveto{\pgfqpoint{1.846250in}{0.525000in}}%
\pgfpathlineto{\pgfqpoint{1.846250in}{2.162500in}}%
\pgfusepath{stroke}%
\end{pgfscope}%
\begin{pgfscope}%
\pgfsetbuttcap%
\pgfsetroundjoin%
\definecolor{currentfill}{rgb}{0.000000,0.000000,0.000000}%
\pgfsetfillcolor{currentfill}%
\pgfsetlinewidth{0.602250pt}%
\definecolor{currentstroke}{rgb}{0.000000,0.000000,0.000000}%
\pgfsetstrokecolor{currentstroke}%
\pgfsetdash{}{0pt}%
\pgfsys@defobject{currentmarker}{\pgfqpoint{0.000000in}{-0.027778in}}{\pgfqpoint{0.000000in}{0.000000in}}{%
\pgfpathmoveto{\pgfqpoint{0.000000in}{0.000000in}}%
\pgfpathlineto{\pgfqpoint{0.000000in}{-0.027778in}}%
\pgfusepath{stroke,fill}%
}%
\begin{pgfscope}%
\pgfsys@transformshift{1.846250in}{0.525000in}%
\pgfsys@useobject{currentmarker}{}%
\end{pgfscope}%
\end{pgfscope}%
\begin{pgfscope}%
\definecolor{textcolor}{rgb}{0.000000,0.000000,0.000000}%
\pgfsetstrokecolor{textcolor}%
\pgfsetfillcolor{textcolor}%
\pgftext[x=1.286250in,y=0.251251in,,top]{\color{textcolor}\rmfamily\fontsize{9.000000}{10.800000}\selectfont \(\displaystyle m_1\) [eV]}%
\end{pgfscope}%
\begin{pgfscope}%
\pgfpathrectangle{\pgfqpoint{0.726250in}{0.525000in}}{\pgfqpoint{1.120000in}{1.637500in}}%
\pgfusepath{clip}%
\pgfsetbuttcap%
\pgfsetroundjoin%
\pgfsetlinewidth{0.803000pt}%
\definecolor{currentstroke}{rgb}{0.752941,0.752941,0.752941}%
\pgfsetstrokecolor{currentstroke}%
\pgfsetdash{{2.960000pt}{1.280000pt}}{0.000000pt}%
\pgfpathmoveto{\pgfqpoint{0.726250in}{0.802219in}}%
\pgfpathlineto{\pgfqpoint{1.846250in}{0.802219in}}%
\pgfusepath{stroke}%
\end{pgfscope}%
\begin{pgfscope}%
\pgfsetbuttcap%
\pgfsetroundjoin%
\definecolor{currentfill}{rgb}{0.000000,0.000000,0.000000}%
\pgfsetfillcolor{currentfill}%
\pgfsetlinewidth{0.803000pt}%
\definecolor{currentstroke}{rgb}{0.000000,0.000000,0.000000}%
\pgfsetstrokecolor{currentstroke}%
\pgfsetdash{}{0pt}%
\pgfsys@defobject{currentmarker}{\pgfqpoint{-0.048611in}{0.000000in}}{\pgfqpoint{-0.000000in}{0.000000in}}{%
\pgfpathmoveto{\pgfqpoint{-0.000000in}{0.000000in}}%
\pgfpathlineto{\pgfqpoint{-0.048611in}{0.000000in}}%
\pgfusepath{stroke,fill}%
}%
\begin{pgfscope}%
\pgfsys@transformshift{0.726250in}{0.802219in}%
\pgfsys@useobject{currentmarker}{}%
\end{pgfscope}%
\end{pgfscope}%
\begin{pgfscope}%
\definecolor{textcolor}{rgb}{0.000000,0.000000,0.000000}%
\pgfsetstrokecolor{textcolor}%
\pgfsetfillcolor{textcolor}%
\pgftext[x=0.362441in, y=0.754733in, left, base]{\color{textcolor}\rmfamily\fontsize{9.000000}{10.800000}\selectfont \(\displaystyle {10^{-2}}\)}%
\end{pgfscope}%
\begin{pgfscope}%
\pgfpathrectangle{\pgfqpoint{0.726250in}{0.525000in}}{\pgfqpoint{1.120000in}{1.637500in}}%
\pgfusepath{clip}%
\pgfsetbuttcap%
\pgfsetroundjoin%
\pgfsetlinewidth{0.803000pt}%
\definecolor{currentstroke}{rgb}{0.752941,0.752941,0.752941}%
\pgfsetstrokecolor{currentstroke}%
\pgfsetdash{{2.960000pt}{1.280000pt}}{0.000000pt}%
\pgfpathmoveto{\pgfqpoint{0.726250in}{1.723119in}}%
\pgfpathlineto{\pgfqpoint{1.846250in}{1.723119in}}%
\pgfusepath{stroke}%
\end{pgfscope}%
\begin{pgfscope}%
\pgfsetbuttcap%
\pgfsetroundjoin%
\definecolor{currentfill}{rgb}{0.000000,0.000000,0.000000}%
\pgfsetfillcolor{currentfill}%
\pgfsetlinewidth{0.803000pt}%
\definecolor{currentstroke}{rgb}{0.000000,0.000000,0.000000}%
\pgfsetstrokecolor{currentstroke}%
\pgfsetdash{}{0pt}%
\pgfsys@defobject{currentmarker}{\pgfqpoint{-0.048611in}{0.000000in}}{\pgfqpoint{-0.000000in}{0.000000in}}{%
\pgfpathmoveto{\pgfqpoint{-0.000000in}{0.000000in}}%
\pgfpathlineto{\pgfqpoint{-0.048611in}{0.000000in}}%
\pgfusepath{stroke,fill}%
}%
\begin{pgfscope}%
\pgfsys@transformshift{0.726250in}{1.723119in}%
\pgfsys@useobject{currentmarker}{}%
\end{pgfscope}%
\end{pgfscope}%
\begin{pgfscope}%
\definecolor{textcolor}{rgb}{0.000000,0.000000,0.000000}%
\pgfsetstrokecolor{textcolor}%
\pgfsetfillcolor{textcolor}%
\pgftext[x=0.362441in, y=1.675634in, left, base]{\color{textcolor}\rmfamily\fontsize{9.000000}{10.800000}\selectfont \(\displaystyle {10^{-1}}\)}%
\end{pgfscope}%
\begin{pgfscope}%
\pgfpathrectangle{\pgfqpoint{0.726250in}{0.525000in}}{\pgfqpoint{1.120000in}{1.637500in}}%
\pgfusepath{clip}%
\pgfsetbuttcap%
\pgfsetroundjoin%
\pgfsetlinewidth{0.803000pt}%
\definecolor{currentstroke}{rgb}{0.752941,0.752941,0.752941}%
\pgfsetstrokecolor{currentstroke}%
\pgfsetdash{{2.960000pt}{1.280000pt}}{0.000000pt}%
\pgfpathmoveto{\pgfqpoint{0.726250in}{0.525000in}}%
\pgfpathlineto{\pgfqpoint{1.846250in}{0.525000in}}%
\pgfusepath{stroke}%
\end{pgfscope}%
\begin{pgfscope}%
\pgfsetbuttcap%
\pgfsetroundjoin%
\definecolor{currentfill}{rgb}{0.000000,0.000000,0.000000}%
\pgfsetfillcolor{currentfill}%
\pgfsetlinewidth{0.602250pt}%
\definecolor{currentstroke}{rgb}{0.000000,0.000000,0.000000}%
\pgfsetstrokecolor{currentstroke}%
\pgfsetdash{}{0pt}%
\pgfsys@defobject{currentmarker}{\pgfqpoint{-0.027778in}{0.000000in}}{\pgfqpoint{-0.000000in}{0.000000in}}{%
\pgfpathmoveto{\pgfqpoint{-0.000000in}{0.000000in}}%
\pgfpathlineto{\pgfqpoint{-0.027778in}{0.000000in}}%
\pgfusepath{stroke,fill}%
}%
\begin{pgfscope}%
\pgfsys@transformshift{0.726250in}{0.525000in}%
\pgfsys@useobject{currentmarker}{}%
\end{pgfscope}%
\end{pgfscope}%
\begin{pgfscope}%
\pgfpathrectangle{\pgfqpoint{0.726250in}{0.525000in}}{\pgfqpoint{1.120000in}{1.637500in}}%
\pgfusepath{clip}%
\pgfsetbuttcap%
\pgfsetroundjoin%
\pgfsetlinewidth{0.803000pt}%
\definecolor{currentstroke}{rgb}{0.752941,0.752941,0.752941}%
\pgfsetstrokecolor{currentstroke}%
\pgfsetdash{{2.960000pt}{1.280000pt}}{0.000000pt}%
\pgfpathmoveto{\pgfqpoint{0.726250in}{0.597918in}}%
\pgfpathlineto{\pgfqpoint{1.846250in}{0.597918in}}%
\pgfusepath{stroke}%
\end{pgfscope}%
\begin{pgfscope}%
\pgfsetbuttcap%
\pgfsetroundjoin%
\definecolor{currentfill}{rgb}{0.000000,0.000000,0.000000}%
\pgfsetfillcolor{currentfill}%
\pgfsetlinewidth{0.602250pt}%
\definecolor{currentstroke}{rgb}{0.000000,0.000000,0.000000}%
\pgfsetstrokecolor{currentstroke}%
\pgfsetdash{}{0pt}%
\pgfsys@defobject{currentmarker}{\pgfqpoint{-0.027778in}{0.000000in}}{\pgfqpoint{-0.000000in}{0.000000in}}{%
\pgfpathmoveto{\pgfqpoint{-0.000000in}{0.000000in}}%
\pgfpathlineto{\pgfqpoint{-0.027778in}{0.000000in}}%
\pgfusepath{stroke,fill}%
}%
\begin{pgfscope}%
\pgfsys@transformshift{0.726250in}{0.597918in}%
\pgfsys@useobject{currentmarker}{}%
\end{pgfscope}%
\end{pgfscope}%
\begin{pgfscope}%
\pgfpathrectangle{\pgfqpoint{0.726250in}{0.525000in}}{\pgfqpoint{1.120000in}{1.637500in}}%
\pgfusepath{clip}%
\pgfsetbuttcap%
\pgfsetroundjoin%
\pgfsetlinewidth{0.803000pt}%
\definecolor{currentstroke}{rgb}{0.752941,0.752941,0.752941}%
\pgfsetstrokecolor{currentstroke}%
\pgfsetdash{{2.960000pt}{1.280000pt}}{0.000000pt}%
\pgfpathmoveto{\pgfqpoint{0.726250in}{0.659569in}}%
\pgfpathlineto{\pgfqpoint{1.846250in}{0.659569in}}%
\pgfusepath{stroke}%
\end{pgfscope}%
\begin{pgfscope}%
\pgfsetbuttcap%
\pgfsetroundjoin%
\definecolor{currentfill}{rgb}{0.000000,0.000000,0.000000}%
\pgfsetfillcolor{currentfill}%
\pgfsetlinewidth{0.602250pt}%
\definecolor{currentstroke}{rgb}{0.000000,0.000000,0.000000}%
\pgfsetstrokecolor{currentstroke}%
\pgfsetdash{}{0pt}%
\pgfsys@defobject{currentmarker}{\pgfqpoint{-0.027778in}{0.000000in}}{\pgfqpoint{-0.000000in}{0.000000in}}{%
\pgfpathmoveto{\pgfqpoint{-0.000000in}{0.000000in}}%
\pgfpathlineto{\pgfqpoint{-0.027778in}{0.000000in}}%
\pgfusepath{stroke,fill}%
}%
\begin{pgfscope}%
\pgfsys@transformshift{0.726250in}{0.659569in}%
\pgfsys@useobject{currentmarker}{}%
\end{pgfscope}%
\end{pgfscope}%
\begin{pgfscope}%
\pgfpathrectangle{\pgfqpoint{0.726250in}{0.525000in}}{\pgfqpoint{1.120000in}{1.637500in}}%
\pgfusepath{clip}%
\pgfsetbuttcap%
\pgfsetroundjoin%
\pgfsetlinewidth{0.803000pt}%
\definecolor{currentstroke}{rgb}{0.752941,0.752941,0.752941}%
\pgfsetstrokecolor{currentstroke}%
\pgfsetdash{{2.960000pt}{1.280000pt}}{0.000000pt}%
\pgfpathmoveto{\pgfqpoint{0.726250in}{0.712974in}}%
\pgfpathlineto{\pgfqpoint{1.846250in}{0.712974in}}%
\pgfusepath{stroke}%
\end{pgfscope}%
\begin{pgfscope}%
\pgfsetbuttcap%
\pgfsetroundjoin%
\definecolor{currentfill}{rgb}{0.000000,0.000000,0.000000}%
\pgfsetfillcolor{currentfill}%
\pgfsetlinewidth{0.602250pt}%
\definecolor{currentstroke}{rgb}{0.000000,0.000000,0.000000}%
\pgfsetstrokecolor{currentstroke}%
\pgfsetdash{}{0pt}%
\pgfsys@defobject{currentmarker}{\pgfqpoint{-0.027778in}{0.000000in}}{\pgfqpoint{-0.000000in}{0.000000in}}{%
\pgfpathmoveto{\pgfqpoint{-0.000000in}{0.000000in}}%
\pgfpathlineto{\pgfqpoint{-0.027778in}{0.000000in}}%
\pgfusepath{stroke,fill}%
}%
\begin{pgfscope}%
\pgfsys@transformshift{0.726250in}{0.712974in}%
\pgfsys@useobject{currentmarker}{}%
\end{pgfscope}%
\end{pgfscope}%
\begin{pgfscope}%
\pgfpathrectangle{\pgfqpoint{0.726250in}{0.525000in}}{\pgfqpoint{1.120000in}{1.637500in}}%
\pgfusepath{clip}%
\pgfsetbuttcap%
\pgfsetroundjoin%
\pgfsetlinewidth{0.803000pt}%
\definecolor{currentstroke}{rgb}{0.752941,0.752941,0.752941}%
\pgfsetstrokecolor{currentstroke}%
\pgfsetdash{{2.960000pt}{1.280000pt}}{0.000000pt}%
\pgfpathmoveto{\pgfqpoint{0.726250in}{0.760081in}}%
\pgfpathlineto{\pgfqpoint{1.846250in}{0.760081in}}%
\pgfusepath{stroke}%
\end{pgfscope}%
\begin{pgfscope}%
\pgfsetbuttcap%
\pgfsetroundjoin%
\definecolor{currentfill}{rgb}{0.000000,0.000000,0.000000}%
\pgfsetfillcolor{currentfill}%
\pgfsetlinewidth{0.602250pt}%
\definecolor{currentstroke}{rgb}{0.000000,0.000000,0.000000}%
\pgfsetstrokecolor{currentstroke}%
\pgfsetdash{}{0pt}%
\pgfsys@defobject{currentmarker}{\pgfqpoint{-0.027778in}{0.000000in}}{\pgfqpoint{-0.000000in}{0.000000in}}{%
\pgfpathmoveto{\pgfqpoint{-0.000000in}{0.000000in}}%
\pgfpathlineto{\pgfqpoint{-0.027778in}{0.000000in}}%
\pgfusepath{stroke,fill}%
}%
\begin{pgfscope}%
\pgfsys@transformshift{0.726250in}{0.760081in}%
\pgfsys@useobject{currentmarker}{}%
\end{pgfscope}%
\end{pgfscope}%
\begin{pgfscope}%
\pgfpathrectangle{\pgfqpoint{0.726250in}{0.525000in}}{\pgfqpoint{1.120000in}{1.637500in}}%
\pgfusepath{clip}%
\pgfsetbuttcap%
\pgfsetroundjoin%
\pgfsetlinewidth{0.803000pt}%
\definecolor{currentstroke}{rgb}{0.752941,0.752941,0.752941}%
\pgfsetstrokecolor{currentstroke}%
\pgfsetdash{{2.960000pt}{1.280000pt}}{0.000000pt}%
\pgfpathmoveto{\pgfqpoint{0.726250in}{1.079437in}}%
\pgfpathlineto{\pgfqpoint{1.846250in}{1.079437in}}%
\pgfusepath{stroke}%
\end{pgfscope}%
\begin{pgfscope}%
\pgfsetbuttcap%
\pgfsetroundjoin%
\definecolor{currentfill}{rgb}{0.000000,0.000000,0.000000}%
\pgfsetfillcolor{currentfill}%
\pgfsetlinewidth{0.602250pt}%
\definecolor{currentstroke}{rgb}{0.000000,0.000000,0.000000}%
\pgfsetstrokecolor{currentstroke}%
\pgfsetdash{}{0pt}%
\pgfsys@defobject{currentmarker}{\pgfqpoint{-0.027778in}{0.000000in}}{\pgfqpoint{-0.000000in}{0.000000in}}{%
\pgfpathmoveto{\pgfqpoint{-0.000000in}{0.000000in}}%
\pgfpathlineto{\pgfqpoint{-0.027778in}{0.000000in}}%
\pgfusepath{stroke,fill}%
}%
\begin{pgfscope}%
\pgfsys@transformshift{0.726250in}{1.079437in}%
\pgfsys@useobject{currentmarker}{}%
\end{pgfscope}%
\end{pgfscope}%
\begin{pgfscope}%
\pgfpathrectangle{\pgfqpoint{0.726250in}{0.525000in}}{\pgfqpoint{1.120000in}{1.637500in}}%
\pgfusepath{clip}%
\pgfsetbuttcap%
\pgfsetroundjoin%
\pgfsetlinewidth{0.803000pt}%
\definecolor{currentstroke}{rgb}{0.752941,0.752941,0.752941}%
\pgfsetstrokecolor{currentstroke}%
\pgfsetdash{{2.960000pt}{1.280000pt}}{0.000000pt}%
\pgfpathmoveto{\pgfqpoint{0.726250in}{1.241600in}}%
\pgfpathlineto{\pgfqpoint{1.846250in}{1.241600in}}%
\pgfusepath{stroke}%
\end{pgfscope}%
\begin{pgfscope}%
\pgfsetbuttcap%
\pgfsetroundjoin%
\definecolor{currentfill}{rgb}{0.000000,0.000000,0.000000}%
\pgfsetfillcolor{currentfill}%
\pgfsetlinewidth{0.602250pt}%
\definecolor{currentstroke}{rgb}{0.000000,0.000000,0.000000}%
\pgfsetstrokecolor{currentstroke}%
\pgfsetdash{}{0pt}%
\pgfsys@defobject{currentmarker}{\pgfqpoint{-0.027778in}{0.000000in}}{\pgfqpoint{-0.000000in}{0.000000in}}{%
\pgfpathmoveto{\pgfqpoint{-0.000000in}{0.000000in}}%
\pgfpathlineto{\pgfqpoint{-0.027778in}{0.000000in}}%
\pgfusepath{stroke,fill}%
}%
\begin{pgfscope}%
\pgfsys@transformshift{0.726250in}{1.241600in}%
\pgfsys@useobject{currentmarker}{}%
\end{pgfscope}%
\end{pgfscope}%
\begin{pgfscope}%
\pgfpathrectangle{\pgfqpoint{0.726250in}{0.525000in}}{\pgfqpoint{1.120000in}{1.637500in}}%
\pgfusepath{clip}%
\pgfsetbuttcap%
\pgfsetroundjoin%
\pgfsetlinewidth{0.803000pt}%
\definecolor{currentstroke}{rgb}{0.752941,0.752941,0.752941}%
\pgfsetstrokecolor{currentstroke}%
\pgfsetdash{{2.960000pt}{1.280000pt}}{0.000000pt}%
\pgfpathmoveto{\pgfqpoint{0.726250in}{1.356656in}}%
\pgfpathlineto{\pgfqpoint{1.846250in}{1.356656in}}%
\pgfusepath{stroke}%
\end{pgfscope}%
\begin{pgfscope}%
\pgfsetbuttcap%
\pgfsetroundjoin%
\definecolor{currentfill}{rgb}{0.000000,0.000000,0.000000}%
\pgfsetfillcolor{currentfill}%
\pgfsetlinewidth{0.602250pt}%
\definecolor{currentstroke}{rgb}{0.000000,0.000000,0.000000}%
\pgfsetstrokecolor{currentstroke}%
\pgfsetdash{}{0pt}%
\pgfsys@defobject{currentmarker}{\pgfqpoint{-0.027778in}{0.000000in}}{\pgfqpoint{-0.000000in}{0.000000in}}{%
\pgfpathmoveto{\pgfqpoint{-0.000000in}{0.000000in}}%
\pgfpathlineto{\pgfqpoint{-0.027778in}{0.000000in}}%
\pgfusepath{stroke,fill}%
}%
\begin{pgfscope}%
\pgfsys@transformshift{0.726250in}{1.356656in}%
\pgfsys@useobject{currentmarker}{}%
\end{pgfscope}%
\end{pgfscope}%
\begin{pgfscope}%
\pgfpathrectangle{\pgfqpoint{0.726250in}{0.525000in}}{\pgfqpoint{1.120000in}{1.637500in}}%
\pgfusepath{clip}%
\pgfsetbuttcap%
\pgfsetroundjoin%
\pgfsetlinewidth{0.803000pt}%
\definecolor{currentstroke}{rgb}{0.752941,0.752941,0.752941}%
\pgfsetstrokecolor{currentstroke}%
\pgfsetdash{{2.960000pt}{1.280000pt}}{0.000000pt}%
\pgfpathmoveto{\pgfqpoint{0.726250in}{1.445900in}}%
\pgfpathlineto{\pgfqpoint{1.846250in}{1.445900in}}%
\pgfusepath{stroke}%
\end{pgfscope}%
\begin{pgfscope}%
\pgfsetbuttcap%
\pgfsetroundjoin%
\definecolor{currentfill}{rgb}{0.000000,0.000000,0.000000}%
\pgfsetfillcolor{currentfill}%
\pgfsetlinewidth{0.602250pt}%
\definecolor{currentstroke}{rgb}{0.000000,0.000000,0.000000}%
\pgfsetstrokecolor{currentstroke}%
\pgfsetdash{}{0pt}%
\pgfsys@defobject{currentmarker}{\pgfqpoint{-0.027778in}{0.000000in}}{\pgfqpoint{-0.000000in}{0.000000in}}{%
\pgfpathmoveto{\pgfqpoint{-0.000000in}{0.000000in}}%
\pgfpathlineto{\pgfqpoint{-0.027778in}{0.000000in}}%
\pgfusepath{stroke,fill}%
}%
\begin{pgfscope}%
\pgfsys@transformshift{0.726250in}{1.445900in}%
\pgfsys@useobject{currentmarker}{}%
\end{pgfscope}%
\end{pgfscope}%
\begin{pgfscope}%
\pgfpathrectangle{\pgfqpoint{0.726250in}{0.525000in}}{\pgfqpoint{1.120000in}{1.637500in}}%
\pgfusepath{clip}%
\pgfsetbuttcap%
\pgfsetroundjoin%
\pgfsetlinewidth{0.803000pt}%
\definecolor{currentstroke}{rgb}{0.752941,0.752941,0.752941}%
\pgfsetstrokecolor{currentstroke}%
\pgfsetdash{{2.960000pt}{1.280000pt}}{0.000000pt}%
\pgfpathmoveto{\pgfqpoint{0.726250in}{1.518818in}}%
\pgfpathlineto{\pgfqpoint{1.846250in}{1.518818in}}%
\pgfusepath{stroke}%
\end{pgfscope}%
\begin{pgfscope}%
\pgfsetbuttcap%
\pgfsetroundjoin%
\definecolor{currentfill}{rgb}{0.000000,0.000000,0.000000}%
\pgfsetfillcolor{currentfill}%
\pgfsetlinewidth{0.602250pt}%
\definecolor{currentstroke}{rgb}{0.000000,0.000000,0.000000}%
\pgfsetstrokecolor{currentstroke}%
\pgfsetdash{}{0pt}%
\pgfsys@defobject{currentmarker}{\pgfqpoint{-0.027778in}{0.000000in}}{\pgfqpoint{-0.000000in}{0.000000in}}{%
\pgfpathmoveto{\pgfqpoint{-0.000000in}{0.000000in}}%
\pgfpathlineto{\pgfqpoint{-0.027778in}{0.000000in}}%
\pgfusepath{stroke,fill}%
}%
\begin{pgfscope}%
\pgfsys@transformshift{0.726250in}{1.518818in}%
\pgfsys@useobject{currentmarker}{}%
\end{pgfscope}%
\end{pgfscope}%
\begin{pgfscope}%
\pgfpathrectangle{\pgfqpoint{0.726250in}{0.525000in}}{\pgfqpoint{1.120000in}{1.637500in}}%
\pgfusepath{clip}%
\pgfsetbuttcap%
\pgfsetroundjoin%
\pgfsetlinewidth{0.803000pt}%
\definecolor{currentstroke}{rgb}{0.752941,0.752941,0.752941}%
\pgfsetstrokecolor{currentstroke}%
\pgfsetdash{{2.960000pt}{1.280000pt}}{0.000000pt}%
\pgfpathmoveto{\pgfqpoint{0.726250in}{1.580470in}}%
\pgfpathlineto{\pgfqpoint{1.846250in}{1.580470in}}%
\pgfusepath{stroke}%
\end{pgfscope}%
\begin{pgfscope}%
\pgfsetbuttcap%
\pgfsetroundjoin%
\definecolor{currentfill}{rgb}{0.000000,0.000000,0.000000}%
\pgfsetfillcolor{currentfill}%
\pgfsetlinewidth{0.602250pt}%
\definecolor{currentstroke}{rgb}{0.000000,0.000000,0.000000}%
\pgfsetstrokecolor{currentstroke}%
\pgfsetdash{}{0pt}%
\pgfsys@defobject{currentmarker}{\pgfqpoint{-0.027778in}{0.000000in}}{\pgfqpoint{-0.000000in}{0.000000in}}{%
\pgfpathmoveto{\pgfqpoint{-0.000000in}{0.000000in}}%
\pgfpathlineto{\pgfqpoint{-0.027778in}{0.000000in}}%
\pgfusepath{stroke,fill}%
}%
\begin{pgfscope}%
\pgfsys@transformshift{0.726250in}{1.580470in}%
\pgfsys@useobject{currentmarker}{}%
\end{pgfscope}%
\end{pgfscope}%
\begin{pgfscope}%
\pgfpathrectangle{\pgfqpoint{0.726250in}{0.525000in}}{\pgfqpoint{1.120000in}{1.637500in}}%
\pgfusepath{clip}%
\pgfsetbuttcap%
\pgfsetroundjoin%
\pgfsetlinewidth{0.803000pt}%
\definecolor{currentstroke}{rgb}{0.752941,0.752941,0.752941}%
\pgfsetstrokecolor{currentstroke}%
\pgfsetdash{{2.960000pt}{1.280000pt}}{0.000000pt}%
\pgfpathmoveto{\pgfqpoint{0.726250in}{1.633874in}}%
\pgfpathlineto{\pgfqpoint{1.846250in}{1.633874in}}%
\pgfusepath{stroke}%
\end{pgfscope}%
\begin{pgfscope}%
\pgfsetbuttcap%
\pgfsetroundjoin%
\definecolor{currentfill}{rgb}{0.000000,0.000000,0.000000}%
\pgfsetfillcolor{currentfill}%
\pgfsetlinewidth{0.602250pt}%
\definecolor{currentstroke}{rgb}{0.000000,0.000000,0.000000}%
\pgfsetstrokecolor{currentstroke}%
\pgfsetdash{}{0pt}%
\pgfsys@defobject{currentmarker}{\pgfqpoint{-0.027778in}{0.000000in}}{\pgfqpoint{-0.000000in}{0.000000in}}{%
\pgfpathmoveto{\pgfqpoint{-0.000000in}{0.000000in}}%
\pgfpathlineto{\pgfqpoint{-0.027778in}{0.000000in}}%
\pgfusepath{stroke,fill}%
}%
\begin{pgfscope}%
\pgfsys@transformshift{0.726250in}{1.633874in}%
\pgfsys@useobject{currentmarker}{}%
\end{pgfscope}%
\end{pgfscope}%
\begin{pgfscope}%
\pgfpathrectangle{\pgfqpoint{0.726250in}{0.525000in}}{\pgfqpoint{1.120000in}{1.637500in}}%
\pgfusepath{clip}%
\pgfsetbuttcap%
\pgfsetroundjoin%
\pgfsetlinewidth{0.803000pt}%
\definecolor{currentstroke}{rgb}{0.752941,0.752941,0.752941}%
\pgfsetstrokecolor{currentstroke}%
\pgfsetdash{{2.960000pt}{1.280000pt}}{0.000000pt}%
\pgfpathmoveto{\pgfqpoint{0.726250in}{1.680981in}}%
\pgfpathlineto{\pgfqpoint{1.846250in}{1.680981in}}%
\pgfusepath{stroke}%
\end{pgfscope}%
\begin{pgfscope}%
\pgfsetbuttcap%
\pgfsetroundjoin%
\definecolor{currentfill}{rgb}{0.000000,0.000000,0.000000}%
\pgfsetfillcolor{currentfill}%
\pgfsetlinewidth{0.602250pt}%
\definecolor{currentstroke}{rgb}{0.000000,0.000000,0.000000}%
\pgfsetstrokecolor{currentstroke}%
\pgfsetdash{}{0pt}%
\pgfsys@defobject{currentmarker}{\pgfqpoint{-0.027778in}{0.000000in}}{\pgfqpoint{-0.000000in}{0.000000in}}{%
\pgfpathmoveto{\pgfqpoint{-0.000000in}{0.000000in}}%
\pgfpathlineto{\pgfqpoint{-0.027778in}{0.000000in}}%
\pgfusepath{stroke,fill}%
}%
\begin{pgfscope}%
\pgfsys@transformshift{0.726250in}{1.680981in}%
\pgfsys@useobject{currentmarker}{}%
\end{pgfscope}%
\end{pgfscope}%
\begin{pgfscope}%
\pgfpathrectangle{\pgfqpoint{0.726250in}{0.525000in}}{\pgfqpoint{1.120000in}{1.637500in}}%
\pgfusepath{clip}%
\pgfsetbuttcap%
\pgfsetroundjoin%
\pgfsetlinewidth{0.803000pt}%
\definecolor{currentstroke}{rgb}{0.752941,0.752941,0.752941}%
\pgfsetstrokecolor{currentstroke}%
\pgfsetdash{{2.960000pt}{1.280000pt}}{0.000000pt}%
\pgfpathmoveto{\pgfqpoint{0.726250in}{2.000338in}}%
\pgfpathlineto{\pgfqpoint{1.846250in}{2.000338in}}%
\pgfusepath{stroke}%
\end{pgfscope}%
\begin{pgfscope}%
\pgfsetbuttcap%
\pgfsetroundjoin%
\definecolor{currentfill}{rgb}{0.000000,0.000000,0.000000}%
\pgfsetfillcolor{currentfill}%
\pgfsetlinewidth{0.602250pt}%
\definecolor{currentstroke}{rgb}{0.000000,0.000000,0.000000}%
\pgfsetstrokecolor{currentstroke}%
\pgfsetdash{}{0pt}%
\pgfsys@defobject{currentmarker}{\pgfqpoint{-0.027778in}{0.000000in}}{\pgfqpoint{-0.000000in}{0.000000in}}{%
\pgfpathmoveto{\pgfqpoint{-0.000000in}{0.000000in}}%
\pgfpathlineto{\pgfqpoint{-0.027778in}{0.000000in}}%
\pgfusepath{stroke,fill}%
}%
\begin{pgfscope}%
\pgfsys@transformshift{0.726250in}{2.000338in}%
\pgfsys@useobject{currentmarker}{}%
\end{pgfscope}%
\end{pgfscope}%
\begin{pgfscope}%
\pgfpathrectangle{\pgfqpoint{0.726250in}{0.525000in}}{\pgfqpoint{1.120000in}{1.637500in}}%
\pgfusepath{clip}%
\pgfsetbuttcap%
\pgfsetroundjoin%
\pgfsetlinewidth{0.803000pt}%
\definecolor{currentstroke}{rgb}{0.752941,0.752941,0.752941}%
\pgfsetstrokecolor{currentstroke}%
\pgfsetdash{{2.960000pt}{1.280000pt}}{0.000000pt}%
\pgfpathmoveto{\pgfqpoint{0.726250in}{2.162500in}}%
\pgfpathlineto{\pgfqpoint{1.846250in}{2.162500in}}%
\pgfusepath{stroke}%
\end{pgfscope}%
\begin{pgfscope}%
\pgfsetbuttcap%
\pgfsetroundjoin%
\definecolor{currentfill}{rgb}{0.000000,0.000000,0.000000}%
\pgfsetfillcolor{currentfill}%
\pgfsetlinewidth{0.602250pt}%
\definecolor{currentstroke}{rgb}{0.000000,0.000000,0.000000}%
\pgfsetstrokecolor{currentstroke}%
\pgfsetdash{}{0pt}%
\pgfsys@defobject{currentmarker}{\pgfqpoint{-0.027778in}{0.000000in}}{\pgfqpoint{-0.000000in}{0.000000in}}{%
\pgfpathmoveto{\pgfqpoint{-0.000000in}{0.000000in}}%
\pgfpathlineto{\pgfqpoint{-0.027778in}{0.000000in}}%
\pgfusepath{stroke,fill}%
}%
\begin{pgfscope}%
\pgfsys@transformshift{0.726250in}{2.162500in}%
\pgfsys@useobject{currentmarker}{}%
\end{pgfscope}%
\end{pgfscope}%
\begin{pgfscope}%
\definecolor{textcolor}{rgb}{0.000000,0.000000,0.000000}%
\pgfsetstrokecolor{textcolor}%
\pgfsetfillcolor{textcolor}%
\pgftext[x=0.306885in,y=1.343750in,,bottom,rotate=90.000000]{\color{textcolor}\rmfamily\fontsize{9.000000}{10.800000}\selectfont \(\displaystyle m_i\) [eV]}%
\end{pgfscope}%
\begin{pgfscope}%
\pgfpathrectangle{\pgfqpoint{0.726250in}{0.525000in}}{\pgfqpoint{1.120000in}{1.637500in}}%
\pgfusepath{clip}%
\pgfsetrectcap%
\pgfsetroundjoin%
\pgfsetlinewidth{1.003750pt}%
\definecolor{currentstroke}{rgb}{0.000000,0.000000,0.000000}%
\pgfsetstrokecolor{currentstroke}%
\pgfsetdash{}{0pt}%
\pgfpathmoveto{\pgfqpoint{1.037371in}{0.515000in}}%
\pgfpathlineto{\pgfqpoint{1.043018in}{0.526501in}}%
\pgfpathlineto{\pgfqpoint{1.054331in}{0.549543in}}%
\pgfpathlineto{\pgfqpoint{1.065644in}{0.572585in}}%
\pgfpathlineto{\pgfqpoint{1.076957in}{0.595628in}}%
\pgfpathlineto{\pgfqpoint{1.088270in}{0.618670in}}%
\pgfpathlineto{\pgfqpoint{1.099583in}{0.641712in}}%
\pgfpathlineto{\pgfqpoint{1.110896in}{0.664754in}}%
\pgfpathlineto{\pgfqpoint{1.122210in}{0.687797in}}%
\pgfpathlineto{\pgfqpoint{1.133523in}{0.710839in}}%
\pgfpathlineto{\pgfqpoint{1.144836in}{0.733881in}}%
\pgfpathlineto{\pgfqpoint{1.156149in}{0.756923in}}%
\pgfpathlineto{\pgfqpoint{1.167462in}{0.779966in}}%
\pgfpathlineto{\pgfqpoint{1.178775in}{0.803008in}}%
\pgfpathlineto{\pgfqpoint{1.190088in}{0.826050in}}%
\pgfpathlineto{\pgfqpoint{1.201402in}{0.849092in}}%
\pgfpathlineto{\pgfqpoint{1.212715in}{0.872135in}}%
\pgfpathlineto{\pgfqpoint{1.224028in}{0.895177in}}%
\pgfpathlineto{\pgfqpoint{1.235341in}{0.918219in}}%
\pgfpathlineto{\pgfqpoint{1.246654in}{0.941261in}}%
\pgfpathlineto{\pgfqpoint{1.257967in}{0.964304in}}%
\pgfpathlineto{\pgfqpoint{1.269280in}{0.987346in}}%
\pgfpathlineto{\pgfqpoint{1.280593in}{1.010388in}}%
\pgfpathlineto{\pgfqpoint{1.291907in}{1.033430in}}%
\pgfpathlineto{\pgfqpoint{1.303220in}{1.056473in}}%
\pgfpathlineto{\pgfqpoint{1.314533in}{1.079515in}}%
\pgfpathlineto{\pgfqpoint{1.325846in}{1.102557in}}%
\pgfpathlineto{\pgfqpoint{1.337159in}{1.125599in}}%
\pgfpathlineto{\pgfqpoint{1.348472in}{1.148641in}}%
\pgfpathlineto{\pgfqpoint{1.359785in}{1.171684in}}%
\pgfpathlineto{\pgfqpoint{1.371098in}{1.194726in}}%
\pgfpathlineto{\pgfqpoint{1.382412in}{1.217768in}}%
\pgfpathlineto{\pgfqpoint{1.393725in}{1.240810in}}%
\pgfpathlineto{\pgfqpoint{1.405038in}{1.263853in}}%
\pgfpathlineto{\pgfqpoint{1.416351in}{1.286895in}}%
\pgfpathlineto{\pgfqpoint{1.427664in}{1.309937in}}%
\pgfpathlineto{\pgfqpoint{1.438977in}{1.332979in}}%
\pgfpathlineto{\pgfqpoint{1.450290in}{1.356022in}}%
\pgfpathlineto{\pgfqpoint{1.461604in}{1.379064in}}%
\pgfpathlineto{\pgfqpoint{1.472917in}{1.402106in}}%
\pgfpathlineto{\pgfqpoint{1.484230in}{1.425148in}}%
\pgfpathlineto{\pgfqpoint{1.495543in}{1.448191in}}%
\pgfpathlineto{\pgfqpoint{1.506856in}{1.471233in}}%
\pgfpathlineto{\pgfqpoint{1.518169in}{1.494275in}}%
\pgfpathlineto{\pgfqpoint{1.529482in}{1.517317in}}%
\pgfpathlineto{\pgfqpoint{1.540795in}{1.540360in}}%
\pgfpathlineto{\pgfqpoint{1.552109in}{1.563402in}}%
\pgfpathlineto{\pgfqpoint{1.563422in}{1.586444in}}%
\pgfpathlineto{\pgfqpoint{1.574735in}{1.609486in}}%
\pgfpathlineto{\pgfqpoint{1.586048in}{1.632528in}}%
\pgfpathlineto{\pgfqpoint{1.597361in}{1.655571in}}%
\pgfpathlineto{\pgfqpoint{1.608674in}{1.678613in}}%
\pgfpathlineto{\pgfqpoint{1.619987in}{1.701655in}}%
\pgfpathlineto{\pgfqpoint{1.631301in}{1.724697in}}%
\pgfpathlineto{\pgfqpoint{1.642614in}{1.747740in}}%
\pgfpathlineto{\pgfqpoint{1.653927in}{1.770782in}}%
\pgfpathlineto{\pgfqpoint{1.665240in}{1.793824in}}%
\pgfpathlineto{\pgfqpoint{1.676553in}{1.816866in}}%
\pgfpathlineto{\pgfqpoint{1.687866in}{1.839909in}}%
\pgfpathlineto{\pgfqpoint{1.699179in}{1.862951in}}%
\pgfpathlineto{\pgfqpoint{1.710492in}{1.885993in}}%
\pgfpathlineto{\pgfqpoint{1.721806in}{1.909035in}}%
\pgfpathlineto{\pgfqpoint{1.733119in}{1.932078in}}%
\pgfpathlineto{\pgfqpoint{1.744432in}{1.955120in}}%
\pgfpathlineto{\pgfqpoint{1.755745in}{1.978162in}}%
\pgfpathlineto{\pgfqpoint{1.767058in}{2.001204in}}%
\pgfpathlineto{\pgfqpoint{1.778371in}{2.024247in}}%
\pgfpathlineto{\pgfqpoint{1.789684in}{2.047289in}}%
\pgfpathlineto{\pgfqpoint{1.800997in}{2.070331in}}%
\pgfpathlineto{\pgfqpoint{1.812311in}{2.093373in}}%
\pgfpathlineto{\pgfqpoint{1.823624in}{2.116416in}}%
\pgfpathlineto{\pgfqpoint{1.834937in}{2.139458in}}%
\pgfpathlineto{\pgfqpoint{1.846250in}{2.162500in}}%
\pgfusepath{stroke}%
\end{pgfscope}%
\begin{pgfscope}%
\pgfpathrectangle{\pgfqpoint{0.726250in}{0.525000in}}{\pgfqpoint{1.120000in}{1.637500in}}%
\pgfusepath{clip}%
\pgfsetrectcap%
\pgfsetroundjoin%
\pgfsetlinewidth{1.003750pt}%
\definecolor{currentstroke}{rgb}{0.000000,0.000000,0.000000}%
\pgfsetstrokecolor{currentstroke}%
\pgfsetdash{}{0pt}%
\pgfpathmoveto{\pgfqpoint{0.726250in}{0.744957in}}%
\pgfpathlineto{\pgfqpoint{0.737563in}{0.745282in}}%
\pgfpathlineto{\pgfqpoint{0.748876in}{0.745646in}}%
\pgfpathlineto{\pgfqpoint{0.760189in}{0.746054in}}%
\pgfpathlineto{\pgfqpoint{0.771503in}{0.746510in}}%
\pgfpathlineto{\pgfqpoint{0.782816in}{0.747021in}}%
\pgfpathlineto{\pgfqpoint{0.794129in}{0.747593in}}%
\pgfpathlineto{\pgfqpoint{0.805442in}{0.748233in}}%
\pgfpathlineto{\pgfqpoint{0.816755in}{0.748948in}}%
\pgfpathlineto{\pgfqpoint{0.828068in}{0.749748in}}%
\pgfpathlineto{\pgfqpoint{0.839381in}{0.750641in}}%
\pgfpathlineto{\pgfqpoint{0.850694in}{0.751639in}}%
\pgfpathlineto{\pgfqpoint{0.862008in}{0.752753in}}%
\pgfpathlineto{\pgfqpoint{0.873321in}{0.753996in}}%
\pgfpathlineto{\pgfqpoint{0.884634in}{0.755381in}}%
\pgfpathlineto{\pgfqpoint{0.895947in}{0.756925in}}%
\pgfpathlineto{\pgfqpoint{0.907260in}{0.758642in}}%
\pgfpathlineto{\pgfqpoint{0.918573in}{0.760552in}}%
\pgfpathlineto{\pgfqpoint{0.929886in}{0.762674in}}%
\pgfpathlineto{\pgfqpoint{0.941199in}{0.765029in}}%
\pgfpathlineto{\pgfqpoint{0.952513in}{0.767638in}}%
\pgfpathlineto{\pgfqpoint{0.963826in}{0.770526in}}%
\pgfpathlineto{\pgfqpoint{0.975139in}{0.773718in}}%
\pgfpathlineto{\pgfqpoint{0.986452in}{0.777240in}}%
\pgfpathlineto{\pgfqpoint{0.997765in}{0.781120in}}%
\pgfpathlineto{\pgfqpoint{1.009078in}{0.785385in}}%
\pgfpathlineto{\pgfqpoint{1.020391in}{0.790066in}}%
\pgfpathlineto{\pgfqpoint{1.031705in}{0.795191in}}%
\pgfpathlineto{\pgfqpoint{1.043018in}{0.800790in}}%
\pgfpathlineto{\pgfqpoint{1.054331in}{0.806892in}}%
\pgfpathlineto{\pgfqpoint{1.065644in}{0.813524in}}%
\pgfpathlineto{\pgfqpoint{1.076957in}{0.820714in}}%
\pgfpathlineto{\pgfqpoint{1.088270in}{0.828485in}}%
\pgfpathlineto{\pgfqpoint{1.099583in}{0.836860in}}%
\pgfpathlineto{\pgfqpoint{1.110896in}{0.845858in}}%
\pgfpathlineto{\pgfqpoint{1.122210in}{0.855495in}}%
\pgfpathlineto{\pgfqpoint{1.133523in}{0.865784in}}%
\pgfpathlineto{\pgfqpoint{1.144836in}{0.876732in}}%
\pgfpathlineto{\pgfqpoint{1.156149in}{0.888343in}}%
\pgfpathlineto{\pgfqpoint{1.167462in}{0.900617in}}%
\pgfpathlineto{\pgfqpoint{1.178775in}{0.913548in}}%
\pgfpathlineto{\pgfqpoint{1.190088in}{0.927128in}}%
\pgfpathlineto{\pgfqpoint{1.201402in}{0.941343in}}%
\pgfpathlineto{\pgfqpoint{1.212715in}{0.956176in}}%
\pgfpathlineto{\pgfqpoint{1.224028in}{0.971607in}}%
\pgfpathlineto{\pgfqpoint{1.235341in}{0.987614in}}%
\pgfpathlineto{\pgfqpoint{1.246654in}{1.004170in}}%
\pgfpathlineto{\pgfqpoint{1.257967in}{1.021250in}}%
\pgfpathlineto{\pgfqpoint{1.269280in}{1.038825in}}%
\pgfpathlineto{\pgfqpoint{1.280593in}{1.056865in}}%
\pgfpathlineto{\pgfqpoint{1.291907in}{1.075342in}}%
\pgfpathlineto{\pgfqpoint{1.303220in}{1.094226in}}%
\pgfpathlineto{\pgfqpoint{1.314533in}{1.113488in}}%
\pgfpathlineto{\pgfqpoint{1.325846in}{1.133101in}}%
\pgfpathlineto{\pgfqpoint{1.337159in}{1.153036in}}%
\pgfpathlineto{\pgfqpoint{1.348472in}{1.173268in}}%
\pgfpathlineto{\pgfqpoint{1.359785in}{1.193773in}}%
\pgfpathlineto{\pgfqpoint{1.371098in}{1.214526in}}%
\pgfpathlineto{\pgfqpoint{1.382412in}{1.235506in}}%
\pgfpathlineto{\pgfqpoint{1.393725in}{1.256692in}}%
\pgfpathlineto{\pgfqpoint{1.405038in}{1.278066in}}%
\pgfpathlineto{\pgfqpoint{1.416351in}{1.299609in}}%
\pgfpathlineto{\pgfqpoint{1.427664in}{1.321306in}}%
\pgfpathlineto{\pgfqpoint{1.438977in}{1.343142in}}%
\pgfpathlineto{\pgfqpoint{1.450290in}{1.365103in}}%
\pgfpathlineto{\pgfqpoint{1.461604in}{1.387176in}}%
\pgfpathlineto{\pgfqpoint{1.472917in}{1.409352in}}%
\pgfpathlineto{\pgfqpoint{1.484230in}{1.431618in}}%
\pgfpathlineto{\pgfqpoint{1.495543in}{1.453966in}}%
\pgfpathlineto{\pgfqpoint{1.506856in}{1.476388in}}%
\pgfpathlineto{\pgfqpoint{1.518169in}{1.498875in}}%
\pgfpathlineto{\pgfqpoint{1.529482in}{1.521422in}}%
\pgfpathlineto{\pgfqpoint{1.540795in}{1.544022in}}%
\pgfpathlineto{\pgfqpoint{1.552109in}{1.566668in}}%
\pgfpathlineto{\pgfqpoint{1.563422in}{1.589358in}}%
\pgfpathlineto{\pgfqpoint{1.574735in}{1.612085in}}%
\pgfpathlineto{\pgfqpoint{1.586048in}{1.634846in}}%
\pgfpathlineto{\pgfqpoint{1.597361in}{1.657637in}}%
\pgfpathlineto{\pgfqpoint{1.608674in}{1.680456in}}%
\pgfpathlineto{\pgfqpoint{1.619987in}{1.703298in}}%
\pgfpathlineto{\pgfqpoint{1.631301in}{1.726162in}}%
\pgfpathlineto{\pgfqpoint{1.642614in}{1.749046in}}%
\pgfpathlineto{\pgfqpoint{1.653927in}{1.771946in}}%
\pgfpathlineto{\pgfqpoint{1.665240in}{1.794862in}}%
\pgfpathlineto{\pgfqpoint{1.676553in}{1.817791in}}%
\pgfpathlineto{\pgfqpoint{1.687866in}{1.840733in}}%
\pgfpathlineto{\pgfqpoint{1.699179in}{1.863686in}}%
\pgfpathlineto{\pgfqpoint{1.710492in}{1.886648in}}%
\pgfpathlineto{\pgfqpoint{1.721806in}{1.909619in}}%
\pgfpathlineto{\pgfqpoint{1.733119in}{1.932598in}}%
\pgfpathlineto{\pgfqpoint{1.744432in}{1.955584in}}%
\pgfpathlineto{\pgfqpoint{1.755745in}{1.978576in}}%
\pgfpathlineto{\pgfqpoint{1.767058in}{2.001573in}}%
\pgfpathlineto{\pgfqpoint{1.778371in}{2.024575in}}%
\pgfpathlineto{\pgfqpoint{1.789684in}{2.047582in}}%
\pgfpathlineto{\pgfqpoint{1.800997in}{2.070592in}}%
\pgfpathlineto{\pgfqpoint{1.812311in}{2.093606in}}%
\pgfpathlineto{\pgfqpoint{1.823624in}{2.116623in}}%
\pgfpathlineto{\pgfqpoint{1.834937in}{2.139642in}}%
\pgfpathlineto{\pgfqpoint{1.846250in}{2.162665in}}%
\pgfusepath{stroke}%
\end{pgfscope}%
\begin{pgfscope}%
\pgfpathrectangle{\pgfqpoint{0.726250in}{0.525000in}}{\pgfqpoint{1.120000in}{1.637500in}}%
\pgfusepath{clip}%
\pgfsetrectcap%
\pgfsetroundjoin%
\pgfsetlinewidth{1.003750pt}%
\definecolor{currentstroke}{rgb}{0.000000,0.000000,0.000000}%
\pgfsetstrokecolor{currentstroke}%
\pgfsetdash{}{0pt}%
\pgfpathmoveto{\pgfqpoint{0.726250in}{1.452362in}}%
\pgfpathlineto{\pgfqpoint{0.737563in}{1.452371in}}%
\pgfpathlineto{\pgfqpoint{0.748876in}{1.452382in}}%
\pgfpathlineto{\pgfqpoint{0.760189in}{1.452394in}}%
\pgfpathlineto{\pgfqpoint{0.771503in}{1.452407in}}%
\pgfpathlineto{\pgfqpoint{0.782816in}{1.452422in}}%
\pgfpathlineto{\pgfqpoint{0.794129in}{1.452439in}}%
\pgfpathlineto{\pgfqpoint{0.805442in}{1.452458in}}%
\pgfpathlineto{\pgfqpoint{0.816755in}{1.452479in}}%
\pgfpathlineto{\pgfqpoint{0.828068in}{1.452503in}}%
\pgfpathlineto{\pgfqpoint{0.839381in}{1.452529in}}%
\pgfpathlineto{\pgfqpoint{0.850694in}{1.452559in}}%
\pgfpathlineto{\pgfqpoint{0.862008in}{1.452593in}}%
\pgfpathlineto{\pgfqpoint{0.873321in}{1.452631in}}%
\pgfpathlineto{\pgfqpoint{0.884634in}{1.452673in}}%
\pgfpathlineto{\pgfqpoint{0.895947in}{1.452720in}}%
\pgfpathlineto{\pgfqpoint{0.907260in}{1.452773in}}%
\pgfpathlineto{\pgfqpoint{0.918573in}{1.452833in}}%
\pgfpathlineto{\pgfqpoint{0.929886in}{1.452900in}}%
\pgfpathlineto{\pgfqpoint{0.941199in}{1.452975in}}%
\pgfpathlineto{\pgfqpoint{0.952513in}{1.453059in}}%
\pgfpathlineto{\pgfqpoint{0.963826in}{1.453154in}}%
\pgfpathlineto{\pgfqpoint{0.975139in}{1.453259in}}%
\pgfpathlineto{\pgfqpoint{0.986452in}{1.453378in}}%
\pgfpathlineto{\pgfqpoint{0.997765in}{1.453511in}}%
\pgfpathlineto{\pgfqpoint{1.009078in}{1.453661in}}%
\pgfpathlineto{\pgfqpoint{1.020391in}{1.453828in}}%
\pgfpathlineto{\pgfqpoint{1.031705in}{1.454016in}}%
\pgfpathlineto{\pgfqpoint{1.043018in}{1.454226in}}%
\pgfpathlineto{\pgfqpoint{1.054331in}{1.454462in}}%
\pgfpathlineto{\pgfqpoint{1.065644in}{1.454727in}}%
\pgfpathlineto{\pgfqpoint{1.076957in}{1.455023in}}%
\pgfpathlineto{\pgfqpoint{1.088270in}{1.455355in}}%
\pgfpathlineto{\pgfqpoint{1.099583in}{1.455727in}}%
\pgfpathlineto{\pgfqpoint{1.110896in}{1.456143in}}%
\pgfpathlineto{\pgfqpoint{1.122210in}{1.456609in}}%
\pgfpathlineto{\pgfqpoint{1.133523in}{1.457131in}}%
\pgfpathlineto{\pgfqpoint{1.144836in}{1.457715in}}%
\pgfpathlineto{\pgfqpoint{1.156149in}{1.458368in}}%
\pgfpathlineto{\pgfqpoint{1.167462in}{1.459099in}}%
\pgfpathlineto{\pgfqpoint{1.178775in}{1.459915in}}%
\pgfpathlineto{\pgfqpoint{1.190088in}{1.460827in}}%
\pgfpathlineto{\pgfqpoint{1.201402in}{1.461846in}}%
\pgfpathlineto{\pgfqpoint{1.212715in}{1.462983in}}%
\pgfpathlineto{\pgfqpoint{1.224028in}{1.464251in}}%
\pgfpathlineto{\pgfqpoint{1.235341in}{1.465665in}}%
\pgfpathlineto{\pgfqpoint{1.246654in}{1.467240in}}%
\pgfpathlineto{\pgfqpoint{1.257967in}{1.468992in}}%
\pgfpathlineto{\pgfqpoint{1.269280in}{1.470940in}}%
\pgfpathlineto{\pgfqpoint{1.280593in}{1.473103in}}%
\pgfpathlineto{\pgfqpoint{1.291907in}{1.475504in}}%
\pgfpathlineto{\pgfqpoint{1.303220in}{1.478163in}}%
\pgfpathlineto{\pgfqpoint{1.314533in}{1.481106in}}%
\pgfpathlineto{\pgfqpoint{1.325846in}{1.484357in}}%
\pgfpathlineto{\pgfqpoint{1.337159in}{1.487944in}}%
\pgfpathlineto{\pgfqpoint{1.348472in}{1.491894in}}%
\pgfpathlineto{\pgfqpoint{1.359785in}{1.496235in}}%
\pgfpathlineto{\pgfqpoint{1.371098in}{1.500996in}}%
\pgfpathlineto{\pgfqpoint{1.382412in}{1.506208in}}%
\pgfpathlineto{\pgfqpoint{1.393725in}{1.511898in}}%
\pgfpathlineto{\pgfqpoint{1.405038in}{1.518097in}}%
\pgfpathlineto{\pgfqpoint{1.416351in}{1.524831in}}%
\pgfpathlineto{\pgfqpoint{1.427664in}{1.532127in}}%
\pgfpathlineto{\pgfqpoint{1.438977in}{1.540010in}}%
\pgfpathlineto{\pgfqpoint{1.450290in}{1.548500in}}%
\pgfpathlineto{\pgfqpoint{1.461604in}{1.557616in}}%
\pgfpathlineto{\pgfqpoint{1.472917in}{1.567374in}}%
\pgfpathlineto{\pgfqpoint{1.484230in}{1.577786in}}%
\pgfpathlineto{\pgfqpoint{1.495543in}{1.588857in}}%
\pgfpathlineto{\pgfqpoint{1.506856in}{1.600592in}}%
\pgfpathlineto{\pgfqpoint{1.518169in}{1.612989in}}%
\pgfpathlineto{\pgfqpoint{1.529482in}{1.626042in}}%
\pgfpathlineto{\pgfqpoint{1.540795in}{1.639742in}}%
\pgfpathlineto{\pgfqpoint{1.552109in}{1.654073in}}%
\pgfpathlineto{\pgfqpoint{1.563422in}{1.669020in}}%
\pgfpathlineto{\pgfqpoint{1.574735in}{1.684560in}}%
\pgfpathlineto{\pgfqpoint{1.586048in}{1.700671in}}%
\pgfpathlineto{\pgfqpoint{1.597361in}{1.717328in}}%
\pgfpathlineto{\pgfqpoint{1.608674in}{1.734502in}}%
\pgfpathlineto{\pgfqpoint{1.619987in}{1.752166in}}%
\pgfpathlineto{\pgfqpoint{1.631301in}{1.770290in}}%
\pgfpathlineto{\pgfqpoint{1.642614in}{1.788845in}}%
\pgfpathlineto{\pgfqpoint{1.653927in}{1.807802in}}%
\pgfpathlineto{\pgfqpoint{1.665240in}{1.827132in}}%
\pgfpathlineto{\pgfqpoint{1.676553in}{1.846807in}}%
\pgfpathlineto{\pgfqpoint{1.687866in}{1.866799in}}%
\pgfpathlineto{\pgfqpoint{1.699179in}{1.887084in}}%
\pgfpathlineto{\pgfqpoint{1.710492in}{1.907637in}}%
\pgfpathlineto{\pgfqpoint{1.721806in}{1.928434in}}%
\pgfpathlineto{\pgfqpoint{1.733119in}{1.949454in}}%
\pgfpathlineto{\pgfqpoint{1.744432in}{1.970676in}}%
\pgfpathlineto{\pgfqpoint{1.755745in}{1.992083in}}%
\pgfpathlineto{\pgfqpoint{1.767058in}{2.013656in}}%
\pgfpathlineto{\pgfqpoint{1.778371in}{2.035380in}}%
\pgfpathlineto{\pgfqpoint{1.789684in}{2.057240in}}%
\pgfpathlineto{\pgfqpoint{1.800997in}{2.079223in}}%
\pgfpathlineto{\pgfqpoint{1.812311in}{2.101317in}}%
\pgfpathlineto{\pgfqpoint{1.823624in}{2.123509in}}%
\pgfpathlineto{\pgfqpoint{1.834937in}{2.145792in}}%
\pgfpathlineto{\pgfqpoint{1.846250in}{2.168154in}}%
\pgfusepath{stroke}%
\end{pgfscope}%
\begin{pgfscope}%
\pgfpathrectangle{\pgfqpoint{0.726250in}{0.525000in}}{\pgfqpoint{1.120000in}{1.637500in}}%
\pgfusepath{clip}%
\pgfsetbuttcap%
\pgfsetroundjoin%
\pgfsetlinewidth{1.003750pt}%
\definecolor{currentstroke}{rgb}{0.000000,0.000000,0.000000}%
\pgfsetstrokecolor{currentstroke}%
\pgfsetdash{{3.700000pt}{1.600000pt}}{0.000000pt}%
\pgfpathmoveto{\pgfqpoint{1.495543in}{0.793842in}}%
\pgfpathlineto{\pgfqpoint{1.506856in}{1.081792in}}%
\pgfpathlineto{\pgfqpoint{1.518169in}{1.205486in}}%
\pgfpathlineto{\pgfqpoint{1.529482in}{1.288902in}}%
\pgfpathlineto{\pgfqpoint{1.540795in}{1.353704in}}%
\pgfpathlineto{\pgfqpoint{1.552109in}{1.407793in}}%
\pgfpathlineto{\pgfqpoint{1.563422in}{1.454947in}}%
\pgfpathlineto{\pgfqpoint{1.574735in}{1.497272in}}%
\pgfpathlineto{\pgfqpoint{1.586048in}{1.536060in}}%
\pgfpathlineto{\pgfqpoint{1.597361in}{1.572160in}}%
\pgfpathlineto{\pgfqpoint{1.608674in}{1.606161in}}%
\pgfpathlineto{\pgfqpoint{1.619987in}{1.638488in}}%
\pgfpathlineto{\pgfqpoint{1.631301in}{1.669455in}}%
\pgfpathlineto{\pgfqpoint{1.642614in}{1.699305in}}%
\pgfpathlineto{\pgfqpoint{1.653927in}{1.728225in}}%
\pgfpathlineto{\pgfqpoint{1.665240in}{1.756363in}}%
\pgfpathlineto{\pgfqpoint{1.676553in}{1.783840in}}%
\pgfpathlineto{\pgfqpoint{1.687866in}{1.810753in}}%
\pgfpathlineto{\pgfqpoint{1.699179in}{1.837183in}}%
\pgfpathlineto{\pgfqpoint{1.710492in}{1.863196in}}%
\pgfpathlineto{\pgfqpoint{1.721806in}{1.888849in}}%
\pgfpathlineto{\pgfqpoint{1.733119in}{1.914190in}}%
\pgfpathlineto{\pgfqpoint{1.744432in}{1.939258in}}%
\pgfpathlineto{\pgfqpoint{1.755745in}{1.964089in}}%
\pgfpathlineto{\pgfqpoint{1.767058in}{1.988712in}}%
\pgfpathlineto{\pgfqpoint{1.778371in}{2.013152in}}%
\pgfpathlineto{\pgfqpoint{1.789684in}{2.037432in}}%
\pgfpathlineto{\pgfqpoint{1.800997in}{2.061571in}}%
\pgfpathlineto{\pgfqpoint{1.812311in}{2.085586in}}%
\pgfpathlineto{\pgfqpoint{1.823624in}{2.109490in}}%
\pgfpathlineto{\pgfqpoint{1.834937in}{2.133298in}}%
\pgfpathlineto{\pgfqpoint{1.846250in}{2.157020in}}%
\pgfusepath{stroke}%
\end{pgfscope}%
\begin{pgfscope}%
\pgfsetrectcap%
\pgfsetmiterjoin%
\pgfsetlinewidth{1.003750pt}%
\definecolor{currentstroke}{rgb}{0.000000,0.000000,0.000000}%
\pgfsetstrokecolor{currentstroke}%
\pgfsetdash{}{0pt}%
\pgfpathmoveto{\pgfqpoint{0.726250in}{0.525000in}}%
\pgfpathlineto{\pgfqpoint{0.726250in}{2.162500in}}%
\pgfusepath{stroke}%
\end{pgfscope}%
\begin{pgfscope}%
\pgfsetrectcap%
\pgfsetmiterjoin%
\pgfsetlinewidth{1.003750pt}%
\definecolor{currentstroke}{rgb}{0.000000,0.000000,0.000000}%
\pgfsetstrokecolor{currentstroke}%
\pgfsetdash{}{0pt}%
\pgfpathmoveto{\pgfqpoint{1.846250in}{0.525000in}}%
\pgfpathlineto{\pgfqpoint{1.846250in}{2.162500in}}%
\pgfusepath{stroke}%
\end{pgfscope}%
\begin{pgfscope}%
\pgfsetrectcap%
\pgfsetmiterjoin%
\pgfsetlinewidth{1.003750pt}%
\definecolor{currentstroke}{rgb}{0.000000,0.000000,0.000000}%
\pgfsetstrokecolor{currentstroke}%
\pgfsetdash{}{0pt}%
\pgfpathmoveto{\pgfqpoint{0.726250in}{0.525000in}}%
\pgfpathlineto{\pgfqpoint{1.846250in}{0.525000in}}%
\pgfusepath{stroke}%
\end{pgfscope}%
\begin{pgfscope}%
\pgfsetrectcap%
\pgfsetmiterjoin%
\pgfsetlinewidth{1.003750pt}%
\definecolor{currentstroke}{rgb}{0.000000,0.000000,0.000000}%
\pgfsetstrokecolor{currentstroke}%
\pgfsetdash{}{0pt}%
\pgfpathmoveto{\pgfqpoint{0.726250in}{2.162500in}}%
\pgfpathlineto{\pgfqpoint{1.846250in}{2.162500in}}%
\pgfusepath{stroke}%
\end{pgfscope}%
\begin{pgfscope}%
\definecolor{textcolor}{rgb}{0.000000,0.000000,0.000000}%
\pgfsetstrokecolor{textcolor}%
\pgfsetfillcolor{textcolor}%
\pgftext[x=0.862357in,y=0.563119in,left,base]{\color{textcolor}\rmfamily\fontsize{9.000000}{10.800000}\selectfont \(\displaystyle m_1\)}%
\end{pgfscope}%
\begin{pgfscope}%
\definecolor{textcolor}{rgb}{0.000000,0.000000,0.000000}%
\pgfsetstrokecolor{textcolor}%
\pgfsetfillcolor{textcolor}%
\pgftext[x=0.862357in,y=0.875137in,left,base]{\color{textcolor}\rmfamily\fontsize{9.000000}{10.800000}\selectfont \(\displaystyle m_2\)}%
\end{pgfscope}%
\begin{pgfscope}%
\definecolor{textcolor}{rgb}{0.000000,0.000000,0.000000}%
\pgfsetstrokecolor{textcolor}%
\pgfsetfillcolor{textcolor}%
\pgftext[x=0.862357in,y=1.518818in,left,base]{\color{textcolor}\rmfamily\fontsize{9.000000}{10.800000}\selectfont \(\displaystyle m_3\) (NO)}%
\end{pgfscope}%
\begin{pgfscope}%
\definecolor{textcolor}{rgb}{0.000000,0.000000,0.000000}%
\pgfsetstrokecolor{textcolor}%
\pgfsetfillcolor{textcolor}%
\pgftext[x=1.394112in,y=0.659569in,left,base]{\color{textcolor}\rmfamily\fontsize{9.000000}{10.800000}\selectfont \(\displaystyle m_3\) (IO)}%
\end{pgfscope}%
\end{pgfpicture}%
\makeatother%
\endgroup%

  \caption{Illustration of the behavior of the three neutrino masses, in normal and
  inverse ordering, as a function of $m_1$ (which is not yet well constrained by the
  experimental data).}
  \label{fig:neutrino_mass_scale}
\end{marginfigure}

In addition, we don't know the absolute scale of the masses, although we have several,
independent evidences, from both $\beta$ decay and cosmological metrics, that none
of the $m_i$ cannot be much larger than 1~eV. The implication is that, as summarized in
figure~\ref{fig:neutrino_mass_scale}, if $m_1$ is zero (or close to zero), then
$m_2 \approx 10$~meV and $m_3 \approx 50$~meV, assuming normal ordering. On the
other hand, if $m_1$ is larger than $\sim 100$~meV, then the three masses are more
or less all equal, in both normal and inverted ordering.


\subsection{Three flavor mixing}

Three-flavor neutrino mixing is significantly more complicated than the two-flavor
version from a purely algebraic point of view, although the basic derivation proceeds
exactly in the same way. We shall not work out the full calculation, here, but if we
forget for a second the CP violating phase (which data indicate might be trivial,
after all) the oscillation (or survival) probability between two flavors $\nu_\alpha$
and $\nu_\beta$ reads
\begin{align}
  P(\nu_\alpha \rightarrow \nu_\beta) = \delta_{\alpha\beta} -
  4\sum_{i=1}^{3}\sum_{j < 1} U_{\alpha i} U_{\beta i}U_{\alpha j} U_{\beta j}
  \sin^2\qty(\frac{\Delta m^2_{ij} L}{4\hbar c E})
\end{align}


Nature was kind enought, though, to help un in two ways: $\theta_{13}$ is
\emph{small} and $\Delta m^2_{21} \ll \Delta m^2_{31}$. As we shall see in a second,
this allows for significant simplifications in relevant setups.

Let us start by examining the phases
\begin{align*}
  \phi_{ij} = \frac{\Delta m^2_{ij} L}{4 \hbar c E} =
  1.27\, \frac{\Delta m^2_{ij}~[\text{eV}^2]~L~[\text{m}]}{E~[\text{MeV}]}.
\end{align*}
For solar neutrinos $\nicefrac{L}{E}$ is very large compared to all $\Delta m^2$,
and the trigonometric terms averages out quickly---starting from the highest mass
differences. In this case we have, only keeping the phase corresponding to the
smallest mass difference
\begin{align*}
  P(\nu_e \rightarrow \nu_{\mu,~\tau}) =
  \cos^2(\theta_{13})\sin^2(2\theta_{12})\sin^2\qty(\frac{\Delta m^2_{21} L}{4\hbar c E}) +
  \frac{1}{2}\sin^2(2\theta_{13}),
\end{align*}
that, in the limit of small $\theta_{13}$ becomes
\begin{align*}
  P(\nu_e \rightarrow \nu_{\mu,~\tau}) \approx
  \sin^2(2\theta_{12})\sin^2\qty(\frac{\Delta m^2_{21} L}{4\hbar c E}).
\end{align*}
This is exactly the result we have derived early in the two-generation scheme---only
now we have explicitly named the $\Delta m^2$ and mixing angle involved.

For atmoshperic neutrinos the situation is different, as $\nicefrac{L}{E}$ is
small compared to $\Delta m^2_{12}$, but not with respect $\Delta m^2_{23}$ and
$\Delta m^2_{13}$, and the relevant oscillation probabilities read
\begin{align*}
  P(\nu_\mu \rightarrow \nu_\tau) & =
  \cos^2(\theta_{13})\sin^2(2\theta_{23})\sin^2\qty(\frac{\Delta m^2_{32} L}{4\hbar c E})
  \rightarrow \sin^2(2\theta_{23})\sin^2\qty(\frac{\Delta m^2_{32} L}{4\hbar c E})\\
  P(\nu_e \rightarrow \nu_\mu) & =
  \sin^2(\theta_{13})\sin^2(2\theta_{23})\sin^2\qty(\frac{\Delta m^2_{32} L}{4\hbar c E})
  \rightarrow 0\\
  P(\nu_e \rightarrow \nu_\tau) & =
  \sin^2(\theta_{13})\cos^2(2\theta_{23})\sin^2\qty(\frac{\Delta m^2_{32} L}{4\hbar c E})
  \rightarrow 0.
\end{align*}
Nowadays all the analyses are properly done with a three-flavor mixing framework,
but we have learned that, due to the particular combinations of mixing parameter
values, both solar and atmospheric neutrino oscillations can \emph{approximately}
be interpreted in a two-flavor mixing scenario, the former being sensitive to
$\theta_{12}$ and $\Delta m^2_{21}$, and the latter to $\theta_{32}$ and $\Delta m^2_{32}$.
This is the very reason for the particular decomposition of the PMNS matrix, as
well as the naming of the different terms.


\section{Cosmic neutrinos}


\subsection{Neutrinos from Supernov\ae}


\section{State of the art and outlook}

% https://physics.stackexchange.com/questions/341904/how-is-delta-m2-12-is-identified-with-the-solar-mixing-angle
% https://www.worldscientific.com/doi/pdf/10.1142/9789811269776_0225



\subsection{Current open problems}
