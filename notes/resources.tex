\chapter*{Resources}

Admittedly, the material covered in this course is various and diverse, and one
of the main goals of these lecture notes is to to provide a succinct, one-stop
reference covering (more or less) everything. That said, the world is literally
full of wonderful books, and the web offers a plethora of additional useful
(and freely available) resources.

Infinite lists are useless, and I will try and keep this short and to the point,
trying to emphasize why each single entry is relevant and where it could help.
It goes without saying, everybody has their own preferences, and you might be unable
to find here what you really need, but at least this is a starting point.


\section*{Books}

I have no doubt that, if you are motivated enough, you will be able to find pdf
versions of all the books I might come up with posted on the web---legally or not.
Unpopular as this might sound, I encourage you to consider buying the books you
like and use\sidenote{Full disclaimer: I got free instructor copies for some (but
not all) of the books listed, so take my suggestion for what is worth. One way or
the other, I get no money whatsoever\ldots}
By all means I violently in favour of full open access to scientific research, but
at the same time I fully recognize that writing a book takes time and effort, and
I think it is ultimately up to the author to decide how to distribute their work.

\begin{itemize}
    \item Sean Carrol, \emph{Spacetime and Geometry: An Introduction to General Relativity},
    Cambridge University Press (2019): this is a beautiful introduction to general
    relativity, and if you want to invest money on a single volume, this one is
    definitely a good choice. The chapter on cosmology is quite terse, but will find
    more than you need about the first part of chapter~\ref{chap:cosmo} of these
    lecture notes, in a tone that is rigorous and fun to read at the same time.
    (A trimmed-down version of the material, predating the book, is freely
    available on the web---scroll down for more details.)
    \item Barbara Ryden, \emph{Introduction to Cosmology}, Cambridge University
    Press (2016): a basic and enjoyable introduction to cosmology, focused on the
    underlying physical ideas and with a very limited amount of mathematics. The
    book basically uses no general relativity, and it is a nice complement to the
    previous suggestion.
\end{itemize}



\section*{Other resources}

This includes some of the lecture notes and review articles freely available on the
web that I found most useful while preparing these lecture notes. Note the titles
are clickable and should bring you straight to the corresponding resource.


\begin{itemize}
    \item Pasquale Blasi,
    \href{https://arxiv.org/abs/1311.7346}{The Origin of Galactic Cosmic Rays}:
    a very nice review article on cosmic rays, with a focus on the origin of the
    galactic component. Pasquale is truly one of the leading experts in the field,
    and this shows up.
    \item Sean Carrol,
    \href{https://arxiv.org/abs/gr-qc/9712019}{\emph{Lecture Notes on General Relativity}}:
    this covers roughly half of the material that went into the book
    \emph{Spacetime and Geometry: An Introduction to General Relativity}
    by the same author, and it might be a cheaper, good alternative to that. Most
    of the fun that I mentioned for the book was already present here, so buckle up
    for a bumpy ride.
    \item Daniel Baumann,
    \href{http://cosmology.amsterdam/education/cosmology/}{\emph{Cosmology}}:
    simply stunning; enlightening and beautifully written. The material vastly exceeds
    what we cover in this course, but you definitely want to check out Chapter~3
    on the thermal history of the universe.
    \item Timm Wrase,
    \href{http://hep.itp.tuwien.ac.at/~wrasetm/cosmology2019.html}{\emph{Cosmology and Particle Physics: Lecture Notes}}:
    another excellent set of lecture notes, with significant overlap with the previous
    one.
\end{itemize}
